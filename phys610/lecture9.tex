[missed first half]
\section{Cylinder-plane conformal mapping}
Consider the mapping $w\to z$ defined by
\begin{align}
    w &= \ln\frac{z + a}{z - a}
\end{align}

Let $w=u + iv$ and $z = x + iy$.
We want to answer these two questions:
(a) What region of the $z$-plane corresponds to the region bewteen
$u=0$ and $u=u_0$?

(b) what region of the $z$-plane corresponds to the region of the $w$-plane
between $v=0$ and $v=v_0$,
where $v_0$ is a constant that lies between $0$ and $\pi$.

(a)
\begin{align}
    w &= u + iv = \ln \frac{z + a}{z - a}
\end{align}
where $z = x + iy$.
So
\begin{align}
    e^{w} &= \frac{z + a}{z - a}
\end{align}
and the modulus is
\begin{align}
    \left| e^{u} e^{iv} \right|
    &=
    \left|
        \frac{(x + a) + iy}{(x - a) + iy}
    \right|
\end{align}
and so
\begin{align}
    e^{2 u} &=
    \frac{(x + a)^2 + y^2}{(x - a)^2 + y^2}
\end{align}

For $u=0$, $e^{2 u} = 1$.
\begin{align}
    (x + a)^2 + y^2 &=
\end{align}


[missed some lines]

For $u=u_0$,
$e^{2 u_0} =: C^2$,
we have
$C^2 > 1$ for $u_0 > 0$.
Along $v=u_0$,
\begin{align}
    C^2\left[
        (x - a)^2 + y^2
    \right] &=
    (x + a)^2 + y^2
\end{align}
and after some algebra you can find in the notes,
you find that
\begin{align}
    x^2 + y^2 + a^2 *=
    \underbrace{\left(\frac{c^2 + 1}{c^2 - 1}\right)}_{\lambda} 2ax
\end{align}
So call that thing $\lambda$ becuase it's the only place where $C^2$ appears.
\begin{align}
    \lambda &:= \frac{c^2 + 1}{c^2 - 1}\right)
\end{align}
It's possible to rewrite the quation
\begin{align}
    (x - a\lambda)^2 + y^2 = a^2 (\lambda^2 - 1)
\end{align}
and you can convince yourself that $\lambda > 1$.
Because $C^2$ is a positive number bigger than 1,
so $C^2 - 1$ is a positive number,
and $C^2 + 1$ is a bigger positive number,
so $\lambda > 1$.
This equation here is an equation of a circle
centred at $(a\lambda, 0)$.
So the centre of the circle is at $x=a\lambda$.
So it's on the $x$-axis.
The radius of the circle is
$R= a\sqrt{\lambda^2 - 1}$.

Let me continue for a couple more minutes.
Here's what we're finding.

We started with 2 straight lines in the $u-v$ plane.
$u=u_0$ and $u=u_0$.
Those are 2 straight lines.
The $u=0$ line maps to the $y$-axis
and the $u=u_0$ line maps to a circle with centre sitting on the $x$ axis.
The region between $u=0$ and $u=u_0$ is mapped into the region between the
$y$-axis and the circle.

So let's draw that.

[picture here]

This is a fairly standard map.
We don't have enough time in the class to go thorugh all the standard maps.
If you're familiar with all the standard maps you can go very ar.
It's just like how in integration we hvae standar contours.
That's why it's important I did almost all of the standard contours.

Maybe 30 to 40 years ago,
conformal mapping was improtant,
but these days we can solve things using more powerful computers and this is
becoming less important.
30 years ago,
a big chunk of the course would have been dedicated to conformal mapping
just like how I did contour integrals in the course.

\begin{question}
    At the bottom of the $y$-axis is $u=0$?
\end{question}
The entire $y$-axis is mapped from the line $u=0$.
Sorry about my handwriting.
This whole line is $u=0$.

Any other questions?

(b)
Let's do problem b now.
So we start with what we had before
\begin{align}
    e^{u}e^{iv}
    &=
    \frac{(x + a) _+ iy}{(x - a) + iy}
    &=
    \frac{(x^2 + y^2 - a^2) - 2 i ay}{(x - a)^2 + y^2}
\end{align}
and then skipping a few steps you can find in the notes,
you get
\begin{align}
    e^{iv} &=
    e^{-u}\left\{
        \frac{(x^2 + y^2 - a^2) - 2 iay}{(x - a)^2 + y^2}
    \right\}
\end{align}
But from the previous problem, we know
\begin{align}
    e^{-u} &= \cdots
\end{align}
and with a little bit of algebra gain which I'm not going to show you,
you get
\begin{align}
    e^{iv} &=
    \frac{(x^2 + y^2 - a^2) - 2iay}{\sqrt{
        [(x - a)^2 + y^2]
        [(x + a)^2 + y^2]
    }}
\end{align}
So now we have a direct relation between $v$ and $x,y$.
But you can see this is a messy expression.
Not straightforward to handle,
but let's see how to make progress.

So then let me notice that
\begin{align}
    \tan v &=
    \frac{-2ay}{x^2 + y^2 - a^2}
\end{align}
and if $v$ is a constant, then $\tan v$ is also a constant.
For reasons that will be clear in a moment,
let me also write
\begin{align}
    \sin v &=
    \frac{-2ay}{\sqrt{
        [(x - a)^2 + y^2]
        [(x + a)^2 + y^2]
    }}
\end{align}
So why am I writing both tan and sin?
It's because the tan is not unique,
it's the same for both $v$ and $v+\pi$.
If I tell you the tanget and the sign,
then I specify $v$ uniquely.
If I only specify $\tan v$,
there is a degeneracy and it's not clear what $v$ is.

Let's go back to the quesiton.
For $v=0$, $\tan v =0 $ and $y=0$.
So $v=0$ maps onto the $x$ axis.

Let $\tan v_0 = k$ be a constant.
Then
\begin{align}
    \frac{-2ay}{x^2 + y^2 - a^2} = k
\end{align}
With some algebra I'm skipping,
if you do it,
you can rewrite this as
\begin{align}
    x^2 = \left( y + \frac{a}{k} \right)^2
    &=
    a^2 \left( 1 + \frac{1}{k^2} \right)
\end{align}
and this is the circle centred at
$\left( 0, \frac{-a}{k} \right)$
with readius
$a\sqrt{1 + \frac{1}{k^2}}$.

Alright,
so let's draw it.
This is the $x$-axis and this is the $y$-axis.
Let's draw a circle.
This is the centre of the circle.
I've drawn part of the circle dotted.
Well you're wondering this is the equatino for the full circle,
but why have I drawn part of it solid and part of it dotted.
Well here's why.
Remember what I did was set $\tan v = k$.
That's how I got a circle.
But remember $\tan v$ is the same for both $v$ and $v+\pi$.
I'm going to argue that the dotted part corresponds to $v+\pi$,
and the solid part is the circle that maps to what we want.
It's actually quite tricky.

[picture here]

Recall the the denominator of the $\sin v$ expression is positive.
But how we were asked to look at the problem for $0\le v_0< \pi$.
So if $v < \pi$, then $\sin v$ > 0 and $y <0$,
so only the portion of the circle below the $y=0$ $x$-axis is what we want.
The part of the circle above the $x$-axis corresponds to when
$\sin v < 0$.
Let me write a couple of things and then I can ask questions.

Since we restricted $0< v_0 < \pi$,
$\sin v_0 > 0$.
Then only the part of the circle with $y<0$
corresponds to $v_0$.

Now you can convince yourself that the circle crosses the $x$-axis at
$x=-a$,
which maps to $u-\infty$,
and at $x=a$,
which maps to $u=+\infty$.
The lowest point of the circle,
which is at $x=0$,
corresponds to $u=0$.

So as you change $u$ holding $v$ fixed,
you trace out an arc.
Convince yourself that this is true.

Are there any questions about this?

We have a grand total of 5 minutes to solve this part C.
Alright.

(c) Consider the geometry below in the $w$-plane.
You have two infinite plates,
one at $u=0$ and one at $u=u_0$.
And this is the line $v=0$.
This is the direction of increasing $v$.
Now we're going to set this $u=0$ plane to potential
$V=0$
and this other plate to potential
$V=V_0$.

The solution for region between the two plates is
\begin{align}
    V &= V_0\left\{
        \frac{u}{u_0}
    \right\}
\end{align}
And remember this mapping maps from plane to cylinder.
We already showed that.

This maps into the geometry below,
a plane at $x=0$
and cylinder at $(\lambda a, 0)$
with radius $a\sqrt{\lambda^2 - 1}$.
But you remember in the formulation of the question,
the cylinder is centred distance $d$ from the plane
and has radius $R$.
So we equate
\begin{align}
    d &= a \lambda\\
    R &= a\sqrt{\lambda^2 - 1}
\end{align}
So then after some algebra I have in the notes,
you find that
\begin{align}
    a &=
    R \sqrt{\frac{d^2}{R^2} - 1}\\
    \lambda &=
    \frac{d/R}{\sqrt{\frac{d^2}{R^2} - 1}}
\end{align}
To solve the problem,
all we have to do is rewrite $V=V_0 u/u_0$,
rewriting $u$ in terms of $x$ and $y$.
But we know what that is because we have this thing from (a),
and here's what you end up with
\begin{align}
    V &= \frac{V_0}{u_0} \frac{1}{2}
    \log\left\{
        \frac{(x + a)^2 + y^2}{(x - a)^2 + y^2}
    \right\}
\end{align}
and we have to rewrite $u_0$ and $a$ in terms of
$d$ and $R$ using the expressions above.
It's given in the notes.
This is the solution
and we just have to rewrite the constants in terms of $d$ and $R$.
You can find the closed-form expression in the notes.
You can check by direct substitution into Laplace's euqation,
and you find that it satisfied Laplace's equation and also
that it satisfies the boundary conditions.

Let me stop here.
I know that was rather quick.
