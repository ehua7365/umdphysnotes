\section{Uniqueness of Solutions to Laplace equation}
Assume 2 solutions $\phi_1$ and $\phi_2$ with
$\phi_2\ne \phi_1$.
$\phi_1,\phi_2$ satisfy some boundary conditions,
Dirichlet or Neumann
$\vec{\nabla}\cdot\hat{n}$.

Consider $U=\phi_2 - \phi_1$.
We know $\nabla^2 U=0$ so
\begin{align}
    \int d^3x\,
    \underbrace{
    U \nabla^2 U
    }_{
    \vec{\nabla}\cdot\left( U\vec{\nabla} U \right)
    -
    \left( \vec{\nabla} U \right)^2
    }
    &=0\\
\end{align}
which implies
\begin{align}
    \int_V d^3x\,
    |\vec{\nabla} U|^2
    &=
    \underbrace{
    \int_V d^3x\, \vec{\nabla}\cdot\left( U\vec{\nabla}U \right)
    }_{
    \oint_S dS\, U\left( \vec{n}\cdot\vec{\nabla}U \right)
    }
\end{align}
hence
\begin{align}
    \int_V d^3x\, |\vec{\nabla}U|^2
    &=
    \oint_S dS\, U\left( \vec{\nabla} U\cdot\hat{n} \right)
\end{align}
For Dirichlet boundary conditions,
value of $\phi$ is specified on boundary $S$.
Then $U=0$ on $S$
and
\begin{align}
    \int d^3x\,
    |\vec{\nabla}U|^2 = 0
\end{align}
and we're integrating a positive quantity to be zero,
which means
\begin{align}
    \vec{\nabla}U = 0
\end{align}
everywhere in $V$
and so $U$ is constant.
But since $U=0$ on boundary,
$U=0$ everywhere.

For Neumann case,
$\hat{n}\cdot\vec{\nabla}\phi$ is specified on the boundary.
This is basically saying the normal component of the electric field is
specified.
Then $\vec{n}\cdot\vec{\nabla}U=0$
on the boundary.

So once again,
\begin{align}
    \int d^3x \left| \vec{\nabla} U\right|^2 = 0
\end{align}
Now by exactly the same logic,
$U$ is a constant everywhere.
But we don't know it's actually zero on the boundary.
And so this is as far as we can go.
So all we can really show is that
$\phi_2 - \phi_1$
is a constant in $V$.
So the solution is unique up to an arbitrary constant.

You can always shift the minimum of the potential,
you can change your potential by a constant everywhere and the electric field
doesn't change.
It's the potential differences that matter,
not the absolute value.

You specify the Neumann boundary conditions,
you're just specifying the electric field of the boundary,
the normal component.
Since you didn't specify the potential anywhere,
you can choose your zero potential anywhere,
so you can shift your potential by a constant and the electric field will still
be the same.

In that sense the solution is unique,
because the physical thing,
which is the electric field,
is unique and doesn't change.

The electric field everywhere is completely specified.

The next thing to consider is electrostatics.

\section{Electrostatic Energy}
Change $q_1$ located at $\vec{x}_1$.
Potential at $\vec{x}$ is due to $\vec{q}_1$.
\begin{align}
    \phi\left(\vec{x}\right)
    &=
    \frac{1}{4\pi\epsilon_0}
    \frac{q_1}{\left|\vec{x} - \vec{x}_1\right|}
\end{align}
Now move another charge $q_2$ from $\infty$
to $\vec{x}_2$.
Potential energy is the work done
\begin{align}
    W &=
    q_2 \phi\left( \vec{x}_2 \right)\\
    &=
    \frac{1}{4\pi\epsilon_0}
    \frac{q_1 q-2}{\left|\vec{x}_2 - \vec{x}_1\right|}
\end{align}
And now if we bring in a third charge $q_3$,
you have to include the potential of the two existing charges.
\begin{align}
    W &=
    \frac{1}{4\pi\epsilon_0_0}
    \left\{ 
    \frac{q_1 q_2}{\left| \vec{x}_2 - \vec{x}_1 \right|}
    + \frac{q_1 q_3}{\left| \vec{x}_3 - \vec{x}_1 \right|}
    + \frac{q_2 q_3}{\left| \vec{x}_3 - \vec{x}_2 \right|}
    \right\}
\end{align}
This generalizes pretty easily to the case where you have $N$ charges.
You add it all up,
up to when you bring the $n$th charge to one point and consider the potential of
all the other $N-1$ charges.

This generalizes in a simple way.
In the case of $n$ charges,
\begin{align}
    W &=
    \sum_{i=1}^{n}
    \sum_{j<i}
    \frac{1}{4\pi\epsilon_0}
    \frac{q_i q_j}{\left| \vec{x}_i - \vec{x}_j \right|}\\
    &=
    \frac{1}{2}
    \sum_{i=1}^{n}
    \sum_{j=1: i\ne j}^{n}
    \frac{1}{4\pi\epsilon_0}
    \frac{q_i q_j}{\left| \vec{x}_i - \vec{x}_j \right|}
\end{align}
For continuous charge distributions,
\begin{align}
    W &=
    \frac{1}{2}
    \int d^3\vec{x}
    \int d^3\vec{x}\,
    \frac{\rho\left( \vec{x} \right) \rho\left( \vec{x}' \right)}{
    \left| \vec{x} - \vec{x}' \right|
    }
    \frac{1}{4\pi\epsilon_0}
\end{align}
Notice that almost all those factors are there.
We can rewrite this in terms of the potential
$\phi(\vec{x})=\frac{1}{4\pi\epsilon_0} \frac{\rho(\vec{x}')}{
\left|\vec{x} - \vec{x}'\right|}$
to get
\begin{align}
    W &=
    \frac{1}{2}
    \int d^3x\,
    \rho\left( \vec{x} \right)
    \phi\left( \vec{x} \right)
\end{align}
And now what we can do is use Poisson's equation to eliminate this.
Using Poisson's equation,
this becomes
\begin{align}
    W &=\frac{1}{2}\epsilon_0 \int d^3x\,
    \underbrace{\phi\left( \vec{x} \right) \nabla^2 \phi(\vec{x})}_{
    \vec{\nabla}\cdot\left\{ 
    \phi \vec{\nabla} \phi
    \right\}
    -
    \left( \vec{\nabla} \phi \right)^2
    }\\
    &=
    \frac{1}{2}\epsilon_0 \int d^3x\,
    \left( \vec{\nabla}\phi \right)^2
    -
    \frac{1}{2}\epsilon_0
    \underbrace{
    \int d^3x\,
    \vec{\nabla}\cdot\left\{ 
    \phi \cdot \vec{\nabla} \phi
    \right\}
    }_{
    \int_S dS \phi\left( 
    n\cdot \underbrace{\vec{\nabla}\phi}_{=0}
    \right)
    }
\end{align}
where the surface we are integrating over is at infinity.
So then
\begin{align}
    W &=
    \frac{1}{2} \epsilon_0 \int d^3x\,
    \left( \vec{\nabla}\phi \right)^2\\
    &=
    \frac{1}{2}
    \epsilon_0
    \int d^3x\,
    \left|\vec{E}\right|^2
\end{align}
Note even far apart,
there is self energy of the charges.
On the other hand,
you're calculating the work needed (change in potential energy)
to bring charges from infinity into your configuration.
That is a physical thing,
because it's a potnetial energy difference.
That's a convnetion.

The self-energy has a long and interesting history I could give several lectures
on, but we don't have time for that.
The energy density stored in electrostatic field.
\begin{align}
    w &=
    \frac{1}{2}
    \epsilon_0
    \left| \vec{E} \right|^2
\end{align}
So you can think of the energy density per unit energy stored in the electric
field.

\section{Force on a conducting surface}
There are two ways you can compute this.
\begin{enumerate}
    \item Energy change from a virtual displacement.
    \item Directly from the electric field.
\end{enumerate}

First method.
The energy density is $u = \frac{1}{2} \epsilon_0 \left| \vec{E} \right|^2$.
At the surface of conductor $\vec{E} = \sigma/\epsilon_0$
directed normal to the surface.

Consider a small displacement $\Delta x$ in a direction normal to a surface
element $\Delta A$.

Change in energy
\begin{align}
    \Delta W &=
    -\frac{\epsilon_0}{2}
    |\vec{E}|^2 \Delta A \Delta x\\
    &=
    -\frac{1}{2\epsilon_0} \sigma^2 \Delta A \Delta x
\end{align}
So the force is
\begin{align}
    F &=
    -\frac{\Delta W}{\Delta x}\\
    &=
    \frac{\sigma^2 \delta A}{2\epsilon_0}
\end{align}
directed outward from surface.
Force per unit area is
\begin{align}
    f &=
    \frac{\sigma^2}{2\epsilon_0}
\end{align}
outwards.
To get the total force on conductor integrate $f$ over the surface of conductor
bar.

Second method.
The force on an element of area $\Delta A$
is given by
\begin{align}
    \vec{F} &=
    \left( \sigma \Delta A \right)\vec{E}_{ext}
\end{align}
where $\vec{E}_{ext}$ is the electric field after contribution from $\Delta A$
has been removed.
The force is thus
\begin{align}
    F &=
    \left( \sigma \Delta A \right)
    \frac{\sigma}{2\epsilon_0}\\
    &=
    \frac{\sigma^2}{2\epsilon_0} \Delta A
\end{align}
outwards.
The hard part is finding the charge distribution $\sigma$,
but once you've found it,
it's easy to get the force.

\section{Capacitance}
The capacitance is defined as
charge divided by potential
\begin{align}
    C := \frac{Q}{\phi}
\end{align}
An object with large capacitance means it can hold a large amount of charge
without much potential.

Two conductors carrying equal and opposite charge
$+Q$ and $-Q$,
with potential difference $\Delta \phi$
between them,
the capacitance is by definition
\begin{align}
    C &:=
    \frac{Q}{\Delta \phi}.
\end{align}

\section{Boundary Value Problems}
\subsection{Method of images}
Imagine a charge near a conducting surface.

\begin{example}
    A grounded conducting sphere of radius $a$ is centred at the origin.
    A point of charge $q$ is located at point $\vec{y}$.
    Find the potential in the region outside the sphere.
\end{example}
The potential due to $q$ and $q'$ is
\begin{align}
    \phi(\vec{x})
    &=
    \frac{1}{4\pi\epsilon_0}
    \left\{ 
    \frac{q}{\left| \vec{x} - \vec{y} \right|}
    \right\}
    +
    \frac{1}{4\pi\epsilon_0}
    \left\{ 
    \frac{q'}{\left| \vec{x} - \vec{y}' \right|}
    \right\}
\end{align}
Let $\hat{n}$ be the unit vector in direction of $\vec{x}$.
Let $\hat{n}'$ be in the direction of $y$.
Then
\begin{align}
    \hpi\left( \vec{x} \right)
    &=
    \frac{1}{4\pi\epsilon_0}\left\{ 
    \frac{q}{\left| x\hat{n} - y\hat{n}'\right|}
    + \frac{q'}{\left| x\hat{n} - y\hat{n}'\right|}
    \right\}
\end{align}
where
\begin{align}
    |\vec{x}| &= x\\
    |\vec{y}| &= y\\
    |\vec{y}'| &= y'
\end{align}
We need $\phi\left( |\vec{x}| = a \right) = 0$.
Then
\begin{align}
    \phi(\vec{x}=a)
    &=
    \frac{1}{4\pi\epsilon_0}
    \left\{ 
    \frac{q}{a\left| \hat{n} - \frac{1}{a} \hat{n}'\right|}
    + \frac{q'}{y'\left| \hat{n}' - \frac{1}{y'} \hat{n}\right|}
    \right\}
\end{align}
If we choose
\begin{align}
    \frac{q}{a}
    &=
    - \frac{q'}{y'}
\end{align}
and
\begin{align}
    \frac{y}{a} &= \frac{a}{y'}
\end{align}
then $\phi\left( |\vec{x}| = a \right) = 0$
for all $\hat{n}$ and $\hat{n}'$.
Then the position and magnitude of the image charge is
\begin{align}
    q' &= -\frac{a}{y}q\\
    y' &= \frac{a^2}{y}
\end{align}
