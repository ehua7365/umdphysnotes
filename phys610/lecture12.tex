\section{Hamilton's Principle of Least Action}
3 equivalent formulations
\begin{itemize}
    \item Newton's laws
    \item Lagrange's equation
    \item Hamiton's principle
\end{itemize}
Let us generalize to non-Cartesian coordinates
\begin{align}
    \left( x, y, z \right) \to \left( q_1, q_2, q_3 \right)
\end{align}
For example, $(q_1, q_2, q_3)$
could be
$(r, \theta, \phi)$
or
$(\rho, \phi, z)$.

Necessary that each position $\vec{r}$
uniquely specifies $(q_1, q_2, q_3)$
\begin{align}
    q_i &= q_i(x, y, z)\\
    \vec{r} &= \vec{r}\left( q_1, q_2, q_3 \right).
\end{align}
We can rewrite $x,y,z$
and also
$\dot{x},\dot{y},\dot{z}$
in terms of the
$q_1,q_2,q_3$
and
$\dot{q}_1,\dot{q}_2,\dot{q}_3$.
Then
\begin{align}
    L &=
    \frac{1}{2}m\dot{\vec{r}}^2 - U(\vec{r})
    = L(q_i, \dot{q}_i)
\end{align}
Let's write the action integral
\begin{align}
    S &=
    \int_{t_1}^{t_2} dt\,
    L\left( q_i, \dot{q}_i \right).
\end{align}
Let's draw the picture again.

The value of the action integral for every single coordinate system,
so the action is minimized for all paths.
It's going to be the same path in the extremum value.
And so that means $L$ will satisfy an Euler-Lagrange equation in these
coordinates.

The value of $L$ at any point and therefore the integral $S$ for any path is
unaltered by the change of variables.
Thus it must be true that $S$ is stationary at the correct path even in the new
coordinates system.
This implicitly implies that the Euler-Lagrange equations
\begin{align}
    \frac{d}{dt}\left( \frac{\partial L}{\partial \dot{q}_i} \right)
    - \frac{\partial L}{\partial q_i} &= 0
\end{align}
holds for any set of generalized coordaintes.

\section{Generalized Force and Momentum}
force a particle in 1D,
\begin{align}
    L &= \frac{1}{2} m\doc{x}^2 - U(x)
\end{align}
and
\begin{align}
    \frac{d}{dt}\underbrace{%
        \left(
            \frac{\partial L}{\partial \dot{x}}
        \right)
    }_{m\dot{x} = p}
    =
    \underbrace{%
        \frac{\partial L}{\partial x}
    }_{%
        -\frac{\partial U}{\partial x}
    }
\end{align}
So that gives
\begin{align}
    \frac{dp}{dt} &= -\frac{\partial U}{\partial x}.
\end{align}
So in general.
\begin{align}
    \text{generalized momentum} &:= \frac{\partial L}{\partial \dot{q}_i}\\
    \text{generalized force} &:= \frac{\partial L}{\partial q_i}.
\end{align}
The Euler-Lagrange equation says that
the rate of change of generalized momentum equals the generalized force.

\begin{example}[Single particle in polar coordinates]
    The kinetic energy is
    \begin{align}
        T &= \frac{1}{2}mv^2 =
        \frac{1}{2} m\left( \dot{r}^2 + r^2 \dot{\phi}^2 \right)
    \end{align}
    and the Lagrangian is
    \begin{align}
        L &= T - U\\
        &=
        \frac{1}{2}m\left( \dot{r}^2 + r^2\dot{\phi}^2 \right)
        - U(r, \phi)
    \end{align}
\end{example}
Then the Euler-Lagrange equation is
\begin{align}
    \frac{d}{dt}
    \underbrace{%
        \left(
            \frac{\partial L}{\partial\dot{r}}
        \right)
    }_{%
        m\dot{r}
    }
    =
    \underbrace{%
        \frac{\partial L}{\partial r}
    }_{%
        -\frac{\partial U}{\partial r} + mr\dot{\phi}^2
    }
\end{align}
which leads to the radial equation of motion.
\begin{align}
    m\underbrace{\ddot{r}}_{\text{radial acceleration}}
    - \underbrace{m r\dot{\phi}^2}_{\text{centripetal force}}
    &=
    \underbrace{- \frac{\partial U}{\partial r}}_{%
        \text{radial component of force}
    }
\end{align}
Then the angular equation of motion is
\begin{align}
    \frac{d}{dt}\underbrace{%
        \left(
            \frac{\partial L}{\partial\dot{\phi}}
        \right)
    }_{mr^2\dot{\phi}}
    &=
    \underbrace{\frac{\partial L}{\partial \phi}}_{
        -\frac{\partial U}{\partial \phi}
    }
\end{align}
so you get
\begin{align}
    \frac{d}{dt}\underbrace{\left( mr^2\dot{\phi} \right)}_{\text{angular
    momentum}}
    &=
    \underbrace{- \frac{\partial U}{\partial \phi}}_{r F_{\phi}=\text{torque}}
\end{align}
Then the force is
\begin{align}
    \vec{F} = - \vec{\nabla} U
    = -\frac{\partial U}{\partial r} \hat{e}_r
    - \frac{1}{r} \frac{\partial U}{\partial \phi} \hat{e}_\phi
\end{align}

\section{Constrained system}
Consider a pendulum of length $l$.
The position is given by $(x,y)$
but there is a constraint
\begin{align}
    \sqrt{x^2 + y^2} &= l.
\end{align}
So you only need one degree of freedom.

The system only has one degree of freedom because location of bob is specified
once we know the value of $x$.
Alternatively,
location is specified by the value of generalized coordinate $\theta$.

The coordinate transform is
\begin{align}
    x &= l \sin\theta\\
    y &= l \cos\theta.
\end{align}

The kinetic energy is
\begin{align}
    T &= \frac{1}{2}m \left( \dot{x}^2 + \dot{y}^2 \right)\\
    &= \frac{1}{2}m l^2 \dot{\theta}^2
\end{align}
and the potential energy
is
\begin{align}
    U &= mgl(1 - \cos\theta)
\end{align}
Then the Lagrangian is
\begin{align}
    L &= T - U\\
    &= \frac{1}{2} ml^2 \dot{\theta}^2
    - mgl(1 - \cos\theta)
\end{align}
The Euler-Lagrange equation is
\begin{align}
    \frac{d}{dt}\left(\frac{\partial L}{\partial \dot{\theta}} \right)
    -
    \frac{\partial L}{\partial \theta} = 0
\end{align}
which gives the equation of motion
\begin{align}
    ml^2\ddot{\theta} + mgl\sin\theta &= 0\\
    \ddot{\theta} + \frac{g}{l}\sin\theta &= 0.
\end{align}

\section{Generalized Coordinates for a Constrained System}
We say that the parameters
$q_1,q_2,\ldots,q_n$ are generalized coodrinates for the system of
$N$ particles
$\alpha = 1,2,\ldots,N$
with psoitions $\vec{r_}_{\alpha}$
if
\begin{enumerate}
    \item Each position $\vec{r}_{\alpha}$ can be expanded as a function of
        $\left( q_1,q_2,\ldots,q_n \right)$
        and (possibly) $t$.
        \begin{align}
            \vec{r}_{\alpha} &=
            \left( q_1, q_2, \ldots, q_n, t \right)
        \end{align}
        for $\alpha = 1,2,\ldots,N$.
    \item Each $q_i$ can e expressed in terms of the $\vec{r}_{\alpha}$
        and (possibly) $t$.
        \begin{align}
            q_i &= q_i\left( \vec{r}_1, \vec{r}_2,\ldots,\vec{r}_N, t \right)
        \end{align}
    \item The number of generazed coordinates is the smallest number that allows
        the system to be parameterized in this way
        \begin{align}
            n &= 3N & \text{for unconstrained system}\\
            n &< 3N & \text{for constrained system}
        \end{align}
\end{enumerate}
\begin{example}[Double pendulum]
    There is a pivot point to which a rigid bar of length $l_1$ is free to
    swing,
    at the end of which is a mass $m_1$.
    It makes an anble $\phi_1$ with the veritical.
    Another rigid bar is attached to where $m_2$ is,
    but is also free to swing,
    of length $l_2$ which makes an anble $\phi_2$ with the vertical,
    at the end of which is a mass $m_2$.
\end{example}
The position of the first mass is
\begin{align}
    \vec{r}_1 &= \vec{r}_1(\phi_1)\\
    \vec{r}_2 &= \vec{r}_2(\phi_1, \phi_2)
\end{align}
