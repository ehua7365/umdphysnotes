\section{Green function method}
Let's do an example.
Consider the differential equation
\begin{align}
    \frac{d^2y}{dx^2} + y &= f(x)
\end{align}
Boundary conditions $y=0$ at $x=0$
and $y=0$ at $x=\frac{\pi}{2}$.
Introduce the green function $G(x, x')$ which satisfies
\begin{align}
    \frac{d^2 G}{dx^2} + G &=
    \delta\left( x - x' \right)
\end{align}
Why introduce $G$?
In general,
$G$ is easier to find than $y$.
Once $G$ has been determined,
we can obtain $y$ with
\begin{align}
    y(x) &=
    \int_{0}^{\pi/2}
    G\left( x, x' \right)
    f\left( x' \right)\,
    dx'
\end{align}
To see that this is a solution,
substitute it into the original ODE.
\begin{align}
    \frac{dd^2 y}{dx^2} + y
    &=
    \int_{0}^{\pi/2}
    \underbrace{\left\{ 
    \frac{d^2 G}{dx^2} + G
    \right\}}_{
    \delta\left( x - x' \right)
    }
    f\left( x' \right)
    \,dx'\\
    &=
    f(x).
\end{align}
It also satisfies boundary conditions
\begin{align}
    y(x=0) &=
    \int_{0}^{\pi/2}
    \underbrace{\left.G\left( x, x' \right)\right|_{x=0}}_{=0}
    f\left( x' \right)\,dx' = 0
\end{align}
and
\begin{align}
    y\left( x=\frac{\pi}{2} \right)
    &=
    \int_{0}^{\pi/2}
    \underbrace{\left.G\left( x, x' \right)\right|_{x=\pi/2}}_{=0}
    f\left( x' \right)
    \,dx'
    =0
\end{align}
But how do we find $G$?
\begin{align}
    \frac{d^2G}{dx^2} + G &= \delta\left( x - x' \right)
\end{align}
Away from $x=x'$,
\begin{align}
    \frac{d^2 G}{dx^2 + G} = 0
\end{align}
The solutions are of the form
\begin{align}
    G &=
    A\sin(x) + B\cos(x)
\end{align}
In general the solution for $x<x'$ and $x>x'$ will be different
because they must satisfy different boundary conditions at each end.
\begin{align}
    G &=
    \begin{cases}
        G_{<} = A_{<}\sin(x) + B_{<}\cos(x) & \text{for } x < x'\\
        G_{>} = A_{>}\sin(x) + B_{>}\cos(x) & \text{for } x > x'\\
    \end{cases}
\end{align}
Requiring $G=0$ at $x=0$,
we get $B_{<}=0$.
Requiring $G=0$ at $x=\frac{\pi}{2}$,
we get $A_{>}=0$.
Now we need match the two pieces across $x'$.

In quantum mechanics have you solved for a delta function potential?
Now let's look at this differential equation here.

I claim that this is what we mean.
I want $G$ to be continuous across the boundary,
but $dG/dx$ is not necessarily continuous.
That's the way this works for this example.

Requiring that $G$ be continuous at $x=x'$,
we have
\begin{align}
    G_{<}\left( x = x' \right)
    =
    G_{>}\left( x = x' \right)
\end{align}
we get
\begin{align}
    A_{,}\sin\left( x' \right)
    &=
    B_{>}\cos\left( x' \right)
\end{align}
But we need one more condition,
and that has to come from the derivatives.
So what we do is integrate that differential equation for $G$ across $x=x'$.
\begin{align}
    \lim_{\epsilon\to 0}
    \int_{x'=\epsilon}^{x' + \epsilon}dx\left\{ 
    \frac{d^2 G}{dx^2} + \underbrace{G}_{O(\epsilon)}
    \right\}
    &=
    \underbrace{\int_{x' - \epsilon}^{x' + \epsilon}
    dx\,
    \delta\left( x - x' \right)}_{1}
\end{align}
which gives
\begin{align}
    \left.\frac{dG}{dx}\right|_{x'-\epsilon}^{x'+\epsilon}
    = 1
\end{align}
which means
\begin{align}
    \left. \frac{dG_{>}}{dx}\right|_{x=x'}
    -
    \left. \frac{dG_{<}}{dx}\right|_{x=x'}
    &= 1\\
    -B_{<}\sin\left( x' \right) - A_{<}\cos\left( x' \right) &= 1
\end{align}
When all the dust settles,
the solution is
\begin{align}
    A_{<} &= -\cos\left( x' \right)\\
    B_{>} &= -\sin\left( x' \right)
\end{align}
Hence
\begin{align}
    G\left( x, x' \right)
    &=
    \begin{cases}
        G_{<}\left( x, x' \right)
        =
        -\sin\left( x \right)\cos\left( x' \right)
        &\text{for } x<x'\\
        G_{>}\left( x, x' \right)
        =
        -\cos\left( x \right)\sin\left( x' \right)
        &\text{for } x> x'
    \end{cases}
\end{align}
Notice that
\begin{align}
    G\left( x, x' \right) &= G\left( x', x \right)
\end{align}
You should check that your Greens function has this property.
If it doesn't your probably screwed up.
It's a powerful consistency check.

Some people like to write it in shorthand
\begin{align}
    G\left( x, x' \right)
    &=
    -\sin\left( x_{<} \right)\cos\left( x_{>} \right)
\end{align}
where $x_{<}$ refers to the smaller of $x$ and $x'$
and $x_{.}$ refers to the larger of $x$ and $x'$.
To get the final solution
\begin{align}
    y(x) &=
    \int_{0}^{\pi/2}dx'\,
    G\left( x, x' \right)
    f\left( x' \right)\\
    &=
    \int_{0}^{x}
    dx'
    f\left( x' \right)
    G_{>}\left( x, x' \right)
    +
    \int_{0}^{\pi/2}dx\,
    f\left( x' \right)
    G_{<}\left( x, x' \right)\\
    &= -\cos(\pi)\left[ 
    \int_{0}^{\pi}dx'\,
    f\left( x' \right)
    \sin\left( x' \right)
    \right]
    -
    \sin\left( x \right)
    \left[ 
    \int_{x}^{\pi/2}dx'\,
    f\left( x' \right)
    \cos\left( x' \right)
    \right]
\end{align}
So that's the solution.
This is a standard way to solve inhomogeneous differential equations.

You've actually been using this method for a long time.
The potential from a point charge admits an interpretation as the Green function
of the Laplacian operator $\nabla^2$.
Consider the Poisson equation
\begin{align}
    \nabla^2\phi &=
    -\frac{\rho}{\epsilon_0}
\end{align}
For a point charge $q$ located at $\vec{x}'$,
\begin{align}
    \rho\left( \vec{x}' \right)
    &=
    q\delta^{(3)}\left( \vec{x} - \vec{x}' \right).
\end{align}
The resulting potential $\hat{\phi}$ satisfies
\begin{align}
    \nabla^2\hat{\phi}
    &=
    -\frac{q}{\epsilon_0}
    \delta^{(3)}\left( \vec{x} - \vec{x}' \right)
\end{align}
with
\begin{align}
    \hat{\phi}
    &=
    \frac{1}{4\pi\epsilon_0}
    \frac{q}{\left| \vec{x} - \vec{x}'\right|}
\end{align}
By definition,
the Green function for the Laplacian operator is required to satisfy
\begin{align}
    \nabla^2 G
    &=
    -4\pi \delta^{(3)}\left( \vec{x} - \vec{x}' \right)
\end{align}
This Poisson equations is the kind of equation we're trying to solve.
We set up a Greens function,
replacing the source term by a delta function,
just the way we did earlier.
We use the same differential operator,
but we replace the arbitrary source by a $\delta$ function.
The only difference is the factor of $-4\pi$.
Previously we set it to 1,
which is more standard,
but in electrostatics,
it's just a convention we put a $-4\pi$ there.

Comparing we see that
\begin{align}
    G\left( \vec{x}, \vec{x}' \right)
    &=
    \frac{4\pi\epsilon_0}{q}\hat{\phi}
    =
    \frac{1}{\left|\vec{x} - \vec{x}'\right|}
\end{align}
Then the solution of the Poisson equation is
\begin{align}
    \phi\left( \vec{x} \right)
    &=
    -\frac{1}{4\pi}
    \int d^3\vec{x}\,
    G\left( x, x' \right)
    \left\{ 
    -\frac{\rho(x)}{\epsilon_0}
    \right\}\\
    &=
    \frac{1}{4\pi\epsilon_0}
    \int d^3x
    \frac{\rho(x)}{\left|\vec{x} - \vec{x}'\right|}
\end{align}

Let's summarize the results.
Consider the example we had
\begin{align}
    y'' + y &= f(x)\\
    G'' + G &= \delta\left( x - x' \right)\\
    G\left( x, x' \right) &= -\sin\left( x_{<} \right)\cos\left( x_{>}
    \right)\\
    y &= \int dx\, f\left( x' \right) G\left( x, x' \right)
\end{align}
The same method applies for electrostatics.
\begin{align}
    \nabla^2 \phi &=
    -\frac{\rho\left( \vec{x} \right)}{\epsilon_0}\\
    \nabla^2 G &=
    -4\pi \delta^{(3)}\left( \vec{x} - \vec{x}' \right)\\
    G\left( \vec{x}, \vec{x}' \right)
    &=
    \frac{1}{\left|\vec{x} - \vec{x}'\right|}\\
    \phi &=
    -\frac{1}{4\pi}\int d^3\vec{x}\,
    G\left( \vec{x}, \vec{x}' \right)
    \left\{ 
    \frac{-\rho\left( \vec{x}' \right)}{\epsilon_0}
    \right\}
\end{align}
So this is the general method for solving linear inhomogeneous differential
equations.
$G\left( \vec{x}, \vec{x}' \right)$
is the potential from a charge $q=4\pi\epsilon_0$
located at $\vec{x}=\vec{x}'$.

Consider a charge $q$ near a grounded conducting sphere.
For a point charge $q$,
\begin{align}
    \phi(\vec{x})
    &=
    \frac{1}{4\pi\epsilon_0}
    \left\{ 
    \frac{q}{\left|\vec{x} - \vec{x}'\right|}
    -
    \frac{aq}{\left|\vec{x}'\right|}
    \frac{1}{\left|\vec{x} - \frac{a^2}{\left|\vec{x}'\right|^2}\vec{x}'\right|}
    \right\}
\end{align}
which we obtained by the method of images.
Now consider 
\begin{align}
    \nabla^2 \phi
    &=
    -\frac{q\delta^{(3)}\left( \vec{x} - \vec{x}' \right)}{\epsilon_0}
\end{align}
with boundary conditions
$\phi=0$ at $\left|\vec{x}\right|=a$.
$\phi=0$ at $\left|\vec{x}\right|=\infty$.

Then the Green function is
\begin{align}
    G\left( \vec{x}, \vec{x}' \right)
    &=
    \left\{ 
    \frac{1}{\left|\vec{x} - \vec{x}'\right|}
    -
    \frac{a}{\left|\vec{x}'\right|}
    \frac{1}{\left|
    \vec{x}
    -
    \frac{a^2}{\left|\vec{x}\right|^2}
    \vec{x}'
    \right|}
    \right\}
\end{align}
such that for a more general charge distribution,
\begin{align}
    \phi(\vec{x})
    &=
    \frac{1}{4\pi\epsilon_0}
    \int d^3\vec{x}'
    G\left( \vec{x}, \vec{x}' \right)
    \rho\left(\vec{x}'\right)
\end{align}

We need to generalize the Green function method to problems where $\phI$ or
$\vec{E}\cdot\hat{r}$ are specified on boundaries.
Start from Green's theorem.
\begin{align}
    \int_V
    \left( 
    \phi \nabla^2 \psi
    - \psi \nabla^2 \phi
    \right)
    dV
    &=
    \oint_S
    \left[ 
    \phi\left( \vec{\nabla}\psi\cdot\hat{n} \right)
    - \psi\left( \vec{\nabla}\phi\cdot\hat{n} \right)
    \right]
    dS
\end{align}
To prove this is straightforward.
\begin{align}
    \psi\nabla^2 \psi &=
    \vec{\nabla}\cdot\left( 
    \phi\vec{\nabla}\psi
    \right)
    - \vec{\nabla}\psi
    \cdot \vec{\nabla} \psi\\
    \psi\nabla^2\phi
    &=
    \vec{\nabla}\cdot\left( 
    \psi\vec{\nabla}\phi
    \right)
    -
    \vec{\nabla}\phi
    \cdot\vec{\nabla}\cdot\vec{\nabla}\psi
\end{align}
Subtract
\begin{align}
    \phi\nabla^2\psi
    - \psi\nabla^2\phi
    &=
    \vec{\nabla}\cdot\left( 
    \phi\vec{\nabla}\psi
    -
    \psi\vec{\nabla}\phi
    \right)
\end{align}
Integrate over volume and apply divergence theorem.

For $\phi$ choose the potential.
For $\psi$ choose the Green function $G\left( \vec{x}, \vec{x}' \right)$.

\begin{align}
    \vec{\nabla}'^2 G\left( \vec{x}, \vec{x}' \right)
    &= -4\pi \delta^{(3)}\left( \vec{x}' - \vec{x} \right)
\end{align}
Then
\begin{align}
    \int_V \left[ 
    \phi\left( \vec{x}' \right)
    \underbrace{
    {\nabla'}^{2}
    \psi\left( \vec{x}' \right)
    }_{-4\pi\delta^{3}\left( \vec{x}' - \vec{x} \right)}
    -
    \psi\left( \vec{x}' \right)
    \underbrace{{\nabla'}^2 \phi\left( \vec{x}' \right)}_{
    -\rho\left( \vec{x}' \right)/\epsilon_0
    }
    \right]
    d^3\vec{x}'
    &=
    \int \left[ 
    \phi\left( \vec{x}' \right)
    \left( 
    \vec{\nabla}' \psi\left( \vec{x}' \right) \cdot \hat{n}'
    \right)
    -
    \psi\left( \vec{x}' \right)
    \left( 
    \vec{\nabla}' \phi\left( \vec{x}' \right) \cdot \hat{n}'
    \right)
    \right]
    dS'
\end{align}
So
\begin{align}
    \phi\left( \vec{x} \right)
    &=
    \frac{1}{4\pi\epsilon_0}
    \int_V
    \rho\left( \vec{x}' \right)
    G\left( \vec{x}, \vec{x}' \right)
    d^3\vec{x}'
    +
    \frac{1}{4\pi}
    \oint_S
    \left[ 
    G\left( \vec{x}, \vec{x}' \right)
    \left( 
    \vec{\nabla}' \phi\left( \vec{x}' \right)\cdot\hat{n}'
    \right)
    -
    \phi\left( \vec{x}' \right)
    \left( 
    \vec{\nabla}' G\left( \vec{x}, \vec{x}' \right)
    \cdot \hat{n}'
    \right)
    \right]
    dS'
\end{align}
