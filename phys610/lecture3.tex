\section{Line Integrals}
Under what circumstances is the line integral
$\int_{A}^{B} f(z) \, dz$
independent of the path from $A$ to $B$?

\begin{theorem}[Cauchy]
    If $f(z)$ is analytic in a simply connected,
    bounded domain $D$,
    for every simple closed path $C$ in $D$,
    \begin{align}
        \oint_C f(z) \, dz = 0
        \label{eqn:1}
    \end{align}
\end{theorem}
To prove this, consider
\begin{align}
    \oint f(z)\, dz
    &= \oint_C (u + iv) (dx + i\, dy)\\
    &= \oint_C (u\,dx - v\,dy) +
    i \oint_C (v\,dx + u\,dy)
\end{align}
Then use Stokes theorm
\begin{align}
    \oint\vec{A}\cdot d\vec{r}
    = \int(\vec{\nabla}\times\vec{A})\cdot d\vec{s}.
\end{align}
Let the line integral Lie in hte $xy$ plane.
\begin{align}
    d\vec{r}
    &= dx \hat{e}_x + dy\hat{e}_y\\
    \vec{A} &= A_x \hat{e}_x + A_y \hat{e}_y.
\end{align}
Then
\begin{align}
    \oint_C \left(
        A_x\,dx + A_y\,dy
    \right)
    = \int\left(
    \frac{\partial A_y}{\partial x}
    - \frac{\partial A_x}{\partial y}
    \right)\,dx\,dy
\end{align}
if we go around the path counterclockwise.
Notice this is the same form as before.

Let us apply this to the two line integrals in Equation~\ref{eqn:1}.
For the first integral,
set $A_x = v$ and $A_y = -v$.
Notice that by the Cauchy-Riemann condition,
\begin{align}
    \oint\left(
        u\,dx - v\,dy
    \right)
    =
    \int\left(
    -\frac{\partial v}{\partial x}
    - \frac{\partial u}{\partial y}
    \right)\,dx\,dy
    =0.
\end{align}
For the second integral, set
$A_x = v$ and $A_y = u$.
Then by the other Cauchy-Riemann condition,
\begin{align}
    \oint\left(
        v\,dx + u\,dy
    \right)
    =
    \int\left(
    \frac{\partial u}{\partial x}
    - \frac{\partial v}{\partial y}
    \right)\,dx\,dy
    =0.
\end{align}
Hence
\begin{align}
    \oint_C f(z)\,dz = 0
\end{align}
Suppose $C$ is made up of two paths $C_1$ and $C_2$.
\begin{align}
    \int_{C_1} f(z)\,dz
    - \int_{C_2} f(z)\,dz
    = \oint_C f(z)\,dz
    = 0.
\end{align}
Hence
\begin{align}
    \int_{C_1}f(z)\,dz
    = \int_{C_2}f(z)\,dz
\end{align}
What Cauchy is saying is that this integral is actually a function of the final
point for fixed $A$
\begin{align}
    \int_{A}^{P}f(z)\,dz = F(P).
\end{align}


\section{Cauchy's Integral Formula}
Cauchy is saying if I know the values of an analytic function on the boundary of
a simply connected region in $\mathbf{C}$,
then I can tell you the value of the function anywhere inside the region.

I was shocked when I first saw this theorem.
There's no analogue of this for real functions.

If you consider a complex function
\begin{align}
    f(z) = u(x, y) + i v(x, y).
\end{align}
Then you can solve Laplace's equation in 2D if you know the boundary.
\begin{align}
    \frac{\partial^2 u}{\partial x^2}
     + \frac{\partial^2 v}{\partial y^2}
     = 0.
\end{align}
That's the application we're interested in.

Cauchy's integral formula states that if $f(z)$ is analytic in a simply
connected domain $D$,
then for any point $z=a$ in $D$ and any closed path $C$ in $D$ which encloses
the point $a$,
we have
\begin{align}
    \oint_C \frac{f(z)}{z - a} \, dz = 2\pi i f(a)
\end{align}
where the integration is being taken in the counterclockwise sense.
If you know the values on the boundary then you know the value of any point
inside the region.
The reason it's possible is because $f$ is composed of harmonic functions.
In principle,
if you have the values on the boundary for Laplace's equation,
then in principle you have enough information to solve Laplace's equation in the
bulk.

Define
\begin{align}
    \phi(z) := \frac{f(z)}{z - a}.
\end{align}
This is analytic everywhere in $D$ except at $z=a$.
We need to evaluate
\begin{align}
    \oint_C \phi(z)\, dz.
\end{align}
Let us go counterclockwise.
Then around $a$ draw a small circle,
and I call that circle $C'$.
Then I make a cut between $C$ and $C'$.
Let there be a point $A$ on $C$ and $B$ on $C'$.
Make a cut along $AB$.
Integrate $\phi(z)$ along the closed path from $A$ to $C$ to $B$
around $C'$ and then back to $A$.
The contribution from the cut vanishes since we integrate first in one
direction and then again in the opposite direction.
\begin{align}
    \oint_C \phi(z)\, dz + \oint_{C'}\phi(z)\, dz = 0
\end{align}
where we are going counterclockwise in $C$ and clockwise in $C'$.
Then
\begin{align}
    \oint_{C}\phi(z)\,dz = \oint_{C'}\phi(z)\,dz
\end{align}
when both integrals are performed counterclockwise.
Along the circle $C'$,
we have
$z=a + \rho e^{i\theta}$.
Remember this is a circle and the differential is
\begin{align}
    dz = i\rho e^{i\theta} \,d\theta
\end{align}
So the integral is
\begin{align}
    \oint_{C}\phi(z)\,dz &= \oint_{C'}\phi(z)\,dz\\
    &=
    \oint_{C'}\frac{f(z)}{z - a} \,dz\\
    &= \int_{0}^{2\pi}\frac{f(z)}{\rho e^{i\theta}}\left( 
        i\rho e^{i\theta}
    \right) \,d\theta\\
    &= i\int_{0}^{2\pi} f(z)\, d\theta
\end{align}
but because the circle is so small $f(z)\to f(a)$ and so
\begin{align}
    \oint_C \phi(z)\, dz
    = \oint \frac{f(z)}{z - a}\,dz
    = 2\pi i f(a).
\end{align}

The following is useful for integer $n$.
\begin{align}
    \oint \frac{dz}{{(z - z_0)}^n}
    =
    \begin{cases}
        2\pi i & \text{if } n = 1\\
        0 & \text{otherwise}
    \end{cases}
\end{align}
You can do this by using $z = z_0 + \rho e^{i\theta}$
and doing the integral explicitly on a circle.
If it's not a circle,
use the same trick with the cut like how we proved the Cauchy integral formula.

\section{Liouville's theorem}
Every analytic function is either constant or blows up somewhere in the complex
plane.
\begin{theorem}
    If $f(z)$ is analytic and bounded in absolute value in the entire complex
    plane,
    then it must be a constant.
\end{theorem}
\begin{proof}
    Assume $|f(z)|$ is bounded,
    so that $|f(z)| < M$
    for all $z\in\mathbb{C}$.
    Since $f(z)$ is analytic,
    it can be expanded like
    \begin{align}
        f(z) &= \sum_{n=0}^{\infty} a_n z^{n}.
    \end{align}
    By the Cauchy integral formula,
    \begin{align}
        |a_n|
        &= \left|\frac{1}{2\pi i} \oint \frac{f(z)}{z^{n + 1}}\, dz\right|\\
        &=
        \left|
            \sum_\gamma \frac{1}{2\pi i}
            \frac{f(z)}{z_{\gamma}^{n + 1}}
            \Dlta z_\gamma
        \right|\\
        &\le
        \sum_{\gamma}
        \left|\frac{1}{2\pi i}\right|
        \left|\frac{f(z_\gamma)}{z_\gamma^{n + 1}}\right|
        \left|\Delta z_\gamma \right|\\
        &\le
        \sum_{\gamma}
        \frac{1}{2\pi}
        \frac{f(z)|_{\max}}{R^{n + 1}} |\Delta z_\gamma|\\
        &\le \frac{1}{2\pi} \frac{f(z)|_{\max}}{R^{n + 1}} 2\pi R
    \end{align}
    where the contour is a circle centred at the origin and
    we have used the identities
    \begin{align}
        |ab| &= |a||b|\\
        |a + b| &\le |a| + |b|
    \end{align}
    Hence
    \begin{align}
        |a_n| \le \frac{M}{R^n}
    \end{align}
    for some constant $M$.
    We can take $R$ to be arbitrarily large.
    As $R\to\infty$, $a_n\to 0$ for $n\ge 1$.
    Hence $f(z)=a_0$,
    which is a constant.
\end{proof}
