\section{Spherical Harmonics}
Since $V\to 0$ at $r\to\infty$, then $a_{lm}=0$.
Impose boundary condition at $r=a$.
\begin{align}
    \sum_{l=0}^{n}\sum_{m=-l}^{+l}
    \frac{b_{lm}}{a^{l + 1}} Y_{lm}\left( \theta, \phi \right)
    =
    V_0 \left( \theta, \phi \right)
\end{align}
note
\begin{align}
    V\left( \theta, \phi \right)
    &=
    \sum_{l,m}
    \left( a_{lm}
    r^l
    +
    \frac{b_{lm}}{r^{l + 1}}\right)
    Y_{lm}\left( \theta, \phi \right)
\end{align}
Use orthogonality of spherical harmonics to find $b_{lm}$,
\begin{align}
    \int d\Omega \, Y_{l'm'}^* \left( \theta,\phi \right)
    \left\{ 
    \sum_{l,m}
    \frac{b_{lm}}{a^{l + 1}}
    Y_{lm\left( \theta, \phi \right)}
    \right\}
    &=
    \int d\Omega\,
    Y_{l'm'}^*\left( \theta, \phi \right)
    V_0\left( \theta, \phi \right)
\end{align}
so
\begin{align}
    b_{lm}
    &=
    a^{l + 1}
    \int d\Omega\,
    Y_{lm}^*\left( \theta, \phi \right)
    V_0\left( \theta, \phi \right)
\end{align}
so finally
\begin{align}
    V\left( r, \theta, \phi \right)
    &=
    \sum_{l=0}^{\infty}
    \int_{m=-l}^{+l}
    \left( \frac{a}{r} \right)^{l + 1}
    Y_{lm}\left( \theta, \phi \right)
    \int d\Omega'
    Y_{lm}^*\left( \theta', \phi' \right)
    V_0\left( \theta', \phi' \right)
\end{align}
I'll show you your formula sheet before the exam.
Usually you just express the answer in terms of Legendre polynomials.
Usually you're going to leave your answer in terms of the $Y_{lm}$ except where
there is spherical symmetry and you can write it in terms of Legendre
polynomials.

You can either not bother with $Y_{lm}$ and use Legendre polynomials,
or just use spherical harmonics and reduce it to the Legendre polynomials.
Remember the relation between the spherical harmonics and the Legendre
polynomials.

Let's do another example.
\begin{example}
    Find the Dirichlet Green function for a sphere of radius $b$ centered at
    origin.
    \begin{align}
        \nabla^2 G\left( \vec{r}, \vec{r}' \right)
        &=
        -4\pi \delta^{3}\left( \vec{r} - \vec{r}' \right)
    \end{align}
    such that $G\left( \vec{r}, \vec{r}' \right) = 0$
    at $\left|\vec{r}\right| = b$.
\end{example}
This is a very standard problem.
I have a charge here somewhere in the sphere and I want to find the charge here.
It's Laplace's equation except for that one point,
so we look for a solution like that.

Awa from $\vec{r}=\vec{r}'$,
the Green function satisfies Laplace's equation
\begin{align}
    G\left( \vec{r}, \vec{r}' \right)
    &=
    \sum_{l=0}
    \sum_{m=-l}^{l}
    g_{lm}\left( r \right)
    Y_{l}^{\,m}\left( \theta, \phi \right)
\end{align}
So we look for a separable solution.
Now take this and plug it in to the definition of the Green function.

And then yo get
\begin{align}
    \sum_{l,m}
    \left\{ 
    \frac{d^2 g_{lm}}{dr^2}
    +
    \frac{2}{r}
    \frac{dg_{lm}}{dr}
    -
    \frac{l(l + 1)}{r^2}g_{lm}
    \right\}
    Y_{l}^{\,m}\left( \theta, \phi \right)
    &=
    -4\pi \delta^3 \left( \vec{r} - \vec{r}' \right)
\end{align}
Then we multiply both sides by ${Y^*}_{l'}^{\, m'}\left( \theta, \phi \right)$
and integrate over $\Omega$.
\begin{align}
    \frac{d^2 g_{l'm'}}{dr^2}
    +
    \frac{2}{r} \frac{dg_{l'm'}}{dr}
    -
    \frac{l'\left( l' + 1 \right)}{r^2} g_{l'm'}
    &=
    -4\pi
    \int d\Omega\,
    {Y^*}_{l'}^{\,m'}\left( \theta, \phi \right)
    \underbrace{\delta^3\left( \vec{r} - \vec{r}' \right)}_{
    \frac{\delta\left( r - r' \right)}{r^2}
    \delta\left( \cos\theta - \cos\theta' \right)
    \delta\left( \phi - \phi' \right)
    }\\
    &=
    -\frac{4\pi}{r^2}
    {Y^*}_{l'}^{\,m'}\left( \theta, \phi \right)
\end{align}
This is much simpler because this is just an ODE,
so like the 1D problem we solve.

Away from $r=r'$,
\begin{align}
    g_{lm}\left( r \right)
    &=
    A_{lm} r^l + B_{lm} \frac{1}{r^{l + 1}}
\end{align}
So it's just like Laplace's equation.
The only difference is that you have a sphere and a delta function source
somewhere here.
This source is at $\vec{r}'$.
So we have Laplace's equation everywhere except $\vec{r}'$.
So we're going to get one value of the $A_{lm}$ for $r<r'$
and another value for $r>r'$.
So we are solving Laplace's equation for $r<r'$ and then for $r>r'$
and then we stick the two solutions together.
This is similar to what we did when we were solving for the Greens function for
the box,
where we had a point charge in the middle of the box and we had to divide the
box into two pieces,
which we solved and matched the distributions,
did using a more careful sophisticated way using the convergence theorem,
but here I'm doing it much quicker,
as the two are mathematically equivalent,
and you won't have time in an exam to do all those divergence theorems.
So this is equivalent to what we did.
Here we are breaking the sphere into two pieces,
one $r<r'$ and $r>r'$,
just done much more directly than the box,
and this is much faster,
because very quickly it gets reduced to a 1D ODE.

So in general,
$A_{lm}$ and $B_{lm}$ will be different for $r<r'$
and $r>r'$.
This $1/r^{l+1}$ blows up when $r=0$,
so since $g_{lm}(r)$ must be regular at $r=0$,
we have
\begin{align}
    g_{lm}^{<}\left( r \right)
    &=
    A_{lm}^{<} r^l
\end{align}
which is the solution for $r < r'$.
Also, we need that
\begin{align}
    g_{lm}\left( r \right) = 0
\end{align}
at $r=b$,
so
\begin{align}
    g_{lm}^{>}\left( r \right)
    &=
    A_{lm}^{>}
    \left( 
    r^l
    -
    \frac{b^{2l + 1}}{r^{l + 1}}
    \right)
\end{align}
for $r>r'$.
The solutions are different for $r<r'$ and $r>r'$,
which is similar to the solution for the box
where the solution was different above and below the box
and you had to stick the solutions together at hte boundary.
So now let's stitch the solutions together.

Since solutions must be continuous at $r=r'$,
\begin{align}
    g_{lm}^{<}\left( r' \right)
    &=
    g_{lm}^{>}\left( r' \right)
\end{align}
so you get this condition
\begin{align}
    A_{lm}^{<}
    &=
    A_{lm}^{>}
    \left( 
    1
    -
    \frac{b^{2l + 1}}{r'^{2l + 1}}
    \right)
\end{align}
And then integrate the ODE on both sides over $r$
from $r=r' - \epsilon$ to $r=r' + \epsilon$,
which is the region that contains the delta function.
Then you get
\begin{align}
    \left\frac{dg_{lm}}{dr}\right|_{r' + \epsilon}
    -
    \left\frac{dg_{lm}}{dr}\right|_{r' - \epsilon}
    &=
    -\frac{4\pi}{r'^2}
    {Y^*}_{l}^{m}\left( \theta', \phi' \right)
\end{align}
So this is the second condition.
And this is a lot messier.
\begin{align}
    A_{lm}^{>}
    \left\{ 
    l {r'}^{l - 1}
    +
    \left( l + 1 \right)
    \frac{b^{2l + 1}}{ {r'}^{l + 2}}
    \right\}
    -
    A_{lm}^{,}
    \left\{ 
    l {r'}^{l - 1}
    \right\}
    &=
    - \frac{4\pi}{ {r'}^2 }
    {Y^*}_{m}^{\, l} \left( \theta', \phi' \right)
\end{align}
And then you do some algebra,
and the final answer is
\begin{align}
    G\left( \vec{r}, \vec{r}' \right)
    &=
    \sum_{l,m}
    \frac{4\pi}{\left( 2l + 1 \right)}
    r_{<}^{l}
    r_{>}^{l}
    \left\{ 
    \frac{1}{r_{>}^{2l + 1}}
    -
    \frac{1}{b^{2l + 1}}
    \right\}
    {Y^*}_{l}^{m}\left( \theta', \phi' \right)
    Y_{l}^{\;m}\left( \theta, \phi \right)
\end{align}
