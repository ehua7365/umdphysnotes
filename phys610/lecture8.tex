[30 min late, first zoom lecture]

\section{The Point on Infinity}
The transformation $w=1/z$ establishes a one-to-one correspondence between
points in the $z$-plane and there is the $w$-plane,
except for the singular point at $z=0$.
The resolve this,
we attach an ``improper point'' to the $w$-plane.
This point is called ``the point of infinity''
and is denoted by $\infty$.
The entire complex plane together with the point of infinity is called the
``extended complex plane''.

You need to know what it is and move on.

Certain functions are analytic in the entire complex plane,
they are \emph{entire functions}.
However, they need not be analytic at the point of infinity.
To check whether something is analytic at infinity,
you have to do this mapping $w=1/z$.
Once you do that mapping,
look at how the function behaves at $w=0$.

To check whether a function $g(z)$ is analytic at $\iinfty$,
substitute $w=1/z$ and check analyticity at $w=0$.

\begin{question}
    There is no number line,
    it's a complex plane.
    What does infinity mean in a complex plane?
\end{question}
It's not a simple thing.
The claim is that in the complex plane,
infinity can be represented by a single point,
which is not true in the real line where $+\infty$
is different to $-\infty$.
But in the complex plane,
there is only one $\infty$.
The only reason I am discussing is that you will hear this occasionally so you
just need to have some idea what people are talking about.
It's not like we're actually going to use it for anything.
If you want to know more,
you're going to have to do the reading yourself.
Infinity is a treacherous number,
it's not a simple thing.

Any other questions?
OK let's move on.

\section{Conformal mapping}
Mappings by analytic functions possess and important geometric property known as
\emph{conformal}.
A mapping in the plane is said to be ``conformal''
if it preserves the angle between any two curves,
both in magnitude and orientation.

Let me illustrate that with a picture.
Let's say this is the $z=x+iy$ plane,
with real part $x$ and imaginary part $y$.
Let's say that is the curve $C_1$
and that is the curve $C_2$.
They intersect at a point and
draw their tangents at this point.
Let's say there's an angle $\beta$ between the tangents of the curves at that
point.
Let's look at the map of these curves $z\to w = u + iv$.
Every point of the curve $C_1$ gets mapped to a curve in the $w$ plane $C_1'$.
Every point of the curve $C_2$ gets mapped to a curve in the $w$ plane $C_2'$.
Again, $C_1'$ and $C_2'$ still intersect at a point.
The claim is that the angle between the tangents of the curves in the map is the
same as before, $\beta$.
So yeah.

In other words,
if the mapping is conformal,
the angle between the curves is preserved.
And the claim is that if you do a mapping via an analytic function,
it's automatically conformal.
Mappings with analytic functions are automatically conformal.
It's guaranteed with one caveat I will explain in a moment.

We want to show that the mapping $w=f(z)$
is conformal at every point where $f(z)$ is analytic
except where the derivative $f'(z)=0$.
A point where $f'(z)=0$ is called a \emph{critical point}.

Consider a small change $\Delta z$ at a point $z$.
Then the derivative is
\begin{align}
    \frac{df}{dz} = \frac{dw}{dz}
    = \lim_{\Delta z\to 0} \frac{\Delta w}{\Delta z}
\end{align}
taking the argument which is the angle
\begin{align}
    \arg \lim_{\Delta z\to 0} \frac{\Delta w}{\Delta z}
    = \lim_{\Delta z\to 0} \arg \Delta w
    - \lim_{\Delta z\to 0} \arg \Delta z
    = \arg \frac{df}{dz} =: \alpha
\end{align}
which is a constant for a given point.

Take the ratio of two complex numbers and take the argument to get the angle
between them.

Any line that passes through $z_0$
has its image rotated by an angle $\alpha$.
That means if you had two lines passing through $z_0$,
they would both be rotated by the same angle $\alpha$.
Hence for any pair of lines through $z_0$,
their images in the $w$-plane are rotated through the same angle.
Hence the angle between the lines is the same as in the $z$-plane.

Remember,
you have two curves $C_1$ and $C_2$,
both of which get rotated by the same angle.

You might remember there was a caveat.
At points $z$ where $f'(z)=0$,
the angle $\alpha$ is not well-defined,
and the argument fails.

That's the caveat.
Any analytic function where there is a region where the derivative doesn't
vanish,
if you have this property of conformality,
the image of any curves is rotated by a fixed angle at that point,
so the angle between pairs of curves is preserved.

\begin{question}
    What do you mean by orientation?
\end{question}
What I mean is that you are rotated by this angle $\alpha$ in the same
direction.

This always ends up taking a bit longer than I expect.

So we agree that the angle is preserved.
But what about relative sizes?

From the definition of a derivative,
we have
\begin{align}
    \lim_{z\to z_0}
    \underbrace{\left|
        \frac{f(z) - f(z_0)}{z - z_0}
    \right|}_{
        = \frac{|f(z) - f(z_0)|}{|z - z_0|}
    }
    =
    |f'(z_0)|
\end{align}
Let's think about what this means.
$|z - z_0|$ is the length of $\Delta z$
and $|f(z) - f(z_0)|$ is the length between the image points.
So in the limit $\Delta z\to 0$,
the denominator is the distance in the $z$ plane,
but the numerator is the distance between the same two points in the $w$ plane.
That means there's a rescaling between the points and the images.
So if you have a distance between two points,
that distance is rescaled by a factor
which only depends on $z_0$.

That means if you have a curve that passes through a point,
in the neighbourhood of hat point,
the image is rescaled by that factor.
If you have multiple curves passing through that point,
all the curves are rescaled by that factor.
You also know they're rotated.
So every cure is rotated by some amount
and rescaled by some amount
in that neighbourhood.

I understand it's not easy to observe these things.
If I write everything down,
I can go on and on,
so let me explain again from scratch.

We want $z-z_0$ to go infinitesimal.
The denominator is the distance between two points in the $z$-plane.

Let me write this.
The mapping $w=f(z)$ rescales the lengths of infinitesimal lines passing through
$z_0$ by the factor $|f'(z_0)|$.
Therefore the image of a small figure always has the same shape as the original
figure,
even though it has been rescaled and rotated.
However, a large figure may have an image with a completely different shape.

\section{Conformal Mapping and Harmonic Functions}
OK so here is the claim,
which I proved in my notes,
but the proof is tedious and I wonder if I want to do it.
I will state the claim and give and outline of the proof,
and you can have the pleasure of filling in the proof.

\begin{theorem}
    A harmonic function of two variables $(x, y)$
    remains harmonic under a change of variables from $(x, y)$
    to $(u, v)$ if $z = x + iy$ and $w= u + iv$
    are related by a conformal transformation.
\end{theorem}
I'll explain in a moment what that means.

OK so here's the idea.
Let $h$ be a harmonic function.
What that means is
\begin{align}
    \frac{\partial^2 h}{\partial x^2}
    + \frac{\partial^2 h}{\partial y^2} = 0.
\end{align}
The claim is that if $u=u(x, y)$ and $v=v(x, y)$
such that $w = f(z)$,
$w=u+iv$ and $z=x+iy$.
Then
\begin{align}
    \frac{\partial^2 h}{\partial u^2}
    + \frac{\partial^2 h}{\partial v^2} =0
\end{align}
provided $f'(z)\ne 0$.

This is a very useful result,
used to prove lots of results in fluid dynamics and electrostatics.
You're going to apply this to electrostatics.
Unfortunately,
it's a trick that only works in 2D.
Unfortunately it doesn't help you solve Laplace's equation in arbitrary
dimensions,
but it is a powerful method in 2D.

I look at my notes and I actually did prove this.
Let me just outline how it goes.
The proof is kind of like brute force.
There are more clever proofs in the books I suggested,
but all of them rely on some additional assumptions or rely on some theorem I
didn't prove.

\begin{proof}
    Write using the chain rule
    \begin{align}
        \frac{\partial h}{\partial x} &=
        \frac{\partial h}{\partial u} \frac{\partial u}{\partial x}
        + \frac{\partial h}{\partial v} \frac{\partial v}{\partial x}
    \end{align}
    then differentiate again
    \begin{align}
        \frac{\partial^2 h}{\partial x^2} &=
        \left\{
            \frac{\partial^2 h}{\partial u^2}
            \left(\frac{\partial u}{\partial x} \right)^2
            +
            \frac{\partial^2 h}{\partial u \partial v}
            \left( \frac{\partial u}{\partial x} \right)
            \left( \frac{\partial v}{\partial x} \right)
            + \frac{\partial h}{\partial u}
            \frac{\partial^2 u}{\partial x^2}
        \right\}\\\nonumber
        &\qquad + 
        \left\{
            \frac{\partial^2 h}{\partial v^2}
            \left(\frac{\partial v}{\partial x} \right)^2
            +
            \frac{\partial^2 h}{\partial u \partial v}
            \left( \frac{\partial u}{\partial x} \right)
            \left( \frac{\partial v}{\partial x} \right)
            + \frac{\partial h}{\partial v}
            \frac{\partial^2 v}{\partial x^2}
        \right\}
    \end{align}
    Similarly,
    \begin{align}
        \frac{\partial^2 h}{\partial y} = \cdots
    \end{align}
    and it gets messy\ldots
    Then you have to use the Cauchy-Riemann equations to cancel out some terms.
    Then you use the fact that $u$ and $v$ are harmonic functions themselves.
    After a lot of algebra,
    you get
    \begin{align}
        \frac{\partial^2 h}{\partial x^2} +
        \frac{\partial^2 h}{\partial y^2} &=
        \left(
            \frac{\partial^2 h}{\partial u^2}
            + \frac{\partial^2 h}{\partial v^2}
        \right)
        |f'(z)|^2
        = 0
    \end{align}
    where we need the assumption $f'(z)\ne 0$.
    Then
    \begin{align}
        \frac{\partial^2 h}{\partial u^2}
        + \frac{\partial^2 h}{\partial v^2}
        = 0
    \end{align}
\end{proof}
