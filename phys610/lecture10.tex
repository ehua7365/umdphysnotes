\section{Method of Steepest Descent}
This is a method of approximating complex line integrals.
Let us first consider the simpler case of a real integral.
We look for an approximation to the Gamma function
$\Gamma(x + 1)$
when $x$ is large,
starting from its definition as an integral
\begin{align}
    \Gamma(x + 1)
    &=
    \int_{0}^{\infty} t^x e^{-t}\, dt
\end{align}
We want to get an approximate integral in the limit that $x$ is very large.
OK so how do we begin?

The first thing is note that the integrand has a maximum when
$t=x$.
To see this, note the integrand is extreme when
\begin{align}
    \frac{d}{dt}\left(
        t^x e^{-t}
    \right) = 0
\end{align}
at $t=t_0$.
Solving $t_0=x$,
for large $x$,
the integrand is very highly peaked around $t=t_0$.

So to a good approximation,
the value of the integral is dominated by they region around the peak.
For large $x$,
the contribution of the integral is dominated by the region around the peak.
For a first pass,
we can just consider the region around the peak and discard everything else.
That's the approximation we make.
It's similar to what we do when we look at the method of steepest descent.

The first thing to do is to write the integrand in an exponential form,
like $e^{f(t)}$
and expand $f(t)$ in a Taylor series
about the maximum.

For example,
\begin{align}
    e^{f(t)} &=
    t^{x} e^{-t}
    = e^{x \log t - t}
\end{align}
Then
\begin{align}
    f(t) &= x\log t - t\\
    f'(t) &= \frac{x}{t} - 1\\
    f''(t) &= -\frac{x}{t^2}
\end{align}
Then,
expanding about $t=x$,
we have
\begin{align}
    f(t) &=
    (x \log x - x)
    - \frac{1}{2}
    \frac{1}{x} {(t - x)}^2 + \cdots
\end{align}
Now we approximate the integral by contribution from the region around the
maximum,
\begin{align}
    \Gamma(x + 1) &\approx
    \int_{0}^{\infty} dt\,
    \exp\left\{
    x\log x - x \frac{1}{2x}{(t - x)}^2
    \right\}\\
    &\approx
    e^{x\log x - x}\int_{-\infty}^{\infty}\exp\left\{
    -\frac{1}{2x}{(t - x)}^2
    \right\}
\end{align}
The error in changing the integral lower bound from 0 to $\infty$
is negligible
and it's a Gaussian integral so you get
\begin{align}
    \Gamma(x + 1) \approx
    e^{x \log x - x}\sqrt{2\pi x} = \sqrt{2\pi x} x^x e^{-x}
\end{align}
This is the first term in Stirling's formula.

The method of steepest descent is generally applied to integrals of the form
\begin{align}
    I(s) &=
    \int_C g(z) e^{s f(z)} dz
\end{align}
where $s$ is real, large and positive
and $f(z)$ and $g(z)$ are analytic functions.
This restriction on $s$ being real is not important because if it were 
complex,
you can pull out the imaginary part and sign
and absorb it into $g$ and $f$.
Assume you've already done that.

For $f(z)=u& + iv$
we expect most of the contribution to the integral to come from the region
where $u$ is large.
But by Cauchy's theorem,
we have to freedom to move the contour,
so the idea is the move the contour to pass through where the value is
peaked.

The idea behind the method of steepest descent it to choose the contour $C$
so that it runs over a region where $u$ is large and highly peaked
so that $I(s)$ is dominated by the contribution from this region.

Now we come to the complication.
Remember, you had to prove that neither $u$ nor $v$
can have a maximum or a minimum in that region.
All you have are saddle points.
But actually this turns out not to matter.
The question is not what is the global maximum of this function.
All we have to do is choose a path such that it has a maximum
\emph{along} that path.
When going along a path,
you approach a point from only one direction.
So choose your path such that the path has a maximum along the path,
and the integral is dominated by the contribution from that peak.
On the real line,
there's only one path of integral,
either you have a peak or you don't.
In the complex plane,
there is no maximum,
but you have the flexibility to choose the path,
so choose the path so that along that path you have a maximum,
and the integral will be dominated by the region around that maximum
on the path.

So let's write that down.

However, because $f(z)$ is an analytic function,
neighbour $u$ nor $v$ can have a maximum or a minimum

Any extrema of $u(x, y)$ where
\begin{align}
    \frac{\partial u}{\partial x}
    &=
    \frac{\partial u}{\partial y} = 0
\end{align}
must be a saddle point,
where the surface looks like a saddle or a mountain pass.
At any such point,
we also have
\begin{align}
    \frac{\partial v}{\partial x}
    =
    \frac{\partial v}{\partial y} = 0
\end{align}
because of the Cauchy-Riemann conditions.

At the saddle point, we have
$f'(z) = 0$.
Then we can approximate $f(z)$ near the saddle point,
\begin{align}
    f(z) &=
    f(z_0)
    + \frac{1}{2} f''(z_0){(z - z_0)}^2.
\end{align}
Here $z=z_0$ is the saddle point.

\begin{align}
    f''(z_0) &= |f''(z_0)| e^{i\theta}\\
    z - z_0 &= |z - z_0| e^{i\alpha}
\end{align}
and so approximate
\begin{align}
    u(x, y) &\approx
    u(x_0, y_0) + \frac{1}{2}|f''(z_0)|
    |z - z_0|^2 \cos(\theta + 2\alpha)\\
    v(x, y) &\approx
    v(x_0, y_0) + \frac{1}{2}|f''(z_0)|
    |z - z_0|^2 \sin(\theta + 2\alpha)\\
\end{align}
So this $\alpha$ controls the direction that you're approaching the saddle
point.
You can choose $\alpha$ such that as you leave the saddle point,
$u$ is increasing,
or you can choose $\alpha$ such that as you leave the saddle point
$u$ is decreasing.
I think that's clear.

On the surface $u(x, y)$,
the part of paths of steepest decent from the saddle point
into the valleys are those for which $\cos(\theta + 2\alpha)=-1$,
so that
\begin{align}
    \alpha &= - \frac{\theta}{2} \pm \frac{\pi}{2}
\end{align}
We can pick a path that rises very fast and falls very fast.
Why do we approach along this direction?
We want the path of steepest descent.

Along these directions,
$\sin(\theta + 2\alpha)=0$
so $v(x, y)$ is constant.
Then the oscillating factor $e^{isv}$
in the integrand will not produce large cancellations
along that line.
This is pretty important.
This $e^{isv}$ is a phase factor,
and this phase can be positive or negative,
so it doesn't help if $u$ is huge and this phase factor is
oscillating like crazy.

It's very important that  there are no large oscillations in this phase as we
integrate along this saddle point.
Everything is working out just the way we want.
$e^{u}$ is large about that specific point,
and there are no large cancellations coming from that phase.

In contrast,
the paths for which
$\cos(\theta + 2\alpha)=1$
are the paths of steepest ascent.
Those are the directions along which $u$ is growing as opposed to falling
most quickly as you leave the saddle point.
Along those directions,
it's like you are going along a mountain ridge.
These paths are at right angles with respect to path of steepest descent.

We now deform the contour of integration so as to pass over the saddle point
along the path of steepest descent.
Since the saddle point is a maximum of $u(x, y)$
\emph{along this path},
for large $s$ we expect that the contribution to the integral will be dominated
by the region around the saddle point.

Writing $z = z_0 + l e^{i\alpha}$,
\begin{align}
    I(s) &= \int_{C'}
    g\left( z_0 + l e^{i\alpha} \right)
    \exp\left[
        s f(z_0 + l e^{i\alpha})
    \right]
    e^{i\alpha}\, dl
\end{align}
So now we approximate by Taylor expanding $f$ and $g$ around the point.
\begin{align}
    f\left( z_0 + l e^{i\alpha} \right) &\approx
    f(z_0) + \frac{1}{2}f''\left( z_0 \right) l^2 e^{2i\alpha}
    + \cdots\\
    g\left( z_0 + l e^{i\alpha} \right) &\approx
    g(z_0) + \frac{1}{2}g''\left( z_0 \right) l e^{i\alpha}
    + \cdots\\
\end{align}
Assuming $g(z)$ is slowly varying,
we can approximate
\begin{align}
    I(s) &\approx
    g\left( z_0 \right)
    e^{s f\left(z_0 \right)}
    e^{i\alpha}
    \int_{-\infty}^{\infty}
    e^{-\frac{1}{2}\left|f''\left( z_0 \right)\right| sl^2}
    \,dl
\end{align}
Now this is a Gaussian integral,
and we changed the limits to $(-\infty, \infty)$,
but it's not going to change the answer significantly.
And when the dust settles
\begin{align}
    I(s) &\approx
    \frac{%
        \sqrt{2\pi} g(z_0) e^{sf(z_0)} e^{i\alpha}
    }{%
        \sqrt{s |f''(z_0)|}
    }
\end{align}
This is the one big formula you're going to have to remember when you take the
exam,
because you're not going to be allowed to bring cheatsheets.

You have to remember $\alpha$ is related to $\theta$ in that way
and $\theta$ is the phase of $f''$.

Now there is one complication,
which applies in the real world,
but probably not your homework problems.

Recall that there are 2 possible values of the angle $\alpha$,
which differ by $\pi$.
Which value of $\alpha$ should we choose?
\begin{align}
    \alpha &= -\frac{\theta}{2} \pm \frac{\pi}{2}
\end{align}
That depends on the direction we cross the saddle point.
If the direction is reversed,
the value of the integral will flip sign.

Let me explain what I mean.
Let's say you have to integrate from $A$ to $B$.
Let's say the saddle point is somewhere in the middle.
You want to cross the saddle point along that direction from $A$ to $B$.
And so this would be the angle $\alpha$.

[picture]

But suppose $B$ is here and $A$ is here.

[picture]

Now you have to go in the opposite direction.
The angle $\alpha$ is this angle,
which is different from the original $\alpha$ by $\pi$.
Depending on where you're starting or ending,
you're going through the saddle point in a different direction,
in which case you get a different value of $\alpha$.
So it's either you can go this way or that way,
and it's just a minus sign difference.
You just reverse the direction you're integration.

So that ambiguity is hidden here:
$\alpha = -\frac{\theta}{2} \pm
\frac{\pi}{2}$.

Sometimes you can have weird contours.
In the real world,
you need to create the full ridge.
Make sure you only cross it once over the direction you chose.
Remember if you're signing a paper,
you're putting your reputation on the line,
and these subtleties do show up.
This is not going to be in the homework of course.

I have 2 examples in the notes,
but I don't have time to go other both of them.
Let's do the shorter example because we don't have a whole bunch of time.
Most of what I talk about is based on Matthews and Walker.

\begin{example}
    The Hankel function of the first kind,
    which is a solution of the Bessel equation,
    may be represented by the contour integral
    \begin{align}
        H_{\nu}^{(1)}(s) &=
        \frac{1}{\pi i}
        \int_{\epsilon}^{\infty e^{i\pi}}
        e^{\frac{s}{2}\left( z - \frac{1}{z} \right)}
        \frac{dz}{z^{\nu + 1}}
    \end{align}
    where the $(1)$ means first kind.
    Here $\epsilon\to 0$, and the contour is shown in the figure.
    There is a branch cut because of the $1/z^{\nu + 1}$,
    and you choose the branch cut to go along the negative real axis.
    The contour to integrate over looks like this:
    Start from $\epsilon$ then spiral anticlockwise around to negative
    $-\infty$ staying above the branch cut.
\end{example}
This is a solution to the Bessel equation,
because if you substitute it in,
the integrand doesn't satsify it,
but inside the integral,
you take a total derivative and these limits are cunningly chosen so that at the
limits the derivative piece once you do the integration vanishes.
We won't have to discuss that,
since this course doesn't cover special functions at all.
We won't have time to discuss,
but I can just outline.
The thing you're left with is a total derivative,
and since it's sitting on a derivative,
the endpoints you're left with are 0 at the end points.
There is an extended discussion Arfken.

Our problem is just to find an approximation to this integral in the region of
large $s$.

The integrand has a branch cut along the negative $x$-axis
and the contour is to be taken from $\epsilon$ to $-\infty$
above this branch cut.

We want to approximate this integral,
so we have a formula for the method of steepest descent.
The integral we began from is this integral
\begin{align}
    I(s) &=
    \int_C g(z) e^{s f(z)} dz
    \approx
    \frac{%
        \sqrt{2\pi} g(z_0) e^{sf(z_0)} e^{i\alpha}
    }{%
        \sqrt{s |f''(z_0)|}
    }
\end{align}
So we choose
\begin{align}
    g(z) &= \frac{1}{z^{\nu + 1}}\\
    f(z) &= \frac{1}{2}\left( z - \frac{1}{z} \right)
\end{align}
Take derivatives
\begin{align}
    f'(z) &= \frac{1}{2}\left( 1 + \frac{1}{z^2} \right)\\
    f''(z) &= -\frac{i}{z^3}
\end{align}
So there are saddle points at $z = \pm i$,
but we want $z_0 = +i$
because that's above the branch cut.

\begin{align}
    f(z_0) &= \frac{1}{2}\left( i - \frac{1}{i} \right) = 1\\
    f''(z_0) &= -\frac{1}{i^3} = -i
\end{align}
Then
\begin{align}
    f''(z_0) &= |f''(z_0)| e^{i\pi/2}
\end{align}
so
\begin{align}
    \theta = - \frac{\pi}{2}
\end{align}
so we get
\begin{align}
    \alpha &= - \frac{\theta}{2} \pm \frac{\pi}{2}\\
    &= \frac{3\pi}{4} \quad \text{or} -\frac{\pi}{4}
\end{align}
Because of the direction we're going,
the correct angle is
\begin{align}
    \alpha=\frac{3\pi}{4}.
\end{align}
Then,
\begin{align}
    I(s) &\approx
    \frac{%
        \sqrt{2\pi} g(z_0) e^{sf(z_0)} e^{i\alpha}
    }{%
        \sqrt{s |f''(z_0)|}
    }\\
    &=
    \sqrt{\frac{2}{\pi s}}
    e^{i\left(
        s - \nu \frac{\pi}{2} + \frac{\pi}{4}
    \right)}
\end{align}
So, any questions?

We're done for the mathematical methods part of the course.

Next lecture,
we're back to in person,
and we will begin the classical mechanics part of the course.

I'll see you next week.
