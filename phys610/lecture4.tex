\section{Laurent Series}
Let $c_1$ and $c_2$ be two circles centred at $Z_0$.
Let $f(z)$ be analytic in the region $R$ between the circles.
Then $f(z)$ can be expressed in a series of the form.
\begin{align}
    f(z) =
    \left[
        a_0 + a_1(z - z_0) + a_2(z - z_0)^2 + \cdots
    \right]
    + \left[
        \frac{b_1}{z - z_0}
        + \frac{b_2}{(z - z_0)^2}
        + \cdots
    \right]
\end{align}
The series converges and represents $f(z)$ in the open annulus obtained from the
given annulus by continuously increasing the circle $C_1$ and decreasing $C_2$
until each of the two circles reaches a point where $f(z)$ is singular.

[picture of two annulii]

Note that $f(z)$ just has to be analytic inside the region between the two
circles,
but not necessarily outside.
This series has two parts.
The first part looks like a Taylor series.
The second part is different because it has negative powers.
Both these series have in principle infinitely many terms,
but they could terminate,
not that they have to.
These are the Laurent series.
I'm not going to prove this,
but the proof is in the book.
It's not that difficult.
If I had time I would prove it, but we don't.

\begin{question}
    The converse is not true?
    If you have a function that is represented by this formula is it analytic?
\end{question}
Good question I don't know the answer.

\begin{question}
    Is it just for that region?
\end{question}
Yes, it's not guaranteed to work.

$f(z)$ is assumed to be analytic in this region.
You could keep increasing the size of the outer circle radius
and at some point you may hit a singularity.
Putting it in this form is fine,
and you can similarly keep reducing the size of the inner circle until you
hit a singularity.
And that's as far as you can get this formula to work.

This series has a name, \emph{Laurent series}.
This part is called the \emph{principal part} of the Laurent series.
\begin{align}
    \left[
        \frac{b_1}{z - z_0}
        + \frac{b_2}{{(z - z_0)}^2}
        + \cdots
    \right]
\end{align}
It's a standard problem in complex analysis to find the Laurent series of a
function.
It's a very standard problem.
We have to do a few examples of Laurent series.

Formally,
we can find the coefficients $a_n$ and $b_n$ with
\begin{align}
    a_n &=
    \frac{1}{2\pi i}
    \oint \frac{f(z)}{ {(z - z_0)}^{n + 1}}\, dz\\
    b_n &=
    \frac{1}{2\pi i}
    \oint \frac{f(z)}{ {(z - z_0)}^{-n + 1}}\, dz
\end{align}
This does not quite come from Cauchy theorem,
but actually this uses this result from the last class
\begin{align}
    \oint \frac{dz}{{(z - z_0)}^n}
    =
    \begin{cases}
        2\pi i & \text{if } n = 1\\
        0 & \text{otherwise}
    \end{cases}
\end{align}
and you can see only one term survives when you multiply by the appropriate
power of ${(z - z_0)}^n$ and integrate.
Formally this is the way to find the coefficients,
but in practice there are easier ways to find the coefficients.

\begin{example}
    Find the Laurent series of $f(z) = z^2 e^{1/z}$ centred at $z = 0$.
\end{example}
\begin{proof}[Solution]
    Using the expansion
    \begin{align}
        e^{z} &=
        1
        + z
        + \frac{z^2}{2!}
        + \frac{z^3}{3!}
        + \cdots
    \end{align}
    which has an infinite radius of convergence,
    we can write
    \begin{align}
        z^2 e^{1/z}
        &= z^2\left\{
            1
            + \frac{1}{z}
            + \frac{1}{2!}\left(\frac{1}{z} \right)^2
            + \frac{1}{3!}\left(\frac{1}{z} \right)^3
            + \cdots
        \right\}\\
        &= z^2 + z + \frac{1}{z} + \frac{1}{3!}\frac{1}{z} + \cdots
    \end{align}
    This converges for $|z|>0$.
\end{proof}
Let's do another example.
\begin{example}
    Find the Laurent series of
    \begin{align}
        f(z) = \frac{1}{1 - z^2}
    \end{align}
    centred at $z=1$.
\end{example}
\begin{proof}[Solution]
    Before you do any work you should find out where the singularities are.
    The singular points are $+1$ and $-1$.
    Notice that we're asked to find the Laurent series at the singular point
    $z=1$.
    There will be a Laurent series that is valid inside a circle of radius 2
    about $z=1$.
    And then there will be a second Laurent series outside radius 2.
    So there are two separate Laurent series.
    If someone tells you to find the Laurent series,
    you need to find all the Laurent series there are.
    That's why the first thing to do is stare at it and figure out where the
    singularities are and thus how many to find.

    You guys suggested breaking into partial fractions like this
    \begin{align}
        \frac{1}{1 - z^2}
        &=
        \frac{1}{2}\left[
            \frac{1}{1 + z}
            + \frac{1}{1 - z}
        \right]
    \end{align}
    Let me tell you why this is not a good idea.
    This is easier to do
    \begin{align}
        \frac{1}{1 - z^2}
        &= \frac{1}{(1 - z)(1 + z)}
        = \frac{1}{1 - z}
        \underbrace{\left\{ \frac{1}{1 + z} \right\}}_{
            \text{expand about }z=1
        }.
    \end{align}
    So let's write it like this.
    \begin{align}
        \frac{1}{1 + z}
        &= \frac{1}{2 + (z - 1)}\\
        &= \frac{1}{2}\left\{
            \frac{1}{1 + \frac{z - 1}{2}}
        \right\}
    \end{align}
    Then we expand using the binomial theorem to get
    \begin{align}
        \frac{1}{1 + z}
        &= \frac{1}{2}\sum_n (-1)^n
        \left(\frac{z - 1}{2} \right)^n\\
        &= \sum_n (-1)^n \frac{(z - 1)^n}{2^{n + 1}}
    \end{align}
    This series converges if $\left|\frac{z - 1}{2}\right| < 1$,
    which is $|z - 1|<2$.
    Hence we have
    \begin{align}
        \frac{1}{1 - z^2}
        &= \left(\frac{-1}{z - 1}\right)
        \sum_n
        \frac{(-1)^n}{2^{n + 1}}
        (z - 1)^n\\
        &= \sum_{n=0}^{\infty} \frac{(-1)^{n + 1}}{2^{n + 1}}
        (z - 1)^{n - 1}
    \end{align}
    which is the Laurent series for $|z - 1|<2$.
    Now we need to find the other one for $|z - 1| > 2$.
    We just need a slightly different binomial expansion.
    What we do is full out an additional factor of $(z-1)/2$ like this
    \begin{align}
        \frac{1}{1 + z} &=
        \frac{1}{2}\left\{
            \frac{1}{1 + \left(\frac{z - 1}{2} \right)}
        \right\}\\
        &= \frac{1}{2} \frac{1}{\left(\frac{z - 1}{2} \right)}\left\{
            \frac{1}{\left(\frac{2}{z - 1} + 1 \right)}
        \right\}\\
        &= \frac{1}{z - 1} \frac{1}{1 + \left( \frac{2}{z - 1} \right)}\\
        &= \frac{1}{z - 1}\sum_{n = 0}^{\infty} (-1)^{n}\left(
        \frac{2}{z - 1}
        \right)^n \qquad\text{converges if }|z - 1| > 2\\
        &= \sum_{n=0}^{\infty} (-1)^n \frac{2^n}{(z - 1)^{n+1}}
    \end{align}
    Hence the Laurent series for $|z - 1|>2$ is
    \begin{align}
        \frac{1}{1 - z^2}
        &= \sum_{n=0}^{\infty} (-1)^n \frac{2^n}{(z - 1)^{n+2}}
    \end{align}
\end{proof}
You can usually use this binomial trick for rational functions.

Let's do another example.
\begin{example}
    Find the Laurent series of
    \begin{align}
        f(z) &= \frac{12}{z(2 - z)(1 + z)}
    \end{align}
\end{example}
\begin{proof}[Solution]
    First thing to do is find the singular points.
    They are $z=-1$, $z=0$, $z=2$.
    So we have 2 circles we can draw about $z=0$
    and 3 regions to evaluate Laurent series on.
    So we need to get 3 Laurent series.

    If you have a problem like this in the exam,
    you should draw this picture and say how many Laurent series there are and
    you will get some easy points.
    You might make a mistake in the binomial expansion,
    but at least you'll get some points here.

    For this we actually have to do some work to find the Laurent series.
    We want expansion in powers of $z$.
    The $1/z$ is already in the right form, so factorise that out.
    Then we break it out into partial fractions.
    \begin{align}
        f(z) &=
        \frac{1}{z}\left\{
            \frac{12}{(2 - z)(1 + z)}
        \right\}\\
        &= \frac{4}{z}\left\{
            \frac{1}{2 - z}
            + \frac{1}{1 + z}
        \right\}
    \end{align}
    so now we have broken it up into partial fractions.
    Notice each term looks like what we had before,
    so we can keep them separate and then add.
    Let's first find the Laurent series for the region $|z|<1$,
    where we expand binomially.
    Note the geometric series
    \begin{align}
        \frac{1}{1 - z} = 1 + z + z^2 + \cdots \qquad
        \text{for }|z| < 1
    \end{align}
    is a special case of the binomial expansion.
    It's useful to know the more general form.
    Anyway,
    \begin{align}
        \frac{1}{1 + z} =
        \sum_{n=0}^{\infty}(-1)^n z^n
    \end{align}
    and
    \begin{align}
        \frac{1}{2 - z} &=
        \frac{1}{2} \frac{1}{1 - \frac{z}{2}}\\
        &= \frac{1}{2}\sum_{n=0}^{\infty}\left( \frac{z}{2} \right)^n\\
        &= \sum_{n=0}^{\infty}\frac{z^n}{2^{n + 1}}.
    \end{align}
    Then for $|z|<1$, the Laurent series is
    \begin{align}
        f(z) &= \frac{4}{z}\left\{
            \sum_{n=0}^{\infty}(-1)^n
            + \sum_{n=0}^{\infty} \frac{z^n}{2^{n + 1}}
        \right\}\\
        &= \frac{6}{z} - 3 + \frac{9}{2}z - \frac{15}{4} z^2 + \cdots.
    \end{align}
    We could find the Laurent series in the intermediate region,
    but I like to find the center region first,
    then the outer-most region second,
    and then the intermediate regions last.
    You'll see why in a minute.

    For $|z|>2$,
    use the binomial expansion once again.
    \begin{align}
        \frac{1}{1 + z} &=
        \frac{1}{z} \frac{1}{1 + \frac{1}{z}}\\
        &= \frac{1}{z} \sum_{n=0}^{\infty} (-1)^n \left(
        \frac{1}{z}
        \right)^n\\
        &= \sum_{n=0}^{\infty}(-1)^{n} \frac{1}{z^{n+1}}.
    \end{align}
    which is valid for $|z| > 1$.
    Then for the other term
    \begin{align}
        \frac{1}{2 - z} &=
        \frac{1}{z} \frac{-1}{1 - \frac{2}{z}}\\
        &= -\frac{1}{z}\left\{
        \sum_{n=0}^{\infty}\left(\frac{2}{z} \right)^n
        \right\}\\
        &= -\sum_{n=0}^{\infty}\frac{2^n}{z^{n+1}}
    \end{align}
    which is valid for $|z|>2$.
    Putting the pieces together,
    \begin{align}
        f(z) &=
        \frac{4}{z}j\left\{ 
        \sum_{n=0}^{\infty} \frac{-2^{n}}{z^{n+1}}
        + \frac{(-1)^n}{z^{n+1}}
        \right\}\\
        &= -\frac{12}{z} - \frac{12}{z^4} - \frac{36}{z^5} -
        \frac{60}{z^6} + \cdots
    \end{align}
    which is the Laurent series for $|z|>2$.

    There's one Laurent series left in the middle.
    The reason I left is the following.
    The binomial expansion is what is different.
    We can no longer use the expansion $1/(1 - 2/z)=\sum_{n=0}^n(2/z)^n$,
    but we did have a similar expansion for $1/(1 - z/2)$ that is valid for
    $|z|<2$, which we have previously computed.
    That is why I left it till last, 
    so we can just reuse the binomial expansion for the $1/(2 - z)$ term
    \begin{align}
        \frac{1}{2 - z} &= \sum_{n=0}^{\infty}\frac{z^n}{2^{n+1}}.
    \end{align}
    for $|z|<2$.
    So then
    \begin{align}
        f(z) &= \frac{4}{z}\sum_{n=0}^{\infty}\left\{
            (-1)^n \frac{1}{z^{n+1}}
            + \frac{z^n}{2^{n+1}}
        \right\}\\
        &= \left(
            1 + \frac{z}{2} + \frac{z^2}{4} + \frac{z^3}{8}
        \right)
        + \frac{2}{z}
        + \left(
        \frac{4}{z^2} - \frac{4}{z^3} + \frac{4}{z^4} - \cdots
        \right)
    \end{align}
    which is the Laurent series for $1 < |z| < 2$.
\end{proof}

Now time for some definitions.
\begin{definition}[analytic, regular point]
    If all the $b_n$ are zero in the Laurent expansion that is valid close to
    the center $z_0$,
    then $f(z)$ is \emph{analytic} at $z_0$
    and $z_0$ is a \emph{regular point}
\end{definition}
If it's analytic at $z_0$,
thep principal part of the Laurent series will vanish
and it will just be a Taylor series.
\begin{definition}[pole]
    If $b_n\ne 0$,
    but all the $b$'s after $b_n$ are zero,
    then $f(z)$ is said to have a \emph{pole of order} $n$.
    If $n=1$,
    we say $f(z)$ has a \emph{simple pole} at $z=z_0$.
\end{definition}
These are just definitions.

\begin{definition}[essential singularity]
    If an infinite number of $b$'s are non-zero,
    we say that $f(z)$ has an \emph{essential singularity}
    at $z=z_0$.
\end{definition}

\begin{definition}[residue]
    The coefficient $b_1$ of $\frac{1}{z - z_0}$ is called the
    \emph{residue} of $f(z)$ at $z=z_0$.
\end{definition}

You should look at the examples to see if it was an essential or not.
