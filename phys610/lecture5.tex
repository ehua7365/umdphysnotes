\section{Residue Theorem}
Today I want to talk about how to evaluate integrals using the residue theorem.

Let $z_0$ be an isolated singular point of $f(z)$.
Consider the value of the closed line integral $\oint_C f(z) dz$
around a simple closed curve $C$ surrounding $z_0$
but enclosing no other singularities.
The little circle means it's a closed line as opposed to an open line.
It means it comes back to where you started.
Let $f(z)$ be expanded in a Laurent series about $z=z_0$ that converges near
$z=z_0$.

Then,
\begin{align}
    f(z) = a_0 + a_1(z - z_0) + \cdots
    + \frac{b_1}{z - z_0} + \frac{b_2}{(z - z_0)^2} + \cdots.
\end{align}

\begin{question}
    What is converging near $z_0$?
\end{question}
It means converging in the neighbourhood around that point.
This is important because you could have several Laurent series about a point
that converge in different regions.
We want the series that converges near the point we're expanding about.

The point of the residue theorem is that there are an infinite number of $b$'s,
but $b_1$ is special.
Out of all the infinite terms, this $b_1$ is special.
This $b_1$ is called the \emph{residue}.
That one is special.

The terms in the $a$ series do not contribute to the integral,
because they are analytic.
The thing is, of the terms in the $b$ series,
only $b_1$ contributes.
This is because of this identity we see again and again.
\begin{align}
    \oint \frac{dz}{\left( z - z_0 \right)^n} =
    \begin{cases}
        2\pi i & \text{if } n =1\\
        0 & \text{otherwise}
    \end{cases}
\end{align}
It's easy to prove with a circle.
Just substitute $z=\rho e^{i\theta}$.
If the curve is not a circle,
first prove arbitrary curves give the same result
using Cauchy's theorem using a cut as we showed before.
Then if you know the results of the circle,
you know the result for arbitrary curves.
Anyway,
it's a homework problem and the solution will be provided at some point.

The point is,
that if you use this integral,
the $b_1$ is the only term that satisfies $n=1$.
All the other $b$s have an $n$ that is bigger than one and doesn't contribute.
The $a$'s don't contribute because they are analytic and Cauchy's theorem tells
you they don't contribute,
and of all the $b$'s only the $b_1$ contributes because of that formula.

So then, therefore
\begin{align}
    \oint_C f(z)\, dz = 2\pi b_1
\end{align}
and this $b_1$ is called the \emph{residue} of $f(z)$ at $z=z_0$.
It's the only term that survives this integration.
Any questions about all this?

So this particular result corresponds to when you have just one singular point
inside the contour $C$.
What if there are half a dozen,
what if there is more than one singularity inside that point?
Then what?

So we have to generalize this result.
We generalize to when there are multiple isolated singularities.
And we will see there is a very simple generalization of this thing.

Now let's say this is the region [picture]
and there are two isolated singular points enclosed by the contour $C$.
Let's call this point $z_1$ and this point $z_2$.
Now what you do is draw a circle around $z_1$
and another circle around $z_2$.
This contour $C$ goes counter clockwise.
Then we do what we did before.
We create a cut like this from $C$ to $C_1$ and a cut from $C$ to $C_2$.
Now the contour is deformed.
Follow the arrows and see you have a simply connected closed loop.

A couple of things I want you to realise.
This region in between $C$ and the circles,
that is simply connected.
If I put a rubber band anywhere,
I can shrink it to a point,
and I can apply Cauchy's theorem.

So around the deformed contour,
we have
\begin{align}
    \oint f(z)\, dz = 0
\end{align}
by Cauchy's theorem,
because we have cut out the singularities and it's analytic.
The contribution to the integral from the cut vanishes.
So then you can convince yourself using the logic very similar to our proof of
the Cauchy integral formula that
\begin{align}
    \oint_C f(z) \, &=
    \oint_{C_1} f(z)\,dz
    + \oint_{C_2} f(z)\, dz
\end{align}
In other words,
the contribution from the cuts vanishes,
but you can see that $C_1$ and $C_2$ are going clockwise,
but if you make everything counterclockwise,
the signs work out.
We are going around $C$, $C_1$ and $C_2$ counterclockwise
in the equation above.

And this generalizes no matter how many singularities you have.
It is the sum of the contour integral around each point.
But we just proved the integral around one singularity is just the residue times
$2\pi i$.
Each circle contains only one singular point
and we can immediately apply the result we just derived.
So it should be $2\pi i$ times the residue of each singular point.
And this is the residue theorem we're going to apply again and again.

So this is my claim.
This leads to the residue theorem
\begin{align}
    \oint_C f(z)\, dz &= 2\pi i\times
    \text{sum of residues of $f(z)$ from singular points inside $C$}
\end{align}
This is the thing that you guys have to remember.

\begin{question}
    That's the closed contour $C$?
\end{question}
Yes it's the closed contour $C$ that contains the isolated points.
Any questions?

The rest of this class and most of the rest of the next class will be giving
examples where we use the residue theorem to evaluate integrals.
This is a place where you use complex analysis a lot.
There are many integrals where the simplest way to do them is to use a contour
integral,
sometimes you know for example,
the most convenient way to express something is as a contour integral.
This result gets used a lot,
especially if you do theoretical physics,
you will see this over and over again.
Let's do some examples.

You should look at all the literature on ELMS
and try to work through the examples there.
Do several of them.
Unfortunately I can't cover the full range of possibilities.
As you'll see,
you have to be clever in most cases,
choose your contour cunningly,
hence it's helpful to do some examples so you can learn how to choose.
But time constraints.

Let's start with some very simple ones.
\begin{example}
    Integrate $f(z) = \sin(z)/z^4$ around the unit circle counterclockwise.
\end{example}
\begin{proof}[solution]
    This potentially has a pole at $z=0$.
    There's nowhere else where it might have a pole.
    The first thing to do is expand this in Taylor series.
    \begin{align}
        \frac{\sin(z)}{z^4} &=
        \frac{1}{z^4}\left\{
            z
            - \frac{z^3}{3!}
            + \frac{z^5}{5!}
            - \cdots
        \right\}
    \end{align}
    What is the order of the pole?
    It's order 3.
    What is the residue?
    It's $-1/6$.
    You can see this
    \begin{align}
        f(z) &=
        \frac{1}{z^3}
        - \frac{1}{6} \frac{1}{z}
        + \frac{z}{5!} - \cdots
    \end{align}
    Don't forget that negative sign,
    the residue is $-\frac{1}{6}$,
    not $\frac{1}{6}$.
    So by the residue theorem,
    \begin{align}
        \oint \frac{\sin z}{z^4}\,dz &=
        2\pi i \left( -\frac{1}{6} \right)\\
        &= -\frac{\pi i}{3}
    \end{align}
\end{proof}
Any questions?
OK let's do another example.

\begin{example}
    Integrate $\frac{4 - 3z}{z^2 - z}$
    around the circle $|z|=2$ counterclockwise.
\end{example}
\begin{proof}[solution]
    This is the circle of radius 2 centred at the origin.
    Look at this and tell me where it has poles.
    $z=0$ and $z=1$ where the denominator vanishes.
    The poles are where the denominator vanishes.
    And $0$ and $1$ are both inside the circle.
    And so we really have to use the residue theorem
    but first we need to find the residue at both $0$ and $1$.
    That's the procedure.

    The function has poles at $z=0$ and $z=1$.
    Near $z=0$,
    \begin{align}
        f(z) &=
        \frac{4 - 3z}{z(z - 1)}\\
        &= \frac{1}{z}\left\{
            \frac{4 - 3z}{z - 1}
        \right\}\\
        &= -\frac{4}{z} + \text{analytic terms near $z=0$}
    \end{align}
    The fraction in the bracket is analytic at $z=0$,
    and it's equal to $-4$.
    What's where it comes from.

    Near $z=1$,
    \begin{align}
        f(z) &=
        \frac{4 - 3z}{z(z - 1)}
        = \frac{1}{z - 1}\left\{
            \frac{4 - 3z}{z}
        \right\}
    \end{align}
    Near $z=1$, the things in the braces are analytic,
    so then the whole thing behaves like
    \begin{align}
        f(z) = \frac{1}{z - 1}
    \end{align}
    and hence $1$ is the residue.
    
    By the residue theorem,
    \begin{align}
        \oint f(z)\, dz &= 2\pi i\sum \text{residues}\\
        &= 2\pi i \left\{ -4 + 1 \right\}\\
        &= -6\pi i.
    \end{align}
    These are just some really simple examples to illustrate.
\end{proof}
Let's suppose this is $|z|=1/2$.
How would this change the answer?

If you draw it,
the circle would then have radius $1/2$,
and $z=1$ is outside the circle,
so then this $z=1$ residue wouldn't have contributed
and the result would be $-8\pi i$.
The boundary is not well defined for $|z|=1$.
You may have to take some limiting procedure,
and the answer may depend on how you take that limiting procedure.

Let's go on to the next problem.
\begin{example}
    Evaluate the definite integral
    \begin{align}
        I = \int_{0}^{2\pi} \frac{d\theta}{5 + 4\cos\theta}.
    \end{align}
\end{example}
\begin{proof}[Solution]
    Strictly speaking,
    you don't need complex analysis to do this integral,
    but it's a tough one.
    You'd have to use
    \begin{align}
        t = \tan\frac{\theta}{2}
    \end{align}
    but you would have had to know it.
    I'll show you it's easy with complex analysis.
    Here's what you do.
    There's all classes of problems where you integrate trig functions.
    The steps are going to be the same.

    Change variables to $z=e^{i\theta}$.
    Then $\theta$ goes from $0$ to $2\pi$
    around the unit circle in the complex plane.
    As $\theta$ goes from $0$ to $2\pi$,
    we go around the unit circle in the complex plane,
    which of course is $|z|=1$.

    So then, you have to change variables from $\theta$ to $z$.
    \begin{align}
        dz = i e^{i\theta} d\theta\qquad
        \implies\qquad
        \frac{1}{i} \frac{dz}{z} = d\theta.
    \end{align}
    And then
    \begin{align}
        \cos\theta = \frac{1}{2}\left(
            z + \frac{1}{z}
        \right).
    \end{align}
    To see this
    just use the Euler formula
    \begin{align}
        e^{i\theta} = \cos\theta + i\sin\theta.
    \end{align}
    So then the problem has reduced to evaluating
    \begin{align}
        I &=
        \oint \frac{1}{iz}
        \frac{1}{5 + 2\left( z + \frac{1}{z} \right)}\,dz\\
        &= \frac{1}{i}\oint \frac{dz}{2z^2 + 5z + 2}.
    \end{align}
    That's what you get when the dust settles.
    Now we have to figure out what the poles of this thing are.
    You have to find the roots first.
    I'm going to skip the step and claim it can be written like this.
    \begin{align}
        I &=
        \frac{1}{i}
        \oint
        \frac{dz}{(2z + 1)(z + 2)}.
    \end{align}
    If I draw this in the complex plane,
    it has a pole here at $z=-1/2$ and then it has a pole here at $z=-2$.
    And we want the contour integral of the unit circle
    centred at $z=0$.
    You can see that only one of the two poles is inside the circle.
    We're going counterclockwise because $\theta$ is increasing this way.
    $\theta$ is increasing from $0$ to $2\pi$ so we're going counterclockwise,
    which is good,
    so we don't have to put an extra sign in
    and just blindly apply the the theorems.

    Only the pole at $z=-1/2$ contributes.
    Now let's calculate the residue.
    Let's look at this
    \begin{align}
        \frac{1}{(2z + 1)(z + 2)} &=
        \frac{1}{z + \frac{1}{2}}\left[ 
            \frac{1}{2(z + 2)}
        \right]\\
        &= \frac{1}{z + \frac{1}{2}}\left[
            \frac{1}{3}
        \right]
    \end{align}
    near $z=-1/2$,
    where we have evaluated the analytic part at $z=-1/2$.
    So $1/3$ is the residue,
    so
    \begin{align}
        I &= \frac{1}{i}2\pi i \frac{1}{3}\\
        &= \frac{2\pi}{3}
    \end{align}
\end{proof}

Any questions?
\begin{question}
    When evaluating the analytic part,
    can we ever just say this region is analytic and just plug in the value of
    $z$.
    Do we have to do the Taylor series?
\end{question}
You don't have to do the Taylor series,
only the first term matters.
This thing is non-zero,
but if it is zero,
the residue is zero.

We have to be a little bit careful.
Suppose if our function had a square in it
\begin{align}
    \frac{1}{(2z + 1)^2(z + 2)}
    = \frac{1}{\left( z + \frac{1}{2} \right)^2}\left[
    \frac{(z + \frac{1}{2})}{2(z + 2)}
    \right]
\end{align}
there is a higher order pole and you have to expand.
In physics problems you typically only have simple poles
and you don't have to expand.

You always want to make it a closed curve.

\begin{question}
    What happens if the integral is not from $0$ to $2\pi$?
\end{question}
Can we try $z=e^{i2\theta}$ for example if it's from $0$ to $\pi$.

\begin{question}
    Can you explain how you'd do it if it were a complex pole?
\end{question}
Let's say instead we had
\begin{align}
    \frac{1}{(2z + 1)^2 (z + 2)}
    = \frac{1}{\left(1 + \frac{1}{2} \right)^2}
    \frac{1}{4\left(z + 2 \right)}
\end{align}
Then you have to expand in a Taylor series,
so the coefficient of the next term in the Taylor series is what gives you the
residue.
In this situation with a higher order pole,
you need more terms in the Taylor series to work it out.

\begin{example}
    Evaluate
    \begin{align}
        I = \int_{-\infty}^{\infty}
        \frac{dx}{1 + x^2}
    \end{align}
\end{example}
\begin{proof}[Solution]
    If you were to use elementary methods,
    you would get
    \begin{align}
        \tan^{-1}(x)|_{-\infty}^{\inftyy}
        = \frac{\pi}{2}
        - \left( -\frac{\pi}{2} \right)
        = \pi.
    \end{align}
    Let's do it using complex analysis,
    just to show you the kind of contours you can integrate with.
    Consider the complex line integral
    \begin{align}
        \lim_{\rho\to\infty}
        \int_{-\rho}^{\rho}
        \frac{dz}{1 + z^2}
    \end{align}
    where the path is along the real line.
    This line integral is equal to $I$.
    Now let's draw this thing in the complex plane.
    Let's look at this integrand.
    It has poles where this $1+z^2$ vanishes.
    The denominator vanishes at $z=+i$ and $z=-i$.
    Right now the fact it has two poles is irrelevant for now.
    Consider this thing integrated not just along the real line,
    but back along a big semicircle in the upper half plane.

    Consider
    \begin{align}
        \oint \frac{dz}{1 + z^2}
    \end{align}
    integrated over the closed contour shown in the figure,
    which consists of a semicircle over the upper half plane,
    in addition to the integral over real line.

    This closed line integral we can evaluate using the residue theorem.
    Even though it has two poles,
    only $z=+i$ is inside the contour.
    Let's evalaute it.
    Close to $z=i$,
    \begin{align}
        \frac{1}{1 + z^2} = \frac{1}{(z + i)(z - i)}
        = \frac{1}{z - 1} \underbrace{\frac{1}{2i}}_{\text{residue}}
        + \text{analytic function}
    \end{align}
    Hence by the residue theorem, the contour integral is
    \begin{align}
        \oint \frac{dz}{1 + z^2} &=
        2\pi i \frac{1}{2i} = \pi.
    \end{align}
    We have evaluated this integral not over the real line,
    but over this closed contour which includes the real line,
    but also includes the contribution from this big semicircle.
    I'm going to show you the contribution from the big semicircle vanishes as
    you take $\rho\to\infty$.
    And so this closed integral is actually equal to the integral we want.
    That's going to be the logic here.

    We need to prove the line integral over the semicircle vanishes and then
    we're done.
    Let's prove that.
    \begin{align}
        \underbrace{\oint \frac{dz}{1 + z^2}}_{\pi} &=
        \underbrace{\int_{\text{real line}} \frac{dz}{1 + z^2}}_{I}
        + \underbrace{\int_{\text{semicircle}} \frac{dz}{1 + z^2}}_{?}
    \end{align}
    Set $z=\rho e^{i\theta}$.
    Then consider
    \begin{align}
        \int_{\text{semicircle}} \frac{dz}{1 + z^2} &=
        \int_{0}^{\pi} \frac{i\rho e^{i\theta}}{1 + \rho^2 e^{2i\theta}}\,
        d\theta
    \end{align}
    This has $\rho$ in the numerator and $\rho^2$ in the denominator,
    this should go to zero as $\rho\to\infty$.
    But you still have to be careful because it's complex,
    but you'll see this intuition is correct.
    It's just going to vanish.
    \begin{align}
        \left|
            \int_{\text{semicircle}} \frac{dz}{1 + z^2}
        \right|
        &=
        \left|
            \int_{0}^{\pi} \frac{i\rho e^{i\theta}}{1 + \rho^2 e^{2i\theta}}\,
            d\theta\\
        \right|
        &\le
        \int_{0}^{\pi}
        \left|
            \frac{i\rho e^{i\theta}}{1 + \rho^2 e^{2i\theta}}
        \right|
        \left|
            d\theta
        \right|\\
        &\le
        \int_{0}^{\pi}
        \frac{\rho}{\rho^2} |d\theta|\\
        &=
        \frac{\pi}{\rho}\\
        &\to 0
    \end{align}
    as $\rho\to\infty$.
    So in the end the contribution of this semicircle vanishes.
\end{proof}

This shows you can do an integral over the real line by doing an integral like
this with a contour like this.
This is one of the very standard contours where you have an integral over the
real line and make it a contour by closing the upper half plane,
or lower half plane.

\begin{question}
    What if you took the semicircle in the lower half plane?
\end{question}
Everything would work, but you're not going counterclockwise and you have to
deal with signs.
It still works though.
There's more than one way to do these problem.
Choose the most convenient one.
As you do more provlems, you'll see.
I want you guys to convince yourselves that it can be closed in the negative
half plane and get the same result.

This contour works for this function works for a lot of functions,
but it also doesn't work for a lot of functions.
You could have used either up or down in this problem,
but we're going to look at at problem where you have to use one of them.
It's actually more common.

\begin{example}
    Evalaute
    \begin{align}
        I = \int_{0}^{\infty} \frac{\cos(z)}{1 + x^2}\,dx
    \end{align}
\end{example}
\begin{proof}[Solution]
    Notice that it's similar to before.
    Notice that the integrand is even, so this is equal to
    \begin{align}
        I &=
        \frac{1}{2}\int_{-\infty}^{\infty}
        \frac{\cos(x)}{1 + x^2}\,dx.
    \end{align}
    It's an even function so we can be a bit clever.
    We want to make this an integral over some line in the complex plane.
    There's more than one way to do this,
    but here's the trick.
    Write $z = \Re e^{iz}$.
    Then write
    \begin{align}
        I &= \re\left\{
        \frac{1}{2}
        \int_{-\infty}^{\infty}
        \frac{e^{iz}}{1 + z^2}\,
        dz
        \right\}
    \end{align}
    where the integral is over the real line in the complex plane.
    Now it's beginning to look a lot like the problem we just saw.
    You have a function integrating over the whole real line.
    It has singularities at exactly the same points $z=\pm i$.
    But can we use the same contour?

    The integrand is a bit different.
    Can we convince ourseves that the contribution from this semicircle
    vanishes?
    If it does, then we can use the same contour.
    It turns out it does vanish.

    Along that big semicircle, $z$ has a real and imaginary part
    \begin{align}
        z = \rho_R + i\rho_I
    \end{align}
    Unless you're at $+\rho$ or $-\rho$,
    the imaginary part is infinite.
    If the imaginary part is positive and infinite,
    $e^{iz}$ because $e$ raised to minus infinity,
    so it goes to zero exponentially quickly
    if you're in the upper half plane.
    That means this integrand vanishes incredibly quickly in the upper half
    plane.
    So we can use exactly the same contour as we used before
    and use the residue theorem.
    At the same time,
    we can't use a contour with a semicricle in the lower half plane,
    because it blows up incredibly quickly.
    Last time we had a choice of contours up or down,
    but here we don't have a choice,
    at least we don't have that choice,
    you have to choose to go in the upper half.
\end{proof}
