\section{Green functions}
Up to now we have left the boundary conditions that $G$ has to satisfy
unspecified.

Which boundary conditions we choose depends on convenience.
\begin{align}
    \phi(\vec{x})
    &=
    \frac{1}{4\pi\epsilon_0}
    \int_V \rho\left( \vec{x}' \right) G\left( \vec{x}', \vec{x} \right)\,
    d^3\vec{x}'\\\nonumber
    &\qquad
    + \frac{1}{4\pi}
    \oint_S\left[ 
    G\left( \vec{x}', \vec{x} \right)
    \left( 
    \vec{\nabla}' \phi\left( \vec{x}' \right)
    \cdot \hat{n}'
    \right)
    -
    \phi\left( \vec{x}' \right)
    \left( 
    \vec{\nabla}' G\left( \vec{x}', \vec{x} \right)
    \right)
    \right]\, dS'
\end{align}

The simplest consistent choice is
\begin{align}
    \vec{\nabla}' G_N \left( \vec{x}', \vec{x} \right)
    &=
    -\frac{4\pi}{S}
\end{align}
where $S$ is the teal area of the boundary surface.
\begin{align}
    \phi\left( \vec{x} \right)
    &=
    \langle \phi\rangle_S
    +
    \frac{1}{4\pi\epsilon_0}
    \int_V \rho\left( \vec{x}' \right)
    G_N\left( \vec{x}', \vec{x} \right)
    \, d^3 \vec{x}'
    +
    \frac{1}{4\pi}
    \oint_S G_N\left( \vec{x}', \vec{x} \right)
    \left( 
    \vec{\nabla}' \phi\left( \vec{x}' \right)
    \cdot
    \hat{n}'
    \right)\, dS'
\end{align}
where $\langle \phi\rangle_S$
is the average of $\phi$ over the bounding surface.

Finding the Green function is the hard part.
Everything else is easy.

\subsection{Example: Two hemispheres}
Let's do an example.

Consider a sphere made up of two conducting hemispheres that are kept at
different potentials.
And the centre of the sphere is the origin.
The upper sphere is at potential $+V$ and the lower potential is at potential
$-V$.

The problem is to find the potential due to the sphere everywhere in space.
A very standard problem.
It doesn't look very much like a Greens function problem.
Why are we able to solve this?
Because using them method of images,
we found the Green's function for a grounded conducting sphere.

Recall that using them method of images,
we found the potential due to a point charge near a conducting sphere,
and from just that we can obtain the Greens function.
\begin{align}
    G_0\left( \vec{x}', \vec{x} \right)
    &=
    \frac{1}{\left| \vec{x} - \vec{x}'\right|}
    -
    \frac{a}{r'\left|
    \vec{x}
    -
    \frac{a^2}{r'^2}\vec{x}'
    \right|}
\end{align}
Here is the magic.
We didn't solve this for an arbitrary potential,
we solved it for a conducting sphere grounded.
But that is the boundary condition for the Dirichlet Green function,
which is chosen so that $G=0$ on the surface.
So it happens that when we solve this problem for the grounded conducting
sphere,
we were actually finding the Dirichlet Greens function.
Zero potential on the bounded surface,
so it's grounded.
That is exactly the Greens function.
What we were solving the problem,
what we were really solving is the Dirichlet Greens function.
So we know the answer.
That's the Dirichlet greens function.
Now you might say what does that have to do with  completely different
problem where the potential nd boundary have nothing to do with the grounded
sphere we originally solved.
That's the magic of the Greens function method.

Once you know the Greens function of the Dirichlet function,
then for any Dirichlet boundary function of the potential,
the answer is given in terms of these integral.s

Recall that
\begin{align}
    \phi\left( \vec{x} \right)
    &=
    \frac{1}{4\pi\epsilon_0}
    \int_V 
    \rho\left( \vec{x}' \right)
    G_D\left( \vec{x},\vec{x}' \right)
    \, d^3\vec{x}'
    -
    \frac{1}{4\pi}
    \oint_S
    \phi\left( \vec{x}' \right)
    \left\{ 
    \vec{\nabla}' G_D \left( \vec{x}', \vec{x} \right)
    \cdot \hat{n}'
    \right\}
    \,dS'
\end{align}
Now here's the thing.
What is $\rho\left( \vec{x}' \right)$?
It's zero, because there are no charges.
We know the Greens function,
so we can calculate this gradient.
And $\phi(\vec{x}')$ is specified on the surfaces.

And then we can do the problem we want to solve by evaluating this integral.
You see the power of this method.
The problem you originally solved looks noting like this problem.
The only thing the same is the surface.
This is a sphere
that's a sphere.
Everything else is changed.
Charges are changed,
boundary conditions have changed.

Once you know the Greens function for that surface,
then it just doesn't matter.
It's the surface that matters.
Once you have solved the Greens function for that surface with Dirichlet
boundary conditions,
you're done.
Just plug it into this formula.
That's the magic.
Then of course you may say this is a messy integral,
but that's completely different.
That's not the hard work.
The real hard work was finding the Greens function.

It was easy with the method of images here,
but for more general geometries it's much harder.

Any problem you can solve by the method of images you can use this method.

In spherical coordinates
\begin{align}
    \left|
    \vec{x} - \vec{x}'
    \right|
    &=
    \sqrt{%
    r^2 + {r'}^2
    - 2rr' \cos\gamma
    }
\end{align}
where $\gamma$ is the angle between $\vec{x}$ and $\vec{x}'$.

Then
\begin{align}
    r'
    \left|
    \vec{x}
    -
    \frac{a^2}{r'^2}\vec{x}'
    \right|
    &=
    \sqrt{
    r^2 r'^2 - 2a^2 rr' \cos\gamma
    + a^4
    }
\end{align}
and
\begin{align}
    \cos \gamma &= \hat{e}_r \hat{e}_{r'}
\end{align}
where
\begin{align}
    \hat{e}_r
    &=
    \sin\theta \cos\phi \hat{e}_x
    + \sin\theta \sin\phi \hat{e}_\phi
    + \cos\theta \hat{e}_z\\
    \hat{e}_{r'}
    &=
    \sin\theta' \cos\phi' \hat{e}_x
    + \sin\theta' \sin\phi' \hat{e}_\phi
    + \cos\theta' \hat{e}_z\\
\end{align}
And in spherical coordinates
\begin{align}
    \cos\gamma
    &=
    \cos\theta \cos\theta'
    + \sin\theta \sin\theta' \cos\left( \phi - \phi' \right)\\
    G_D\left( \vec{x}, \vec{x}' \right)
    &=
    \frac{1}{\sqrt{
    r^2 + r'^2 + 2r r' \cos\gamma
    }}
    -
    \frac{1}{\sqrt{
    \frac{r^2 r'^2}{a^2} 
    + a^2
    - 2 r r' \cos\gamma
    }}
\end{align}
We need to calculate $\vec{\nabla}'G_D$.
You can check that the Green function is symmetric
\begin{align}
    G\left( \vec{x}, \vec{x}' \right)
    &=
    G\left( \vec{x}', \vec{x}' \right)
\end{align}
Coming back to this,
what is $\hat{n}'$.
What should I choose for it in the coordinate systems I've chosen.
You should choose
\begin{align}
    \hat{n}' &= -\hat{e}_{r'}
\end{align}
But why the minus sign?
Remember,
we're living in the volume outside the sphere.
$\hat{n}'$ is the unit vector going \emph{out} of our volume
where we are trying to solve for the potential,
so it should be going into the sphere.
That's a common thing to mess up.

So then this becomes
\begin{align}
    \vec{\nabla}' G_D \left( \vec{x}', \vec{x} \right)
    \cdot \hat{n}'
    &=
    - \frac{\partial G_D}{\partial r'}
\end{align}
and it's just a lot of algebra,
but all the steps are given in my notes.
When all the dust settles,
\begin{align}
    - \frac{\partial G_D}{\partial r'}
    &=
    \frac{-\left( r^2 - a^2 \right)}{
    \left( r^2 + a^2 - 2ar\cos\gamma \right)^{3/2}
    }
\end{align}

So after doing some steps,
\begin{align}
    \phi\left( \vec{x} \right)
    &=
    \frac{1}{4\pi}
    \int_S
    \phi\left( a, \theta', \phi' \right)
    \frac{-\left( r^2 - a^2 \right)}{
    \left( r^2 + a^2 - 2ar\cos\gamma \right)^{3/2}
    }
    d\Omega'
\end{align}
where $d\Omega' = \sin\hteta' \, d\theta'\, d\phi$
is the solid angle,
with $dS' = a^2 d\Omega'$.

This thing reduces to
\begin{align}
    \phi\left( \vec{x} \right)
    &=
    \frac{V a^2\left( r^2 - a^2 \right)}{4\pi a}
    \int_{0}^{2\pi}
    d\phi'
    \left[
    \int_{0}^{1}
    d\left( \cos\theta' \right)
    -
    \int_{-1}^{0}
    d\left( \cos\theta' \right)
    \right]
    \frac{1}{\left( a^2 + r^2 - 2ar\cos\gamma \right)^{1/2}}
\end{align}
At this point the physics is kind of done.
In other words,
knowing this Greens function,
you can find the potential for this sphere,
reduced to some integral,
which we may not do in closed form,
but the answer is here.

\begin{question}
    What if there are multiple surfaces?
\end{question}
Then you have many integrals.

But this is about as an easy of a problem as you can get with Green's function.
That was a very easy problem.

\section{Method of Orthogonal Basis Functions}
Suppose you know the Greens function for some boundary,
for arbitrary boundary conditions you can find the potential.
But how do you find the Green's function?
In general it's hard.

The method of images works some times.

I will now discuss another method.

The idea here is that in the end,
solving a partial differential equation,
like Poisson's equation or Laplace's equation,
the physicist favorite method is separation of variables.
There are certain geometries where the Laplacian geometry is separable.
There are a dozen or so where Laplace's equation is separable.
We're not going to do all the dozen.
If you took the math methods class 40 years ago,
you would learn a whole dozen.
Nowadays you only have to study 3,
which are Cartesian, spherical and cylindrical.
That's pretty much the rest of the course.

So we're just going to do a couple of examples with Cartesian,
spherical, cylindrical.
For some reason,
the number of cylindrical examples in the books is fewer,
even though it's harder.

Let's start with the easy case of Cartesian.

Here's the problem.
Not sure if you've solved this as an undergraduate.
The problem is that you have a rectangular box
with sides $\Delta x = a$,
$\Delta y = b$ and $\Delta z = c$.
This box is sitting with one of its corners in the origin.
This height is $c$,
this distance is $a$ and this distance if $b$.
It's a rectangular box.
So you understand this geometry or is it mysterious?
Anyway that's what it is.

All the sides of the box are at zero potential except the top
$z=c$,
which is at potential $V(x, y)$,
which is some specific function that is specified.
You're told that at the top of the box,
the potential satisfies this.

The problem is to solve for $\phi$ everywhere inside the box.
This is not particularly hard.
You just have to solve Laplace's equation inside the box.
Since the sides of the box parallel to $xyz$,
it's most convenient to use Cartesian coordinates.
\begin{align}
    \frac{\partial^2 \phi}{\partial x^2}
    +
    \frac{\partial^2 \phi}{\partial y^2}
    +
    \frac{\partial^2 \phi}{\partial z^2}
    &=
    0
\end{align}
We're going to solve this using the method of separation of variables.

Here's the idea.
We begin by looking for a solution
of very specific form.
\begin{align}
    \hat{\phi}\left( x, y, z \right)
    &=
    X(x) Y(y) Z(z)
\end{align}
In the end the final solution will not be of this form,
but it will be a superposition of solutions of this form.
So you find solutions like this and then you add them up to satisfy the boundary
conditions.
Substituting into Laplace's equation,
\begin{align}
    \frac{1}{X}\frac{d^2 X}{d x^2}
    +
    \frac{1}{Y}\frac{d^2 Y}{d y^2}
    +
    \frac{1}{Z}\frac{d^2 Z}{d z^2}
    &=
    0
\end{align}
Before they were partial derivatives,
but here they are ordinary derivatives.
Rewrite this as
\begin{align}
    \frac{1}{X} \frac{d^2 X}{dx^2}
    + \frac{1}{Y} \frac{d^2 Y}{dx^2}
    &=
    -\frac{1}{Z}
    \frac{d^2 Z}{dz^2}
\end{align}
Now comes the crucial step.
The right side of the equation only depends on $z$.
The left side depends on $x,y$ but not on $z$.
Suppose I change $z$,
the right hand side should change,
but the left hand side can't change,
and the only way that can be true is if the right side doesn't actually depend
on $z$ at all.
Hence
\begin{align}
    -\frac{1}{Z}
    \frac{d^2 Z}{dz^2}
    =
    \text{constant}
\end{align}
that is independent of $z$
Let's give this constant a name.
\begin{align}
    -\frac{1}{Z}
    \frac{d^2 Z}{dz^2}
    =
    -\gamma^2.
\end{align}
Why the minus sign?
Because I know the answer.
Sometimes you can get the solution and find the sign is not nice and you have to
go back to this step to change the sign.
I know the answer and this choice is good.

Similarly,
you can say the same thing about the other side,
with $x$ and $y$.

Then we have two equations.
We started wiht one equation,
but now we have 2 equations.
\begin{align}
    \frac{1}{X}\frac{d^2 X}{dx^2}
    +
    \frac{1}{Y}\frac{d^2 Y}{dy^2}
    &=
    -\gamma^2
\end{align}
and we have
\begin{align}
    -\frac{1}{Z} \frac{d^2 Z}{dz^2} &= -\gamma^2
\end{align}
So now we can rewrite the first equation as
\begin{align}
    \frac{1}{X} \frac{d^2 X}{dx^2}
    &=
    -\frac{1}{Y} \frac{d^2 Y}{dy^2}
    - \gamma^2
\end{align}
and now we apply the same logic as before.
If I change $x$,
I should expect the left hand side ot chnage,
but the right side dosn't change,,
and the only way for that to be possible is that the left doesn't actually
depend on $x$ and so it's some constant independent of $x$ and $y$,
and let's call it $-\alpha^2$.
Once again there's a sign because I know the answer.
\begin{align}
    \frac{1}{X} \frac{d^2 X}{dx^2}
    &=
    -\alpha^2.
\end{align}
We can then define a parameter
\begin{align}
    \beta^2 &:= \gamma^2 - \alpha^2
\end{align}
and now we have 3 equations.
\begin{align}
    \frac{1}{X} \frac{d^2 X}{dx^2}
    &=
    -\alpha^2\\
    \frac{1}{Y}
    \frac{d^2 Y}{dy^2} &=
    -\beta^2\\
    \frac{1}{Z}\frac{d^2 Z}{dz^2} &=
    + \gamma^2.
\end{align}
So we have 3 ODEs and we know the boundary conditions
\begin{align}
    X(x) &= 0 &\text{at }x = 0 \text{ and } x = a\\
    Y(y) &= 0 &\text{at }y = 0 \text{ and } y = b
\end{align}
Then
\begin{align}
    X(x) &=
    \sin\left( \frac{n\pi x}{a} \right)\\
    Y(y) &=
    \sin\left( \frac{m\pi y}{b} \right)
\end{align}
where $n,m$ are integers
and we know
\begin{align}
    Z(z) &= 0 &\text{at } z = 0
\end{align}
so then
\begin{align}
    Z(z) &= \sinh\left( \gamma_{nm} z \right)
\end{align}
where
\begin{align}
    \gamma_{nm} &=
    \pi \sqrt{
    \left( \frac{n}{a} \right)^2
    +
    \left( \frac{m}{b} \right)^2
    }
\end{align}
There's still one boundary condition at the top we haven't used yet.
The thing about the hyperbolic sine is that it's monotonic.
If $\phi$ is zero at the top,
then I just get $\phi(z)=0$ everywhere,
so it's important I have some interesting boundary conditions.

To satisfy boundary conditions at $z=c$ for a linear superposition,
\begin{align}
    \ph(x, y, z)
    &=
    \sum_{n,m=1}^\infy
    A_{mn}
    \sin\left( \frac{n \pi x}{a} \right)
    \sin\left( \frac{m \pi y}{b} \right)
    \sinh\left( \gamma_{mn} z \right)
\end{align}
and we require that
\begin{align}
    \phi\left( x, y, c \right) &= V\left( x, y \right)
\end{align}
This is really just double Fourier seris for $V(x, y)$.
What we need to do is invert this and find the Foufier coefficients 
$A_{nm}$.
Use the identity
\begin{align}
    \frac{2}{L}\int_{0}^{L} dx\,
    \sin\left( \frac{n\pi x}{L} \right)
    \sin\left( \frac{n'\pi x}{L} \right)
    &=
    \delta_{n,n'}
\end{align}
iSo then
\begin{align}
    \int_{0}^{a}dx\,
    \left( \frac{z}{a} \right)
    \sin\left( \frac{n'\pi x}{a} \right)
    \int_{0}^{b}dy\,
    \left( \frac{z}{b} \right)
    \sin\left( \frac{m' \pi y}{b} \right)
    V\left( x, y \right)
    &=& 
    \sum_{n,m}
    A_{nm} \delta_{nn'} \delta_{mm'}
    \sin\left( \Gamma_{nm} z \right)
\end{align}
which gives
\begin{align}
    A_{nm}
    &=
    \frac{4}{ab}
    \frac{1}{\sinh\left( \gamma_{nm} z \right)}
    \int_{0}^{a} dx
    \int_{0}^{L} dy\,
    \sin\left( \frac{n\pi x}{a} \right)
    \sin\left( \frac{m\pi y}{b} \right)
    V\left( x, y \right)
\end{align}

Meake sure you can solve the problems in hte homework chop chop.
The problems aresimliar but not the same.
You need to do those right quicklky.
Some questions are straightforward but longer,
but you can do those later once you've finished the fast ones.
You guys should do well in this exam.
EM no matter how easy is going to be a hard exame.
