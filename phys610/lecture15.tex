\section{Hamiltonian Mechanics}
Recall hte Lagrangian is
\begin{align}
    L = T - U
\end{align}
which is a function $L(q_1,\ldots,q_n, \dot{q}_1,\ldots,\dot{q}_n)$.
Let use define
\begin{align}
    p_i &= \frac{\partial L}{\partial q_i}
\end{align}
where $p_i$ is called the \emph{canonical momentum} or momentum conjugate to
$q_i$.
The \emph{Legendre transform} is
\begin{align}
    H &= \sum_i p_i \dot{q}_i - L
\end{align}
where $H=H(q_i, p_i)$.

The $n$ q_i$ and $n$ p_i$ define a point in a $2n$-dimensional space,
called \emph{phase space}.
Hamilton's equations determine a unique path in phase space,
starting from any initial point.

Let's derive Hamilton's euqations for a 1D system.
\begin{align}
    H &=
    p\dot{q}(q, p) - L\left( q, \dot{q}(q, p) \right)
\end{align}
The point is that $\dot{q}$ is determined implicitly in terms of $q$ and $p$
by the definition of $p$.
\begin{align}
    \left(\frac{\partial L}{\partial \dot{q}}\right)_{q} = p.
\end{align}
where we are holding $q$ fixed in the derivative.
Let's calculate the derivative
\begin{align}
    \left( \frac{\partial H}{\partial q} \right)_p
    &=
    p \left( \frac{\partial\dot{q}}{\partial q} \right)_p
    - \left[ 
    \left( \frac{\partial L}{\partial q} \right)_p
    \underbrace{\left(\frac{\partial q}{\partial q}_\right)_p}_{1}
    +
    \underbrace{\left( \frac{\partial L}{\partial q} \right)_q}_{p}
    \left( \frac{\partial \dot{q}}{\partial q} \right)_p
    \right]
\end{align}
It's just an exercise of partial derivatives.
You see that one of the terms cancel.
So using Lagrange's equation,
\begin{align}
    \left( \frac{\partial H}{\partial q} \right)_p
    &=
    - \left( \frac{\partial L}{\partial q} \right)_{\dot{q}}\\
    &=
    -\frac{d}{dt}\left( \frac{\partial L}{\partial \dot{q}} \right)_{q}\\
    &= -\frac{dp}{dt}
\end{align}
where it's true on a classical path

Similarly,
\begin{align}
    \left( \frac{\partial H}{\partial p} \right)_{q}
    &=
    \dot{q}
    +
    p \left( \frac{\partial \dot{q}}{\partial p} \right)_{q}
    -
    \left[ 
    \left( \frac{\partial L}{\partial q} \right)_{\dot{q}}
    \underbrace{\left( \frac{\partial q}{\partial p} \right)_{q}}_{0}
    +
    \underbrace{\left( \frac{\partial L}{\partial \dot{q}}
    \right)_{q}}_{p}
    \left( \frac{\partial \dot{q}}{\partial p} \right)_{q}
    \right]\\
    &= \dot{q}
\end{align}
And so we have derived Hamilton's equations
\begin{align}
    \left( \frac{\partial H}{\partial q} \right)_{p} &= -\dot{p}\\
    \left( \frac{\partial H}{\partial p} \right)_{q} &= -\dot{q}
\end{align}
These are first order equations.
This tells you how a point $(q,p)$ changes with time in phase space.

In the Lagrangian approach,
fora system with one degrees of freedom,
we obtain a single second order equation.
In the Hamiltonian approach,
we obtain two first order equations.

\subsection{Atwood's Machine}
Once again,
we consider the Atwood's machine as an example.

We have a string on a pulley with two masses on its end
$m_1$ and $m_2$.
The length of the string on the $m_1$ is  $x$
and the length of the string on the $m_2$ side is $y$.
The constraint is that
\begin{align}
    x + y = \textrm{constant}
\end{align}
which means the string is inextensible,
so if one goes up the other goes down.

The Lagrangian is
\begin{align}
    L &=
    \frac{1}{2}m_1\dot{x}^2
    +
    \frac{1}{2}m_2 \dot{y}^2
    - \left( 
    -m_1 gx
    - m_2 gy
    \right)\\
    &=
    \frac{1}{2}m_1\dot{x}^2 + \frac{1}{2}m_2\dot{y}^2
    + m_1 gx + m_2gy\\
    &=
    \frac{1}{2}\left( m_1 + m_2 \right) \dot{x}^2
    +
    \left( m_1 - m_2 \right) gx
\end{align}
Then we can calculte the momenta
\begin{align}
    p &= \frac{\partial L}{\partial x}\\
    &=
    \left( m_1 + m_2 \right)\dot{x}
\end{align}
and write the velocity in terms of momentum
\begin{align}
    \dot{x} &= \frac{p}{m_1 + m_2}
\end{align}
which we can substitute into the Lagrangian
\begin{align}
    L &=
    \frac{1}{2} \frac{p^2}{m_1 + m_2}
    +
    \left( m_1 - m_2 \right)gx
\end{align}
so the Hamiltonian is
\begin{align}
    H &= p\dot{x} - L\\
    &=
    p \frac{p}{m_1 + m_2} - L\\
    &=
    \frac{1}{2}\frac{p^2}{m_1 + m_2}
    -
    \left( m_1 - m_2 \right)gx
\end{align}
Using Hamilton's equations
\begin{align}
    -\dot{p} &= \frac{\partial H}{\partial q}\\
    \dot{x} &= \frac{\partial H}{\partial p}
\end{align}
you can convince yourself it gives
\begin{align}
    \dot{p} &= \left( m_1 - m_2 \right)g\\
    \dot{x} &= \frac{p}{m_1 + m_2}
\end{align}
Differentiate the second equation with respect to time and we gte
\begin{align}
    \ddot{x} &=
    \frac{\left( m_1 - m_2 \right)g}{m_1 + m_2}
\end{align}

You could of course get the same differential equation from the Euler-Lagrange
equations in the Lagrangian approach.
\begin{align}
    \frac{d}{dt}\left( \frac{\partial L}{\partial \dot{x}} \right)
    - \frac{\partial L}{\partial x} &= 0\\
    \frac{d}{dt}\left[ 
    \left( m_1 + m_2 \right)x
    \right]
    -
    \left( m_1 - m_2 \right)g
    &= 0\\
    \ddot{x} &=
    \frac{\left( m_1 - m_2 \right)g}{m_1 + m_2}
\end{align}
Why bother with Hamilton's equations with a fraction of the work.

After Hamilton came up with these equations they made him a professor at Dublin
when he was just an undergrad!
So what is the great insight of Hamilton?

The minus sign changes the symmetry from the obvious one into a more subtle
symmetry.
If you are a theoretical physicist,
you see they have a symplectic geometry.
The Hamiltonian formalism leads to a symmetry between the $p$ and $q$'s.
You know how much easier it is to solve a system if it has the right coordinate
system?

Now the fact is,
there is a symmetry between the coordinates and momentum,
which means you have the option to choose coordaintes which are combinations of
coordaintes and momenta.
You can make transformations between coordaintes and moemnta together,
which increases the choics of possible coordinates,
with so much more freedom to choose coordinates.

There are problems you can solve using thismethod no one knows how to solve
otherwise.
It's because of this greater flexibility.
Unfortunatley,
there's no problem you will lsolve in this course where this is actually true.
That's why it's important even in classical mechanics.
These transformations are called \emph{canonical transformations}.

\section{Phase space orbits}
The generalization of Hamilton's equations to many degrees of freedom is
\begin{align}
    \dot{q}_i &= \frac{\partial H}{\partial p_i}\\
    -\dot{p}_i &= \frac{\partial H}{\partial q_i}
\end{align}
You will need to remember these equatinos,
but it's pretty easy to remember,
just need to know where the sign is.
If you can't remember,
just consider the free particle with Hamiltonian
$H=p^2/2m$ and so $\partial H/\partial q = p/m$ and $\dot{q}=p/m$.

\begin{align}
    \dot{q}_i &= \frac{\partial H}{\partial p_i} = f_i(q_j, p_j)\\
    -\dot{p}_i &= \frac{\partial H}{\partial q_i} = g_i(q_j, p_j)
\end{align}
Introduce the $2n$ dimensional vectors
\begin{align}
    \vec{z} &=
    \begin{pmatrix}
        q_1\\
        \vdots\\
        q_n\\
        p_1\\
        \vdots\\
        p_n
    \end{pmatrix}
\end{align}
and
\begin{align}
    \vec{h} &=
    \begin{pmatrix}
        f_1(q_j, p_j)\\
        \vdots\\
        f_n(q_j, p_j)\\
        g_1(q_j, p_j)\\
        \vdots\\
        g_n(q_j, p_j)
    \end{pmatrix}
\end{align}
so we can write all of Hamilton's equations as
\begin{align}
    \dot{\vec{z}} &= \vec{h}(\vec{z}).
\end{align}
The vector $\vec{z}$ defines the ``position'' of the system in phase space.

There's a $2n$-dimensional phase space.
There's one dimension for every $p$ and one dimension for every $q$.
If you know what $z$ is,
you know the particle position in phase space at any given time,
and this equation tells you how the particle moves in phase space as a function
of time.

Let's say you have some trajectory on phase space.
Because the vector $\dot{\vec{z}}$ is unique for a given $\vec{z}$,
there should be no crossing of paths in phase space.

\subsection{Harmonic Oscillator in 1 dimension}
The Lagrangian is
\begin{align}
    L &= \frac{1}{2}m\dot{x}^2 - \frac{1}{2}kx^2
\end{align}
so the Hamiltonian is
\begin{align}
    H &= p\dot{x} - L\\
    &= \frac{p^2}{m} - \left[ 
    \frac{1}{2}m\left( \frac{p}{m} \right)^2
    - \frac{1}{2}kx^2
    \right]^2\\
    &=
    \frac{1}{2}\frac{p^2}{m}
    + \frac{1}{2}kx^2
\end{align}
so Hamilton's equations gives
\begin{align}
    \frac{\partial H}{\partial p} &= \dot{x}
    &\implies
    \frac{p}{m} &= \dot{x}\\
    \frac{\partial H}{\partial q} &= -\dot{p}
    &\implies
    kx &= -\dot{p}
\end{align}
Eliminating $p$,
we get
\begin{align}
    m\ddot{x} &= -kx
\end{align}
and if we define $\omega^2 = k/m$,
we get the solution
\begin{align}
    x &= A \cos(\omega t - \eta)\\
    p &= m\dot{x} = - mA\omega \sin(\omega t - \eta)
\end{align}
The phase space for the harmonic oscillator is the 2D space with coordinates
$(x,p)$.
As time changes,
the particle traces out a path in phase space,
called the \emph{orbit} in phase space.
Despite the name, it's generally not a closed path.

For a Harmonic oscillator in 1 dimensions,
the orbit is an ellipse
\begin{align}
    \frac{x^2}{A^2} + \frac{p^2}{mA^2\omega^2} &= 1
\end{align}
and it's going clockwise.

\section{About the Midterm}
I've written the exam already.
If you did the homework,
understand how to solve those problems,
it should be quite straightforward to get 50\% very easily.
I think it's an exam where to get 100\% is not easy,
to get 90\% is tough.
Some questions are straightforward,
but some are had.
All carry equal points.
It's not that easy questions are first.
You can read the exam,
recognise the questions.
Make sure you do the ones you know how to do.
In the end,
the most important thing is to pass.
Nail all those you know.
Once you're confident they're right,
spend time on the tough ones.

You need to memorize the formula for the Method of Steepest descent.
There is no cheat sheet.

It's not like the qualifier,
where you need to know
all of classical, all of quantum, and you have to pass them all at once,
or some BS like that.
Trust me, the midterm is much easier than being tested on all of physics.

The questions are like this.
2 from the first homework, 2 from the second homework and 2 from the last.
All questions are are of equal weight.
On Thursday, try to get here early!
Class is 9:30 to 11:20 but try to get here early.
I will hand out the question papers early enough so you begin exactly at 09:30.
You can look at the question paper before that,
but you can only start writing at 09:30.

I will try to provide paper,
but just in case bring your own.

I suggest spending time doing the homework problems,
make sure you can absolutely do all the homework problems.
Once you've done that,
look at all the worked examples I did in class.
That's basic preparation.
If you have time beyond that,
there's all those books,
some of them have problems,
look through the problems and see if you know how to do those problems.

Do we need a calculator?
I don't think you should need a calculator.

Are we scanning or handing in physical paper at the end?
I'm thinking of handing in physical paper,
because it will take you 10 minutes of scanning.

A calculator might be useful for binomial coefficients,
but you can calculate the first 2 or 3.
You'll get some points for sure.

Will we be given the contours to integrate over?
Choosing contour shapes, should we learn?
Look very carefully through all the homework problems.
I'm not going to tell you which contour to choose,
you're going to have to figure it out yourself.
That's important.

Will the mapping for mapping problems be provided?
The choice of mapping will be provided.

Also, one last thing, there's a new set of notes online.
