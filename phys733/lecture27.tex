\section{Quantum Dimer Model}
Each site is involved in no more than one dimer.

Complete dimer covering each site involved in exactly 1 dimer.
Different dimer configurations are assumed orthogonal states.
This is not the case in real systems,
but is assumed in the dimer model.

Then we considered a square lattice,
with a Hamiltonian that has a hopping term where the configurations get flipped
plus a potential energy term.
\begin{align}
    H_{qdm} &=
    \sum -t \left( \ket{||}\bra{=} + \mathrm{h.c.} \right)
    +
    V\left( 
    \ket{||} \bra{||} + \ket{=}\bra{=}
    \right)
\end{align}


Last time we considered hot topological sectors.
We have a non-bipartite lattice.
Fix reference line count parity of dimer crossings $n_c\mod 2$.
And $n_c\mod 2$ is invariant under logical dimer moves.

This winding number $n_c$,
the reference line matters.
If you choose a different reference line,
this number does change,
so to define the topological sector you must fix the reference line first.
You can see this if you draw a triangular lattice.
Because each site is connected to exactly one dimer,
if I change my reference line to cross a particular site near it,
I will always cross an extra dimer when I deform the line like this.
Because each site is part of 1 dimer,
$n_c\mod 2$ depends on the reference.
Once we fix our reference looks and do this counting,
the result is going to be an invariant of these local moves.
On a torus,
we can consider two reference loops corresponding to non-contractible cycles of
the torus and we get the 2 non-contractible sectors.

On a bipartite lattice,
we pick 2 reference lines and nay integer associated with 2 reference lines.


\subsection{Topological Excitations}
\begin{align}
    \ket{g.s}
    &=
    \sum_c A_c \ket{c}
\end{align}
Then we talked about topological excitations.
It's probably better to call them topological defects.
Depending on what phase you're in,
they may or may not be deconfined excitations.
We talked about visons,

Say you have a vison at $i$ and a vison at $j$.
Suppose that you had a ground state which was the sum over some dimer
configurations with some amplitudes.
Then you can introduce the visons by picking some reference line.
And $i$ and $j$ are the endpoints of the line.
So a vison $i$ and vison at $j$ connected by this line would be the same sum
with configurations
but a $(-1)^{n_c}$.
\begin{align}
    \ket\left\{ V_i V_j \right\}
    &=
    \sum_c
    A_c \left( -1 \right)^{n_c} \ket{c}
\end{align}
Think of it as a $\mathbb{Z}_2$ vortex,
because it's like a branch cut,
and along the branch cut,
we change the state depending on how many dimers are crossing it.
So $-1$ associated with branch cuts.

Then you can break them into 2 monomers,
and you allow the monomers to move around.
The local moves only break the dimers.
If I had a dimer on some particular link,
then that creates two monomers or vice versa.
\begin{align}
    \left( \ket{:}\bra{|} + \mathbb{h.c.} \right)
\end{align}
A single monomer requires some string of moves to break and move far apart.
A single monomer is thus a topological excitation
because it cannot be created by any local operator.

Let's rephrase some notation.
Let's call the reference line $\gamma$,
and the number of crossings along it $n(\gamma)$.
Then
\begin{align}
    \ket\left\{ V_i V_j \right\}
    &=
    \sum_c
    A_c
    \left( -1 \right)^{n_c\left( \gamma \right)}
    \ket{c}
\end{align}
If you fix $\gamma$,
this will have multiple configurations,
but I haven't told you they are all related by local moves.
This $\gamma$ here is just a line segment.
The number of line crossings is not topologically protected.
It could be topologically protected if it were a closed loop though.
But here the $\gamma$ is a line segment that starts at $i$ and ends at
$j$.

\begin{question}
    Different dimer coverings could ave different $n_c(\gamma)$'s?
\end{question}
Yes that's right.
If $\gamma$ is a non-contractible loop,
then local moves will not change $n_c\left( \gamma \right)$.

\begin{question}
    Are monomers topological excitations?
\end{question}
There's no operator that creates a single monomer.
The only way is to break a dimer,
get 2 monomers and move one far away.
This has become important recently,
because last year in Harvard,
in a Rydberg atom array,
they managed to create the quantum dimer model,
and a particular ground state that corresponds to a $\mathbb{Z}_2$ spin liquid.

\begin{question}
    Dimer models can be defined even without translational symmetry
    on arbitrary tribulations?
\end{question}
Yes it's more general.

\begin{question}
    If you have a square lattice,
    and or have a dislocation,
    it will become non-bipartite.
\end{question}
We can do it on bipartite or non-bipartite lattices.

\subsection{Square Lattice Dimer Model}
There is a particular point in the parameter space where you set $V=t$.
This point is called the RK point, which stands for Rocksta-GUilson.
When you set $V=t$,
then you can write the Hamiltonian as
\begin{align}
    H &=
    V\sum
    \left( 
    \ket{||}
    -\ket{=}
    \right)
    \left( 
    \bra{||}
    -
    \bra{=}
    \right)\\
    &= V \sum_{\square} P_{\square}
\end{align}
And the point here is that this is actually a sum of projectors,
so you can define a projector on each plaquette,
so you're just summing over plaquettes.
Projector means if you square it you get the same thing.
\begin{align}
    P_{\square}^2 = P_{\square}
\end{align}

The nice thing is that the ground state is the state that is annihilated by
every projector.

\begin{question}
    Is that always possible?
\end{question}
You don't know it's always possible,
but here it is.
You can have projects that are weird.
Here it will be possible.

A generic wave function that is annihilated by projectors takes the form
of a sum over configurations $c$.
\begin{align}
    \ket{\Phi} &= \sum_{c} A_c \ket{c}
\end{align}
If the configurations $c$ and $c'$ differ by some flipped plaquette,
then $A_c = A_{c'}$.
So you can write a ground state as a uniform superposition over all coverings,
but a restricted sum over all configurations that can be obtained by local flip
moves.
\begin{align}
    \ket{\Psi_0}
    &=
    \sum_{c} \ket{c}
\end{align}
This is a uniform superposition over dimer coverings related by flipping terms.

This RK point is nice because we have a nice understanding of the wavefunction
as just a uniform superposition over dimer coverings related by these local
flips.
But the actual properties of the state,
like what phase of matter it corresponds to depends sensitively on what
dimension you're in and the specific geometry of the lattice.
Here we're on a square lattice,
but it could be triangular.
What phase and what properties depends sensitively on lattice geometry and
dimension.

Depending on lattice geometry and dimension,
RK point has different physics.
If you're on a 2D bipartite lattice,
for example a square or honeycomb lattice,
the RK point is actually a critical point
between 2 different ordered phases.
On the other hand,
if you're on a non-bipartite lattice,
then the RK point is just one point in a larger phase that corresponds to a
$\mathbb{Z}_2$ short
range RVB phase,
sometimes called a $\mathbb{Z}_2$ spin liquid.
I think the RK point in 3D is part of really interesting gapless phase of spin
liquid.
To figure this out,
you in general need to do complicated numerics,
but writing the Hamiltonian and doing numerics to diagnose what phase you're
in,
and perturb the Hamiltonian to see what the properties are.
It's generally not something to be understood analytically.

\begin{question}
    The ground state is a uniform superposition over states related by local
    moves,
    but with different topology sectors,
    we have different ground state for each sector?
    Also, the classification depends on which reference loop we choose?
\end{question}
That might affect how you label the states,
but not that they are different states.

So that's what happens when $V=t$,
but what about other limits?

What if $V/t \to \infty$?

One phase is called the \emph{columnar phase},
where you have dimers just along columns like this.
This breaks translation symmetry,
and it breaks rotation symmetry.
Here this would happen in the limit $t>0$ and $V\to -\infty$.
You want to satisfy that second term in the Hamiltonian,
which really likes plaquettes which look like this,
so it has the maximum number of flippable plaquettes.

On the other hand,
if $t>0$ and we take $V\to +\infty$,
then we want to minimize the number of flippable plaquettes.
If you really want to violate that second term in the Hamiltonian,
you would have a staggered configuration.
No plaquette looks like that second term.
These are different kinds of ordered phases that break translational and rotating
symmetry.

Spin liquid type phases on the critical point are particularly interesting
because they are described by gauge theories.

This quantum dimer model so far,
may look familiar to you,
they arise in the context of gauge theory,
like $U(1)$ and $\mathbb{Z}_2$ gauge theory.
There's in fact a mapping from quantum dimer model to gauge theory.
And this RK point will correspond to some gapless U(1) gauge theory,
and iff you're non-bipartite,
the RK point corresponds to the deconfined phase of a $\mathbb{Z}_2$ gague
theorey.

\begin{question}
    In the 2D non-bipartite phase,
    is there a more complciated orderd phase?
\end{question}
There's generally huge different kinds of orderd phasess.
You essentially break the symmetyr into any subgroup of that.
If you write down a paritcular Hamiltnoian on the tringaulr lattice,
it's a complicated story you need to do mnumerics to figure out.

\begin{question}
    How does this relate to the original system without orthogonal states?
\end{question}
You could try to hink of the Dimer model on its own and try to realise that,
like the Rydberg atom array.
So physiical models on themselves.

You can think of this quantum dimer model,
imagine on each link htere is a particle with occupation number 0 or 1,
so you can map this quantum dimer model to some kind of model of bosons defined
on links.
The fact that dimer coverings will be orthognal,
but there will be a dimer constraint,
so no two sites will have 2 paricles,
so each site will have 1 boson.
So you map it to a model of bosons,
where you forbid any link that talks to a different site.

each dimer maps to the occupation number of a boson.
The constraint that each site is connected to 1 dimer becomes a constraint that
the sum of the occupation number of the bosons for each link,
\begin{align}
    \sum_{l \in +} n_{b, l} = 1
\end{align}
for any link that's conected to a star site has to equal 1.
You can map the dimer model to a boson hopping and make the constraint an
energetic constraint,
so the Hamiltonani is bosons hopping on a link,
plus a sum over all sites and sum over all links with some coefficietn $K$
infinitely large.
\begin{align}
    H &=
    -t \sum\left( b_i^\dagger b_i + \mathbb{h.c.}\right) 
    - K
    \sum_s
    \left( 
    \sum_{l \in +}
    n_{b,l} - 1
    \right)^2
\end{align}
If you start with the trignaulr lattice,
you get the Kagome lattice,
which is the balanced Fisher-Girven mdoel,
a model of bosons hopping on a Kagome lattice,
with dimer constraints made energetic.

Just think of the dimer mdoel as some constarinted boson model,
or just think of it as inspirational from the original spin-singlet problem.
Both perspectives are used.

\section{Quantum Lattice Gauge Theory}
But first,
let's talk about lattice gauge theory.

Let's start with U(1) lattice gauge theory on a square lattice.
We have gauge fields defined on links.
I'm going to call it $a_{ij}$,
which is a $U(1)$ gauge field defined on links.
And then $a_{ij}$ is actually an angle $a_{ij}\in [0,2\pi)$.
Furthermore,
if we have some gauge field defined on some link,
we should think of the link as being directed.
When I say $a_{ij}$,
I really mean them link is directed from $i$ to $j$,
and if it's the other away around,
\begin{align}
    a_{ij} &= -a_{ji}
\end{align}
In addition  to the gauge field,
we have some matter fields which are defined on site vertices,
$c_i$ which could be a boson or fermion operator,
agnostic for now.
It carries charge 1 under this $U(1)$ gauge field,
which means under a gauge transformation
\begin{align}
    c_i &\to e^{i\theta_i} c_i\\
    a_{ij} &\to a_{ij} + \theta_i - \theta_j
\end{align}

\begin{question}
    How are the links directed?
\end{question}
There's no constraint?
You just pick everything in the same direction.

A generic Hamiltonian will have nearest-neighbour hopping terms
and a magnetic field and electric field term.
\begin{align}
    H_{U(1)}
    &=
    -\sum_{\langle ij \rangle}
    t c_i^\dagger c_j e^{ia_{ij}} + \mathbf{h.c.}
    - K
    \sum_{\square}
    \cos\left( b_{\square} \right)
    +
    J
    \suM_{\langle ij \rangle} e_{ij}^2
\end{align}
where $b$ is a magnetic field associated with each plaquette.
Each $b_{\square} =
a_{ij} + a_{jk} + a_{kl} + a_{li}$
can be thought of as a holomony term.
You could think of $a_{ij}$ as the phase of some rotor variable,
so a phase on a $U(1)$ rotor,
and $e_{ij}$ would be its corresponding angular momentum.

So $e_{ij}$ would have integer eigenvalues.
And canonical commutation relations between them are
\begin{align}
    \left[ e_{l}, a_{l'} \right] &=
    i \delta_{ll'}
\end{align}
Equivalently,
this is just saying $e^{ia}$ acts like a raising operator for the electric field
\begin{align}
    \left[ e_{l}, e^{ia_{l'}} \right]
    &=
    e^{ia_{l'}}
    \delta_{ll'}
\end{align}
So $e^{ia}$ is the raising operator for $e$.
So it's like having rotor degrees for freedom where $e$ is the angular
momentum and $a$ is the phase.
But there is an important gauge redundancy,
which is the same as saying that we need to have a Gauss law,
$\nablea\cdot e = n$
where $n$ is charge density.
But here this is a lattice divergence.
To be specific,
suppose I have a site $i$,
and I have links directed inwards $l_1, l_2, l_3, l_4$.
What that means is that I add up electric field coming in from these links and
that should be equal to the density on that site.
\begin{align}
    e_{l_1} + e_{l_2} + e_{l_3} + e_{l_4} &= n_i
\end{align}
Thus the Gauss law \emph{generates}
the gauge transformation,
which means that suppose you had some state $\ket{\left\{ a_{ij}, n_i \right\}}$
then
\begin{align}
    e^{i\sum_j \left( \nabla\cdot e - n_j \right) \theta_j}
    \ket{}\left\{ a_{ij}, n_i \right\}
    &=
    \ket{\left\{
    a_{ij} + \theta_j - \theta_j
    ,
    n_i
    \right\}}
\end{align}
The first term,
you can integrate by parts,
and what a single $e$ does is that $e^{ie\theta}$ shifts $a$ by $\theta$.
This is just the same as the gauge transformation I wrote down before for $c_i$
and $a_{ij}$.
The gauge transformation changes the phase of the state by this amount.
Because
$e^{i\sum_j \left( \nabla\cdot e - n_j \right) \theta_j}=1$,
it means that
\begin{align}
    \ket{\left\{ a_{ij}, n_i\right\} }
    =
    e^{-i\sum _j n_j \theta_j}
    \ket\left\{ a_{ij} + \theta_j - \theta_i , n_i\right\}}
\end{align}
This is the gauge redundancy in our description.
Let's write it down explicitly.
For the second term,
\begin{align}
    e^{-in_j \theta_j} \ket{n_j}
\end{align}
is equivalent to
$c_i \to e^{i\theta_i} c_n$.
For the first term,
\begin{align}
    e^{e_{ij\left( \theta_j - \theta_i \right)}}\ket{a_{ij}}
    &=
    \ket{a_{ij} + \theta_j - \theta_i}
\end{align}
In the continuum limit,
this becomes a continuum $U(1)$ gauge field coupled to matter.
\begin{align}
    H &=
    \int d^2 r
    \left[ 
    c^\dagger (r)
    \frac{\left( i\partial - a \right)^2}{2m}
    c(r)
    -
    \tilde{K} \cos b
    - \tilde{J} e^2
    \right]
\end{align}
And remember pure maxwell theory has a Hamiltonian that is proportional to $E^2
- B^2$ with
\begin{align}
    H_{\text{Maxwell}}
    \propto e^2 - b^2
\end{align}
Here,
the coefficients for the $e^2$ and $b^2$ terms are not the same,
and remember the reason they're the same in Maxwell is because of Lorentz
invariance,
but there is no Lorentz invariance in the lattice gauge theory,
so there's no reason why $\tilde{K}$ and $\tilde{J}$ would be related to each
other.
So we have no special relationship between $\tilde{K}$ and $\tilde{J}$.

In the Hamiltonian framework,
$e$ is the canonicmal momentum relative to $a$.
But $b$ would be the curl of $a$.
In the continuum,
you would have
\begin{align}
    b &= \nabla\times a
\end{align}
while $e$ and $a$ are canonically conjugate.

\subsection{$\ZZ_2$ gauge theory}
So $\mathbb{Z}_2$ gauge theory is just breaking the $U(1)$ down to
$\mathbb{Z}_2$.
Can understand by ``spontaneous gauge symmetry breaking''
It will spontaneously obtain an expectation value.
You pick an operator with charge 2,
so something like $\langle C^2 \rangle$.
Then you calculate the expectation value of $C^2$,
and if it's non-zero,
then that's gauge symmetry breaking.
The transition where this happens is sometimes called the Anderson-Higgs
transition,
but colloquially we sometimes call it ``Higgsing''.
Higgs is a verb now.

More formally,
we can define $\mathbb{Z}_2$ gauge theory as follows.
Again we have this square lattice,
which has gauge fields on links,
and matter fields on sites.
Here our $\ZZ_2$ gauge fields are $\sigma_{ij}^z= \pm 1$ Pauli $Z$ operators.
Then $\ZZ_2$ charged matter defined on sites.
I'm going to take the matter fields to be Ising spin variables,
so there's a basis where its eigenvalues are $+1$ or $-1$.
So we have a spin operator defined on sites $S_{i}^{x}$
which takes $-1$ to $+1$.
Though in general the matter can be either bosonic or fermionic.
To make a connection to $U(1)$ gauge theory,
you could think
\begin{align}
    \sigma_{ij}^{z} &7=
    e^{ia_{ij}\pi}
\end{align}
and then
\begin{align}
    \sigma_{ij}^{x}
    &=
    e^{i e_{ij} \pi}
\end{align}
can be thought of as an electric field.
Here $a_{ij}$ and $e_{ij}$ have eigenvalues 0 or 1 mod 2.
Then the gauge transformation is
\begin{align}
    \sigma_{ij}^z \to \lambda_i \lambda_j \sigma_{ij}^z
\end{align}
for $\lambda_i = \pm 1$,
while
\begin{align}
    a_{ij} \to a_{ij} + \theta_j + \theta_i \mod 2
\end{align}
and 
\begin{align}
    S_i^x \to \lambda_i S_i^x
\end{align}

For simplicity,
consider nearest-neighbour hopping.
Remember these $S^x$ are boson creation and annihilation operators,
but because it's mod 2,
creation and annihilation are the same thing.
\begin{align}
    H &=
    -\sum_{\langle ij \rangle}
    t 
    S_i^x S_j^x \sigma_{ij}^Z
    - K
    \sum_{\square}
    \prod_{\langle ij\rangle \in \square}
    \sigma_{ij}^z
    - h
    \sum_{\langle ij \rangle}
    \sigma_{Ij}^x
\end{align}
And this is the thing that minimizes the magnetic fields at the plaquettes.
This is what you would call $\mathbb{Z}_2$ lattice gauge theory
coupled to matter.
Although in this specific example,
the matter itself is an Ising spin.

One last thing we haven't done is the Gauss law,
which we did in the U(1) case but no the $\ZZ_2$ case.
Before we had a charge on each site,
but now we need a $\ZZ_2$ charge on each site.
So let us define a $\ZZ_2$ charge on each site.
Call our $\ZZ_2$ charge $n_i = 0 \mod 2 $ in some basis.
And so
\begin{align}
    \left( -1 \right)^{n_i}
    \ket{S_i^z} = s_i^z \ket{s_i^z}
\end{align}
The matter field is an Ising spin that is pointing in up or down direction,
but if it's up we say it's $\ZZ_2$ charge even,
and odd if down.
If I start with $\ZZ_2$ charge even on two neighbouring sites,
then that's going to create $\ZZ_2$ charge odd on the neighbouring sites.

Let $\sigma^x = \ZZ_2$ be the electric field and then
\begin{align}
    \left( \nabla\cdot e \right) - n_i = 0 \mod 2
\end{align}

\begin{question}
    Is there a generalization to higher neighbours?
\end{question}
Yes.
Consider a term like
\begin{align}
    \sum_{ij}
    c_{i_1}^\dagger c_{i_n}
    e^{i\left( 
    a_{i_1 i_2}
    + a_{i_2 i_3}
    + \cdots +
    a_{i_{n-1} i_n}
    \right)}
\end{align}

\begin{question}
    What's the classical analogue?
\end{question}
There's multiple things you could say.
First you could ignore the magnetic term,
and it's purely classical,
this is the 2D classical $\ZZ_2$ gauge theory coupled to matter.
If you think fo the path integral or partition function of this quantum model,
spacetime is 3D,
youcauld map that to a classical 2D Ising model.

If I draw some site $i$ and consider its neighbouring links $l_1,l_2,l_3,l_4$,
then what the Gauss law is saying si that
\begin{align}
    \sigma_{l_1}^x
    \sigma_{l_2}^x
    \sigma_{l_3}^x
    \sigma_{l_4}^x
    &=
    s_i^z
\end{align}
And this means that
\begin{align}
    \left( 
    \prod_{l\in +} \sigma^x
    \right)
    S_i^z
    = +1
\end{align}
which is the Gauss law constraint.

The cleeer thing:
Inseead of thinking it as a hard constrain on the Hilbert space,
you cna think of it as an energetic constraint you add as an extra term in the
Hamiltonian.
\begin{align}
    H &=
    -\sum_{\langle ij \rnagle}
    t S_i^x S_j^x \sigma_{ij}^z
    -
    K \sum_{\square} \prod_{\langle ij \rangle} \sigma_{ij}^z
    - h \sum_{\langle ij \rangle} \sigma_{ij}^x
    - J \sum_i
    \left( 
    \prod_{l \in +} \sigma_l^x
    \right)
    S_i^z
\end{align}
with $J\to\infty$.


Now let's assume $S$ forms a trivial paramagnet.
\begin{align}
    \langle S_i^x \rangle &= 0\\
    \langle S_i^z \rangle &= 1
\end{align}
Then we can do a mean field approximation
\begin{align}
    H_{mf} &=
    - K \sum_{\square} \prod_{l\in\square} \sigma_l^z
    - J\sum_{+} \prod_{l \in +} \sigma_l^x
    - h\usm_{\langle ij\rangle} \sigma_{ij}^x
\end{align}
THis is pure $\ZZ_2$ lattice gauge theory because there are no matter fields,
but it's a pure $\ZZ_2$ lattice gague theory iwthteh Gaus law treated
energetically.

If you set $h=0$,
you get the famous Kitaev toric code Hamiltonian.
IT's really just $\ZZ_2$ latice gauge theory wher you treat Gauss law as an
energetic cosntraitn.
This was fundanetmally a conseptaionlly umpotatn thing to  do.
Rather than not treating the Gauss law as a fundmenatl Hilbert space constrat,
was not a trivia step.
Befoer the Hilbert space had a hard constarin,
adn we canont factor.
Now instead,
the Hilbert space is much bigger,
but it does factorize.
Before when you implmeent the constraint it was not a spin model.
Here we havea spin mdoel that reproduces the gauge field.
Kitaev became famous for making this 1 important step.
everything was knonwn since the 1907s,
but hte new insight was to make the Gauss law constraint energetic.
