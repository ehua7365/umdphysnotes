\section{Quantum Phases of Gapped Hamiltonians}
Revision.
There are two definitions of phase.

The first one is this:
$H_1$, $H_2$ are in the same quantum ($T=0$) phase of matter
if and only if
we have a continuous parameter $\lambda$ and a path connecting them such that
$H(\lambda)$ stays open.

The second definition is the traditional thermodynamics definition.
$H_1$ and $H_2$ are in the same phase if and only if
there is a path $H(\lambda)$ such that
\begin{align}
    \lim_{N\to\infty} \frac{E(\lambda)}{N}
\end{align}
is smooth.

The third definition is this:
$H_1$ and $H_2$ are the same phase if and only if
there is a path $H(\lambda)$
with ground states $\ket{\psi_i(\lambda)}$
such that the ground state expectation value
${\langle\mathcal{O}\rangle}_{\lambda}$
is a smooth function of $\lambda$ for every local operator $\mathcal{O}$.

Some remarks:
\begin{itemize}
    \item We will use (1)
    \item (2) is the usual thermodynamic definition in terms of free energy
        density.
    \item (2) and (3) apply for gapless systems.
\end{itemize}

There is a conjecture.
\begin{conjecture}
    If $H(\lambda)$ remains gapped,
    then ${\langle\mathcal{O}\rangle}_{\lambda}$
    is a smooth function of $\lambda$.
\end{conjecture}
This implies that if two phases are the same according to (1)
then they are also different according to (3).

\section{Constant Depth Circuits}
You can have two states in the same phase
but they might not be related by a constant depth local circuit.

We have a bunch of local gates that act on our qubits.
Local quantum circuit is a circuit where each layer has gates with finite
support locally.
It is a composition of these layers.
Constant depth means that the number of layers is less than a constant as the
size goes to infinity.

Let $U$ be a constant depth local unitary circuit.
If $\ket{\psi}$ is the ground state of some Hamiltonian $H$,
then $U\ket{\psi}$ is in the same phase.

However, it is not true that if I have two states $\ket{\psi_1}$ and
$\ket{\psi_2}$ 
then $\ket{\psi_1} = U\ket{\psi_2}$.

An example of this is the following.
Suppose $\ket{\psi_1}$ is a product state.
The correlations must be exactly zero beyond a certain range,
because it cannot create correlations beyond a certain distance $\xi$ in space
that has to do with the gates and range.

If you have one state which is a trivial phase of matter
but correlations have exponentially decaying tails,
you cannot go to one with $U$.

You can fix this by allowing the notaries have tails,
but then it's tricky
and I've never seen anyone made this work.
It's not that useful.
If only useful to show two states are in the same state.

If you allow logarithmic depth,
you can connect states that are in different phases
and that's too much.

One hope would be to give up the fact you want it to be exactly
$\ket{\psi_1} = U\ket{\psi_2}$
and make it constant error instead.

In particular,
if you have a gapped ground state $\ket{\psi_1}$
and this unitary circuit $U$,
then there should be an adiabatic path between the Hamiltonians,
and you should be able to construct that unitary path
given this circuit.

One more point is that you can add symmetry to the problem.
If you assume the problem has symmetry,
then you must make sure that whole path respects the symmetry.
For example,
quantum hall phases have paths that are sometimes incompatible with
time-reversal symmetry.

When you add symmetry,
you end up splitting into many phases that cannot be connected by
symmetry-respecting paths.

\section{Topology in Gapped Ground States}
Now let's connect this to the mathematical structure of TQFT.

Here is some gapped phase of matter,
but where is the topology?

The properties robust to arbitrary perturbations.

Universal properties that distinguish different phases are usually characterised
by TQFTs.
On the one hand,
I can think of gapped quantum phases of matter.
The idea is that as far as we know, although there's no proof,
these are in one-to-one correspondence with deformation classes of TQFTs.

This statement as I said it is not quite right,
depending on how you define TQFT.
We believe this is true in $1+1$ and $2+1$ spacetime dimensions,
but in higher dimensions,
you could have more complicated things happening.

Topological liquids are interesting.
There is the space of all gapped phases of matter.
Then inside this space,
there are topological liquids
which have one-to-one correspondence with deformation classes of TQFTs.
Whereas gapped phases for $D\ge 3 + 1$,
we're not really sure,
but we thin they are in one-to-one correspondence with
deformation classes of TQFTs with \emph{defect networks}.

The mathematical structure TQFT has a log of topology.
They essentially classify gapped phases of matter with some caveats.
TQFTs may come with continuous parameters.
You have to be really careful what these objects are.

\begin{question}
    What are deformation classes?
\end{question}
A TQFT $\mathcal{T}_\lambda$ can be split up into spaces and points.
You could have some theory $\mathcal{T}^1_\lambda$ for some continuous parameter
$\lambda$,
you could have another one $\mathcal{T}^2_\lambda$
and then you have some which are just points $\mathcal{T}^3_\lambda$, etc.

That there are TQFTs without lattice descriptions,
some TQFTs do not have an exact lattice description.
Every TQFT useful for gapped phases of matter
can arise from some microscopic description on a lattice.
It might not be exactly solvable.

The rough intuition is that if you want to define TQFTs,
you want to make sure all quantities that come up are
topological invariants.
If you use triangulation of the space you're describing,
it should be independent of the triangulation.
Some gapped phases of matter require you to specify some microscopic geometry,
and you can't avoid it.

There's nothing outside of topological liquids outside of 1 and 2 spatial
dimensions,
but in 3 dimensions it's weird and clever.

\section{Topological Quantum Field Theory Basics}
I want to give you the plain vanilla version.
The discussion is abstract and mathematical,
but it's worth doing before moving into examples.

I want to tell you some basics of TQFTs.
I want to talk about ``plain'' TQFT,
but there are additives like oriented and unitary.
I will tell you the plain TQFT and then soup it up.

TQFT is this really amazing and intricate math structure that \emph{emerges} out
of some messy quantum many-body system with an energy gap.
It's not obvious why it should emerge from a gapped quantum many-body system.

To me, it's an absolutely amazing and profound thing.
You can have something so mathematically rich just emerge from some quantum
many-body system.
What I'm going to discuss are the \emph{Atiyah's axioms}.
Atiyah is one of the foremost mathematicians of the century
who died recently.
When discussing with Witten,
Atiyah saw what Witten was doing and axiomised it.
So it listed some axioms TQFTs should satisfy.
So it is an axiomatic framework that describes TQFTs.
One interesting aspect is that it is a way of defining a restricted class of
QFTs in a rigorous fashion.
In QFT, you have problems with lattice discretisation and perturbation theory.
But TQFT can be mathematically rigorously defined and used in physics,
so it's beautiful common ground.

These Atiyah axioms are related to some other axioms proposed at around the same
time, including those by Segal,
who was trying to define 2D \emph{conformal field theory} axioms,
which has a lot more structure that TQFTs.

These are useful for eventually giving a rigorous definition of quantum field
theory.
A few months ago there was an article in Quanta magazine about the mathematical
problem of defining QFT.\@
How do you define a QFT,
no one really knows.
TQFT might be useful.

\begin{question}
    What's the problem with QFT?\@
\end{question}
We know how to rigorously define free field theory.
We know how to rigorously define perturbation theory of free field theory.
There was this whole program of axiomatic quantum field theory in the late 20th
century that never became useful and cannot capture what we know.
We can't understand 2D CFT from the ideas that we had.
You need a whole new set of tools,
like operators algebras, a complicated story.
You can rigorously define some aspects of simple QFTs
but people are after something more broad
that can explain all the things we know about all the QFTs we know.

I'm not going to say anything mathematically complicated.
Feel free to say if you don't understand some mathematical terminology,
don't be shy.

There's an abstract definition of plain vanilla TQFT.
\begin{definition}
    A $(d+1)$-dimensional TQFT is is a symmetric monoidal functor
    between two categories $\mathcal{F}: \mathrm{Cob}_{d + 1}\to \mathrm{Vec}$.
\end{definition}

$\Cob_{d + 1}$ is a tensor category of $(d + 1)$-dimensional cobordisms.

A \emph{category} is a mathematical structure that consists of objects
and \emph{morphisms} between objects.
At a naive level,
it's a set of objects and arrows between objects called morphisms.
And this is a 1-category.
In our case,
objects of $\Cob_{d + 1}$ are $d$-dimensional closed manifolds
and a morphism between two objects,
which are $d$-dimensional manifolds,
say from $\Sigma_1^d\to\Sigma_2^d$,
is a $(d+1)$-dimensional manifold $M^{d+1}$
such that its boundaries are $\Sigma_1^d$ and $\Sigma_2^d$,
i.e.
$\partial M^{d + 1} = \Sigma_1^d\sqcup \bar{\Sigma}_2^d$.

Let's do an example $\Sigma_2 = S^1$, which is a circle
and we have a cobordism to $\Sigma_1 = S^1\sqcup S^1$.
A morphism looks like this
[picture of pair of pants]
The manifold $M^{d+1}$ is called a (co)-bordism
from $\Sigma_1^d$ to $\Sigma_2^d$.
Bord means boundary in French.
The point is that if you hae two manifolds,
and if you can go from one another in a highr-dimensional manifold,
then that's a bordism, sometimes called a co-bordism.

\begin{question}
    So there's nothing differnt betwen cobordism and bodirsm?
\end{question}
Two words for the same hting,
but in some cnotexts,
they'll call maps from this into $U(1)$ as a cobordism.
Mahtematicians do yuse oth interchangably.

\begin{question}
    Can you go other that such that clause?
    What is $\sqcup$?
\end{question}
Objects of this category are $d$-dimensiona closed manifiolds.
A morphism is a way to gor from one objcet toa nother.
A cobordism is a $d+1$-dimeisional manifold with boundaries that are the
objects.
It's disjoint union.

\begin{question}
    Category of sets, category of topological sets.
    Weird notation?
\end{question}
Objects of Vec are vector spaces and the morphisms are maps betwene vector
spaces, so it is a bit weird.

\begin{question}
    Are there some topological properties?
\end{question}
If two manifolds are bordred with each other,
it just means they're boundaries of some higher-dimesional manifold,
but that's about it.

\begin{question}
    Are two manifolds always connected by a $(d+1)$-dimensional manifolds?
\end{question}
No, there's a cobordism group notion,
group by equivalence classes based on whether there is a cobordism betwen them.

Every 2D manifold is related to every 2D manifold,
so that's trivial,
but it's not true in higher-dimensional space.

\begin{question}
    What are the points in between?
\end{question}
A morphism here is this higher dimensional manifold.

For every object, there may or may not be a morphism between them.

It's more than a map.
The TQFT gives a map, but $\mathrm{Cob}_{d+1}$ is the whole manifold.


\begin{question}
    What is a tensor category?
\end{question}
I'll come back to that.

\begin{question}
    What is they're not circles but spheres?
    What does that look like?
\end{question}
You want me to draw in 4D?\@

You have $\Sigma_1^2 = S^2 \sqcup S^2$ and
$\Sigma_2^2=S^2$.
Then the bordism is
$M^3 = B^3\sqcup (S^2\times I)$
and the boundary is
$\partial M^3 = S^2 \sqcup (S^2 \times S^2)$

\begin{question}
    Does the cobordism have to be connected?
\end{question}
No, it doesn't have to connected.
There are many possible morphisms between objects.

\begin{question}
    Can you give an example of what these objects are physically?
\end{question}
To some extent I can,
but it's related to why a miracle TQFT emerges out of a system.
II haven't gotten there yet.
I can define stuff on all kinds of spacetime manifolds.
I'm going to define stuff for eveyr $(d+1)$-dimensional spacetime manifodls.
But Condensed matter systmes don't have spacetime,
only time and space the system lives on.
It's hard to give examples which use all this technology irght now.
But a simple example is this.
Actually, hold that question until I get further.
Let me define a few things about it and then I'll try to address your question.


The next thing is the tensor category.
Tensor means there's a notion of a tensor product here.
There is a tensor product betwene objects
$\Sigma_1\otimes\Sigma_2$.
I need to give a meaning for tensor product,
which I say is just the disjoint union.
$\Sigma_1 \otimes \Sigma_2 := \Sigma_1 \sqcup \Sigma_2$.
Now I need to define $\mathrm{Vec}$, which is a tensor category.
Objects are finite-dimensional vector spaces of $\mathbb{C}$.
Morphisms are linear maps between vector spaces.

In our case,
because we're doing \emph{unitary} TQFT,
we want these linear maps to be unitary maps.
Actually, no unitary for now, we'll add it later.
The tensor aspect should be obvious.
The tensor product is just the tensor product of vector spaces.

A functor is just a map between categories.
For every object on the left side,
I have an object on the right side.
And for every morphism on the left I have on on the right.
Monoidal means it respects the tensor product on the left and right side.
Symmetric means it applies the same on each side.

What the TQFT is telling is is that
for every closed $d$-manifold $\Sigma_d$,
we have a finite-dimensional vector space $V(\Sigma^d)$.
For every $M^{d+1}$ cobordism between
$\Sigma_1^d$ and $\Sigma_2^d$,
we have an unitary linear map
$Z(M^{d+1}):V(\Sigma_1^d)\to V(\Sigma_2^d)$.

This is the outline of the more abstract definitoin.
Now I'm going to be more specific about exactly what data the axioms satisfy.
It's all hidden in this very abstract construction of a symmetric monoidal
uniatry functor between two categories.



\begin{question}
    You can only have unitary vector spaces of the same size?
\end{question}
Actually, it should be a linear map,
only unitary if they are the same dimension.
That's why you want the manifold to be closed.

\begin{question}
    Is this always defined for two objects?
\end{question}
The tensor product $\Sigma_1\otimes\Sigma_2$ defines for me a single object.
Then if I have another one
$\Sigma_1\sqcup\Sigma_2$.
[pair of pants picture]

Again, let me emphasize,
it's totally not obvious why something like this should come out of a quantum
many-body system.
I have linear maps associated with every $(d+1)$-dimensional bordisms.
Why should this structure arise?
It's not obvious.

\begin{question}
    $V(\Sigma^d)$
\end{question}
Every manifold is associated with a vector space.
You can think of $\Sigma^d$ as an argument.
For every $\Sigma^d$ you have a different vector space.

\begin{question}
    Does the dimension of the vector space have something to do with the
    dimension of the manifold?
\end{question}
No.

\begin{question}
    What is monoidal and symmetric?
\end{question}
I can write it down.

Monoidal means that
$\mathcal{F}(\Sigma_1^d\otimes \Sigma_2^d)
\simeq
V(\Sigma_1^d)\otimes V(\Sigma_2^d)$.
Here $\simeq$ means isomorphic.

Symmetric means that
$\mathcal{F}(\Sigma_1^d\otimes\Sigma_2^d)\simeq
\matcal{F}(\Sigma_2^d\otimes\Sigma_1^2)$.


\subsection{Defining data}
Let's define what's the data.
$V(\Sigma^d)$ is a finite-dimensional vector space.
There is a path integral or partition function $Z(M^{d+1})$.
When $M^{d+1}$ is closed,
then $Z(M^{d+1})\in\mathbb{C}$.
When $M^{d+1}$ has boundary, then
$Z(M^{d+1})\in V(\partial M^{d=1})$.

This is an alternative way of thinking about it.
When $M^{d+1}$ is closed, you have a point,
but when it has a boundary, you have a state.

\begin{question}
    If I have a 5D TQFT,
    can we define a state on $\mathbb{CP}^2$.
\end{question}
No path integral will define this.
But you still have a vector space.

\section{Atiyah's Axioms}
You could think of this as the data of a TQFT
and these are the axioms.

\begin{axiom}
    All manifolds have orientation.
\end{axiom}

\begin{axiom}
    $Z$ and $V$ are \emph{functorial} with respect to orientation-preserving
    diffeomorphisms of $M^{d+1}$ and $\Sigma^d$.
\end{axiom}
Let me try to explain what this means.

$M^{d+1}$ had two boundaries,
and $Z(M^{d+1})$ was thought of as a map.

Actually, we're out of time,
I'll tell you what this axiom means next time.

\begin{question}
    What does $V$ mean physically?
\end{question}
$V$ is the ground state degeneracy on that manifold.
