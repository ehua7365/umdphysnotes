$Z$ only depends on topology of $M^{d+1}$,
not on any geometry.
\begin{align}
    Z(\Sigma^d\times [0,1])
    = Z(\Sigma^d\times[0,2])
\end{align}
by assumption.

$Z(\Sigma^d\times I$ is the time evolution operator $e^{iHt}$.
This means the ``Hamiltonian'' of this TQFT is zero.
But often we can write a lattice mode,
with $H<0$ ground of $H$ will be described by TQFT.
\section{Some Consequences of Axioms}
\begin{proposition}
    If $f:\Sigma^d\to\Sigma^d$,
    then $\mathcal{U}_f$ is a linear map,
    where $\mathcal{U}_f$ is a homotopy invariant.
    It is indpeendent of contiuous defomrations of $f$.
\end{proposition}
Everytime I do a diffeo if that is continuously conneted to anohter
diffeomorphism,
I get another map.

The mapping class group
\begin{align}
    \mathrm{MCG}\left( \Sigma^d \right)
    = \frac{\mathrm{diff}\left( \Sigma^d \right)}{\mathrm{diff}_0\left( \Sigma^d
    \right)}
\end{align}
which implies $g\in \mathrm{MCG}\left( \Sigma^d \right)$
for $\mathcal{U}_g:V(\Sigma^d) \to V(\Sigma^d)$.
For unitary TQFT,
$\mathcal{U}_g$ is unitary.

A TQFT furnishes the MCG of a manifold.
Often you can characterise a TQFT completely by the representations of the MCG.

Another consequence is the following.
Let's see what the path integral actually means.
$Z(\Sigma^d\times S^1)$.
You can glue a tube into a torus over a circle.

This is the meaning of thep ath integral
\begin{align}
    Z\left( \Sigma^d\times S^1 \right)
    = \Tr_{V\left( \Sigma^d \right)} II
    = \dim V\left( \Sigma^d \right)
\end{align}
Consider gluing maps defined by
$f: \Sigma^d \to \Sigma^d$.
The other important thing is that
\begin{align}
    Z\left( \Sigma^d \times_f S^1 \right)
    = \Tr_{V\left( \Sigma^d \right)}\mathcal{U}_f.
\end{align}

\section{Physical interpretation}
$V(\Sigma^d)$ is the ground state subspace on $\Sigma^d$.
We say topological ground state degeneracy
because it depends on the topology of $\Sigma^d$.
There are other reasons.
The physical meaning is that $V$ is the ground state degeneracy.

Then $Z\left(\Sigma^d\times S^1 \right) = \dim\left(\Sigma^d \right)$
is the dimension of the ground state degeneracy.

Our system has some non-trivial operators
we can associate with diffeomorphisms,
and this guy is just the trace of that.
\begin{align}
    Z\left(\Sigma^d\times_f S^1 \right)
    = \Tr_{V\left(\Sigma^d \right)} \mathcal{U}_f
\end{align}
for a gluing map $f$.

In general
\begin{align}
    Z\left(M^{d+1} \right) &=
    Z\left( M_1^{d+1} \cup M_2^{d+1} \right)\\
    &=
    \langle
        Z(M_{1}^{d+1}), Z(M_{2}^{d+1})
    \rangle
\end{align}

We can build more and more complicated manifolds by gluing.

\begin{question}
    Are you saying where the mapping is the trace of the operator,
    how does that change how the operator acts on the system?
\end{question}
What we're saying is that we have some gapped local Hamiltonian,
that there are some unitary transformations we can do on our system
that map the ground state subspace back onto itself,
which correspond to various kinds of map $f$.

Let me give you one example.
Imagine a system defined on a square lattice,
where I identify opposite sides,
so a torus.
I can identify opposite edges with a shift, like a twist.
There's going to be some mapping between stsates of different identifications,
and if I go in a circle,
I go where I started,
and it corespodns to a unitary acting on the ground state subspace,
and this is an example of a non-trivial $\mathcal{U}_f$.
It maybe hard to find,
but it does exist.
The trace of the unitary defines for me this map.

That was not super concrete.

\begin{question}
    Is this space or space-time?
\end{question}
If our space was $\Sigma^d$ and suppose it was a torus.
If I cut my torus and cut,
and rotate by $2\pi$,
that's a large diffeomorphism,
and gives rise to a large non-trivla map $\mathcal{U}_f$.

If $\Sigma^d = T^2$ is my spacetime,
it's topologically the same.

\begin{question}
    There should be a theorem saying the ground state energies are\ldots
\end{question}
If you have a gapped phase of matter which is stable,
then you're going to have that.
You're right we do need an assumption on the ground state subspace.
Your Hamiltnoian might have a symmetry that is spontaneously broken by the
gorund state.
That wouldn't be a topological ground state,
but a symmetry-protected ground state.
I'm talking about properties that are stable whether or not symmetries are
broken.

\begin{question}
    If I take $T^2\times I$  and glue it up into $T^2\times S^1$,
    do I get the same manifold?
\end{question}
This is going to be a different manifold.
This is detailed topology question,
let's address that later.

That's for my review of plain vanilla TQFT if you like.
But the TQFTs that actually arise when we study topological phases of matter
have more structure than what I've given you right now.
I want to give you a brief review of what this extra structure is
when we use TQFT in nature,
or at least what we consider to be realistic Hamiltonians.

\section{Additional structure}
So far we have $V(\Sigma^d)$ and $Z(M^{d+1})$.
They are subject to axioms and consistency conditions.
But hte actual TQFTs that come up in matter,
there are 4 or 5 of them.
Let me write down their names and explain what they mean later.
\begin{enumerate}
    \item Defect TQFT
    \item TQFT with background gauge fields
    \item TQFT with spin structures (Spin TQFT)
    \item Framed TQFT, relevant for $(2+1)$-d chiral topological phases.
    \item Extended TQFT
\end{enumerate}
The rest of this lecture,
I'm going to do a brief review simple introduction to what each of these
actually mean.

The first most impoant thing is defect TQFT.
The idea is that instead of just conidsering $\Sigma^d$,
we're going to remove submanifodlds of space and spacetime,
and label those removed sjbmanifodls.
And then all objects depend on the labelling of these remove submanifkds.
Rename submanifolds at various co-dimension from
$\Sigma^d$ and $M^{d+1}$ and give labels to them.

\begin{example}
    Rename points from $\Sigma^d$,
    made with labels from finite label set.
\end{example}
Physically,
we have quasiparticles
which come in topological super selection sectors.
They might be something to the action of local operators.
I.e. topological equivalence classes of quasi-particles,
or zero-dimensional defects.
Suppose we have points $a,b,c,d$.
Then we get $\sum_{a,b,c,\ldots}^{d}$ and
$V\left(\sum_{a,b,c,\ldots}^d \right)$.
$M^{d=1}$ will have 1d variables of marked points.
These marked points are described by some sort of link.
$\left( \Sigma^d\times I, \mathrm{Link}\right)$.

Or I can think of $S^3$,
and we cna have links
$Z(S^3, \mathrm{Link}_{a,b,c,\ldots})$.

We can remove lines, points and surfaces.
As we oncreaseigcal jjoWe'd thh n=rules.
Physics is interested in the excitations,
like string, membranes.

Hamilotians of TQFT are jusoOr=fSome quantiies only appear in a mi=acrosjjjk&=&
No every one coresponds toa stable quasiparticle oprati
You might use the word defect instead.
A couple of weeks again,
 did that,
 added a oitetakmm
 Imaginiea sutatiln
 thhhhhhhhhhhhhhhhhhhhhhhhhhhhhhhhhhhhhhhhhhhhhhhhhhhhhhhhhhhhhhhhhhhhhhhhhhhhhhhhhhhhhhhhhhhhhhhhhhhhhhhhhhhhhhhhhhhhhhhhhhhhhhhhhhhhhhhhhhhhhhhhhhhhhhhhhhhhhhhhhhhhhhhhhhhhhhhhhhhhhhhhhhhhhhhhhhhhhhhhhhhhhhhhhhhhhhhhhhhhhhhhhhhhhhhhhhhhhhhhhhhhhhhhhhhhhhhhhhhhhhWhats
 theooint of jjjGj
 
 \susection{TQCF with background gauage field}
 TQFT phaseppoewe haveaoWhenever oCan always turn
 backwrads gauage fields into speed.)
 Everything should be familiar with this.
 Suppose the system has $U(1)$ charge summetry.


 Suppose we have a global symmetry $G$.
 Can turn on background guage field $A$.
 Mathematially, $A$ is a conection on principal $G$-bundle.
 We want to equpi these manifolds iwth background guage fields.
 So $\Sigma^d$ and $M&^{d+1}$ are equipped with background gauge fields
 \begin{align}
     V\left( \Sigma^d, A \right)\qquad
     Z\left( M^{d+1}, A \right)
 \end{align}
 For example, consider a 1+1 dimensional TQFT.
 \begin{align}
     Z\left( M^2, A \right) =
     Z(M^{d}, A=0) e^{i\theta \int_{M^2}F}
 \end{align}
 recalling that the field strength is given by
 $F_{\mu\nu} = \partial_\mu A_\nu - \partial_\nu A_\mu$
 for $G=U(1)$.
 For 3+1 dimensions, we can do something similar with
 \begin{align}
     Z\left( M^4, A \right) =
     Z\left( M^4, A=0 \right)
     e^{i\theta \int F\wedge F}
 \end{align}
 where 
 $F\wedge F =\epsilon_{\alpha\beta\gamma\sigma} F_{\alpha\beta}
 F_{\gamma\sigma}$.
 These are called $\theta$-terms.

 This is where the notion of deformation classes becomes important.
 Deformation class of TQFT.
 The kinds of terms I want to add are topological terms.
 Meaning we can write down these terms without picking any metric for spacetime.
 If you did have a metric,
 you could have even more terms,
 but because we're working with TQFT,
 we're working with terms independent of the metric.

 \begin{question}
     Is there a reason to use a background gauge field
     with the symmetry?
 \end{question}
 Sometimes, you might want to describe a TQFT gauging another TQFT which does
 have a symmetry.

 \begin{question}
     ???
 \end{question}
 You can certainly consider the ratio
 $Z\left( M^d, A \right)/Z\left( M^d, 0 \right)$.

 \begin{question}
     $A$ is the connection of the principal $G$ bundle.
     Is a TQFT with gauge field just a TQFT on that principal $G$ bundle.
     Can I think of a TQFT with a background gauge field
     as a plain TQFT on a principal $G$-bundle.
 \end{question}
 Actually, I think so.
 I should maybe think more.

 One final comment.
 In recent years, it's becoming vogue to think about more symmetries than a
 regular symmetry.
 $A$ is a one-form,
 it lives on the links if you discretize spacetime simplex.
 $A$ could also be a higher $p$-form.
 You could ave background gauge fields which are $p$-forms
 which are associated with $p-1$-form symmetrized.
 So you have zero-form symmetries and a 1-form gauge field,
 it's something you can generalize.
 I'm not going to expand on what that means.

 \begin{question}
     How can I form the form of the path integral after gauging the fields?
 \end{question}
 I can't give a general answer to that.
 Often we classify different kinds of TQFTs to see what kinds of $\thata$-terms
 we can put here.
 There are ways to do that we'll get there.
 It's quite non-trivial.

 \section{Spin TQFTs and fermions}
 If a microscopic system has fermions like electrons,
 if you want to describe it,
 you need to specify whether a fermion going on any loop has periodic or
 anti-periodic boundary conditions.
 You need some extra conditions to specify this,
 and that's basically what a spin structure is.

 \begin{question}
     Why do we need this BC?
     Why fermions only?
 \end{question}
 If don't specify it,
 how do you know which one it is?
 For bosons,
 if my system has a $Z_2$ symmtery,
 through non-trivial loops,
 I can insert $Z_2$ flux,
 and with bosons it can change from anti to periodic.
 For bosons,
 specifiying BCs is just like having a gauge field with a $Z_2$ symmetry,
 could go around any loop and get any phase I want.
 That's covered by background gauge fields.

 Fermions,
 there's alwasy choice of anti-periodic or periodic.
 Fermions always have fermion parity symmetry.
 I can put fermion parity flux through loops.
 It's not quite the same as a $Z_2$ gauge field.
 THere's more intricate structure.

 Microsopic system with fermions implikes periodic or anti-periodic boundary
 conditions along any loops.

 A \emph{spin structure} $\eta$ is on a manifold is an assignment of
 periodic or anti-periodic boundary conditions to each loop.

 For example, on a ring $S^1$,
 there are two choices $+$ or $-$.
 On a torus,
 we have four choices $++$, $+-$, $-+$, $--$
 We can get a different spin structure
 by inserting fermion parity flux through non-contractible loops.

 this is an obvious statement in words.
 Mathematically, this means that spin structures form what's called a
 \emph{torsor} over the co-homology group $H_1(M,\mathbb{Z}_2)$ of the manifolds
 with coeffciients in $\mathbb{Z}_2$.
 Torsor means that once I'm given a structure,
 I can alwways get another one if you give me $\pi$ flux.

 So the number of choices is the number of elements of this group.

 Torsor means that this different spin structures are related to each other by
 elements of $H_1$.
 Temperature is not a torsor because there's an aboluste zeor.
 But energy is because there's no notion of zero energy.
 There's no identity of the gourp fora torsor,
 only actions on the group.

 \begin{question}
     Why does inserting fermion parity flux is not enough to determine the spin
     structure?
 \end{question}
 I don't have an intuitive reason for it.

 The point is just there's no natural notion of what zero fermion parity flux is
 for a generic manifold.
 You can only say that if you insert fermion parity flux you change the boundary
 condition.

 Not all manifolds admit spin structures,
 because you have to consistently assign boundary conditions.
 In fact there are restrictions on what kind manifolds even admit spin
 structures.

 Not all manifolds admit spin structures.
 There are lots of cool theorems proven about this in the lsat 50 years.
 The most famous is $\mathbb{CP}^2$ does not admit a spin structure.
 The obstructure to defining a spin structure
 is called the second Stiesel-Witney class $W_2$,
 an element of the second cohomology group $H^2(M, \mathbb{Z}_2)$.

 Every manifold always has a Stiefl-Whitney class.
 If it is trivial,
 then you can define a spin structure.

 There is always a $H^1$ and $H^2$.
 I you give me a spin structure,
 I can always get another by action of $H^1$,
 that's inserting fermioin parity flux thorugh nontrivla loops.
 When $W_2$ vanishes,
 there's a choice of spin structure,
 but the choice is not unique,
 so it forms a torsor.

 $\eta$ is really a 1-cochain.
 $d\eta = W_2$ then $\eta\to \eta + \alpha$,
 where $\alpha\in H^1$.

No spin structure means you can't define the ground state on the manifold,
and you can't define the path integral on that spacetime manifold.

The sphere has a $W_2 = 0\mod 2$.
$W_2$ is the mod-2 reduction of the Euler class.
The Euler class of a sphere is zero.
Euler characteristic is always even.

\begin{question}
For CP2 we cannot define fermions in this spacetime?
\end{question}

A spin TQFT needs to have $V(\Sigma^d, \eta)$
and $Z(M^{d+1},\eta)$.
We will study this with a $(1+1)$-dimensional Majorana chain.
This is a 1D phase of matter with fermions in it.
And the fermion description needs a spin TQFT.

\section{Framed TQFT}
As far as I know,
this is only relevant when we want to study $(2+1)$-dimensional chiral
topological phases.
If we want to study superconductors, integer quantum hall states,
fraftional quantum hall states,
we need Framed TQFT.

To define $Z(M^3)$,
we need not just an orientation,
but also need a \emph{framing}.
A framing of a $d$-manifold is a choice of $d$ vector fields in the tangent
space that are always linearly independent.
In chiral topological phases,
we can always continuously deform a framing.
In chiral topological phases,
the path integral is going to depend on $M^3$
together with the homotopy class of the frame,
meaning framing up to continuous deformations.
Mathematically,
this is called a trivialization of the tangent bundle.

If someone tells you the fractional quantum hall effect is described by TQFT,
you have to tell them no,
it's described by a framed TQFT.
It doesn't depend on the metric,
but depends on a frame.

This only comes up in 2+1 dimensional chiral topological phases.

There's an obstruction to finding a framing in general,
just like the obstruction to finding spin structures.
The obstruction is called the Euler characteristic.
If the Euler characteristic of the manifold is 0,
then a framing is possible,
otherwise you can't.
A sphere has Euler characteristic 2, which is not 0,
and the famous theorem is the hair ball theorem.
You cannot comb a hairy ball to not have singularities.
It turns out that for closed 3-manifolds,
the Euler characteristic is always 0.
We never have to worry about this obstruction in $3+1$-dimensions.

\begin{question}
    Does it break Lorentz invariance?
\end{question}
You still have to choose some kind of information.
We're already thinking of curved manifolds
which are not Lorentz invariant.
We're just asking if it's completely independent any metric.
I don't think it's Lorentz invariant,
just some extra structure.
If you spacetime has a metric,
you cna get a framing.

CS theory has no dependence on a metric,
but when you quantize it,
you need to introduce a metric.
It turns out the path integral has some weak dependence,
and that is just the frame.
That is Witten's 1989 paper.

\begin{question}
    What is trivialization of the tangent bundle?
\end{question}
It means pick some basis for the tangent space for every tangent space.
If you're really inetereted, yo ushould read more.
A local trivialisation is making a product,
so a global trivialisation is a base space times a fibre.
It's some choice of basis for the fibre,
how you decompose into a product.

\begin{question}
    How does this relate to Chern number?
\end{question}
No, it's the chiral central charge that gives dependence on the frame.
We're jumping ahead here!

There are a few other things about TQFTs I want to mention.
Then we'll finally get to the 1D Majorana chain,
the simplest 1D topological phase.
