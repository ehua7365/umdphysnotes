Please do your homework.
\section{U(1) Chern-Simons theory}
Last there were $|k|^g$ states on genus $g$ surface.
We also figured out we have these loop operators
\begin{align}
    W_\gamma
    &=
    e^{i\oint_\gamma a\cdot dl}
\end{align}
and if you're on a torus,
we have loops going in two directions,
which I just called 1 and 2.
And you find that
\begin{align}
    W_1 W_2 &=
    W_2 W_1 e^{-2\pi i/k}
\end{align}
What this means is that you can consider a basis of states where one of them is
diagonal
\begin{align}
    W_2 \ket{n}
    &=
    e^{2\pi in/k} \ket{n}
\end{align}
and the other one acts like a clock operator that shifts $n$ by 1.
\begin{align}
    W_1\ket{n}
    &=
    \ket{n + 1\mid k}
\end{align}
That there are $k$ ground states,
is related to the fact that around a given loop there are $k$ distinct
operators.

So that $k$ states on a torus is directly related to the fact that there are $k$
distinct Wilson loop operators along a given curve $\gamma$.
When I write a loop operator for a curve $\gamma$,
I can consider a charge $\ell$ particle going around a loop.
\begin{align}
    W_\gamma^{\ell}
    &=
    e^{i\ell \oint_\gamma a\cdot dl}
\end{align}
and furthermore this acts like the identity on the Hilbert space.
\begin{align}
    W_\gamma^k &= \mathbf{1}
\end{align}
For example, if I consider $W_1^k$,
it's going to make the phase go away.
Hence the distinct values of $\ell$ are $\ell=0,\ldots, k-1$.

From the condensed matter perspective,
there are $k$ topologically distinct types of particles.
And these $k$ distinct types of particles are what we call anyons in the system.

So far because we're working entirely in CS theory,
I haven't defined what topologically distinct,
but in CS theory,
it just means there are $k$ Wilson loop operators that act non-trivially on your
subspace.

\begin{question}
    This is difeernt from $k$ speciies of particles.
    This seems strange?
\end{question}
The way to thik about this is imagine some physical system described by CS
theory at long wave lengths.
There are $k$ distinct ground staes.
Thew ay to go from one ground state to anohter,
you create a quasiparticle, and a quasihole,
take one around the torus,
and annihliiate them.
If it is trivial,
such an operation has no effect no the ground state.
The corresponding Wilson loop should act in the ground state subspace.


\begin{question}
    There are $k$ ground states?
\end{question}
In the CS hteory,
there are only $k$ states.
In CS-Maxwell,
we have $k$ ground states and a gap to other states.
There's a ground state sector described by CS and then excitations above a gap.
there are infinitely many excitations that take you from the ground state to
anexcited state,
but they can be classified by topological operators
to make equivalence classes.
The topological operators are Wilson loops.

\begin{question}
Should we say there are $k$ different particeles?
\end{question}
I would say htere are $k$ ddistinct particles.
each are $k$ inequivalent types of paritcles,
But there is a single generator,
and elementary particle.

\begin{question}
    Discrete translation for mangetic field.
    Is htere an analogy for magnetic flux?
\end{question}
If you start with CS-Maxwell theory,
your Lagrangian would be
\begin{align}
    \mathcal{L}
    &=
    \frac{k}{4\pi}a\, da
    -
    \frac{1}{4g^2}f^2
\end{align}
If you look at the ground state sector,
you can define
\begin{align}
    X_i \propto \int d^2x \, a_i
\end{align}
but then you fnid the Lagrangian winds up looking like
\begin{align}
    L &=
    X_1 \dot{X}_2
    -
    \frac{m^2}{2}\left( 
    \dot{x}_1^2
    + \dot{x}_2^2
    \right)
\end{align}
which is literally a particle in a magnetic field.
You can think of the particle as a zero mode of the gauge field.
that makes the more specific connection.

When you go to fractional quantum Hall,
the translatioanl symmetry in FQHE has the same effect.
Translation symmetry acts the same way as Wilson loops act.
There's a deep connection there.


\begin{question}
    What was the motivaiton for WL?
\end{question}
When you have a gauge theory,
the natural thing to talk a bout his lop operaotrs.
One way to think is
suppose you hvae some action $S(a)$ for your gauge theoyr,
you could always add a term for some exeternal source current
\begin{align}
    S(a) + a_\mu j^\mu
\end{align}
So if I take my path integral,
I could always add a term here for some source current $j^\mu$,
\begin{align}
    \int \mathcal{D}a\,
    e^{i S(a)}
    \left( 
    e^{a_\mu j^\mu}
    \right)
\end{align}
some external source yo put in yoru system.
If this $j$ describes a paritcle travelling around,
this is literally the Wilson loop.
You could write the charge
\begin{align}
    j^0 &= \delta(\vec{x} - \vec{x}_*(t))
\end{align}
and then the current would be
\begin{align}
    j^i &= \frac{dx_*'}{dt} \delta^{(3)}(\vec{x} - \vec{x}_*(t))
\end{align}
So it describes a particle travelling in a loop.
And then if you write the integral,
that just becomes a loop integral on a curvve.

So the loop operator coresponds ot inserting a source in the current.

\begin{question}
    With this source current,
    that allows you todistinguish these staes?
\end{question}
Every time you hvae agauge thoery,
Wilson loops are a natural thing.
What we find ni CS theory,
we have a degenerate set of states.
If we're on a torus,
taking a paritcle around a loop 
either shifts the state or measures which state we're in.

Suppose you fix your state to be an eigenvalue,
you can insert your loop in hte other direction adn shuffle your states.
There's a deep connection betwen the number of distinct typeso f apritcles you
have and the number of ground states on your torus.
The number of ground states on your torus is equal to the number of
topologically distinct types of particles.

I havne't given you a general proof,
but we've seen it play out.

\begin{question}
    So each of these loops,
    were you able to?
\end{question}
If I loop athte Wilson loops,
I have $W_\gamma^{\ell} = e^{i\ell \oint_\gamma a\cdot dl}$
I have charge $\ell$.
And if I put $\ell=k$,
I get the trivial operation.

Now the fact that those Wlison loops have this non-trivial algebra,
you can infer that hte associated particls that go around these trajectories
have fracitonal statsitcs.

Imagine that my gournd stae is a on a torus.

SUppose I apply a Wilson loop operator for this guy,
then another one.

Suppose I do it in an oppsoite order,
and we know the phase idfferenece is $e^{2\pi i/k}$.
So if one particle goes aroudn another loop,
I get a phse $e^{2\pi i/k}$ relative to unlinked.

We can see this more precisely in the following way by looking at the equation
of motion.
From the equation of motion,
we have
\begin{align}
    j^0 &=
    \frac{\delta S}{\delta a_0}
    \\
    &=
    \frac{k}{2\pi}
    \underbrace{\left( \partial_1 a_2 - \partial_2 a_1 \right)}_{b}
\end{align}
which is just hte magnetic field inthe parentheses
\begin{align}
    b &= \frac{2\pi}{k}j^0
\end{align}
This tells us a charge 1 particle carries flux $2\pi/k$.
Because magnetic field is bound ot particle density iwth coefficeitn $2\pi/k$.
If you introduce a paritcle of charge $1$,
the CS term is that it dresses the particle with flux.
A particle with charge 1 is dressed wiht flux $2\pi /k$.
The CS theory binds flux to the charge.
That's the ssense of CS theory.
Now e see fracitonal statitcs.

SUpoose I have another paritcle of chagge 1 and also $2\pi/k$ flus.
I have charge going around flux,
so I have Aaronov-Bohm phase.
The phase for braiding this round another is going to be
\begin{align}
    e^{i2\pi/k}
\end{align}
Naively you should expect $e^{2\cdot 2\pi i/k}$,
but it turns out it's half of what you naively expect if you go through the
subtle calculations of CS theory.
Let me sketch how to do that calculation.
Let's go through the CS calculation of the fractional statistics.

Consider the manifold $M^3 = S^3$.
Consider the expectation of product of Wilson loop operators
\begin{align}
    \left\langle
    W_{\gamma_1}^{\ell_1}
    W_{\gamma_2}^{\ell_2}
    \cdots
    W_{\gamma_n}^{\ell_n}
    \right\rangle
    &=
    \int \mathcal{D} a\,
    e^{i S_{CS}(a)}
    e^{i
    \sum_\alpha \ell_\alpha \oint_{\gamma_\alpha} a\cdot dl
    }
\end{align}
Suppose I have some complicated knot that is $\gamma_1$,
and $\gamma_2$ is knotted and linked with it in some complicatd way.
Just as I mentioned Wilson loops corresponding to source current,
whatwe have is that the 
$ e^{i \sum_\alpha \ell_\alpha \oint_{\gamma_\alpha} a\cdot dl }$
is like adding an external source current.

So we can write our Lagrangian as
\begin{align}
    \mathcal{L} &=
    \frac{k}{4\pi} a_\mu \partial_\nu a_\lambda \epsilon^{\mu\nu\lambda}
    +
    a_\mu j^\mu
\end{align}
where the current is
\begin{align}
    j^\mu &=
    \sum_{\alpha} \ell_\alpha j_{\alpha}^{\mu}
\end{align}
and then the charge density part is just a particle
\begin{align}
    j_{\alpha}^{\mu}
    &=
    \delta^{(3)}\left( 
    \vec{x} - \vec{x}_{\alpha}(t)
    \right)
\end{align}
and the current
\begin{align}
    j_\alpha^i
    &=
    \frac{dx_\alpha^i}{dt}
    \delta^{(3)}\left( 
    \vec{x} - \vec{x}_{\alpha}(t)
    \right)
\end{align}

In order to understand how braiding gives fractional phases,
we need to calculatine this espcation vuae.
Therea re two ways to think about how you'd calcaulte the epectaiotn vlaue.
One is that this is a Gaussian integral,
the actoin is quadratic in hte fh field.
You could literally integrate out.

A slighly different way is to solve hte classical equation of motino
and plug that back in.
It works because this is a Gaussian integral.

To calculate the expectaiton vlaue,
solve hte quation of motion
\begin{align}
    \frac{\partial S}{\delta a_\mu}
\end{align}
which implies
\begin{align}
    \frac{k}{2\pi} e^{\mu\nu\lambda} \partial_\nu a_\lambda = -j^\mu
\end{align}
Imagine we solve for the classical solution $a^{cl}$ and plug that back into the
action.
Then substitutling it into the action,
\begin{align}
    S\left( a^{cl} \right)
    &=
    \int \left( 
    \frac{k}{4\pi} a^{cl} da^{cl} 
    + a^{cl}\cdot j
    \right)\\
    &=
    \int\left( 
    -\frac{1}{2} a^{cl} j
    + a^{cl}\cdot j
    \right)\\
    &=
    \int \frac{1}{2} a^{cl}\cdot j
\end{align}
where I'm using a shorthand to avoid writing the $\epsilon$.
And so then
\begin{align}
    \left\langle
    \prod_{\alpha} W_{\gamma_\alpha}^{\ell_\alpha}
    \right\rangle
    &=
    \mathcal{N}
    e^{-i \frac{k}{4\pi} \int a^{cl} \, da^{cl}}\\
    &=
    \mathcal{N}
    e^{i\frac{1}{2}\int a^{cl}\cdot j}
\end{align}

\begin{question}
    Does the Wilson loops not commuting matter in the ordering?
\end{question}
We can just define the expecation value using this.
The fact we're on $S^3$ also simplifies the situation,
as the only thing to talk about is hte linkingof the loops.


\begin{question}
    There should be some paht ordering in order to define the linking?
\end{question}
Usually a Wilson loop is defined like this,
but it only matters if it's a non-abelian gague field.
\begin{align}
    W &= \mathcal{P} e^{i\oint a\cdot dl}
\end{align}
Path ordering only shows up for non-Abelian things.
Here we're working on $S^3$,
so there are no non-contractible cycles where things don't commute.
T could have just written
\begin{align}
    \left\langle
    W_{\gamma_1}^{\ell_1}
    W_{\gamma_2}^{\ell_2}
    \cdots
    W_{\gamma_n}^{\ell_n}
    \right\rangle
    &=
    \int \mathcal{D} a\,
    e^{i S_{CS}(a)}
    \prod_{\alpha}
    e^{i
    \ell_\alpha \oint_{\gamma_\alpha} a\cdot dl
    }
\end{align}
and these are just phases.

\begin{question}
    How to do you classify a systme of knotted loops.
    Does that mean I just consider the knots and evaluate thep ahte integral?
\end{question}
That was Witten's point.
Even before Witten,
people know you get linking numbers of loops when you calculate these ntegarls.
Witten did this for non-Abelian CS theory,
and show the corresponding expectation value
evaluates the Jones polynomial of the knot,
which si asophisticated knot invariatn mathematicains had invented earlier.
there was no intrinsic 3D version.
He won a Fields medal.

\begin{question}
    If w'ere on a sphere,
    toologically there's no difference betwen a path that takes you around the
    north pole and a path that doesn't take you aroud the north pole.
\end{question}
You're getting confusing it with $S^2$.
We're in $S^3$.
You're right htere is a configuation in $S^2$,
but this configuration cannot happen
because a chage 1 particle has flux $2\pi /k$.
So htere has to be $k$ of them.

\begin{question}
    Does $k$ has to be even?
\end{question}
No I don't think $k$ being even is needed.

\begin{question}
    Is there a physical reason tht he charge around the fluxes is not twice?
\end{question}
No,
I don't have a good reason for that excpt that's what CS theory gives.
If you have some lattice model,
like $\mathbb{Z}^n$ toric code state,
you would get stuff consitstent with CS theory.
Our intuitition comes for Maxwell,
and this is CS theory,
whici s difernet.

\begin{question}
    For 2D,
    we sometimes impose $\pI$ flux on particles,
    which makes paritcles go from boson to fermions,
    but in this case hte full braidig braiding is trivial.
    If I take $k=2$,
    the full braiding is $e^{i\pi}$.
\end{question}
But that describes semions,
not fermions.

I might put this in your homework and guide you through it,
but the result is that you get hte Gauss linking number,
meaning htat this expectaiton vlaue of curves is equal
up to to smoe overall normalization
\begin{align}
    \left\langle
    \prod_{\alpha} W_{\gamma_\alpha}^{\ell_\alpha}
    \right\rangle
    &=
    \mathcal{N}
    e^{\frac{2\pi i}{2k}
    \sum_{\alpha,\beta}
    \ell_\alpha \ell_\beta
    L\left( \gamma_\alpha, \gamma_\beta \right)
    }
\end{align}
where the Gauss linking number is defined by
\begin{align}
    L\left( \gamma_\alpha, \gamma_{\beta} \right)
    &:=
    \frac{1}{4\pi}
    \int_{\gamma_\alpha} dx^{\mu}
    \int_{\gamma_\beta} dy^{\nu}
    \epsilon^{\mu\nu\lambda}
    \frac{x^\lambda - y^\lambda}{|x - y|^3}
\end{align}
and this means unlinking loops gives you a $e^{2\pi i/k}$ phase,
which si what we mean by fractional statistics.

But there's a problem,
when $\alpha=\beta$,
this is not well-defined,
because $x=y$ and it blows up,
so it doesn't make sens for the diagonal piece,
the self-linking.
CS theory gave us something we can't make sense of.
When ever you find infinities in QFT,
you try to get a well-defined ansewr by regularizing hte theory in some way.

In this case,
there is self-linking $L(\gamma_\alpha, \gamma_\beta)$ is ill-defined.
So to regularize,
introduce framing of the loop.
Framing means you consider it as a ribbon instead of just a string.
You have $\gamma$ and then you have $\gamma'$ that is infintetsminally pushed
off from the original $\gamma$.
You have a world-line as a ribbon,
then the self-linking is the linking between $\gamma$ and $\gamma'$.
In general,
it's completely arbitrary how I define this framing.
I could have chosen a different framing.

I could have made the ribbon twist around,
and I would get a differnt number.
So the ansewr is ambigiuos until I pick a framing,
and we know how the resutl changes when we change the framing.
So it is well-defined how it changes under changes of framing.
The point is that the flat ribbon is related ot the one-twist ribbon
by a fator of $e^{\frac{2\pi i}{2k}}$.

It's only well-defined if we pick a framing,
which is arbitrary,
but we can clacualte how te expecation vlaues chnage under framing changes.

\begin{question}
    Is it depednent on the overall framing on the manifold?
\end{question}
No, it's independent of hte overall framing.

Why does it have to be a ribbon?
If I had a particle,
it doesn't makesense to talk about it spinning aorund ont itself..
But if it's a ribbon,
you have an arrow spniing around.
You have charge and flux
in CS thery,
so your particle has a little arrow,
and you can then talk about how the arrow twists around by $2\pi$.
We talked about topological spin.

We can write
\begin{align}
    e^{\frac{2\pi i}{2k}}
    = e^{2\pi i h_1}
\end{align}
so we can define
$h_1 = \frac{1}{2k}$ as topological spin,
or more generally
\begin{align}
    h_\ell
    =
    \frac{\ell^2}{2k}
\end{align}

The idea is to think of a ribbon instead of a loop.
If you had a ribbon,
but the ribbon is in a loop,
then you can't actually smoothly deforma framing.
You owuld have to cut it,
twist it,
and glue it back.
It's nota global property of the loop.

\begin{question}
    How do we change the linking number as we change the framing?
\end{question}
The point is the framing was wa choice.
Imagine two lops with different fraings.
Then what is the hange in the path ingerals if I consider the two lops in
didfetn frames.
It's just $e^{\frac{2\pi i}{2k}}$ phase difference.

\begin{question}
    The difference comes even looknig at self-linking.
\end{question}
yes,
otherwise the off-diagonal terms are linking diferent loops,
and the selflonking wouldn't matter at all.

\begin{question}
    This almost seems like picking up an arbitrayr numbe fo reach diagonal term,
    so we just pick a finite linking nubmer?
\end{question}
Yes,
but the point is we have rules for how that number changes under hcange of
framing.
Wihtout hte rule,
this would be meaningless.
that this does havea rule means we can make sense.

\begin{question}
    What did you mean by topological spin?
\end{question}
I just meant whatehr comes in hte exponent.
The change ni phase under change in framing si topological spin.
It's $\ell^2$ because there is the $\ell_\alpha \ell_\beta$.
I implicitly assumed it was a charge 1 wilson loop,
but if it was carge $\ell$ it's $\ell^2$.

\begin{question}
    Whatis the physcial relvance of framing?
    Why not just set it all to zero and not care?
\end{question}
Suppose I can just ermove al hte idagonal terms.
The problem is that htins wouldn't relaly make sense.
The point of this thing is that these phases are important.
They contain important physics.
One way of undersatndin the physics is that these spins show up in a log of
idffertn eways.
One way they show up for example,
is suppose I hae torus and I take awilson lop of a particle thoruhg here.
I can do certain mapping class group operations,
such as Dane twists,
which you can think of cutting hte manifold alnog a cycle,
twisting it by $2\pi$,
and clugin it back together.
Thati s the same a changing the framing by 1.
thew ay the ground state changes under mapping clas group actions
is these charges.
The naive thing of forgettigng about them will fail to reproduce a lot of
physics in systems that do show up.
The point is there's something smarter you can do than just setting htem to
zero.
The point is ther is a way to regularie them.

\begin{question}
    What is hte physical meaning of Wilson loops in physical space?
\end{question}
You mean these $h_\ell$'s?
These $h$s were from changing the framing of Wilson loops.
they were not well defined,
but we need to choose a framing,
and there are many,
but they are related to geter by the phase.

\begin{question}
    This framing and expectation values of Wilson loops?
\end{question}
Suppose I have a state where I have some particles at various places.
Then I have that evole to another state.
then they go around in some braid.
The fact htese particle have factoinal statistics means
that if I have my initial state and pply a stransmtation $U$,
the point is I get some phases,
which gives me my final state.
\begin{align}
    U\ket{\psi_{in}}
    &=
    e^{i \cdots}
    \ket{\psi_{in}}
    =
    \ket{\psi_{final}}
\end{align}
These phaes are essentially given by these kinds of expecatation values.c
In this case,
consider $M^3=\mathbb{R}^2\times I$.
That causes all these particles to braid around each other.
If I take the inner product,
that essentially becomes an execation vlaueof Wilson loops.
\begin{align}
    \bra{\psi_{in}}U\ket{\psi_{in}}
\end{align}
If I thiknk of time as periodic,
this guy would be the expeation of these loops knotted in some way.
Physicaly it corespnods to the evoution of quazparticles moving aorund in space
and time and what phases you get.

\begin{question}
    If you're chosing some framing to calcualte expecation value,
    where is the comparison between these framings that inttoduces these phases.
\end{question}
When you write down this $W$,
I need to implicitly specifiy the $\gamma_\alpha'$ the shifted version.
For every $\gamma$ you need to tell me what he shiftd version is like,
and then we know the value of the expectaion value.
We would also know how differnt shiftings would relate to the orignial $\gamma$.

\begin{question}
    What's the situation you would comapre thse to?
\end{question}
You might be interested ina situation where you're intested in some evoultion
wherethe patricles evolv like this,
but you want o compare to a situatio where the paritcles eolve like this
(pictures with braidings).
In a sense you're asking what hte application is.
I feel like we'll see it when we get there.
One applicatoin isin that Dane twist picture.
If I have a paicle sittin on the plane,
I can imagine that the particle twists as it evovles in time,
relative to not twisting in time,
and you want to know hwath e phase idffernece is.

\begin{question}
    Isn't htat depending in hte framing you choose,
    how do you standadaris it?
\end{question}
In whatI was tlaking about,
imagine some evolution of your system that corespond to this (not doing
anything).
Then theren's another eovlution where you twist the space around.
Ify ou compare it to that,
if your system was evolging,
you would get smoe transition amplitud e from intial tofinal.
Insteado f donig that,youhave your system evolving,
rotate space by $2\pi$. and then come back to where you started.
Comapring htse two situatinos,
I get exaclty the same thing up to some phase,
the topological spinn.

In QFt we have the spin-statistics theorem.
It turns out
if I have some paticle that spins aorund on itself,
relative to not hvaing spun around itself,
we get this topological spin $e^{2\pi ih}$.
Another way to think about htis picture is what happend?
A particle was going here,
particle and antiparitcle got eated ot of vacuum,
then they sexhcnaged.
So three is  deep relation with exchange statsitics.
Thsi $e^{2\pi i h}$ is also the exchanges tatistcs
becasue of the spin statistics theorem.
This topological spin just tells us what hte exchange statistics is.
If I have two fermiosn,
spinnig it aorund should give $-1$,
which is the same as exchanging them.

Here I have a phase of $e^{i2\pi/k}$
but if I exchange them I also get $e^{i2\pi /k}$.

\section{Global U(1) symmetry}
This CS thoery has global U(1) symetry,
which si not to be confused with the gauge symetry that defines the teory.
We have a U(1) gauge fidl $a$ which is dynamical.
We can introduce particles that are charged under $a$,
which describes a conserved current $j$.
Then we can write down
\begin{align}
    \mathcal{L}
    &=
    \mathcal{L}_{CS}
    + a_\mu j^\mu
\end{align}
For example,
you could couple it to a scalar field,
\begin{align}
    \mathcal{L}
    &=
    \frac{k}{4\pi} a\, da
    +
    |\left( \partial + ia \right)\phi|^2
\end{align}
and this term would contain a $j\cdot a$ term,
where $j$ is the curretn of thse $\phi$ particles.

But there's another conserved cucrent beyond any kind of particle you cna couple
with $a$.
You have a conserved current
\begin{align}
    J^\mu &=
    \frac{1}{2\pi}
    \epsilon^{\mu\nu\lambda}
    \partial_\nu a_\lambda
\end{align}
with $\partial_\mu J^\mu = 0$.
We can write out the components
\begin{align}
    J^0 &= \frac{1}{2\pi}
\end{align}
the magnetic field and the eltric cfiel
\begin{align}
    J^1 &\propto \frac{1}{2\pi}e^2\\
    J^2 &\propto \frac{1}{2\pi} e^1
\end{align}
Flux of $a$ describes  conserved current.
$J^\mu$ is a conserved cucfrent for flux $a$.

This isa ddep thing for $(2+1)D$ in genreal.
Any time you have some theory iwhta glboal U(1) symmetry,,,
you can write down the symmetry for hatt ocnsierved *(1)
and yu can ``dualize'' it and get a conserved U(1) current,
and that is something deep I can discuss more if we have time.

So this conserved current for flux,
you can think of as dual to current of charges of $a$.

\begin{question}
    Is this statment true for arbitrary 2D manifolds or just $\mathbb{R}^2$.
\end{question}
I think it should be true in teeral.

We can couple a conserved U(1) current to a background U(1) gauge field.
You can couple the global U(1) to a background U(1) gauge field $A$.
So you can write a Lagrangian
\begin{align}
    \mathcal{L} &=
    \frac{k}{4\pi}\epsilon^{\mu\nu\lambda} a_\mu \partial_\nu a_\lambda
    +
    \underbrace{
    \frac{q}{2\pi}\epsilon^{\mu\nu\lambda} A_\mu \partial_\nu a_\lambda
    }_{
    a\cdot A_\mu J^\mu
    }
\end{align}
and here $a$ would be a dynmical U(1)
but $A$ is a fixed static $U(1)$ field.
And $q$ is a new parameter that tells us how to couple this with the global U(1)
symmetry.


Once consequence of this is that when we go back to the non-trivla particls of
the original theory,
those trivial particels which were associated with $2\pi/k$ flux of $a$,
but now we see those particles also carry a fractional charge under the
background $A$.
we can write down the current.
You can stare at it and find
\begin{align}
    J^0 &=
    \frac{\delta \mathcal{L}}{\delta A_0}
    =
    \frac{q}{2\pi}b
\end{align}
which means that $2\pi/k$ flux of $a$ has charge 
\begin{align}
    Q_{2\pi/k} = q/k
\end{align}
So putting it togethe.
If I have a particle that has cahrge $\ell$ under $a$,
then htere will be a flux $2\pi \ell / k$
of $a$,
which gives a charge of
\begin{align}
    Q &=
    \frac{q\ell }{k}
\end{align}
so these have fractional statistics but also fractional charge.
