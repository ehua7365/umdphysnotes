\section{Group Cohomology Model for SPT}
So we've been discussing this Dijkgraaf-Witten theory.
Because there were questions about this last time,
let me mention the big picture.

Before discussing topological phases,
all models were free fermion models.
We could exactly solve them,
understand everything about them,
the ground state wave function,
the topological invariant and so on.

Now with strongly interacting systems,
they are not free.
For bosons,
you need strongly interaction,
because if not they will condensed into a superfluid.

These group cohomology models are kind of like free fermion system,s
in that they are exactly solvable classes of models.
You could exactly write down a path integral you could exactly compute,
and exact ground sate wave functions.
They are idealized exactly solvable models analogous to the free fermion case.

Of course these are not free,
they are strongly interacting.

Once we work with these exactly solvable modes,
these TQFTs,
this opens the entire door to a huge class of exactly solvable TQFTs with idea
Hamiltonians,
and so on.
This is the tip of the iceberg.

We start off discussing the path integrals,
then discuss the ground state wave function.

We're discussing the 1+1D case.
What we did was triangulate spacetime.

We add a branching structure,
which is a local ordering of vertices.

Then we put group elements on vertices.

And then we define the path integral,
which depends on a background gauge field $A$,
which is essentially the twist you can put on these non-contractible paths.
\begin{align}
    Z(M^2, A) &=
    \frac{1}{|G|^{N_V}}
    \sum_\left\{{g_i\right\}}
    \prod_{\Delta_2 \ni (i,j,k)}
    \nu_2^{S\left( \Delta_2 \right)} \left( g_i, g_j, g_k \right)
\end{align}
So then this $\nu_2$ is a homogeneous 2-cocycle.
So then if you write
\begin{align}
    \nu_2\left( g_i, g_j, g_k \right)
    &=
    \nu_2^{\sigma\left( g_i^{-1} \right)}
    \left( 1, g_{ij}, g_{ik} \right)
\end{align}
it doesn't change if you multiply everything by $g$,
where
$g_{ij} := g_{i}^{-1} g_j$.
And hen I can define the in-homogeneous 2-cochain which is
\begin{align}
    \omega_2(g_{ij}, g_{ik}) :=
    \nu_2\left( 1, g_{ij}, g_{ij} g_{jk} \right)
\end{align}
and the reason they needed to satisfy the 2-cocycle equation is that we needed
this topological action phase
\begin{align}
    e^{iS_{\mathrm{top}}\left( \left\{ g_i \right\} \right)}
    \prod_{\Delta_2 \ni (i,j,k)}
    \nu_2^{S\left( \Delta_2 \right)} \left( g_i, g_j, g_k \right)
\end{align}
to be retriangulation-invariant,
which led us to the 2-cycle equation.

I also said this 
$\frac{1}{|G|^{N_V}} \sum_\left\{{g_i\right\}}$
which is averaging over gauge transformations is actually unnecessarily.

We could have instead,
if you had some triangle,
just defined $g_{ij},g_{jk},g_{ik}$,
and have gauge field $A_{ij} = g_{ij}$.

What we picked here is a flat gauge field.
Flat means that
$g_{ij}, g_{jk}, g_{ki} = 1$.
And so the action is just the product over all triangulations $\Delta_2$
of the 2-cocycles.
\begin{align}
    Z\left( M^2, A \right)
    &=
    \prod_{\Delta_2}
    \omega_2^{S\left( \Delta_2 \right)}
    \left( g_{ij}, g_{jk} \right)
\end{align}
The $S\left( \Delta_2 \right)$ is the orientation of the triangulation.
This path integral is independent of the choice of gauge.
Let's say we have a gauge transformation $h_i$ and we define
$g_{ij}:= g_i^{-1} g_j$.
Then the gauge transformation can be defined to act like
\begin{align}
    g_{ij} \to
    h_i g_i^{-1}g_j h_j
\end{align}
So if I think of taking
\begin{align}
    g_i \to g_i h_i
\end{align}
by right multiplication.
To see this path integral is independent of gauge transformations,
take a group element assigned to a single vertex,
then get the new group element,
then show the path integral is independent of this transformation.

One way to convince yourself that this would work is that you could draw your
initial system,
say here,
which has some choice of $g_i$,
then you could draw the exact copy of this.
Say you have some vertex $g_i$.
And then we change it go $g_i h_i$,
a gauge transformation.
You could get from this picture to this picture,
by doing a series of Pachner moves.
One way to think how it works,
is think of it as an evolution of a 2D system from some initial time  a late
time,
so we have a 3D system and now we triangulate the 3D system.
Draw some additional lines,
and there's a way of triangulating this system,
which is the cobordism
So we have a cobordism
from the initial slice to the final slice.

In Pachner moves,
you could think of it as stacking a simplex of one higher dimension.
If I had a single triangle,
and I do a Pachner move that adds an extra vertex in the middle,
then you could think of it as pasting a 3-simplex on top of this triangle,
so the original triangle is just the bottom of the 3-simplex tetrahedron.
And so the 1-3 Pachner move is pasting a 3-simplex.
You could think of these Pachner moves as starting with some initial
configuration,
and some final configuration,
I have a 3-manifold that interpolates between them,
I triangular the manifold,
and all I'm doing is the sequence of Pachner moves of 3-simplices that I
continuously paste onto here to get the final configuration.
This whole thing is invariant under triangulation and Pachner moves,
so we could start with some $g_i$ and do some triangulation of 3-manifolds,
and those are some series of Pachner moves,
and ultimately we end up with $g_i h_i$ on that vertex.

Ultimately,
this is an explanation that we can do Pacher moves to to implement the gauge
transformation $g_i \to g_i h_i$.
This then means that this whole amplitude $Z\left( M^2, A\right)$
is gauge-invariant.

And because it's gauge invariant,
when we do the sum,
which is just picking different group elements in the vertices,
we don't have to do the sum,
because changing the group elements on the vertices is equivalent to doing a
gauge transformation.
Because of that,
the path integral we wrote there is just a phase,
an element of $U(1)$.

\begin{question}
    What is the gauge field?
\end{question}

If you have a cylinder,
you could have holomony when you go around a loop.
You could do that by cutting the cylinder surface open along its axis,
and then I glue it back with a twist.
So if this is $g_i,g_j,g_k,\ldots$ on the bottom edge,
I would just identify it with a twist $hg_i,hg_j,hg_k,\ldots$,
which effectively determines some gauge transformations.

There's a slightly different way.
You could define gauge field separately on the links.
That is exactly the same as doing what I said.

\begin{question}
    Where do we encode the fact that we are describing physical systems which
    are gapped and have unique ground sate.
\end{question}
There's no notion of gapped yet,
because there's no Hamilton defined yet.
After I define it,
I will show these do define a gapped system.

\begin{question}
    Can we say this describes only gapped systems?
\end{question}
Depends what you mean by describe.
I'm going to do it right now,
so in a minute,
suing this path integral
I can define a wave function on the boundary.
That wave function on the boundary is going to be the unique gapped ground state
of a certain ideal exactly solvable Hamiltonian.
You could ask,
can the aground sate also be the ground state of a gapless Hamiltonian.
I don't know if there is a proof if that can or cannot happen.

But hits thing gives a lot of information.
This can't really describe a gapless system in a comprehensive way,
although it could describe gapped systems.

Any questions about the exact setup.
It's a bit strange,
because in condensed,
we want these on the vertices,
but if you're a field theorist you don't care.
It's awkward by you will see why.

\begin{question}
    What will change in the formula with gauge fields?
\end{question}
Let me write it in this one over here.
You  think of $\omegag_2\left( g_{ij}, g_{jk} \right)$
as
$\omegag_2\left( A_{ij}, A_{jk} \right)$.
But maybe it's more clear,
if there is a background gauge field $A_{ij}$,
which is flat,
and the path integral is
\begin{align}
    Z\left( M^2, A \right)
    &=
    \prod_{\Delta_2}
    \omega_2^{S\left( \Delta_2 \right)}
    \left( A_{ij}, A_{jk} \right)
\end{align}
and then if I want,
I can sum over gauge transformations to get
\begin{align}
    Z\left( M^2, A \right)
    &=
    \frac{1}{|G|^{N_V}}
    \sum_\left\{ {g_i \right\}}
    \prod_{\Delta_2}
    \omega_2^{S\left( \Delta_2 \right)}
    \left( g_i^{-1} A_{ij} g_j, g_j^{-1} A_{jk} g_k \right)
\end{align}
So I fix my flat background gauge field with $A_{ij}$ on the links.
Then I write down the amplitude for each 2-simplex,
and because this amplitude is invariant,
if I do a gauge transformation,
that changes
$A_{ij} \to g_{i}^{-1} A_{ij} g_j$.

And the point is that this formula,
I ca do what I said about cutting open the cylinder and gluing together with a
twist.
Maybe that was a cleaner way of presenting it in terms of $A$,
but I didn't write it explicitly.
But this is probably the most best comprehensive definition.

If you take any two triangulations,
you could definitely construct this manifold,
which is this piece times $I$ basically.
And you can triangulate the 3-manifold.
It's clear that this is basically this with some 3-simplices attached to it.
You could just think of it just one by one.

\begin{question}
    Why did you introduce this picture?
\end{question}
I wanted to introduce this picture to show you could change $g_i$ on a vertex to
$g_i h_i$ by Pachner move.

\begin{question}
    Did you really show that?
\end{question}
i just sketched it.
I showed this $Z$ is invariant under Pachner moves,
and then I showed you could get from one to another.
I have a cobordism that I triangulate,
and I have 3-simplices I glue on one-by-one,
each of which is equivalent to a Pachner move.

\begin{question}
    We have lines that are intersecting the triangle.
\end{question}
that might be a function of how you look at it.
It might be useful to take a simple example.
Suppose you have a triangle,
with a vertex in the centre and rays to vertices.
Then I want to take this $g_i$ on the centre vertex I want to transform to
$g_i'$.
I just do a Pachner move to delete the vertex.
And then I do another Pachner move to reintroduce the vertex with $g_i'$ on the
vertex.
Just play  a bunch of examples it works.
But this is the general proof.

So far we defined the path integral by triangulating and labelling the group
elements coupled to this background gauge field,
and that defines a path integral.
Now we discuss the wave function.
Remember from TQFT,
If I evaluate a path integral on a manifold with a boundary,
then I get a state on the boundary.

Imagine a 1D system on a circle like this.
I can write down a wave function for the group elements that live on the edge of
my manifold,
and I write down a path integral where I sum over all the group elements in the
bulk
\begin{align}
    \Psi\left( \left\{ g_i^{\textrm{edge}} \right\} \right)
    &=
    \frac{1}{|G|^{N_{v,bulk}}}
    \sum_\left\{ {g_i^{bulk} \right\}}
    \prod_{\Delta^2\ni (i, j, k)}
    \nu_2^{S\left( \Delta^2 \right)}
    \left( g_{i}, g_{j}, g_{k} \right)
\end{align}
So now I have a wave function defined on a circle defined on chain.
And by studying the wave function,
I can learn some interesting things.

Firstly,
we have a system that has a global symmetry.
Here,
the global symmetry would act just by left multiplication.
If I multiply every group element on the edge by $g$,
you can see what that does is \ldots.
It's easy to see this in the special case.
Suppose you have vertices on the boundary, say 4 vertices.
And the bulk is retriangulation-invariant,
I can do enough retriangulations such that there is only t only a single vertex
in the bulk, say it's called $g_*$,
and on the edge we have $g_1, g_2, g_3, g_4$.
Suppose $g_*$ has the largest ordering number,
so they the arrow lines go like this (toward the centre vertex and around the
edges clockwise)
But because $\nu_2$ has this property where it's symmetric itself,
I can multiply $g^{-1}$ everywhere,
and I could just raise it to $\sigma(g)$,
which is the complex conjugate symbol if it's anti-unitary.
\begin{align}
    \Psi\left( \left\{ g g_i^{\text{edge}} \right\} \right)
    &=
    \frac{1}{|G|}
    \sum_{g_*}
    \prod_{\Delta^2}
    \left[ \nu_2^{S\left( \Delta_2 \right)}
    \left( g_i, g_j, g^{-1} g_* \right)\right]^{\sigma(g)}\\
    &=
    \left[ \Psi\left( \left\{ g_i^{\textrm{edge}} \right\} \right)
    \right]^{\sigma(g)}
\end{align}
where I relabelled $g^{-1} g_i^* \to g_*$.

The local Hilbert space on the edge is
\begin{align}
    \mathcal{H}_i &=
    \span\left( \left\{ \ket{g_i} \right\} \right)
\end{align}
The degrees of freedom compose the Hilbert space on the boundary.

The physical degrees of freedom are the group elements on the vertices,
the $A$'s are just a background gauge field which tells you whether you have
twist on the boundaries.

Even though this is technically a sum over $g_*$,
the actual $g_*$ on the bulk doesn't matter,
because we can just retriangulate to completely get rid of $g_*$.
So the amplitude shouldn't even depend on $g_*$ at all.

\begin{question}
    What is $\sigma(g)$?
\end{question}
To be clear,
\begin{align}
    \sigma(g) &=
    \begin{cases}
        * & \text{if $g$ is anti-unitary symmetry}\\
        1 & \text{if $g$ is unitary symmetry}\\
    \end{cases}
\end{align}


\begin{question}
    Why is it 2+1D?
\end{question}
Usually spacetime is $M^{d+1} = \Sigma^d\times \mathbb{R}$,
but in TQFT,
we consider arbitrary spacetime manifolds.
For example,
for states on the edge of a circle,
you could thick of the radial direction as ``time''.
I still like to say 2+1,
because it's not clear if it's 2+1 or 3D if I just say 3D.

Anyway,
we can actually get rid of that bulk vertices by Pachner moves.
I can just imagine that the triangulation looks something like this.
(triangulation without vertices in the bulk)
So each term in the sum should be independent of the bulk.
And so we can actually just pick an arbitrary $g_*$.
And write the wave function without the sum.
\begin{align}
    \Psi\left( \left\{ g_i^{\textrm{edge}} \right\} \right)
    &=
    \prod_{\Delta_2}
    \left[ 
    \nu_2\left( g_i, g_{j}, g_* \right)
    \right]^{S\left( \Delta_2 \right)}
\end{align}
which means that the wave function is just a phase
\begin{align}
    \left|
    \Psi\left( \left\{ g^{\textrm{edge}} \right\} \right)
    \right|
    = 1
\end{align}
So there is no intrinsic topological order.
And so that means there is no intrinsic topological order,
so we can disentangle by constant depth circuit.
The state itself
\begin{align}
    \prod_{\Delta_2}
    \ket{\Psi_{1D}}
    &=
    \sum_\left\{ {g_i^{\textrm{edge}} \right\}}
    \left[ \nu_2 \left( g_i, g_j, g_* \right) \right]^{S\left( \Delta_2 \right)}
    \ket{
    g_i^{\textrm{edge}}
    }
\end{align}
And so we can define a new basis
\begin{align}
    \ket{g_i'}
    &=
    U\ket{g_i}\\
    &=
    \prod_{\Delta_2}
    \left[ \nu_2\left( g_i, g_j, g_* \right) \right]^{S(\Delta_2)}
    \ket{\left\{ g_i \right\}}
\end{align}
This unitary is just a local unitary circuit.
That means,
we can view this as a constant depth local unitary.


It's a local unitary because every term only depends on the $g_i$ and $g_j$ that
are close to each other with this $g_*$ in the bulk.
It's constant depth,
because I could just imagine that I do these links on the edge all at the same
time,
and then apply the links to the center vertex.

So with depth 2 I could implement the sequence of phases.
YOu could also redefine the basis as the original $\ket{g_i}$ times a unitary of
constant depth circuit.

It's a fairly trivial state in the new basis.
In the new basis,
\begin{align}
    \ket{\Psi_{1D}}
    &=
    \sum_{\left\{ g_i' \right\}}
    \ket{\left\{ g_i' \right\}}\\
    &=
    \bigotimes_i\left( 
    \sum_{g_i} \ket\left\{ g_i' \right\}
    \right)
\end{align}
which is just a tirvial product state.
hence no intrinsic topological order.

But this $U$ does not actually preserve the symmetry,
unless the cohomology class of $\nu_2$ is trivial,
that is the class $[\nu_2]$
is trivial in $H^2\left( G, U(1) \right)$.

If I do  symmetyr transofmraiton,
I'm changing the ones on the edge.
This itself does not equal.
\begin{align}
    \nu_2\left( gg_1, gg_j, g_* \right) \ne
    \nu_2\left( g_i, g_j, g_* \right)
\end{align}
because to get the symetry you also need to transafomr the $g_*$.
only the trivial cocycle would satisfy the above equality.

Let me just say that
two cocycles in the same $H^2(G, U(1))$ class
define wave functions that can be related by a constant depth symmetric circuit.
The idea is that if I take some cocylce $\nu_2' = \nu_2\, d\mu_1$,
and if I draw a picture with $g_i, g_j, g_k, g_l$ on four corners,
then the ones on the bulk will come iwth compelx conugation,
the only one left is the $\mu$ on the boundary,
so you get a factor of $\mu\left( g_i, g_j \right)$ on the boudnary because
there is no $g_*$ anymore.
Yougeta different wave fucntio nif you change this on the coboundary,
but it onlydiffers by the gound elmetns on the boundary,
so we won'th ave this issue iwht the $g_*$ that spoils the symmetry so you can
get a symmetric constant depth circuit.

The pathe integral defines a wvae funtion that is symmetric.
It has no intrinsci toploical order.
IT can be distentagled by a constant depth circuti,
butit can not be distentabled by a constant depth symmetryc circuit.
This cohomology class determines equivlaence classes of ground satte wwave
ufnctions that cannot be reached to each other by aconstantt depth symmetric
circuit.

\begin{question}
    Is it easy to see how the ege transforms projectively?
\end{question}
Not yet.
So far I've only defined the ground state wave funiton on a ring.
What I coulddo is look at the entanglement of the ave fucntion,
I can loko ath the reduced density matrix on the interval,
and how it decomposes into a product of terms.
You should be able to do that,
but I havne't gone thoruh the anlaysis myself.

\begin{question}
    What's the symmetic of the constant dpeth circuit?
\end{question}
The symmtery group is $G$.
I said that on each site,
we have $\ket{g_i}$ that takes this form,
and the symmetry is
\begin{align}
    \ket{g_i} \to \ket{ gg_i}
\end{align}
which is the symmetric action nthe state.
This constant depth circuit $U$ doesn't respect that symmetry,
because if it did,
we would need to hae this equlity.
\begin{align}
    \nu_2\left( gg_1, gg_j, g_* \right) =
    \nu_2\left( g_i, g_j, g_* \right)
\end{align}

Now that we have this wave fucntion,
we could alos write down an ideal Hamiltonian.
Three's a one-line way of writing down the ideal Hamiltonian.

\subsection{Ideal Hamiltonian}
I can start with the trivial Hamiltonian
\begin{align}
    H_{\text{trivial}} &=
    \sum_{i}
    H_{i,\text{trivial}}
\end{align}
where
\begin{align}
    H_{i,\text{trivial}}
    &=
    \ket{\phi_i}\bra{\phi_i}
\end{align}
where
\begin{align}
    \ket{\phi_i} &=
    \sum_{i} \ket{g_i}
\end{align}
It's like the paramagnet.
Each site is uniform to position with respect to all the $g$'s.

Then I can write the SPT Hamiltonian as
\begin{align}
    H_{SPT} &=
    U H_{\text{trivial}} U^\dagger
\end{align}
where $U$ is the constant depth circuit that disentangles 
$\ket{\Psi_{1D}}$.
So I start with the trivial one and I do this basis transformation.

A more useful way of thinking is the following.
The ground state wave function is a path integral defined on a disk,
where we are summing over group elements on the interior.
Let's focus on just some segment of the chain,
with vertices
$\ldots,g_{i-1},g_i,g_{i+1},\ldots$.
Now let's introduce a bulk vertex $g_i'$
that has links to each
$g_{i-1},g_i,g_{i+1}$.
So then imagine doing a bulk transformation.
The new wave function is now
\begin{align}
    \Psi\left( \ldots, g_{i-1}, g_i', g_{i+1},\ldots \right)
    &=
    \frac{1}{|G|}
    \sum_{g_i}
    \nu_2\left( g_{i-1}, g_i, g_i' \right)
    \nu_2^*\left( g_{i-1}, g_i', g_{i+1} \right)
    \Psi\left( \ldots, g_{i-1}, g_i, g_{i+1}, \ldots \right)
\end{align}
So that's how they are related under bulk transformation.
And now I can just write down the Hamiltonian
\begin{align}
    H &=
    \sum_i
    \sum_{g_i, g_i'}
    \frac{\nu_2\left( g_{i-1}, g_i, g_i' \right)}{
    \nu_2\left( g_i, g_i', g_{i+1}\right)
    }
    \ket{\ldots, g_i', \ldots}
    \bra{\ldots, g_i, \ldots}
\end{align}
So now we have a ground sate and the Hamiltonian.
Now e can study the system on an open chain with boundaries,
and we have a complete solvable model.

\begin{question}
    We considered the Hamiltonian hat considered one edge state and change it to
    a different group element,
    why don't we consider it simultaneously and onwards?
\end{question}
That's just doing it twice.

The beautiful thing about this group cohomology model,
is that everything we do generalizes to higher dimensional models.
I could have $D$-cochains and $D$-cocycles.
That's why it's a nice way of thinking about a system.
The MPS description on the other hand,
while deeply related to this group cohomology construction,
this is really what's going on behind the scenes.

\section{Higher dimensions}
Firstly,
let me define the higher dimension cohomology groups.
Consider some $n$-cochain
\begin{align}
    \omega_d \in C^n_\rho (G, M).
\end{align}
$M$ is going to be an Abelian group,
which used to be $U(1)$
with a $G$-action $\rho$.
It is a map that takes in a group element 
\begin{align}
    \rho: G \times M &\to M
\end{align}
so this $M$ is a $G$-module.
And this $\omega_$ has values
\begin{align}
    \omega_2\left( g_1,\ldots, g_n \right) \in M
\end{align}
Then define the coboundary map
\begin{align}
    d: C^n \to C^{n+1}
\end{align}
and the way we define this is 
\begin{align}
    d\omega(g_1, \ldots, g_{n+1})
    &=
    \rho_{g_1}\left( 
    \omega\left( g_2,\ldots, g_{n+1} \right)
    \right)
    \times
    \prod_{j=1}^{n}
    \left[\omega^{\left( -1 \right)^j}
    \left( g_1,\ldots, g_{j-1}, i, g_{j+1}, g_{j+2}, \ldots \right)
    \right]
    \times
    \left[
    \omega\left( g_1,\ldots, g_n \right)
    \right]^{\left( -1 \right)^{n+1}}
\end{align}
This is a complicated expression,
but it does have the property that
\begin{align}
    d^2\omega = 1
\end{align}
Even though this thing looks weird,
one way of thinking about where this complicated expression comes from is the
triangulation invariance we want to impose.
We want our path integral to be independent of triangulation.
This invariance under triangulation literally gives you this.
That's geometrically where this complicated expression actually comes from.

Now we have this sequence of cochains.
\begin{align}
    C_\rho^{n-1}
    \xrightarrow{d_{n-1}}
    C_\rho^n
    \xrightarrow{d_n}
    C_\rho^{n+1}
    \to \cdots
\end{align}
Cohomology groups are
\begin{align}
    \frac{\ker d_n}{\im d_{n-1}}
    =
    H_\rho^n\left( G, M \right)
    =
    \frac{Z_\rho^{n}(G, M)}{B_\rho^n(G,M)}
\end{align}
The $n$-cocycles are
\begin{align}
    Z_\rho^n(G, M) &=
    \left\{
    \omega \in C_\rho^n
    \mid
    d\omega = 1
    \right\}
    =
    \ker d_n
\end{align}
and the $n$-coboundaries are
\begin{align}
    B_\rho^n &=
    \left\{ 
    \omega \in C_\rho^n
    \mid
    \omega = d\mu\,
    \text{ for some }
    \mu\in C_\rho^{n-1}
    \right\}
    =
    \im d_{n-1}
\end{align}
Time-reversal actually complex conjugates things,
because it's anti-unitary.
So we need this $\rho$ for SPTs.
