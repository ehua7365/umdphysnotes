\section{Unitary Modular Tensor Categories}
This is sometimes called the algebraic theory of anyons.

Suppose you ave a UMTC $\mathcal{C}$.
We have a quantity called the chiral central charge $C_-$.
For a Chern insulator the central large is 1,

$\mathcal{C}$ determines $C_- \mod 8$.
Depending on people,
it could be mod 24,
but that's because they have a slightly different definition of $\mathcal{C}$.

It's conjectured, or believed to be the case,
basically everyone I know believes this,
that if you ignore symmetries:
\begin{conjecture}
$(2+1)d$ topological phases are classified by this pair $(\mathcal{C},C_-)$.
This is a condensed matter conjecture.
\end{conjecture}


There is also another conjecture.
$\mathcal{C}$ always comes from some quantum Chern-Simons theory.

For any UMTC,
there is some appropriate CS theory with some appropriate choice of gauge group
that defines it.

The important conjecture is the classification of $(2+1)D$ topological phases.

So far in this class we saw 3 examples.

\subsection{$U(1)_k$ CS theory}
When we were studying $U(1)_k$ CS theory,
we found there are $k$ distinct anyons.
You can label them by
$[0], [1], \ldots, [k - 1]$.
Remember these were described by some Wilson loop operators
\begin{align}
    W_{[l]}(C)
    &=
    e^{i l \oint_C a}
\end{align}
And we described how these anyons have exchange statistics
\begin{align}
    \theta_l
    &=
    e^{2\pi i l^2/k}
\end{align}
We talked about how in CS theory if you want to define the exchange statistics,
you need to think of the worldline as a ribbon,
which is also sometimes called giving a framing to a worldline.
If you twist the framing by $2\pi$,
then that's related to the untwisted version by a phase $\theta_l$.

Another way of thinking about this is that if you have two particles both of
type $l$,
then you get a phase of $\theta_l$.
This is the exchange statistics and this is thinking about it as a topological
spin,
like spinning around by $2\pi$ gives a phase of $\theta_l$.

Then we also had metal statistics $M_{l,l'}$.
This means that if you have a $l$ and $l'$ particle,
and you braided around,
you would get a phase
\begin{align}
    M_{l,l'}
    &=
    e^{2\pi i l l'/k}
\end{align}
Furthermore,
you could also fuse these particles together
\begin{align}
    [l] \times [l'] = [l + l']
\end{align}
That you could see from the Wilson loop perspective
\begin{align}
    W_{l} W_{l'} = W_{l + l'}
\end{align}
So that's a summary of $U(1)_k$ CS theory.


\subsection{$\ZZ_2$ toric code}
There are four kinds of particles
$1,e,m\epsilon=\e\times m$.
And they have properties
\begin{align}
    e\times e = 1 = m\times m = \epsilon\times \epsilon
\end{align}
And the exchange statistics are
\begin{align}
    \hteta_e = \theta_m = 1
\end{align}
And mutual statistics

\subsection{Non-abelian fusion rules}
MZMs in vortex comes in $p + ip$ superconductors.
$\sigma$ is a $\pi$ flux vertex.
If you had a pair of $\sigma$ MZMs,
they could have even or odd fermion parity.

When I have two,
they can fuse to either even or odd fermion parity.
\begin{align}
    \sigma\times \sigma = 1 + \psi
\end{align}
And fusing two $\sigma$ is trivial.
\begin{align}
    \psi\times\psi &= 1
\end{align}
The interesting thing is that when I fuse two $\sigma$ together,
I don't get a unique outcome,
but a distinct outcome.

\begin{question}
    Should it be $\frac{1+\psi}{\sqrt{2}}$ instead?
\end{question}
No. When I write $1+\psi$,
I means either two $\sigma$ fuse to odd or even fermion parity,
so the notion of a square root doesn't make sense.

The idea of UMTC is to generalise these phenomenon that provides a mathematical
framework to describe what's going on in the most general way.

UMTC gives a mathematical framework to generalize these examples.

Once we define a UMTC,
what we can do from that data I specify
is that it can be used to define a TQFT,
but it defines more than just a TQFT int the sense of a path integral associated
with closed manifolds,
remember how before you could also have defect TQFTs where you have defects and
worldlines of defects.

So UMTC is a way to define TQFT with worldlines of anyons,
which you can think of as zero-dimensional defects.
Its world line is one-dimensional that can form knots and links,
so we can define path integrals in the presence of these world lines.

TQFT defines the Hilbert space describes the topological sectors of topological
phases of matter.
An important direction a subject of current research is connecting the UMTC that
I'm going to define for you to microscopic properties of a many-body system.
This is an ongoing research topic.
It's slow-going because not many people are working on it,
so it's not fully developed.
There is a nice paper by Kawagoe and Levin where they do nice work of
connecting the abstract UMTC to a physical many-body system that they're using
to describe.

\begin{question}
    Is it Hamiltonian level or state level?
\end{question}
State level.


\subsection{Topological charges (anyons)}
If you're a mathematician you might call these objects of a category,
but mathematicians would think of it as isomorphism classes of simple objects.
If you look up category theory in Wikipedia,
it will tell you category theory is objects and morphisms.
In these categories,
there s a special class of objects called simple objects and isomorphisms of
those and we call those anyons.

We have a finite set of these topological charges, or anyons.
Let's label them
$\left\{ 0,a,b,\ldots \right\}$.
The 0 is the same thing I used to call 1,
but I'll explain.

These are the topologically inequivalent class of anyons,
but the important thing is that there are a finite number.

The first thing that comes up is the notion of a \emph{fusion space}
or \emph{splitting space}.

Define a vector space $V_c^{ab}$ as a ``splitting space'' graphically.

We have a Y shape, $c$ going into a vertex $\mu$ that branches out into $b$ and
$c$.
In going from the graphical notation to the bra-ket notation,
there is a normalization convention 
\begin{align}
    \left( \frac{d_a d_b}{d_c} \right)^{1/4}
    \ket{a,b;c,\mu}
    \in V_c^{ab}
\end{align}
In addition,
there is the dual vector space $V^{c}_{ab}$
called the \emph{fusion space}.
Graphically,
we have an upside down $Y$ with two entering edges $a$ and $b$
into vertex $\mu$ and leaving edge $c$.
\begin{align}
    \left( 
    \frac{d_a d_b}{d_c}
    \right)^{1/4}
    \bra{a,b;c,\mu}
\end{align}
Then I define fusion coefficients
\begin{align}
    N_{ab}^{c}
    &=
    \dim V_{ab}^{c}
    =
    \dim V_{c}^{ab}
\end{align}
And so formally,
we can define the fusion rules as
\begin{align}
    a\times b
    &=
    \sum_{c}
    N_{ab}^{c} c
\end{align}

So,
for each $a,b,c$,
we have a vector space $V^{ab}_{c}$
We can think of these diagrams as operators,
or equivalently,
as a spacetime process.
So in other words,
we have splitting operators as fusing operators.
When I draw this $Y$ diagram,
you can literally think of an operator that takes an anyon $c$
and splits out anyons $a$ and $b$.
You can think $c$ moves in time in fixed location and splits into $a$ and $b$.
There is an index $\mu$ because there are in principle different orthogonal
states that have $a,b$ at different locations,
because the process that causes splitting could be distinct.
So the index $\mu=0,\ldots,N_{ab}^{c} - 1$
labels distinct to split

\begin{question}
    Can $N_{ab}^c$ be greater than 1?
\end{question}
It turns out yes there are examples,
but they are not simple.
The simplest example is $SU(3)$ level 3 CS theory,
which you quantize and gives $N_{ab}^{c}>1$.
In general $N_{ab}^{c}>1$ is considered exotic.

OK, if $N_{ab}^c=0$,
it means there's no way to split or fuse $a,b$ into $c$.
Usually,
$N_{ab}^{c}=0$ or $1$.
Examples where $N_{ab}^{c}>1$ are kind of exotic.
Using these spaces,
I can build up vector spaces with more and more anyons.

For example,
one can define a vector space
\begin{align}
    V_{c}^{a_1 \cdots a_n}
\end{align}
which you would depict as this tree that starts from $c$,
and branches out at $\m_2$ to $a_n$
and $b_n$,
which splits out into $a_{n-1}$ and $b_{n-1}$, etc.
So you can see
\begin{align}
    V_{c}^{a_1 \cdots a_n}
    &=
    \bigotimes_{b_1,\ldots,b_{n-2}}
    V_{b_1}^{a_1 a_2}
    \otimes
    V_{b_2}^{b_1 a_3}
    \otimes \cdots \otimes
    V_{c}^{b_{n-2} a_n}
\end{align}
When I have a state with anyons far apart from each other,
in general there could be  some residual topological degeneracy.
If I tell you where some MZMs are,
you may not know the fermion parity between pairs of anyons.
In general,
in order to specify a state,
what you do is pick a sequence of larger and larger regions,
then pick a basis where $a_1,a_2$ will fuse to some particle $b_1$,
then in you can fuse $b_2$ and $a_3$ to give $b_2$, etc,
and finally they all fuse to $c$.
Just knowing where all the anyons are is not enough to specify which topological
sector we're in,
so you can decompose the state into a fusion tree that provides a basis which
allows one to write the state.
\begin{align}
    \ket{\Psi_{a_1\cdots a_n}}
    &=
    \sum \Psi_{\cdots}
    \ket{a-1,a_2;b_1,\mu_1}
    \otimes
    \ket{a_3,b_1,b_2,\mu_2}
    \otimes
    \cdots
\end{align}

\begin{question}
    Can I think of these as spin-$\frac{1}{2}$?
\end{question}
Yes that's a perfect example,
but wiht spin-$\frac{1}{2}$,
you could in general have arbitrarily large,
so you could just truncate up to $k$.
That just gives $SU(2)_k$ CS theory,
or something similar.

At this stage,
we can already make a distinction 2 kinds of theories.
Abeliean thoeries where
\begin{align}
    a\times b
\end{align}
has a unique outcome
That is,
\begin{align}
    N_{ab}^c &=
    \begin{cases}
        1 &\text{for unique }c\\
        0 &\text{otherwise}
    \end{cases}
\end{align}
This means that $\dim V_c^{a_1\cdots a_n}=1$ or $0$,
depending on $c$.
It's $1$ for a unique $c$ and zero otherwise.

On the other hand,
in a non-Abelian theory,
$a\times b$ would generally have more than one fusino outcome.
And so that implies
\begin{align}
    V_{c}^{a_1\cdots a_n}
\end{align}
can be multi-dimensional,
and because thise $V$ is just telling us about our topological sectors of our
topologicla phase of matter,
it turns out no local operators can distinh what tstates I'm in,
so the number of states I have is tooplogically protected degeneracy.
An example of that is MZMs.


Just a few consequences of the graphical calculus.
If I want to take an inner product between two states,
the inner projject is obtained just by stacking the diagrams.
That just menas taking the inner product
\begin{align}
    \braket{a,b;c,\mu}{a',b';c',\mu'}
    &=
    \delta_{cc'}
    \delta{\mu\ma'}
    \sqrt{\frac{d_a d_b}{d_c}}
    I
\end{align}
where the identity $I$ is jst an edge from $c$ to $c$.

We can also hvae the $\mathbf{1}_{ab}$
that is the trivial operator 
\begin{align}
    \mathbf{1}_{ab}
    &=
    [\text{two parallel edges}|
    =
    \sum_{c,\m} \sqrt{\frac{d_c}{d_a d_b}} [Y\text{ shape}]
\end{align}
There exists soeme unique vacuum charge 0,
somtiems denoted $1$.

There is a dual of $a=\bar{a}$ defined by
\begin{align}
    N_{a\bar{a}}^{0} &= 1\\
    N_{ab}^{0} &= \delta_{a\bar{b}}
\end{align}
Assume each $a$ has a unique $\bar{a}$.
\begin{align}
    0 &= \bar{0}\\
    \bar{\bar{a}} &= \bar{a}
\end{align}
You can also reverse the arrows $a$ becomes $\bar{a}$.

The quantum dimension $a_ \ge 1$ is defined as a loop of $a$.
The total quantum dimension is
\begin{align}
    \mathcalD{D}
    &=
    \sqrt{\sum_a d_a^2}
\end{align}

You can think of vacuum $0$ branching into $a$ and $\bar{a}$ which fuse to
become $0$ again.

I chose a particular wya to define the vector space.
There should be a basis transformations that take from one decompositon to
another.
These basis transfomrations in the most baiscaly fom take up

\subsection{$F$-symbols}
Imagine a process that splits $d$ into $a,b,c$.
The $F$ symbol tells you how you should relate one set of splitting processes
to another set of splitting processes.
\begin{align}
    V_d^{abc}
    \simeq
    \bigoplus_e
    V_e^{ab}
    \otimes
    V_d^{ec}
    \simeq
    \bigoplus_f V_d^{af}\otimes V_f^{bc}
\end{align}

These $F$ are basis transformations we want to be unitary operators.
\begin{align}
    F_{d}^{abc}:
    \bigoplus_e V_e^{ab}
    \otimes
    V_d^{ec}
    \to
    \bigoplus_{f}
    V_d^{af}
    \otimes V_f^{bc}
\end{align}
is a unitary operator,
but if you're a mathematician,
you won't require that $V$ has a positive semi-definite inner product and $F$ be
unitary,
but we do.

\subsection{Pentagon equation}
If you have a vector space of may anyons,
you could choose some way of decomposing $V$ into some basis,
and you want all the ways to consistent with one another.
This consistency equation is called the \emph{pentagon equation}.

The pentagon equation which says that when I go from here via 2 moves,
it's the same as going the other way by 3 rules.
\begin{align}
    FF &= \sum FFF
\end{align}
summing over a bunch of indices.
[picture of 5 branch diagrams connected by arrows]

If you look at this equation,
it should remind you of the 2-3 Pachner moves on the triangulation of the SPT
phase.
It requires us to consider the 2-3 Pachner move which gave us an equation that
looked very similar to this.
Note this is formally similar to the 2-3 Pachner move which gave us 3-cocycle
equations
$\omega \omega = \omega \omega \omega$
for $H^3\left( G, U(1) \right)$ in group cohomology.

\subsection{Maclane's coherence theorem}
The question is whether there are other kinds of consistency equations for the
$F$-moves to be fully consistent.
The answer is no.
\begin{theorem}
    Pentagon equation is sufficient for $F$-moves to be fully consistent.
\end{theorem}
Here are some interesting facts for you prove as an exercise.
\begin{enumerate}
    \item
    \begin{align}
        d_a d_b = N_{ab}^c d_c
    \end{align}
    \item Define fusion matrices $N_a$
    \begin{align}
        \left[ N_a \right]_{bc} &= N_{ab}^{c}
    \end{align}
    Then $d_a$ is the largest eigenvalue of $N_a$.
    \item Only the largest eigenvalue contributes.
        \begin{align}
            \sum_{c} \dim V_c^{\overbrace{a\cdots a}^{n}}
            &=
            \sum_{c} \left[ N_a^n \right]_{oc}
            \sim d_a^n
        \end{align}
\end{enumerate}
So the quantum number $d_a$ is the average number of quantum degrees of freedom,
because the dimension of the state space grows as the quantum dimension to the
$n$-th power.
If $d_a=1$,
then the dimension is just one of these anyons,
and that's an abelian anyon.
If $d_a>1$,
then you have a non-trivial degeneracy that grows exponentially with the number
of anyons and $a$ is a non-abelian anyon.

It turns out that the $F$-symbols determine $d_a$.
You can think of $d_a$ as a parameter.
You have some fusion rules,
$F$-symbols,
and so on.
Then there are some relationships between them,
because the Pentagon equation is a consistency equations that forces only some
choices of $F$, $d_a$ and $N_a$ being consistent.

The $F$-symbols themselves are basis-dependent,
so if I change the basis the $F$-symbols themselves are going to change.
They are not gauge invariant.

\subsection{Gauge transformations}
There is a basis transformation you can do of these $V_{c}^{ab}$
\begin{align}
    \Gamma_{c}^{ab}:
    V_{c}^{ab}
    \to
    V_{c}^{ab}
\end{align}
And you can relate
\begin{align}
    \ket{a,b; c,\mu}
    &=
    \sum_{\nu}
    \left[ \Gamma_c^{ab} \right]_{\mu\nu}
    \ket{a,b;c,\nu}
\end{align}
And then
\begin{align}
    \tilde{F}_d^{abc}
    &=
    \Gamma_e^{ab}
    \Gamma_d^{ec}
    F_d^{abc}
    \left( \Gamma_f^{bc} \right)^\dagger
    \left( \Gamma_d^{af} \right)^\dagger
\end{align}
If you have an inequivalent consistent choice of this data
$\left\{ N_{ab}^{c}, F_{d}^{abc} \right\}$,
then you have defined a unitary fusion category.

There are two Hilbert spaces.
There's the Hilbert space of the TQFT,
then there's the Hilbert space of the topological phase of matter,
which consists of microscopic constituents interacting.
This is the Hilbert space of the TQFT,
which models the topological sectors the ground states of the topological phase
of matter than are pinned to these locations.

So you can either think of them as the topological sectors of the real phase of
matter,
or just the Hilbert space of the TQFT.

\begin{question}
    Is this enough to define the ground state of a torus?
\end{question}
Suppose you want a vector space on a torus $V\left( T^2 \right)$.
I can start with a sphere here
with two punctures $a$, $\bar{a}$.
And another sphere with punctures $\bar{a}, a$,
which I glue together to give a torus.
And then
\begin{align}
    \bigoplus_{a} V_{0}^{a\bar{a}}
    \otimes
    V_0^{a\bar{a}}
    &=
    V\left( T^2 \right)
\end{align}
So then
\begin{align}
    \dim V\left( T^2 \right) &=
    \sum_a 1
    = \text{\# of anyons}
\end{align}
\begin{question}
    Can you extract all the data from the path integral of the TQFT?
\end{question}
I haven't described how to find a TQFT from this structure yet.
It's almost another lecture.

Let me give you a nice example to have in mind of a fusion category which
connects to this group cohomology.

Our unitary fusion category will be denoted by
\begin{align}
    C &=
    \Vec_{G}^{[\omega_3]}
\end{align}
where $[\omega_3]\in H^3\left( G, U(1) \right)$.
The anyons are
$\left\{ g \mid g \in G \right\}$
with
$g\times h = gh$
and $N_{g,h}^{k} = \delta_k,gh$.
Then
\begin{align}
    F_{g,h,k}^{g,h,k}
    &=
    \omega_3\left( g,h,k \right)
\end{align}
So the pentagon equation gives a 3-cocycle equation for $\omega_3$.
The gauge transformation corresponds to 3-coboundaries.

Any UTF $\mathcal{C}$ can be used to define a $(2+1)D$ TQFT in terms of a state
sum,
which generalized the Digraph-Witten theory.

The amplitude for the tetrahedra is the $F$-symbols.
The pentagon equation you can view as retriangulation invariance of the path
integral.

These TQFTs define ground state wave functions on their boundaries.
These are Tuaev-Viro-Barrett-Westbury TQFTs,
which leads to Levin-Wen string-net models.

\subsection{Braiding}
I want to take anyons around each other and exchange them and so on.
I take a particle $c$ up to vertex $\mu$ that then branches and braids into $b$,
$a$.
You should really compare this to a diagram where there is no braiding,
and that is via $R$-symbols.
\begin{align}
    \left[ \text{diagram} \right]
    &=
    \sum_c \left[ R_c^{ab} \right]_{\mu\nu}
    \left[ \text{diagram} \right]
\end{align}
If I had a state
\begin{align}
    \ket{\Psi_{a_1\cdots a_n}}
    &=
    \sum
    \Psi_{\cdots}
    \ket{a_1,a_2;b_1,\mu_1}
    \ket{a_3b_1; b_2, \mu_2}\cdots
\end{align}
The braid operator can act like full braiding
\begin{align}
    B_{12}
    \ket{\Psi_{a_1,\cdots a_n}}
    &=
    \sum
    \Psi_{\cdots}
    \left[ 
    R^{a_2 a_1} R^{a_1 a_2}
    \right]_{\mu-1,\mu_1'}
    \ket{a_1,a_2; b_1, \mu_1'}
    \cdots
\end{align}
I can also define the $R^{ab}$ operator,
which is a diagram of $a$ and $b$ crossing.
\begin{align}
    R^{ab}
    &=
    [\text{diagram of $a$ and $b$ crossing}]\\
    &=
    \sum_{c,\mu}
    \sqrt{\frac{d_c}{d_a d_b}}
    [\text{diagram}]\\
    &=
    \sum_{c,\mu}
    \sqrt{\frac{d_c}{d_a d_b}}
    \left[ R_c^{ab} \right]_{\mu\nu}
    [\text{diagram}]
\end{align}
Unitary here means that
\begin{align}
    \left[ R^{ab} \right]^{-1}
    &=
    \left[ R^{ab} \right]^\dagger
\end{align}
so $R^{ab}$ is unitary.
Braiding commutes with fusion.

[diagrams]

Using these diagrams,
you can prove this famous relationship called the Yang-Baxter relation.

[diagram]

Just like we had the pentagon equation,
we need a consistency equation,
which leads us to a set of equations called the
\emph{hexagon equations},
which is compatibility between braiding and fusion.

Let me draw one of the diagrams.

[6 diagrams with arrows]

There is also another equation where the braid goes under instead of over.

If you actually include all the Greek indices, you will have
\begin{align}
    \sum_{\text{greeks}} RFR
    &=
    \sum_{\text{greeks}} \sum FRF\\
    \sum_{\text{greeks}} R^{-1}FR^{-1}
    &=
    \sum_{\text{greeks}} \sum FR^{-1}F
\end{align}

\begin{question}
    Do people do this day-to-day?
\end{question}
It is a research topic in maths to find solutions to these equations
systematically,
although it's very difficult because they're extremely complicated equations,
but yes,
there are all sorts of tricks to solve them.

Once we have these $R$-matrices,
we can define some other things.
\begin{align}
    \theta_a
    &=
    \theta_{\bar{a}}
    =
    \sum_{c,\mu}
    \frac{d_c}{d_a}
    \left[ R_c^{aa} \right]_{\mu\mu}\\
    &=
    \frac{1}{d_a}
    \left[ \text{diagram of figure 8}\right]
\end{align}

\subsection{Topological $S$-matrix}
The other important diagram to define is the topological $S$-matrix.
This is a matrix we diagrammatically define like this
\begin{align}
    S_{ab} &=
    \frac{1}{\mathcal{D}}
    \left[ \text{diagram of two rings $a$ and $b$ interlocking }\right]
\end{align}
Using the $S$-matrix there are some important identities one can prove.
Consider a strand $b$ and you take an $a$ loop around it,
then that's equivalent to $S_{ab}/S_{0b}$ times $b$.

You can verify this diagram by closing it up,
or taking the trace of the operator.
Understanding this relationship is useful for describing this very important
identity.

Consider a line of $x$ with two loops $a$ and $b$ going around it.
They can go around individually like
$\frac{S_{ax}}{S_{0x}} \frac{S_{bx}}{S_{0x}}$
times $x$,
or I can fuse the $a$ and $b$ together to get
\begin{align}
    \sum_{c} N_{ab}^c [\text{diagram}]
    &=
    \sum_c N_{ab}^c \frac{S_{cx}}{S_{0x}}
\end{align}
so
\begin{align}
    \frac{S_{ax}}{S_{0x}}
    \frac{S_{bx}}{S_{0x}}
    &=
    \sum_{c}
    N_{ab}^{c}
    \frac{S_{cx}}{S_{0x}}
\end{align}
which basically means these obey the fusion rules.

\subsection{Summary}
So under gauge transformation
\begin{align}
    R_{c}^{ab}
    \to
    \tilde{R}_{c}^{ab}
    =
    \Gamma_{c}^{ba}
    R_{c}^{ab}\left( \Gamma_{c}^{ab} \right)^\dagger
\end{align}
If you give me an inequivalent consistent choice of $\left\{ N,F,R \right\}$,
it can define a unitary braided fusion category (UBFC).
So if you give me some very crazy links and knots,
I can give you a number using these rules.

There's one more: modularity.
Modularity basically just meas $S$ is unitary,
which means I can invert that relation on the right hand side to write
\begin{align}
    N_{ab}^{c}
    &=
    \sum_{x}
    \frac{S_{ax} S_{bx} S_{cx}^*}{S_{0x}}
\end{align}
which is the famous Verlinde formula.
There's a nice relationship the $S$ matrix and the twists satisfy.
There is also the charge-conjugation matrix
\begin{align}
    C_{ab} &=
    \delta_{a\bar{b}}
\end{align}
And let me define the twist
\begin{align}
    T_{ab}
    &=
    \theta_a \delta_{ab}
\end{align}
And it turns out there is a really beautiful relationship
\begin{align}
    \left( ST \right)^3 &=
    e^{2\pi i C_{-}/S} C\\
    S^2 &= C\\
    C^2 &= \mathbf{1}
\end{align}
Here $C_-$ is the chiral central charge.
And it turns out there is 
\begin{align}
    e^{2\pi i C_{-}/8}
    &=
    \frac{1}{\mathcal{D}}
    \sum_{a}
    d_a^2
    \theta_a
\end{align}
The modular tensor category defines the central charge mod 8 because of this
formula.
You may recognize this is a projective representation of $SL(2,\ZZ)$.

One more important fact.
Modularity is actually equivalent to braiding non-degeneracy,
which just means that
for all $a\ne 0$,
there exists a $b$ such that
\begin{align}
    R^{ab}R^{ba} \ne \mathbf{1}
\end{align}
That means every anyon is detectable by braiding.
For any anyon $a$,
there is some $b$ such that if I braid it all around,
I don't get something trivial,
so there exist no invisible particles by braiding.
This is equivalent to modularity.

So a braided fusion category with a unitary $S$ matrix is a modular fusion
category.
Once you have this,
you have a topological order,
if you don't care about symmetry.
If you do care about symmetry,
then you need to soup up this theory with a symmetry group $G$.
That's another lecture.

It's also another lecture for how to get from a UMFC to  $(2+1)D$ TQFT.
