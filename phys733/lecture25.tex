\section{Chern-Simons and Edge theory}
\begin{align}
    \mathcal{L} &=
    \frac{m}{4\pi} a_\mu \partial_\nu a_\lambda \epsilon^{\mu\nu\lambda}
\end{align}
The boundary theory is
\begin{align}
    \mathcal{L}_{\mathrm{edge}}
    &=
    \frac{m}{4\pi}
    \partial_x \phi \partial_t \phi
    -
    V \left( \partial_x \phi \right)^2
\end{align}
The field doesn't commute with itself at different locations
\begin{align}
    \left[ \phi(x) , \phi(y) \right]
    &=
    \frac{-i\pi}{m}
    \sgn\left( x - y \right)
\end{align}
Then
\begin{align}
    \partial\mathcal{L}
    &=
    \lambda\cos\left( m\phi \right)
\end{align}

This edge theory with $m=1$ is the bosonize theory.
This chiral Luttinger liquid is the ``bosonized'' description
($m=1$) of the chiral free fermion.
Based on what we said last time,
we can write the fermion
\begin{align}
    \Psi(x)
    &\propto
    e^{im\phi}
\end{align}
for $m=1$.
Your question was why can we gap out free fermions when they don't commute with
each other.
You actually cannot gap out chiral free fermions.

If we had counterpropagating fermions,
we have $\Psi_L$ and $\Psi_R$,
then we could add terms like
$\Psi_L^\dagger \Psi_R + \mathrm{h.c.}$.
And that wed be like adding
\begin{align}
    e^{-i\phi_L} e^{i\phi_R} + \mathrm{h.c.}
    = \cos\left( \phi_L - \phi_R \right)
\end{align}
So if you had two counter-propagating modes,
and you looked at the commutation relation,
you would find that
\begin{align}
    \left[ \phi_L - \phi_R(x),
    \phi_L - \phi_R(y)
    \right] = 0
\end{align}
Fully chiral modes however cannot be gapped out.
The key is to put everything in terms of bosonized description,
add a cosine term and see if you can pin the argument.
You can't if it's fully chiral.

Last time,
we looked at the ``electron operator''
\begin{align}
    \Psi(x)
    \propto e^{im\phi}
\end{align}
And we wanted just to get an operator with charge 1.
We derived it by requiring its commutation with the density operator
\begin{align}
    \left[ \rho(x), \Psi^\dagger(y) \right]
    &=
    \Psi^\dagger (y) \delta\left( x - y \right)
\end{align}
and
\begin{align}
    \rho &= \frac{1}{2\pi} \partial_x \phi.
\end{align}
And we knew the statistics had this algebra
\begin{align}
    \Psi(x) \Psi(y)
    &=
    \Psi(y)\Psi(x)
    \left( -1 \right)^{m}
\end{align}
which came from the commutation relation.

We can also write down the quasiparticles on the edge.
Here we would have operators like
\begin{align}
    V_a &\propto
    e^{ia\phi}
\end{align}
There are proportionality factors that don't really matter.
The important piece is this part.
These operators just from studding how it commutes with the density operator,
you can see carry charge $a/m$.

So then $V_a$ and $V_a^\dagger$ are the creation and annihilation operators for
charge $a/m$ excitations.
This is a representation of those to on the edge theory.

If you applied this operator here on the edge,
you would locally create an excitation on the edge.

It's fractional statistics by looking athte commutation relations of this
operator.
\begin{align}
    V_a(x) V_b(y)
    &=
    V_b(y) V_a(x)
    e^{i\frac{\pi}{m} a\cdot b}
\end{align}
That's like a manifestation of hte fact that exchanging the two has fracitonal
statistics
that arises on the edge from that algebra.

Previously,
I mentioned that if you take smoe FQHE state descibed by a $U(1)_m$ CS theory,
then we put a metal near it,
and we can have electrons tunnel from the metal to the FQH edge.
The tunneling conductance
\begin{align}
    \frac{dI}{dV} \propto
    \text{density of states}
    \propto
    V^{m-1}
\end{align}
measures the density of states of the system you're tunneling into.
And the density of states is given by the imaginary part of the Greens function
or correlation function.

The prediction is that there is a quantized conductance given by this integer
$m$.
Experiments try to look for this.
They never saw good quantization,
believed to be because the actual physical systems that arise
aren't really perfectly described by this idealized theory.
It's a whole topic to talk about what the departures from the idealized theory
actually are and so on.

You could also study quasipartciles tunneling from say one edge to another.
If you had some FQH system,
like a strip with two edges,
you could bring the two edges close to each other
and realize what's called a ``quanutm point contact''.
Because the bulk supports these $m-1$ excitations,
you could have edges tunneling into each other.
So your Lagrangian would have left mover and right mover
plus a particulars point that cause the quasiparticle from one edge
to tunnel to another one
\begin{align}
    \mathcal{L}
    &=
    \mathcal{L}_L
    + \mathcal{L}_R
    +
    \delta(x) \left( 
    V_{a,R}^\dagger V_{a,L} + \mathrm{h.c.}
    \right)
\end{align}
This cause the current to backscatter.
Electrical current can only go in one direction,
but the tunnel could go across,
and you could try to measure that backscatter current.
It turns out properties of this $V^_{a,R}^\dagger V_{a,L}$ show up in the
backscatter current.

You could try to measure the noise in the backscatter current
$\langle I_{bs}^2\rangle$
which would contain properties of the quantization.

I just want to give you some flavour of what you can do once you have this
theory.
You could study all kinds of properties,
and non-trivial properties of these functions,
and the scaling and so on enter.

\begin{question}
    Is $a$ arbitrary?
\end{question}
Here $a$ must be an integer.
The reason is because $\phi$ is a compact boson,
so $\phi$ is equivalent to $\phi + 2\pi$.
That's why $a$ needs to be an integer.
Where does that come from?
That realises the gauge field $a$ in
\begin{align}
    \mathcal{L} &=
    \frac{m}{4\pi} a\, da
    +
    \frac{q}{2\pi} A\, da
\end{align}
And we found the charge of excitations is
\begin{align}
    \frac{q\cdot \ell}{m}
\end{align}
where $\ell$ is the charge carried under $a$.
The reason that $q$ and $\ell$ had to be integers is because $A$ and $a$ are
compact gauge fields.
To make sure the theory is invariant under large gauge transformations,
you need to make sure any particle charged under $a$ has to be integer,
and that's why the $a$ in $V_a$ has to e integer.
When you go to the edge theory,
you  find $\phI$ itself is a compact variable and that's why this $a$ is
quantized too.
Maybe I should call it $\ell$ instead.
\begin{align}
    V_{\ell} \propto e^{i\ell \phi}
\end{align}
and the $V_{\ell}$ and $V^_\ell^\dagger$ are the creation and annihilation
operators for charge $\ell/m$ excitations.

\begin{question}
    Are there exactly solvable models for this.
\end{question}
Some aspects can be solved,
but no.
This comes down to something deep in the study of topological phases of matter.
There is a dichotomy between phases of matter with gapless edge modes,
like this $U(1)_m$ CS theories,
and those that do admit gapped boundaries,
which have exactly solvable models.
Models with gapless edge modes can be solved by numerics to see these properties
though.


\section{Quantum Spin Liquids}
I want to discuss various physical contexts in which these abstract theory arise
in.
FQH states are the most important context,
but we'll come back to that.
For now I want to talk about quantum spin liquids.
Quantum spin liquids are phases of matter that have the properties that they are
electrical insulators,
and furthermore,
this is not an essential property,
but it's where the name comes from,
that the spin not order at $T=0$.
Here the spins don't order,
so they form some liquid state at zero temperature.
This is unusual,
because if you go to low enough temperature,
spins should order,
like a ferromagnetic,
spin density wave or whatever.
Hiving the spins not order at zero temperature is a little unusual.

TO have these two properties is extremely non-trivial if you have a system with
one electron per unit cell.

If you remember band theory,
in order to form a band insulator,
you need to have one fermion per unit cell,
but with a spin degeneracy,
you have 2 electrons per unit cell
to form an insulator.
If you form an insulator with only one electron per cell,
and you have spin symmetry,
interactions must be involved and you cannot understand it with just band
theory.

Band insulator with with SO(3) spin interaction symmetry requires 2 electrons
per unit cell.

Such a thing is called a Mott insulator.
There's no precise definite,
but the closest to a definition is something like this.
More importantly,
it is a system that is insulating because of Coulomb interactions.

So you have electrons,
and one electron per unit cell,
but even with half filling,
they don't want to move on top of each other because of a strong Coulomb effect,
and the system could potentially form an insulator.
Coulomb interaction is stopping electrons from going on top of each other and
therefore forming and insulating state.

Any insulator with 1 electron per unit cell plus OS(3) spin with symmetry
requires strong interactions.
It's not something you can understand with free fermions and weak interactions.

\section{LSM theorem}
So far I said that you can have a system with one electron per unit cell,
it can still form an insulator,
but suppose I move away from 1 electron per unit cell.
If your system is clean without disorder,
you have mobile charges moving around,
so your system comes metal or superconducting,
because they might pair up an and condense.
Unless you have Impurity states that trap this extra charge.

It's not just this,
there's something much more powerful you can say.
that goes back to the Lieb-Schulz-Mattis theorem,
which places very strong constraints on the ground state properties of a system
with one spin-one-half particle on each unit cell.
the statement is from 1961 in 1D.

A spin-$\frac{1}{2}$ chain has an excitation gap $\Delta E \le \mathrm{const}/L$
where $L$ is the size of the system.
Assuming the Hamiltonian $H$ has $SO(3)$ spin rotation symmetry
and it has translation symmetry.

This is a really deep statement.
It's saying that I cannot havea unique ground state wihta finite gap ot a uniuqe
ground state
if I want o preseve these smmetries.

what are the options then?
What states could I have?

The implication of this thoery is that he ground state is either
\begin{enumerate}
    \item Spontaneous symmetry breaking.
        For example, the system could spntaneously dimerize,
        break the translational symmetry,
        so you have 2 spin-$\frac{1}{2}$
        per unit cell,
        so neighbouring spins can form singlests,
        and it costs a finite gap to creating excivatiosns
        butecause you have to beraak some bonds.
        But this state is an example of a valence bond solid that soona-=onesly
        break the symmetry.
        Asside from the finitegrund state,
        the spectrum has a finiteeeerg gap.
    \item The other possibility is that hte systme is just gaplesss,
        which means tpower la correlations infinite correaltion length.
\end{enumerate}
You still have finite corrrelation lnegth,
The tower of aroanother onseuqneceis th the tiial gappped state is not
possibe=l.e

This is spin-$\frac{1}{2}$ which is Go

This is the new result.
For $d>1$,
(Hastings $N$).
Let there be $V/m^3$.
For an odd number of unit spin$_Y_2$  per unit cell.
\begin{align}
    SO(3) \times \mathbb{Z}^d
\end{align}
and
\begin{align}
    \Delta E \ge \frac{c\log L}{L}
\end{align}

If I have $SO(3($ s
$V$ unit cells $L_i$ length ni $i$ dimension.
$L:= L_i$.

\subsection{Oshikava (2000)}
For systems vertices $U(1)\times \mathbb{Z}^2$ unique gapped ground filling
charge to inconstant equivalent ${Z}^2$.
There is a unique ground state filling charge per unit cell
\begin{align}
    \nu = n \in m\mathbb{Z}.
\end{align}
there is spin-$\frac{1}{2}$ per unit cell with SO(3) symmetry
equals half-filling $Y_2$ for any $I(1)$ subgroup.
\begin{align}
    \ket{\downarrow} &= \ket{0}\\
    \ket{\uparrow} &= \ket{1}\\
    \langle S^Z\rangle &= 0
   \textrm{implies}
   \langle n\rangle &= \frac{1}{2}.
\end{align} 

For all $d>1$,
if odd numbers of spin$\frac{1}{2}$ per unit cell and or fractional filling,
these are the possibilities
\begin{enumerate}
    \item ground states spontaneously breaks $SO(3)$\\
    \item Gapless (compatible with symmetries)\\
    \item topological order (compatible with symmetries)
        Typically degenerate local ground states with periodic body conditions
\end{enumerate}
This is a powerful theorem.
It constraints the infrared from UV information.

LSM constraints are examples of anomalies in QFT.

The first most important thing in QFT are symmetries,
the second most important things are anomalies.
like 't Hooft anomaly.

Anomalies provide strong constraints,
but the LSM theorems is more general than anomalies.
This i the are instances where you have low energy constraints on high energy
data.

There is a poor mans' proof of LSM,
which you can see in 30 seconds see why it has to be true,
but is not rigorous.
The point is the following.

Considers a system with spin-$\frac{1}{2}$ per unit cell,
and you have spin rotation symmetry.
If I have $n$ unit cells,
the state space if fusing sin-half tensor spin-half $n$ times.
If odd,
that has to fuse to a half-integer spin just by presentation theory of SU(2) or
SO(3)
If the Hamiltonian preserves that,
then this Hamiltonian has to come with double degeneracy at least.
With an odd length system,
I obviously have to have a bound on the excitation gap,
because I need to have at least two degenerate ground states.

For a system with an odd number of unit cells with an odd number
of spin $\frac{1}{2}$ spins per unit cell,
the ground state must be at least 2-fold degenerate because
$\frac{1}{2}\otimes \frac{1}{2}\otimes
\cdots
= \frac{1}{2}\oplus \frac{3}{2} \oplus$.

Now for the even.
If the system is gapped,
then it has a finite correlation length,
so the number of unit cells should not matter for energy spectrum up to
$O\left( e^{-L/\xi} \right)$ corrections..
If it has a finite correlation length,
if I put the system on a large ring
and I change the number of  unit ells by one,
it should be sensitive only to what's happening in some small region,
roughly a correlation length away,
and it shouldn't care what the overall size of the ring is up to exponentially
small corrections.

From this,
we can conclude that if the system is gapped,
then it must come with at least a 2-fold degeneracy up to exponentially small
corrections,
Regardless of whether we have even or odd.

Consider a torus $T^d$,
and consider if hte total numbero f unit cells is even or odd.
If gappe,d
the toal numbero f unit cells should not matter,
so it should be true up to these corrections.

In $d=1$,
the only way of having a gapped ground state with degenerate ground states,
is through spontaneous symmetry breaking.
The only way to have a gapped degenrate ground state,
aside from accidental degeneracies,
is through spontaneous symmetry breaking.
The example of valence bond crystals that deimerizes,
foms singlests,
then fors some crystal arrangmenet of singlets.
For $d>1$,
we can have topological order.
Specifically,
we mena non-invertible topologcial order.
So we cna have topological degeneracy on a non-trivla space.
In this case,
we want a degenerate set of ground states.
Likethese examples of CS theories that we studied.

The generalization to other groups is obvious.
Consider your symetry to be an on-site symmetry cross some integer symmetry
$G = G_{\mathrm{on-site}}\times \mathbb{Z}^d$.
Consider a projective rpresenation of $G_{\mathrm{on-site}}$ per unit cell.
So spin-$\frac{1}{2}$.
Projective reperenstaions are classiifed by a second cohomology group that is in
general some finite abelian group $A$.
\begin{align}
    [w]
    \in H^2\left( G_{\mathrm{on-site}}, U(1) \right)
    = A = \mathbb{Z}_{n_1}\times\mathbb{Z}_{n_2}\times\cdots
\end{align}
and projective representaions necessarily have dimension greater than one,
just like spin-half has dmiension 2.
As long as the numbero f unit cells is not commensurate iwth these factors
$n_i$,
then the overall Hilbert space of the large system needs to also form a
projective represntation for a smilar reaons that an odd number of spin-half
fuse ot an odd number of spins.

This poor man's argumetn si not a proof and gives not much insight.

\subsection{More Detailed Argument}

Consider a $U(1)$ symmetry,
filling $\nu$.
Consider a cylinder with $x$ azimuthal of length $L$ around it.
Consider a ground sate and its translation
\begin{align}
    \hat{T}_x\ket{\Psi_0}
    &=
    e^{i\hat{P}_x}\ket{\Psi_0}\\
    &=
    e^{iP_x^0} \ket{\Psi_0}
\end{align}
and now we adiabatically insert flux into the Hamiltonian $H[\Phi]$.
Then the state will adiabatically transform to another state after inserting one
flux quantum.
And after one flux quantum,
\begin{align}
    \ket{\Psi_0} \to \ket{\Psi_0'}
\end{align}
and if the ground state is unique,
then
\begin{align}
    \ket{\Psi_0'} &= c \ket{\Psi_0}
\end{align}
for some constant $c$.
After inserting one unit of flux,
we can do a large gauge transformation and come back to the original
Hamiltonian.

The large gauge transformation looks like
\begin{align}
    U &=
    e^{i \frac{2\pi}{L} \sum_{\vec{r}} x \hat{n}_{\vec{r}}}
\end{align}
so that
\begin{align}
    U H[2\pi] U^\dagger &= H[0]
\end{align}
and after inserting $2\pi$ flux plus a large gauge transformation,
then
\begin{align}
    \ket{\Psi_0} &\to U\ket{\Psi_0'}
    = cU \ket{\Psi_0} = \ket{\tilde{\Psi}_0}
\end{align}
and now the idea is let's calculate the eigenvalues of translation on
$\ket{\Psi_0'}$,
or alternatively,
let's calculate the momentum of $\ket{\Psi_0'}$

We're going to use the property that
\begin{align}
    U^\dagger T_x U &=
    e^{-\frac{2\pi i}{L} \sum_r x n_r} T_x
    e^{+ \frac{2\pi i}{L} \sum_r x n_r}\\
    &=
    e^{\frac{2\pi i}{L} \sum_r n_r} T_x
\end{align}
so then the momentum can be calculated by
\begin{align}
    T_x \ket{\tilde{\Psi}_0 }
    &=
    T_x c U\ket{\Psi_0}\\
    &=
    c e^{\frac{2\pi i}{l} \sum_r n_r}
    U T_x \ket{\Psi_0}
\end{align}
so then
\begin{align}
    T_x \ket{\tilde{\Psi}_0 }
    &=
    e^{\frac{2\pi i}{L} \sum_r n_r}
    e^{i P_x^0}
    \ket{\tilde{\Psi}_0}
\end{align}
so the state $\ket{\tilde{\Psi}_0}$ has momentum
\begin{align}
    P_x^0 + 2\pi \nu
\end{align}
which is the momentum of the new state.
If $\nu$ is not an integer,
the new state $\ket{\tilde{\Psi}_0}$ has a momentum which is different from the
original state, and so therefore must be orthogonal.
If it's orthogonal,
then that contradicts the original assumption that there is a unique ground
state.

Another  way of saying this is that if you have fractional filling,
you can construct another state which is really low energy which is orthogonal
to it.
That's the more sophisticated argument for why the filling needs to be an
integer if you want a unique ground state.

here I just showed if I have $U(1)$ symmetry and a unique gapped ground state,
then the filling must be integer.
