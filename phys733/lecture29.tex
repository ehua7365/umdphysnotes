\section{Last lecture}
Finish off parton construction with quantum spin liquids and spend the last 20
minutes for an overview of what we did in the course and try to tie things
together a bit.
Fermionic parton construction.
\begin{align}
    \vec{S} &=
    \begin{cases}
        \frac{1}{2}f^\dagger \vec{\sigma} f\\
        \frac{1}{2}z^\dagger \vec{\sigma} z
    \end{cases}
\end{align}

We had spin operators with 2 states per site.
And two flavours of fermions per site.
We actually get 4 states per site for the fermions,
but we deemed two of them unphysical,
so that's how we describe the two initial states.

So we have $\ket{\uparrow}=\ket{01}$
and $\ket{\downarrow}=\ket{10}$
which are physical states but the states
$\ket{00}$ and $\ket{11}$ are not gauge invariant and hence unphysical.

There is a $U(1)$ gauge redundancy that keeps physical operators invariant.
\begin{align}
    f &\to e^{i\theta}f
\end{align}
Actually there is an $SU(2)$ gauge redundancy.

To implement this gauge constraint,
we need to enforce a constraint,
which is that the number of particles per site should always be 1.
\begin{align}
    n_1 + n_2 = 1
\end{align}
So we're in the $01$ and $10$ subspace.
If we decompose the operator this way and enforce the constraint,
that gives us new mean-field approximations that we didn't have access to
before.

We started with the Heisenberg model,
and rewrote it as just  model of fermions hopping.
\begin{align}
    H &=
    \sum_{ij} \vec{S}_i\cdot\vec{S}_j J_{ij}\\
    &=
    \sum_{ij}
    \left[ 
    -\frac{1}{2} J_{ij} f_{i\alpha}^\dagger
    f_{j\alpha}
    f_{j\beta}^\dagger
    f_{i\beta}
    +
    J_{ij}
    \left( 
    \frac{1}{2}n_i
    -
    \frac{1}{4} n_i n_j
    \right)
    \right]
\end{align}
And, we can then write the ansatz that
\begin{align}
    \chi_{ij} &=
    \langle f_{i\alpha}^\dagger f_{j\alpha}\rangle
\end{align}
which if we put nito the Hamiltonian gives the mean-field Hamiltonian
\begin{align}
    H_{mf}
    &=
    \sum_{ij}
    \left[ 
    \left( 
    -\frac{1}{2} J_{ij} f_{i\alpha}^\dagger f_{i\alpha}
    \chi_{ji}
    +\mathrm{h.c.}
    \right)
    -
    \left| \chi_{ij} \right|^2
    \right]
\end{align}
where we have replaced $f_{i\alpha}^\dagger f_{j\alpha}$
by the mean field.
And for consistency,
we want ot enforce the equation
$ \chi_{ij} = \langle f_{i\alpha}^\dagger f_{j\alpha}\rangle$.

The zeroth order thing to do is to not require this constraint exactly,
but require it on average.
So there is a zeroth order mean field which is to treat the constraint on
average.

One thing we can do is add an extra Lagrange mulitiplier in the Hamiltonian
which acts like a chemical potential that fixes this average
$\langle n_i \rangle = 1$.
So we have
\begin{align}
    H_{0-mf}
    &=
    H_{mf}
    +
    \sum_i a_0 
    \left( 
    f_{i\alpha}^\dagger f_{i\alpha}
    - 1
    \right)
\end{align}
so this is basically a site-dependent chemical potential to enforce
$\langle n_i \rangle = 1$.
We hvae this Hamiltonian that is afree-fermion Hamitonian,
and we hvae thse parameters $a_0$ and $\chi$ we can tune,
but we want to choose them to be self-consistent such.
Choose $a_0$ such that $\langle n_i\rangle = 1$.

We are assuming $\chi_{ij} &= \bar{\chi}_{ij}$
ad $a_0(i) = \bar{a}_0$
satisfy self-consistency equations.
The fermions carry spin half but are electrically neutral.
These excitations are electrically neutral spin-$\frac{1}{2}$ fermions.
We call these fermionic \emph{spinons},
which are excitations that carry spin-$\frac{1}{2}$.
Local excitaitons always carry integer spin,
because every spin flip operation changes the spin by one,
when it goes from $-\frac{1}{2}$ to $\frac{1}{2}$.

Any logcal opeator can only create integer spin excitations,
so a spin-$\frac{1}{2}$ is a non-trivial thing.
They also happen to be electrically neutral fermions.
The zeroth order mean field theory is useful because you can deduce you can have
fermionic spinon excitations,
but in some sense you can only treat the constrant on average,
but you should rally treat hte constaint exactly.

We don't know if hte meanf ifeld theory is stable to flutcuations either,
which could be in phase or amplitude of the $\chi_{ij}$ mean field.
Another way of arriving at the $\chi_{ij}$
is instead of doing a mean field theory is do a
Hubbard-Stratonvich transformation,
where you replace
$\phi^4 \to \phi^2 \phi^2 + \chi^2$
completing the square.
It's a common trick,
but if you're familiar wiht it,
just consider it a mean field hteory.

\subsection{First-order mean field}c:w
we consider a first-order mean field theory
to check the stability of the zeroth order mean field
and treat hte constraint exactly.

The first fluctuations to consider are phase fluctuations.
\begin{align}
    \chi_{ij} &=
    \bar{\chi}_{ij} e^{i a_{ij}}
\end{align}
The amplitude fluctuation is gapped due to the $|\chi_{ij}|^2$
term so we don't have to worry about it.

A useful way of thinking about this is the integral language.
\begin{align}
    Z &=
    \int \mathcal{D} f \mathcal{D} \bar{f}\,
    e^{iS}
    \prod_{i,t} \delta\left( f_i^\dagger f_i - 1 \right)
\end{align}
Every site and time has a delta function.

And we can introduce fields $a_0(i, t)$ 
\begin{align}
    Z &=
    \int \mathcal{D} f \mathcal{D} \bar{f} \mathcal{D} a_0\,
    e^{i \sum_{i, t} a_0 (i, t)} \left( f_9^\dagger f_i - 1 \right)
    e^{iS}
\end{align}
and hte Hamiltonian becomes
\begin{align}
    H_{1,mf}
    &=
    \sum_{ij}
    -\frac{1}{2} J_{ij}
    \left[ 
    \left( 
    f_{i\alpha}^\dagger f_{j\alpha}
    \bar{\chi}_{ji} e^{-ia_{ij}}
    + \mathrm{h.c.}
    \right)
    -
    \left| \bar{\chi}_{ij} \right|^2
    \right]
    -
    \sum_{i} a_0\left( i \right) \left(
    f_{i\alpha}^\dagger f_{i\alpha} - 1
    \right)
\end{align}
and now $a_0$ becomes a dynamical field.
Then $a$ can be thought of as a $U(1)$ gauge field,
with the transformation
\begin{align}
    f_{i} &\to e^{i\theta_i} f_i\\
    a_{ij} &\to a_{ij} + \theta_j - \theta_i
\end{align}
keeping it invariant.
Then $a_0$ literally enters the theory as the time-componento f the gauge field.
\begin{align}
    \mathcal{L} &=
    i \sum_{i}
    f_i^\dagger \left( \partial_t - i a_0 \right) f_i
\end{align}
So basically we get to fermionic spinons plus $U(1)$ gauge fields.
We should emphases the gauge field is a dynamical emergent gauge field.
It wasn't there microscopically,
but once we do this decomposition,
it has to be there due to the gauge redundancy,
because we expanded the Hilbert space got expanded and we need to restrict it
back down.

The fluctuations of $\chi_{ij}$ correspond to the gauge field fluctuations.
If we want to understand the stability of the mean-field ansatz,
we need to understand the stability to fluctuations of this $U(1)$ gauge field.

The stability of mean field is related to the stability to $U(1)$ gauge field
fluctuations.
There's  lot to do in this direction,
and you could consider all kinds of mean field ansatzs and you can go back to
see that we have an $SU(2)$ gauge field fluctuations which are much larger,
and you can put those in the theory which requires you to havea few more
paramteres,
so you can analyse all sorts of mean field theories.

But now I want ot switch gears and not talk about hte Hamiltonian and the mean
field gauge theory,
but instead I want to talk about the wave functions.

This leads to a whole class of trial wave functions,
or variational wave functions,
which are basically projected wave functions.

This is a spatial component and this is a time component.

Whence have this mean field ansatz,
we have to look at the ground state
\begin{align}
    \ket{\Psi_{\mathrm{mean}}\left( \chi_{ij} \right)}
\end{align}
which is the mean field state of partons.
this mean field by itself is not a physical state because it doesn't live in the
spin Hilbert space.
Because it's a ground state of the zeroth order mean field theory,
it contains unphysical states where there are 0 or 2 fermions per site.
To get back into the original Hilbert space,
we can take this mean field ground state and project it into the physical
Hilbert space.
\begin{align}
    \ket{\Psi_{\mathrm{phys}}
    &=
    P\ket{\Psi_{\mathrm{mean}}(\chi_{ij})}
\end{align}
where $P$ is the projection onto 1 particle-per-site states.
This is called the Gutewillen projection.

Start with the Fock vacuum $\ket{0}$ and add fermions,
and take that inner product with the state.
\begin{align}
    \Psi_{\mathrm{spin}}^{\left( \chi_{ij} \right)}
    \left( \left\{ \alpha_i \right\} \right)
    &=
    \bra{0}
    \prod_{i} f_{\alpha i}
    \Ket{\Psi_{\mathrm{mean}^{(\chi_{ij})}}
\end{align}
Consider a different ansatz
\begin{align}
    \tilde{\chi}_{Ij}
    &=
    \chi_{ij} e^{i\left( \theta_i - \theta_j \right)}
\end{align}
which is the same as taking $f_i \to e^{i\theta_i} f_i$,
so then
\begin{align}
    \Psi_{\mathrm{spin}}^{\left( \tilde{\chi} \right)}
    &=
    e^{i\sum_i \theta_i} \Psi_{\mathrm{spin}}^{\left( \chi \right)}
\end{align}
and the only difference is the overall phase which is the same physical state.
It's incredible that this class of wave functions is so incredibly rich.
Fractional quantum hall states,
quantum spin liquids,
any exotic state you can write down parton construction,
and then do mean field and then do a projection.

We can not just get ground states,
but also variational states for excited states.
I can put some winding or fluctuation of $\chi$,
then do some projection and get some candidate excited wave function.

To do it more explicitly,
I can do a spinon wave function.

The spinon wave functions
\begin{align}
    \Psi_{\mathrm{spin}}^{\mathrm{spinon}}
    \left( i_1, \lambda_1; i_2, \lambda_2 \right)
    &=
    \bra{0}
    \left( 
    \prod_{i} f_{i,\alpha_i}
    \right)
    f_{i_1\lambda_1}^{\dagger}
    f_{i_2 \lambda_2}
    \ket{\Psi_{\mathrm{mean}}^{\chi}}
\end{align}
There's been a lot of work trying to write down new fancy wave functions using
neural networks and tensor networks and machine learning,
and PEPS and so on,
but these are the ones that actually work,
and describe everything.
All the popular ones people talk about but are not nearly as powerful as this
class of wave functions.
Those other classes of wave functions are fun to think about they have not been
as powerful so far as these wave functions.

You set up this mean field theory,
and you set up spins with spinons,
that gets yo to $U(1)$ gauge field,
and then you get wave functions by projection to the physical Hilbert space.
Now you consider all sorts of mean field theories.
$\chi$ and $a_0$ are the variational parameter.

\begin{question}
    Have there improvmentes by including gague flucationas?
\end{question}
You can improve the wave functions by including gague fluctuations.
It is an interesting reearch direction but I have seen very littel work.
One thing cold be to get better wave functions by doing exactly that.

At this point the name of hte game is to consider all sorts of mean field
ansatzs and describe all sorts of phases.

\section{$\ZZ_2$ spin liquid}
There's another we of arriving at very similar states in terms of projected wave
functions and mean field theory approach.
One way is to assume the spinons form an $s$-wave superconductor.
Assume when spinons form $s$-wave superconductor
\begin{align}
    \langle f_{i\uparrow} f_{i\downarrow}\rangle
    &\ne 0
\end{align}
The mean field wave function si going to be a mean field wave function for an
$s$-wave superconductor,
and then sou do a projection onto one particle per site.
This gives a physical wave function in the Hilbert space,
which si a $\ZZ_2$ spin liquid wave function.
When this parton model acquires a non-zero expectation,
there is an Anderson-Higgs gauge symmetry
and what's left behind is this $\ZZ_2$ gauge theory.
\begin{align}
    \ket{\Psi_{\ZZ_2 \mathrm{QSL}}}
    &=
    P
    \ket{\Psi_{\Psi_{mf,s\text{-wave s.c.}}}}
\end{align}

In parton mean field theory,
$U(1)$ gauge field plus $f_{i,\alpha}$.
And $U(1)$ breaks by Higgs to $\ZZ_2$.
The $\ZZ_2$ gauge field coupled $f_{i,\alpha}$.

Then we have topological excitations.
The first kind of quasiparticle $f$ is BdG quasi-particles of $s$-wave conductor
created by breaking ``cooper pairs`` of the neutral spin-half spinons.
Each one is fermionic spin-$\frac{1}{2}$ excitations.

The second kind.
Instead of writing the mean field state you insert a vortex nito this
$s$-wave superconductor.
The $\pi$-flux vortex of an $s$-wave superconductor after projection becomes a
vison,
which we talked about in the context of the quantum dimer model,
and in the context of the $\ZZ_2$ toric code vortex where this was the $m$
particle.

This $f$ is a fermion and this $m$ particle is a boson.
Every other topological class of excitations we could get from here.
We could consider composite $f$ and $m$ particles,
which gives what we would normally call the $e$ particle in the toric code
state.

The third kind of $f\times m = e$
When $f$ goes around $m$,
that picks up a minus sign,
I would pick up the spin of the $f$ which is minus $-1/2$,
and I would pick up a spin of the $m$ which is $+1/2$ and then the composite is
a boson.

The fourth kind is local excitations.
Even numbers of spinons.
You need more work to show that two visons coming together is local,
but I'll leave that for you to think about.
The whole point is that this is another way of arriving at this $\ZZ_2$ spin
liquid,
but there is something which didn't come up previously.

We see the fermion $f$ is a spinon that carries spin-$\frac{1}{2}$,
whereas the vison does not carry spin-$\frac{1}{2}$.
All particles in the toric code were spin-less,
so this differs from the quantum dimer model discussion and $\ZZ_2$ toric code
discussion due to the quantum numbers these excitations have under the $SO(3)$
rotations.
This didn't exist in the previous discussions.

It could have existed if we demanded the dimness had spin-$\frac{1}{2}$.
It's distinct from the toric code in that the spinons carry
spin-$\frac{1}{2}$,
so this whole state carries symmetry fractionalization.
All these particles carry fractional spin.

You can think of a spin-liquid as a superconducting state of spinons.
That's one way of thinking about a quantum $\ZZ_2$ spin liquid.

There is another important class of states is fractional quantum Hall states,
also known as fractional Chern insulators.
We can do it in terms of spins but we don't have to.
In terms of spins,
you can think of the boson $b$.
We don't necessarily there is full SO(3) full spin rotation symmetry.
We can write
\begin{align}
    b &=
    f_1 f_2
\end{align}
which si basically the same thing as writing the spins as
$S^\dagger &= f_1^\dagger f_2 = \tilde{f}_1 \tilde{f}_2$.

In this description $U(1)$ gauge theory goes
\begin{align}
    f_1 &\to e^{i\theta} f_1\\
    f_2 &\to e^{-i\theta} f_2
\end{align}
Each of these form some kind of insulating phase with Chern number 1.
Consider mean field state where $f_1,f_2$
are each in Chern insulator with $C_1=C_2=1$.
I want ot track the background external gauge field under which $b$ carries
charge 1.
So let $A$ be a background $U(1)$ gauge field,
not to be confused with the internal emergent gauge field that emerges from the
$U(1)$ gauge transformation.

\begin{table}[h]
    \centering
    \begin{tabular}{ccc}
        & $q_a$ & $q_A$\\
        $f_1$ & + 1 &  1\\
        $f_2$ & -1 & 0
    \end{tabular}
    \caption{z2spin}
    \label{tab:z2spin}
\end{table}
Imagine we have some effective Lagrangian
\begin{align}
    \mathcal{L}_b
    &=
    \mathcal{L}\left( f_1, a + A \right)
    +
    \mathcal{L}\left( f_2, a \right)
\end{align}
$f_2$ is a sate which si a Chern insulator with Chern number 1.
And it's coupled to this $a$ gague field.
What's an effective Lagrangian for such a csitaution.
You may not remember,
but we covered this before,
to introduce a $U(1)$ level Chern-Simons theory.
We can write the effective theory as
\begin{align}
    \mathcal{L}\left( f_2, a \right)
    &=
    -\frac{1}{4\pi}
    \alpha_{\mu} \partial_{\nu} \alpha_{\lambda}
    \epsilon^{\mu\nu\lambda}
    -
    \frac{1}{2\pi}
    \epsilon^{\mu\nu\lambda}
    a_{\mu} \partial_{\nu} a_{\lambda}
\end{align}
so then the current is
\begin{align}
    j_{\mu}^{\left( f_2 \right)}
    &=
    \frac{1}{2\pi}
    \epsilon^{\mu\nu\lambda}
    \partial_\nu
    \alpha_{\lambda}
\end{align}
So this is also a Chern-insulator 1 state.
\begin{align}
    \mathcal{L}\left( f_1, a + A \right)
    &=
    -\frac{1}{4\pi}
    \beta \, d\beta
    +
    \frac{1}{2\pi}
    \left( a + A \right)
    d\beta
\end{align}
If we only keep leading order terms in these gauge fields,
\begin{align}
    \mathcal{L}_b
    &=
    -\frac{1}{4\pi}
    \alpha\, d\alpha
    -
    \frac{1}{4\pi}
    \beta\, d\beta
    +
    \frac{1}{2\pi}
    a\, d\left( \beta - \alpha \right)
    +
    \frac{1}{2\pi} A\, d\beta
\end{align}
The whole point of hte parton mean field ansatz is that you can treat all the
gauge fluctuations perturbabitvely.
If you summarised in one sentence the parton constution.
Decomose the physcial operaotrs in temrso f p
Then aassume some mean field state,
then assume fluctuiatons are weak.

We assume the gague fluctuiaotns are greated perturbabilty,
and all subleadng terms are irrelevant.
And then we can just integrate out the $a$.
And because thsi si a free theory,
we can just solve the equation of motion for $a$.
Either way you're going to get the same thing.
\begin{align}
    \epsilon^{\mu\nu\lambda}
    \partial_\nu \beta_{\lambda}
    =
    \epsilon^{\mu\nu\lambda}
    \partial_\nu a_\lambda
\end{align}
which assuming it's not topologically weird space,
\begin{align}
    \beta_{\mu} &=
    a_\mu + \partial_\mu \phi
\end{align}
and we can just ignore the second term,
and we arrive at
\begin{align}
    \mathcal{L}_b
    &=
    -\frac{2}{4\pi}
    \alpha\, d\alpha + \frac{1}{2\pi}A\, d\alpha
\end{align}
and we saw this before.
This is $U(1)$ level 2 CS theory or $U(1)_2$.
And you will find the Hall conducatnce is
\begin{align}
    \sigma_H
    &=
    \frac{1}{2} \frac{1}{2\pi}
\end{align}
and so this ansatz describes a fractional quantum Hall state.

\begin{question}
    Why is that $1$ and $0$ is a choice in the charge table?
\end{question}
In ternral you could have $a_{A1}$ and $q_{A2}$,
but all you need is $q_{A1}+q_{A2}=1$.
In that case your Lagrangian will be
\begin{align}
    \mathcal{L}_b
    &=
    -\frac{2}{4\pi}
    \alpha\, d\alpha + \frac{1}{2\pi}A\, d\alpha
    +
    \frac{1}{2\pi}
    A\, \alpha
    \left( 
    q_{A1} \beta +
    q_{A2} \alpha
    \right)
\end{align}
But in consdensed matter physics,
people are find not worrying about global issues,
and you could just wirte $\frac{1}{2}$ and $\frac{1}{2}$,
but field theorists yell at me to write $1$ and $0$.
But the correct thing to do is do the field throeeists integer charge $1$ and
$0$.
You'll get the same answer for the Hall conducatnce eithere way.
They mess aorund with the global compactness of the gauge field.
If you don't worry about non-trivial topologies you won't worry about this
issue.

Now you can also write down the wave funciton.
Each mena field state is in some Chern number 1.

Now the projection.
We have two flavours of fermions.
\begin{align}
    \Psi\left( \left\{ r_i \right\} \right)
    &=
    P_{n_1=n_2}
    \Phi_{C=1}\left( \left\{ r_i^2 \right\} \right)
    \Phi_{C=1}\left( \left\{ r_i^1 \right\} \right)\\
    &=
    \left( \Phi_{c=1}\left( \left\{ r_i \right\} \right) \right)^2
\end{align}
If you're familiar with this,
this is the $\frac{1}{2}$-Laughlin FQH wave function.

\begin{question}
    ???
\end{question}
It's again the same thing,
you identify
$\ket{0}_b=\ket{00}$ and $\ket{1}_b=\ket{11}$ sa the physical states
and $\ket{10}$ and $\ket{01}$ as non-phsyical.

This parton construction is much better than the other approaches,
because it unifies quantum spin liquids and quanutm Hall states.

\begin{question}
    How did you get to the fact I need to consider 2 Chern nisulators?
\end{question}
In principle,
there's an infinite numbero f mean field ansatz options,
but why did I decide on the ansatz $b=f_1f_2$?
The parton construciton is a way of describing any phase with an emergent gauge
theory,
even if gapless.
When you do MF,
you can consider many ansatzs.
You can just pick one,
see what its energetics looks like,
and the one with least energy is the most likely.
You can consdier all sorts of mean field ansatz whcih gives you many variational
wave functions,a
nd that gives you many.

How did Laughlin know it's a square or cube of the free fermion wave function?
It was just his guess based on his experience.

\begin{question}
    How general is this?
    If you write $b=$ 3 fermions?
\end{question}
You could do in general $b=f_1f_2\cdots f_k$.
the gauge symmetry is different,
but you can still do it.
You end up with $U(1)_k$ CS theory,
if you assume all are in $C=1$ Chern-insulator states.

\begin{question}
    Math procedure for free fermion model and mean field ansatz to output some
    topolgocial orderd satte.
\end{question}
There probably is one,
but it hasn't been elucidated.
Perhaps there's something more formal that can be developed.

\section{Summary}
I want to give you an overview of what we covered.
A brief review of what we've done and why it all fits together.
There's a lot more things to cover for another course.
There are important things we didn't cover,
like the algebraic theory of anyons,
this modular tensor category.
I can give another lecture about modular tensor categories,
but since the lectures for the course are over we don't technically have to do
that.
Are you interested in a bonus lecture on Wednesday?

Let me try to take 10 minutes to give you a summary or overview.

We start off by talking about general htings,
like locality in quantu mssytesm,
how gapped implies correlation length.
We didn't directly do naything wiht these,
but we talked about Hilbert spaces decompsoing into local tensor products.
We did use finite correlation lneght in some places.

Then we described TQFTs with path integrals $Z\left( M^{d+1} \right)$,
but the important point is that topologicla phases of matter are believed to be
in 1-1 correspondtance to deformaiton clases of TQFTs.
An important distinction was the notion of invertible vs non-invertible.
Invertible in this contextz means that $|Z|=1$ 
and non-invertible menas $|Z|\ne 1$.
Invertible means unique ground state,
but non-invertible means degenerate ground states.

We did some examples of TQFTs.
They came in two flavours.
We talked about this,
but there was $U(1)_k$ CS theory,
which came in two flavours.
We could think of this as a response theory,
meaning that we have a background gauge field
$Z\left( M^3, A \right) = Z\left( M^3, 0 \right) e^{i k S_{CS}[A]}$.
This can describe Chern insulator integer quantum Hall states,
in other words invertible topological phase.

Then the way of thinking about it is that we have a dynamical gauge theory.
Here,
we're actually integrating over all gauge field configurations.
\begin{align}
    Z\left( M^3 \right) &=
    \int \mathcal{D} a\,
    e^{ikS_{CS}(a)}
\end{align}
and this describes non-invertible topological phases,
like fractional quantum Hall states,
quantum spin liquids.
One of the problems is thinking about the toric code in terms of CS theories
with 2 by 2 matrices.

It depends on whether you treat the gauge field as background or dynamical.

Another example is SPT path integrals, or Digraph-Witten theory,
where you pick a cohomology class inside a cohomology group
\begin{align}
    \left[ \nu_{d + 1} \right]
    \in
    H^{d + 1}
    \left( G, U(1) \right)
\end{align}
You can think of it as an invertible TQFT
\begin{align}
    Z\left( M^{d+1}, A \right)
    &=
    e^{i\int_M A^* \nu_{d+1}}
\end{align}
We triangulated our manifold,
and defined group elements on links,
and the path integral value was always a phase.
This describes SPTs.
One utility of this construction was that it gave exact wave functions,
and it gives parent Hamiltonians,
a full solvable model to describe a huge class of SPTs.
And remember SPTs are a subset of invertible phases.
An invertible phase is a subset that is invertible on all manifolds.
In a ground states of an invertible topological state,
it has an inverse state that if you stack them can adiabatically go to a trivial
state.
If you don't care about the symmetry,
you can always adiabatically connect it to a trivial state,
but not if you want to respect the symmetry.

Then here what we can do is consider non-invertible TQFTs,
\begin{align}
    Z(M) \propto
    \sum_{[a]} e^{i \int a^* \nu}
\end{align}
where the sum of over all flat gauge fields.
If you take $G=\ZZ_2$ and $[\nu] = [0]$,
then this describes the $\ZZ_2$ spin liquid,
which is the toric code.

There's one other example that we did,
which was the TQFT for the $(1+1$-D Majorana.
We had a path integral over a 2D manifold with a 2-dimensional spin structure.
\begin{align}
    Z\left( M^2, S \right) &=
    \left( -1 \right)^{\Arf(S)}
\end{align}

And then we talked about concrete examples of topological phases.

We had the $(1+1)$D Majorana.
Then we talked about the $(2+1)D$ Chern insulator,
TRI TI and $p+ip$ superconductor.

The invertible free fermion model.
Time-reversal invariant topological insulator.

We talked about the $(3+1)$D TRI TI.

Then we tlked about boson SPTs,
including the AKLT construction,
and group cohomology models.

Then we talked about non-invertible topological phases,
which includes quantum spin liquids,
including quantum dimer models,
parton construction,
lattice gauge thoery and toric code.

FQH state $U(1)_k$ CS theory by parton construction.

Thep oint of the anyon pahses was topological excitations wiht fractional
statistics.
We had topological ground state degeneracies that depended on the topology of
the manifold and so on.
In practice,
this class is a survey of the huge classes of topological phaes.
It's a deep and rich subject.
In the next bonus lecture,
I'll tell you about non-Abelian topological phaes and the algebraic theory.
