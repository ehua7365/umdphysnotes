\section{Mean-field approximation for Ising Model}
This is the main equation
\begin{align}
    h_{\mathrm{eff}}
    &=
    h
    +
    4J T gh \beta h_{\mathrm{eff}}
\end{align}
The critical temperature
\begin{align}
    k_B T_c &= 4J
\end{align}
The susceptibility
\begin{align}
    \chi
    &=
    \frac{1}{\beta}
    \left.\frac{\partial m}{\partial h}\right|_{T=T_c, h=0}
    \sim
    \left| T - T_c \right|^{\overbrace{-1}^{\gamma}}
\end{align}
and the magnetisation is
\begin{align}
    m &\sim
    \left| T - T_c \right|^{\overbrace{1/2}^{\beta}}
\end{align}
I can then write
\begin{align}
    \underbrace{h_{\mathrm{eff}}}_{h + 4Jm}
    &=
    h
    +
    4 J T gh \beta h_{\mathrm{eff}}
\end{align}
which implies
\begin{align}
    m &=
    Tgh \beta \left( 4J m + h \right)
\end{align}
so
\begin{align}
    4J m + h
    &=
    \frac{1}{\beta_c}
    \arctanh m\\
    &\approx
    \frac{1}{\beta_c}
    \left( 
    m
    +
    \frac{m^3}{3}
    + \cdots
    \right)
\end{align}
so then
\begin{align}
    h \sim m^{\overbrace{3}^{\delta}}
\end{align}
which is not obvious at all.

The relation between $m$ and $h$ depends on that equation,
but depends only on the fact that this s an odd function and that it is smooth.
This is very robust because even though it's only approximate.

There are perfect analogies of mean field all over physics.
In fact you've seen those before.
So I'm going to erase this and I'm going to make some drawings in two different
colours.
On the left we have the mean field approximation,
and on the right I'm going to tell another thing that you already know,
which are fluids that obey the van der Waals equation of state.

\subsection{Ising}
I'm going to draw the external magnetic field vs the temperature.
This is $T_c$.
What happens when I'm above $T_c$ and I have a negative $h$?
It favours spin down so magnetisation is going to be down.
If $h>0$,
then up is favoured.
If the magnetic field is zero,
then it's zero.
So nothing interesting happens here when I cross $h=0$ for $T>T_c$.

Now below $T_c$,
the magnetisation doesn't go to zero.
When $h$ is slightly negative,
they are all spin down.

The discontinuity disappears in the critical region.

The analogy with the van der Waals fluid is
\begin{align}
    h &\sim P\\
    m &\sim \frac{1}{V}
\end{align}

[lots of pictures]

In a lot of systems,
there is an order parameter $\phi$ such that its average value is
\begin{align}
    \begin{cases}
        \langle \phi \rangle
        =
        0 & T > T_c\\
        \langle \phi \rangle
        \ne
        0 & T > T_c
    \end{cases}
\end{align}
If there is such a situation,
then there is spontaneous symmetry breaking.
The order parameter doesn't have the symmetry of the Hamiltonian,
because when it flips it flips sign,
but the Hamiltonian doesn't change.
The thermal average is not zero,
even though it should be naively if you look at the Hamiltonian.

A lot of cases,
when the expectation picks up a non-zero expectation value like this,
frequently this means spontaneous symmetry breaking.

It happens even when equal to zero.

It's worth while going close to the critical point and looking at the
configuration that contributes the most.
The ones with smallest energy are ordered and there are few of them.
They are not typical.
The more typical ones are the high energy ones.

I'm going to try to draw some configurations that are critical.

[picture]

When above the critical temperature,
the typical size of clusters $\xi\approx a$,
the lattice size.
When at $T=T_c$,
you're going to get clusters of all sizes,
from the size of a lattice spacing to large clusters that go from one corner of
the system to another,
and everything in between.

So we talk about the critical exponents in the mean field model.
I want it to be the centre piece of everything we compute here.
It turns out that close to the critical point,
the heat capacity diverges.
\begin{align}
    C &\sim \left| T - T_c \right|^{-\alpha}
\end{align}
Let's compute $\alpha$.

The magnetisation when you approach the critical temperature is like
\begin{align}
    m ~ \left| T - T_c \right|^{\beta}
\end{align}
and the susceptibility
\begin{align}
    \chi &=
    \left| \frac{\partial m}{\partial h} \right|_{h=0}
    \sim
    \left| T - T_c \right|^{-\gamma}
\end{align}
Suppose we calculate the correlation
about how close or far spins influence each other
\begin{align}
    \left\langle
    \sigma_0
    \sigma_{0 + R}
    \right\rangle
    &\sim
    e^{-R/\xi}
\end{align}
and this parameter $\xi$ is the correlation length,
which is measurer f how far you need to go to not get correlation.
The correlation length diverges too at the critical temperature.
\begin{align}
    \xi &\sim
    \left| T - T_c\right|^{-\nu}
\end{align}
I didn't show you but $\nu=\frac{1}{2}$ for mean field Ising.

I can take a Fourier transform.

This is the only integral you need to do in your head
\begin{align}
    \sum_{\vec{R}}
    \left\langle
    \sigma_0
    \sigma_{0 + R}
    \right\rangle
    e^{i\vec{k}\cdot\vec{R}}
    \sim
    \frac{1}{k^{2 - \eta}}
\end{align}
where $\eta$ is called the \emph{anomalous dimension}.

\section{Universality classes}
Numerical experiments and exact results.
The Ising model in 2D was solved exactly on paper.
First surprise,
it's not what mean field theory says.
It maybe shouldn't surprise you much,
because mean field was a bit hand waving anyway.

But the surprising thing was a different one.
It wasa big surprise back them.

There was a theory called Landau mean field theory,
a lot of hand wavin,
but it's Landau hand waving,
and he talks to God directly.
It's like mean field theory on steroids.
Tremendous theory,
no argument whatsoever.

Then you find something interesting with the exact solution.
There are what we call Universality classes.

People looked at the 2D Ising model.
People made this relation between this and the liquid phase transition.
You can measure the critical exponents of water evapoerating,
and you find it's 3D Ising model.
One is a stupid magnet model,
one is a real system.
They agree within error bars.
But then you look at other systems,
and there are sets of critical exponents that appear frequently.

The critical exponents of the 3D Ising model are common,
probably the most common one.
And then there are other common ones.
The systems are completely diffrent.
In the notes,
I give some critical exponents calculated from numerics,
conformal bootstrap.
Let me just write these here.
\begin{table}
    \centering
    \begin{tabular}{cccc}
        & 3D Ising & CO$_2$ & Ni Magnet\\
        $\alpha$ & 0.11008(1) & 0.10 & 0.0(?)\\
        \vdots\\
        $\delta$ & 4.28984(1) & 4.20 & 4.22
    \end{tabular}
    \caption{Critical Exponents Theory vs Experiment}%
    \label{tab:criticalexponents}
\end{table}
There's an uncertainty for Ni magnet that I don't know.
All systems in nature seem to fall into 

Here are some common universality classes
\begin{itemize}
    \item 3D Ising
    \item 2D Ising
    \item Mean field
    \item Kosterlitz-Thouless
    \item 3-Potts model
\end{itemize}
I knew Thouless.
A lot of things come from experience.
People get data and they see a pattern.
Roughly the pattern people see is the following.

The first thing you have to discern is the dimensionality of the state.
The other one is more subtle,
it's the ``pattern of spontaneous symmetry breaking''.
For example,
consider the Ising model.
The Hamiltonian has a symmetry $\mathbb{Z}_2$
which is basically spin flip.
This symmetry has 2 elements,
do nothing and flip all the spins.
The point is that at low temperatures,
the magnetisation,
the order parameter,
is not zero.
It's pointing either up or down.
When I did a spin flip,
I get a different state.
What's the symmetry of the sate below the critical temperature?
That's what I mean by a matter of symmetry breaking,
how it's broken to the final state
$\mathbb{Z}_2 \to \textrm{nothing}$.

Another model is the Heisenberg model,
every point on a lattice has a spin-$\frac{1}{2}$ on it.
You can make it ferromagnetic or antiferromagnetic or whatever.
The point is that it's not just up or down,
it could be many other states you can rotate by.

The dot product between two spins stays the same under rotation.
People call this SO(3),
which are 3 by 3 orthogonal matrices.
The special orthogonal matrices.
Also known as rotations.
You don't know this,
but the ferromagnetic model,
the ground state is all spins pointing in the same direction,
because that maximizes the dot product and minimized the energy just like thei
Ising model.

Suppose this direction is picked up,
there's still some symmetry left.
If I do rotations around this axis,
the magnetisation order parameter doesn't change.
$SO(3) \to O(2)$.

There is a particular type of liquid crystal called nematic liquid crystal.
How can something be a liquid and a crystal at the same time?
It's made of long molecules,
and the head and tails are different.
It turns out below a certain critical temperature,
they organize themselves like this.
Their position is not organized on a crystal,
like a liquid,
not organized in a nice pattern.
Their direction is not organized either,
there are as much head up and head down.
The only thing is that they are all aligned that way.
The Hamiltonian has a symmetry that they can rotate any direction they like.
Here there's nothing to point,
the order parameter is a tensor.
But it can still flip.
\begin{align}
    SO(3) \to O(2)\times \mathbb{Z}_3
\end{align}
Rotation goes to 2D rotation and flips.

Meanwhile there are other liquid crystals
where the centres of molecule positions are fixed by the orientation are not.

Remember the Bose gas,
which has a phase transition,
a condensate.
I claimed that it's superfluid.
I'm not even going to attempt to explain.
The pattern of symmetry breaking in that case is $U(1)$.
That's exactly the same group as $O(2)$ by the way.
This is broken to almost nothing.
This phase has to do with changing the wave function by a phase
$\psi \to e^{i\beta} \psi$.
Superfluid helium and BEC follow this pattern,
and they have their own universality class.

You probably heard something that if you look at the physics of elementary
particles,
there is a symmetry
\begin{align}
    SU_c(3)
    \times
    SU_w(2)
    \times
    U_Y(1)
    \to
    SU_c(3)
    \times
    U_{EM}(1)
\end{align}
If you have 2 systems with the same dimension,
they follow the same universality class.
That was an experimental observation.

Examples.

Consider the Ising model in 2D
\begin{align}
    H &=
    -J \sum \sigma_i \sigma_j
\end{align}
consider my made up Paulo model
\begin{align}
    H^P &=
    -J \sum_{\textrm{plaquette}}
    \sigma_{i} \sigma_{j} \sigma_{k} \sigma_{l}
\end{align}
They have the same dimensionality and the same pattern of spontaneous symmetry
breaking so they should have the same critical exponents.
Same for the sum
\begin{align}
    H &=
    -\tilde{J} \sum \sigma_i \sigma_j
    -J \sum_{\textrm{plaquette}}
    \sigma_{i} \sigma_{j} \sigma_{k} \sigma_{l}
\end{align}
