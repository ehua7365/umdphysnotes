\section{Background Gauge Fields}
Let $P$ be the fermion parity operator.
There is a fermion parity symmetry
\begin{align}
    P \gamma_i P = -\gamma_i
\end{align}

Consider a $\mathbb{Z}_2^f$ background gauge field with $s=\pm 1$.
Let there be some centired Majoranas $\gamma_{i,\alpha}$
where $i$ labels site location
and $\alpha$ labels flavour.

In one dmiensions the Hamiltonian is
\begin{align}
    H &= \frac{i}{4}\sum_{i,j}
    M_{ij}^{\alpha\beta} \gamma_i^\alpha \gamma_j^\beta
\end{align}

You can introduce a $\mathbb{Z}_2$ gauge field so than whenever you hop from $i$
to $j$,
there is a $\mathbb{Z}_2$ string from $i$ to $j$.
\begin{align}
    H[s] &=
    \frac{1}{4}\sum_{i,j}M_{i,j}^{\alpha\beta}
    \gamma_i^d \gamma_i^\beta
    \prod_{i=0}^{j - 1} S_{k,k + 1}
\end{align}
This is the $\mathbb{Z}_2$ analogue of say $U(1)$ electron symmetry
where you would take an $e^{i\in_1^2 A\, dl}$.

\begin{question}
    Why do we have multiple fermion flavours?
\end{question}
For example we could have two fermions on each site,
and furthermore if these had spin,
these would also have spin.

Back to the spinless particle superconductor,
\begin{align}
    H[s] &=
    \underbrace{
        \frac{1}{2}\sum_{i}(1? + t) \gamma_{2i}\gamma_{2i + 1} s_{i,i+1}
    }_{\text{intersite}}
    - \underbrace{\mu \gamma_{2i - 1}\gamma_{2i}}_{\text{intersite}}
\end{align}
You can write
$s_{i,i+1}=e^{i\pi \sigma_{i,i+1}}$
where $\sigma\in\left\{ 0, 1 \right\}$.
There are $\mathbb{Z}_2^+$ gauge transforms for
$\lambda_i\in\left\{ 0, 1 \right\}$
and we get 
\begin{align}
    \sigma_{i, i + 1} &\to
    \left(
        \sigma_{i, i+1} + \lambda_{i + 1} - \lambda_i
    \right)\mod 2\\
    \gamma_{2i + 1} &\to
    (-1)^{\lambda_i}
    \gamma_{2i + 1}\\
    \gamma_i &\to
    (-1)^{\lambda_i} \gamma_{2i}
\end{align}
I want to emphasize the effect of changing boundary conditions from periodic to
anti-periodic results is equivalent to inserting $\pi$ net flux through the
ring.

\begin{example}
    For periodic boundary conditions $c_{i + N} = c_i$
    where $c$ is complex fermion.
    Singular gauge transform.
    Define
    \begin{align}
        \tilde{c}_i &=
        c_i \prod_{k=1}^{i} s_{k,k+1}
    \end{align}
    where
    \begin{align}
        \prod_{i=1}^{N} s_{i,i+1} = -1
    \end{align}
    then
    \begin{align}
        \tilde{c}_{i+N} &=
        c_{i + N} \prod_{k=1}^{N} s_{k, k+1}\\
        &=
        \underbrace{c_i \prod_{k = 1}^{i} S_{k, k + 1}}_{\tilde{c}_i}
        \underbrace{\prod_{k=i + 1} s_{k, k + 1}}_{-1}\\
        &=
        - \tilde{c}_i
    \end{align}
    But for periodic boundary conditions and $s=1$,
    we have anti-periodic boundary conditions with $\pi$ flux through ring.
\end{example}

\section{Topological invariant}
non-trivial phase.
Suppose you have a ring and a gap in between through close together to allow the
zero modes to interact.
The hamiltonian is
\begin{align}
    H_{bdy} &= \tilde{t} i \gamma_{L}\gamma_{R}
\end{align} 
Depending on the sign of $\tilde{t}$,
the ground state has either even or odd fermion parity.

Changing the sign of $\tilde{t}$ is equivalent to
inserting $\mathbb{Z}_2^f$ flux thorugh the ring,
which is equivalent to chnaging boundary conditions.
It is also equivalent to changing fermion parity of ground state.

Trivial phase:
No Majorana Zero Mods,
so changing boundary conditions
doesn't chnage ground state parity.

Let us define the topological invariant
\begin{align}
    \mathcal{I} &=
    P_P P_A
\end{align}
where $P_P$ is the fermion parity of the ground state with periodic boundary
conditions and $P_A$ is the parity with anti-periodic boundary conditions.

The claim is that
\begin{align}
    \mathcal{I} &=
    \begin{cases}
        -1 & \mathrm{topological phase}\\
        1 & \mathrm{trivial phase}\\
    \end{cases}
\end{align}
In spinless $p$-wave supercondutors,
\begin{align}
    P_A = + 1, \qquad P_P = -1
\end{align}

\subsection{Relation to free bond invariant}
For free fermions,
the topological invariant is
\begin{align}
    \mathcal{I} &=uuuuu
    P\left( H_P \right) P\left( H_A \right)
\end{align}
where
\begin{align}
    P(H) = \frac{i}{4} \sum_{i} \gamma_i \gamma_j A_{i,j}
\end{align}
where $P(H)$ is given by the Pfaffian
\begin{align}
    P(H) &= \textrm{sgn}(\textrm{Pf}(A))
\end{align}
Go into $k$-space the Hamiltonian becomes
\begin{align}
    H &=
    \frac{1}{2} \sum_{k}
    \Psi_k^\dagger H_k \Psi_k
\end{align}
where
\begin{align}
    \Psi_k &=
    \begin{pmatrix}
        \Psi_k\\
        \Psi_{-k}^\dagger
    \end{pmatrix}
\end{align}
For $k=0$,
$H$
only 
involves $\Psi_i\Psi^\dagger$.
\begin{align}
    H &=
    \frac{1}{2} \sum_{k = 0, \pi}
    \Psi_k^\dagger
    H_k\Psi_{k}
    + \frac{1}{2}\cdots
\end{align}

At the end of the day,
the fermion parity for odd length ring
$k=2\pi n/N$ periodic boundary conditions
$k=0$, with $k=\pi$ so
\begin{align}
    k &= \frac{2\pi}{N}\left( n + \frac{1}{2} \right)
\end{align}
with antiperiodic boundary conditions.

Three is no $k=0$,
but there is $k=\pi$.
\begin{align}
    I = S_0 S_\pi = \nu.
\end{align}

For even length ring,
$k=0,\pi$ allowed by periodic boundary conditions,
but neigher is allowed for anti-periodic boundary conditions.
Then we have
\begin{align}
    I = S S_{\pi} = \nu
\end{align}

\subsection{TQFT}
When you have fermions and wnat TQFT,
you want spin TQFt.
For every $d$-dimensional manifold,
we have a vector space.
We have 2 possible spin structure, $+$ periodic and $-1$ anti-periodic.
In both cases we know we have a unique ground state.
\begin{align}
    V(S^1, +)
    \simeq
    V(S^1,-)
    \simeq\mathbb{C}
\end{align}
Let's try to figure out the path integral on $T^2$
with our choice of spin structure $\eta$.
I can think of the two directions of the torus as time and spcae.
So I have $x,t$,
and the spin structure $(x,t)$
can have boundary conditions $(++)$, $(+-)$, $(-+)$ or $(--)$.

When you have a system of verimions,
you can write it like a propagator
\begin{align}
    Z &= \bra{j} e^{iHt} \ket{i}
\end{align}
which is the coherent sate path integral.
In the coherent state path integral for fermions,
when you do the derivation,
the fermions have naturally have anti-periodic boundary conditions.
It's counterintiutive,
because you'd think periodic boundary conditions are what you should have,
but it turns out anti-periodic is natural.
The path integral for the TQFT is just the trace for the identity operator.
We have anit-periodic boundary conditions for the boundary operator.
This is equal to
\begin{align}
    Z(T^2, +-) &= \Tr_{V(S^1,+)}\mathbf{1} = 1\\
    Z(T^2, -+) &= \Tr_{V(S^1,-)}\mathbf{1} = 1
\end{align}
Remember when discussing TQFt,
if you're gluing the time boundary,
and you're twisting the boundary condition,
that is inserting flux,
which is the fermion parity operator.
\begin{align}
    Z(T^2, ++) &= \Tr_{V(S^1,+)}P = -1\\
    Z(T^2, --) &= \Tr_{V(S^1,-)}P = +1
\end{align}
You can choose the convention the $-1$ case if periodic and the $+1$ case is
anti-periodic.

Now you can ask what is the path integral in a closed genus $g$ surface
$\Sigma_g$.

We have a genus $g$ surface and a spin structure $\eta$.
We should get a $\pm 1$ because there are only two phases.
There are canonical invariants associated with spin surfaces
and it's callled the Arf invariant of the spin structure.
\begin{align}
    Z\left( \Sigma_g, \eta \right) &= (-1)^{\mathrm{Arf}(\theta)}
\end{align}
Arf invariants are associated with quadratic forms.
Definition of Arf.
Suppose $V$ is a vector space over $\mathbb{Z}_2$.
Let $\phi: V\times V\to \mathbb{Z}_2$ be a map
that is a non-degenerate symmetric bilinear form.
Then suppose you have a quadratic form $q: V\to \mathbb{Z}_2$
that satisfies by definition
\begin{align}
    q(x + y) &= q(x) + q(y) + \phi(x, y)
\end{align}
and $q\in Q(V, \phi)$ is an element in the space of all quadratic forms.
Then the Arf invariant is defined as $\mathrm{Arf}:Q\to\mathbb{Z}_2$
by
\begin{align}
    (-1)^{\textrm{Arf}(q)} &=
    \frac{1}{\sqrt{|V|}}
    \sum_{\lambda\in V} (-1)^{q(x)}
    = \pm 1
\end{align}
There is a theorem which I will not prove.
\begin{theorem}
    Spin structures on genus $g$ surfaces in 1-1 correspondence with elements of
    \begin{align}
        Q\left(
        H_1\left( \Sigma_g, \mathbb{Z}_2 \right),
        \underbrace{\phi}_{\text{intersection form}}
        \right)
    \end{align}
\end{theorem}
The intersection form takes 2 loops and tells you whether they intersect.
Given a spin structure,
you have a quadratic form like this,
and then you can evaluate the Arf invariant.
I can't give you an explanation of this,
that would be far beyond what we want.

\begin{question}
    What is a genus $g$ surface?
\end{question}
You should have stopped me earlier!
Surfaces are classified by their genus,
which is basically the number of holes they have.
Sphere has $g=0$,
Torus has $g=1$,
and if you have 2 holes then $t=2$.
The Euler characteristic of a genus-$g$ surface is
\begin{align}
    \chi(\Sigma_g) = 2 - 2g
\end{align}

\begin{question}
    What is $Q(V,\phi)$?
\end{question}
It is the space of all quadratic forms on $V$,
functions which satisfy the definition with $\phi$ as the form.

\begin{question}
    What is $H_1$?
\end{question}
$H_1$ is the space of formal linear combinations of loops.
OK too complicated.
Let's just say it's the space of loops modulo the space of contractible loops.
I'm not going to give you a rigorous definition.
I'll just define by example.

If I give you a surface, (genus 2)
there are 4 independent loops you can draw,
which are called $x_1,y_1,x_2,y_2$.
Then you can take formal linear combinations of them like
\begin{align}
    C = \alpha x_1 + \beta x_2 + \gamma y_1 + \delta y_2
\end{align}
where $\alpha,\beta,\gamma,\delta\in \mathbb{Z}_2$
and then this linear combination
$C\in H_1(\Sigma_g, \mathbb{Z}_2$.

Then you can have forms $a,b\in H^1(\Sigma_g, \mathbb{Z}_2)$
which map from $H_1\to \mathbb{Z}_2$.

So $H^1$ are linear combinations of loops (cohomology),
but $H_1$ are the actual loops themselves (homology).

Given a loop $X$,
you can define
\begin{align}
    \int_X a = \pm 1
\end{align}

I should have just said $x$ intersects $y$.

You can go to the torus,
first enumerate all possible functions from the loops on the torus to
$\mathbb{Z}_2$,
there are 16 of them,
then find all possible functions like this quadratic form,
there are 4 of them,
and then take the corespondence with periodioc and anti-periodic.

\begin{question}
    What is the Arf invariant invariant under?
\end{question}
I don't know.
It's invariant under all mapping class diffeommorphisms on the surface and the
Arf invariant will still be invariant.

To fully specify the TQFT,
I need to give you more information.
I need to specify the path integral on manifold with boundary.
for example,
I can tell you what the path integral on a disk $D^2$ is
(like a hemisphere with axis in the time direction).
\begin{align}
    Z(D^2) \in V(S^1, -)
\end{align}
On $S^1$,
two spin structures, $\pm 1$.
Only $-$ can be the boundayr of the disk.
In path, $-1$ is called the bounding spin structure
and $+$ is called the non-bounding spin structure.

The way ot think about spin structure is this.
You QFT,
a fermion is described by a fermion,
and you want a framing of the worldline of hte fmrion,
which goes on a loop,
and you want a frame for the world line.
Thereare two frames.
There's one that goes like this,
pointing orthogonal to the line.
Or you could consider one like this, all pointing in a direction.
The first is no-bounding,
the second one is bounding.

As you go along this second line,
and you unfold it,
as you go along, it kind of twists $2\pi$.
And a $2\pi$ rotation gives you a $-1$ for fermions.

Whereas in the first case, there is no twisting.
That's why the anti-periodic boundary condition gives something that can be a
boundary.

This comes up in a lot of sitautions with topological phases and fermions.
If you are confused why anit-periodic is showing up when it is,
this is why I think it's useful.
I haven't given you a deep understanding of what framing is rigorously,
but it's whether it twists or not relative to its worldline and whether the
boundary should be periodic or not.

In QFT fermions are defined in $SO(1,d)$.
If space-time is 2D,
you should have a 2D frame field,
and the two components are the frame field and the tangent ot the curve.
Relativisitc QFT somehow winds up giving the correct language for TQFT aspects
of things.
And somehow the condensed matter stuff
ends up fitting well into TQFT.

Relativistic fermions are spinors of $SO(3)$,
but if they're relativistic Minkowski space,
it's $SO(1,d)$.

Microsopically,
the fermions don't need this structure,
but when doing TQFT with fermions,
you have spin structures,
and a way to think about spin structures is to think of relativistic
fermions.

Boundary needs
\begin{align}
    Z\left( M^{d+1}, \eta \right) &=
    V\left( \partial M^{d + 1}, \eta_{\partial M^{d+1}} \right)
\end{align}

I will post the paper for reference.

The reason I'm telling you this,
is I want to show you wnat structure is giving you this TQFT,
then I'm going ot guess what kind of structure classifies all kinds of
topolgoical atter.
One of the goals is how do we classify all phases of topological matter.
One of thoese asnswers is in terms of TQFT.
We're going to see how to think about phases fro mthe persepective of TQFT,
and know what to expect how to classify phases.
