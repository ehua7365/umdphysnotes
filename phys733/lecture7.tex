\section{Majorana Fermions}
Define $\nu=S_0 S\pi$ to be a topological band invariant.
$\nu=+1$ is the non-topological phase.

The band looks like $E_k = -2t\cos k$,
which mas a minimum of $\mu = -2t$ and a maximum of $\mu= 2t$.

$\nu = -1$ is the topological phase.

Majorana fermions obey
\begin{align}
    \left\{ \gamma_i, \gamma_j \right\} &= 2\delta_{ij}
\end{align}
which implies
\begin{align}
    \gamma_{i}^2 = 1.
\end{align}
Here the operators
$\gamma_i$ act on sites $i=1,\ldots,2N$.

Let $\tilde{\gamma_i}:=\sum_{j} W_{ij}\gamma_j$
where $w\in O(2N)$ is an orthogonal matrix.
Then you still get
\begin{align}
    \left\{ \tilde{\gamma}_i, \tilde{\gamma}_j \right\} = 2\delta_{ij}
\end{align}

Thus the largest possible symmetry of $2N$ Majoranas is
$O(2N)$.

One may define creation and annihilation operators
$c_i$ and $c_i^\dagger$ for ``complex fermions''
which obey
\begin{align}
    \left\{ c_i^\dagger, c_j \right\} &= \delta_{ij}\\
    \left\{ c_i, c_j \right\} &= 0.
\end{align}
The two can be related by
\begin{align}
    c_i &= \frac{1}{2}\left(
        \gamma_{2i - 1} + 1\gamma_{2i}
    \right) 
\end{align}
where $i=1,\ldots,N$.

Thus $2N$ Majoranas is equivalent to $N$ complex fermions.

$N$ complex fermions has $2^{N}$ states,
since the number operator $c_i^\dagger c_i = 0$ or $1$.

So we can say that 1 complex fermion has a
``quantum dimension'' of 2,
but 1 Majorana fermion has a quantum dimension of $\sqrt{2}$.

Let us introduce the fermion parity operator
\begin{align}
    P_i = {\left( -1 \right)}^{c_i^\dagger c_i}
    = -i \gamma_{2i - 1}\gamma_{2i}=
    \begin{cases}
        +1\\
        -1
    \end{cases}
\end{align}
From this we can also define the total fermion parity
\begin{align}
    P &= \prod_{i=1}^{N} P_i.
\end{align}
Under $O(2N)$ rotation this transforms as
\begin{align}
    P \to P\det W
\end{align}
so $SO(2N)$ fixes $P$.

Consider just 1 pair with
\begin{align}
    \gamma_{1} &= c + c^\dagger\\
    \gamma_2 &= \frac{1}{2}\left( c - c^{\dagger} \right)
\end{align}
and such as system only has two states
$\ket{0}$ for empty and $\ket{1}$ for filled.
Then the Majorana operators act like
\begin{align}
    \gamma_1\ket{0} &= \ket{1}\\
    \gamma_2\ket{0} &= i\ket{1}
\end{align}
so in fact
\begin{align}
    \gamma_1 &=
    \begin{pmatrix}
        0 & 1\\
        1 & 0
    \end{pmatrix}
    = \sigma^{x}\\
    \gamma_1 &=
    \begin{pmatrix}
        0 & -i\\
        i & 0
    \end{pmatrix}
    = \sigma^{y}
\end{align}
and their product satsifies
\begin{align}
    -i \gamma_1 \gamma_2 = -i^2 \sigma^z = \sigma^z.
\end{align}
The most general free ferion Hamiltonian assuming there being
$2N$ Majoranas is
\begin{align}
    H &= \frac{i}{4} \sum_{ij} A_{ij} \gamma_i \gamma_j
\end{align}
where $A_{ij} = - A_{ji} = A_{ij}^*$
so $A$ is a $2N\times 2N$ real skew symmetric matrix.

We can always bring $A$ into canonical form
\begin{align}
    D &= WAW^\dagger =
    \begin{pmatrix}
        0 & \varepsilon_1 & & & & & \\
        -\varepsilon_1 & 0 & & & & & \\
        & & 0 & \varepsilon_2 & & & \\
        & & -\varepsilon_2 & 0 & & &\\
        & & & & \ddots & &\\
        & & & & & 0 & \varepsilon_N\\
        & & & & & -\varepsilon_N & 0\\
    \end{pmatrix}
\end{align}
so we can set $\tilde{\gamma} = W\gamma$
and
\begin{align}
    \tilde{c}_j &= \frac{1}{2}\left( \tilde{\gamma}_{2j} - 1
    + \tilde{\gamma}_{2i}\right)
\end{align}
The Hamiltonian in this form is
\begin{align}
    H &=
    \frac{i}{2}\sum_{j=1}^{N}
    \varepsilon_j \tilde{\gamma}_{2j - 1} \tilde{\gamma}_{2j}\\
    &=
    \sum_{j=1}^{N} \varepsilon_j\left( \tilde{c}_j^\dagger \tilde{c}_j -
    \frac{1}{2}\right)
\end{align}
We can then do the so-called ``particle-hole transformation''
\begin{align}
    C: \quad
    \tilde{c}_j^\dagger &\to \tilde{c}_j\\
    \tilde{\gamma}_j &\to {(-1)}^{j+1}\tilde{\gamma}_j
\end{align}
which also can transform the Hamiltonian under conjugation with
\begin{align}
    C:\quad
    H\to CHC = -H.
\end{align}
This means tthat eigenvalues must come in pairs $(E, -E)$.

If all $\varepsilon_i \ne 0$
then the lowest energy state is uniqued and gapped.


Consider the fermion parity of the ground state.
Consider a single pair
\begin{align}
    H &= \frac{i}{2} \sum \gamma_1 \gamma_2
\end{align}
which for the ground state if $\varepsilon > 0$ with
$i\gamma_1 \gamma_2 = -1$
implies
$P = -i \gamma_1 \gamma_2 = \pm 1$.
The parity $P=\sgn(\varepsilon)$.

For the whole system,
\begin{align}
    P &=
    \prod_{i=1}^{N} \sgn\left( \varepsilon_i \right)
    =
    \sgn\left[ \Pf(D) \right]
    = \sgn\left( \Pf\left( A \right) \det W \right)
\end{align}
where $\Pf$ denotes the Pfaffian,
which is defined for an $N\times N$ matrix with $N$ even matrix by
\begin{align}
    \Pf(M) &=
    \frac{1}{2^{N}}
    \frac{1}{N!}
    \sum_{\text{permutations }\sigma}
    {(-1)}^{\sgn(\sigma)}
    \prod_{i=1}^{N}
    A_{\sigma(2 i - 1), \sigma(2i)}.
\end{align}

For example,
\begin{align}
    \Pf
    \begin{pmatrix}
        0 & \varepsilon\\
        -\varepsilon & 0
    \end{pmatrix}
    = \varepsilon
\end{align}
and
\begin{align}
    \Pf
    \begin{pmatrix}
        0 & \varepsilon & &\\
        -\varepsilon & 0 & &\\
        & & 0 & \varepsilon'\\
        & & -\varepsilon' & 0
    \end{pmatrix}
    = \varepsilon\cdot \varepsilon'
\end{align}
A useful property is that
\begin{align}
    \det(X) = {\left[ \Pf(X) \right]}^2
\end{align}

If we fixed the definition of fermion parity,
only consider $W\in SO(2N)$,
then
\begin{align}
    P(H) &= \sgn \Pf A
\end{align}

\section{Boundary Majorana Zero Modes}
Consider the Hamiltonian
\begin{align}
    H &=
    \sum_{i}\left[
        -t \left( c_i^\dagger c_{i + 1} + c_{i+1}^\dagger c_i\right)
        - \mu\left( c_i^\dagger c_i^\dagger - \frac{1}{2} \right)
        + \Delta c_i c_{i+1}
        + \Delta^* c_{i+1}c_{i}
    \right]
\end{align}
where $\Delta = |\Delta| e^{i\phi}$,
in which case we can just redefine
\begin{align}
    c_i \to \tilde{c}_i = e^{i\phi/2}c_i
\end{align}
to absorb the $\phi$.
Then you would write
\begin{align}
    \tilde{c}_i &= \frac{1}{2}
    \left( \gamma_{2i - 1} + i\gamma_{2i} \right).
\end{align}
Then
\begin{align}
    H &= \frac{i}{2}\sum_{i}\left[
        \left( |\Delta| + t \right)\gamma_{2i}\gamma_{2i+1}
        + \left( |\Delta| - t \right) \gamma_{2i - 1} \gamma_{2i + 2}
        - \mu \gamma_{2i - 1} \gamma_{2i}
    \right]
\end{align}
Let us consider two limiting cases with open boundary conditions.

Case 1: $|\Delta| = t = 0$ and $\mu > 0$.
Then
\begin{align}
    H &= -\frac{i}{2}\mu\sum_{i}\gamma_{2i - 1}\gamma_{2i}
\end{align}
so each site has
\begin{align}
    -i\gamma_{2i - 1}\gamma_{2i} = 1 = P_i
\end{align}
This is an ``atomic superconductor''
with tightly bound Cooper pairs
and strong pairing phases.
The cooper wave function is exponentially decaying.

Case 2: $|\Delta| = t > 0$ and $\mu = 0$.
$\gamma_1$ and $\gamma_{2N}$ are boundary Majorana zero modes with
\begin{align}
    \left[ H, \gamma_1 \right] =
    \left[ H, \gamma_{2N} = 0 \right]
\end{align}
which means there is a 2-fold degeneracy of the 0 energy state
corresponding to
\begin{align}
    i\gamma_1 \gamma_{2N} = \pm 1
\end{align}

Away from the idealized points,
still free fermion,
we can look for solutions to
\begin{align}
    \left[ H, \tilde{\gamma}_{L} \right]
    = \left[ H, \tilde{\gamma}_{R} \right] = 0
\end{align}
in the limit $N\to\infty$ where
\begin{align}
    \tilde{\gamma}_L &= \alpha_1 \gamma_1 + \alpha_2 \gamma_2 + \cdots\\
    \tilde{\gamma}_R &= \beta_{2N}\gamma_{2N} + \beta_{2N - 1}\gamma_{2N - 1} +
    \cdots
\end{align}
where
\begin{align}
    \alpha_x &\sim e^{-x/\xi}\\
    \beta_{2N - x} &\sim e^{-x/\xi}.
\end{align}
For $\nu=-1$,
can find solution $\alpha$ and $\beta$
decay experimentally away from the boundary.

The finite size system boundayr Majorana zero modes will have some wave function
overlap.
This effective Hamiltonian ground state is in a subspace
with effective Hamiltonian
\begin{align}
    H_{\textrm{eff}} &=
    \frac{i}{2} t_{\textrm{eff}} \tilde{\gamma}_L \tilde{\gamma}_R
\end{align}
where the effective hopping coefficient is
$t_{\textrm{eff}}\sim e^{-N/\xi}$.

At the phase transition,
$\xi\to\infty$,
the MZMs become delocalized.
This is the critical point that is a field theory of 1D gapless Majorana modes.

Note that when
\begin{align}
    \left[ H, \tilde{\gamma}_L \right]
    = \left[ H, \tilde{\gamma}_R \right] = 0
\end{align}
it is called a ``string zero mode.''.

If we turn on interactions,
\begin{align}
   \delta H \propto
   \sum_{ijkl} \gamma_i \gamma_j \gamma_k \gamma_l
\end{align}
The boundary MZMs still exist,
but they have a doubly degenerate ground state
in the $N\to\infty$ limit,
but generically no longer commute with $H$.

The ground state degeneracy is ``topological''
in the following sense.
\begin{enumerate}
    \item Robust to perturbations.
    \item No \emph{local} bosonic operator that can distinguish between the
        ground states.
        Specifically,
        \begin{align}
            \bra{j}\mathcal{O}\ket{i} \propto \delta_{ij}
            + O\left( e^{-L/\xi} \right)
        \end{align}
        for any local bosonic operator $\mathcal{O}$.
        It is not protected against local fermionic operators though.
        \begin{align}
            \bra{j}c_1\ket{i}, \bra{j} c_N\ket{i}
            \not\propto \delta_{ij}
        \end{align}
        As long as we can protect the system from stray fermions,
        we have a
        ``topologically protected qubit''.
        Otherwise,
        we have so-called ``quasi-particle'' poisoning.
\end{enumerate}
Boundary Majorana Zero Modes can be understood from field theory.

Suppose we have a spinless $p$-wave superconductor with Hamiltonian
\begin{align}
    H &= \frac{1}{2}\sum_{k\in\mathrm{BZ}}
    \Psi_k^\dagger
    \begin{pmatrix}
        \varepsilon_k & \Delta_k^*\\
        \Delta_k & -\varepsilon_k
    \end{pmatrix}
    \Psi_k
\end{align}
where
\begin{align}
    \Psi_k =
    \begin{pmatrix}
        \Psi_k\\
        \Psi_k^\dagger
    \end{pmatrix}
\end{align}
and $t=|\Delta|$, $\mu\approx -2t$ and
\begin{align}
    \varepsilon_k &= -2t\cos k - \mu\\
    \Delta_k &= -2i\Delta\sin k.
\end{align}

The the limit $k\to 0$ and we get
\begin{align}
    \varepsilon_k &\to -2t - \mu = m\\
    \Delta_k &\to -2i\Delta k
\end{align}
where $m$ is defined as the mass term in the field theory.
Assume $\Delta$ is real so that $\tilde{\Delta}:= 2\Delta$.
The Hamiltonian thus can be written as
\begin{align}
    H_k &=
    \begin{pmatrix}
        m & i\tilde{\Delta}k\\
        -i\tilde{\Delta}k & -m
    \end{pmatrix}
    = \tilde{\Delta}k\sigma^y + m\sigma^z.
\end{align}
In real space continuous variables,
\begin{align}
    H &=
    \frac{1}{2}\int dx\,
    \Psi^\dagger(x)
    \left( -i\tilde{\Delta}\sigma^y \partial_x + m\sigma^z \right)
    \Psi(x)
\end{align}
where
\begin{align}
    \Psi(x) &=
    \begin{pmatrix}
        \Psi(x)\\
        \Psi^\dagger(x).
    \end{pmatrix}
\end{align}
here, we have
\begin{align}
    \Psi^*(x) &= \left( \Psi^\dagger(x) \right)^T
\end{align}
and the Majorana condition is
\begin{align}
    \Psi(x) &= \sigma^x \Psi^*(x).
\end{align}
The ``trivial phase'' is for $\mu + 2t < 0$ which means $m>0$.
The topological phase occurs when
$\mu + 2t > 0$ which implies $m < 0$.
And the critical point is when $\mu = -2t$ and we get $m=0$.
This is a massless $(1+1)D$ Majorana field.

\subsection{Digression: Jackiw-Rabbi system}
Consider a $(1 + 1)D$ massive Dirac fermion.
Then if we define
\begin{align}
    \Psi =
    \begin{pmatrix}
        \Psi_1\\
        \Psi_2
    \end{pmatrix}
\end{align}
consider a $(1+1)D$ massive Dirac fermion
\begin{align}
    H_{\textrm{Dirac}} &=
    \frac{1}{2}\int dx\,\left[
    -i \nu \Psi^\dagger \sigma^y \partial_x \Psi
    + m\Psi^\dagger \sigma^z \Psi
    \right]
\end{align}
which arises as the continuum limit of the SSH model
\begin{align}
    H_{\mathrm{SSH}} &=
    -\sum_{i}\left[
    \left( t + {(-1)}^{i} \delta t \right) c_i^\dagger c_{i+1} + \textrm{h.c.}
    \right]
\end{align}
This is used to model polyaccetaline if we enable two spin $\frac{1}{2}$.
Lable each unit cell by
\begin{align}
    \Psi_r &=
    \begin{pmatrix}
        c_{2r - 1}\\
        c_{2r}
    \end{pmatrix}
\end{align}
Also $\delta t \approx m$ of Dirac point.

Consider a domain wall in $m(x)$.
Look for eigenstates of $H_{\textrm{Dirac}}$.
\begin{align}
    q^\dagger &= \int dx\,\left[ 
    \phi_1(x) \psi_1^\dagger(x)
    + \phi_2(x) \psi_2^\dagger(x)
    \right]
\end{align}
defines eigenstate of energy $\varepsilon$ since
\begin{align}
    \left[ H_{\mathrm{Dirac}}, q^\dagger \right] &=
    \varepsilon q^\dagger.
\end{align}
where $\phi =
\begin{pmatrix}
    \phi_1\\
    \phi_2
\end{pmatrix}$.

Consider a simple limit with 1 fermion pre unit cell.
And we have 2 chains.
1 fermion = 2 domain wall.
1 domain wall = $\pm\frac{1}{2}$ charges.
Example of charge fractionalization in charge topological insulator.

Generalization: $\frac{1}{m}$ fillings.
CDW with on sites per unit cell $\to$
elementary domain wall
$\implies$
$\pm \frac{1}{m}$ charges.

\begin{enumerate}
    \item Modulation arises from spontaneous symmetry breaking
        both dimerization patterns have some energy.
        Domain walls are ``deconfined''.
    \item Modulation from ``explicit'' symmetry breaking terms domain wall.
        Then two dimerization patterns have different energy
        $\to$ confined linear energy cost.
\end{enumerate}

\subsection{MZMs and J-R soliton}
\begin{align}
    \Phi(x) &=
    \begin{pmatrix}
        \Psi\\
        \Psi^\dagger
    \end{pmatrix}\\
    \Psi &= \sigma_x \Psi^*\\
    \Phi_0 &=
    \begin{pmatrix}
        \phi_0\\
        \lambda \phi_0
    \end{pmatrix}\\
    q^\dagger &=
    \int dx\, \left[
    \phi_0\Psi^\dagger(x) + \lambda \phi_0\Psi(x)
    \right]
\end{align}
$\phi_0$ is real $\implies$ $q= \lambda q^\dagger$ and
\begin{align}
    \gamma &=
    \begin{cases}
        q & \lambda = 1\\
        iq & \lambda = -1
    \end{cases}
\end{align}
which implies
\begin{align}
    \gamma^\dagger &= \gamma\\
    \gamma^2 &= 1
\end{align}
which are Majorana zero modes bound to domain walls.
