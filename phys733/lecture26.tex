\section{Quantum Spin Liquids}
\subsection{LSM Constraints}
If Hamiltonian has symmetry
\begin{align}
    G = \underbrace{SO(3)}_{\text{spin rotation}}\times
    \underbrace{ \mathbb{Z}^d}_{\text{trans sym}}
\end{align}
and an odd number of spin-$\frac{1}{2}$ per unit cell,
then no trivial unique gapped ground state is possible.

In $d=1$, either
\begin{enumerate}
    \item Ground state either breaks symmetry
    \item Or gapless
\end{enumerate}
In $d>1$
\begin{enumerate}
    \item  ''''
    \item ''''
    \item Topological order (compatible with symmetry)
\end{enumerate}
With
\begin{align}
    G &= U(1) \times \mathbb{Z}^d
\end{align}
filling $\nu$ is charge per unit cell.

If $\nu$ is fractional then no unique gapped ground state is possible.

The general version.
\begin{align}
    G &=
    G_{\textrm{on-site}} \times \mathbb{Z}^d
\end{align}
If projective representation per unit cell
\begin{align}
    [w] \in H^2 \left( G, U(1) \right)
\end{align}
the no unique gapped ground state is possible.

Consider $U(1)$ symmetry.
Consider a cylinder with space like $T^d$.
Let $L_x=: L$.
$V$ is then umber of unit cells, and $C=V/L$.
Then translational symmetry acts like
\begin{align}
    T_x \ket{\Psi_0} &=
    e^{iP_x^0}
    \ket{\Psi_0}
\end{align}
Suppose ground state is unique and gapped.
Insert one flux quantum adiabatically,
then large gauge transformation
\begin{align}
    U &= e^{i \frac{2\pi}{L} \sum_r x n_r}
\end{align}
acts like
\begin{align}
    \ket{\Psi_0} &\to
    U c \ket{\Psi_0} = \ket{\tilde{\Psi}_0}
\end{align}
and
\begin{align}
    T_x \ket{\tilde{\Psi}_0}
    &=
    e^{i \tilde{P}_x^0}
    \ket{\tilde{\Psi}_0}
\end{align}
Then the new momentum is
\begin{align}
    \tilde{P}_x^0
    &=
    P_x^0
    +
    \frac{2\pi}{L} \sum n_r\\
    &=
    P_x^0 + \frac{2\pi N}{L}\\
    &=
    P_x&0 + 2\pi \frac{V}{L} \frac{p}{q}
\end{align}
where we let $\nu = N/V$.
Assume $\nu = p/q$ for some integers $p$ and $q$ which are coprime.
Pick $V/L$ to be coprime with $q$ so that
$\frac{V}{L} \frac{p}{q}$ is not an integer.
So then
\begin{align}
    e^{i\tilde{P}_x^0}
    \ne e^{i P_x^0}
\end{align}
which implies
$\ket{\Psi_0}$ and $\ket{\tilde{\Psi}_0}$
must be orthogonal,
which is a contradiction.

\begin{question}
    This is not a full proof?
\end{question}
No, to get a full and rigorous proof,
it was done originally by LSM.
They considered a variational state,
and showed you get could to $e^{1/L}$ energy of the ground state.
Hastings gave a long rigorous proof,
but add to give a number of assumptions about then umber of unit cells.
He was able to prove it for even cells,
but still had restrictions on the size.

\begin{question}
    How do you have the freedom to pick $V/L$.
\end{question}
We're assuming we have enough freedom to choose the case.
This proof would not work for specific vlaues of $V/L$.
Then ext step is that if I cannot have a unique ground state for this system
size,
then I also cannot have it for any system size.
The intuition is that if you did have a gapped ground state,
even if it wre degenerate,
then it has a finite coreelaiton elngth,
so the spectrum shouldn't care xactly what hte systme size is because it'sa
lgobal thing up to exponentially small corrections.
Go from speicfic system sizes to generic syste msizes,
by arguing correlation lengths.

\begin{question}
    This smells like the FQH argument,
    where $V/L$ is the number of orbitals.
\end{question}
In FQH yo uhave fractional filling,
and htis is exactly a way of proving FHQ has degenerate ground states,
and if you did the work,
then number of degenerate ground states if $q$.
If you went further,
you could show that if I insert one flux quantum,
that toggles from one state to another,
and you can go futher and argue that you have to have $q$ states.

\begin{question}
    What if there is no translational symmetry?
\end{question}
Whatwe learn her is that if you have translational symmetry and on site
symmetry,
no unique ground state is possbile.
For $d>1$ there are 3 psosiblities.
Topologcial order is compatible wiht symmetry.
The degenarcy itself is robust to breaking tansliaontl symmetry.
The topological degeneracy when you insert flux does change which degenerate
state you are in.
The main point is that once you have topolgoical order you can satisfy this
theorem,
but the teorem itself sdone't require TI,
but jsut that TI requires you have these almsot degenerate states.
The tarnslation symetry and this filling requirement imply that you musth ave a
degenearcy,
but that degenracy can still be robust to breaking these tnralsioantl symmetry.
Nothing says the egeneracy isn't robust if I brek the translational symmetry.

\begin{question}
    Here you asumed periodiciiyt in the $x$ direction,
    but where did we use it?
\end{question}
Technically we used it because I asumed it had trnalaiotnl symmetry in any
direciton,
but if I had a boundary,
I would technically break tranlationl symmetry.
And I needdd translaitonal symmetry to define a filling.
It's there implicitly,
but explicitly I didn't use it.
In practise it would get a bit more subtle wiht boundaries,
becase you would have chiral edg modes on the boundary,
and it wouldn't necessarily gapped.

This result applies to way more states if I put it on a torus.

\section{Spin liquids}
That was all motivation.
Once you have a system with fractional filling,
then topological order is the only way to still have a gapped state and be
compatible with the symmetry.
Interestingly if you have such a system with an odd number of spin-$\frac{1}{2}$ 
per unit cell,
plus the other conditions,
then that implies topological order.
You have systems with fractional filing and spin-half per unit cell,
ad if the system wants to preserve the symmetry and be gapped,
the only thing that can make that happen is topological order.

The physical picture for what the quantum spin liquid is,
comes from the idea of a resonating valence bond.
\begin{align}
    H &=
    \sum_{\langle ij\rangle}
    J_{ij} \vec{s}_{i} \cdot \vec{s}_{j}
\end{align}
for $J_{ij}>0$.

For a single pair $J_{ij} S_i S_j$,
the lowest energy state is a spin singlet.
Idea of resonating valence bond,
is the idea that it's a superpositions of different singlet configurations.
This goes back to chemistry,
where you have molecules like benzine,
where you have 6 carbon atoms,
and hydrogen out here,
and then the electrons can form singlet configurations.
But the actual state of the molecule is a superposition between this singlet
configurations and the other.
People knew about it in Chemistry,
and Linus Pauling started thinking about many-body valences bond resonance in
metal,
and that motivated Phil Anderson to think about resonating valence bond in
insulators.
Anderson in 1973 proposed the idea that you could have an electrically
insulating system where the spin degrees of freedom form resonating valence
bonds.
Meaning that the ground state is a superposition over valence bond
configurations.

The system is forming a quantum liquid of singlets,
a gorund stae superpositon of all posssible singlet configuraotns.
Andreson talked about it in 1973 andit ws completely ignored.
In 1986,
physicist dicovered high temperature superconductivity in the cuprates.
Right after that there was  big explosion in the field.
People tried to come pu with theories for why cuprates where high-temperature
superconductors,
and Anderson had this spin liquid lying in a draw,
and he said look,
maybe this spin liquid could have something to do with it.

Since 1987,
there's a huge explosion about understanding spin liquids and resonance states.
I'm not going to mention the relationship with high-temperature superconductivity,
because it's very tenuous and controversial.
It may or may not have anything to do with high-$T_c$ superconductivity,
and there are opposing camps of physicists who hate each other.
It led to the destruction of the careers of many young scientists in the
1990s,
and there should be dramatic books written about the sociology about this,
extremely fascinating.

Now,
we're getting to a point where those people are dying and retiring,
so maybe the new generation can solve this problem.
It was pretty bad.

The kind of ground states natural to think about,
once you realise that the lowest energy state is a singlet,
the natural states to think about are a crystal of these valence bonds,
sometimes called VBS or valence band solid.

These bonds form singlets,
for do so in some ordered pattern.
This kind of state breaks translation symmetry,
but it preserves the SO(3).
In the context of LSM,
it's the first possibliity where you just breka the symmetry.

Another state possible is a spin-density wave.
The simplest example is a ferromagnet.
So you have spins up and down.
An anti-ferromagnetic is a special spin density wave with wave vector $k=\pi$.
This thing breaks SO(3) as translatoin,
but it preserves some combination.

The third interesting possiblitiy s htis RVB state,.
There are two kinds of RVB states we talk about.
One is short-range RVB,
where spins near each other form valence bonds,
so only a superposition over states where nearby spins for valence bonds.
THis kind of thing means systems have finite correlation lnegth.
Anything happening in some reaosn,
spins are only connected to spins nearby,
so they come with finite corelation length,
which is typically associated with gapped states.

Ify ou had a short range RVB states,
a gapped state,
then it has to have topological order by the LSM theorem.
Specifically,
if you put it on a torus,
there iwlll be a gorund satte degenearcy as required by LSM.

Then you have long-range RVB,
where you have valnece bonds arbirarily far apart,
but of course the amplitude for having long range valence bonds will decrease
far away but perhaps power law decay.


It's the same reason why you have gapless systems at all.
A gapless system is one where the correlation decays as a power law.
How can that happen when I only have short-range interactions?
The system some how conspires to have long-range correlations,
power law decay,
but still long range.

One thing you can think of physically is suppose that these bonds are really
strong.
This guy has really strong bonds wiht this one,
but the bonds iwht this guy are weak,

So far I've given you a picutre,
from which you can construct variatonal states,
where you can write down a ground sate that is a superpositon of different
configuraitons.
You can try to figure out are these energetically favourable ground states.
That's what Anderson did.
It's not super concrete either.
I have a Hamiltonian whose ground state is going to look like this form.

\section{Quantum Dimer Model}
An importnat devlopment after 1987 was quantum dimer models,
which gave a nice way to think about spin liquids,
changing the rules a little to make thisp icutre more precise.
It changed the rules a bit so people don't like that.
But the good review article is arXiv 0809.3051,
and the original paper was Rokhsar-Kivelson 1987.

The quantum dimer model says let's forget about the spins entirely,
and think of a quantum state system of dimers,
which are just colorings on edges.

The idea is to directly model valence bonds.
We replace valence bonds with dimers.
The key point is that we treat different dimer configurations as orthogonal
quantum states.
This is the key thing.
This is actually incorrect for spin systems.
If I take a spin system,
consider some condition nation of singlets, valence bonds,
that's not going to be orthogonal to every other configuration.
But the idea of the dimer model is to forget that and just demand they are
orthogonal.

The next thing is to assume the dimers cannot touch each other.
So if one spin forms a singlet with one spin,
it cannot also form a singlet with another spin.
So dimers cannot touch.

Here we just set up the Hilbert space.
Now we want a Hamiltonian,
for example on a square lattice.
The Hamiltonian will have two terms,
the sum over plaquettes of the square lattice.
\begin{align}
    H_{QDM}
    &=
    \sum_{\square}
    -t \left(
        \ket{=}\bra{||}
        + \mathrm{h.c.}
    \right)
    +
    V
    \left( 
    \ket{=}\bra{=}
    +
    \ket{||}\bra{||}
    \right)
\end{align}
The properties of the ground state depend very sensitively what dimension we are
considering,
whether square, cubic,
and it also depends very sensitively on the geometry,
like triangle, honeycomb, etc.

Because we have these local rules which flip configurations,
let me define some terminology.

We could have a \emph{flippable plaquette}
which takes $=$ to $||$.

A non-flippable plaquette,
consider 3 plaquettes,
each with 1 singlet,
but the bonds are not touching.
Then that's not flippable.

We could also have a \emph{flippable loop}.

I can draw a fictitious loop that surrounds these guys.
For a dimer configuration on this loop,
if I can flip it with local moves to get the complementary configuration,
then if I can get from one to another by dong local moves in the interior of
the loop and maybe a little outside of it,
then that would be  flippable loop.

Once we introduce these local moves,
the question is which dimer configuration can be converted to another dimer
configuration by local moves.
And that introduces topology and winding into the system.

You can ask if there are dimer configurations that cannot be transformed into
each other by local moves,
and if there are are,
then you can argue they are topologically distinct from one another.

It turns out the biggest difference in dimer models is whether your lattice
bipartite or non-bipartite.
Bipartite means that you can colour the vertices with 2 colours such that no
nearest neighbour has the same colour.
Square lattice you can do that so it's bipartite,
but you can't do that on a triangular lattice, so it's non-bipartite.

The notion of topological sector is that you count the number of dimers that are
crossing some line.
What I mean is that suppose you draw a triangular lattice.
Then you draw a dotted line that crosses some plaquettes.
You draw your line and you count how many crossings with dimers there are.
One thing you'll notice is that the local moves only change the number of
crossings by an even amount.

Local moves here means a specific move on two triangles.
The key point it's always 2 to 2.
That's where this parity thing comes up.

\begin{question}
    Is there a physical reason we have this constraint.
\end{question}
No.
We're just writing down interesting Hamiltonians for these and considering
interesting classes of moves.
We'll find these map to lattice gauge theory,
and changing the kind of moves allowed is changing the gauge constraints,
but there's really no physical reason.
It's made up anyway.
Physically motivated,
but the actual Hamiltonian is argued from another starting point.

\begin{question}
    Perhaps it's conservation of energy?
\end{question}
There are terms you can write down like
$S_i^+ S_j^+ S_k^- S_k^-$.
But then you can write down 3 spin terms as well\ldots
Suppose I have a full dimer covering,
so there are no extra dimers you can place anywhere that is compatible with the
rules.
Then you cannot move a single dimer.
Requiring we have  full dimer covering will restrict the local moves you can
have.

The point is that if I count the parity of the number of crossings,
that's a topological invariant because it's invariant under local moves.

Putting this on a torus gives four topological sectors,
because we have 2 non-contractile cycles,
one called $\alpha$ and another called $\beta$.
Suppose the number of crossings is $n_c^\alpha$ and $n_c^\beta$,
then the invariants are
\begin{align}
    n_c^\alpha\mod 2\\
    n_c^\beta\mod 2
\end{align}
You can get these from local plaquette moves.
We don'th ave such a term n the Hamilonitna htat allowed me to do that.

On a bipartitlelattcek we have the number o crossngs $\mod 2$.

If you have a bipartitle lattice,
then you have A sites and B sites.
Imagine that all of our dimers actually have an arrow on them,
but points to the A subslattice to the B sublattice,
so they implcitly have arrows on a to b aso on.

I can count the winding number,
which is
\begin{align}
    n_{\textrm{winding}}
    &=
    n_{\textrm{left-crossings}}
    - n_{\textrm{left-crossings}}
\end{align}
Ona a torus,
there are $\mathbb{Z}\times mathbb{Z}$

\section{Topological Excitations}
There are two kind of excitations we can have.
First first one are called \emph{visons}.

\begin{question}
    It's not clear it's the same for each vison
\end{question}

???

I vison is a dual excitation that lives on plaquettes.
Imagine a line that connects the two.
Then count the dimer crossings on that line.
The number of crossings is $n_c$.

The wave function of a vison is
\begin{align}
    \ket{\mathrm{ground state}} = \sum_c (-1)^{n_c} A_c \ket{c}
\end{align}
but the ground state is
\begin{align}
    \ket{\mathrm{ground state}} = \sum_c A_c \ket{c}
\end{align}

Why are they of this form?
Because it's like vortex.
These excitations become defined finite energy excitations.

In general,
the choice of line matters,
and the energy wold even depend on\ldots

The next type of topological excitation is called a \emph{monomer}.

I can have terms like this to $H$
\begin{align}
    \ket{|}\bra{:} + \ket{:} \bra|{}
\end{align}
The dots just means separate spins by themselves.
The name vison is controversial.
It was introduced in 1997 by Matt Fischer.
All people 10-years liquid than $\mathbb{Z}_2$ vortices got renamed to something
new.
there's a reason to change the name.
One way to think about what's going on,
say you have a normal superconductor with vortices.
You can have a $+1$ vortex,
or a $-1$ vortex.
A vison is kind of a superposition between a $+1$ and $-1$ vortex.
You can think of it as all quantum superposition of odd-number winding
vortices.
The -on is for particle.
V for vortex, and is for Ising.
So vison.

The key point is that depending on whether the ground state is ordered,
valence bond crystal,
or resonating valence bond state,
a uniform superposition over all kinds of valence bond configurations,
pending on which type of ground state we're in,
these excitations will either be confined or deconfined excitations.

Confined means linear in separation.
Deconfined means you can separate them arbitrarily far apart without energy cost.

There are mutual statistics between visons and monomers.

In ordered states they are confined and in spin liquid RVB they are deconfined.

Ordered VBS confined vs spin liquid RVB deconfined.
