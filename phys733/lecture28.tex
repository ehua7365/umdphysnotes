\section{Toric Code}
Last time I talked about $U(1)$ and $\ZZ_2$ lattice gauge theory,
and I said that we can make the Gauss law constraint an energetic constraints.
I emphasised that this is dramatic change to the system,
because it's not just change the Hamiltonian but fundamentally changes the
Hilbert space.
Before we had a constrained Hilbert space,
whereas afterwards,
our Hilbert space is just that of a many-spin system.
Our Hilbert space decomposes into a tensor product of qubits on sites
and qubits on links.
\begin{align}
    \mathcal{H}
    =
    \underbrace{\bigotimes_{i\in\mathrm{sites}} \CC^2}_{\text{matter}}
    \otimes 
    \underbrace{\bigotimes_{i\in\mathrm{links}} \CC^2}_{\text{gauge field}}
\end{align}
This is an important conceptual shift.
Before you think gauge theory is something special and were does it come
from.Gauge theory was put on some crazy pedestal.
Once you understand this point,
you can think of the gauge field as just some phase of a many-body spin system.
You automatically have an emergent gauge field that is dynamical and so on.
You realise that gauge theory can emerge from some interacting spin system.

In a way,
this addresses one of the great fundamental question in physics,
and we now understand gauge theory can come from pretty many any many-body system
emerging dynamically.

\begin{question}
    Can we create a QED theory from this?
\end{question}
Yes.
You could get nonce normalizable gauge theories.

We removed the constraint entirely,
but we have some particular term in the Hamiltonian that enforces.

\begin{question}
    Is this a computational or conceptual advance?
\end{question}
It's more a conceptual advance,
but we could simulate spin system.so
It convinces us that gauge theories can emerge from a system that has no gauge
fields.
Whether the standard model arises from this is its own question.
The standard model has Lorentz invariance,
there are a lot of gapless particles and they all have the same speed.
It's not obvious how to get that.

You can show that for some systems if you go to low energy Lorentz invariance
can emerge,
but it's slow and logarithmic so people are not convinced.
But gravity is nowhere here.

People think it can emerge if the cosmological spacetime is anti-de Sitter
space,
but that's a different story.

Last time,
we went to pure $\ZZ_2$ lattice gauge theory
with Gauss law implemented energetically.
And we had a Hamiltonian on plaquettes and vertices.
\begin{align}
    H &=
    -K \sum_{\square} B_{\square}
    - J \sum_{+} A_{+}
    - h\sum_{l} \sigma_l^{x}
\end{align}
where
\begin{align}
    B_{\square} &=
    \prod_{l \in \square}
    \sigma_l^z\\
    A_{+} &=
    \prod_{l\in +}
    \sigma_l^x
\end{align}
If we set $h=0$,
this model is the famous $\ZZ_2$ toric code model of Kitaev.
And what's special about his $h=0$ limit is that every term commutes with each
other and the mole becomes exactly solvable.

From now on,
I will only talk about the $h=0$ limit that is exactly solvable because
the $B$'s and the $A$'s all commute with each other.
It's easy to see that because if you take a plaquette and a vertex star,
if they don't overlap they obviously commute.
But if they overlap,
they will always share two links with two $\sigma_x$ and $\sigma_z$.

Thus we can simultaneously diagonalize every operator here.
Because they are all Paulis,
$B^2=1$ and $A^2=1$,
so we can just go into  some basis that simultaneously diagonalises these and the
star is just labelled by the eigenvalues $\pm 1$ of these operators.
Excited states would correspond to flipping these eigenvalues.

We can actually add a constant to this Hamiltonian and make it a sum of
commuting projectors.
We can define the projector for a plaquette as
\begin{align}
    P_{\square}
    &=
    \frac{1}{2}\left( 
    1 - B_{\square}
    \right)
\end{align}
so that $P_{\square}^2 = p_{\square}$
and similarly for the vertices
\begin{align}
    P_{+}
    &=
    \frac{1}{2}\left( 1 - A_+ \right)
\end{align}
and so the Hamiltonian is just a sum of commuting projectors
\begin{align}
    H &=
    \frac{K}{2} \sum_{\square} P_{\square}
    +
    \frac{J}{2} \sum_{+} P_{+}
\end{align}
Before in the AKLT it was as sum of projectors but he projectors weren't
commuting.
There's a whole world here.

A nice way to think of the ground star of this model is basically as a loop gas,
a sum over all loop configurations.

Pick a basis for $\sigma_z$.
Let's say $\sigma^z = +1$ is the absence of a string on an edge.
And if it's $-$ it's  the presence of a string on an edge.
The product of $\prod_{l\in\square}\sigma_l^z=1$
means that there should be an even number of links that have strings.
Then what does $\prod_+ \sigma_l^x$ on a star do?
If I have no strings on a star,
then it's going to create strings on the star when you apply it.

Actually, I got it backwards.

If $\sigma^z = +1$ for a link,
it means that there is a string that crosses it.
If $\prod_{l\in\square} \sigma_l^z = 1$ then that means there is an even number
of strings on a plaquette.
This means that the strings on the dual lattice do not end.

You can think of $\prod_{l\in +} \sigma_l^$ as it as a plaquette on the dual
lattice.

If we have an even number of strings around the plaquettes of the dual lattice,
I better have a string somewhere else,
basically the strings cannot end and have to keep going.

So basically we have to have closed strings on a dual lattice,
and this other term says that if I apply it I get another closed string.
So the $+1$ eigenstate of
$\sum_{\square}B_{\square}$
is the uniform superposition state of only closed loops on the dual lattice.

And the $+1$ eigenstate of the $A_+$ operator is a uniform superposition of all
closed loop configurations.
It's not every closed loop,
but only loops that can be made by deforming a single operator.

At least in the situation where $K$ and $J$ are positive,
we can think of the ground state as a uniform superposition of closed loops.
For $K,J>0$,
the ground state is a uniform superposition of closed loops.

A basis of $\sigma^z$ is loops on the dual lattice.
$\sigma^x$ is loops on the primal lattice,
but it's just a change in perspective depending on which basis you pick.

This means that on this torus there are 4 topologically degenerate states,
given by whether you have an even or odd number of closed loops around each
cycle.

This is an example of a non-invertible topological phase,
because we have this topological degeneracy.
You can in fact not need to define it on a square lattice,
but could have used any graph with the same basic properties,
planar graph,
then you can also define it on any triangulation on a genius $g$ surface,
and you would have $2^{2g}$ different sectors.
So you can define a similar model on any triangulation of genus $g$ surface with
$2^{2g}$ topologically degenerate ground states.

\subsection{Excitations}
This model is basically $\ZZ_2$ lattice gauge theory,
and you should know what the excitations are.

You can think of creating excitations by applying local operators.

Suppose you apply a $\sigma^x$ operator to a single link.
That's going to flip the eigenvalues of the $B_{\square}$ operator on nearby
plaquettes.
If I apply an $\sigma^x$ on a link,
that's going at flip the eigenvalues next to it.
If you apply a product of $\sigma^x$ operators along some string $\gamma$,
then this think creates $2$ plaquette excitations far away
\begin{align}
    W_m(\hat{\gamma}) &= \prod_{l\in \hat{\gamma}} \sigma_l^x
\end{align}
And the energy cost is just $2K$ at one end and $2K$ at the other end,
and we can separte them arbitrarily far apart.
These are deconfined excivations because they can be separated arbtrarily far
without using energy besides the initial creation.
These are $m$ particles.
We say $\hat{\gamma}$ denotes a string on the dual lattice.
Then

Similarly,
we can apply $\sigma^z$ to a link and that does exactly the same thing,
except on the dual lattice.
Creating it on the original lattice creates a star excitation.
\begin{align}
    W_e\left( \gamma \right)
    &=
    \prod_{l\in \gamma} \sigma_l^z
\end{align}
These are $e$ excitations.
No local operator can create any individual one,
and you cannot convert one to another by local operators either,
so they are topologically distinct.
The way we think about it is this.

\subsection{Topological classes of excitations}
We have topologically classes of excitations.
1, $e$, $m$, and the composite particle $\epsilon = e\times m$.
These are 4 topological distinct excivations.
And we have fusion rules for these,
whch tell us how different topological excivations fuse together to give other
classes of topolgoical excitaitons.
\begin{align}
    e\times e &= 1\\
    m\times m &= 1\\
    \epsilon \times \epsilon &= 1\\
\end{align}

They have fractional statiics if yo utake these particls around each other and
you have intersting phases.
Imagine applying a string operator that creates a pair of $m$ particles
using $W_m\left( \hat{\gamma} \right)$,
and then we take $W_e\left( \gamma \right)$ around one of the $m$ particles.

And then I consider the order they apply
\begin{align}
    W_e\left( \gamma \right) W_{m}^{\hat{\gamma}}
    \ket{\mathrm{g.s.}}
    &=
    -W_m\left( \hat{\gamma} \right)
    W_e\left( \gamma \right)
    \ket{\psi_{\mathrm{g.s.}}}
\end{align}
This $-$ sign is the mutual statistics between $e$ and $m$.

One more thing to mention.
If $\gamma$ is a closed loop,
then
\begin{align}
    W_e\left( \gamma \right)
    \ket{\psi_{\mathrm{g.s.}}}
    &=
    \ket{\psi_{\mathrm{g.s.}}}
\end{align}
and if $\hat{\gamma}$ is a closed loop then
\begin{align}
    W_m\left( \hat{\gamma} \right)
    \ket{\psi_{\mathrm{g.s.}}}
    &=
    \ket{\psi_{\mathrm{g.s.}}}
\end{align}
So we have these loop operators that keep the ground state invariant.
So actually the ground state has a symmetry,
but it's a strange kind that is generated by loop operators.
The reason we have this symmetry because we actuallyhad it microsocpially,
and these operators commute with the Hamiltonian.

$H$ also commutes with $W_e\left( \gamma \right)$ and $W_{m}\left( \hat{\gamma}
\right)$
if $\gamma$ and $\hat{\gamma}$ are closed.

Comapred to conventional symmetries,
like flipping spins,
this Hamiltonian has a spcial clsass of symmetreis,
called $1$-symmetries,
becaue thyeare dimension-1 operators.

Even if we broke the symmetry they the level few the Hamiltonian by adding some
perturbation,
for example the $h \sigma^x$ perturbation,
then we would break the $W_m$ symmetry.
We could also add terms to break the $W_e$ symmetry.
The Hamiltonian will still have these loop symmetries,
but these operators will be slightly different.

So the existence of these loop symmetries is a \emph{stable} property.
These days we say that these have an emergent 1-form symmetry.

There exist loop operators associated with $e$ and $m$ particles even if those
symmetries don't exist in in the original Hamiltonian.
At least  the toric code Hamiltonian,
those statements are certainly true.

Coming back here,
we have these mutual statistics between $e$ and $m$.
The exact form of the operators may change,
but you can think of topologically orderd phases as being characterisd by
topological order,
but they can also be cahactersied by these emergent symmetries,
which are just lop operators for hte non-trivial topological operators.

For example,
here we but the $m$ loop open to get an $m$ segement,
and we go through it with a $W_e$ loop.
So you cna in general get signs and phases,
and more complicated things can happen.
In general,
they not even be invertible,
and that leads at the concept of non-invertible symmetries.

That you have a minus sign n the mutual statistics is just a property of the
toric code.

You can imagine a different gound satte with $e$ particles
The point is that as $r\to\infty$,
$E(r)\to\mathrm{const}$..
Don't tkae the gneral thing I said too literally.

Now the $e$ and the $m$ partio
$e,m$ are bososns.
$\epsilon$ is a fermion.
You could do an exchange to find out  the phase.
Or you can find out if you did a $2\pi$ rotation,
locally this guy goes around itself.

Thisis all I wantd to say about $\ZZ_2$ lattice gauge theory.

\begin{question}
    Is it a loop?
\end{question}
You can have a more non-trivial algebra so they don't multiple = t

We noted how these can come form lattice guage htoery,
and pure lattice gauge theory without magnteic field,
that'sj s t Kitaev toric code.
There are 4 topological excitations,
generated by the $\ZZ_2$ charge $e$
and the $\ZZ_2$ flux $m$.

\ssection{Parton/Slave particle approach}
What I want to do now is to discuss another approach for describing spin
liquids.
The phrase slave-particle has been cancelled,
so it's called Parton, projected construction.
XG-Wen 2004 Chapter 9 has a good description of this.

Where gauge theory comes from in a spin model,
you stat with microscopic spins,
embed the 2-state Hilbert space in a larger Hilbert space with a constraint.
The idea here is that we parametrize microscopic Hilbert space in a real way by
expanding the Hilbert space and adding a constraint.

So let me say what I mean by the way.
At the opeartor level,
take my spin operator and represent it in terms of other operaotrs,
for example.

\begin{align}
    \vec{S} &=
    \begin{cases}
        \frac{1}{2}f^\dagger \vec{\sigma} f\\
        \frac{1}{2}z^\dagger \vec{\sigma} z
    \end{cases}
\end{align}
$f$ is a 2-component operator and it's a fermion.
\begin{align}
    f &=
    \begin{pmatrix}
        f_1\\
        f_2
    \end{pmatrix}
\end{align}
sometimes called a slave-fermion or
Schwinger fermion.
Meanwhile,
\begin{align}
    z &=
    \begin{pmatrix}
        z_1\\
        z_2
    \end{pmatrix}
\end{align}
is a complex calar that describes boson,
sometimes called ``slave boson'' or Schwinger bosons.

This i the Fermionic parton.
So we can write
\begin{align}
    S^\dagger &=
    S^x + iS^y\\
    &=
    \begin{pmatrix}
        f_1^\dagger & f_2
    \end{pmatrix}
    \begin{pmatrix}
        0 & 1\\
        0 & 0
    \end{pmatrix}
    \begin{pmatrix}
        f_1\\
        f_2
    \end{pmatrix} \\
    &=
    f_1^\dagger f_2
\end{align}
and then
\begin{align}
    S^z &=
    \frac{1}{2}
    \left( 
    f_1^\dagger f_1
    -
    f_2^\dagger f_2
    \right)
    S^-\\
    &=
    f_2^\dagger f_1
\end{align}
It is an exercise to check that
\begin{align}
    \left[ S^a, S^b \right]
    &=
    2i \epsilon^{abc} S^c
\end{align}
There are two states $\ket{\uparrow}$ and $\ket{\downarrow}$.
But in the fermionic picture,
there are 4 states
$\ket{00}, \ket{01}, \ket{10}, \ket{11}$.
So then $\ket{\uparrow}=\ket{10}$ and $\ket{\downarrow}=\ket{01}$
and we consider $\ket{00}$ and $\ket{11}$ to be unphysical state.
To remove these unphysical states,
we enforce the constraint that there is just one fermion on each site.
\begin{align}
    n_1 + n_2 = 1\\
    n_1 &= 1- n_2
\end{align}

We can think of the number of $f_1$s being equal to the number of holes of
$f_2$.

Constraints on a Hilbert space is the same as saying there's a gauge redundancy
invovled.
We can introduce a gauge transformation such that the physcial states are
invariatn undergauge transfomraiotn,
but the unphysical states are not.
And the gauge transformation is just to project on to the gauge invariant
states.

\subsection{Gauge redundancy}
Consider the $U(1)$ gauge redundancy.
\begin{align}
    f_{\alpha}
    &\to
    e^{i\theta} f_{\alpha}
\end{align}
which keeps $\vec{S}$ invariant.

I have introduced these Fock states for the fermions,
and consider how they transform.
\begin{align}
    \ket{0}_{f_1} &\to \ket{0}_{f_1}\\
    \ket{1}_{f_1} =
    f^\dagger \ket{0}_{f_1}
    &\to
    e^{-i\theta} f_1^\dagger f_1^\dagger \ket{0}_{f_1}
    =
    e^{-i\theta}\ket{1}_{f_1}
\end{align}
Menawhile
\begin{align}
    \ket{0}_{f_2} &\to e^{i\theta} \ket{0}_{f_2}\\
    \ket{1}_{f_2} &\to \ket{1}_{f_2}
\end{align}
So that means $\ket{01}$ and $\ket{10}$ are gauge invaritn but $\ket{00}$ and
$\ket{11}$ are not gauge invariant.
That meansa the physical space is equal to the gauge invariant subspace of the
fermions.

It turns out the redundancy is not just $U(1)$,
but we actually ahve $SU(2)$ gauge redunancy.
Let me just tell you what it is.

If we consider
\begin{align}
    \begin{pmatrix}
        f_1\\
        f_2^\dagger
    \end{pmatrix}
    &\to
    W
    \begin{pmatrix}
        f_1\\
        f_2^\dagger
    \end{pmatrix}
\end{align}
for some $W\in SU(2)$,
then the spin vector $\vec{S}$ will be invariant.
Let me rewrite this as
\begin{align}
    S^+
    &=
    f_1^\dagger f_2\\
    &=
    \frac{1}{2} \epsilon_{\alpha\beta} \tilde{f}_{\alpha} \tilde{f}_{\beta}
\end{align}
noting that
\begin{align}
    \tilde{f}_1 &= f_1^\dagger\\
    \tilde{f}_2 &= f_2
\end{align}
THis is paritcle-hole tranfomaitno.
Because fermions anticommute,
I just do this with the $\epsilon$ tensor and put in $\frac{1}{2}$.
If we do it in this langgue

And then we can write this as
\begin{align}
    \begin{pmatrix}
        \tilde{f}_1^\dagger\\
        \tilde{f}_2
    \end{pmatrix}
    &\to
    W
    \begin{pmatrix}
        \tilde{f}_1^\dagger\\
        \tilde{f}_2^\dagger
    \end{pmatrix}
\end{align}
which I can write as a shorthand
\begin{align}
    \tilde{f} \to W \tilde{f}
\end{align}
And what that tells us is that
\begin{align}
    S^\dagger &\to
    \frac{1}{2} \epsilon_{\alpha\beta} W_{\alpha\alpha'}
    W_{\beta\beta'} \tilde{f}_{\alpha'}
    \tilde{f}_{\beta'}
\end{align}
and for $W\in SU(2)$,
we have
\begin{align}
    \epsilon_{\alpha\beta}
    W_{\alpha\alpha'}
    W_{\beta\beta'}
    &=
    \epsilon_{\alpha'\beta'}\\
    W_{1\alpha'}
    W_{2\beta'}
    -
    W_{2\alpha'}
    W_{1\beta'}
\end{align}
For $\alpha',\beta'=1$,
\begin{align}
    W_{11}W_{21} - W_{21}W_{11} &=0
\end{align}
and for
$\alpha'=1$ and $\beta'=2$,
\begin{align}
    W_{11} W_{22} - W_{21} W_{12} &= |W| = 1
\end{align}
So the spin operator is kept invariant by this $SU(2)$ transformation.
The point is that once you write it in this way with the expanded Hilbert space,
that gives you access to new kinds of approximations you didn't have access to
before,
in particular new types of mean field theory.

This parton decomposition gives access to new mean-field approximations.
To see this,
consider the usual mean field theory on the Heisenberg model
\begin{align}
    H
    &=
    \sum_{ij} J_{ij} \vec{S}_{i} \cdot \vec{S}_{j}
\end{align}
and you just replace the $S$ with its expectation value,
so the mean-field Hamiltonian is
\begin{align}
    H_{\mathrm{m.f.}}
    &=
    \sum_{ij}
    J_{ij}
    \left( 
    \langle\vec{S}_i\rangle
    \cdot
    \vec{S}_j
    +
    \vec{S}_i\cdot
    \langle \vec{S}_j \rangle
    -
    \langle\vec{S}_i\rangle
    \cdot
    \langle\vec{S}_j\rangle
    \right)
\end{align}
But you want to enforce consistency with
\begin{align}
    \langle \vec{S}_i \rangle
    &=
    \bra{\Phi_{mf}}
    \vec{S}_i
    \ket{\Phi_{mf}}
\end{align}
This is useful a lot of the time,
but this is useless for $\langle S_i \rangle=0$,
because everything is just zero.

So this could only be useful for ordered states of spins.
If I want to describe states that have topological order,
usual mean field theory is no good.
But if I rewrite in terms of partons,
that goes me a new way of doing mean-field theory,
which gives me a window into describing a whole new class of phases of matter
that I couldn't have described just by talking about expectation values of $S$.

Let me just write it down and analyse it next time.
For the parton mean field,
I replace the $S$ with my fermion operators.
\begin{align}
    H &=
    \sum_{ij}
    J_{ij}
    \vec{S}_i
    \cdot
    \vec{S}_j\\
    &=
    \sum_{ij}
    \sum_{a}
    J_{ij}
    f_{i\alpha}^\dagger
    f_{i\beta}
    f_{i\gamma}^\dagger
    f_{i\delta}
    \sigma_{\alpha\beta}^{a}
    \sigma_{\gamma\delta}^{a}
\end{align}
and the identity to use here from QFT class is
\begin{align}
    \sum_{a}
    \sigma_{\alpha\beta}
    \sigma_{\delta\gamma}
    &=
    2\delta_{\alpha\beta'}
    \delta_{\alpha'\beta}
    -
    \delta_{\alpha\beta}
    \delta_{\alpha'\beta'}
\end{align}
and the Hamiltonian simplifies to
\begin{align}
    H &=
    \sum_{ij}
    \left[ 
    -\frac{1}{2}
    J_{ij}
    f_{i\alpha}^\dagger
    f_{j\alpha}
    f_{j\beta}^\dagger
    f_{i\beta}
    +
    J_{ij}
    \left( 
    \frac{1}{2} n_i
    -
    \frac{1}{4} n_i n_j
    \right)
    \right]
\end{align}
with a constraint that
\begin{align}
    n_i = 1
\end{align}
So far this is just an exact rewriting.

\begin{question}
    ???
\end{question}
In gauge theory,
you could have defined our gague field in the spin sstem,
and eery time the Guass law is violated
I'm going to call that thing matter,
and then you can exacly embed the gague field sapce with matter,
and it becomes and exat mapping.
But if you want pure gauge theory,
it always has to be a subset.
Here there is a mtter field.
Thera is a conjecture.
If you want ot exactly map the gauge field to a spin mdoel,
you need to mave am tter field,
that's the completeness conjecture.
But you're right,
it seems like I embeeed it in hte larger space,
but there's this other thing yo ucould do,
and that is call every Gauss law violation ``matter''
and that maps on to the Hilbert space.

\begin{question}
    Do these necesasrily follow fermionic commutation relations?
\end{question}
I don't know.
It's not entirely obvious how you would define that.
You're thiknig of oeprators htat create a aprticular fluctuation and what its
properties are.
It oden'st have to be bosonic or fermionic.
I might be able to talk about that level fo genearlity.
I would have some gauge field,
and I would have some matter,
so I can't decompose anything,
I can only talk about the whole of what's happening.
In this particular relation,
I can decompose what I call matter and what I call gauge field and it might be
tricky to do that.
So I don't have a good answer.

Certainly in the theories we write down,
it's gauge fields the couple to matter that is fermionic or bosonic.

\begin{question}
    If bosonic matter,
    you could call any violation of the Gauss law constraint bosonic,
    because yo can have as many bosons at a site as you want?
\end{question}
I think there will be fermionic and bosonic violations.
