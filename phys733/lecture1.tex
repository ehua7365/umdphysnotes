\section{Lecture 1}
The most famous place topology enters physics is in the discussion of
topological defects in ordered media.
For example you can have a vortex in a superfluid where a superconductor with an
order parameter winds in space.
The winding of some smooth order parameter.
Just to mention some examples, this would be vortices in superconductors.
Normal fluids are unordered media.
There are things called skyrmions in magnets, which if you think of things as
non-trivial windings along a sphere.

1. You have dislocations and disclinations in crystals and so on.


The mathematics that goes into describing this is homotopy theory.
You have some order, some symmetry that gets broken into some subgroup, then
it winds in a non-trivial way.
Mermin, RMP is a good reading.

These defects so far can be treated classically and is a well-trodden topic but
we're not going to focus on this.


Another example is the topological configurations of gauge fields
(magnetic monopoles) here the maths gets more sophisticated.
Gauge fields are connections on fibre bundles and fibre bundle has non-trivial
topology.
To really understand you need to study the topology of fibre bundles,
which is is mathematics.

You may have heard monopoles don't exist, like people have been looking for
magnetic monopoles for a long time, but they do exist in emergent gauge field
descriptions and you have monopoles of emergent gauge fields and those are
important.
They have even been experimentally observed in spin ice.
You can google to learn more about spin ice.

3.
Another place is that once you have non-trivial defect configurations like
solitons, if you have fermions living in your system, like electrons, then the
spectrumo f hte Hamiton might have zero-eenrgy states bound to the position of
these defects.
This subject is called zero modes of solitons and monopoles.
Here to understand this,
For example ou could have am assive field whre the mass changes sign.

So you have $x$ space, and $m$ mass of the field. It could go from negative to
positive, but as you cross zero there is some mode localized that is pinned to
exactly zero energy.
Understanding this brings us into the mathematics of index theory.


4. Topological band theory.
This is something the class is going to touch on a lot.
Here hte idea is that ify ou have a wave or a single particle in aperiodic
medium, hten thant wave or particle is going to acquire a dispersion relation.
As a function of momentum $k$ you're going to have enrgy bands $E(k)$.
Each band can have smoe topological character associated iwth them.
At each point in momentum space, you have states called Bloch states
$u_{f,k}(r)$ where $f$ is the band index and $k$ is the momentum.
As you track how it varies in momentum, that phase winds around in non-trivial
ways and that gives rise to topolgoical band theory.
This $u$ defines a vector bundle that can have non-trivial topology.
I don't expect you to understand what I'm talking about,
just want ot to give a broad interview.

THe most famous is the TKNN number, also called the \emph{Chern number}
which is an integer you can assign to every band once you know how to understand
how it behaves in momentum space.
This is a striking phenomenon and leads to the integer quanutm Hall efect.
When you fully fill a band a get a band instlualtion,
you get an electircal induslator, buti t has a Hall conductance, which si
atransverse response.
Iti s exactly quantized $\sigma_H= C e^2/h$.
$C$ is the Chenr number.

It's just about bands, so it's really something that describes single-particle physics.
These depict the one-elctonc states, where every electron can fill every
electron states.
You fill partilly or fuully any one of these bands.

The maths that describes this is the maths that came up early.
It's just the topology of figure bundles.
If you want a full understanding of this subject, you need to understand
$K$-theory.

Single-particle means the physics of interactions are not included in this
problem.
Also index theory shows up in this problem.

Q: What happens if you add interactions?

Sometimes the topological numbers are no longer meaningful if you add
interactions.
Then there are some cases where the topological invariant still has meaning with
interactions, but then we need new ways to describe it.
It could have different topological quantum numbers identified under
interactions, there's a possibility there are topological properties with no
analogues in theband theory too.
Both can happen.

5. What are all the possible phases of matter?
Two phases of matter are different if I tune the paramers o the sytem, I cannot
continuous go from one phase to another wihtout a phase transiton.
A phase transition is a singularity in the free energy density.

Let's try to classify all phases of matter.
Immediately when you pose htis question, topolgoy enters.
Imagine the space of all possible equilibrium systems let's just call it $X$
which is at all temperatures..
THey have some spatial dimension $d$ and some symmetyr $G$.
Imagine tunning the parameters of the system keeping these $d,G$ fixed.
As you parameters, you might encounter phase transition when the free energy
necessarily goes through a singularity.
In this vast space we're trying to imagine, you can remove all subspaces where
there's a phase transition and that gives us a new space $\tilde{X}$.
The problem of classifying spaces of matter is the problem of classifying all
the \emph{connected components} of $\tilde{X}$.
So the connected components of a space $\pi_0(\tilde{X})$ is called the zeroth
homotopy space.
Even the basic question of asking what the phases of matter are is immediate a
topology question asking for the connected components of a vast space we're
tyring to imagine.

6. Characterising topological phases of matter.
Most of this class is about this.
This class is mainnig about number 4 and 6.
When you htink about what are difrerntk inds of ordered phases.
For a lot of the 20th centruy, pepole thought you have symmetries spontaneously
broken, ask what local order parameters are from $G$ to some subgroup.
Then ask the effective free energy by expanding in gradients of the order
parmtere, then describet he long-wavelength properties by looking at smooth
breaking symmetries.
So we have the Ginzburg-Landau theory of symmetry breaking.
This is a powerful theory.
The heat capacity of a crystal.
Fluctations of atoms are phonons, then take expansions in the strain field and
you get it.
Fluids, magnets, just identify the local order parameter and write a GZ thoery.

In hte 1970s, Wegner studied $\mathbb{Z}_2$ lattice gauage theory.

He showed you can have phases with no local order parameter.
You need something more subtle.
This leads us to topological phases of matter.
In one sentence, a topolgoical phase of matter is the properties of the phase
canot be signalled by a local order parameter.
To understand it you need to look at non-local orperats and how they behave in
the ground state.

Q: Why ground state?
Good question.
Implicitly, when I talk about topological phases of matter, I'm talking about
zero-temperature, whichi shte ground state,
And also low-energy exciatitons fo the gorudn astate.
Some phases are distinct at zero-temperature, but at finite temperature the
distinction may melt away.

In topological phases of matter, the description you need is very intricate, has
a lot of fascinating mathematical structure, structure that is a more modern
topic in mathematics, not some 1940s maths we came up with before.

So instead, what you need are topological quantum numbers.
You insert defects which extend in various directions and dimensions and they
can have interesting topological properties.
Ultimately there is a topological quantum field theory that magically emerges
(TQFT) to describe the universal robust properties of the system.
The maths that goes into this is a very different kind of math.
It goes under the name of quantum topology, TQFT and Category Theory.
These topics are frontiers in pure mathematics as awell.

For me aI ifind it fascinating.
We have intricate math struytures called TQFTs, and somehow there are phases of
matter where TQFTs magically arise for describing phases of that system.

Q: IS TQFT is the same sense of GZ theory for long wavelength or is it a
differnet sense?
y
It's a similar sense, but it depends what you eman by similar.
GZ is useful for dynmiacal porperties and critical exponents.
TQFT is not useful for understanding dynamics, but it's useful for undestanding
braiding paritcles, so more like kinematics than dynamics.


I'm going to briefly mention a few others.

When you hear peopel talking about topological phases of matter, they're talking
about topological band thoery or TQFT.

7.Toplogical pumps.
You have some Hamltoinian that depends on some parmater.
This hamiltonian could be periodic $H(\tau + T) = H(\tau)$.
Imagine slowly changing $H$ so that by the around it comes back you're back to
where it started, but something about the systme has changed.
As you go through this system, you can pump charge across the system for
example.
The most famous is the integer quantum Hall state on the cylinder.
As you insert flux $\Phi$, flux of $0$ is the same as flux $2\pi$.
Your wave function will slowly move to the right.
Byt he time the flux goes to $2\pi$, but htere is a net winding that pumps
charge from left to right.
This is called the \emph{Thouless pump}.
It arises in the study of the integer quantum Hall effect on a cylinder.

Recently, there has been some work where the idea of a toological pump is vastly
generalised for QFT.
There's a paper called Anomalies in the couplling constants that abstracts this
notion.

8. Effective action.
Your effective action depends on some set of fields, which has topological terms
in the effective action.
.
9. 't Hooft anomaly.
Every QFT has one.
This is invertible TQFT, cobordisms and cohomology.

10. 
Metals have a Fermi surface that separates occupied and unoccupied states.
That surface can have a topology.
In a 3D metal, the surface is a closed 2D surface,
so it could be a sphere, a torus and so on.
In apritcular, there' s an invariant called the Euler characteristic that
distinguishes these possibilities.
THere's a cool paper that says the Euler charctrsic shows up in ballistic
conductance.
No disorder, just pure sytem, add contacts, measure conductance, and it turns
our the euler characteristic turns up in that measurement.
It's well-knonw in 1D but this paper generalizes that.

11. Early attempt to 
This is a failed attempt.
Lord Kelvin and Pepter Tate attempted to think of atoms as knots in the aether.
Cool idea, because as knot is stable and has a well-defined character.
But it went down the drain when the discovered there is no aether.

Another places is when we think of world-lines of particles in 3D spacetimes.
Depending on the topology of knots and links.
It could depend only on the topological invariance of knots and links.
It comes in up in 2+1D topological quantum field thoereis.

In higher dimensions, we need membranes, world-volumes and more non-trivial
kinds of knotting.

That's it for my overview.

Any other ideas?

Studnet: Feynman diagrams planar and non-planer.

I hope the lesson is that topology shows up everywhere.
The mathematics can be veyr differnet, the actual thing you're tlaking about
cane very differnt, so you have to clear.
It's maiing about characterirising toplogical phases of matter using TQFT.
Many topological phases of matter can be understood simply in terms of
topological band theory.


\subsection{Local quantum systems}
When startin fof a sstudent learning QM and may=body QM, at some point it may
occur to you a Hamlitoinin is just a big matrix and why is it differnt ot Linaer
algebra.
The fundamental didferencis the concept of locality.
We're intersted in systems with some notion of locality, often geometry.
W'ere interested in sum of linear terms.
Wihtout locality we hayve nothing to say.
It's locality in a many-body system.
What are some basic properties local quantum systems have we can demonstrate on
general grounds.

The first thing is the Hilbert space and hte locla tensor product decomposition.
We have some quantum ssytme with some hilbert space.
We're interested in Hilbert spaces that factorize
\begin{align}
    \mathcal{H} = \bigotimes_{i=1}^{N}\mahtcal{H}_i
\end{align}
where $i$ labels sites on a graph $\Lambda$, for example a lattice.
A general graph is useful.
$\mathcal{H}_i$ is the local Hilbert space, which might be finite-dimensional,
in which case it's isomorphic to $\matbb{C}^k$.
This could be useful for a spin system for example,
so spins that can be up or down, 
or hardcore bosons, I can have bosons with hardcore repulsion.
Or it could be an infinite-dimensional space like a harmonic oscillator at that
particular site.

If we have fermions, then this $\mathcal{H}_i$ will be a \emph{graded vector
space}
which means we have to specifiy which staes have odd or even fermion numbers.
And $\mathcal{H}$ is a fermionic Fock space.
Fermions don't commute, so you have to be careful to take care of the
non-locality and anti-commutation.

I immediately have a system essentially discrete.
This is not a continuum system.
To the extent we're interested in continuum systems,
we're only interested in long-wavelength approximation.
It's hard to directly define this tensor product in the continuum,
not sure what the best way is.
If you want to do it you wind up in the world of operator algebras.

We'll be interested in continuum systems that have UV completion on $\Lambda$
iwth $\mathcal{H}$.
This menas if I go to really small scales, this is what my system actually looks
like, but if I look at low-wavelneght regimes, this continuum systme is
sufficient to describe.
Often we write down fermions in the conituum like quadaratic dispersion or
llinear band structure.
These all arise from long-wavelenth approximations of fermions hopping on a
lattice.y
That's kind of what we have in mind.

Imporatntly, not all quanutm systmes have UV completion, not all QM systems
have a tensor product factorsiation.

Q: Any examples?

Yes, not all quanutum systems factorize.c
An example is pure gauge thoery, like Maxwell theory with no electric charges,
just electric and magnetic fields.
There's Gauss law constraint where div is 0.
Such a pure gauge theory does not have all local tensor product decomposition,
because there is a constraint that has to be satisfied at every point.
You could put gauge theory on a lattice to get lattice gauge theory.
In general you can have constrained quantum systems where the Hilbert space does
not factorize into such a nice form.

This leads to the question of \emph{emergeability}:
Which QFTs can arise as long-wavelength approximations to a system that has this
local tensor-product factorisation.
For exmaple, there's a lw that all guage thoeries can emerge from such systems
iwth local tensor factorisations. 
However, the guage thoery needs to have charges in every representation of the
guage group.
There's a notion of completeness for that guage theory to be emergable.
You can think of this as emergeable from qubits if you like.
What's a qubit?
Just a local 2-state system.
This is just saying my Hilbert space is just a bunch of qubits.
The questions is whatk ind of QFTs can emerge from qubit models.
We know some QG models can arise frmo qubit models.
QG models can arise from super-Yang Mills theory and that can arise from some
interacting qubit thoeyr.

We don't know how to write the standard mdoel starting from this construction,
because it has chiral fermions, neutrinos and we don't know how to put that on a
discrieitsed structure.
Some people believei t can be done, but no one has figured it out.

So the questions is: How to get standard model of particle physics.

This questions has received attention later with TQFTs and anomalies and it has
been useful thinking about hits question.

Q: Is there a verison of the tandard model thatl ooksl ike the standard model
without neutrinos?

Yes you can put QCD on a lattice.
You can just put EM on a lattice.
But if you want everything including neutrinos,
then you have to deal with chirality.
It's sometimes inpossible to get on a lattice, but sometimes possible but very
hard.

Q: Cna you do QCD wiht weka force?

The weak force is the issue because it has neutrinos.

THer'es 2 statements: Can you usefully approximate the standard mdoel by
numerical methods? Yes but it's not honestly done.
You could do domain-wall construction, with a guage field that leaks ot hihger
dimensions.
There are tricks you cna play to get numerics.
But if want non-perturbative approximations, then no.

The systems that have some chiral fermions are on the bounadries of some
systems.
You don't want the guage field to leak out ot higher dimensiosn.

Q: Is there a relation between renormalzabilit and emergeability.

No. Don't know.

Emerge ability is a new word.
It's natural from a condensed matter perspective.
If you're big into everything comes from qubits,
then this is exactly what you're interested in.

Let me define now local Hamiltonian.

Before I do that, we have a graph $\Lambda$.
We need geometry so we need to define a distnace function dist$(i,j)$
that is symmetric and satisfies the triangle inequlaity.
The most useful thing to do do is say the distance between two points is hte
lenght of hte minimum path.

If you have a set $Z\subset \Lambda$,
then the diameter of $Z$ is $\mathrm{diam}(Z)=\max_{i,j}\mathrm{dist}(i, j)$.
Then
\begin{align}
    H = \sum_{Z\subset\Lambda} h_Z
\end{align}
where each $h_Z$ only has support on $Z$.

We're interested in geometric locality.

Q: Do we need $Z$ to be finite diameter?

Well we're interested in $N\to\infty$.
I'm not done yet.
y
Geometric locality means $|h_Z|$ decays fast enough.

\begin{align}
    |h_Z| =
    \begin{case}
        0 & \text{if }\mathrm{diam}(Z) > k\\
        C e^{-\mu \mathrm{diam}(Z)}\\
        (C/\mathrm{diam}(Z))^\alpha
    \end{case}
\end{align}

Power laws do occur, but often you have screening which essentially makes it
exponential.

Q: What is hte norm of an operator?
\begin{align}
    |\mathcal{O}| = \max_{\ket{\psi}} |\mathcal{O}\ket{\psi}|
\end{align}
If $\mathcal{O}$ is just a matrix acting on some finite-dimensional space, them
this is the maximuum eigenvalue of $\mathcal{O}$.

It turns out this isn't everything we want in the most general case.

There's something useful to add.
$\Lambda$ is some general graph,
but we do not want something with a single site talking to an extensive number
of sites,
because that could be pathological.
What we want is that every site is only involved in a few number of terms in the
Hamiltonian.
that's an additional thing.

For each $i$,
\begin{align}
    \sum_{X|i\in X} ||h_X|| \le C
\end{align}
so a single site cannot be involved in too many terms.

If you're interstedi nin  QEC, you're interested in this too.
If it's just a regular lattice like a square lattice, this is going to be
naturally bounded.

It's useful to have in mind that we can relax geometric locality and instead
just htink about \emph{sparsity}.
Here we could say that we want each term in the Hamiltonian to have low weight.
So we say that $||h_Z||\le Ce^{-|Z|\mu}$ where $|Z|$ is hte cardinality of hte
set.
We just want it to be a sum of low-weight terms,
that $H$ only has support over a low number of sites,
but we don't care where the sites are.
And we also have
\begin{align}
    \sum_{X|i\in X} \le C.
\end{align}

Sparsity shows up in the Sachdev-Ye-Kitaev model.
This mdoel is
\begin{align}
    H = \sum_{ikl} J_{j,k,l} \gamma_i\gamma_j\gamma_k\gamma_l
\end{align}
where the sum is random coupling.
They're low weight, only 4, but there's no geometric locality.
This actually faiils that constartin, but there's naother version that does.


Anohter example is finite-rate quantum error correcting codes.
Where you encounter Hamiltonains with no geometric locality,
but they do have this sparsity term.

Q: SYK have support on 4 sites, how is that not local?

Each term could be a random set of 4 sites.
I don't care about what's far and close.

Some basic properties about local Hamiltoanins.
Lee-Robinson bound.
Even though this is not relativistic,
there is an effective light cone, where $x=_{LR} t$,
the correlations will only be appreciatble inside the light code,
and will decay exponentially outside the light cone.
So there's a light cone even if there's no relativity.
This Lee-Robinson bound bounds the spread of information propagation in a local
system.
The minute you define local, you can prove htis bound.
It's extrememly powerful and geenreal.

Gapped Hamiltonian implies finite correlation length.

Q: Where is the syllabus.

Do you know elms.umd.edu.
