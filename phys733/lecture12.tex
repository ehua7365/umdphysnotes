\section{Massive Dirac Fermion Model}
The dispersion relation $E(k_x, k_y)$ looks like a cone.

New band touching,
we get Dirac fermion
\begin{align}
    H &=
    \int d^2 x\,
    \Psi^\dagger\left(
        -i\sigma^x v_x D_x
        - i\sigma^y D_y
        + m\sigma^z
    \right)\Psi
\end{align}
where
\begin{align}
    D_j &= \partial_j - i A_j
\end{align}
We can rescale space to get isotropic velocity
\begin{align}
    y \to \frac{v_y}{v_x}y
\end{align}

In the Lagrangian formulation
\begin{align}
    S &= \int d^3x\, \mathcal{L}
\end{align}
the Lagrangian density is
\begin{align}
    \mthcal{L} &=
    \overline{\Psi} \left( i \slash{D} - m \right) \Psi
\end{align}
where we use Feynman slash notation
\begin{align}
    \slash{D} = \gamma^{\mu}D_{\mu}
\end{align}
and
\begin{align}
    \overline{\Psi} &= \Psi^\dagger \gamma^0
\end{align}
and the Dirac matrices here are
\begin{align}
    \gamma^0 &= \sigma^z\\
    \gamma^1 &= i \sigma^x\\
    \gamma^2 &= i\sigma^y.
\end{align}
In continuum,
can also define on curved space-time a metric $g_{\mu\nu}$
with action
\begin{align}
    S &= \int d^3x\,
    \sqrt{|g|} \mathcal{L}
\end{align}
where you can diagonalize the metric with orthonormal frame fields
\begin{align}
    g_{\mu\nu}^a \eta_{\alpha\beta} e_{\nu}^b
\end{align}
and we use the Minkowski metric
\begin{align}
    \eta &=
    \begin{pmatrix}
        1 & &\\
        & -1 &\\
        & & -1
    \end{pmatrix}.
\end{align}
The group of transformations which leaves the metric invariant is
$SO(2, 1)$, which means there is a redundancy in this description,
as transformations like
\begin{align}
    e_{\mu}^a \to \sum_{b} W_{ab} e_\mu^b
\end{align}
where $W\in SO(2,1)$ describe the same thing.
This gauge symmetry leads to a $SO(2, 1)$
non-Abelian gauge field
\begin{align}
    W_\mu^{ab}
\end{align}
which is also called the ``spin connection''.
The holomony of this gauge filed will tell you how you wind up rotated as you go
in a closed loop.
It tells you how to do parallel transport in your space to relate frame fields to
each other in different points in space.

So the field strength of the spin connection will then give you the curvature of
spacetime.

The way you couple Dirac fermions to space-time curvature is to couple them to
this non-Abelian gauge field,
the spin connection.

\begin{question}
    What was the redundancy?
    If $W$ is just a redundancy,
    how can it define a field?
\end{question}
When you have $U(1)$ gauge symmetry,
you have
\begin{align}
    \psi \to e^{i\theta} \psi
\end{align}
Then we gauge the global symmetry.

\begin{question}
    Although it's called spin connection,
    it seems only related to the $SO(2, 1)$ spacetime,
    but not the field that lives on the spacetime?
\end{question}
What you're alluding to will come up next,
how to couple the field to the Dirac spinor,
during which we must pick a representation,
and we have to pick the spinor representation.

\begin{question}
    How will we be gauging this?
    These are local transformations to begin with,
    do we need to gauge them?
\end{question}
It's like turning on gravity.
When you turn on $g$,
you have curved spacetime,
and to describe it,
it's turning on a gauge field associated with $SO(2, 1)$.
It gives a way of thinking of curvature as an ordinary gauge term,
of $SO(2,1)$ in this case.

\begin{question}
    If you want your theory to be invariant,
    there should be some cancellation of fermionic field,s
    and since the field is present,
    it can demonstrate the curvature of spacetime.c
    But sine the $W$ is just a local Lorentz transformation,
    it doesn't carry.
\end{question}
I'm not spelling this out,
I'm just sketching the general thing.
Just to see what fields are involved in the description.
So fermions should couple to this gauge field $W_\mu^{ab}$.
So the covariant derivative $D_\mu \psi_\alpha$.

The spinor should be written as
\begin{align}
    \Psi =
    \begin{pmatrix}
        \psi_1\\
        \psi_2
    \end{pmatrix}
\end{align}
and then we want to couple to the gauge field the following way.
\begin{align}
    D_{\mu}\psi_\alpha &=
    \left( \partial_\mu - iA_\mu \right)\psi_\alpha
    + \frac{1}{2} W_\mu^{ab}\left( S_{ab} \right)_{\alpha\beta}\psi_\beta
\end{align}
where the $S_{ab}$ is the spinor representation of the Lorentz group,
can be written in terms of Gamma matrices.

So now we can consider integrating out the fermions.

By integrating out the fermions,
I mean we do something like this.
\begin{align}
    e^{iS_{\mathrm{eff}}[A,W]} &=
    \int \mathcal{D}\overline{\Psi}\,\mathcal{D}\Psi\,
    e^{iS[\Psi, A, W]}.
\end{align}
It will be homework,
but the point  is,
to lowest order,
you get Chern-Simons terms.
\begin{align}
    S_{\mathrm{eff}} &=
    \frac{1}{4\pi} \frac{\sgn(m)}{2}
    \int \epsilon^{\mu\nu\lambda} A_\mu \partial_\nu A_\lambda
    + \frac{\sgn(m)}{2} S_{\mathrm{CS,grav}}
\end{align}
where the second term is the gravitational Chern-Simons term for the non-Abelian
gauge field.

And this term is
\begin{align}
    S_{\mathrm{CS,grav}} \propto
    \int\Tr\left( w\wedge dw + \frac{2}{3} w\wedge w \wedge w \right).
\end{align}
Changing the sign of the mass gives
\begin{align}
    \Delta S &= \frac{1}{4\pi} \int A\, dA
    + S_{\mathrm{CS,grav}}
\end{align}
The point of this was to see what the effective action is from the point of view
of the Dirac fermions and see how we can reproduce the
$A_\mu \partial_\nu A_\lambda$
Chern-Simons term and that we also get this over gravitational term.

Every time we describe topological phases with free Fermions,
there's always some way to also describe it in terms of Majorana fermions
or Dirac fermion.

\begin{question}
    Why is there a 1/2 factor in
    \begin{align}
        \frac{1}{2}W_\mu^{ab} \left( S_{ab} \right)_{\alpha\beta}\psi_\beta.
    \end{align}
\end{question}

Now we have a Dirac fermion of 2+1D and we can chase the phase we're in by
changing the mass.
The same thing happens in other dimensions,
and by changing the mass term,
you can go through different topological phases.

\subsection{Chiral fermion on domain wall in mass}
If you have a system with boundaries,
you can have chiral fermion modes.
Because changing the sign of the mass changes the Chern number,
because it changes the effective action by one unit of the CS term.
It turns out,
at the domain wall,
we're going to have exponentially localized to the domain wall,
that's gapless and chiral.
Let $m(x)$ is negative for negative $x$ and positive for positive $x$.
It reduces to the Jakiw-Rabbi soliton in 1D.
The Hamiltonian is
\begin{align}
    H &=
    \int dx\, dy\,
    \Psi^\dagger\left(
        -i\sigma^x \partial_x - i\sigma^y \partial_y + m(x) \sigma^z
    \right)
\end{align}
where I set the velocities to 1.

For a plane wave
\begin{align}
    \Psi(x, y) &= \Psi(x) e^{ik_x y},
\end{align}
the eigenvalue equation is
\begin{align}
    \left( -i\sigma^x \partial_x + m(x) \sigma^z \right)\psi
    &=
    \left( E_{k_y} - k_y \sigma^y \right)\psi.
\end{align}
If you remember what we did in the Jakiv-rabbi Soliton,
there was a zero-energy solution where
$E_{k_y} + k_y \sigma^x$ was the solution
and it was exponentially localized to the domain wall.

Recall in the J-R soliton,
there is a $z$ mode solution $\psi_0$
with
\begin{align}
    \sigma^x \psi_0 = \lambda \psi_0
\end{align}
where $\lambda = \pm 1$.
This zero-energy solution means that we get a solution localized to the domain
wall with an energy
\begin{align}
    E_{k_y} &= -\lambda k_y.
\end{align}
If we had a domain wall where $m(x)<0$ on one side
but $m(x)>0$ on the other side, 
we have a propagating fermion exponentially confined to the domain wall.

We have a chiral anomaly and a gravitational anomaly,
which is partially responsible for a quantized thermal quantum hall effect.

The $(1+1)D$ chiral fermion has a
$U(1)$ chiral anomaly,
which is a t' Hooft anomaly.
There is also a gravitational anomaly.
Both of these anomalies can be cancelled by the
$(2+1)D$ bulk,
which is the Chern insulator.
This is an example
of how you have anomalies in one dimension,
but are cancelled in higher dimensions.
Related to this is the fact that the chiral fermion is gives a quantized thermal
hall conductance.

The gravitational anomaly and the quantized thermal Hall conductance
defines a chiral central charge $C_{-}$.
It is the same as the Chern number for free fermion systems,
but that breaks down for interacting systems.

If you view them as boundaries of a Chern insulator,
there's no longer anything wrong with the anomalies.

U(1) chiral anomaly

The effective action for Dirac fermions is
\begin{align}
    S_{\mathrm{eff}} &=
    \int A\, dA
    + S_{\textrm{CS,grav}}
\end{align}

Then if we turn on the electric field
\begin{align}
    \frac{dp}{dt} &= E
\end{align}
and suppose we increase the momentum by $\Delta t= \int E\, dt$.
Imagine putting it on a  ring with periodic boundary conditions.
$p$ is quantized in units $2\pi/L$.
Then the charge is
\begin{align}
    \Delta Q &=
    \frac{\Delta p}{2\pi /L}\\
    &= \frac{L}{2\pi} \int E\, dt\\
    &= \frac{1}{2\pi} \int E\, dy dt\, j
\end{align}
and also nothing that the charge is just
\begin{align}
    \int \partial_t n \, dt\, dy
\end{align}
where $n$I s the charge density,
you get
\begin{align}
    \partial_t n &= \frac{1}{2\pi E}.
\end{align}
The covariant versions of this is
\begin{align}
    \partial_\mu j^\mu &= \frac{1}{2\pi}\epsilon_{\mu\nu}F^{\mu\nu}.
\end{align}
The key is that some conservation law breaks down when you turn on the field.

Suppose you have a cylinder with boundaries 
The bulk cancels the anomaly because it tells you exactly where the charge comes
from.
If you have both edges at the same time,

Include both edges.
\begin{align}
    \partial_\mu j_{\nu}^{\mathrm{axial}} &=
    \frac{1}{2\pi} \epsilon_{\mu\nu} F_{\mu\nu}..
\end{align}
Sometimes axial current is called chiral current.

Earlier in the course I said you can have 't Hooft anomalies
which allow you to couple to the background gauge do
Trying to $A_\mu$
\begin{align}
    S_{\mathrm{coupling}} &=
    \int A_\mu j^\mj
    = \to \int_{} \left( A_\mu - \partial_\mu f \right) \gamma^\mu\\
    \delta S &= \int f \partial_\mu j^\mu &= 0
\end{align}

\subsection{Gravitational Anomaly}
To couple a field theory to gravity,
that means we want to couple it to some spacetime metric.
If we want to couple to $g_{\mu\nu}$,
we need a conserved stress energy tensor $T_{m\nu}$
which is conserved meaning
\begin{align}
    \Delta_\mu T^{\mu\nu}.
\end{align}
And the gravitational anomaly is a breakdown of this equation.
For example,
for the chiral fermion,
\begin{align}
    \nabla_\mu T^{\mu\nu} \propto C_{-}\epsilon^{\nu\sigma}\partial_{\sigma}R
\end{align}
where $C_{-}$ is the chiral central charge
and $R$ is the space-time curvature.

The next thing I want to describe is the last property,
which is the quantized thermal Hall conductance.

\subsection{Quantized Thermal Hall Conductance}
Suppose we have a $(2+1)$D Chern insulator region
where $C=1$,
which is coupled to a thermal bath at temperature $T_1$ with $C=0$
and the other side is coupled to a thermal bath at temperature $T_2$.
So we have a \emph{thermal current}
\begin{align}
    \vec{j}_Q = -\kappa \vec{\nabla} T
\end{align}
But here there is a quantized thermal hall conductance
\begin{align}
    \kappa_H &=
    \frac{\pi^2}{3} \frac{k_B^2}{h}T
\end{align}
per chiral edge mode
so the quantized thermal current is
\begin{align}
    j_{Q,X} &= -\kappa_H \Delta T.
\end{align}

The intuition is as follows.
If you couple the top to some temperature,
you're creating a thermal density of excitations,
where the average energy is $k_B T_1$
and same for the other side.
But because it's chiral,
you have a flow of energy along that boundary,
and a flow of energy along the other boundary in the opposite direction,
and if the temperatures and thus energies different,
there is a net flow of energy.

Consider a single chiral edge mode at temperature $T$.
The average energy density
\begin{align}
    n_Q &=
    \int \frac{dk}{2\pi}
    \left( \varepsilon_k - \mu \right)\left[ 
    \frac{1}{e^{\beta\left( \varepsilon_k - \mu \right)} + 1}
    - n_{FD;k}(T = 0)
    \right]
\end{align}
where the energy is $\varepsilon_k = \hbar v\cdot k$ and the thermal current is
\begin{align}
    j_Q = v\cdot n_Q = \frac{\pi^2}{6} \frac{k_B^2}{h}T^2.
\end{align}
For two edge modes,
one going to the right with current
\begin{align}
    j_{Q,1} &=
    \frac{\pi^2}{6} \frac{k_B^2}{h} T_1^2
\end{align}
and one going to the left
\begin{align}
    j_{Q,2} &=
    \frac{\pi^2}{6} \frac{k_B^2}{h} T_2^2
\end{align}
so the total current is their difference
\begin{align}
    j_{Q,\mathrm{tot}} &=
    \frac{\pi^2}{6} \frac{k_B^2}{h}
    \left( T_1^2 - T_2^2 \right)\\
    &=
    \underbrace{\frac{\pi^2}{6} \frac{k_B^2}{h} 2T_1 \Delta T}_{\kappa_H}
    + \mathcal{O}\left( \Delta T^2 \right)
\end{align}
where $T_2 = T_1 + \Delta T$.
In general,
\begin{align}
    \kappa_H &=
    C_{-} \frac{\pi^2}{3} \frac{k_B^2}{h}T
\end{align}
and here the chiral central charge $C_{-}$ is the number of right-moving chiral
fermion modes minus the number of left-moving chiral fermion modes on a given
edge.

The reason we introduce this is because
Majorana fermions are half a complex fermion,
so it has a $\frac{1}{2}$ chiral central charge.
In fact,
it's much more general,
and has to do with a conformal field theory.
More complicated topological quantum phases of matter
with boundaries not described by chiral fermions,
but there is still a chiral central charge.

\begin{question}
    About the $U(1)$ anomaly there's a way to see the bulk.
    Is there a way to see how the gravitational anomaly comes from the bulk?
\end{question}
The gravitational anomaly,
you can do calculations and put your system on a curved spacetime,
you see extra energy,
and if you write the extra CS term,
it will account for the non-conservation of the stress energy tensor.
But you won't be able to relate it to the quantized thermal Hall effect,
because there is no boundary.
The boundaries are supposed to be moving as we adiabatically insert flux.
You can't really relate it to the thermal hall effect in a clear way.

\begin{question}
    Is the only way you can have these chiral fermions is in relation to the
    bulk?
\end{question}

\begin{question}
    Is the gravitational anomaly what leads to the quantized hall conductance?
\end{question}
I don't have a good explanation except that $C_-$ appears in both cases.
Just like how we derive conductivity by turning on an electric field,
you can turn on thermal conductivity by turning on a gravitational field.
In the context of the quantum Hall effect,
there have been many studies,
but I don't think it led to a direct connection between the thermal Hall effect
and the gravitational response.
From my perspective,
there's central charges appearing in both places.
There might e ways of introducing thermal conductivity on the edge
by introducing gravitational potential.
There's a fundamental mismatch,
because the bulk can never have thermal quantum hall effect.
It's possible other people might have a better answer.

\begin{question}
    A cylinder with a gradient around the cylinder?
\end{question}
That doesn't even make sense.
You can't really put a temperature gradient on the boundary of this thing
because it's chiral.
I didn't put a gradient along the edge,
I don't think it's even something you can do.
