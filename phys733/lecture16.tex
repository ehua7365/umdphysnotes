\section{Lecture 16}
In (1+1)D there are two types of anomaly.
Chiral anomaly $U(1)$, characterised by some integer, the Chern number.
Gravitational anomaly, characterised by Chiral central charge.
They are different,
but the minimal example of a theory with a chiral anomaly is a 1D complex
fermion.
A chiral fermion will have chiral anomaly.
When they say chiral anomaly,
they mean two different things.
If you have a left and right mover,
there is a symmetry corresponding to the left and right,
sometimes called the axial current.
The fact there's a chiral anomaly here
means that if you take two copies,
that anomaly will persist.
If you take a single chiral fermion,
and it has this gravitational anomaly.

If you have a general interacting CFT,
these don't have to be the same thing.

Hopefully I'll put up another homework assignment to cover what we've done since
last homework assignment.
We've been talking about topological insulators,
discovered by Charlie Kane who's here today.

Let me clarify some things I didn't explain,
then we'll go to 3D.

\section{2+1D case}
Here we have a topological insulator.
The fermionic symmetric group was
\begin{align}
    G_F &= U(1)^f \rtimes \mathbb{Z}_4^{T,f} / \mathbb{Z}_2
\end{align}
$U(1)$ is the same as fermion parity.
Time reversal also square the fermion parity.
This is class AII.

Then then are matrix elements of the time-reversal operator.
\begin{align}
    w_{m,n}(k) &=
    \bra{U_m(k)} \tau \ket{U_n(-k)}
\end{align}
and there are 4 time-reversal invariant momenta (TRIM) labelled $\Lambda_a$.

So we have $(0,0)$, $(\pi ,0)$, $(\pi, \pi)$, $(0, \pi)$.
Then we have $\mathbb{Z}_2$ topological band invariant.
\begin{align}
    \left( -1 \right)^{\nu} &= \prod_{a=1}^{4} \delta_a
    \prod
\end{align}
where
\begin{align}
    \delta_a &=
    \frac{\Pf(w(\Lambda_a))}{\sqrt{\det(w(\Lambda_a))}}
\end{align}
and we need to pick Bloch wave functions $\ket{u_n(k)}$ such that
$\sqrt{\det(w(k))}$ is continuous.

For 1D systems,
time-reversal polarization
\begin{align}
    (-1)^{P_T} &=
    \prod_{a=1}^{2}\delta_a = \delta_1 \delta_2
    = \delta_` \delta_2
\end{align}
When the time-reversal polarization is odd,
it has a Kramers degeneracy on the boundary,
and none if it's even.
A 1D system is not a well-defined thing,
because putting an electron at the end will ad a Kramers degeneracy.
When you do a gauge transformation the Bloch wave functions,
you can actually change $P_T$ from $0$ to $1$.

$P_T=0,1$ tells us if the boundary of a 1D system has Kramers degeneracy,
but there is a gauge transformation taking
$\ket{u_n(k)}\to e^{i\theta_k}\ket{u_n(k)}$.
So there is some phase change of the Bloch wave function that will change $P_T$
from $0$ to $1$.

There is an integer ambiguity of the charge polarization in 1D.
When we talk about Chern number,
the Chern number tells us how polarization changes affects the 1D changes.
If we had a 1D system,
if we insert flux,
the polarization as we insert flux changes from 0 to 1,
means we pump change from one end to another.
Adiabatic changes to polarization around a cycle are well defined and define
a topological pump.
Whether it's 0 to 1 in a given 1D system is not useful,
however if you think of a 2D system this way,
when you insert flux, or tune $k_y$,
the fact that the time-reversal polarization is changing between 0 and $\pi$
is the basic effect that describes what's going on here.
In equations,
this invariant is the same as
\begin{align}
    (-1)^{\nu} = (-1)^{P_T(k_y=0) - P_T(k_y = \pi)}
\end{align}
And it could be a plus.

What does it mean for the surface stats.
If $\nu$ is odd,
we have helical edge modes with counter propagating complex fermions as I said
last time.

Now, I want to explain the connection to the surface states a little more
clearly.

\section{Connection to Surface States}
The equality I didn't derive last time
\begin{align}
    (-1)^{\nu} = (-1)^{P_T(k_y=0) - P_T(k_y = \pi)}
\end{align}
I didn't describe,
so I might put that in your homework.
What does it actually mean?

We conclude that we have to have gapless surface states if the symmetry remains
intact.
Taking $k_y$ from 0 to $\pi$.
Remember $k_y + A_y$ enters the equation when we insert flux.
So $k_y + A_y$ is what enters the Hamiltonian.
So
\begin{align}
    A_y &= \Phi/L_y
\end{align}
Taking $A_y$ from $0$ to $\pi$ is equivalent to taking
$\Phi$ from $0$ to $\pi L_y$.
The cylinder is along $x$.

$2\pi$ flux is equivalent to no flux,
so this is just $\pi$ flux if $L_y$ is odd.
So take $L_y$ odd and consider inserting $\pi$ flux.
$\phi=0\to \pi$.

If we start with a Kramers singlet on a boundary,
we need to get to a Kramers doublet.
What this tells us is that if we start off with a non-degenerate ground sate,
we adiabatically insert flux.
If $\nu$ is odd,
we better get to a Kramers doublet.
That immediately tells us it's gapless?

Why?
Suppose it's gapped.
Then inserting flux can only change the ground state energy by an amount
$\Delta E \propto e^{-L_y/\xi}$.
So that means if it starts off in a Kramers singlet,
there's no way it could become a Kramers doublet.
It has to remain a Kramers singlet.

So if the boundary is gapped,
there are 2 possibilities.
Either this $\nu=0$ is trivial,
so a singlet stays singlet.
Or, we break time-reversal symmetry,
so it doesn't make sense to talk about Kramers doublet or singlet anymore.

If we break $U(1)$ charge conservation,
we have asuperconductor on the boundary,
inserting flux is well defined anymroe if you want to stay on the ground state
manifold.
If I insert flux, it gives lare enrgy cost.

We can conlude that if $\nu=1$ mod 2,
then either
\begin{enumerate}
    \item The boundary is gapped and symmetry-breaking.
    \item The boundary is gapless, so it can preserve the symmetry.
\end{enumerate}
If it's gapless it actually may or may not preserve the symmetry,
but the point is that if we want to preserve the symmetry,
the boundary must be gapless.
Remember this picture where I drew $E$ as a function of $k_y$,
where we had one bad down here and one bad up there.

At $k_y=\pi$,
they had to be Kramers degenerate,
and these Kramers pairs split at $k_y=0$.
There are two ways they can connect going from $k_y=0\to \pi$.
Either they can switch partners like this [picture].

Or they don't switch partners.
In the case of no switching pairs,
there is symmetry,
and if it's gapped, it's in conflict,
so it must be the pair switching case if $\nu$ is odd,
and not switching if $\nu$ is even.

This establishes a connection betwen the bulk nivariatn $\nu$ and the way the
edge modes connect on to the boundary,
which guarantees these robust edge states.

You could go further.
Suppose you had one of these situations,
can you convinceyouself that the insertion of $\pi$ flux would change the bands.

Now I'm going to draw the bnads from $-\pi$ to $\pi$.
Suppose your chemical potential is exactly at this point,
so htis state which isa Kramers doublet,
and bothsates in the doublet are occuppied.
And then all the states below that chemical potential are also occupied.
So then you can insert the $\pi$ flux,
which shifts all tese states over to hte right by $\pi/L_y$.
And we get a picture where one of these states goes up a little,
and hte other states shifts down.
But here there's an empty state.
So this guy becomes empty and this guy becomes filled.
Under $\pI$ flux,
these states hsfit over,
and you have occupied and unoccpied states.
This start is Kramers doublet because there's an exact anaolgous state with a
Kramers degeneracy with a degeneracy.
You can draw these diagrams in all scenarios and convince yourself that iserting
$\pi$ flux changes a state from an even number of doublets to an odd number of
doublets, and vice versa.
That makes the surface and bulk correspondnce even more explicit.

I wanted to fill in this hole last time.
I didn't make a connecino with suraace staes.

THis (1+1)D helical boudnary,
in the Chern number case,
we had a U(1) anomaly.
Something has ot cancel out the anomaly,
which si the bulk.
Here the anomaly is a mix between $U(1)$ and time-reversal.
If we want to preserve both of them so these gapless modes are robust,
that's a signature that we hav ea mixed anomaly between U(1) and TR symmetry.
The fact you insert $\pI$ flux and the Kramers degeneracy of the ground state
chagnes,
you can think of it as a symptom of the anomaly.
So the ground sate went from beinga Karmers singlet to Kramers doublet.

If you take this helical theroy,
noe way to see hwat's wrong is the study it on a spacteim that iss Klein bottle.
It's useful to put sapceteime on a non-orientable manifold,
it turns out you can also in that case see more things wrong and inconsistent
with the theory.
Just like the chiral fermion, thei nconsistency wwas curretn conservation breka
down.

Seeing anomalies wiht discrete ymmetries is a muc hmore complicated game poeple
only started to understand 10 years ago.
Studying on a Klein bottle is one thing you could do to se the anomaly.
But I'm not going to go down that path here.

So now let's go to (3+1)D.
We're still in Class AII and TI still squares to -1.
And here we have 8 TRIM,
which I write as
\begin{align}
    \Lambda_a=(n_1,n_2,n_3)
    =
    \frac{1}{2}n_1 \vec{b}_1
    + \frac{1}{2}n_2 \vec{b}_2
    + \frac{1}{2}n_3 \vec{b}_3
\end{align}
where the $b$s are primitive reciprocal lattice vectors and $n$s are iether 0 or
1.
This defines a 3D topological insulator.
The topological invariant is
\begin{align}
    (-1)^\nu &= \prod_{a=1j^{8}}\delta_a.
\end{align}
Consider a cubic lattice.
Consider a surface in the $z$ direction.
We can look at hte surface states.
At the time-reversal invaritn mometna,
at a stte in this TRIM,
we have Kramers degenercy,
but as soon as we go away,
the bands split.
So that means we have a Dirac node,
when we locally have some state at some energy.

So surface states at $(k_x,k_y)=(0,0)$, $(0,\pi)$, $(\pi, 0)$, $(\pi, \pi)$
come in Kramers pairs,
and that means we have Dirac cones every time we have states with these momenta.

We can draw a picture.
We have these different points here,
these TRIM,
and wheneter you have some state that comes iwth degeneracy,
there's a bunch of dirac cones when we hvae some staes at these moemnta.
For exmplae
at this momenta,
we have Dirac cones at vairous energis.
And just like this dimension,
just like here,
it was a question of how the states connect up.
Similarly here,
it's a uestion of how these diract cones all connect up with each other.
This index $\nu$ ultimately gives informatino about the way these cones connect
up to each other.

The fermi level could be anywhere.

The question is how do these cones connect up.
And basically hte answer is that if $\nu=1$.
Let me mention nadi mportant property of Dirac cones,
I hsould put on the homework.
If you calculate the Bery phase of going around a dirac cone,
you get a $\pI$ Berry phase,
it's the most important property of a Dirac cone.

When $\nu=1$,
the cones connect up in such a way that wheterver your chemical potential is,
if you lok tahte Fermi suaces,
and claculate the Berry phase to go aorund all your fermi surfaces,
you get a Berry phase of $\pi$.
If $\nu=1$,
the cones connect up sothat the Fermi surface encloses an odd numbero f Dirac
cones,
or more precisely there is a $\pi$ Berry phase.

For example, suppose you had $k_x$ and $k_y$.
At each $\Lambda_a$, define
\begin{align}
    \pi_a = \delta_{\Lambda_a} \delta_{\Lambda_a + \pi \hat{z}} = \pm 1
\end{align}
Depending on the sign of this guy,
what you can argue is that every time you want to go from a filled to an empty
circle,
youhave to cross the Fermi circle.
Every time you went from 0 to $\pi$ we had to cross the band so that the Fermi
surface lies here.
It's anlogous in higher dimension,
that if you want to gor from one poitn ot another,
whether you hae to cross a Fermi surface is determined by this number here.

So far I've just told you what happens,
but I haven't explaine dhtis very much,
but I won't explain this to you,
just too much band theory.
I hope you got the idea.
It's imlar to 1D.
You want to go from one side to another,
whether you cross the band depends on some invariants,
and in this case happens to be this one.
It's more confusing to think of this 3D case.
If you are interested in understanding this point in detail,
I can recommend references.

If $\nu=1$ mod 2,
then the surface can be modelled by an odd number of Dirac cones at some
chemical potnetial nd the minimal surface model would be having just 1 Dirac
cone.

So we've writtne Dirac cones before in class, for example
\begin{align}
    H &=
    \int d^2x\,
    \left[
    \Psi^\dagger\left( 
        i\sigma^x D_x - i\sigma^y D_y
    \right)\Psi
    - \mu \Psi^\dagger \Psi
    \right].
\end{align}
Can the Fermi surface be gapped?

In this model,
time-reversla symmetry can e implmented by
\begin{align}
    \mathcal{T} &= i\sigma^y K
\end{align}
where $K$ is complex conjugatino.
In 1D it's either gapped and breaks the smmetry
or it's not gapped and preserves the symmetry.
Same thing here.

A surface is either
\begin{enumerate}
    \item Trivially gapped and symmetry-breaking.
        The things you can dd to the Hamiltonian to gap out the Fermi surface
        would be a mass term
        \begin{align}
            \delta H_1 &= m \Psi^\dagger \sigma^z \Psi
        \end{align}
        which breaks TR symmetry, but conserves $U(1)$.
        If you ahvea single massive dirac fermion,
        integrate out the fermion,
        look tah te electrmagnetic response,
        you geta Chern-Simons term in the background gauge field with a value of
        half.
        so this leads to a $\frac{1}{2}$ integer Hall conductivity,
        meaning hte Hall conducatnce is
        \begin{align}
            \sigma_H &= \frac{1}{2} \frac{e^2}{h}
        \end{align}
        So breaknig TR symmetyr leads to Hall condcityity that is half.
        But if there is an integer ambiguoulity more generally,
        it's
        \begin{align}
            \sigma_n &= \left( n + \frac{1}{2} \right) \frac{e^2}{h}
        \end{align}
        the half shift of the Landau levels is related to the Dirac cone.
        You will never see the surface in a pure 2D system,
        because in a 2D system,
        you only have an even numbero f Dirac cones.
        Having an odd number of Dirac cones gives that 1/2 conduance,
        and is non-trivial.

        Another term you could add to theHamiltonain to gap the Fermi surface is
        \begin{align}
            \delta H_2 &=
            \Delta \Psi^\dagger i\sigma^y \Psi + \mathrm{h.c.}
            \propto
            \Delta \Psi_1\Psi_2 + \mathrm{h.c.}
        \end{align}
        This term by itsefl opens a gap at the Fermi surface.
        Superconducting terms usually produced gapped at the surface.
        there's no way of introducing a gap trivially,
        by trivally I mean adding a Fermion bilinear and gapping hte surface.
        These are really the only two terms you could add.
    \item The other possiblity is that you don't hvae these terms and the
        surface is symmetry-preseriving.
        But now you have in general an odd number of Dirac cones,
        with this symmetry and time-reversal,
        and this has a ``parity anomaly''.
        And in our situtiona,
        this is really a mixed anomaly between U(1) and time-reversal.
        What this mixed anomaly means is that if you take a theory of a single
        Dirac cone,
        make it well defined,
        then you run into problems and you need to regularize it,
        and any way of regualirazting necessarily breaks U(1) or T symmetry.
        For examle, compue the harge of the U(1) symmetry,
        but you need to do a sum you need to regularize,
        but ther's no way of regularlizing wihtout brekaign T.
        You need to introduce an extra dimensino.
        Alternatively,
        you could try to get teh sign of the path integral real to be
        consistentwtih TR symetry without messing the bulk.
        This discussion is probably a lecture of its own.
        There's a really nice review by Ed Witten from 5 or 6 years ago where he
        does it and he releates it to TSE index theorems,
        which I encourage you to look up.
    \item This is new to this dimensino and wasn't discovered until after 5 or 6
        years after topological insulators were discovered.
        This is the possbility that the surface is gapped and
        symmetry-preserving,
        but the surface has a non-invertible topological order.
        Somtimes people call this ``intrinsic toplogical order''.
        This means anyonic excitaitons,
        particles iwth topological nontrivial faactional statsitics,
        and the particles act on the anyons in anomalous ways that show upin a
        2+1D system.
        We haven't discussed non-invertible toolgoical ordre yet,
        so we're not in a position to discuss this third possibility,
        there there is such a possibility.
        The ground state on a nontrivial surface is degenerate,
        in fact a topological degeneracy.
        For example, if the surface of your system isa torus,
        you would havea ground sate degenarcy coming from hte boundaries,
        likethe Majoranas.
        In 1 you also have a ground state degenearcy,
        but it is trivial.
\end{enumerate}

\begin{question}
    When it says we have to enclose an odd numbero f Dirac coenes.
    Say I sweep the fermi level,
    that means it should come in paris?
    If I sweep up the Fermi level,
    and I need to retain an odd number of Dirac cones,
    then i have to pass 2 Dirac cones at a time?
\end{question}
Just ignore ``Fermi surface enclosed odd number of Dirac cones''.
What matters is that when you go around the Dirac cone,
you have a $\pi$ Berry phase.

You have to think pretty carefully about what happens in 2D, 3D, but 3D is much
more complicated.
It's analysis you have to go through.


Those are just a few words bout hte surface.o
Bulk response theory.
I assert what the bulk response theory is .

\begin{question}
    Trivial gapping is being introduced by fermion bilinears,
    by non-trivial you mean higher order terms?
\end{question}
Let me be more specific.
Trivially gapped means that the surface does not have intrinsic topological
order.
Now,
if the surface does not have intrinsic topological order,
usually those kinds of gapped surfaces which don'th ave intrinsic gapped order
can always be cancleeleld by adding fermion bilinears.
But if you add a four-fermion term,
that could lead ot expectation values for two-fermion blinears.
That's what happesn in superconductivity.
what I really mean is more naunced.

\begin{question}
    Given a material,
    would you be able to tell from this band diagram?
\end{question}
You have to find whether the symmetry is preserved or not,
which you can convince experiennatally.

\begin{question}
    Possiblity 4 with gappless and symmetry breaking?
\end{question}
That's kind of trivial.
It could be gapless and symmetry breaking.
The reason I emphasies symmetry preserving is because that's usually the
possiblity that occurs if you preserve the symmetry.

Once you break the symmetries,
you could think of what's hapening as another manifestatino of hte anomaly.
Symmetry brekaing ofa system with an anolay is an intersting subject itself.
Usulaly it means various defects have nontrivial properties because of the
anomaly.

One thing I didn't mention.
This superconductivity is superconducting in a single fermion band.
Ify ou inteoduced supercondcutivty in a single band systme,
you expect ot et Majorana zero modes.
The vortices of this superconector $\delta H_2$ carry Majorana zero modes.
This gave the first example beyond
which doesn't have a non-trivial pairing symmetry,
you can have an odd numberof so vortices carry MZMs.

It's really a single band,
there's no spin degenearcy.

\section{Electromagnetic response}
I want to discuss the electromagnetic response.
The electromagneti response has an effective action that is topological
\begin{align}
    S_{3+1}^{top}[\theta, A] &= \theta\cdot P
\end{align}
this is the so-called $\theta$ term in gauge theory.
This is called the instanton number.
\begin{align}
    P &= \frac{1}{32\pi^2} \int d^3x\, dt\,
    \epsilon^{\mu\nu\rho\sigma} F_{\mu\nu} F_{\rho\sigma}
    =
    \frac{1}{4\pi^2} \int d^3x\, dt\, \vec{E}\cdot\vec{B}
\end{align}
Let me define the non-abelian Berry conncetion,
where we take Bloch states from different bands.
\begin{align}
    a_i^{\alpha\beta} &=
    -i \bra{\alpha,k}
    \frac{\partial}{\partial k_i}
    \ket{\beta,k}
\end{align}
Before we were taking Blcoh staes from the same band,
but now we're taking them from different bands.
the non-abelian field strength
\begin{align}
    f_{ij}^{\alpha\beta} &=
    \partial_i a_j^{\alpha\beta}
    - \partial_j a_i^{\alpha\beta}
    + i[a_i, a_j]^{\alpha\beta}
\end{align}
Then the $\theta$ term is
\begin{align}
    \theta &=
    \frac{1}{2\pi}\int d^3k \, \epsilon^{ijk}\Tr\left( 
    \left( f_ij - \frac{1}{3}[a_i, a_j] \right)a_k
    \right)
\end{align}
The trace is over occupied bands.
This is called the ``theta term'',
the ``topological magnetoelectric effect''
or the magnetoelectric polarizability.

Note that $F\wedge F$ is a total derivative because
\begin{align}
    \epsilon^{\mu\nu\rho\sigma} F_{\mu\nu} F_{\rho\sigma}
    \propto
    \epsilon^{\mu\nu\rho\sigma} \partial_{\mu}\left( 
    A_\nu \partial_\rho A_\sigma
    \right)
\end{align}
One way of thinking about this is either a surface effect,
and you can think of it as a property that does something to monopoles in your
system.
We see both of those.

Just like the Chern number arises from integrating fields strength over a
surface.
But here, w'ere integrating $F\wedge F$ over a closed $4$-manifold.

$P$ is an integer if space-time is closed.
And because $P$ is an integer and spacetime is closed,
our path integral,
or effective parition function looks like
\begin{align}
    e^{i\theta P}
\end{align}
and because $P$ is an integer,
changing $\theta$ to $\theta+2\pi$ doesn't change anything,
so in the bulk it doens't change anything
$\theta \sim \theta + 2\pi$.
We're just trying to undertand the bulk of hte sytem by putting it on a closed
sacetime manifodl,
and ther's no effect changing by $2\pi$.
Then you can ask what time-versal does.c
It doesn't do anytign to $\vec{E}$, 
but it flips $\vec{B}$.
So effeively time-reversal is like flipping $\theta$.

So, you could say a time-versed system would have a $-\theta$ response,
which means that $T:\theta \to -\theta$.

And because $\theta=0$ or $\pi$,
but are consistent with timer-reversal.

\begin{question}
    The anlog of $\theat$ in 2+1 dimensisno is anot integer,
    but now we can define something up to an integer?
\end{question}
There's a fundamental distinction betwen even and odd idmensions.
In odd dmidension,
youhave $\theat$ terms.
If you have a insulator wiht no particular symemtries,
$\theat$ can be anything.
But now we add a symmetry and we quatnzie it.
The coefficient in the CS theory was integer.
The short answer is there is a fundamental distcintion betwene odd and even
dimensions.
There's a long answer how to relate 2+1 and 3+1 dimensinos,
but I can't go into that at the moment.

As an aside,
let me say that if you're intersted in the TQFT aspect of it,
you might also be interested in not only turning on a non-trival spacteime,
but also be interested in putting it on a non-trivial spacetime so it has
curvature.
So if you include the gravatiational response,
here we have not just $F\wedge F$,
but also $R\wedge R$ where $R$ is the Riemann tensor.
So
\begin{align}
    S[A, g] &=
    \theta\cdot P
    -
    \frac{1}{48} \int \frac{1}{\left( 2\pi \right)^2} \Tr R\wedge R
\end{align}
Let me mention some physical consequences of this magnetoelectric term.

\begin{enumerate}
    \item Surface quantum Hall effect.
        Suppose that we have a domain wall of $\theta$ as a function of $z$.
        And $\theta$ goes from $0$ at $z=-\infty$ to $\pi$ at $z=+\infty$.
        So then
        $\theta(z)=\pi H(z)$ as a step function and its derivatve is
        $\theta'(z) = \pi \delta(z)$.
        Then
        \begin{align}
            \frac{1}{8\pi^2}\int\theta(z)
            \epsilon^{\mu\nu\lambda}\partial_z\left( 
            A_\mu \partial_\nu A_\lambda
            \right)
            &= -\frac{1}{8\pi}\int \partial_z\theta\cdot
            \epsilon^{\mu\nu\lambda} A_\mu \partial_\nu A_\lambda\\
            &= -\frac{1}{2} \frac{1}{4\pi}\int_{z>0}
            A_\mu \partial_\nu A_\lambda \epsilon^{\mu\nu\lambda}
        \end{align}
        and so we get the Hall conductance
        \begin{align}
            \sigma_H &= -\frac{1}{2}
        \end{align}
    \item Witten effect.
        Suppose we have a bulk cube with boundaries.
        Suppose $\theta$ goes from $\pi$ to $0$ as discussed.
        Imagine a monopole form the outside and bring it inside.
        The basic effect is that monopole carries half charge.
        One quick way of seeing that is tht the surface has half-integer QHE,
        I take the monompole iwth strnght $2\pi$,
        and this monopole feels flux $2\pI$,
        so it changes by $2\pi$.
        Becasue of the half integer efefct,
        it has to bind the half charge,
        so I get a half-charge at the surface and that needs to be compensated
        somewhere else,
        and that's in the monopole.
        That's a quick way to see the half-integer hall effect.
        You could do a few lines of algebra and actually derive it a little more
        concretely.
        Let me derive it quickly.
        \begin{align}
            j^{\mu} &=
            \frac{\delta S}{\delta A^{\mu}} =
            \frac{1}{2\pi} \epsilon^{\mu\nu\lambda\rho}
            \partial_\nu \theta
            \partial_\lambda A_\rho
        \end{align}
        Suppose $\theta$ is uniform in spce,
        but time-varying
        \begin{align}
            \partial_i j^i &=
            \frac{1}{2\pi}
            \epsilon^{itjk} \partial_t \theta
            \partial_i \partial_j A_k\\
            &= -\frac{1}{2\pi} \partial_t \theta \partial_i B^i
        \end{align}
        and this is the same as the time-derivativeof the charnge
        \begin{align}
            \partial_t \rho &=
            \frac{1}{2\pi} \partial_t \theta \partial_i B^i
        \end{align}
        So then
        \begin{align}
            \rho &= \frac{1}{2\pi}\theta\nabla\cdot\vec{B}
        \end{align}
        and so tthe electric charge is half the monopole charge
        \begin{align}
            Q_e &= \frac{1}{2} Q_m
        \end{align}
\end{enumerate}

Ther'es one other consequence I want to discuss next time,
then I'll do dimensinoal reduction from 4+1D Chern insulator.

Let me just take a few minutes and do one another thing,
which is closely replated to the Hall conductivity called the Witten effect.
