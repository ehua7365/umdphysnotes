\section{1D boson SPTs}
Last time,
we started talking about SPT states,
bosons in 1D.
We talked about AKLT states,
which is the ground state for spin-1 chains.
We saw the Hamiltonian was
\begin{align}
    H &= \sum_i \vec{S}\cdot\vec{S}_j
    + \frac{1}{3} \left( S_i \cdot S_j \right)^2 + \frac{2}{3}
\end{align}
and we saw this is the sum of projects where neighbouring spins are projected
onto the spin-2 chain.
\begin{align}
    H &= \sum_i P_{i,i+1}^{(2)}
\end{align}
and then we have spin-$\frac{1}{2}$s projected onto spin-1s on each site.
Before you do that though,
you put the left and right on different sites into a spin-singlet state,
and we so we said that this ground state,
AKLT proved that this is the unique gapped ground state
of this $H_{ALKT}$.

In particular, this ground state is actually annhilated by every single term
here,
and this Hamiltnoian is called a frustration-free Hamiltonian.

It's useful to note that the motivation for AKLT to study this was actually how
Haldane conjectured that the spin-1 Heisenberg chain is gapped.

The motivation for AKLT was the Haldane conjecture.
The spin-one-half Heisenberg chain was known to be gapless,
but Dr Haldane predited the spin-1 Hesenberg chain is gapped.
That was suprising,
bcause poeple asuemd the spin half wouldh ave the same behaviour,
but Haldane's conjectre said that spin-1 and spin-half behave very
differently,
which is intrinsically QM.

\begin{question}
    Does it have Goldstone bosons?
\end{question}
No, it doesn't order, no long range order in 1D.

It's non-trivial the spin-half chain is gapless,
but the  spin-1 half is gapped.

SO this AKLT,
you can think of this thing as smoothly connected,
or close,
to the HEisenberg point,
because you can slowly ramp up this parameter from 0 to 1/3.
AKLT noticed this Hamiltonian was similar to the Heisenberg model,
but you could prove it was gapped and has this interesting ground state.

So the spin-1 Heisenberg chain is in the same phase as this,
turning from 0 to 1/3,
and show the gap is still open,
so this isa mdoel to see whath e spin-1 Heisenberg chain looks like.

The key point about this tate,
is that you see youhaave these edge modes,
whic his a key non-trivial aspect of this tate as opposed to trivial.

\begin{question}
    How do you see there's nophase transition from 0 to 1/3.
\end{question}
I think numerically you can see it?
This is 1D,
so there are all sorts of analytically solvable tricks,
maybe yes,
but I don't know.
Therei s aphase transitoin,
but it's beyond 1/3.
I forgot exactly what.

They key point si that you hvae these dangling spin-1/2 things on the edge,
which are dge modes.
The global symmetry is SO(3),
but actaully the chain has more symmetries,
it is time-reverla invaraint,
tralsationally invaraitn,
reflection symmetry,
but we're interested in the OS(3) global symmetry.

The property of SO(3) is that it acts anomalously on the boundary.
Here, anomalous means microscopically each latice site is pin-1,
so it forms a linear repseration of SO(3),
but spin-1/2 is a projective representation of SO(3).
I will remind you mathematically what projective means.

Spin-$\frac{1}{2}$ is a projective representation of SO(3),
but microscopically we have spin-1 per site,
which is a linear representation of SO(3).

So let me tell you what projective representations are,
in case you don't know exactly what this means.
The short summary is taht if you do a combination of rotaitons in SO(3),
that for examle add up to dientity,
whereaase Spin-1/2 would give you a minus sign,
but linaer represetnations would not give you a minus sign.

Suppose you have a group $G$ with a finite-dimensional representation.
\begin{align}
    R: G \to GL(d; \mathbb{C})
\end{align}
What that means is that for eveyr gorup element,
you have the group of $d\times d$ complex matrix.

A \emph{faithful} or \emph{linear} representation has the property that when you
compose group elements,
youget the representation for the product.
\begin{align}
    R_g R_h = R_{gh}
\end{align}
However, this is not always the case,
because in general you could have
\begin{align}
    R_g R_h = w(g, h) R_{gh}
\end{align}
where $w(g,h)$ is a phase in U(1).
So in other words,
\begin{align}
    w \in C^2(G, U(1))
\end{align}
so $\omega$ is a function of two group elements and it gives you a U(1) phase.
If you have this rule,
you could consider associativity for 3 group elements,
which tells you you can first multiply them in any way.
\begin{align}
    R_g R_h R_k &=
    \omega(h, k) \omega(g, hk) R_{ghk}\\
    &= \omega(g, h)\omega(gh, k) R_{ghk}
\end{align}
And this takes us to an important equation.
It tells us the phases must satisfy an important equality
\begin{align}
    \omega(h, k) \omega(g, hk)
    =
    \omega(g, h) \omega(gh, k)
\end{align}
and this is called the \emph{2-cocycle condition}.

We can define this differential map $d$ that maps $n$-cochains to
$n+1$-cochains.
\begin{align}
    d: C^n(G, U(1)) \to C^{n+1}(G, U(1))
\end{align}
and we can define this differential as
\begin{align}
    d\omega_2 (g, h, k)
    =
    \frac{\omega(h, k) \omega(h, hk)}{\omega(g, h) \omega(gh, k)}
\end{align}
and so the 2-cocyle equation is really just the equation that
\begin{align}
    d\omega_2 = 1
\end{align}
and if it satisfies this eqaution,
we call it a 2-cocycle
so actually
$\omega_2 \in Z^2(G, U(1))$,
which is the space of 2-cocycles.
In other words,
\begin{align}
    Z^2(G, U(1)) &=
    \left\{
    x_2 \in C^2(G, U(1)) \,|\,
    dx_2 = 1
    \right\}
\end{align}
However,
there is some reduencay here,
because you could change any of these by a phase and potentially get rid of
these $\omega$s.
That is,
if $R_g R_h = \omega(g, h) R_{g,h}$,
you could just redefine
\begin{align}
    R_g \to R_g \epsilon(g) = R_g'
\end{align}
so then
\begin{align}
    R_g' R_h' &= \omega_2'(g, h) R_{gh}'\\
    R_g R_h &=
    \underbrace{\frac{\omega_2^1(g, h) \epsilon(gh)}{\epsilon(g)
    \epsilon(h)}}_{\omega_2(g, h)} R_{gh}
\end{align}
and thus
\begin{align}
    \omega_2'(g,h) &=
    \omega_2(g, h) \frac{\epsilon(g) \epsilon(h)}{\epsilon(gh)}
\end{align}
So you need to consider 2-cocycles modulo this equivalence relation,
which youccan think of as equilvanet proejctive rerepstionas.
This $\epsilon(g)\in C^1(G, U(1))$ is a 1-cochain associated iwth a single grup
elements,
with
\begin{align}
    d\epsilon &=
    \frac{\epsilon(g) \epsilon(h)}{\epsilon(gh)}
\end{align}
And now you can define the space of 2-coboundaries,
\begin{align}
    B^2(G, U(1)) := \left\{ 
    X_2 \in C^2(G, U(1)) \, \mid\,
    X_2 = d\epsilon_1,\,
    \epsilon_1 \in C^1(G, U(1))
    \right\}
\end{align}
that are differentials of 1-cochains
\begin{align}
    \omega_2 \sim \omega_2 \cdot d\epsilon_1
\end{align}

So projective rpesenations should be classified by 2-cocycles modulu by
2-coboundaries.

You have these rlrestnations,
you have hese extra phases,
andthat phase,
you might be agble to get rid of it by redefingin $R_g$,
but if you can't,
then you have a non-trivial projetive rpesetnation,
and two projective represtnatinos are equivalnet if you can just rephase these
$R_g$'s so distinct ones are lcassifed by this group.

Thus distinct projective representations are classified by
the second cohomology group
\begin{align}
    \frac{Z^2(G, U(1))}{B^2(G, U(1))} =:
    H^2(G, U(1))
\end{align}
And indeed,
you could also generalize this wigh
\begin{align}
    H^n (G, U(1)) &=
    \frac{Z^n(G, U(1))}{B^n(G, U(1))}\\
    Z^n(G, U(1)) &=
    \left\{ X_n \in C^n(G, U(1)) \mid dX_n = 1 \right\}\\
    B^n\left( G, U(1) \right)
    &=
    \left\{ 
    X_n \in C^n(G, U(1)) \mid
    X_n = d\epsilon_{n-1},
    \epsilon_{n-1} \in C^{n-1}(G, U(1))
    \right\}
\end{align}
The only thing I haven't defined is how the differential map is defined,
but we only need $n=2$ for now for 1D spin-1 chains.
We will come back to higher-dimensional SPTs later.

\begin{question}
    Is $H^2$ always a group because the coefficient is $U(1)$?
\end{question}
Well $Z^2$ is always a group.
You can actually replace $U(1)$ with a module over $G$ and still define all
these things.
For $U(1)$,
we don't really need all that,
we just need $U(1)$ coefficients.

Let's just stick to the math we need to do the physics.


\begin{question}
    The stateis symmetric under $G$ means linear represatatino?
    Is there any reason why linear represetnations are special in that way?
\end{question}
No, not ncesarily.
Linear represetnations are special for a wide vairteyt of reasons,
but you could defintiely hav astate symmetric under $G$ but forms a projetcive
rpesetations.
There are theorems which govern what ht low-energy behaviour has to be.
It's imporant to know forh tsystem,
but they are both equally valid as a system and ans as symmetry.


\begin{question}
    Is there a simple quantum example of a projective prestation?
\end{question}
Yes, spin-1/2.

\begin{question}
    We can either say it forms a linear represetnation of spin-1
    or a spin
\end{question}
Let me reprhase your qeustion.
A spin-1/2 chain can be thought of as a projective rpestenation of SO(3),
or it can be thought of a linear representation of SO(3).
It's also thel inear reprsetnation of the cover of some group.
Suppose I have a spni-1/2 chian,
should I think of the symmetry of the sytem being SU(2) or SO(3).
This is actually a pretty delicate quesiton.

It depends on how many spin-1/2s you have.
if you ahve an even numberof spin-1/2s, it should be SO(3),
but if you had an odd number of spin-1/2s, it should be SU(2) technically.
It depends on odd or even.

\begin{question}
    Regardless of even or odd,
    excitaitons are alwasy even spin,
    is there somtehting to interpet?
\end{question}
The local operators for a spin-1/2 are the Pauli matrices $\sigma^i$,
and they themselves form a spin-1 represetnation of SO(3).
The way that $\sigma^i$ transforms under an SO(3) represatatnion,
is the same as a spin-1.
So the Pauli operatosr transform as Spin-1 under SO(3).
Local operators always carry integer spin,
because these $\sigma^i$ transforma nd tranform as a spin-1 represetnation.

\begin{question}
    This is true even if we have a spin-2 chain?
\end{question}
If you had a spin-2 chain,
then the local operatosr would not e pauli matrices,
it would be whatever for spin-2.
Spin-2 is not the best examle.
Spin-3/2 is better,
where they are 4x4 matrices,
but under SO(4) rotations would transform as a local spin.
The formal statement is that clocal exctiations carry integer spin.

Let me say that
\begin{align}
    H^2(SO(3), U(1)) = \mathbb{Z}_2
\end{align}
So it's linear in the spin-1 represetantio nand porjective in the SU(2)
represtnation.

There's a relation between $H^2$ and $\pi^1$.
I think there may be smoe theorem for Lie groups,
$H^2$ is related to $\pi^1$.
$H^1$ is related to $\pi^1$.
$H^1$ is athe abelianization of $\pi^1$.
This is group cohomology.

OK, so going back to AKLT,
the edge modes form this projective representation and we can actually see
what's going on there.
In general for SPT states in 1-dimensions,
suppsoe you have a ground state and you apply your symmetry operation,
on the represetantion on the full-many-body state,
and because in the bulk it's symmetric and you hav a finite-correlation length,
ify ou apply $R_g$ on a ring,
youget back the original state,
\begin{align}
    R_g \ket{\psi} = \ket{\psi}
\end{align}
but on an open chain,
you get the ground satte back,
but up to some boundary opeartions near the left boundary $U_g^{(L)}$
and some operations acting on the right boundary,
up to corrections exponentially small in the isze of teh system.
\begin{align}
    R_g \ket{\psi} \approx
    U_g^{(1)} U_{g}^{(R)} \ket{\psi}
\end{align}
and so
\begin{align}
    R_g R_h \ket{\psi} &\approx
    U_g^{(L)} U_h^{(L)} 
    U_g^{(R)} U_h^{(R)} 
    \ket{\psi}\\
    R_{gh}\ket{\psi} &= U_{gh}^{(L)} U_{gh}^{(R)} \ket{\psi}
\end{align}
Now this is a genearl statemetn that doesn't even need ALKLT.
In 1D, you hvae as ymemtric gapped state,
and you apply your symemtry transfmatoin on an open chain,
then yo hvae these local operators acting on the boundary,
and you compre aply $R_{gh}$ and $R_g R_h$,
and the only way they can be eqaul is that the $U$s are equal up to a phase.
Because some are localized onthe left and some are localized on the right
boundary,
and the only wa this could possibly hold is that
\begin{align}
    U_g^{(L)} U_{h}^{(R)} \ket{\psi} &=
    \omega_2(g, h) U_{gh}^{(L)} \ket{\psi}\\
    U_g^{(R)} U_{h}^{(R)} \ket{\psi} &=
    \omega_2^{-1}(g, h) U_{gh}^{(R)} \ket{\psi}
\end{align}
In general,
you have one projective erepsertaniotn on the left,
and the inverse one on the right.
It's just he thing about SO(3) that you find the projective rpesetnation is its
own inverse and so you have spin-$\frac{1}{2}$ in each direciton.

And so this tells us the equivalence class
\begin{align}
    [\omega_2] \in H^2(G, U(1))
\end{align}
classifies the symmetry action on the edge.

Let's do another example.
AKLT was not the simplest thing.
Even thoguh it's fustration free,
it's nontrivial to prove the ground state is simple and gapped.

There's an even simpler example.
This is called the 1D cluster state model,
sometimes called the $ZXZ$ model,
or sometimes $XZX$ model.
Here, $X$ and $Z$ are the Pauli operators.
\begin{align}
    H &=
    -\sum_{j=2}^{N-1} Z_{j-1} X_j Z_{j+1}
\end{align}
Note that every term commutes with each other.
If I have $ZXZ$ on 3 sites,
and I have another $XZX$ on 3 sites shifted by 1,
they commute with one another.
So every term in the Hamiltnoian commutes and furthermore squares to the
identity,
so the gournd state is jsut the simultaneous $+1$ eigenstate of the $XZX$.

And furthermore,
the global symmetry we're intersted here is the
$G=\mathbb{Z}_2\times \mathbb{Z}_2$,
which are generated by
\begin{align}
    \bar{X}_1 &=
    \prod_{k\mathrm{ odd}} X_k\\
    \bar{X}_2 &=
    \prod_{k\mathrm{ even}} X_k\\
\end{align}


\begin{question}
    Is this encoding $k$ logical qubits into $n$ physical qubits?
    Is it something like stabilizer groups?
\end{question}
These are just symmetries,
I wouldn't call it logical oeprators.
These are jsut symmetry operators.
They square to the identity and they commute with the whole Hamiltonian.

On each boundary,
say on the left boundary,
we actually have a set of operators that commute with the Hamiltonian exactly.
Let's call them
\begin{align}
    \bar{X}_L &= X_1 Z_2\\
    \bar{Z}_L &= Z_1
\end{align}
and these commute with the Hamiltonian.
You see $Z_1$ commutes iwth teh Hamitna,
because the only operator that hits site 1 is the first ermm,
which has a $Z$ there, so that commutes.
You can also see that with the $X_1 Z_2$.

But these have a non-trivial algebra among themselves.
They anti-commute with one another.
\begin{align}
    \bar{X}_L \bar{Z}_L = - \bar{Z}_L \bar{X}_L
\end{align}
Because they comute with te Hamiltonai,
the ground state must frm a rperestaion of this algebra,
and because they anticommute,
there must be a two-fold degeneracy on each edge.
There's a simlar one on the right one as well.
This is not an accidental degenearcy.
No local perturbation can lift this degeneracy as long as the
$\mathbb{Z}_2\times \mathbb{Z}_2$ symmetry is preserved.
Oneway to lift hte dgenracy is to add these 2 tersm to the Hamiltonian,
but these don't commute with one another,
so you cannot add these both and preservet the symmetry.

You can directly see this behaviour exactly in this model.
Suppose that you want to apply the first symmetry operator to the ground state.
Let me consider a half-infinite chain,
so we only have the left boundary,
but then it goes off to inifitey on the other end so we don't have to worry
about it.
\begin{align}
    \bar{X}_1\ket{\psi_{gs}} &=
    \bar{X}_1 \prod_{k=2}^{\infty} Z_{2k - 2} X_{2k -  1} Z_{2k}
    \ket{\psi_{gs}}
\end{align}
and then I pull some things out,
and the nice thing about pulling them out is that if I get a lot of cancelation
\begin{align}
    \bar{X}_1 Z_2 X_3 Z_4 Z_4 X_5 Z_6\cdots &=
    (X_1 X_3 X_5\cdots) Z_2 (X_3 X_5 \cdots)\\
    &= X_1 Z_2
\end{align}
so actually you have
\begin{align}
    \bar{X}_1\ket{\psi_{gs}} &=
    \bar{X}_1 \prod_{k=2}^{\infty} Z_{2k - 2} X_{2k -  1} Z_{2k}
    \ket{\psi_{gs}}\\
    &=
    X_1 Z_2 \ket{\psi_{gs}}\\
    &=
    \bar{X}_L \ket{\psi_{gs}}
\end{align}

\begin{question}
    Something about $C^*$ algebra.
\end{question}
I'll leave it as an exercise.
If you really do a more honest calfcaultion end at $N$ rather than infinity,
you should get $\bar{X}_L$ and $\bar{X}_R$.
If you simillarly did 
$\bar{X}_2\ket{\psi_{gs}}$,
and instead of pulling the even rather than the odd one,
you find.
\begin{align}
    \bar{X}_2\ket{\psi_{gs}} &=
    Z_1 \ket{\psi_{gs}}
\end{align}
and this is really similar to how
$\bar{X}_1$ and $\bar{X}_2$ act as
$\bar{X}_L$ and $\bar{Z}_2$
on the left edge

And $\bar{X}_L$, $\bar{Z}_L$ form a projective represetnation of
the global symmetry
$\mathbb{Z}_2\times \mathbb{Z}_2$.

\begin{question}
    Is it clear this is a gapped state?
\end{question}
yes, every term commutes,
and each terms is +1 or -1,
and it costs an energy of 2 to raise it frmo the ground state to the next
excited state.

If you had a gapless situation on the ege,
you wouldn't have a sitaution where exactly the boundayr gets affected but the
bulk doesn't get affected at all.

You could have a sysetm fully symmetric on a chain,
but if you did have edge modes,
you owud have poewr law correaltions,
and if wasn't exaclty symmetric,
you wouldh ave some power law decay.
Also, if it wasn't gapped,
it wouldn't give an SPT phase which is by definition gapped.

This $R_g\ket{\psi} \approx U_g&L U_g^R \ket{\psi}$
really relies on the fact that it is gapped.
Are there any other questions?

\begin{question}
    How do you see there's no edge modes connected with other symmetries and
    translations?
\end{question}
In this particular model,
you can probably just count the number of degrees of freedom and see that we've
accounted for everything.

We have $Z_1 X_2 Z_3, \cdots Z_{N-2}X_{N-1}Z_N$.
The entire Hilbert space is accounted for already by counting.

\begin{question}
    Are we just partitioning the system?
\end{question}
They coexist.
Ify ou had a symmetry that is $G\times H$,
on each edge you would have a projective represenation of $G\times H$.
Spatial symmetries like translation and reflection are special.
here, the point is that even if we have a boundary,
the systeme iwth the boudnary stil rferelcets th symmetery.
Hwoever, wiht transltaiton symemtry,
the sysetme in the prescenece of the boudndary doesn't relaly resepct that,
so it desno't really makes sense to talk about hte action of tranlation symetron
when there is a boundary.

Often wiht spatial symmetries,
you couldh ave nootrivial SPTs associtated iwth te systemery,
but there may not be edge modes.
It's a more delicate task to figure out what ist he desintct property of the
SPTs given  it's hard to probe them becasue the boundary breaks the symmetry.

\section{Full classification of bosonic 1+1D topological phases}
What we saw so far is that we have this classfificaiton of how the symmetry acts
on the edge modes.
In fact,
this is the full classification of 1D SPTs.

The full classification of $(1+1)$-D topological phases is by
$H^2(G, U(1))$.
There are 2 pieces of evidence more easy to come by.

\begin{enumerate}
    \item We can use matrix product states.
    \item Try to develop a classification of TQFTs and see that deformation
        classes.
\end{enumerate}

\section{Matrix product states}
Suppose you have a chain with $N$ spins with periodic boundary conditions.
Then a \emph{matrix product} state is of the following form
\begin{align}
    \ket{\psi} &=
    \sum_{i_1,\ldots,i_N} 
    \Tr\left( 
    A_{i_1}^{[1]}
    A_{i_2}^{[2]}
    \cdots
    A_{i_N}^{[N]}
    \right)
    \ket{i_1,\ldots,i_N}
\end{align}
where $i_k=1,\ldots,d$ where $d$ si the local qudit dimension,
each $A_{i_k}^{[k]}$ are $D\times D$ matrices
where $D$ is the bond dimension.

The beauty of this state is that there are very few parameters.
There are only $dND^2$ parameters,
as opposed to the $d^N$ parameters for the most general $N$-qudit state.
This is an efficient parameterization of a very large class of systems.

The point of these matrix product states is that the entanglement entropy of a
matrix product state is upper-bounded by its bond dimension.
So if you had a chain,
let's say you have an open chain,
and you split it up into a left and a right piece,
then the entanglement entropy would satisfy
\begin{align}
    S = -\Tr \rho_L \ln \rho_L \le 2 \ln D
\end{align}
It is bounded by the bond dimension $D$.
In 1D,
any generic gapped ground state forms an area law.
What that means is that any gapped 1D ground state has ``area law''
entanglement,
which means that if I look at the entanglement entropy of some segment,
then $S_{1d}=\mathrm{constant}$,
meaning that it is independent of system size.

Actually, the more precise statement is that the entanglement entropy $S$ is of
order
\begin{align}
    S = O\left( 
    \frac{\log_3 d}{E_{gap}}
    \right)
\end{align}
This was originally proven by Hastings in 2007,
but then it was exponentially improved 6 years later by
Arod, Kitaev, Landau, Vazirani in 2013,
and the statement above is from the latter paper.
This latter papaer also improved another result of Hastings,
which is that you can approximate any gapped ground state to an accuracy that is
$1/poly(N)$ where $N$ is system size with an MPS of sublinear bond dimension,
sublinear means the bond dimension grows slower than the size of the system.
We don't care what the polynomial is.
The more speicfic thing they proved is that
\begin{align}
    D &=
    O\left( 
    e^{\log_{3/4} N / E_{gap}^{1/4}}
    \right)
\end{align}
The main ponit is taht gapped ground states have area law entalmenet in 1D,
and MPS is a way of parameterizing states iwth area law entanglmeent.
So if you have a gapped ground state,
you can have a fairly good approximation with MPS.

You need fewer than $dND^2$ parameters to very accurately approximate the ground
state.

\begin{question}
    is the converse true?
    Any state that satisfies area law entanglement is gapped?
\end{question}
I don't have a counter example.
Every system I know has an entropy that grows logarithmically at worst.
The converse of these statements usually have some sick counterexample.
For example,
finitely correlated Hamiltonians don't imply gapped.
I don't recall hearing any for this one though.

What we can actually do is use matrix product states to prove the
classification.
In particular, there is no topological order for 1D bosonic states.
