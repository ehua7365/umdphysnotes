\section{Response Theory for (3+1)D Topological Insulator}
The response theory has action
\begin{align}
    S[A,g] &=
    \theta\cdot \frac{1}{32\pi^2}
    \int_{M^4} F_{\mu\nu}F_{\lambda\sigma} \epsilon^{\mu\nu\lambda\sigma}
    - \frac{1}{48}
    \int \frac{1}{\left( 2\pi \right)^2} \Tr R\wedge R
\end{align}
where $\theta=\pi$ for non-trivial and $\theta=0$ for trivial.
Recall that
\begin{align}
    \int F_{\mu\nu}F_{\lambda\sigma} \epsilon^{\mu\nu\lambda\sigma}
    =
    \vec{E}\cdot\vec{B}
\end{align}

The physical consequences are TME:
\begin{enumerate}
    \item Domain wall in $\theta$, $\frac{1}{2}$ integer Hall conductance at
        surface.
    \item Witten effect
        \begin{align}
            Q_e &= \frac{1}{2}Q_m
        \end{align}
    \item Magnetic field induces charge polarization
\end{enumerate}
the current is
\begin{align}
    j^{\mu} &= \frac{\delta S}{\delta A^\mu}
    = \frac{1}{2\pi} \epsilon^{\mu\nu\lambda\sigma}
    \partial_\nu \theta \partial_\lambda A_\sigma
\end{align}
Let $\theta$ be uniform and time-dependent
\begin{align}
    j^i &=
    -\frac{1}{2\pi}
    \epsilon^{tijk} \partial_t \theta \partial_j A_k\\
    &= -\frac{1}{2\pi}\partial_t \theta\cdot B^i
\end{align}
so then the current is
\begin{align}
    \vec{j} &=
    \frac{\partial\vec{p}}{\partial t}\\
    &=
    -\frac{1}{2\pi} \partial_t \theta \vec{B}
\end{align}
and so the magnetic polarizability is
\begin{align}
    \vec{P} &=
    -\frac{\theta}{2\pi}\vec{B} + \mathrm{const}
\end{align}
For a topological insulator $\theta=\pi$.
It's a bulk polarization but poarlizaiton only has an effect on the surface.
When I change $\theta$ from 0 to $3\pi$ for example,
my quantum hall effeect on the surface is $\left(1 + \frac{1}{2}\right)e^2/h$.

In 1D I can always add an integer charge on the surface.
If I put $2\pi$ here,
the efect is tackign on a quantum hall effect on the surace.
The bulk didn't play a special role,
it's just something intrinsicaly 2D you can do.

\begin{question}
    Why is QHE layer the same as surface charge?
\end{question}
The magnetic field induces the surface carge.
Charge density is attached to magnetic field in the integer quantum hall effect.

\begin{question}
    How do you physically increase $\theta$?
    Whatif you wanted to?
\end{question}
This was just a trick to get this eqution.
Every insulator in 3D has a $\theat$.
Could be anything.
Buti f it's time-reversal invariatn,
it's pinned to $0$ or $\pi$.
What $\theta$ is for a specific material,
its' a aproperty of hte matieral.

\begin{question}
    Insead of magnetic field, the charge goes ot the edge of the system?
\end{question}
If I insert a magnetic field localy,
I locally get a charge.
If I insert a uniform magnetic field then I get a uniform charge.

\begin{question}
    So the divergence of $\vec{B}$ is zero,
    does that mean there's no net charge within anywhere in the system?
\end{question}
From the Witten effect,
charge is attached to the monopole.
If you have no monopoles,
then you won't have any.

I told you all thsi stuff,
just asserted the response theory of the TI  has a $\theta$ term,
but I didn't derive it from any mdoel.
I want to give you one type of model.
This goes back to the dimensional reduction from (4+1)D.

To describe (3+1)D TI,
it's useful to start with a (4+1)D analogue of the Chern insulator.

What is a (4+1)D Chern insualtor?
In 2+1D,
we wrote the berry connection.
But now you have to define a non-abelian Berry connection.
\begin{align}
    a_i^{\alpha\beta} &=
    -i \bra{\alpha,k} \frac{\partial}{\partial k_i} \ket{\beta,k}\\
    f_{ij}^{\alpha\beta} &=
    \partial_i a_j^{\alpha\beta}
    - \partial_j a_i^{\alpha\beta}
    + i\left[ a_i, a_j \right]^{\alpha\beta}
\end{align}
where $i,j$ are spatial indices and $\alpha\beta$ are band indices.
The second Chern number is
\begin{align}
    C_2 &=
    \frac{1}{32\pi^2} \int d^4k
    \epsilon^{ijkl} \Tr f_{ij}f_{kl}
\end{align}
Here we get a (4+1)D Chern-Simons term
\begin{align}
    S_{CS,4+1} &=
    \int
    \frac{C_2}{24\pi^2} A_{\mu}\partial_\nu A_\lambda \partial_\rho A_\sigma
    \epsilon^{\mu\nu\lambda\sigma}
\end{align}
There isa Maxwell term with two factors of $A$,
but here we have3 factors of $A$,
so this is subleading,
but this term is topological,
wherease Maxwell si not toolgocial.

In (2+1)D we had a model for the Chern Insulator,
where it's basically a massive Dirac fermion you put on a lattice.

Here we write a $(4+1)D$ massive Dirac fermion instead.
So the mdoel for the $(4+1)D$ Chern insulator is a massive Dirac fermion,
which in momentum space you would write
\begin{align}
    H &= \sum_k \psi_k^\dagger d_a(k)\Gamma^a \psi_k
\end{align}
where the $\psi_k$ are 4-component fermions,
and these $\Gamma$ are $(4+1)D$ Dirac matrices which satisfy the Clifford
algebra
\begin{align}
    \left\{ \Gamma^\mu, \Gamma^\nu \right\} = 2\delta_{\mu\nu}
\end{align}
where $\mu,\nu=0,\ldots,4$.
And our lattice vectors are
\begin{align}
    d(k) &=
    \left( 
    m + c\sum_{i}\cos k_k,
    \sin k_x,
    \sin k_y,
    \sin k_z,
    \sin k_w,
    \right)
\end{align}
and to be concrete,
we can write explicityly the Gamma matrices
\begin{align}
    \Gamma^a &=
    \sigma^a \otimes \tau^2
\end{align}
for $a=1,2,3$
and
\begin{align}
    \Gamma^4 = I\otimes \tau^1\\
    \gamma^0 &= I\otimes \tau^3
\end{align}
and the lattice vectors
\begin{align}
    \hat{d}(k) &=
    \frac{\vec{d}(k)}{|d(k)|}
\end{align}
The second Chern number is
\begin{align}
    C_2 &=
    \frac{3}{8\pi^2}
    \int d^4k
    \epsilon^{abcde}
    \hat{d}_a
    \partial_x \hat{d}_b
    \partial_y \hat{d}_c
    \partial_z \hat{d}_d
    \partial_w \hat{d}_2\\
    &=
    \begin{cases}
        0 & m < -4c, m >4c\\
        1 & -4c < m < -2c
    \end{cases}
\end{align}
and just like before,
there are $(3+1)D$ gapless chiral fermions on the surface with
$|C_2|$ flavours.

The surface Hamiltonian will be
\begin{align}
    H_{\mathrm{surface}} &=
    \sgn(C_2)
    \int \frac{d^3k}{(2\pi)^3}
    \sum_{i=1}^{|C_2|}
    V_i \psi_i^\dagger \vec{\sigma}\cdot\vec{k} \psi_i
\end{align}
where these are 2-component fermions.

There is a chiral anomaly
\begin{align}
    \partial_\mu j^\mu \propto FF
\end{align}

Now that we have this $(4+1)D$ cherin insulator,
we can dimension-reduce and view one of these $A$ componetns as a parameter,
and we effectively get an $F\wedge F$ term in the dmiensinoally reduce dmouel

These chircal fermions onthe surface descend into Dirac cones on the surface of
the $(3+1)D$ topological insulator.

\begin{question}
    Why are there $C_2$ flavours?
\end{question}
It is the analogue of $C_1$ flarous in $(2+1)D$.

It's always a 2-component fermion,
buti t's a question of how many you have.
The Chern numbero f bulk is just like the $(2+1)D$ case.

\begin{question}
    Why 2 component fermions?
\end{question}
I don't have a quick answer.

The point is you decompose your fermions
with a chirality operator,
so you split the 4 componet into 2, 2 components,
and because it's chiral,
you only have 2.
They are $C_2$ flarous of the 2-compoent fermion.

\begin{question}
    How do you visualize a $(2+1)D$ chiral fermion.
\end{question}
I don't have a godo visualization.
There's chirality operator,
and you project them onto two eigenvectors.

If you have massless particles moving in a direciton,
you can attribute a handedness of the particles,
which you cannot Loretnz boost out of.

Yeah, it's basically in this language,
$1\pm \Gamma^0$,
that's how you define the Chirality operator.
Projection onto some particular chirality.

\begin{question}
In 2 spatial dimensions,
you have only clockwise and anti-clockwise.
What's the analog of this?
\end{question}
I don't usually think of chiral fermions in 3 dimensions.

\begin{question}
    In the 1D case,
    you could think of the Chern number as the wrapping of the sphere?
\end{question}
Oh yes.
The $\hat{d}$ vector is a map from the Brilluoin zone $\hat{d}:T^4\to S^4$.
This is computing the winding numbero f the map from $S^4\to S^4$.

If you take $k\to\infty$, it is the case that it all points in the same
direction,
I don't know if there's an easy way of saying it.

His point is if I take $k_x$ to $\infty$ but not the other ones,

I was plannig on presenting the dimensinoal reduction from $(4+1)D$
to $(3+1)D$.

The point is the $(4+1)D$ model can be thought of as
decoupled $(3+1)D$ models parameterized by $k_w$.
We did a version of this in $(2+1)D$,
where we thoguht of it as a $(1+1)D$ Hamiltonian parametrized by $k_y$.

Let's write the Hamiltnoian first in real space on a lattice.
On real space on a lattice,
that Hamiltnoian will look like
\begin{align}
    H_{(4+1)D} &=
    \sum_{\sigma,a=1,\ldots,4}\left[ 
    \psi_{\vec{r}}^\dagger
    \left(
    \frac{c\Gamma^0 - i\Gamma^i}{2}
    \right)
    \psi_{\vec{r} + \hat{a}}
    e^{i A_{\vec{r},\vec{r} + \hat{a}}}
    + \mathrm{h.c.}
    + m\psi_{\vec{r}}^\dagger \Gamma^0 \psi_r
    \right]
\end{align}
and then I can write it in the ``Landau gauge''
translationlaly invariatn in $w$ direction
\begin{align}
    H_{(2+1)D}[A] &=
    \sum_{k_w} H_{(3+1)D}[k_w, A]
\end{align}
and then
\begin{align}
    H_{(3+1)D} [k_w, A] &=
    \sum_{\vec{r}, a=1,\ldots,3}\left[ 
    \psi_{\vec{r}, k_w}^\dagger
    \left( 
    \frac{c\Gamma^0 - i\Gamma^a}{2}
    \right)
    \psi_{\vec{r} + \hat{a}, k_w}
    e^{i A_{\vec{r},\vec{r} + \hat{a}}}
    + 
    \mathrm{h.c.}
    \right]\\\nonumber
    &\qquad+
    \sum_{\vec{r}}
    \psi_{\vec{r}, k_w}^\dagger\left[ 
    \sin\left( k_w + A_{\vec{r}, w} \right)\Gamma^4
    +
    \left( m + c\cos\left( k_w + A_{\vec{r},w} \right)\right)\Gamma^0 
    \right]
\end{align}
So if I set $\theta=k_w + A_{\vec{r},w}$,
I just replace $k_w + A_{\vec{r},w} = \theta_{\vec{r}}$
and this gives us
$H_{(3+1)D}[A,\theta]$.

At the level of the effective action,
we can figure out the efective response theory
by doing dimensinoal reduction in the response theory.

And so
\begin{align}
    S_{(4+1)D} &=
    \frac{C_2}{24\pi^2} \int d^4x \, dt\,
    \epsilon^{\mu\nu\rho\sigma\tau}
    A_{\mu} \partial_\nu A_\rho \partial_\sigma A_\tau
\end{align}
and under dimensional reduction $C_2=$,
and I can just rewrie this action as
\begin{align}
    S &=
    \frac{1}{24\pi^2} 3
    \int d^3x \, dt\,
    \epsilon^{w \mu\nu\rho\sigma}
    A_w \partial_\mu A_\nu \partial_\rho A_\sigma\\
    &=
    \frac{\theta}{32\pi^2}
    \int \epsilon^{\mu\nu\rho\sigma} F_{\mu\nu}F_{\rho\sigma}
\end{align}

Why did we set $C_2=1$?
Because I want to describe the non-trivial topological insulator.
If I set $C_2=2$, then I would just get two copies of the TI.

You can just do perturbation theory to see what the response theory looks like.

\begin{question}
    Is $\theta$ here defined on $2\pI$ and does it save the time-reversal
    symmetry?
\end{question}
Here it does, just follows from the same argumetn I made before.

This $k_w + A_{r,w}=\theta_r$,
which you can see is periodic in $2\pi$.

If you write the time-reversal operator,
this term does not commute with the time-reversal operator,
so $\theta$ has to be $0$ or $\pI$.

\begin{question}
    When is dimensional reduction not possible?
    This is quite general argumetn.
\end{question}
You can always do dimensional reduction,
but it's not necesarily the case that you start with some theory,
and you do dimesnional reductino and yo uget an interseting and nontrivial
theory.

For example, if I set $C_2=2$,
I get two copies and I get the trivial insulator.

In $(2+1)D$ where we did dimesniaonl reductnio,
the $(1+1)D$ theory was not trivial because it realised a topological pump.

We start with a topological pahse characterised by integer $C_2$,
and then $\mathbb{Z}_2$ invariants.

You could dmiensional reduce one more time,
and yo uget the $(2+1)D$ topological insulator,
which is time-reversal invariant.

\begin{question}
    Do we need the model gapped?
\end{question}
If you start witha gapped model,
you usually wind up with a gapped model.
But if you started with a gapless mdoel,
you might not end up with a gapless model.


Now you can also do the $(2+1)D$ surface.
If we take $C_2=1$,
we would only have one flavour,
and if I dmeinosaly reduce,
and set $k_w=0$,
then I would just get a single Dirac cone.
This ia nice way to se the (2+1)D surface
would jsut beceom a (2+1)D Dirac cone.
I won't write it out,
because it's obvious from this.


Let me write out one more thing to be a little more clear.
Wen you do dimensinoal reduction,
you can think of putting a system on a cylinder,
in which case the azimuthal direction is $w$.
Then your momentum would be quantized
\begin{align}
    k_w &= \frac{2\pi}{L-w}n
\end{align}
and the low-energy mode is the one where $n=0$.
If you start with Chiral fermions and do dimesnioanl reductio
you have one massless fermion, the zero-ommentum mode
that descends to the massless dirac cone,
but you would have higher dmieinosal massive Dirac fermions.


Intersting commetn:
If you start with the $(2+1)D$ surface Dirac cone,
I mentinoed before this theory has a Perry anomaly,
hwich is a mix of U(1) and TR symmetry,
but you see the theory has problems,
you can decude the fact that the only way of preserving the symmetyra nd makig
making the theory well-define is to inroduce a bulk (3+1) respone theory.

So start with a (2+1)D surface theory with a Dirac cone,
then try to make the path integral well-defined with
$T$, $U(1)$.
Then educe a $(3+1)D$ bulk with
\begin{align}
    S_{eff} &=
    \int F\wedge F + R\wedge R
\end{align}
There's a paper by Witten in 2015,
which is nice because he uses the APS index therem by Atiyah.
I would go thorugh it,
but it would be an entire lecture.

\begin{question}
    Does it go up to higher dimensions?
\end{question}
In Witen's paper,
he only does $(3+1)D$ topological theories,
but the index theorem hsould work.

The thing is,
this parity anomaly you don't have in every dimension,
you have it in every 8 dimensisno.
There's a periodicity of 8.
Everything we did for free fermions has periodicity 8.
Deep,
but we don't care beyond 3 dimensions.

\begin{question}
    Are zero modes of (3+1)D a subset of zero modes in (4+1)D
\end{question}
I don't know, it's not obvious.

\begin{question}
    Experimental progress in realising these?
\end{question}
There are multiple generations of people who predicted various matirerals.
The first material is BismuthAntimony,
but since then many materials have been porpsoed.
There's no problem gettinga band structure exhitibing this topolgoical
insulator,
but the problem is getting the chemical potential in the gap.
You can't tune the chemical potential like in 2D,
you need to use doping,
but it's hard to get hese materials into actual insualtors.
Early on,
it was difficult getting the chmiceal potential in the right place.
But in the band structre,
you could confirm things like an odd numberof Dirac conesno the surface.
I don't know if people actually sucessfully made an insulating matieral that was
topogoloical and gets the chemcial potential in the right palce.

Was anyone following?

Now I want to switch gears a little bit.
Let's summarize what we did.

We talked about a $(1+1)D$ Majorana system with
$G_f = \mathbb{Z}_2^f$,
and there is a $\mathbb{Z}_2$ invariant.

Then we considered $(2+1)D$ systems with $G_f=\mathbb{Z}_2^f$.
There is a $\mathbb{Z}$ invariatn which si the chiral central charge,
which is realised by the $p+ip$ superconductor.
$G_f = U(1)^f$And there is a $\mathbb{Z}$ invariant Chern insulator.
And we studied 
$G_f = U(1)^f \rtimes \mathbb{Z}_4^{t,f}/\mathbb{Z}_2$
which results in a $\mathbb{Z}_2$ invariatn in class AII.

Then in $(3+1)D$,
we geta $\mathbb{Z}_2$ invariant to class AII.

All the above are examples of invertible topological phases.
They all have free fermion realisations.
These are all examples,
once we look at what states are at the free fermion level,
all states were stable to added interaciton.
They all had phsycal propblems.
the $\mathbb{Z}$ invairnat is jus the QHE conducatnce.
And we also solved for hte phsycial properties of these invariats.
We didn't formally argue it,
but it is plausible because iti s a physical quantitiy you could measure its
properties.

There are other tings tha could hapepen.
thera resiutaltios in free fermions you get $\mathbb{Z}$ invariatns,
but when you add interactinos,
it drops to $\mathbb{Z}_8$.

And then you could have invertible models with no fre fermion ralisations.
And then you have non-invertible models.

\section{Bosonic in (1+1)D}
Theseare inrinsically strongly interacting,
because if not they will condense into superfuliud state.
I necessarily need a strongly interacting system.
So all the tools we had for free fermion models become useless.
In $(1+1)D$ for bosons,
bosons can only realize a certina subclass of invertible phases.

There are some aspects.
\begin{enumerate}
    \item There's no topological order in 1D.
        This means if you forget about any symmetries of your model,
        any gapped bosonic state in 1D can be adiabatically conencted to trivila
        state.
        That's difernt to the fermionic case,
        where if you ignore all pases,
        it's still non-trivial bcause thereare stil MZMs.
        For bosons,
        if you ignore symmetries,
        everything can be adiabatically connected to the trivial phase.
        So no topological order in 1D.
    \item If you add symmetry,
        then we couldh ave a class of topological phases called SPT phases.
        SPT stands for symmetry protected topological.
        These are a subset of invertible phases,
        which has the property that if you break the symmetry,
        it can be adiabatically connected to a trivail state,
        but if you have the symmetyr,
        it cannot be in a way that preserves the symmetry.
        These are classified by a quantitiy called the group cohomology
        $H^2(G, U(1))$.
        The second cohomology group with group elements of $G$ wiht coefficients
        in $U(1)$.
\end{enumerate}
Rigorous proof will require mathematical frameowrks not in this class.
We can establish this using matrix product states.

Ther eare no invertible phaess that are not SPTs.
There are no chiral phases.
You wouldn't be able to define chiral,
the boundary is zero dmiensional,
so there are no chiral moedes.
$(2+1)D$ is special,
even bosons have an nivertible phase that is not an SPT.

Let me start off by giving an example of an interesting 1D state.
The most important example is the AKLT state.

These guys considered a spin-1 chain,
with $G=SO(3)$ symmetry.

The Hamiltonian includes a Heisenberg term,
but in addition,
there's an extra term that is the square of he Heisenberg term
and a constant offset
\begin{align}
    H_{AKLT} &=
    \frac{1}{2} \sum_{i} \left( 
    \vec{S}_i\cdot \vec{S}_{i+1}
    + \frac{1}{3} \left( \vec{S}_i\cdot\vec{S}_{i+1} \right)^2
    + \frac{2}{3}
    \right)\\
    &= \sum_i P_{i,i+1}^{(2)}
\end{align}
where $S_i^{\alpha}$ is a $3\times 3$ matrix
and $P_{i,i+1}^{(2)}$ is a projector onto spin-2 subspace of neighbouring spin
1s.

This is simply the explicit form of this projector.
For spin $1/2$ particles,
there is no square term.
Once we have this sum of projectors,
G
the ground state is the state that is annihilated by all projects.

I have a spin chain,
and I am summing over every site on the spin chain.
The first term would be $P_{1,2}$,
where I project 1 and 2 onto spin 2, etc.

Is there a non-trivial state that si annihilated by all $P$s.
You might be worried thesubspace annihlated by this is zero-dimensiona,
but it's one-dimensional on a periodic chain.

The one way to construct the ground state is a cool trick.
Think of every spin 1 as coming from 2 spin 1/2 systems,
but project onto the spin 1 subspace.

This is site $i$ for example,
would have two spin $\frac{1}{2}$ systems,
which I project onto spin-1 subspace.
You'll notice that in order for these two sites $i$ and $i=1$,
the only way this projector is going to give something non-trival
is if every pair of spin-1/2 fuses into a state of spin-1.

If any two spin-$\frac{1}{2}$ fuse to spin 0,
that subspace is going to be annihilated by this projector
$P^{(2)}$.

Imagine if you had
\begin{align}
    \frac{1}{2}\otimes \frac{1}{2}\otimes
    \frac{1}{2}\otimes \frac{1}{2}
    =
    (0 \oplus 1) \oplus (0 + 1)
\end{align}
I can consider where these spin-1/2s from neighbouring parts fouse across.o

consider a state
where each spin-1 you break down into spin-1/2s,
and then project them onto spin1.
Andthen you draw lines connecting adjacent spin-1/2s in different proejctors.

If you put this on a zero chain,
you won't hvae ege modé.o

The 0-energy ground state í the unique gapped grouned sate in AKLT.

This is a ``frustration free'' state.
There will be dangling spin-1/2 degrees of freedon.
On an open chin,

Just likewe found we have anomalies for the boundaries,
this eges system is ano.

Spin-1 degrese offreedom form faithful linear optaions of SO(3) symmetry,
wherease spin-$\frac{1}{2}$ is only a projection.

There will be dangling spin-$\frac{1}{2}$ degrees of freedom.
Edge degeneracy forms a projective representation of $SO(3)$.
Wahereas,

There's anomalies associated iwth global symemtries,
and then there are graviational anomalies,
and they are more severe.
The edges system is 

Some boudnary theories might be in conflict iwththe.
