\section{Anomalies}
Last time we mentioned the types of TQFTs.
There's one more TQFT that you might encounter.
There's something called an extended TQFT.\@
When I define TQFT,
I assign a path integral to a $(d+1)$-dimensional manifold
and we have a path integral
$Z\left( M^{d+1} \right)\in \mathbb{C}$.

For a closed $d$-manifold $V(\Sigma^d)$.

For a closed $(d-1)$-manifold,
we can associate a $1$-category.
This is called once-extended.
The objects get more and more abstract as you go down.

And then for $(d - 2)$-manifold,
you can attach a 2-category.
For a $(d-3)$-manifold,
you attach a 3-category.

I'm not going to define these.
And then for a 0-manifold,
you get a $d$-category.

Useful to note that if you have a fully extended TQFT,
all the way down to $d$-category,
this is going to come up for us,
we won't necessarily use it,
but we will encounter them often.
Fully extended TQFTs are situations where you can write down exactly solvable
Hamiltonians for your topological phases as sums of commuting projectors.
And then the path integral can be written as an exact combinatorial state sum.

We'll encounter examples eventually.
To define the path integral,
triangular the manifold
define simplexes 1 2 3,
assign labellings,
and them sum over all labellings.
That's an exact combinatorial space.
They're useful to work with and have a lot to say about them.
Physicists don't have a good understanding of what these higher structures mean
physically,
so it's an interesting question.


\section{Invertible vs non-invertible TQFT}
The next distinction is between invertible and non-invertible TQFTs.

\begin{definition}
    A TQFT is \emph{invertible} if the magnitude of the path integral has unit
    magnitude
    \begin{align}
        |Z(M^{d+1})| = 1
    \end{align}
    on any closed manifold.
\end{definition}
In particular, if you look at $Z(\Sigma^d\times S^1) = 1
= |\dim V(\Sigma^d)|$.
This is interpreted as the dimension of the Hilbert space on this space
Physically,
this means that there is always a unique state on any closed $\Sigma^d$.
A topological state of matter described by an invertible TQFT always has a
unique ground state on any closed manifold.
Not always the case,
depending on the topology of the space,
and those will be described by non-invertible TQFTs.

The reason it's called invertible,
is that if the path integral is a phase,
there's always a well-defined inverse of the phase,
which is the complex-conjugate of the phase.

Let's say if you have a TQFT,
then the ``inverse'' TQFT has a path integral like
\begin{align}
    Z_{T^{-1}}\left( M^{d+1} \right)
    =
    Z_T^*\left( M^{d+1} \right)
\end{align}
Invertible TQFTs come up a lot.
They describe integer quantum hall states.
Topological insulators and superconductors.
All $(1+1)$-dimensional topological phases of matter are describable by
invertible TQFTs.
Any TQFT that is not invertible is a \emph{non-invertible} TQFT.\@

Actually, one more thing before I get to that.
Invertible TQFTs form an Abelian group.
Suppose that I have the path integral for two theories $T_1$ and $T_2$,
and I want the path integral for $T_1\times T_2$,
then I just multiply their path integrals.
\begin{align}
    Z_{T_1T_2}\left( M^{d+1} \right)
    =
    Z_{T_1}\left( M^{d+1} \right)
    Z_{T_2}\left( M^{d+1} \right)
\end{align}

\begin{question}
    Does the partition function define the TQFT uniquely?
\end{question}
Well $V(\Sigma^d)$ is always going to be a one-dimensional vector space.
Even if it has some boundaries,
it's just going to be a number,
a multiple of that state,
so just an amplitude.
We can extend that definition to the full TQFT.
I'm specifically talking about the invertible ones.

Even in $M^{d+1}$ is open,
the state on the boundary is one-dimensional,
I can just take the inverse theory to give me the complex conjugate.

If a theory is not invertible,
it is a non-invertible TQFT,
and it will have a non-trivial degeneracy on different manifolds.
One thing that is true,
but I'm not sure if there is a proof,
I doubt there's a counterexample,
but I think that this always goes the other way.
There's also an arrow going the other way.
If the path integral magnitude is 1,
then every path integral on the manifold is 1.
I've seen the proof for once-extended,
but I'm not sure if it's general.

So a non-invertible TQFT should have
$Z|\left( \Sigma^d\times S^1 \right)\ne 1|$
for some $\Sigma^d$.
Using ``should'' loosely here.

Let me give you a quick example of an invertible TQFT.\@
This is actually going to be an interesting example,
because it's going to show you a TQFT that is non-trivial,
but it's going to describe a trivial phase of matter.
Imagine we take our path integral on a two-manifold
and have some $U(1)$ gauge field $A$ with field strength $F$.
Then
\begin{align}
    Z\left( M^d, A \right)
    =
    e^{-\theta \int_{M^2} F}
    e^{i\phi \chi\left( M^d \right)}
\end{align}
where
\begin{align}
    \chi\left( M^d \right) = \frac{1}{2\pi}\int_{M^2}R
\end{align}
is the Euler characteristic.

Phases of matter are 1 to 1 correspondence with deformation classes of TQFTs.
In the absence of pinned phases of $\phi$,
this is an example of an invertible TQFT.\@

The first most important question studying a new phase of matter is to ask if
it's describable by an invertible TQFT or a non-invertible TQFT.\@

\section{'t Hooft anomaly}
One big concept about TQFTs I want to mention are called
\emph{'t Hooft anomaly}.
This is something that applies a general QFT,
not just TQFT.
We can consider a QFT which has a symmetry group $G$.
This is the global symmetry group of the QFT.
This may be completely well-defined in the absence of any background gauge field
associated with $G$i.
Then you try to turn on a background gauge field and see something is wrong
with the theory.
Consider your path integral $Z(M^{d+1})$ and turn on $A$,
and you find that the path integral is not gauge-invariant.
\begin{align}
    Z\left( M^{d+1}, A + d\lambda \right)
    \ne
    Z\left( M^{d+1}, A \right)
\end{align}
when $A\ne 0$.
It's not the end of the world.
You might be able to redefine the path integral and make it gauge invariant.
You may be able to define a new gauge-invariant path integral
\begin{align}
    \tilde{Z}\left( M^{d+1}, A \right)
    = Z\left( M^{d+1}, A \right)
    e^{i\int_{M^{d+1}} f(A)}
\end{align}
where the $e^{i\int_{M^{d+1}} f(A)}$
is called a local counter-term that cures the gauge invariance.

\begin{question}
    You mean the action of $G$?
\end{question}
You look at all the correlation functions and you're applying $G$ to all the
operators.
The ground state of the QFT is invariant under $G$.

\begin{question}
    When you couple the background gauge field,
    should we be doing minimal coupling?
\end{question}
It depends how you define this quantity.
Suppose you have some action that is a single Dirac fermion with a mass.
You integrate out the mass
and find a Chern-Simons term with some action.
You wouldn't have written this term down anyway if you were experineced.
But you could actually just add another local functional to this,
and now the coefficients add up to 1 and they are gauge invairant.
If you're not following this discussion it's fine.
I haven't introduced Chern-Simons theory yet.
But you're right.
Usually, you would have defined your path integral in the first place.

\begin{question}
    Is it an anomaly in global symmetry?
\end{question}
Yes. This is the only anomaly worth having a name.
Gauge anomaly means you have a 't Hooft anomaly you try to gauge.
You may not always be able to add counter terms.
I haven't defined 't Hooft anomaly yet.

My point is that a theory with a 't Hooft anomaly is a real theory.
A theory with a gauge anomaly is a theory that doesn't make sense in the first
place.

There's something called a framing anomaly.
There's gravitational anomalies.
Anomaly is an overloaded term in physics,
used everywhere for different things.

There's no such thing as a gauge theory in QM.
It's a classical redundancy in your description.
A theory can have a global symmetry that is anomalies.
That is a 't Hooft anomaly.

You write down a Lagrangian with classical gauge symmetry.
You quantize it and find the symmetry does not hold when quantized,
and that's an anomaly.
There's a whole story.

There could be situations where you can't cancel this gauge apology by adding
counter terms.
You can cancel it not with a local counter term,
but thinking your theory as the surface of a higher dimensional invertible
TQFT.

So,
we can have a situation where the gauge non-invariance is cancelled by a
$(d+2)$-dimensional invertible TQFT.
An invertible TQFT that exists in a 1-higher dimension.
You can think of your system as living on the $(d+1)$-dimension surface of a
$(d+2)$-dimensional bulk and this bulk is an invertible TQFT.
And then you can have some bulk theory
$Z_{\text{bulk}}\left( W^{d+2}, A \right)|_{\text{boundary cond.}}
= Z\left( \partial W^{d+2}, A|_{\partial W} \right)$
perfectly well-defined,
but if you evaluate with some boundary condition.
And the gauge transformation would be governed by the higher-dimensional theory.
So
\begin{align}
    Z\left( M^{d+1}, A + d\lambda \right)
    Z\left( M^{d+1}, A \right)
    = Z_{\text{bulk}}\left( M^{d+1}\times I, \tilde{A} \right)
\end{align}
So the amplitude of going from $A$ to $A + d\lambda$,
there's an extra phase that comes from the invertible TQFT in the bulk,
which cancels out the gauge non-invariance here,
but cancelling in a way that cannot be done by just a local counterterm.

So then 't Hooft anomalies can be classified by TQFTs.

\begin{question}
    Why only just one higher dimension?
\end{question}
If you have to go to 2 higher dimensions,
you can always compactify one of te dimensions
and then you're back to $+1$ extra dimensions.

Let me just explicitly write it out

A QFT has a 't Hooft anomaly if
\begin{enumerate}
    \item It has a global symmetry $G$
    \item The path integral $Z$ is not gauge invariant under gauge
        transformaiton $A\to A + d\lambda$
    \item Gauge non-invariance cannot be cancelled by a local counter-term,
        but it can be cancelled by a $(d+1)$-dimensional TQFT.
\end{enumerate}
In topologcial pahses of matter,
some are described at low energy witha 't Hooft anolmaly
and what that tells us is that it cannot be realised ni $(d+1)$ spacetime
dimensions,
but has to be the surface of a higher dimensional theory.
In condensed matter,
a theory witha 't Hooft anomaly can only be raelised microscopically
and have the symmetry act on-site,
it arises as the $(d+1)$-dimensional surface of a $(d+2)$-dimensional
topological phase.

On-site means the symmetry means a tensor product of on-site symmetry opreators
separately.
That is, 
$g=\bigotimes_i g^{(i)}$.

The symmetry group dictates that the gauge field is a connetion on a principle
$G$-bundle,
so it's a gague field of global symmetry group $G$,
so it controls what gauge transfomatinos you have.
What I'm saying is
your therory may be defined for correlation functions without $A$,
but when you turn on the $A$ and things go wrong.
Let me just mention a few examples of 't Hooft anomalies in case you may have
seen examples before.

If you haven't seem examples before,
these examples will come up in the course.

Famous example.
You can have a $(1+1)$-dimensional chiral fermion
with $U(1)$ symmetry.
It can only exist in the boundary of an integer quantum Hall state,
which is a TQFT.

Another example.
You can have a single $(2+1)$-dimensional massless Dirac fermion
with $U(1)$ and time-reversal symmetry.
But this suffers from the famous parity anomaly.
So htis can only exist at the surface of a $(3+1)$-dimensional topological
insulator.

In particular physics,
there's a famous axial anomaly,
which is an example of a 't Hooft anomaly.

These are the most famous ones.

\begin{question}
    Are massless Dirac fermions topological?
\end{question}
No,
I'm just saying the ntion of a 't Hooft anomaly applies to QFTs in general.
These examples are not topological.

\begin{question}
    Gauge-gravity duality?
\end{question}
It's related in an either loose way or such a profound way that I don't have an
aswer.
Definitely related,
but it's hard to yeah.
It's related,
but it's one of things things,
as time goes on
we're seeing more and more hints.
At the surface it seems pretty good.

\begin{question} 
    Is there some version for CFTs of diffe.rent dimensions?
\end{question}
A CFT can certainly have a 't Hooft anomaly.
They must live at the surface of a higher-dimensional TQFT.
But a higher-dimensiona CFT doesn't really make sense,
because the boundary theory only makes snese
if there are no additional degrees of freedom in the bulk.
If the bulk has degrees of freedom,
there can be leaking of these.
Here the boundary has degrees of freemdom that cannot leak into the bulk.

\begin{question}
    Why should the bulk by invertible?
\end{question}
There's two answers.
People do talk abbout non-invertible anomalies.
There are examples,
but it suffers from the thing I was describing.
If your bulk is non-inverible,
it means there are non-trivial degrees of freedom that are also non-trivial i
hte bulk,
they can leak out and there is no clean separation between the bluk and the
boundary.

An integer quantum Hall state is well-defined,
but the boundary mode cannot exist on its own.

We can open up the bulk and have it open up the space and there is a boundary
theory there.
.
\begin{question}
    So you have abulk theorey
    You restrict it to the boundary,
    you evaluate it on hte boundary?
\end{question}
It's better to htink I have a bublk theory,
and if the manifold has a boundary,
then I will specify some boundary conditions.

\begin{question}
    When you specify the boundary condition,
    what do you mean exactly?
\end{question}
It's only when I specify boundary conditions that I can get a number $Z$.

What is the bulk?
$Z_{\text{bulk}}\left( W^{d+2} \right)\in V(\partial W^{d+2})$
is just a state,
which I take an inner product with something to specify the boundary condition.

It might be more useful that we go to examples later on.
This discussion is a bit abstract.
Different theories may correspond to different boundary conditions.
In all cases, the boundary theory will have a 't Hooft anomaly.

\begin{question}
    TFQT exist on the boundary of a therory in the sense you want to put it on a
    lattice?
    If we don't put it on a lattice and don't couple it to a gauge field?
\end{question}
Call that a macroscopic realisation.

A single fermion is not emergable from a one-dimensional system.

\begin{question}
    The Standard model is not emergable?
\end{question}
If a theory has a 't Hooft anomaly,
if you're willing to give up the symmetry,
you can realise it microscopically.
If you dedmanded you wanted to preserve axial symmetry,
you have to go to the surface of a one-higher dimensino.

In the standard mdoel,
there is actually 2 anomoalies.
There's a gauge $U(1)$ anomaly for chiral fermions,
and there's a gravitaitonal anomaly,
that you cannot give up.
It's tied to energy enregy conservation.
I don't even know what that means,
because then you have a time-dependent Hamiltonian.

I think we're getting ahead of ourselves with this example.

\begin{question}
    In a 1D condensed matter system,
    in a Luttinger liquid for example?
\end{question}
That anomaly you can think of as an axial anomaly in 1+1 dimensinos.
The theory emerges out ofa sitautiaon where it doesn't exist microscopically.

If there are no more questions about TQFTs,
we're going to talk about actual topological quantum phases of matter.

\section{1D topological superconductor}
This is describable by an invertible spin TQFT.

We're going to start off with a $(1+1)$-dimensional
Majorana-Kitaev chain.
Start with a simple toy model that is the 1d spineless $p$-wave superconductor.
That's going to be described by the following.
Imagine you have 1D array of $N$ sites.
WRite the free fermion Hamiltnoia,
which you can think of as a free field for the semiconductor.

\begin{align}
    H &=
    \sum_{i=1}^{N}
    \left[ 
    -t\left( c_i^\dagger c_{i+1} + c_{i+1}^\dagger c_i\right)
    - \mu\left( c_i^\dagger c_i - \frac{1}{2} \right)
    + \Delta c_i c_{i+1}
    + \Delta^* \chi_{i+1}^\dagger c_i
    \right]
\end{align}
and consider periodic boundary conditions $c_{i+N}=c_i$.

Consider momentum space.
\begin{align}
    \tilde{c}_k &=
    \frac{1}{\sqrt{N}}
    \sum_{j=1}^{N}
    c_j e^{ik_j}
\end{align}
where $k = 2\pi n/N$,
$n=0,\ldots,N$.
The Broullin zone is $[0,2\pi] \simeq [-\pi, \pi].
$\Delta is just any compelx number here.

Define this spinner with some redundnacy.
\begin{align}
    \Phi_k =
    \begin{pmatrix}
        c_k\\
        c_{-k}^\dagger
    \end{pmatrix}
\end{align}
then the Hamiltonian is
\begin{align}
    H &=
    \frac{1}{2}\sum_{k\in BZ}
    \Psi_k^\dagger \mathcal{H}_k \Psi_k
\end{align}
So then the Bloch Hamiltonian is
\begin{align}
    \mathcal{H}_k =
    \begin{pmatrix}
        \mathcal{E}_k & \Delta_k^*\\
        \Delta_k & -\mathcal{E}_k
    \end{pmatrix}
\end{align}
where
\begin{align}
    \mathcal{E}_k &= -2t\cos k - \nu\\
    \Delta_k &= -2i\Delta\sin k
\end{align}
The dispersion relation is
\begin{align}
    E_k &= \pm
    \sqrt\left\{ \mathcal{E}_k^2 + |\Delta_k|^2 \right\}\\
    &= \pm
    \sqrt{
        (\mu + 2t\cos k)^2
        + 4|\Delta|^2\sin^2 k
    }
\end{align}2
$E_k$ is always gapped away from $k=0,\pi$.

If $\mu=-2t$,
then it is gapless at $k=0$.
If $\mu=2t$,
then it is gapless $k=\pi$

Let's draw some pictures.

[pictures]
It's called p-wave because $\Delta_k$ is linear in $k$ for small $k$.

If the Hampton is
\begin{align}
    \mathcal{H}_k &=
    h_k^0 \mathbf{1}
    + \vec{h}_k \cdot \vec{\sigma}
\end{align}
Then the combo.
\begin{align}
    h_k^x = -h_{-k}^x
    \qquad
    h_y^y = -h_{-k}^y
    \qquad
    h_k^z = h_-zk
\end{align}

If we know $\vec{h}_k$ from $[0,\pi]$,
then we know is everywhere.
\begin{align}
    \hat{h}_{k=0} &= S_0 \hat^{z}\\
    \hat{h}_{k=\pi} &= \underbrace{S_{\pi}}_{\pm 1}\hat{Z}
\end{align}
depedning on the sign of $\mathcal{E}_\pi$.
Define $\nu = S_0 S_\pi$.

The two spaces $S_0 = S_{\pi}=2$
and $S_0 = -S_{T}$.
They are distinct so long as the energy gap stays open so $h_k$ is well-defined.

[picture]

Everything is region is topological.
Everything outside is topological.
As we tune the chemical potentitoni through kinetic energy,
if we have an odd number of Fermi points,
they we're on a topological phase.
