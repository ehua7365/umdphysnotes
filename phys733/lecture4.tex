\section{Defining data for Oriented TQFT}
This is the defining data for $(d + 1)$-dimensional oriented TQFT.
It Closed $d$-dimensional $\Sigma^d$
has vector space $V(\Sigma^d)$.
When $M^{d+1}$ is closed, meaning it has no boundary,
$Z(M^{d+1})\in\mathbb{C}$ is just a number.
If $M^{d+1}$ has a boundary,
then $Z(M^{d+1})\in V(\partial M^{d+1})$.
This is a state.
$Z$ is the path integral on the spacetime manifold.
You can think of it as a wave function over boundary conditions.

These are the axioms.
All manifolds have orientation.

\begin{enumerate}
    \item $Z$ and $V$ are functorial with respect to orientation-preserving
    diffeomorphisms of $M^{d+1}$ and $\Sigma^d$.
    \item $V$ should be ``involutory'' in the sense that
    $V(\bar{\Sigma^d}) = V(\Sigma^d)^*$.
    Here the $*$ means dual vector space.
    Over bar means hermitian conjugate.
    \item $Z$ and $V$ should be multiplicative.
    \item Non-triviality.
    \item $Z$ is hermitian.
\end{enumerate}

Let's explain the axioms.
\begin{axiom}[Axiom 1]
    Let $f:\Sigma_1^d \to \Sigma_2^d$ be a diffeomorphism.
    This defines an isomorphism
    $\mathcal{U}_f: V(\Sigma_1^d)\to V(\Sigma_2^d)$.
    If we also have $g: \Sigma_2^d \to \Sigma_3^d$,
    then
    $\mathcal{U}_g \mathcal{U}_f = \mathcal{U}_{gf}$.

    If $f$ extends to
    $\tilde{f}: M_1^{d+1}\to M_2^{d+1}$
    where $\partial M_1 = \Sigma_1$
    and $\partial M_2 = \Sigma_2$,
    then
    $\mathcal{U}_f\left( Z\left( M_1^{d+1} \right) \right)
    = Z\left( M_2^{d=1} \right)$.
\end{axiom}
Suppose we have a diffeomorphism from one circle $\Sigma_1 = S^1$ to another
circle.
Then there will be an induced map $\mathcal{U}_f$ from vector spaces between
then too.

\begin{question}
    Isn't $Z$ a path integral that is a number?
    Why is it a vector space?
\end{question}
You can think of $Z\left( M_1^{d+1} \right)$
as a state,
because like in path integrals you have boundary conditions,
which are variable.


\begin{question}
    What is a diffeomorphism?
\end{question}
A diffeomorphism $f:X_1 \to X_2$ is a smooth and continuous map.

\begin{question}
    What is a manifold?
\end{question}
Manifolds can be broken up into charts $\mathcal{U}_\alpha$
with coordinate charts $\phi: \mathcal{U}_\alpha\to \mathbb{R}^d$.

\begin{question}
What is the relation between $\mathcal{U}_{f}$ and
$\mathcal{U}_{\tilde{f}}$?
\end{question}
They are actually the same.

\begin{question}
    The extension from $f$ to $\tilde{f}$ is not unique?
\end{question}
It's not unique.

\begin{question}
    $\mathcal{U}_f$ acting on the partition function
    gives the other partition function.
    This is the sum of all paths.
\end{question}
Don't think of it like that.
This thing is only acting on the partition function.
Now you're saying $Z$ is the sum of something
and how does $U_{f}$ act on terms in the sum.
I'm hesitant to say anything general about that.
There's no general clean answer.
Just think of it as an abstract object $Z$
which gives a state if there is a boundary
and a number otherwise.

Let's talk about axiom 3.
Imagine if I build a manifold by gluing simpler manifolds.
Then I can build my vector spaces by some tensor products and multiplications.
Recall that $\sqcup$ is the disjoint union.
Then we can state the axiom.
\begin{axiom}[Axiom 3]
    Let
    \begin{align}
        V\left( \Sigma_1^d \sqcup \Sigma_2^d  \right)
        =
        V\left( \Sigma_1^d \right) \otimes
        V\left( \Sigma_2^d \right)
    \end{align}
    Suppose the boundary is
    \begin{align}
        \partial M^{d+1} =
        \Sigma_1 \sqcup \bar{\Sigma}_2
    \end{align}.
    Then
    \begin{align}
        Z\left( M_1^{d=1} \right) &\in
        V\left( \Sigma_1 \right) \otimes V\left( \bar{\Sigma}_2 \right)\\
        &\simeq V\left( \Sigma_1 \right) \otimes \overbar{V(\Sigma_2)}
    \end{align}
    so we can think of $Z\left( M^{d+1}_1 \right)$
    as a state in
    $V\left( \Sigma_1 \right)\otimes V\left( \overbar{\Sigma_2} \right)$
    or as a linear map from
    $V\left( \Sigma_2 \right)\to V\left( \Sigma_1 \right)$.
\end{axiom}
I can think of this as a state in a tensor product of two spaces.
\begin{align}
    Z\left( M_1^{d+1} \right)
    = \sum_{n,m} c_{n,m}
    \ket{n}_{V(\Sigma_1)}
    \otimes \ket{m}_{V(\overbar{\Sigma_2})}
\end{align}
or you can think of it as
\begin{align}
    Z\left( M_1^{d+1} \right)
    = \sum_{n,m} c_{n,m}
    \ket{n}_{V(\Sigma_1)}
    \bra{m}_{V(\overbar{\Sigma_2})}
\end{align}

\begin{question}
This only applies when the boundary is a disjoint union of two manifolds?
\end{question}
You can always think of the boundary as $\partial M^{d+1}=\Sigma_1 \sqcup
\emptyset$

\begin{question}
    Why do we need the complex conjugate?
\end{question}
It's a state on $V(\Sigma_1)\otimes V(\ovebar{\Sigma_2})$.
Actually maybe it shouldn't be there,
put a question mark there.
Actually, there should be no complex conjugate.

We can think of the path integral as a state on the vector space
$V(\Sigma_1)\otimes\overbar{V(\Sigma_2)}$
or as a map from
$V(\Sigma_2)\to V(\Sigma_1)$.

I can think of a state as a map from the vector space to the empty manifold.
I'm just moving pieces around no each side.

It's really nothing more than thinking of it as bras and kets.
Just think of this as a definition.
I have a state that lives on a tensor product of two vector spaces.
I can think of it as an operator from one vector space to another.
It's really just that simple.
There's nothing more to that.
Forget that we called it the path integral.
You can think of any tensor product state as an operator from one space to
another.

\begin{question}
    When you have 3 disjoint components of $\Sigma$,
    then how do you define orientation?
    How do you decided which orientation to use?
\end{question}
I don't think you need extra information about what the boundary orientation
looks like.
If this has + orientation, then are you saying that?

Actually I think there should be a star.
\begin{align}
    Z\left( M_1^{d+1} \right)
    = \sum_{n,m} c_{n,m}^*
    \ket{n}_{V(\Sigma_1)}
    \bra{m}_{V(\overbar{\Sigma_2})}
\end{align}
We'll come back to this.

The path integral is a state on the boundary.

We've done the multiplicative axiom for $V$.
We haven't gotten to the multiplicative axiom for $Z$ yet.
We want to glue together $M$s with common boundaries.

Suppose that we have some manifold $M_1$
and another manifold $M_2$.
The boundary $\partial M_1 = \Sigma_1 \sqcup \overbar{\Sigma_2}$
and the boundary
$\partial M_2 = \Sigma_2 \sqcup \overbar{\Sigma_3}$.
Then $M=M_1 \cup_{\Sigma_2} M_2$.
If you think of these $Z$ as linear maps,
then we just compose the maps.
\begin{align}
    Z(M) = Z(M_2)\circ Z(M_1): V(\Sigma_1) \to V(\Sigma_3)
\end{align}
or equivalently,
we can think of $Z(M)$ as the inner product of these two states,
where the inner product
\begin{align}
    Z(M) =
    {\langle Z(M_1), Z(M_1) \rangle}_{V(\Sigma_2)}
    \in V(\Sigma_1) \otimes V(\overbar{\Sigma_3})
\end{align}

\begin{question}
    The bar just means going counterclockwise right?
\end{question}
Yes.


Actually there are many ways of glueing manifolds.
When I identify points together,
I can actually identify with any points I want.
So in general we need a gluing map
$\phi: \Sigma_2 \to \Sigma_2$
with
$M = M_1 \cup_\phi M_2$.

I'm going to use Axiom 1
and this $\phi$ map is going to define a corresponding map
$\mathcal{U}_\phi: V(\Sigma_2)\to V(\Sigma_2)$
so
\begin{align}
    Z\left( M_1 \cup_\phi M_2 \right)
    =
    Z(M_2) \circ \mathcal{U}_\phi \circ Z(M_1).
\end{align}
Or equivalently
$\langle Z(M_2), \mathcal{U}_\phi Z(M_1)\rangle_{V(\Sigma_2)}$.
$\mathcal{U}_\phi$ is an isomorphism between these two vector spaces.

Closed $M^{d=1}$
can be obtained by gluing together
$M_1$, $M_2$ along the their common boundary.
For example,
if you wanted to calculate the path integral on a sphere,
you blue toegether the two on a common boundary.

If you want a torus,
ou glue
\begin{align}
    T^2 = (S_1\otimes I) \union (S^1\times I).
\end{align}

That was the only non-trivial axiom,
the rest are pretty simple.

Funnily enough,.
This is about what happens when we have a non-empty manifold.
the next axiom is called non-triviality

\begin{axiom}[Axiom 4: non-triviality]
    Consider a boundary $\Sigma^d = \emptyset$.
    Then
    V(\emptyset) = V(\emptyset) \otimes V(\emptyset)
    and
    \begin{align}
        V(\emptyset) =
        \begin{cases}
            0 & \text{ordinal}\\
            \mathbb{C} & 1-dimensional
        \end{cases}
        hence we have
        \begin{align}
            V(\phi) = \mathbb{C}
        \end{align}
    \end{align}
    If $M^{d+1}$ is closed,
    $\partial M^{d+1} = \emptyset$,
    then
    $Z(M^{d+1}) \in V(\emptyset) \simeq \mathbb{C}$.
    Consider the case
    \begin{align}
        M^{d+1} = \emptyset.
    \end{align}
    Then
    \begin{align}
        Z(\emptyset) = Z\emptyset) Z(\emptyset)
    \end{align}
    which implies
    \begin{align}
        Z(\emptyset) =
        \begin{cases}
            0 &\\
            1
        \end{cases}.
    \end{align}
    Take $Z(\emptyset)=1$.
\end{axiom}

consider $M^{d+1}= \sigma^d \times I$.
Then
\begin{align}
    Z\left( \Sigma^d \times I \right)
    =
    \begin{cases}
        0\\
        II_{V(\Sigma^d)}
    \end{cases}
\end{align}
Take
$Z(\Sigma^d\times I) = II_{V(\Sigma^d)}
= \sum_{n=1}^{\dim(V(\Sigma^d))}
\ket{n}\bra{n}$
where $\{\ket{n}\}$ is orthonormal basis in $V(\Sigma^d)$.
Or,
\begin{align}
    Z(\Sigma^d\times I) =
    \sum_n
    \ket{n}_{V(\Sigma^d)}
    \otimes
    \ket{n}_{V(\Sigma)}^d.
\end{align}

The last axiom is simple.
\begin{axiom}[Hermitian]
    $Z(\overbar{M^{d+1}}) = Z(M^{d+1})^\dagger$.
\end{axiom}

\section{Non-trivial consequences}
\subsection{Homotopy invariance of maps}
This has some non-trivial consequences.
Consider $M_1^{d+1}=\Sigma^d\times I$
and consider $M_2^{d+1}=\Sigma^2\times I$
with homotopy
$F: M_1\to M_2$.
Then
$\mathcal{F}(\Sigma^d, t)
= f_t(\Sigma^d)$
for $f_t: \Sigma_d\to \Sigma^d$.
This defines a continuous family
$\mathcal{U}_{f_t}: \Sigma^d\to \Sigma^d$ satisfying
$\mathcal{U}_{f_t} = \mathcal{U}_{f_{t'}}$.

[diagram here]

If I have a homotopy group with a homotopy of maps.

\subsection{Mapping Class Group}
That leads us to the important group for a manifold.
For every manifold, there is a mapping class group $\Sigma^d$.
Colloquially, it's the group of ``large'' diffeomorphisms.
You would take all diffeomorphisms and mod out all those that are continuously
\begin{align}
    \mathrm{MCG}(\Sigma^d) = \frac{\mathrm{Diff}(\Sigma^d)}{\mathrm{diff}_0(\Sigma^d)}
\end{align}
Two diffeomorphisms are equivalent if they can be connected to each other.

This is the group taking all possible diffeomorphisms
and modding out all the ones equivalent to the identity.

The mapping class group is related to the homotopy group,
but not the same thing.
You may not understand this,
but
\begin{align}
    \mathrm{MCG}(\Sigma^d)
    = \mathrm{out}(\pi_1(\Sigma^d))
\end{align}
It's isomorphic to the group of all outer maps between loops.
An example of such a map that maps a non-trivial loop on an torus
to the other type of non-trivial loop around the torus.

In TQFT,
all my maps are smooth.
If my maps aren't smooth,
I don't have something that takes me everywhere from the left to everywhere on
the right.
If I have a smooth map,
that's what allows me to take vector spaces on the left to vector space on the
right.
The only reason I can go from left to right is because $\mathcal{F}$ is a
diffeomorphism.

The punchline is this.
For each $g\in \mathrm{MCG}(\Sigma^d)$,
we get some $\mathcal{U}_g: V(\Sigma^d) \to V(\Sigma^d)$.
If you figure out the $\mathcal{U}_g$,
you can completely figure out the TQFT.
These $\mathcal{U}_g$ are very important,
and can basically characterise the TQFT.

Is this the $S$ and $T$ matrices?
Yes.
For every element in the MCG  of a torus,
there is a linear map that acts on the ground state subspace on the system,
hence the $S$ and $T$ matrices on the subspace.
Don't worry if you've never heard of it.

Do you have a concise reference to recommend for the last two lectures?
There's an article from Atiyah 1988.
It's basically called TQFT.
My whole representation is based on this article.

\begin{question}
    Is it readable?
\end{question}
It's as readable as my lectures are understandable.
If you don't understand my lecture,
then I'm not sure.
It's not too bad,
it doesn't assume you know that much.

We're basically done,
but there's one more consequence to talk about.
These TQFTs are very baby versions of TQFTs,
and in real applications, we need some extra structure.
