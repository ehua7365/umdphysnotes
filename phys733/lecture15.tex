\section{}
The chiral central charge tells you the heat current.
If you had a single chiral Majorana mode it's half.

\begin{table}
    \centering
    \begin{tabular}{ccccc}
        (1+1)D & $\mathbb{Z}_2$ invariant &
        $\mathcal{I}=P_0 P_\pi$ &
        $G_f = \mathbb{Z}_2^f$ &
        $\psi\to e^{i\theta}\psi$ Class D\\
        (2+1)D & $\mathbb{Z}$ invariant Chern number &
        C &
        $G=U(1)$ &
        [$G_f=U(1)^f$] Class A\\
        & $\mathbb{Z}$ invariant &
        $\nu=2C_{-}$ (sp. Chern # for free fermions) &
        $G_f=\mathbb{Z}_2^f$
    \end{tabular}
    \caption{Free fermion models}
    \label{tab:freefermions}
\end{table}

\begin{question}
    What is a full Majorana measurement?
\end{question}
Measurement of $\gamma_1 \gamma_j \gamma_k \gamma_l$.
Joint fermion parity.


Consider $U(1)_\uparrow$ with Chern number $C_\uparrow$
and $U(1)_\downarrow$ with Chern number $C_\downarrow$
with
$C_{\uparrow} = -C_{\downarrow} = C$.
There is also time-reversal symmetry.

Chern number breaks time-reversal and reflection symmetry.
It breaks time rerseal because the direction of the current on the edge would
have to switch sign.

But if they have opposite Chern number on both sides with two components then
you can.

A two-component system with $C_\uparrow = -C_\downarrow$ can be time-invariant.

Let's study the edge theory in a particular case here.

\subsection{Edge theory for $C_\uparrow=C_\downarrow=1$}
The Hamiltonian would be like
\begin{align}
    H &=
    \sum_k \left[ 
    |V_\uparrow| k \psi_{\uparrow,k}^\dagger \psi_{\uparrow,k}
    |V_\downarrow| k \psi_{\downarrow,k}^\dagger \psi_{\downarrow,k}
    \right]
\end{align}
Non-trivial topological phases have something interesitng at the edges.
So consider a backscatttering term or a mass term for htis fermion
\begin{align}
    m \psi_\uparrow^\dagger \psi_\downarrow + h.c.
\end{align}
This breans spin consdervation,
which is the relative $U(1)$.
This breaks the $S_z$ spin conservation.
But that's not so surprising because you have an up fermion and that gets
backscattered by a down fermion,
and it's not so surprising that breaks spin conservation.
More interesting,
this term also breaks time-reversal symmetyr.

Let's deine time reversal to hae the following actiona
\begin{align}
    T: \psi_\uparrow &\to \psi_\downarrow\\
    \psi_\downarrow &\to -\psi_\uparrow
\end{align}
which means on a single fermion $T^2=-1$,
and so
\begin{align}
    T^2 &= {(-1)}^F
\end{align}
So you could say this is a $\mathbb{Z}_4^{T,f}$ symmetry.
But there's also a $U(1)^f$ symmetry.
But if you apply one and apply another,
the order matters,
and there's a complex conjugation when you reverse the order.
And you need to mod out to consider equivalence.
So the symmetry of the system is this
\begin{align}
    G_f &=
    \frac{U(1)^f \rtimes Z_4^{T,f}}{\mathbb{Z}_2}.
\end{align}
This is called the class AII symmetry group in the Cartan classification.

Fermionic symmetry group $G_f$.
Consider whath appens when you act the symmetry only on bosonic operators,
then you can ask what symmetry acts on them,
and that allows you to define a bosoinc symmetry gorup
\begin{align}
    G_b &= G_f/\mathbb{Z}_2^f
\end{align}

If you look at the bosoinc symmetry group,
then
\begin{align}
    G_b &=
    U(1) \rtimes \mathbb{Z}_2^T
\end{align}
Under time-reversal symmetry,
\begin{align}
    T (M \psi_\uparrow^\dagger \psi_\downarrow) T^\dagger &=
    - m^* \psi_\downarrow^\dagger \psi_\uparrow
\end{align}
You'd think there's a single total Chern number.
Surprisingly,
there's still something non-trivial left over.
This suggests there's still some non-trivial topological phase still left over
in the system.
You can then consider two copies.
\begin{align}
    C_\uparrow = - C_\downarro = 2
\end{align}
then you can write the same thing,
but there's an extra flavour $s$.
\begin{align}
    H &=
    \sum_{k,s} \left[ 
    |V_{\uparrow,s}| k \psi_{\uparrow,k,s}^\dagger \psi_{\uparrow,k,s}
    |V_{\downarrow,s}| k \psi_{\downarrow,k,s}^\dagger \psi_{\downarrow,k,s}
    \right]
\end{align}
Under time-reversal,
\begin{align}
    T: \psi_{\uparrow, s} & \to \psi_{\downarrow,s}\\
    \psi_{\downarrow,s} &\to -\psi_{\uparrow,s}
\end{align}
And now you can consdier what terms you can add,
and it turns out you can add a backscaettering term.
\begin{align}
    \delta H_{\textrm{backscattering}} &=
    m\left(
    \psi_{\uparrow,1}^\dagger \psi_{\downarrow, 2}
    - \psi_{\downarrow,1}^\dagger \psi_{\uparrow, 2}
    + \mathrm{h.c.}
    \right)
\end{align}
and now you can open an energy gap.
This suggests the bulk topological phase should have
$\mathbb{Z}_2$ invariant.

This is the bulk-edge correspondence.
You can deduce the bulk properties by looking at he stability of the boundary
modes under action of the symmetry group.

Now we can consider whath appens when this time-reversal symmetry does not have
this minus sign.
Suppose that instead
\begin{align}
    T: \psi_\uparrow &\to \psi_\downarrow\\
    \psi_\downarrow & \to \psi_\uparrow
\end{align}
and so $T^2=+1$.
Then the symmetry groups are
\begin{align}
    G_f &= U(1)^f \rtimes \mathbb{Z}_2^T\\
    G_b &= U(1) \rtimes \mathbb{Z}_2^T
\end{align}
and then $C_\uparrow = C_\downarrow = -1$
and it does not have protected edge modes.
The fact that $G_f$ is different has consequences on what ind of backscattering
terms you can add.

Fermions can ermge in a variety of situations in solid state systems.
I don't have a good example,
but you can have fermions show up in solid state systems.
Of ten this kind of time reversal can be thoguht of as a effectie time reversal
where you combien it wiht some kind of rotation.
This may be the true time reversal plus some rotation,
but we still call it time reversal because it's anti hermitian.

The $\pi$ rotation in $G_f$ becomes the $1$ operation in $G_b$.
This becomes implicit in
\begin{align}
    G_b &= G_f / \mathbb{Z}_2^f
\end{align}
This becomes a trivial operation as far as a boson is concerned.

\begin{question}
    Why do we ultimately have to define $G_b$?
    The first symmetry gourp is $G_f$.
    Why do we not just work with $G_f$?
\end{question}
I didn't use $G_b$ anywhere,
I just noted $G_b$.
And depending on how sophisticated the theroy is,
it could be useful,
but not in this particular scenario.

Let me remind you of the important Kramers theorem.

If $T^2=-1$,
it measn the minimal representation is 2-dimensional,
whichm eans if you have a state on which $T^2$ acts as $-1$,
it has to be a doublet.
The proof is two lines.
Suppose you act $T$ on a state and you don't change it
\begin{align}
    T\ket{\psi} &= \lambda\ket{\psi}\\
    -\ket{\psi} = T^2 = \lambda^* \lambda \ket{\psi}
\end{align}
So therefore,
$T$ msut bring it to another state
\begin{align}
    T\ket{\uparrow} &= \lambda \ket{\downarrow}\\
    T\ket{\downarrow} &= \lambda^' \ket{\uparrow}
\end{align}
and $\lambda\lambda' = -1$.
which is a contradiction.
So that's Kramers theorem.

consider a cylinder with a momentum around a circle $k_y$ and let $x$ e the axis
of the cylinder.

Consider the spectrum of $E$ vs $k_y$.
there are bulk states in tethe valence conduction bands.
You can see that $T: k \to -k$,
because intiuitively time reversal changes the sign of momentum.
Then the possible time-reversal momenta are
\begin{align}
    \vec{k} &=
    \begin{cases}
        (0, 0)\\
        (0, \pi)\\
        (\pi 0)\\
        (0, 0)
    \end{cases}
\end{align}
There's two topological distrinct scenarios that can happen.
Either these states connect up in a non-trivial way.
This merges into the bulk and then this comes out.

Thisis topological.
No matter where I put my chemical otential,
I'mg going to hit exactly 1.
The trivial situation is when Karmers pairs connect up in a non-trial wahuoWe
have these states and I can have ooThe Kramrs degenerate states in time-reversal
omentua connect up.

You can geta band invariant bby mulitplying hte hern nmber through the Brilluoin
zone.

\section{\mathbb{$Z_2$}c  band invariant}
Define
\begin{align}
    W_{mn}(k) &=
    \bra{U_m(k)} T \ket{U_n(-k)}
\end{align}
where $w$ is unitary.

\begin{align}
    w^T(k) = -w(-k)
\end{align}
$\Lambda_a$ = $ trim,
$a+1,\dpts.a$4$
\begin{align}
    \lambda_n = \mathrm{TRIM}\qquad a=1,\ldots, 4\\
    w^T(\Lambda_a) &= -w(`A_a)
\end{align}
and
\begin{align}
    \delta_a &= \frac{\Pf(w(\Lambda_a))}{\sqrt\det(w(\lambda_a))}}
    = \pm 1
\end{align}
Here
\begin{align}
    \pm 1 &= (-1)^{\nu}
    = \prod_{a=1}^{N}
\end{align}
Need $\sqrt{deg{w(k)}}$ continuous everywhere in BZ.
Can do this if Bloch wave functions we defined continuously everywhere.
Possible boundary condition that total Chern number = 0 by $T$.


\subsection{Time-reversal polarization in 1D}
Fu-Kane arXiv 0606336.
Can split up Bloch states of occupied bands in the two classes.
\begin{align}
    \ket{U_m^I(k)}, \ket{U_m^{II}(k)}\\
    T\ket{U_m^I(k)} &= e^{i\chi_m(k)}\ket{u_m^{II}(-k)}.
\end{align}
Define polarization $P_I$ and $P_{II}$ fore each class.
Total polarization
\begin{align}
    P &= P_I + P_{II}
\end{align}
and time-reversal polarization
\begin{align}
    P_T &= P_I - P_{II}
\end{align}
\begin{align}
    \pm 1 &= (-1)^{P_T} = \delta_1 \delta_2
\end{align}
for $\Lambda_1=0$ $\Lambda_2=\pi$ where
\begin{align}
    \delta_{a} &=
    \frac{\Pf(w(\Lambda_a))}{\sqrt{\det(w(\Lambda_a))}}
\end{align}

The physical meaning is this.
$(-1)^{P_T}=-1$ means boundary has a Kramers degeneracy in each end.

\subsection{Back to 2D}
Can think of 2D system as parameterized 1D
\begin{align}
    H_{2d} &=
    \sum_{k_y} H_{1d}\left( k_y \right)
\end{align}
Consider change in $P_T$ from $k_y=0$ to $\pi$.
\begin{align}
    {(-1)}^{P_T(k_y=\pi) - P_T(k_y=0)} &=
    \delta\left( (0, 0) \right)
    \delta\left( (\pi ,0) \right)
    \delta\left( (0, \pi) \right)
    \delta\left( (\pi, \pi) \right)\\
    &=
    {(-1)}^{\nu}
\end{align}
You can think of this as localized Wannier states.

Just like there is a story for insulators as superconductors,
this story also has an analogue for topological superconductors.
Instead of Chern number $\pm 1$,
we can also consider $p+ip$ or $p-ip$.

Consider a different symmetry group
\begin{align}
    G_f = \mathbb{Z}_4^{T,f}
\end{align}
and there's no $U(1)$.
This is called class DIII.
This would be for topological superconductors.
Consider 2 layers.
There is an invariant $\nu$ for the $p+ip$ layer and
$-\nu$ for the $p-ip$ layer.

All this stuff about topological pumps.
The story on why states are protected.
We get the $\mathbb{Z}_$ classification of topological phases.
From a high level it's very similar.

If you're okay,
we can go to (3+1)D.
Once we've done that,
we'll just leave the world of free fermion topological phases
and go on to describe some other kinds of things.

So we wrote down this band invariant $\nu$.
If you go up one dimension you can consider a similar invariant.

For (3+1)D Classs AII,
\begin{align}
    G_f &= \frac{U(1)^f\rtimes \mathbb{Z}^{T,f}}{\mathbb{Z}_2}
\end{align}
and
\begin{align}
    (-1)^\nu &= \prod_{a=1}^8 \delta_a
\end{align}
8 TRIM:
\begin{align}
    \Lambda_a &= \left( \pi s_1, \pi s_2, \pi s_3 \right)
\end{align}


\section{(3+1)D massive Dirac fermion}
It is useful to start in (4+1)D.
\begin{align}
    H &=
    \sum_{a=0}^{4}
    \sum_{k} \psi_k^\dagger d_a(k)\cdot \Gamma^a \psi_k
\end{align}
where $\Gamma^\mu$ are $4\times 4$ Dirac matrices wiht $\mu=0,\ldots,4$.
Then the anticommutation is
\begin{align}
    \left\{ \Gamma^{\mu}, \Gamma^{\nu} \right\} = 2\delta_{\mu\nu} II
\end{align}
This can be  parameterized family of 3 + 1 dimensional models,
parameetrized by a parameter $k_w$.
\begin{align}
    d_a(k) &=\left( 
    m +c \sum_{i=1}^{4}\cos(k),
    \sin k_x,
    \sin k_y,
    \sin k_z,
    \sin k_w
    \right)
\end{align}
so
\begin{align}
    H &= \sum_{k_w} H_{3+1Dj}(k_1,k_2,k_3,k_w).
\end{align}
Then we couple this to a background $U(1)$ gauge field.

View $k_w + A_w = \theta$ as a parameter.
I view this fourth dimension as a parameter.
The 4 + 1 D effective action if we integrate out the fermions is basically a
Chern Simons term.
So it's going to give
\begin{align}
    S_{\mathrm{eff}} &=
    \frac{C_2}{24\pi^2}
    \int d^4x\, dt\,
    \epsilon^{\mu\nu\lambda\sigma}
    A_{\mu} \partial_{\nu} A_{\lambda} \partial_{\rho} A_{\sigma}
\end{align}
where $C_2$ is the second Chern number of Bloch states
\begin{align}
    C_2 &=
    \int \mathcal{F}_{\mu\nu} \mathcal{F}_{\lambda\sigma}
    \epsilon^{\mu\nu\lambda\sigma}
\end{align}
where $\mathcal{F}_{\mu\nu}$ is the Berry curvature of the Bloch wave functions.
