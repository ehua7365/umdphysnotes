\section{Lecture 22}
\begin{question}
    Last time for 2D systems,
    something about them not being gapped?
    In 1 spatial dimension you can have degenerate gapless modes,
    but not in 2D?
\end{question}
If I take a 1D system with boundaries,
in general you have edge modes,
and in general there is a ground state degeneracy
with dimension $\dim \mathcal{H}_L \times \dim \mathcal{H}_R$.

In 2D,
you only have a single edge,
so if the ground state is preserved,
the ground state is unique.

Of course,
you could have an annulus with two edges,
but it's still true with an edge far away.
There is no way of having just a single edge in 1D.
You always have to have two edges.
You can have an edge which is connected.

The presence of 2 edge still allow you to have 2 edges?
The edge may spontaneously break a symmetry,
but these tow are in spin singlet together and together they may preserve
symmetry,
but that's not useful and natural to do.
You talk about a single edge and its properties in 2D.

Maybe the better thing to say is that in 2D,
the edge can have a unique ground state that is symmetric,
but of course it doesn't have to be if you had multiple edges,
whereas in 1D it cannot under any circumstances,
unless maybe the edges are in a singlet with one another.


\section{(2+1)D Chern-Simons Theory}
So the idea here is that this is some gauge theory,
based on a gauge group $G$.
You can consider all kinds of gauge groups with strange properties,
but for our purposes,
we're interested in cases where this is a compact gauge group.
This gives rise to a large class of TQFTs

In general if you want to specify a Chern-Simons theory,
let me just mention that,
you want to specify the gauge group and you also want to specify some element
$[\alpha] \in H^4(BG,\mathbb{Z})\simeq H^3(BG, U(1))\simeq H^3(G, U(1))$
So a Chern-Simons theory is defined by a group and some choice of cohomology
class.

The CS theory is very powerful.
It's a class of gauge theories that is topological,
and when you quantize it,l
it leads to a class of TQFTs,
and in general these TQFTs can be non-invertible.
They typically depend on a framing,
and they may also depend on a spin structure
with spin TQFTs in the most general case.

In fact,
it is a conjecture that all $(2+1)D$ TQFTs can come from a Chern-Simons theory.
This conjecture goes back to about 1989 to 1990.
And there's still no definitive counterexample.
Although there are some candidate theories people don't know how to interpret
in terms of CS theory,
so there are some tentative counterexample,s
although it's not clear whether they actually are.
There's n definitive counterexample yet,
so CS theory is very powerful.

I want to talk about the simplest case where the gauge group is $U(1)$.

\section{$U(1)_k$ CS theory}
Consider an action
\begin{align}
    S &=
    \frac{k}{4\pi}
    \int_{M^3} a\, da\\
    &=
    \frac{k}{4\pi}
    \int_{M^3}
    \epsilon^{\mu\nu\lambda}
    a_\mu
    \partial_\nu
    a_\lambda
    \,
    d^3x
\end{align}
where $a$ is a $U(1)$ gauge field.
If you stare at it,
this does not actually make sense in general.

The reason it doesn't make sense in general,
is that a U(1) gauge field is not globally well defined in general.

A gauge field is locally just a 1-form $a_\mu \, dx^\mu$.
\begin{align}
    \int_{Y_2} F = 2\pi
\end{align}
If the field strength through some 2-cycle is $2\pi$,
you cannot pick a global gauge for $A$ if $Y_2$ is closed.
So this expression doesn't make sense,
because there's no globally well-defined choice of a $U(1)$ gauge field.
If $Y_2$ is a closed 2-manifold,
then there is no global gauge $a$.

So we cannot make sense of $S_{CS}$ itself,
but we can make sense of $e^{iS_{CS}}$.

We can in general evaluate this $\theta$ term in one dimension higher.

To make sense of this,
we pick a 4-manifold $W^4$ such that the boundary
$\omega W^3 = M^3$.
Furthermore,
we're going to extend the gauge field $a$
to be defined over $W^4$.
It turns out this is always possible.
For any 3-manifold,
you can always pick a 4-manifold whose boundary is that 3-manifold.
Conceptually,
the reason this can always be done is precisely because this cobordism group is
zero
\begin{align}
    \Omega_3 \left( BU(1) \right) = 0
\end{align}
That this is zero is that every 3-manifold with a U(1) gauge field is 
is the boundary of a 4-manifold with a U(1) gauge field.
The mathematicians tell us this fact.

\begin{question}
    What about other groups?
\end{question}
This doesn't necessarily hold for every group.

But suppose instead you were interested in SU(2).
Turns out this problem doesn't occur anymore,
because for SU(2) the first Chern class if trivial.
That we run into this problem is specific to the U(1) case.
It doesn't happen for SU(2),
but it doesn't happen for SO(3).
I'm not sure how general this statement is.
Everything that comes up in physics,
this cobordism group does always vanish.
I don't quite remember.
Let's just stick to U(1).
For SO(3) you might need to do something a little different.


Anyway, so then define
\begin{align}
    S_{CS} &=
    \frac{k}{4\pi}
    \int_{W^4} \partial_\mu a_\nu \partial_\lambda a_\sigma
    \epsilon^{\mu\nu\lambda\sigma}\\
    &=
    \frac{k}{16\pi}
    \int f_{\mu\nu} f_{\lambda\sigma} \epsilon^{\mu\nu\lambda\sigma}\\
    &=
    \frac{k}{4\pi} \int f\wedge f
\end{align}
This is completely well-defined because $f$ is a locally well-defined quantity.
The reason it gives the CS term on the boundary is because this term is a total
derivative,
and if you evaluate over the boundary it just gives you the CS term.

So this term is a total derivative and it gives a CS term on the boundary.
\begin{align}
    \partial_\mu a_\nu \partial_\lambda a_\sigma
    \epsilon^{\mu\nu\lambda\sigma}
    &=
    \partial_\mu\left( 
    a_\nu \partial_\lambda a_\sigma
    \right)
    \epsilon^{\mu\nu\lambda\sigma}
\end{align}

You might be worried,
because I'm it looks like I'm defining a (3+1)D theory,
but the result should better not depend which 4-manifold I pick,
and how I choose the gauge field in one-higher dimension.
To get an intrinsically $(2+1)D$ theory,
$S_{CS}$ should not depend on the extension $W^4$ and $a$.


In order it makes sure they're not the same,
compare 2 different extensions.

We want this difference to be $2\pI$ times an integer
\begin{align}
    \delta S &=
    \frac{k}{4\pi} \int_{W^4}f \wedge f
    - \frac{k}{4\pi} \int_{W^4'} f' \wedge f'
    \in 2\pi\mathbb{Z}
\end{align}
What I do is glue these two 4-manifolds together on the boundary which is $M^3$
and I end up with a closed 4-manifold.
\begin{align}
    X^4 &= W^4 \cap_{M^3} \overline{W_4'}
\end{align}
with
\begin{align}
    \tilde{A} &=
    \begin{cases}
        a & \text{on } W^4\\
        a' & \text{on } W^4'\\
    \end{cases}
\end{align}
Recall that
\begin{align}
    \frac{1}{4\pi^2}
    \int_{X^4} F\wedge F \in \mathbb{Z}
\end{align}
for a closed $X^4$.
And that's just integrating the entire field strength over this entire manifold.
So then
\begin{align}
    \delta S
    &=
    \frac{k}{4\pi}
    \int _{X^4} \tilde{f} \wedge \tilde{f}
\end{align}
so in order for this to be an extension,
I need $k\in 2\mathbb{Z}$ to be an even integer.

For this CS theory with $e^{iS}$ to make sense,
we need $k$ to be an even integer.

\begin{question}
    global gauge?
\end{question}
No the point is that on a 4-manifold I don't need to pick a global gauge,
because $f$ always makes sense.
On the boundary it's the same,
but it can be whether on the bulk.

Another way to see why $k$ has to be an even integer,
consider $M^3=T^3$ is a torus.
Assume
\begin{align}
    \int_{T^2} F_{12} = 2\pi
\end{align}
Then the action is
\begin{align}
    S_{CS} &=
    \int
    \frac{k}{4\pi}
    \left( 
    a_0 f_{12}
    +
    a_1 f_{21}
    +
    a_2 f_{01}
    \right)
\end{align}
and so let's do a large gauge transformation,
which takes
\begin{align}
    a_0 \to a_0 + 2\pi/L_0
\end{align}
where $L_0$ is the length in the time (0) direction.
So under this large gauge transformation,
there's going to be a change in this action
\begin{align}
    \delta S_{CS}
    &=
    \frac{k}{2}
    \int_{T^2}
    f_{12}
    =\pi k
\end{align}
and in order for this action to be independent of large gauge transformations,
we need $k$ to be an even integer.
Maybe that's a quicker way to see why $k$ needs to be an even integer

\begin{question}
    Why is it called a large gauge transformation?
\end{question}
Because this gauge transformation changes a flux through a hole by $2\pi$.
That kind of gauge transformation is not continuously connected to the identity.
Let's say
\begin{align}
    a \to a - \delta \lambda
\end{align}
what would $\lambda$ be?
We would need $\lambda = \frac{2\pi}{L_0}t$,
which is winding,
which is topologically non-trivial.

You might be a little bothered if you've done nay CS theory,
because now I'm telling you $k$ has to be even,
but we often work with $k$ being odd or 1 in integer quantum Hall states.

So here's the catch.
If you're working with fermions,
then $k$ is odd.

When working with fermions in your theory,
and you're doing QFT,
you want the manifold to have spin structure.

It's a fact that all 3-manifolds
admit a spin structure.
So we can define fermions on them,
but not all 4-manifolds admit a spin structure.
And the counter example would be $\mathbb{CP}^2$.
If you demand your theory has fermions,
then you only ant to consider extending to 4-manifolds
that admit spin structure.

So you require that $W^4$ is actually a spin manifold,
which is just a manifold that admits a spin structure.
So if $W^4$ is a spin manifold,
then $X^4$ is also going to be a spin manifold.

Extension is always possible,
and this time it's because
\begin{align}
    \Omega_3^{spin}\left( BU(1) \right) = 0
\end{align}
which we have to resort to mathematicians telling us this is the case.
Turns out if you have a spin 4-manifold,
then the second Chern class is actually even.
\begin{align}
    \frac{1}{4\pi^2}
    \int_{X^4}
    F\wegde F
    \in 2\mathbb{Z}
\end{align}
for closed spin 4-manifolds.
So what you find is
\begin{align}
    \delta S &= 2\pi k
\end{align}
which implies that
\begin{align}
    k \in \mathbb{Z}
\end{align}
and so $k$ can be odd.

If $k$ is odd,
we have a TQFT that has fermions,
that has to have spin structure,
and in the extension to 4D must also have spin structure.

So odd $k$ defines a \emph{spin} TQFT that describes a theory with fermions.

CM physicists don't worry about this issues,
because often when CM physicists work with CS theory,
they are only interested in the case when the manifold $M^3$ is $\mathbb{R}^3$
or just a ball $D^3$.
If we never consider topologically non-trivial manifolds,
then these issues don't really come up.
And we don't need to worry about the quantization of $k$.

On the other hand,
if the gauge group is $\mathbb{R}$ instead of $U(1)$,
then again we don't have to worry about the quantization of $k$.

So the quantization of $k$ only comes up when we use the fact that the gauge
group is $U(1)$,
which means for large gauge trnasofmation,
integrating magnetic field thorugh large lookps is quantized.

The quantization of $k$ only comes up when thinking of topologically non-trivial
situations.

\begin{question}
    What goes on with this argument?
    You can't even define on patcthes?
\end{question}
With fermisns,
you shouldn't even consider this 3D action,
you sould start with the 4D action,
and this issue won't come up.
You can consider the theory on $\mathbb{R}^3$ or a ball,
but you just can't do it for topologically non-trivial situations.
Maybe ther's a way of putting it together,
but I don't know.

So either you work in patches or $\mathbb{R}^3$,
or just consider the gauge group to be $\mathbb{R}$.
But for fermions,
it's even more necessary because of this issue.

Often,
we continue using this action when we care the gauge group is $U(1)$ and the
situation is topologically non-tvivial,
but usually that's when you have to hae enough expertise to know when and when
not it gives the right answers because of these subtlties.

\section{CS-Maxwell theory}
One more thing I want to just mention,
but I'll put the calculation in your homework.

Consider $\mathbb{R}^3$, but let's add the Maxwell action as well.
\begin{align}
    S_{CS-Maxwell}
    &=
    \int_{\mathbb{R}^3}
    \left( 
    \frac{k}{4\pi}
    a\, da
    -
    \frac{1}{4g^2}
    f_{\mu\nu} f^{\mu\nu}
    \right)
\end{align}
And we know Maxwell theory has gapless photons in Maxwell theory
if $k=0$.
But if $k\ne 0$,
then the photon acquires a mass,
and gets a gap.
It's away of giving the photon a mass gap,
very similar to the Higgs mechanism,
like in superconductors,
with the Meister effect which gaps out.


If we consider pure Chern-Simons theory,
just this $S_{CS} = \frac{k}{4\pi}\int a\, da$,
then the classical equation of motion tells us the field strength is actually
zero with $f_{\mu\nu} = 0$.
But since we're on $M^3 = \mathbb{R}^3$,
there's really nothing to talk about here.
We can write $a_\mu = \partial_\mu \phi$,
and the action is going to be zero because the field strength is zero.
If $S=0$, then the Hamiltonian $H=0$.
So pure classical CS theory on $\mathbb{R}^3$ has nothing in it.
Everything is completely trivial.

\section{Quantizing CS Theory}
The quantization of CS theory is non-trivial and subtle if you want to do it
completely.
We'll start off doing canonical quantization and try to understand the Hilbert
space of the system on some space.
Let me just give a preamble.

Formally,
one thing we can do is to define the path integral.
\begin{align}
    Z 
    \int \mathcal{D} a\,
    e^{i S_{CS}(a)}
\end{align}
but we need to do some gauge fixing,
as you know from QFT.

At the classical level,
we have this action that has no geometry dependence on the metric at all,
unlike maxwell.
So at a classical level,t
her theory is fully topological,
but it's not clear it will remain topological after quantizing.
You have to gauge fix to properly evaluate the path integral,
and you could lose it.
This is where the notion of framing comes in.
When gauge fixing,
you pick some metric to do gauge fixing,
then define a path integral independent of metric,
but it turns out you can't make it completely independent,
so her is some residual which is the framing.
I don't want to do all that right now.
I want to do something simpler,
just the canonical quantization of theory on $M^3$ being the form
torus cross time $M^3 = T^2 \times \mathbb{R}$.

Here things are easier.
I can always just pick a gauge where the time component $a_0=0$
I can't just forget about it entirely,
I need to remember the equation of motion for $a_0$ is non-trivial,
which hence places a constraint.
The equation of motion is
\begin{align}
    \frac{\delta S}{\delta a_0}
    \propto
    \epsilon^{ij}f_{ij} = 0
\end{align}
which becomes a constraint.
I find this explanation a bit opaque.
I prefer the following equivalent explanation.
Consider the Lagrangian
\begin{align}
    \mathcal{L}
    &=
    \frac{k}{4\pi}\left( 
    a_0 \epsilon^{0ij} \partial_i a_j
    +
    \epsilon^{i0j} a_i \dot{a}_j
    +
    \epsilon^{ij0} a_i \partial_j a_0
    \right)
\end{align}
and then I integrate by parts,
change around the order and relabel things,
terms come together
\begin{align}
    \mathcal{L} &=
    \frac{k}{2\pi}
    a_0 \epsilon^{ij} \partial_i a_j
    +
    \frac{k}{2\pi}
    a_2 \dot{a}_1
\end{align}
and note $a_0$ only enters on the first term,
and I can think of $a_0$ as a Lagrange multiplier
enforcing the constraint
\begin{align}
    \frac{k}{2\pi}\epsilon^{ij} \partial_i a_j = 0
\end{align}

\begin{question}
    By framing you mean framing of the gauge?
\end{question}
No, I haven't gotten into it,
but what I meant was fixing a gauge means choosing a metric.
Evaluate the path integral,
and you find the path integral depends on the metric.
But then you try to see if you could have done anything to save the topological
invariance.
You could get rid of the metric almost completely,
but not completely,
and it depends on a choice of framing,
which is a choice of trivialization of the tangent bundle.
You have to choose 3 vectors everywhere that don't degenerate,
which is picking a trivialization of the tangent bundle.
I'm not presenting that, so don't worry about it.


So the gauge field needs to have no magnetic field in space.
We can parametrize the flat gauge field configuration on $T^2$ with
\begin{align}
    a_i(x_1, x_2, t) &=
    \frac{2\pi}{L_i} X_i(t)
\end{align}
The gauge field should have no magnetic field through the torus,
and there will be holonomic which can be parametrized by writing in this form.
These $X_i$'s are called zero modes.
This $X_i$ is like the spatial average of $a_i$.
This is proportional to an integral over space,
and overall average if you like
\begin{align}
    X_i(t) \propto \int d^2x\, a_i
\end{align}
And so $X_i$ is related to $X_i + 1$ by large gauge transformations
with $X_i \sim X_i + 1$.
So $X_1$ and $X_2$ live on a torus.
If I substitute this form back into the Lagrangian,
remembering that $a_0$ is zero,
so only the second term remains,
and when I substitute this 
\begin{align}
    L &=
    \int_{T^2} d^2x \, \mathcal{L}\\
    &=
    2\pi X_2 \dot{X}_1
\end{align}
Doing some classical mechanics,
the canonical momentum is
\begin{align}
    P_{X_1} &=
    \frac{dL}{d \dot{X}_1} = 2\pi k X_2
\end{align}
so the Hamiltonian is
\begin{align}
    H &=
    P_{X_1}\dot{X}_1 - L = 0
\end{align}
which is strange.
But you find that your phase space itself is compact with some finite volume.
And the volume of this phase space is $2\pi k$.
This is a very unusual situation.
Almost always in classical mechanics,
your phase space is infinite volume.
Even if your coordinate is bounded,
the momentum is unbounded.
So this is a weird situation where the phase space itself has finite volume.
If you remmeber QM,
one way of thinknig about quantum sates,
for every phase space volume,
you assing one quantum state.
Here $\hbar=1$,
we get one quantum state for each $2\pi$ volume of phase space.
So we conclude the dimension of this Hilbert space on a torus has to be $k$.
\begin{align}
    \dmi \mathcal{H}\left( T^2 \right) &= k.
\end{align}

So what we learn is that
\begin{align}
    |Z(T^2 \times S^1)|
    =
    \dim \mathcal{H}(T^2) = |k|
\end{align}

\begin{question}
    Does that have something to do with the path integarl not being well defined
    yo umentinoed in the beginningof the course?
\end{question}
No.
Think aoubt this path integral,
it's integrating over all possible gauge fields.
this is perfectly well defined with phase space compact.
Usually in QFt,
you need to make sense of this paht integral hwihhci is infintie,
so you can only define correlation fucntions which are ratios of path integrals.
If you do this precisely,
you can make this path integral well-deifined and convert it to actual numbers.
I haven't told you how to do that yet.
The phase space compact is just a weridness of Chern-Simons theory.

You can repeat the above calculation on any genus $g$ surface.

Now you can think about all these fluxes through all the cycles.
Insteadof just $X_1$ and $X_2$ which parametrized 2 of hte cycles,
we're going to get $g$ copies of that kind of situation that we found,
and we're going to find that the dimension of the Hilbert space on a genus $g$
surafce ends up being
\begin{align}
    \dim \mathcal{H}\left( \Sigma_g \right) = 
    |k|^g
\end{align}
which I'll ask you to prove in your homeowrk.

There's one more generalization I'll put in your homeowrk.
You could instead consider multiple $U(1)$ gaue fields,
so you consdier $U(1)^n$ CS theory,
in which case the Lagrangian is
\begin{align}
    \mathcal{L}
    &=
    \frac{1}{4\pi} K_{IJ}
    a_\mu^{I} \partial_\nu a_\lambda^J \epsilon^{\mu\nu\lambda}
\end{align}
where $K$ is an integer symmetric matrix.
And then
\begin{align}
    \dim\mathcal{H}\left( \Sigma_g \right) = 
    |\det K|^g
\end{align}

Looking back.
So the Hailtonian is zero and we have a finite number of states in total.
There are very few degrees of freedom at all.
We could have instead considred the CS-Maxewll theory.
if you do CS-Maxwel theory,
the Hamiltonian is not zero,
and phase space will be non-compact.
And so what you'll find there is that you'll have $k^g$
ground states degeneracy,
and you'll have phase space that is noncompact,
and the photon becomes massive,
soo you havea finite gap,
but you hvae all these topologically degenerate groud states,
which are zero modes of the gauge field.
Physcally, this $a\, da$ term is only the leading term.
In the cnotex of the faracitoal quantum Hall effect,
there are gapped non-trivail excitations,
like the magneto-roton,
and that's what these excitations describe.
They're all gapped,
becaseu it's a gaped liquid,
they can e modelled by these higher order terms in $a$.
\begin{align}
    \mathcal{L} &=
    -\frac{k}{4\pi} a\, da
    +
    \frac{1}{4g^2} f^2
    + \cdots
\end{align}

When we use CS theory to describe fracitonal quantum Hall effect,
the meaning of these is the density of particles.
\begin{align}
    j^{\mu}
    &=
    \frac{1}{2\pi} \epsilon^{\mu\nu\lambda}
    \partial_\nu a_\lambda
\end{align}
So these are just hydrodynamic modes in a gapped liquid.
In Maxwell-Chern-Simons,
I don't remember the formula,
but it kas to depend on $k$.
If you take $g$ to infinitey,
all these states go away and you'rel eft with pure CS theory.
You'll have to do calcuations yourself to get the gap.

\begin{question}
In thep icutero f illing Landau levels,
what do these $k$ states correspond to?
\end{question}
You're getting two things confused
becauseI ddn't empphasises hte physcal conneiton so much.
We have 2 physical situations where CS thoery turns up.
\begin{enumerate}
 \item background $U(1)$ gauge field $A$.
     A fixed static background.
     And the effectie action for this gauge field for the $\nu=k$ integer
     quantum Hall effect,
     or if you like Chern number $k$,
     \begin{align}
         S_{eff}[A] &=
         \frac{k}{4\pi}
         \int A\, dA
     \end{align}
     The TQFT is invertible,
     with some symmetry $G$ for the gauge field.
 \item Dynamical U(1) guage field (emergent),
     the action is the same,
     but this is no longer fixed and we have to integrate over it
     \begin{align}
         S_{eff}[a] =
         \frac{k}{4\pi} \int a\, da
     \end{align}
     and the path integral is
     \begin{align}
         Z\left( M^3 \right)
         =
         \int \mathcal{D} a\,
         e^{iS[a]}
     \end{align}
     and now it's a non-inveritble TQFt.
     Here we gauge the theory,
     which si just a redundancy in the description,
     and we get a non-invertible theory.
     You could couple this to the backgorund $U(1)$ iwwth
     \begin{align}
         S_{eff}[a] =
         \frac{k}{4\pi} \int a\, da
         +
         \frac{1}{2\pi} A\, da
     \end{align}
     And you have
     \begin{align}
         Z\left( M^3 \right)
         =
         \int \mathcal{D} a\,
         e^{iS[a,A]}
     \end{align}
\end{enumerate}

\begin{question}
    Currents?
\end{question}
You're confusing 2 currents.
This $j^\mu$ is the current of flux.
If you have flux of the gauge field that moves aorund,
this is what it is.
This is hte instanton current.
There are 2 currents.
There si the current of particles charged under $a$ that is conserved wiht
$\partial_\mu j^\mu=0$,
and then there is the monopole instanton conserved current,
which is what I'm talking about.

\section{Wilson loops}
Okay,
so we've quantized the theory at least on this manifold
and we've understood the states.
But now it's useful to talk about Wilson loops.

For any curve $\gamma$,
we can write down a Wilson loop
\begin{align}
    W_\gamma &=
    e^{i\oint_\gamma a\cdot dl}
\end{align}
and let's consider $M63 = T^2\times \mathbb{R}$.
Consider the Wilson loops on the directions on the torus.
\begin{align}
    W_1 &=
    e^{i\oint_{\gamma_1} a\cdot dl} = e^{i 2\pi X_1}\\
    W_2 &=
    e^{i\oint_{\gamma_2} a\cdot dl} = e^{i 2\pi X_2}
\end{align}
And remember when we quantize,
we have
\begin{align}
    \left[ x_1, P_{x_1} \right] = i
\end{align}
we learn that
\begin{align}
    \left[ X_1, X_3 \right] = \frac{i}{2\pi k}
\end{align}
Here capital $X$ is the zero mode and $x$ is the coordinate.
This means the Wilson loops in the two directions acquire a non-trivial value.
\begin{align}
    W_1 W_2 &=
    W_2 W_1 e^{-4\pi^2\left[ X_1, X_2 \right]}\\
    &= W_2 W_1 e^{2\pi i/k}
\end{align}
Since $H=0$, $[H,W_1]=0$.
And so the ground state of the system needs to form a representation of this
algebra,
and every representation of this algebra needs to have an integer multiple of
$k$ and so the minimal representation is $k$-dimensional.
So we must have at least $k$ dimensional Hilbert space.

And we can pick a basis for these Wilson loops to act.
And you can check that this action with the states is compatible with the
algebra.
\begin{align}
    W_2\ket{n} &=
    \ket{n + 1 \mod k}\\
    W_1 \ket{n} &=
    e^{2\pi i/k}\ket{n}
\end{align}
What this means is that these particles have fractional statistics.
You can think of these Wilson loops describing particles of charge 1 moving
around the loop $\gamma$.
If I create a particle and its antiparticle,
moving them around a loop and annihilate.
That they don't commute with each other in different loops implies that these
particles carry fractional statistics.

\begin{question}
    Does this group have a name?
\end{question}
Yes. I think this group is called the Heisenberg group.
