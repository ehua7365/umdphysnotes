\section{Matrix Product States}
MPS closed chain.
\begin{align}
    \ket{\psi} &=
    \sum_{i_1,\ldots,i_N} \Tr\left( 
    A_{i_1}\cdots A_{i_N}
    \right)
    \ket{i_1,\ldots,i_N}
\end{align}
where $A_i$ are $D\times D$ matrices, and $i=1,\ldots, d$ where
$d$ is the dimension of the site Hilbert space.
There are $dND^2$ parameters here to describe a quantum state.

Gapped ground states of local Hamiltonians satisfy the area law for entanglement
entropy.

Can approximate gapped ground states to accuracy $1/\poly(N)$
with MPS with $D$ sublinear in $N$.

A MPS is injective if there exists finite $L_0$ such that
\begin{align}
    \tilde{A}_I &=
    A_{i_1} A_{i_2} \cdots A_{i_N}
\end{align}
spans the whole space of $D\times D$ matrices.

The question is if these matrices span the entire space of $D\times D$
matrices.

The reason injectivity is important is that you can show that an injective MPS
has the following properties.
\begin{enumerate}
    \item Finite correlation length.
    \item There are unique gapped ground states of a frustration-free parent
        Hamiltonian.
\end{enumerate}

There is a book by Xiao-Gang Wen, Xie Chen, Bei Zeng and one other author which
has a nice discussion of MPS and in particular the relation between MPS and SPT
phases.

There's another formulation called PEPS,
projected entangled pair states.

Consider a chain of maximally entangled pairs which are then projected.
So you have on each site,
imagine you have two kinds of spins.
And then you put the spins on nearby states into singlets,
and then you do a projection.

So here, 
this singlet bond,
what that really means,
you can think of it as a maximally entangled state between the Hilbert space on
the two sides.
\begin{align}
    \frac{1}{\sqrt{D}} \sum_{\alpha=1}^{D} \ket{\alpha}\ket{\alpha}
\end{align}
where $D$ is some internal bond dimension.
The idea is that each of these are some virtual spin,
and $D$ would be the dimension of the virtual spin.

And then the projection projects from two virtual spins to a physical degree of
freedom of dimension $D$.

So this projection,
if you write it in the following way
\begin{align}
    P &= \sum_{i,\alpha,\beta} A_{i,\alpha\beta} \ket{i}\bra{\alpha\beta}
\end{align}
Then this state is actually equivalent to the MPS $\ket{\psi}$.

So the MPS is exactly equivalent to this PEPS when the projection is written
like above.

Something that happened in 2010,
that led to people thinking all 1D phases are classified by $H^2(G, U(1))$,
is that if you take the bond dimension,
and what can be analysed by constant depth circuits,
is the following.

One can show that any MPS with fixed bond dimension can be disentangled
to exponential accuracy by local constant depth circuit.

The fact you can do this is evidence for no bosonic topological order,
meaning no topologically ordered phases of bosons without any symmetry.

Disentangle means toke to the direct product state.

It doesn't prove it of course,
because the bond dimension is assumed to be fixed,
but in general you should consider the bond dimension to grow with system size.

Last time in the paper,
they show the bond dimension is sublinear in $N$,
but not necessarily constant in $N$.
This discovery of disentanglement with fixed bond dimension doesn't prove there
is no topological order in 1D but it does strongly suggest there is no bosonic
topological order in 1D.

Then you could add symmetry in the circuit,
and let the circuit be symmetric,
and then you can show from there that you basically get this PEPS form.
That is,
you can show that any MPS with a fixed bond dimension and symmetry $G$
can be converted into an entangled pair form via a symmetric constant depth
circuit.

Furthermore, each virtual spin is in projective representation of $G$
characterized by this cocycle
$[w] \in H^2(G,U(1))$.

This again,
is evidence that the classification of gapped phases should be
$H^2(G,U(1))$.
The caveat is that we're considering MPS,
and constant depth local circuits that disentangle,
or take any generic MPS to this fixed point form.

I'm not going to go through the exact construction,
but this book has s nice account of it.

I want to describe a different model for thinking about SPT states in terms of
path integrals,
state sums and wave functions that can generalize to every dimension.
This whole discussion about MPS is limited to 1D.
While it's interesting and illuminated,
I'm not going to dwell more on it because it's so specific to 1D.
This is the overview of what happened 10 years ago in 1D that led people to go
further and classify higher dimensions.

\begin{question}
    Experimental tests of AKLT?
\end{question}
The spin-1 Heisenberg chain is gapped phase of matter with dangling gapped edge
modes,
and there are experiments.
The Haldane chain is half the reason why Haldane won the Nobel prize,
even though he didn't realize it's an SPT state at the time.

\begin{question}
    Is it because of ground state degeneracy?
\end{question}
Every 1D bosonic topological phase is a trivial phase,
if you forget about symmetry.
For fermions,
it's not the case because we have Majorana chains.
Even though it has no edge modes on a ring,
it's still topological.

It's not exactly because of ground state degeneracy,
because the Majorana chain does not have ground state degeneracy on a closed
chain either.

\begin{question}
    Why is the second statement evidence?
\end{question}
Every sate,
you can run it through some constant depth symmetric circuit,
and take it not a form like this,
where every one is a projective representation,
with a dangling edge state.
You could run some RG procedure that takes a constant number of steps,
but takes you to this idealized fixed-point form to good accuracy.

\begin{qeustion}
    Is the second statement still true for $d=1$.
\end{qeustion}
You can actually get from PEPS to the trivial product state by disentangling
each of these guys,
but he only way to do that is break the symmetry $G$.
But to fully disentangle,
you need that one extra step to break the symmetry.

I encourage to read the relevant chapter in the book by X.G. Wen.
There's also a book by Ignacio Cirac.

\section{Group Cohomology Model (Dijkgraaf-Witten theory)}
I want to turn to a model for SPT state,
first in 1D,
but this model actually generalized to higher dimensions as well.

The idea is that we're going to construct a TQFT that gives SPT states
by constructing the path integral.
Construct a path integral fora TQFT
\begin{align}
    Z(M^2, A)
\end{align}
We have a symmetry $G$, so there is a gauge field $A$ for that $G$ symmetry.

The first step is to triangulate spacetime.

Then the next step is define a branching structure.
A branching structure is a local ordering of the vertices.
Another way of saying what it is is to make all edges directed in a way such
that there are no closed loops.
That's all a branching structure is.

For example, take some triangle,
with vertices labelled 0, 1, 2.
Then arrows pointing from lower to higher number vertices on edge,
and you see there're no loops.

Then we introduce an orientation.
You can reverse the arrows,
and you notice you can't rotate one to make it look like the other,
because if you follow two of the arrow,
your thumb points down,
but it points up on the other orientation.
You call one the $+$ orientation and you call the other the $-$ orientation.

The next thing to do is to assign group elements to vertices.

It effectively introduces a gauge field into the problem,
because if I draw a triangle with
$g_0$, $g_1$ and $g_2$ on vertices,
the links between them will be $g_0^{-1}g_1$ from $g_0$ to $g_1$,
$g_{0}^{-1}g_2$ from $g_0$ to $g_2$ and
$g_1^{-1}g_2$ from $g_1$ to $g_2$.

[picture]

This defines a special kind of gauge field,
it's a \emph{flat} gauge field.
That is,
there is no net flux through a plaquette.

For example,
the flux through one triangle is
$A_{01}A_{12}A_{20}=1$
and in fact the product of any loop is 1 with
\begin{align}
    \prod_{\textrm{loop}} A = 1
\end{align}
so in fact $A$ is a \emph{trivial} flat gauge.

Then for each 2-simplex (triangle),
we're going to associate a phase factor,
which is an amplitude,
that depends on the group elements.

To each 2-simplex $\Delta^2$, define
\begin{align}
    \left[ \nu_2\left( g_0, g_1, g_2 \right) \right]^{S\left( \Delta^2 \right)}
\end{align}
where $\nu_2 \in U(1)$.
I assume the labels are labelled $g_0,g_1,g_2$.
$S$ is the orientation of $\Delta^2$.

\begin{question}
    Is this an invertible TQFT?
\end{question}
This will be an invertible TQFT,
and the fact that this is just a $U(1)$ phase is important why it's invertible.
You can generalize it,
but it's significantly more complex to make $U(1)$ not just a phase.

The path integral that we're going to define is defined as follows.
\begin{align}
    Z &=
    \frac{1}{|G|^{N_V}}
    \sum_\left\{ {g_i \right\}}
    \prod_{\Delta_2 \ni (i, j, k)}
    \left[ 
    \nu_2 \left( g_i, g_j, j_k \right)^{S(\Delta_2)}
    \right]
\end{align}
we're taking the product over all 2-simplices $\Delta_2$
each of which have vertices $i,j,k$ in order.
And I'm going to take the product of all these phases $\nu_2$
and I'm going to raise it to the orientation $S\left( \Delta_2 \right)$.
And then I'm going to sum over all possible choices of vertices $\left\{ g_i
\right\}$
and then I'm going to average by dividing by $|G|^{N_V}$
where $N_V$ is the number of vertices and $|G|$ is the number of elements in
$G$.

We'll come to this,
but if you're a mathematician and see this,
you might think we're crazy,
because you'll see the sum is completely useless for defining the path integral,
but the sum is here for physics.
What you find here is that every term is here is the same.
I haven't told you what $\nu_2$ are.
It turns out every term in the term is the same.

We're just summing over all possible labellings.

\begin{question}
    $G$ survives permutations?
\end{question}
We have some crazy triangulation,
and I'm just summing over every possible value of $G$.
Every vertex has a group element attached to it,
and each vertex has a different group element.

\begin{question}
    Are there as many group elements as vertices?
\end{question}
There are $N_V$ group elements,
which is different from $|G|$.
For example $G=\mathbb{Z}_2$,
then there are $2^{N_V}$.


\begin{question}
    How does it connect to the system we're trying to study?
\end{question}
Maybe you should wait.
If you remember in TQFT,
you can define a path integral for spacetime.
If your spacetime has a boundary,
you have a state on the boundary.
This path integral can give a wave function state on the boundary.


Any boundary is going to be a circle,
or a bunch of disconnected circles,
so we can get a wave function on circles that describes SPTs.
And from those circles,
we deduce an exactly solvable Hamiltonian as well.

This gives a topologically invariant path integral,
form which we extract topological invariants.

\begin{question}
    Is branching structure the same as consistent orientation?
\end{question}
No, it's more than that.
It's even more than a specific set of orientations.
It's a specific set of ordering of vertices.

\begin{question}
    Should neighbouring vertices have the same ordering?
\end{question}
Not necessarily.

\begin{question}
    We take all gauge-equivalent into one term?
\end{question}
Yes, this is going to wind up being an average over gauge transformations.

It's gauge equivalent to no $G$,
we haven't defined $\nu_2$ yet.
We'll be able to relax this product of $A=1$ condition,
so we can introduce twists and things wrap around,
and it won't be gauge-equivalent to nothing.

It's also useful to think of $Z$ in the following way,
as a sum over all $g$s normalized
over the exponential of some topological action.
\begin{align}
    Z &=
    \frac{1}{|G|^{N_V}}
    \sum_\left\{ {g_i \right\}}
    e^{i S_{\mathrm{top}}\left( \left\{ g_i \right\} \right)}
\end{align}
I'm being a bit sloppy here,
when I say $A$,
I really mean equivalence classes of $[A]$,
by gauge equivalence.
But if we know $ZA$ is gauge invariant,
I don't really need to write the square brackets.
\begin{align}
    Z = Z\left( M^2 , [A] \right)
\end{align}

This has a global symmetry,
and what I mean is that the amplitude should be the same as if multiplied by
some global $g$.
Every term in the sum is explicitly the same if I change all $g_j$'s at once by
some global $g$.
\begin{align}
    e^{iS\left( \left\{ g_i \right\} \right)}
    =
    e^{iS\left( \left\{ gg_i \right\} \right)}
\end{align}
which also means
\begin{align}
    \nu_2\left( gg_0, gg_1, gg_2 \right) &=
    \nu_2\left( g_0, g_1, g_2 \right)
\end{align}
We're assuming $g$ is unitary,
but if it's anti-unitary,
you just introduce a complex conjugate in the above equation.

What this allows is to do is define a 2-cochain.
The reason it defines a 2-cochain is because I can define the usual 2-cochain,
which I introduced last lecture,
which is
\begin{align}
    \omega_2(g_1, g_2) &=
    \nu_2 (1, g_1, g_1 g_2)
\end{align}
Furthermore,
I can multiply every element by $g_0^{-1}$ so
\begin{align}
    \nu_2 (g_0, g_1, g_2) &=
    \nu_2 \left( 
    1, g_0^{-1},g_1, g_{0}^{-1} g_1 g_1^{-1} g_2
    \right)\\
    &= \omega_2\left( g_0^{-1} g_1, g_1^{-1} g_2 \right)
\end{align}
We thought of 2-cochains as something that takes in 2 group elements and splits
out a phase factor.
But here,
we can think of it as taking 3 group elements and spits out a factor,
but it has some symmetry in the inputs.
$\nu_2$ is called a homogeneous 2-cochain,
whereas $\omega_2$ is an inhomogeneous 2-cochain.

\begin{question}
    What's inhomogeneous about it?
\end{question}
If you think about it,

\begin{question}
    What is symmetric?
\end{question}
I want the action to be symmetric.
Usually when you talk about a symmetric system,
I say the path integral is a sum over all field configurations.
But I say my system is symmetric if my action has the symmetry.

\begin{question}
    Can't we have each equality up to a phase
    so the phases cancel instead?
\end{question}
I haven't thought about that too more carefully.
I'm saying that this implies this,
but not the other way around.
You could relax things and consider it more general,
but I'm not sure how to do that in a way that's compatible with locality.
At the very least,
it gives us a way of getting a symmetric path integral.

\begin{question}
    Every group element can be mapped to another group element by another group
    element.
    Why are not all the vertices not connected to each other.
    You can always form a side that connects $g_1$ to $g_3$.
    What is the point of triangulation?
    All group elements are connected to each other.
\end{question}
I don't understand the question.
That's the graph of the group.
You can think if this as defining a gauge field on a manifold.

\begin{question}
    We're assuming $\nu_2$ to have a global $g$ symmetry,
    not a local $g$ symmetry.
\end{question}
We have a global $g$ symmetry,
which means we want every amplitude independent of $g$,
but there's also a gauge symmetry,
in that I can do a gauge transformation $A$ and not change the path integral.
There's something much stronger that's happening here in that there's also a
local gauge symmetry.


\section{Topological invariance of the path integral}
Now we get to the topological invariance of $Z$.
So far, 
we had to triangulate spacetime and add a branching structure.
That's just geometry.
But for topological invariance,
we need it to be independent of geometry and choice of branching structure.

The point is,
that we want $Z$ to be independent of triangulation and branching structure.
that's what we want to demand,
but actually we want to demand something even stringer,
that is the action $e^{iS}$ is topological,
meaning independent of triangulation.

And the way that you can require that something be independent of triangulation,
is given any triangulation,
you can always get to another triangulation by a series of moves,
and these moves are called \emph{Pacher moves}.

Suppose I have two triangles like this.
One Pacher move is to have a rhombus with a diagonal,
and change the diagonal to the other diagonal. This is a 2-2 move.

There is a 1-3 move which inserts a vertex on a triangle,
and draws rays from the new vertex.

It's just pasting a simplex of one higher dimension.
In 2D you have a 2-simplex that is a triangle.
But in one-higher dimension,
you have a tetrahedron.
Think of pasting a tetrahedron onto a triangle and flattening it.

So one way of thinking about Pachner moves is pasting $D+1$-simplices onto your
$D$ simplices and flattening it out.


But we actually need branched Pachner moves,
which also deals with the ordering.

I need some extra conditions.
For example,
in the 2-2 Pachner move,
there would be
\begin{align}
    \nu_2 (g_0, g_1, g_2)
    \nu_2 (g_0, g_2, g_3)
    =
    \nu_2 (g_0, g_1, g_3)
    \nu_2 (g_1, g_2, g_3)
\end{align}
[picutre]

To be careful,
you should also check the orientation and make sure you put the right complex
conjugation.
For the 1-3 Pachner move,
you would have
\begin{align}
    \nu_2(g_0, g_1, g_2) =
    \nu_2(g_0, g_1, g_3)
    \nu_2(g_1, g_2, g_3)
    \nu_2(g_0, g_2, g_3)
\end{align}

But secretly,
they are the same equation,
in fact they are just the 2-cocycle equation that
\begin{align}
    d\nu_2 = 1
\end{align}

To make the topological action re-triangulation invariant,
we require that $\nu_2$ to be a 2-cocycle.

And finally,
let's look at the invariant under 2-coboundaries

\subsection{Invariance under coboundaries}
Suppose that we take
\begin{align}
    \nu_2 (g_0, g_1, g_2)
    \to
    \nu_2 \cdot 
    b_1 (g_0, g_1)
    b_1 (g_1, g_2)
    \left[ b_1 (g_0, g_2) \right]^*
\end{align}
and I'm going to require
\begin{align}
    b_1 (g_1, g_2) &= b_1 (gg_1, gg_2)
\end{align}
And $e^{iS}$ is invariant because each edge (1-simplex) appears in exactly two
2-simplices with opposite induced orientation.

The reason it's invariant, is every single 1-simplex appears with 2 neighbouring
triangles.
For example, if I have a triangulation,
every 1-simplex has 2 neighbouring triangles.
And every 1-simplex has opposite orientation relative to each triangle.
That means that if one triangle changes by this factor $b_1(g_1,g_2)$,
the neighbour is going to change by a similar factor,
but it's going to appear complex conjugated in the neighbour so they're all
going to cancel out.

In terms of inhomogeneous 2-cocycles,
applying $g_0^{-1}$ to each term,
we get
\begin{align}
    \nu_2(1, g_0^{-1} g_1, g_{0}^{-1} g_2)
    &\to
    \nu_2 \cdot
    \frac{b_1(1, g_{01}) b_1(1, g_{12})}{b_1(1, g_{02})}
\end{align}
and
\begin{align}
    \omega_2(g_{01}, g_{12}) \to
    \omega_2
    \frac{\epsilon_1(g_{01}) \epsilon_1_1\left( g_{12} \right)}{\epsilon_1\left(
    g_{02} \right)}
\end{align}

That means, if we change our 2-coboundary by 2-cocycles,
the topological boundary doesn't change.
That is,
distinct $G$-symmetric path integrals are classified by
\begin{align}
    [\nu_2] \in
    H^2\left( G, U(1) \right)
    = \frac{\mathbb{Z}^2}{B^2}
\end{align}
Actually,
the last step is a bit of a jump.

Let me give more details.

On closed manifolds,
what we defined so far always gives us one with
$Z(M^2)=1$.
That's because what we have is a trivial $G$-bundle,
meaning that
$\prod_{\textrm{loop}}A=1$.

To get non-trivial results we non-trivial flat bundles with
$\prod_{\textrm{loop}}A \ne 1$.
We want it to be flat but we want holomony over non-contractible loops.

One way of doing this is this construction we've defined,
is we can cut our manifold along whatever loop we're interested with having
holomony,
then inserting some $G$-twist when we glue them back together.

Let me draw a picture.
Suppose w have a cylinder,
and we want to put a $h$ branch cut along the cylinder axis.
So then this loop at the end of the cylinder is $\gamma$,
with $\prod_\gamma A = h$.

Then if you roll out the cylinder flat,
you can draw a triangulation like this.

[picture of unrolled cylinder sheet]

The point is that if we have $g_1,g_2,\ldots$ on the bottom edge of the cut,
which we identify with the vertices on the top edge of the cut to get a
cylinder.
To do the twist,
just make the top edge vertices labelled $hg_1, hg_2, \ldots$.

For a closed $M^2$
non-trivial flat bundle,
\begin{align}
    |Z(M^2, A)| = 1
\end{align}
but the phase is going to be non-trivial,
with $Z(M^2, A)$ being a gauge-invariant polynomial in $\omega_2$.
For every element of $H^2$,
we're going to get a path integral that will spit out a $U(1)$ phase and if we
look at al possible closed manifolds and all possible closed bundles,
we find that all elements of $H^2(G, U(1))$ corresponds to a distinct path
integral $Z(M^2, A)$.

So there is a one-to-one correspondence between this group and these TQFTs by
picking appropriate fluxes along non-contractible cycles.
\begin{align}
    H^2(G, U(1)) \leftrightarrow Z(M^2, A)
\end{align}
I didn't prove that $Z(M^2)=1$ if it's flat and trivial,
because I didn't prove that the sum over $G$ is actually invariant yet.

\begin{question}
    What is the definition of $\prod A$?
\end{question}
If you have a torus.
If you take a product of lops on a contractible bundle,
you get 1.
But you get a holomony over a non-contractible loop?

\begin{question}
    What if we generalize to cellulations?
\end{question}
well this $\nu_2(g_0, g_1, g_2)$ only makes sense over triangles because it has
3 inputs,
but you could consider cellulations,
but the framework will be a different,
but ultimately you get the same answer.
