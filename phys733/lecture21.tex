\section{(2+1) dimensional SPT path integral}
\begin{align}
    H_\rho^n(G, M)
    &=
    \frac{Z_\rho^n(G, M)}{B_\rho^n(G, M)}\\
    &=
    \frac{\ker d_n}{\im d_{n-1}}
\end{align}
where
\begin{align}
    d_n:
    C_\rho^n(G, M)
    \to
    C_\rho^{n+1}(G, M)
\end{align}

You can have $+$ orientation tetrahedrons.
You can also have $-$ orientation tetrahedrons.

Let $\nu_3$ be homogeneous and $\omega_3$
be inhomogeneous.
Then the path integral is
\begin{align}
    Z &=
    \frac{1}{|G|^{N_V}}
    \sum_\left\{ {g_i \right\}}
    \prod_{\Delta^3\ni (i,j,k,l)}
    \left[ 
    \nu_3\left( g_i, g_j, g_k, g_l \right)
    \right]^{S\left( \Delta^3 \right)}
\end{align}
Note that
\begin{align}
    g\cdot \nu_3\left( g_1,\ldots, g_4 \right)
    &=
    \left[ 
    \nu_3\left( gg_1, gg_2, gg_3, gg_4 \right)
    \right]^{\sigma(g)}
\end{align}
due to $g$ -action
and then due to $G$-symmetry,
\begin{align}
    g\cdot \nu_3\left( g_1,\ldots, g_4 \right)
    &=
    \left[ 
    \nu_3\left( gg_1, gg_2, gg_3, gg_4 \right)
    \right]^{\sigma(g)}
    =
    \nu_3\left( g_1, \ldots, g_4 \right)
\end{align}
Then
\begin{align}
    \nu_3\left( g_1,\ldots, g_4 \right)
    &=
    \left[ \nu_3\left( 1, g_{12}, g_{13}, g_{14} \right) \right]^{\sigma\left(
    g_1 \right)}
\end{align}
recalling $g_{ij} = g_i^{-1} g_j$.
And so
\begin{align}
    \omega_3\left( g,h,k \right)
    &=
    \nu_3\left( 
    1,g,gh,ghk
    \right)
\end{align}
Thus we have the path integral
\begin{align}
    Z\left( M^3, A \right)
    &=
    \prod_{\Delta^3\ni \left( i,j,k,l \right)}
    \left[
    \omega_3\left( A_{ij}, A_{jk}, A_{kl} \right)
    \right]^{S\left( \Delta^3 \right)}\\
    &=
    \frac{1}{|G|^{N_V}}
    \sum_\left\{ {g_i \right\}}
    \prod_{\Delta^3\ni (i,j,k,l)}
    \left[
    \omega_3\left( g_i^{-1}A_{ij} g_{j},
    g_j^{-1}A_{jk}g_k,
    g_k^{-1}A_{kl}g_l\right)
    \right]^{S\left( \Delta^3 \right)}
\end{align}

We can draw out the Pachner motes in 3D.

There is the 1-4 Pachner move which inserts a vertex inside a tetrahedron to
produce 4 tetrahedrons.

There is also the 2-3 Pacher move,
which takes two tetrahedrons sharing a face,
to produce 3 tetrahedra.
If the vertices are labelled 0123 and 01234
no the two original tetrahedra,
then new tetrahedra have vertices
01234, 1234, 0234.

We want our path integral to be invariant under Pacher moves.
In mathematics,
there is the famous \emph{pentagon equation}
\begin{align}
    \nu \nu = \nu\nu\nu
\end{align}
which implies
\begin{align}
    d\omega_3 = 1
\end{align}
so that
$\nu_3$ and $\omega_3$ are 3-cocycles.

If we change
$\omega_3 \to \omega_3\, d\mu_2$,
then this cancels out of the path integral.
Every $\mu_2$ enters the product twice with opposite orientation,
so this cancels out the $\mu_2$.

Ultimately,
distinct (2+1)D topological actions will be classified by
\begin{align}
    H_\rho^3\left( G, U(1) \right)
\end{align}
The $\rho$ is there so that every time the symmetry $G$ is anti-unitary,
it's the complex conjugate.
The $U(1)$ can be replaced by another group,
but it has to be a unitary group.
$\rho$ is just a map $\rho: G\times U(1)\to U(1)$ such that
\begin{align}
    \rho_g\left( e^{i\theta} \right)
    &=
    e^{i\theta'}
\end{align}

We can repeat this whole thing in any general dimensions,
so what we get is that in $(d+1)$ dimensions,
we have distinct actions which are classified by
$H^{d+1}\left( G, U(1) \right)$.
This gives a path integral
\begin{align}
    Z_{\left[ \omega_{d+1} \right]}
    \left( 
    M^{d+1},
    A
    \right)
\end{align}
for flat $G$ gauge field $A$
and
\begin{align}
    \left[ \omega_{d+1} \right]
    \in H^{d+1}\left( G, U(1) \right)
\end{align}
is just a cohomology class.
And this gives us invertible TQFT with background flat $G$ gauge field,
because $|Z(M,A)|=1$.

So just like in $1+1$ dimensions,
we can write down the wave function,
we can write down the wave function in $d$ dimensions,
and show there is no intrinsic order in SPT,
and this gives a general solvable model with SPT in general dimensions.

This gives us a whole class of distinct SPTs classified by $H^{d+1}$.
This is not the full classification of SPTs actually.
There is a more general classification in terms of cobordisms.
In the dimensions we're interested in,
this is often the full classification,
but often there are states beyond cohomology,
there a  whole class of SPT phases called ``beyond cohomology''.
The often appear in anti-unitary symmetries in higher dimensions.

So far we have invertible TQFT.

Let me just say a few more things in terms of how much this classifies.
In (1+1)D,
this $H^2\left( G, U(1) \right)$ is the full classification of invertible
bosonic phases.

In (2+1)D,
this $H^3(G,U(1))$ is a full classification of bosonic SPTs,
but there are invertible phases that are not SPTs.
These occur when you have chiral central charge,
you can have bosonic states with chiral edge mods.
Strictly speaking,
if you break the symmetry,
they are not adiabatically connected to the ground state.

And then in $(3+1)D$,
this $H^4\left( G, U(1) \right)$
is a full classification of invertible bosonic invertible theories if $G$ is
unitary.
But it's partial classification if $G$ has anti-unitary elements.
This leads us to so-called \emph{beyond cohomology} SPTs.

There is a theory which is expected to capture everything,
which is the cobordism theory,
which I mentioned earlier on in the course.
For now I'll leave it at this.


\begin{question}
    In 1+1 dimensions,
    there are bosonic phases?
\end{question}
In 1+1 D,
even with no symmetry there is the Majorana chain.

\begin{question}
    In 1D you can always bosonize?
\end{question}
There is a real difference between fermions and bosons.
When you bosonize,
something that is local on one side is no longer local on the other side.
It's precisely because the notion of locality changes.

\begin{question}
    The cobordism is larger than the group cohomology.
    In 6D,
    they become the same cobordism phase.
    Different phases have different topological invariants.
    How do they collect into the same/
\end{question}
I don't know.
The argument why these different phases are different,
in 1D we say the constant depth circuits which trivialize it,
but I don't know what goes wrong in higher dimensions.

\begin{question}
    Example of invertible phases in (2+1)D?
\end{question}
The basic one is called the $E_8$ phase,
which has a chiral central charge of 8.
It does not need any symmetry to exist,
but it has chiral central charge of 8.
One way of thinking about it,
is stack 8 copies of the Kitaev honeycomb model,
condense a bunch of vortices,
and you get this.
Everything is made out of these $C_{-}=8$.

\begin{question}
    Is this related to the $E_8$ the Lie group?
\end{question}
You can take $E_8$ level 1 Chern-Simon theory and you get this.
Or you can get the Cartan matrix of the Lie algebra $E_8$.
It's intriguing this exceptional group $E_8$ comes up.

\begin{question}
    ??
\end{question}
We wrote this class of modes,
but there's no reason to expect this is full classification.
This is not even proven,
I would say,
in every dimensions.
This is just the current dimension.
The current understanding is,
there is a conjecture,
which is that bosonic SPTs in $(d+1)$-dimensions
are classified these cobordism groups
\begin{align}
    \Hom\left(\Omega_{d+1}\left( BG \right), U(1)
    \right)
\end{align}
which means the manifolds are equipped with a $G$-gauge field as well.
This is the Pontryagin dual.
Nothing is really proven in terms of full classification.
To prove rigorously topological phases are fully classified,
means you have to understand all topological phases,
which is really beyond our understanding.
This is based on what we understand so far and what seems to make sense.


This sketch, what is not captured?
In 2+1D the ones not captured are he ones with chiral edge models.
Its' because the edge gives you a Hamiltonian as a sum of commuting projectors,
which you can argue will always gap the edge.
This class of models cannot exhibit fully robust chiral edge modes.
The reason some anti-unitary symmetries are outside the group cohomology models
is a deeper issue.
It ultimately has to do with the fact that group cohomologies characterise the
ability to couple to background gauge fields,
but there's something beyond background gauge fields,
and there's no reason to think $G$-gauge theory should be enough on its own
anyway.

\begin{question}
    Edge modes in 1+1D?
\end{question}
Those are captured,
but hose are gappable.
If I break the symmetry I can get rid of them.
These are ones where even if you break all symmetries you cannot get rid of
them.

\begin{question}
    ??
\end{question}
These chiral central charge has nothing to do with symmetry,
it's like thermal Hall conductance.
It's true that anti-unitary symmetries,
you cannot gauge.
When you have a background $G$-gauge field,
you can promote it to a dynamical gauge field by summing over configurations.
You can't do that if talking about anti-unitary symmetries.

Let's talk a bit more about what the physical meaning of this $H^3$ is.
So far this $H^3$ we used to describe a class of models,
we saw in 1+1D,
we saw a projective action on the edge.
You guys asked if there is an analogy for higher dimensions.
There is.

\section{Physical origin of $H^3(G, U(1))$}
Let's just briefly recall that in $(1+1)$-D,
for the $H^2(G, U(1))$ what happens.
We considered a system on an open chain,
then applied a symmetry operation of $G$,
with representation $R_g$,
which localizes a s an action on the left $U_g^L$ and an action on the right end
point $U_g^R$,
and we get a projective phase we cannot get rid of and that is why we have this
second cohomology class.
\begin{align}
    U_g^L U_h^2
    =
    \omega_2(g, h)
    U_{gh}^L
\end{align}

Now consider $(2+1)$D. (Else-Nayak 2014)
The idea is to apply some symmetry operation and see hat happens on the edge.
Say your system is on a disk,
you apply a symmetry operation,
and see what it looks like on the edge.
We have some restriction of the operator on the edge,
then we cut that operator open,
so it looks like an operator on a segment.
Then we restrict $U_g^{\left( edge \right)}$
on a segment,
and we reduce it down to a point,
and we have a projective phase that turns up.

That's the rough idea.
To get this in more detail,
I want to say more about the idea 
if we have a local unitary,
a unitary operator that arises as a constant depth local unitary circuit,
there's an idea of restricting that unitary to a subregion.

Consider a unitary $U$ acting on a space $C$.
$U$ is a local unitary,
meaning that it's a constant depth local quantum circuit.

Another local unitary $U_M$ is going to be the restriction of $U$ to a
submanifold $M$
if it acts exactly the same as $U$ on the subregion $M$ away from the
boundaries.
That is,
it acts the same as $U$ in the interior of $M$ far way from $\partial M$.
There are a few things you can say about such a restriction.

Firstly,
the restriction always exists.
Suppose we have a unitary that acts as some circuit on a bunch of qubits.
So something like this [picture].

Let's say this is $U$ acting on $C$.
I can always define a restriction
by cutting this unitary along some lines here,
and looking at the unitary only acting on the middle.
Obviously there's some ambiguity on the edge,
but in the bulk it's clear I can define a unitary that acts in the same way as
$U$.

It's important that $U$ is a constant depth circuit,
so there is a kind of light cone,
nothing on this side is going to cross to this side beyond a certain size.
Of course near the boundaries of $M$ there are ambiguities.

The restriction is obviously ambiguous near the boundaries,
and they are ambiguous up to some operators.

\begin{question}
    We must have some conditions on the size of the submanifold as well?
\end{question}
If it's too small compared to the depth of the circuit,
we can't do this.
We're assuming the region is large compared to the depth of the circuit.

The second thing is that this is defined modulo local unitaries near the
boundary $\partial M$.

This is the analogue of what we saw in $1+1$ dimensions.
We applied the symmetries,
acting like $U_g^L$ and $U_g^R$
where we could change the phase near the boundary,
and that's the ambiguity.
But here,
instead of a phase,
This is the higher-dimensional generalization of phase ambiguity.

Let's consider the symmetry operator
acting on the low energy subspace of the edge theory.
It's a unitary.
This state is gapped on the bulk,
but it's gapless on the boundary.
The states we are concerned with are where the edge is in some low energy
subspace,
so the excitations on the edge have a much lower energy than the bulk energy
gap.

Then we apply some symmetry operations.
Because the bulk is in the ground state and symmetric,
after applying $R_g$,
this is just applying $U_g^{edge}$ on the edge.


So let's assume $R_g R_h = R_{gh}$.
That's going to also be true for these edge operators.
If not,
there will be a projective representation no the edge,
which is in conflict that the ground state is unique and in the ground state.
For 1+1 D open chain,
the ground state is not symmetric,
because we have these edge states that can fuse to a singlet,
so we can have a symmetric ground state,
but we have a degeneracy when these things are far away from each other.
In higher dimensions,
the edge is connected,
even though there are gapless excitations on the edge,
it is still unique and symmetric.
\begin{align}
    U_g^{edge} U_h^{edge} u_{gh}^{edge}
\end{align}

If the edge preserves symmetry,
the ground state is unique,
then there is no projective representation for the edge.
Now we take this local unitary on the edge and we cut it open.

So suppose I take a cut.
\begin{align}
    U_g^{cut} U_h^{cut} = \Omega(g, h) U_{gh}^{cut}
\end{align}
where $\Omega(g, h)$
is a unitary operator with only support for the new end points $a$ and $b$.
For the bulk we do satisfy the group law,
but it's only at the edge boundary where we don't necessarily support the group
law.


Now we can consider associativity.
We considered
\begin{align}
    U_g^6cut U_h^{cut} U_k^{cut}
    =
    U_g^{cut}
    \Omega(h, k)
    U_{h,k}^{cut}\\
    &=
    ^g\Omega(h, k)\Omega(g, h, k) U_{ghk}^{cut}
\end{align} 
where
\begin{align}
    ^g\Omega(h, k)
    :=
    U_g^{cut} \Omega(h, k)
    \left( U_g^{cut} \right)^{-1}
\end{align}
so then
\begin{align}
    U_g^6cut U_h^{cut} U_k^{cut}
    &=
    \Omega(g, h)
    \Omega(gh, k)
    U_{ghk}^{cut}
\end{align}
And so we are led to this equation
\begin{align}
    \Omega(g, h) \Omega(gh, k)
    =
    ^g\Omega(h, k)
    \Omega(g, h, k)
\end{align}
But now we realise $\Omega$ is an operator with support on $a$ nd $b$.
Consider restriction of $\Omega(g, h)$
to $\Omega_a(g, h)$.
And so we can consider some phase
\begin{align}
    \Omega_a(g, h)
    \Omega_a(gh, k)
    =
    \omega_3(g, h, k)
    ^g\Omega_a(h, k)
    \Omega_a(g, hk)
\end{align}
where $\omega_3(g,h,k) \in U(1)$
and $\omega_3 \in C^3(G, U(1))$.
But you see that $\Omega_3$ does need to satisfy the $3$-cochain equation
\begin{align}
    d\omega_3 = 1
\end{align}
So then if we change
\begin{align}
    \Omega_a(g, h)
    \to
    \Omega_a(g, h)
    \mu_2(g,h)
\end{align}
for $\mu_2\in U(1)$,
then
\begin{align}
    \omega_3 \to \omega_3\, d\mu_2
\end{align}
This shows us the distinct set of phases is classified by $H^2(G,U(1)$.
So distinct $\omega_3$'s are classified by
\begin{align}
    H^3(G, U(1))
\end{align}
So that associativity condition is satisfied up to a phase and that's where we
get this $H^3$.

There's no projective representation anywhere.
We're just restricting and cutting
until we finally get this.

In order to get this to work,
we need to assume something more specific about how $U_g^{edge}$ looks like.
In higher dimensions,
we have to assume
\begin{align}
    U_g^{edge}
    &=
    N_g
    S_g
\end{align}
where $N_g=\sum_\alpha e^{i N_g'(\alpha)}$ is non-on site but diagonal,
and $S_g=\sum_\alpha \bra{g \alpha}\ket{\alpha}$
And in general,
the docs appear 

\begin{question}
    what happens if you had a system living on an annulus.
\end{question}
If you're taking about intrinsically 2D systems,
you can talk about the edge on a 

I'm just focusing on this operator on one segment.
On that segment,
I satisfy this group up to 1 

There are for mathematicians rigorous definitions and derivations.

I just wanted to mention one more ting.
Three's another way of phrasing these topological.

There is a compact way to define the topological action for group cohomology
models
\begin{align}
    Z\left( M^{d+1}, A \right) &=
    e^{iS_{top}(A)}
\end{align}
Firstly, every group $G$ has a corresponding space called ``classifying
spare'' $BG$.
The defining property is that
$\pi_1(GB) = G$
and
$\pi_k(BG)=0$
for $k>1$.
Every $G$ gauge field is in one-to-one correspondence with
homotopy classes of maps.
You can think of gauge fields as being a map from your manifold into your
classifying space.
$f: M^d \to BG$.
What we did when we triangulated the manifold,
and put gauge fields on links,
that actually does specify a map from $M$ to $BG$.
This is just a fancy way of saying what we did in practice.

The classifying space for the circle is
$B\mathbb{Z} = S^1$.
The classifying space for $\mathbb{Z}_2$ is
$B\mathbb{Z}_2 = \mathbb{RP}^{\infty}$.
The classifying space for phases is
$BU(1) = \mathbb{CP}^{\infty}$.

Secondly, the other thing is that group cohomology
$H^k(G, U(1))$
is equivalent to
``regular'' cohomology (singular de Rham cohomology)
Nakahara has a nice book called geometry, topology and physics..
\begin{align}
    H^k(G, U(1)) \simeq 
    H^k(BG, U(1))
\end{align}
is true for finite $G$,
but is more subtle for groups.


Thirdly,
if I had a cohomology on $Y$,
I can get a class on $X$ and the map that does this is called
$f^*$, the \emph{pullback}.
That is,
if $f:X \to Y$,
then
\begin{align}
    f^*: H_k(Y, U(1)) \to H^k(X, U(1))
\end{align}
is the pullback.

We can put these 3 facts together.
Given a class $[w_{d+1}] \in H^{d+1}\left( BG, U(1) \right) \simeq H^{d+1}(G,
U(1)$
which is isomorphic
and a $G$ gauge field determined by
\begin{align}
    f_A: M^{d+1} \to BG
\end{align}
then we can define an element of the $H^{d+1}$ cohomology class
\begin{align}
    f_A^\left[ \omega_{d+1} \right]
    \in
    H^{d=1}\left( M^{d+1}, U(1) \right)
\end{align}
and then we have our path integral
\begin{align}
    e^{iS_{top}}
    &=
    e^{i\int_{M^{d+1}} f_A^*\left[ \omega_{d+1} \right]}\\
    &= Z\left( M^{d+1}, A \right)
\end{align}
These 3 facts allow you to package this path integral in this nice way.
Unfortunately,
I would need at least 3 lectures to unpack all of this.
I'm just telling you so you know where to look to learn about this.
Are there any questions?
Actually,
if you have questions,
I might not be able to answer it.

In the last minute,
I want to mention where I'm going next.

Suppose we have a TQFT with $G$-symmetry $T$.
We can always ``gauge'' $G$.
That means we promote he background $G$ gauge field to a dynamical gauge field,
which gives a new configuration $T$ mod $G$,
which is $T/G$.
I can then sum over flat gauge field configurations.
I put brackets to say we are summing over inequivalent ones that are not related
by gauge transformation.
\begin{align}
    Z_{T/G}\left( M^{d+1} \right)
    =
    \sum_{\left[ A \right]}
    Z_{\left[ \omega_{d+1} \right]}
    \left( M^{d+1}, A \right)
\end{align}
And so dynamical, topological $G$-gauge theory is classified by
$H^{d+1}\left( G, U(1) \right)$.
We are interested in 2 classes of this phenomena.

1. Discrete and finite $G$.
This leads to models like the toric code,
and more generally quantum double models.
Like Levin-Wen models, etc.
If we start off with an invertible theory,
and we gauge $G$,
we end up with a non-irreversible theory.
SO one way of getting non-invertible theories is this.

2. Continuous $G$.
This takes us toe Chern-Simons theory in $(2+1)$D.
At this point,
the discussion bifurcates.

We can talk about non-invertible phases
you get from gauging invertible theories.

We can also label things on triangulations and sum over all labels.
Next time,
I'll start getting into this quantum Chern-Simon theories,
which allows us to describe fractional quantum hall states,
which are the only experimentally realised non-invertible theory.

Are all non-invertible topological phases obtainable by gauging an invertible
TQFT.
30 year old conjecture by Moore and Cyberg still standing.
There are candidate counterexamples,
but we're not sure.
