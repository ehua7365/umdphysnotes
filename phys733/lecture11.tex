\section{TQFT for Majorana Chain}
I told you that
\begin{align}
    V(S^1, \pm) \simeq \mathbb{C}
\end{align}
I also told you the path integral
\begin{align}
    Z(\Sigma_g,eta) = (-1)^{\textrm{Art}(\eta)}
\end{align}
where $\Sigma_g$ is a closed genus-$g$ surface.
I told you that genus $g$ surfaces are in 1-1 correspondence to a space of
quadratic forms.
\begin{align}
    \text{spin structures on }\Sigma_g \leftrightarrow
    \text{quadratic forms }\in Q\left( H_1\left( \Sigma_g, \mathbb{Z}_2 \right),
    \phi\right)
\end{align}
where $\phI$ is a symmetric bilinear form which is the intersection form that
tells you whether two ops are intersecting or not.
And remember
\begin{align}
    q(x + y) = q(x) + q(y) + \phi(x, y)
\end{align}
I asserted the space of quadratic forms are in 1-1 correspondence with the space
of spins structures.

As for that Arf,
I defined something called the Arf invariant of a quadratic form,
so you take the Arf invariant associated with the spin structure.

I just asserted a bunch of stuff because it was just math.

One example was the Torus, and the partition function was 
\begin{align}
    Z(T^2, \eta) &= -1, 1, 1, 1
\end{align}
depending on whether the boundary conditions are
PP, AP, PA or AA.

If you look at the partition function of a disk,
the disk has a unique spin structure I won't write.
\begin{align}
    Z\left( D^2 \right)
    \in
    V\left( S^1, - \right)
\end{align}
where the $-$ means anti-periodic boundary conditions.
I will denote $+$ for periodic.

Last time,
I asserted the following
\begin{proposition}
    Anti-periodic boundary conditions on a circle $S^1$ give a spin structure
    than can be extended to the disk $D^2$.
    These are also called \emph{bounding} spin structures,
    whereas periodic boundary conditions are \emph{non-bounding}.
\end{proposition}
If you draw a circle,
you can think of the fermion as defining a spinor in your space.
Consider 2 situations.
1 is where you have part of the spinors tangent to the circle,
and the other parts are looking like this.
This is the one that gives you anti-periodic boundary conditions,
because as you go around the circle,
the arrows wind around the circle,
and you get a $2\pi$ rotation,
which is a minus sign for fermions.
This can be extended to the disk.

On the other hand,
this spin structure (all pointing out) cannot be extended to the disk.
I don't have the time to give you background on spin structures,
so take this as a bunch of assertions.
If you want to know what's going on behind this,
we have to talk offline.
But if you do have questions, ask.

\begin{question}
    Assume if you go across the line and rotate multiple lines instead of one
    time.
    These are periodic and anti-periodic?
    We're not sure if those are also have the same.
\end{question}
You want me to convince you that some more complicated rotation on that circle
and show the anti-periodic one will be bounding?
Not sure I can do that right now?

\begin{question}
    What's a spin structure.
\end{question}
You start with a tangent bundle,
say in $d$ dimensions.
The tangent bundle is a $SO(d)$ bundle.
At every point in your manifold there is a space of tangent vectors.
If the space if curved then this will be curved.
For fermions, you don't want $SO(d)$,
you want a double cover of this.
You want a $Spin(d)$ bundle.
So $spin(d)$ is the double cover of $SO(d)$.
Basically,
on every patch originally we had a map to $SO(d)$,
so we had some map $\phi:U_\alpha\to SO(d)$.

The manifold is covered by open charts $\left\{ U_\alpha \right\}$

But instead,
you want to map to $Spin(d)$,
and you have to specify which covering of $SO(d)$ you want to cover.
So really you want to pick it with a sign
\begin{align}
    \tilde{\phi} = (\phi, \eta)
\end{align}
where $\eta$  tells which of $SO(d)\to Spin(d)$
we choose.

A spin structure is something that as you go along the manifold,
you're possibly getting rotated by $SO(d)$,
and something.
You can think of that as one basic definition.

\begin{question}
    How can a manifold not have a spin structure?
\end{question}
It might not be possible to choose $\pm$ signs that's consistent everywhere.
I don't have a 2-line explanation,
but what you find is that you want to make sure this lift is consistent on
triple overlaps.
So make sure lift is 

$SO(d)$ bundle,
there are transition functions
$f_{\alpha\beta}\in SO(d)$
which tell you how to go from one coordinate patch to another.
But when you lift it,
you want
$\tilde{f}_{\alpha\beta} = (f_{\alpha\beta},\eta)$,
and you want this consistency identity
\begin{align}
    \tilde{f}_{\alpha\beta} \tilde{f}_{\beta\gamma} \tilde{f}_{\gamma\alpha}
    = 1
\end{align}
and this is 
\begin{align}
    \delta\eta = W_2(TM)
\end{align}
You need manifolds with trivial second Stiefel-limits.

The spin chain is just 1D,
and all 1D manifolds are just circles if they're closed,
and they always admit a spin structure.
I only know an example in 4D of a manifold that doesn't admit a spin structure.
The only example of a manifold
All 1-, 2- and 3- manifolds admit spin structures.

To finish off the TQFT,
we would also want to understand the path integral on this kind of structure.
If I don't put an markings,
it's anti-periodic.
But imagine changing the boundary conditions,
like a branch cut here that makes this APP.
In principle,
these are all maps from $S^1\times S_1$ to $S^1$.
I haven't specified what the maps are,
but show that everything that is consistent you get $(-1)^{\textrm{Arf}}$.
Because these are 1D Hilbert spaces,
it's just going to be a choice of signs.
With all these signs,
you can build path integrals on closed surfaces using the multiplicative rule
and you can always get the Arf invariant.

That was an example of an invertible TQFT,
with the vector space being 1D.

The Majorana chain is an example of an invertible topological phase of matter.

Furthermore,
we have this $\mathbb{Z}_2$ invariant,
which is a $I=P_P P_A$.
If I took a double layer system,
if I had a $-1$ here, I would get $-1$.
That is,
under ``stacking'',
which means take the total Hamiltonian to be $H=H_1+ H_2$
and the Hilbert spaces to be $H=H\otimes H_2$.
If I two of them together,
I get a trivial phase.
This forms a $\mathbb{Z}_2$ group structure under stacking,
and in this case,
the state is its own inverse.

\begin{question}
    The state is its own inverse or the field?
\end{question}
All of the other.
There is some adiabatic path from
$\ket{\psi_A}\otimes \ket{\psi_B}$
to a product state.

\begin{question}
    What is a product state?
\end{question}
This should just be a state where I start with a Fock vacuum
$\ket{0}$,
I go over all sites and a put a fermion on all sites
$\prod_r c_r^\dagger \ket{0}$.
It should be a state with a fermion or not on every site.

\begin{question}
    In the trivial phase of the Majorana phase,
    it's a superconducting state?
\end{question}
The trivial phase,
there are various versions of it,
(1) was the completely empty phase $\ket{0}$.
(2) another phase was the ``atomic'' superconductor,
where on every fermion pair we had an odd number of fermions.

If we break translational symmetry,
we should be able to get an adiabatic path
$\ket{\psi_A}\otimes\ket{\psi_B} \to \ket{0}$,
when we're allowed to break symmetries.

\begin{question}
    Stacking is a way of combining 2 TQFTs?
\end{question}
Here I'm just talking about microscopic systems.
I'm literally saying it's now 2 chains instead of 1 chain.
That's all I mean by stacking here.

As far as the TQFT is concerned,
stacking means I multiply the path integrals.
$Z_{AB} = Z_A Z_B$.

\begin{question}
    What happens to the vector spaces?
\end{question}
There's a subtlety,
I have a fermion parity in the top system,
and a fermion parity in the second system,
but the parity you want to look at is both at the same time $P=P_1P_2$.
The vector spaces are all 1D and labelled by 1D or 2D and all are related by
the same.

\begin{question}
    How can you get a trivial phase just by grouping 2 together.
\end{question}
The starting Hamiltonian is decoupled but the path is definitely going to couple
them.
Maybe it's more precise to say that there's a path of Hamiltonian $H(s)$
and $H(0) = H_1 + H_2$ are decoupled.


\begin{question}
    Does the TQFT make it easier to see this factor,
    or can it be deduced from the microscopic?
\end{question}
No, this doesn't require TQFT, just that there is a $\mathbb{Z}_2$ invariant
that is either $\pm 1$,
and that together they give $+1$.
That fact is mirrored in the TQFT by $Z=(-1)^{\textrm{Arf}(\eta)}$.

Any other question?

\section{Spin bordism group}
One last point about this TQFT story is that I want to introduce the notion of a
spin bordism group.
A bordism is a high dimensional manifold between two manifolds as boundaries.

I can define a bordism group.
2 $d$-manifolds are equivalent to each other if there exists a bordism between
them.
$M_1^d \sim M_2^d$ if there exists a bordism from $M_1^d$ to $M_2^d$.
So that allows me to define equivalence classes $[M^d]$
of manifolds.
Then 
I can define multiplication by disjoint union
\begin{align}
    [M_1^d]\times [M_2^d]=
    [M_1^d \sqcup M_2^d]
\end{align}
And the inverse is just
\begin{align}
    [M_1^d]^{-1} = [\bar{M}_1^d]
\end{align}
So this group is denoted $\Omega^{\text{or}}_d$
to denote that they are oriented manifolds.
Mathematicians have computed what these groups are for each dimension.

For example, in $d=2$,
the bordism group is trivial.
In 3D it's also trivial.
Then in 4D it's $\mathbb{Z}$.
1D is also trivial.

Now consider not just bordism,
but also spin bordism,
which means each manifolds also comes equipped with a spin structure,
so we're going to have not just the manifolds.

Spin bordism:
We have $(M^d,\eta)$.
The spin structure also must extend to the higher dimension manifold,
and we define multiplication and inverse in exactly the same way.
I won't write it out,
because it's just adding spin structure,
and we call this
$\Omega_d^{\textrm{spin}}$,
which mathematicians have also calculate.

In 1D, it's $\mathbb{Z}_2$.
In 2D, it's $\mathbb{Z}_2$.
In 3D, it's trivial.
In 4D it's $\mathbb{Z}$.

What is the meaning of this?
Let's focus on $d=2$.
In $d=2$, the oriented bordism group is trivial,
which means every closed oriented 2D manifold is the boundary of some 3D
manifold.
That means every surface is boundary of a 3D manifold.

For example,
the boundary of a solid torus
is torus disjoint union nothing,
which means the torus is bordant to empty,
which means the torus is trivial as far as the bordism group is concerned.

Every 2D surface can have its interior filled,
and now I have a 3D manifold inside,
and it's a trivial bordism group.

So spin bordism is not trivial,
and that's related to the point that not all spin structures are boundaries.

On the torus,
suppose I have anti-periodic boundary conditions on both loops.
If one of them is anti-periodic that's enough.
Because if that's anti-periodic,
that's enough.
And if the other one is anti-periodic that's enough.
And so AA, AP and PA are all boundaries,
all bound.
But PP does not bound.
If both of these are periodic,
both of these spin structures cannot be extended to the interior bulk.
That's why its bordism group is $\mathbb{Z}_2$.

Already, you see a tight relation between the path integral,
and the relation of these bordism groups.

What happened is that my path integral is actually
for every element in $\Omega_d^{\textrm{spin}}$,
it gives me $\pm 1$ and that was my path integral for my TQFT.

So my TQFT path integral gave me a $\pm 1$.
Actually, I can take an element
$x \in \Omega_2^{\textrm{spin}}$
and $Z(x)=\pm 1$
depending on which bordism group it's in.
So really, the path integral is a map
\begin{align}
    Z: \Omega_2^{\textrm{spin}} \to \left\{ \pm 1 \right\}
\end{align}
That is our TQFT.
A slightly different way so saying it's a homomorphism
from $\Omega_2^{\textrm{spin}}$ to $U(1)$,
denoted
$\textrm{Hom}\left( \Omega_2^{\textrm{spin}}, U(1) \right)$.
And this is deep,
it's called the
\emph{Pontryagin dual} of $\Omega_2^{\textrm{spin}}$.

I'm telling you this because this is the property that generalises.
Manifolds can have background $g$ gauge fields,
where we define gauge field on our manifold.

\begin{question}
    Why $U(1)$?
\end{question}
$U(1)$ is the thing that generalizes.
Sometimes it's not $\mathbb{Z}_2$ but $\mathbb{Z}_m$,
in which case you get a complex phase.
For an invertible TQFT,
the path integral is just a phase,
which is why it's $U(1)$.
For non-invertible TQFT,
there's no such a nice classification.

\begin{question}
    Why do we choose $\pm 1$ here?
\end{question}
In this example it's $\pm 1$ and this example is the
Pontryagin dual.
It's a much more complex structure called an Anderson dual,
and I don't even know what the complex structure is.

\begin{question}
    Does it just have to be a complex phase or a root of unity?
\end{question}
If your bordism group is finite,
then it must be a root of unity.
But if it's a $\mathbb{Z}$ classification,
it's not clear why.
If it's finite order it's obvious.

The fact you have 2 distinct phases trivial and non-trivial,
has a mirror in the language of TQFTs,
the spin bordism group having two elements as $\mathbb{Z}_2$.

\section{Ising model and Jordan-Wigner duality}
Let's write down the transverse field Ising model in 1D.
\begin{align}
    H &=
    -J\sum_{i} \sigma_i^x \sigma_{i+1}^x
    - h\sum_{i=1}^{N}\sigma_i^{z}
\end{align}
where the $\sigma$ are just Pauli matrices.
Let's do an open chain of $N$ sites.
Now there is a global $\mathbb{Z}_2$ symmetry,
which I am going to write as
\begin{align}
    Z &=
    \prod_{i=1}^{N}\sigma_i^z
\end{align}
and this is a global symmetry that commutes with the Hamiltonian and squares to
1.
$[H,Z]=0$.

There are 2-spin systems per site.
Instead of thinking up and down ad being fermion parity even or odd.
So fermion and no fermion.
So up and down can be reinterpreted as fermion and no fermion.
For each site $i$,
I'm going to define 2 Majorana fermions.
\begin{align}
    \gamma_{2i - 1} &=
    \sigma_1^x \prod_{j=1}^{i-1}\sigma_j^{z}\\
    \gamma_{2i} &=
    \sigma_i^y \prod_{j=1}^{i-1}\sigma_i^{z}
\end{align}
which means
\begin{align}
    \sigma_i^z = -i \gamma_{2i - 1}\gamma_{2i} = P_i
\end{align}
which is the fermion parity operator.
And it's a exercise to show that these satisfy the algebraic conditions for the
Majorana operators.
And the product term becomes
\begin{align}
    \sigma_i^x \sigma_{i+1}^x &=
    -i \gamma_{2i}\gamma_{2i + 1}
\end{align}
and the $Z$ symmetry becomes
\begin{align}
    Z = P = \prod_{i=1}^{N}P_i
\end{align}
which is just the total fermion parity.

And if you substitute this into the Hamiltonian,
you find the system is just a Majorana chain.
\begin{align}
    H &=
    - J \sum_{i=1}^{N-1} i \gamma_{2i} \gamma_{2i + 1}
    + h\sum_{i=1}^{N} i \gamma_{2i - 1}\gamma_{2i}
\end{align}
So there is a mapping from Ising model and the Majorana chain.
And you see how the 2 phases of the Ising model corresponds to the trivial and
non-trivial phase of the Majorana chain.


The ferromagnetic phase $J\gg h$ is when everything is aligned  in $x$ and
that's the
topological phase in Majorana language.
Now you can see the origin of the degeneracy,
with chain,
that had Majorana zero modes,
which led us to a double degeneracy in the ground state,
with even vs odd parity.
In the Ising model,
we also have a degeneracy with all point in $+x$ or all pointing $-x$.


And then the state would be
\begin{align}
    \ket{\rightarrow \cdots \rightarrow}
    \pm
    \ket{\leftarrow \cdots \leftarrow}
\end{align}
and $Z\ket{\pm} = \pm \ket{\pm}$,
which in Majorana language means
$P= \pm 1$,
and $P \propto -i\gamma_L \gamma_R = \pm 1$.

In the $J\ll h$ limit,
we have the paramagnetic phase.

Why bother to do all this just to describe the Ising model?
Well no.
There is a non-local relationship.
That non-local relationship is why there is no local operator to distinguish the
ground states.
No local bosonic operator can distinguish the ground states and connect them.
The degeneracy comes from symmetry breaking in the Ising model,
but it's from topology in Majorana.
Spontaneous symmetry breaking
gets mapped by this non-local mapping to topological degeneracy.

\begin{question}
    Isn't there a duality between strong $J$ and weak $J$,
    how can there be different Majorana parities,
    one is topological phase and another one isn't.
    How can they have different parities if they're dual.
\end{question}
Parity in Majorana language is just total $Z$ in Ising model.
This is going to open a whole can of worms.

The Ising model has 2 dualities.
One of them is this Jordan-Wigner duality.
There's another duality,
which is not really correctly stated in the literature which is a
Kramers-Wannier ``duality'' duality where you swap
$J\leftrightarrow J$.
But it's not a true duality
because the ferromagnetic phase has 2 ground states,
but the paramagnetic phase only has 1 ground state.
So the degeneracy isn't mapped right.

However,
there is a duality,
where the Ising model is coupled to an Ising $\mathbb{Z}_2$ gauge field.

The usual ``Kramers-Wannier duality''
already has a problem with the counting of the ground sate.
That counting is corrected by coupling the Ising model with a
$\mathbb{Z}_2$ gauge field.
And that might solve all the paradoxes you might have.

\begin{question}
    If we such a system with Pauli matrices,
    can we characterise every Hamiltonian?
\end{question}
If you stick to 1D,
you can always do this transformation nd you find that the $\mathbb{Z}_2$
symmetry-broken state will always map to the topological phase.
The Jordan-Wigner in higher dimension,
you'd have to convert to higher $d$.

Yu-An has spent the last few years doing this.
Xiao-gang Wen showed this can be transformed from higher $d$ to some duality.

You have to be careful with non-local transformations to make dualities,
you need to know what's local and what's not local.
If you started with Majorana and went backwards,
you would say everyone missed something,
you could just do topological quantum computing with Ising models!
But you have to be careful because the aping is non-local.

\begin{question}
    What are the possible applications?
\end{question}
There's an application in quantum computing to simulate fermions,
and you want to simulate them without having them,
then this duality could help.
This is a baby version of duality.
If you understand this,
you can try to understand all sorts of ore complicated dualities.

\begin{question}
    I've seen this being related to being on the boundary of a toric code,
    is this related to each other,
    or is this a different kind of structure where the toric code is in 2D?
\end{question}
Um. Well we'll get to the toric code,
but if you had a 2D phase of matter and looked at the boundary,
indeed,
you could map the boundary into a 1D transverse field Ising model,
or a Majorana china.
In fact, the toric code has 2 types of boundaries,
rough and smooth,
and those map to the 2 Majorana phases of matter.
All the analysis will be useful.

Any other questions?
That's all I'm going to say for Majoranas in this lecture.
There's a few more important topics.
I gave you some model Hamiltonians for Majoranas.
How do you realise real Majoranas in a real solid state system.
In the homework I'm going to have such as system.

One big direction is how to do you actually realise Majoranas and what are the
physical systems.

\section{Chern Number}
Now we shift gears to go into higher dimensions.
That was the simplest topological phase of matter in 1D.
It didn't require anything fancy except for fermions.
Now it's associated with an invariant called the Chern number.
And now we're in $(2+1)$ dimensions.

We're going to stat off by defining invariants of band structures,
the Chern number,
but it's also sometimes called the TKNN number,
after Thouless, Kohomoto, Nightingale and den Nijys.
This is some invariant of a band.

Let's say we have a free particle
and it exists as some 2D lattice model,
so we have a tight binding model that looks like
\begin{align}
    H &=
    - \sum_{ij} c_i^\dagger c_j t_{ij} + \text{h.c.}
\end{align}
So we have a Brillouin zone.
The momenta are $(k_x, k_y)$ and we identify
\begin{align}
    \vec{k} \sim \ket{k} + \vec{G}
\end{align}
where $\vec{G}$ is the reciprocal lattice vector.
And so you do the quotient
\begin{align}
    \frac{\mathbb{R}^2}{\mathbb{Z}^2} = T^2 = S^1\times S^1
\end{align}
So the Brillouin zone is actually a torus.

The idea of a Chern number is that you have bands
$E$ vs $k$.
Every bad is going to be associated with some kind of invariant 
$C_1,C_2,C_3,\ldots$ which is an integer.
The way you define this integer for every single band is as follows.

Remember that we have Bloch states
\begin{align}
    \psi_{\vec{k}}(\vec{x})
    = e^{i\vec{k}\cdot\vec{x}}
    u_{\vec{k}}(\vec{x})
\end{align}
where these $u$ plane wave Bloch waves are periodic
\begin{align}
    u_k (x) = u_k(x + e)
\end{align}
where $\hat{e}$ is the lattice vector.
The phase of these wave functions can wind around in non-trivial ways and those
are characterised by these Chern numbers.

Mathematically we define a ``Berry connection'',
which is a gauge field defined in momentum space.
\begin{align}
    A_j(\vec{k}) =
    -i \bra{\vec{u}_k}
    \frac{\partial}{\partial k_j} \ket{\vec{u}_k}
\end{align}
The reason it's a gauge field,
is if I redefine the phase of these Bloch states $u_k$,
then it doesn't change it.
That is,
under gauge transformation
\begin{align}
    \ket{u_{\vec{k}}}
    \to
    e^{i f_{\vec{k}}}\ket{u_{\vec{k}}}
\end{align}
the gauge field transforms like
\begin{align}
    A_j(\vec{k})
    \to
    A_j(\vec{k})
    + \frac{\partial}{\partial k_j} f(\vec{k})
\end{align}
and that gauge field defines a gauge-invariant field strength,
but when you integrate that gauge field through the Brillouin zone,
you get an integer invariant.

So let us define the field strength by
\begin{align}
    F_{xy}(\vec{k})i &=
    \frac{\partial A_y}{\partial k_x}
    - \frac{\partial}{\partial k_y}A_x
\end{align}
and then we can define the Chern number as
\begin{align}
    C &=
    \frac{1}{2\pi} \int_{BZ} d^2 k F_{xy}(\vec{k})
    \in \mathbb{Z}
\end{align}
There's an elementary theorem which says if you integrate a field over a closed
surface, 
you always get an integer.

The flux through any closed surface must be an integer multiple of $2\pi$.
Imagine your surface is a sphere,
and your sphere has some flux through it.
Imagine a particle that goes along some path on the sphere.
And this particle has charge under this gauge field.
The phase it's going to pick up is the look integral
\begin{align}
    e^{i\oint A\cdot dl} =
    e^{i\int_R F}
\end{align}
However you could consider the integral of this loop to be the integral of the
complement of that region.
\begin{align}
    e^{i\oint A\cdot dl} =
    e^{i\int_R F}
\end{align}
It's minus because it has the opposite orientation.
So for these to make sense,
it must be that the flux through the entire closed surface must be 
and integer multiple of $2\pi$.

I'll end there.
