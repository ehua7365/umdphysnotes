\section{Many-body definition of Chern number}
I want to talk about the many-body definition of Chern number.

For every band you can assign a Chern number.
At the face of it,
it's only a property of band structure,
of single-particle TI system.s
There's 2 pieces of evidence for something more general.

1. The Chern number dictate th Hall conductivity,
which is quantized.
So if the Hall conductivity is given by an integer invariant in the free band
theory,
as you slowly turn on interaction,
that shouldn't change,
so there should also be a quantized invariant for that
but how do you extract that from the GS wave function.
2. I had a continuum Dirac theory,
that you have a mass term,
and integrate out the other degrees of freedom.
In that field theory,
you can have higher order field theory interaction
and you get a 4-fermion interaction that is irrelevant by RG,
which you ignore,
so you still have this integer quantity that is still defined if you change
the sign of the mass.

So the Chern number should be well-defined for interacting systems that doesn't
require TI.

I will give you multiple ways of calculating the Chern number fro GS
wavefunctions.


Hall conductivity and twisted boundary conditions.
Suppose we put our system on a torus,
and we insert flux on both holes of the torus,
so consider changing the vector potential in the $x$ and $y$ directions.
\begin{align}
    \delta A_x &=
    \frac{\Phi_0}{L_x} \frac{\theta_x}{2\pi}\\
    \delta A_y &= 
\end{align}
adding these fluxes changes the Hamiltonian,
and we get a term that looks like the gauge fields coupling to the current
\begin{align}
    \delta H &=
    \int d^2x \, \delta A_i J^i\\
    &=
    - \sum_{i=x,y} \frac{\Phi_0}{L_i} \frac{\theta_1}{2\pi}
    \int d^2 x\, J^i
\end{align}
and $H[\theta_x, \theta]$
Twisted boundary conditions vs flux through the hole of the torus are related by
singular gauge transformations.
Given this Hamiltonian,
we have a ground state wave function of this Hamiltonian
\begin{align}
    \ket{\Psi(\theta_x, \theta_y)}
\end{align}
from which we can construct the Berry connection,
just like before in momentum space.

So I can define
\begin{align}
    A_j(\theta_x, \theta_y) &=
    -i \bra{\psi(\theta_x, \theta_y)}
    \frac{\partial}{\partial \theta_j}
    \ket{\psi(\theta_x, \theta_y)}
\end{align}
and from this we can define a field strength
\begin{align}
    F_{ij} &= \partial_i A_j - \partial_j A_i
\end{align}
And here the closed surface is the torus,
because $\theta_x$ can go from $0$ to $2\pi$,
and once it's $2\pi$,
we can do a large gauge transformation
to relate it to the system without the flux.
And we get
\begin{align}
    C &=
    \frac{1}{2\pi}\int d^2\theta\,
    F
    \in \mathbb{Z}
\end{align}
And this is really an integral over the torus.
\begin{align}
    \theta_x $\sim \theta_x + 2\pi\\
    \theta_y $\sim \theta_y + 2\pi
\end{align}
The BZ was a torus,
but here the space is also a torus.
We integrate this field strengh over a closed surface,
it has to be an integer from thes ame argument I gave before.

If we look at hte $\theta$-dependence,
\begin{align}
    \frac{\delta H}{\delta \theta_i} &=
    \frac{\Phi_0}{L_i} \frac{1}{2\pi}
    \underbrace{%
    \int d^2x\,
    J_i
    }_{\text{0-momentum part of current}}
\end{align}
and so we will be able to relate the Chern number
to some current-current corelation function,
which shoud be related to the Hall conductivity.
In fact,
you can prove that the Hall conducivity is
\begin{align}
    \sigma_H &=
    \frac{e^2}{h} \frac{1}{2\pi} F_{xy}(\theta_x, \theta_y)
\end{align}
The proof is in the homework.
The relation between the Hall conductivity and the Chern number is that if you
average the Hall conductivity over the space of wave funtions,
the you find
\begin{align}
    \langle \sigma_H \rangle
    = C \frac{e^2}{h}
\end{align}
that is,
the average over twisted boundary conditions.
In fact,
you find
\begin{align}
    \sigma_H &= \langle \sigma_H \rangle_{BC's}
\end{align}
People understood you get thse two formulae
and you should expect the Hall conducitvity is an average over boudnar
yconditions,
and witha gapped system,
no quantity should care what the flux is thorugh these non-contractible cycles,
in the infinite system size,
because there is a finite-correlation lngh.
If hte cycle is large,
locally you need some coherenetnce thorugh the yccle,
so you need some correlation length that scales with teh size of the system.
Intuitively,
people expected
but in a tour-de-force of mathematical physics a decade,
this was proven about 2010 by Hastings and Michalakis.

If you think of inserting flux,
or you can do a singular gauge transformation,
remove the flux,
and do some twisted boundary conditions,
and the twist is the $\theta$.
You can relate
\begin{align}
    \psi\left( r_1, r_2, \ldots, r_N \right) &=
    e^{i\theta_x}\psi\left( r_1 + L_x \hat{x}, r_2,\ldots \right)
\end{align}
and the point is,
that this is related by a singular gauge transformation.
I'm not spelling out the whole story, because I assume you know already.

The main lesson,
is that this way of defining Chern number
gives a way to define an integer invariant in a many-body system.
We can define a Chern number for any $U(1)$ symmetry,
because the moment I have a U(1) symmetry,
I can turn on a background gauge field,
and insert flux through that torus,
and define a symmetry associated with it.
So every U(1) symmetry can have a  Chern number associated with it,
and that gives a way to define a U(1) symmetry in a gapped many-body system.
And there's few things I want to emphasize here.

This thing is intrinsically many-body.
Not a single free fermions,
not assuming translationally invariant.
So not band theory.

We also assumed implicitly that the system has a unique ground state.

This definition of the Chern number so far assumes you have a unique ground
state.
We're adiabatically inserting flux,
but in some system like fractional quantum hall effect,
if you insert integer flux,
you actually wind to a different ground state,
so you need many flux quanta to wind up in the same ground state.
So here with $2\pi$ flux,
you end up with the same exact ground state.
We get this Berry phase by moving around in one direction,
and if you consider may directions,
you get a single ground ate.

We can still define a Chern number with degenerate ground sates,
but there is an index here,
and this is a non-Abelian gauge field,
and this field strength because a non-Abelian gauge field,
and we need some extra fields,
and the Chern number is the trace of that field strength.
You can still define Chern number,
but it's not a U(1).
I won't say anything more.
You can do it,
but you need to consider non-Abelian Berry gauge fields.

\begin{question}
    To go to a non-Abelian symmetry,
    how would we do that?
\end{question}
So now you're saying,
suppose we have $SU(2)$ symmetry,
can you still define a Chern number?
The quick way to answer this question is,
Chern number we associated with an effective action
and the coefficient was the Chern number.o
With SU(2),
you turn on non-Abelian SU(2) gauge fields,
then write down a non-Abelian CS theory,
and that will also have a coefficient that is also a Chern number.
There will be a Chern number,
but to define it,
you're going to consider.
Then what you do is insert flux associated with some U(1) subgroup of
SU(2)
Just compute it for some U(1) subgroup.
I said that quickly,
because I didn't explain.

\begin{question}
    what about discrete groups instead of U(1)?
\end{question}
We can write down
\begin{align}
    S_{CS}[A] = \int \frac{C}{4\pi} A\, dA
\end{align}
but this doesn't make sense if you have some net flux through your manifold,
like a net magnetic monopole,
you can't globally define $A$,
and make sense.
But here is a remedy,
and it is to go to group cohomology.
I'll just stop there.
Once you view the Chern-Simons theory through that more appropriate lens,
then that framework you can easily extend to discrete groups.

\begin{question}
    Why not glue the state to $SU(2)$ state at the boundary?
\end{question}
Let me say just a bit more about this $SU(2)$ case
For $SU(2)$ symmetry
you would consider SU(2) gauge fields.
And the effective action would be some integer
\begin{align}
    S_{\text{eff}} &=
    \frac{C}{8\pi}
    \int \Tr\left( 
    A\wedge dA
    + \frac{2}{3} A \wedge A \wedge A
    \right)
\end{align}
If the goal was just to know $C$,
you can pick $A$ to live in some $U(1)$ subgroup,
evaluate the action,
and figure out what this coefficient is.
You know the whole thing si $SU(2)$ invariatn,
so you can just pick $A$ to be in a particular $U(1)$ sugroup,
then evaluate to figure out what the coefficient is.

In general,
you could imagine you have some $SU(2)$ flux here,
but you can do an $SU(2)$ rotation,
so this $\theta_x$ lies in some $U(1)$ subgroup,
but then you have to wrory about thsi $\theta_y$.
It's not the most general thing to do,
to figure out Chern number,
just have $A$ lie in some subgroup,
and evaluate the subgroup frmo there.
Once you figured out hte Chern number,
there is only one possible Chern number.

I'm just giving you a trick to calculate the Chern number in the $SU(2)$ case.
It doesn't map analogously.

In general,
you could imagien in the $x$ direciton,
yocuould have a genera l$SU(2)$ matrix,
and you tyr to cnostruct something like the $A_j(\theat_x,\theta_y)$ formula.
But I haven't thought about tha procedure.

\begin{question}
$C$ here can be any integer,
but is it still integer for $SU(2)$?
\end{question}
The CS theory for $U(1)$,
we usually write as
\begin{align}
    S_{eff}^{U(1)} &=
    \frac{C_{U(1)}}{4\pi}\int A\, dA
\end{align}
but for $SU(2)$,
\begin{align}
    S_{\text{eff}} &=
    \frac{C}{8\pi}
    \int \Tr\left( 
    A\wedge dA
    + \frac{2}{3} A \wedge A \wedge A
    \right)
\end{align}
Note the $4\pi$ vs $8\pi$.
For bosonic systems,
\begin{align}
    C_{U(1)} &\in 2\mathbb{Z},\\
    C_{SU(1)} &\in 2\mathbb{Z},
\end{align}
But for fermionic system,s
\begin{align}
    C_{U(1)} &\in \mathbb{Z}\\
    C_{SU(1)} &\in \mathbb{Z}
\end{align}

\section{Thouless Pond}
Let's view the Chern number from polarization in 1D.
Recall if we put our system on a cylinder,
and insert flux through the cylinder.
Let's say we have some $\theta_x = \Phi$ flux.
And the cyclinder is alnog the $x$ direciton.
This causaes an electric field in the $y$ direciton,
which auses a current $j_x$.
Charge is flooiwng alon the $x$ direciton,
and if you view htis as 1D system,
the polarization is changing with time that gives rise to the current.
Think about the Chern number in terms of the polarizaiton of this effecively 1D
system.

That's the perspective,
and see how it relates to the Chern number in 2D.

Let's start off discussing dimensional reduction of a 2D tight binding model.
Suppose we have a tight-binding model on a cylinder.
$x$ is alnog, $y$ is around the cylinder,
because $y$ is periodic, $k_y$ is a good quantum number.

It's a cylinder in the sense there are boundaries at the two ends
in the $x$ direction.
We can do a fermion operator,
with band $\alpha$.
\begin{align}
    C_{k_{y,\alpha}} &=
    \frac{1}{\sqrt{L_y}}
    \sum_y C_{\alpha}(x, y)
    e^{i k_y y}
\end{align}
the Hamiltonian is
\begin{align}
    H &= -\sum_{ij} t_{ij} c_i^\dagger c_j + \textrm{h.c.}\\
    &= \sum_{k_y} H_{1D}[k_y]
\end{align}
It's a sum of independent 1D systems where $k_y$ is just some parameter now.
You can think of your system as a bunch of $L_y$ different 1D chains.
For this 1D system,
$k_y$ is effective just some number.

Let's insert flux.
\begin{align}
    A_y &= - E_y t = \frac{\Phi(t)}{L_y}\\
    A_x &= 0
\end{align}
where $E_y$ is the electric field.
You can think of it as.
\begin{align}
    H &= \sum_{k_y} H_{1D}\left( k_y + A_y \right)
\end{align}
and this causes a current,
and remember the Hamiltonian is decoupled in $k_y$,
\begin{align}
    J_x &=
    \sum_{k_y} J_{1D}(k_y).
\end{align}
Let's calculate the charge flowing across the cylinder
\begin{align}
    \Delta Q &=
    \int_{0}^{\Delta t} dt\,
    \sum_{k_y} J_{1D}(k_y)
\end{align}
Now,
what's happening is,
you can think of this $J_{1D}(k_y)$,
as having some polarization,
and the current is the time derivative of this 1D system.
\begin{align}
    J_{1D}(k_y) &=
    \frac{d P_{1D, x}(k_y)}{dt}
\end{align}
so a changing polarization leads to a current,
charged being pumped in that direction.
And this is just
\begin{align}
    \Delta Q &=
    \int_{0}^{\Delta t} dt\,
    \sum_{k_y} J_{1D}(k_y)
    =
    \left.\sum_{k_y} \Delta P_x (k_y)\right|_{0}^{\Delta t}\\
    &=
    \left.\frac{L_y}{2\pi} \int_{0}^{2\pi} dk_y\,
    \Delta P_x(k_Y)\right|_{0}^{\Delta t}
\end{align}
And we're inserting flux in the adiabatic limit,
and say we insert one flux quanta.
\begin{align}
    E\, \Delta t &=
    \frac{2\pi}{L_y}
\end{align}
and here
I'm assuming all units are 1.
And the change in polarization is just going to be
\begin{align}
    \Delta P_x (k_y) &=
    P_x\left( k_y + \frac{2\pi}{L_y} \right)
    - P_x(k_y)
\end{align}
and in the limit,
this becomes
\begin{align}
    \Delta P_x (k_y) &=
    \frac{dP_x}{dk_y} \frac{2\pi}{k_y}
\end{align}
and what we leran is that the $2\pi/L_y$ factors cancel, so
\begin{align}
    \Delta Q &=
    \int_{0}^{2\pi} dk_y\,
    \frac{dP_x}{d k_y}
\end{align}
and we can just htink of this as a look integral in the Brilluoin zone.
\begin{align}
    \Delta Q &=
    \oint dk_y\,
    \frac{dP_x}{d k_y}
\end{align}
and you can just think of $\Delta Q$
as the winding number of the polarization,
which is equal to the charge pumped across the system.

\begin{question}
    Between the time and frequency arguemnts,
    $\Delta P$ should be the difference in polarizaiton at different times?
\end{question}
The current is the change in polarization,
but the reason it's change in time is becake $k_y(t)$ is in time.
\begin{align}
    J_{1D}(k_y) &=
    \frac{dP_x\left( k_y(t) \right)}{dt}
\end{align}

\begin{question}
    What's the physical meaning of polarization winding?
\end{question}
Physically,
this is what's going on.
Physically,
take the states in a given Chern band,
and because our band has a Chern number,
we cannot write localized Wannier functions,
but they can be paritally localized Wannier functions.
Physically, localized wannier functions for the Chern band
$\ket{W(k_y, x)}$.

As $k_y$ increases,
these Wannier functions are shifting in position,
the states get shifted in $x$,
specifically let's look at the expectation values
\begin{align}
    \bra{W(k_y,x)} \hat{x} \ket{W(k_y, x)}
    = x + P(k_y)
    = \bar{x}_{k_y, x}
\end{align}
Some extra steps of algebra area needed to show this.
we have partially.
The fact the polarization winds
means that if you tried to plot this average
$\bar{x}_{k_y, x}$ vs $k_y$,
it means that if you start at lattice site 0,
you end up at lattice site $C$ after $k_y$ goes up by $2\pi$.
There are two ways of saying it.

What it's saying is that the average positiion of this partially localized
Wannier function changes by $C$ units
\begin{align}
    \bar{x}_{k_y + 2\pi, x} &=
    \bar{x}_{k_y, x} + C
\end{align}
Alternatively,
after inserting $2\pi$ flux,
this average changes by $C$ windings.
At each particular poin,
the polizariotn has changed,
and the net change in polairziaton is just hte net charge that goes from one
side toe another.
This is why you relate $J$ and $P$.

\begin{question}
    What's the picture of the winding number of $P$ around $y$?
\end{question}
The winding
\begin{align}
    \oint dk_y\, \frac{dP}{dk_y}
    &=
    p(k_y + 2\pi) - P(k_y)
\end{align} shift by $2\pi$ and see how much polarization changes.


\begin{question}
    If $C$ is zero,
    we expect this to be locallized?
\end{question}
No,
if $C=0$,
there's no obstruction to write down localied Wannier funcfions,
but we can still write paritally lcoalized Wannier functions.
As we change $k_y$,
these positions would be shifted over by $C$,
which means they don't shift over at all.


\begin{question}
    These Wannier functions are not deinite combinatinos of Bloch states?
\end{question}
You cna think of these nsulators as filling some single-articulare stares,
but which basis you think in is up to you.
You alwasy have the right ot think in the basis they are paritally localized.
winding literally means graduatlly


A nice exercise is to write down the Wannier functino in terms of the Bloch
states of the bands,
and tune the superpsoitiosn so you maximize the lcoalizeation o the wannier
function.

\begin{question}
    Do we have a known quantity for the lcoalization?
\end{question}
For the moment forget about Chern numbers entirely,
and just think about 1D systesm.
More generally,
consider some 1D system that depends on some apramter $\theta$.
\begin{align}
    H_{1D}(\theta)
\end{align}
If $\theta$ varies with time,
we could get some current in this system.
Because $\theta$ is the only thing changing in this setup,
the current is going to be some response linear function $G(\theta)$ times the
derivative.
\begin{align}
    J(t) &= G(\theta) \frac{d\theta}{dt}
\end{align}
and this is varying adiabatically in time,
so we can apply linear response,
as the sytem is always in equilibrim.
Now the current,
is almost by definition
\begin{align}
    J(t) &= \frac{dP}{dt},
\end{align}
but we can also think of it in terms of
\begin{align}
    J(t) &= \frac{dP}{dt}
    = \frac{\partial P}{\partial \theta} \frac{d\theta}{dt}
\end{align}
which means
\begin{align}
    G(\theta) = \frac{\partial P}{\partial \theta}.
\end{align}
what you find for the tight-binding model is that you get a clean expression
\begin{align}
    G(\theta) &=
    \oint \frac{dk_x}{2\pi}\left( 
    \frac{\partial A_x}{\partial \theta} - \frac{\partial A_\theta}{\partial k_x}
    \right)
\end{align}
where this $A_\theta$ is the Berry connection,
with
\begin{align}
    A_x &=
    -i \bra{u(k,\theta)} \frac{\partial}{\partial k_x}
    \ket{u(k,\theta)}\\
    A_\theta &=
    -i \bra{u(k,\theta)} \frac{\partial}{\partial \theta}
    \ket{u(k,\theta)}\\
\end{align}
and these $u$ are Bloch wave functions.
It should look familiar to you.
This is just the field strength of this Berry connection in this
$(k_x,\theta)$-space.
Now interestingly,
if $\theta$ is a periodic parameter so that
\begin{align}
    H(\theta + 2\pi) = H(\theta)
\end{align}
then ifwe take the inetegral,
\begin{align}
    \frac{1}{2\pi}\int d\theta\, G(\theta) &=
    C
\end{align}
where $C$ is some integer Chern number.
This is an amazing cool thing.
You have a family of 1D systems,
you try to look at the current floiwng through the system,
and that current is dteermined by a response function,
and that respnose fucntion
is just het integral of the field strength of the Berry connection.

So to do one mofre thing.
If we pick a gauge where $A_\theta$ is single-valued,
\begin{align}
    G(\theta) &=
    \frac{\partial}{\partial\theta}
    \oint \frac{dk_x}{2\pi} A_x
    = \frac{\partial P}{\partial \theta},
\end{align}
so we learnt something cool:
$P$ itself is equal to the Berry phase.
\begin{align}
    P &=
    \oint \frac{dk_x}{2\pi}A_x
\end{align}
The poliaziaton is taking the Berry conntion,
and looking at hte holonomy of the Berry gauge field in momentum space.
This is the so-called
\emph{Berry phase theory of 1D polarization}.
Once caveat is that this is only well-defined mod integer.
Because,
I can always do a gauge transformation of my insgle-aprticle gauge funtions,
which chagnes the flux thorugh the BZ by $2\pi$.
That is,
this is only well-defined mod $\mathbb{Z}$,
adn we can do large gauge transfomrations to chagne
$\oint_{k_x} A$ by $2\pi$.

This miplies that when $\theta$ is periodic,
\begin{align}
    \Delta Q &=
    \int J\, dt\\
    &= \int G(\theta) \frac{d\hteta}{dt}dt\\
    &=
    \int G(\theta)\, d\theta\\
    &= C \in \mathbb{Z}
\end{align}
So if we do a loop in paratere psace,
we pump charge,
and that can be related to the Chern number,
arising frmo the 1 extra dimension $\theta$.

So topological pumps in 1D are tightly connected to Chern numbers in 2D.

The cna also relate the charge density to the polarization.
The current is the time derivative of the charge,
but if you havet he gradient of the polarization,
that means you have some charge density somehwere.

By the continuity equation,
\begin{align}
    \frac{d\rho}{dt} &=
    -\frac{dJ}{dx}
    =
    -\frac{\partial^2 P(\theta)}{\partial x\, \partial t}
\end{align}
and that tells us the charge density is
\begin{align}
    \rho &=
    - \frac{\partial P(\theta)}{\partial x}
\end{align}
and combining iwth
\begin{align}
    J &= \frac{\partial P}{\partial t},
\end{align}
we get
\begin{align}
    i_{\mu} &=
    \epsilon_{\mu\nu} \frac{\partial P}{\partial x_\nu}
\end{align}
from which we get an effective action
\begin{align}
    S_{\text{eff}} &=
    \int dx\, dt\,
    P e^{\nu \mu} \partial_\nu A_\mu
\end{align}
which means
\begin{align}
    S_{\text{eff}} &=
    \int P\, F_{xt}.
\end{align}
Integrating by parts,
we can confirm that
\begin{align}
    j_{\mu} &=
    \frac{\partial S_{\text{eff}}}{\delta A_\mu}
    = -\epsilon^{\mu\nu} \partial_\nu P
\end{align}
So the Berry phase of the Bloch wave functions as you go around in momentum
space in 1D.
You can think of polairziaton in 1D,
where the effective action is literally just the field strength of $A$.
Finally,
there is a tight relation between pumps in 1D and chern numbers in 2D.

\begin{question}
    Does the correspondence carry in higher dimesnions?
\end{question}
There's a Simons cllaboartion caleled ultra-quantum matter.
There's an ultra quantum theory of polarizaiton.
Andy ou can generalize this to higher dimensions,
ubt it's a research topic.
What enters is you need translation gauge fields in higher dimensinos,
guage fields assocatied with tarnslation symmetry,
and then there's a natural way of witing down this higher dimnsional verios of
this effective action.

\begin{question}
    Why don't we need them now for 1D?
\end{question}
I think it's because in 2D you can go around in a loop,
but in 1D thereis no loop except for the toal loop,
so there are no small loops.
In some sense,
you can get a way with less structure in 1D,
because it just has a lot less structure.

\begin{question}
    How to generalize Chern numbers to 3D?
\end{question}
Several modern things have happend literally in the last few years.
ify ou want to geliraze the pump 1D to Chern number 2D,
therae ra few aperps about amoalies in thespace of coupling constants.
Ify ou chagne $\theta$,
it's not eactly invairatn,
and that's averison fo an anomaly.
You can recase everything that happened here more abstratly.

There's an orthoagonal set of things happend,
that is genralis the theroy of polarization to higher dimesnison.

Ther'es not really a connection betwen those two things,
but maybe that oculd e a resaerch project.

\section{Many-body definition of Chern number}
Let's go back ot the many-body defimtiion of chern number
using our knowledge of polarization in 1D.

Suppose we havea torus.
Consider $\oint A_y = \theta_y$.
Then we have the gorund state $\ket{\psi(\theta_y)}$.

Let me define the exponentiated polarization operator
\begin{align}
    R_x :=
    \prod_{x, y}
    e^{i \frac{2\pi x}{L_x} \hat{n}(x, y)}
\end{align}
and then I want to define
\begin{align}
    \mathcal{T} &=
    \frac{\bra{\psi(\theta_y)} R_x \ket{\psi(\theta_y)}}{%
    \braket{\psi(\theta_y)}{\psi(\theta_y)}
    }
\end{align}
The polarization is $\arg \mathcal{T}(\theta_y)$.
Then the Chern number is
\begin{align}
    C &= \frac{1}{2\pi} \oint d\hteta_y\,
    \frac{d}{d\hteta_y} P(\theta_y)
\end{align}
This is differnt to twisted boundary conditions defiiton.
that was twitisting in x and y,
constructing a berry connection, field strength.
But here,
I'm only inserting $\theta_y$
and seeing how the polarization winds.

The important thing to note about this formula is that before knowing the wave
function as a function of both $\theta_x$ and $\theta_y$,
we only consider the wave function has a function a function of 1 parameter,
$\theta_y$.

\begin{question}
    What's $R_x$?
    What's the meaning of it?
\end{question}
If you write down the formula for polarization in 1D,
it's just
\begin{align}
    P &\propto \sum x n(x)
\end{align}
If you have a 2D system,
\begin{align}
    P &\propto  \sum_{x, y} x n(x, y)
\end{align}
So this $R_x$ you can think of as an exponentiated polarization
\begin{align}
    P_x :\propto e^{i P}
\end{align}

\begin{question}
    What is the minimum number of ground states to find the Chern number?
\end{question}
I'm glad you asked.
The original definition uses the wave function in terms of 2 parameters.
We dropped than down to 1 parameter here.
It turns out you can drop it down to not depend on any parameters.

We can also extrat many-body Chern numbers fom a single GS wave funtion.

Suppose we have a cylinder iwth axis in $x$,
winding in $y$.
Then we have cylinder regions $R_1$, $R_2$, $R_3$, \ldots
and $l_y$ is the lenght in the $y$-direction.
This is pretty amazing
\begin{align}
    \mathcal{T}(\phi) &=
    \bra{0}
    W_{R_1}^\dagger(\phi) 
    \mathrm{SWAP}_{1, 3}
    W_{R_1}(\phi)
    V_{R_1 \cup R_2}
    \ket{0}
\end{align}
where
\begin{align}
    W_R &= \prod_{(x, y)\in R} e^{i\hat{n} (x, y) \phi}\\
    V_R &=
    \prod_{(x, y)\in R} e^{i \frac{2\pi y}{l_y} \hat{n}(x, y)}
\end{align}
then the Chern number is
\begin{align}
    C &= \frac{1}{2\pi} \oint d\phi\,
    \frac{d}{d\phi} \arg\mathcal{T}(\phi)
\end{align}
We have a paper on this.
You think in terms of TQFT and cutting and gluing.
There are open questions about this.
We dropped down from 2 params, 1 param, to 0 params.
But this only works on a cylinder.o
Can we get it on justa patch of space?
Empriically,
this formula works even with patches.

The question is how much topological can you extract from a single ground state
wave function on a disc.
We don't know how to do it,
complexity,
algorithsm,
etc.

\begin{question}
    You've transfoered the parameter ot the operator?
\end{question}
Whether this is really a win depends.
I think it's a win,
it's a matter of if a single ground state wave functino contains all the
topology.

Also,
the homework is due on Monday.
