\documentclass{article}
\usepackage[utf8]{inputenc}
\usepackage{hyperref}
\usepackage{amsmath}
\usepackage{amsthm}
\usepackage{amssymb}
\usepackage{amsfonts}
\usepackage{xcolor}
\usepackage{mathtools}
\usepackage{physics}
\usepackage[
    a4paper,bindingoffset=0.2in, left=1in,right=1in,top=1in,bottom=1in,
    footskip=.25in
]{geometry}
\usepackage{svg}

\newtheorem{theorem}{Theorem}
\newtheorem{example}{Example}
\newtheorem{definition}{Definition}
\newtheorem{question}{Question}
\newtheorem{conjecture}{Conjecture}
\newtheorem{task}{Task}
\newtheorem{axiom}{Axiom}
\newtheorem{proposition}{Proposition}
\newcommand{\overbar}[1]{\mkern 1.5mu\overline{\mkern-1.5mu#1\mkern-1.5mu}\mkern 1.5mu}
\DeclareMathOperator{\sgn}{sgn}
\DeclareMathOperator{\im}{im}
\DeclareMathOperator{\Pf}{Pf}
\DeclareMathOperator{\poly}{poly}


\title{PHYS733 Topological phases of matter}
\author{Eric Huang}
\date{August 2021}

\begin{document}
\maketitle

\begin{abstract}
\end{abstract}

\section{Lecture 1}
The most famous place topology enters physics is in the discussion of
topological defects in ordered media.
For example you can have a vortex in a superfluid where a superconductor with an
order parameter winds in space.
The winding of some smooth order parameter.
Just to mention some examples, this would be vortices in superconductors.
Normal fluids are unordered media.
There are things called skyrmions in magnets, which if you think of things as
non-trivial windings along a sphere.

1. You have dislocations and disclinations in crystals and so on.


The mathematics that goes into describing this is homotopy theory.
You have some order, some symmetry that gets broken into some subgroup, then
it winds in a non-trivial way.
Mermin, RMP is a good reading.

These defects so far can be treated classically and is a well-trodden topic but
we're not going to focus on this.


Another example is the topological configurations of gauge fields
(magnetic monopoles) here the maths gets more sophisticated.
Gauge fields are connections on fibre bundles and fibre bundle has non-trivial
topology.
To really understand you need to study the topology of fibre bundles,
which is is mathematics.

You may have heard monopoles don't exist, like people have been looking for
magnetic monopoles for a long time, but they do exist in emergent gauge field
descriptions and you have monopoles of emergent gauge fields and those are
important.
They have even been experimentally observed in spin ice.
You can google to learn more about spin ice.

3.
Another place is that once you have non-trivial defect configurations like
solitons, if you have fermions living in your system, like electrons, then the
spectrumo f hte Hamiton might have zero-eenrgy states bound to the position of
these defects.
This subject is called zero modes of solitons and monopoles.
Here to understand this,
For example ou could have am assive field whre the mass changes sign.

So you have $x$ space, and $m$ mass of the field. It could go from negative to
positive, but as you cross zero there is some mode localized that is pinned to
exactly zero energy.
Understanding this brings us into the mathematics of index theory.


4. Topological band theory.
This is something the class is going to touch on a lot.
Here hte idea is that ify ou have a wave or a single particle in aperiodic
medium, hten thant wave or particle is going to acquire a dispersion relation.
As a function of momentum $k$ you're going to have enrgy bands $E(k)$.
Each band can have smoe topological character associated iwth them.
At each point in momentum space, you have states called Bloch states
$u_{f,k}(r)$ where $f$ is the band index and $k$ is the momentum.
As you track how it varies in momentum, that phase winds around in non-trivial
ways and that gives rise to topolgoical band theory.
This $u$ defines a vector bundle that can have non-trivial topology.
I don't expect you to understand what I'm talking about,
just want ot to give a broad interview.

THe most famous is the TKNN number, also called the \emph{Chern number}
which is an integer you can assign to every band once you know how to understand
how it behaves in momentum space.
This is a striking phenomenon and leads to the integer quanutm Hall efect.
When you fully fill a band a get a band instlualtion,
you get an electircal induslator, buti t has a Hall conductance, which si
atransverse response.
Iti s exactly quantized $\sigma_H= C e^2/h$.
$C$ is the Chenr number.

It's just about bands, so it's really something that describes single-particle physics.
These depict the one-elctonc states, where every electron can fill every
electron states.
You fill partilly or fuully any one of these bands.

The maths that describes this is the maths that came up early.
It's just the topology of figure bundles.
If you want a full understanding of this subject, you need to understand
$K$-theory.

Single-particle means the physics of interactions are not included in this
problem.
Also index theory shows up in this problem.

Q: What happens if you add interactions?

Sometimes the topological numbers are no longer meaningful if you add
interactions.
Then there are some cases where the topological invariant still has meaning with
interactions, but then we need new ways to describe it.
It could have different topological quantum numbers identified under
interactions, there's a possibility there are topological properties with no
analogues in theband theory too.
Both can happen.

5. What are all the possible phases of matter?
Two phases of matter are different if I tune the paramers o the sytem, I cannot
continuous go from one phase to another wihtout a phase transiton.
A phase transition is a singularity in the free energy density.

Let's try to classify all phases of matter.
Immediately when you pose htis question, topolgoy enters.
Imagine the space of all possible equilibrium systems let's just call it $X$
which is at all temperatures..
THey have some spatial dimension $d$ and some symmetyr $G$.
Imagine tunning the parameters of the system keeping these $d,G$ fixed.
As you parameters, you might encounter phase transition when the free energy
necessarily goes through a singularity.
In this vast space we're trying to imagine, you can remove all subspaces where
there's a phase transition and that gives us a new space $\tilde{X}$.
The problem of classifying spaces of matter is the problem of classifying all
the \emph{connected components} of $\tilde{X}$.
So the connected components of a space $\pi_0(\tilde{X})$ is called the zeroth
homotopy space.
Even the basic question of asking what the phases of matter are is immediate a
topology question asking for the connected components of a vast space we're
tyring to imagine.

6. Characterising topological phases of matter.
Most of this class is about this.
This class is mainnig about number 4 and 6.
When you htink about what are difrerntk inds of ordered phases.
For a lot of the 20th centruy, pepole thought you have symmetries spontaneously
broken, ask what local order parameters are from $G$ to some subgroup.
Then ask the effective free energy by expanding in gradients of the order
parmtere, then describet he long-wavelength properties by looking at smooth
breaking symmetries.
So we have the Ginzburg-Landau theory of symmetry breaking.
This is a powerful theory.
The heat capacity of a crystal.
Fluctations of atoms are phonons, then take expansions in the strain field and
you get it.
Fluids, magnets, just identify the local order parameter and write a GZ thoery.

In hte 1970s, Wegner studied $\mathbb{Z}_2$ lattice gauage theory.

He showed you can have phases with no local order parameter.
You need something more subtle.
This leads us to topological phases of matter.
In one sentence, a topolgoical phase of matter is the properties of the phase
canot be signalled by a local order parameter.
To understand it you need to look at non-local orperats and how they behave in
the ground state.

Q: Why ground state?
Good question.
Implicitly, when I talk about topological phases of matter, I'm talking about
zero-temperature, whichi shte ground state,
And also low-energy exciatitons fo the gorudn astate.
Some phases are distinct at zero-temperature, but at finite temperature the
distinction may melt away.

In topological phases of matter, the description you need is very intricate, has
a lot of fascinating mathematical structure, structure that is a more modern
topic in mathematics, not some 1940s maths we came up with before.

So instead, what you need are topological quantum numbers.
You insert defects which extend in various directions and dimensions and they
can have interesting topological properties.
Ultimately there is a topological quantum field theory that magically emerges
(TQFT) to describe the universal robust properties of the system.
The maths that goes into this is a very different kind of math.
It goes under the name of quantum topology, TQFT and Category Theory.
These topics are frontiers in pure mathematics as awell.

For me aI ifind it fascinating.
We have intricate math struytures called TQFTs, and somehow there are phases of
matter where TQFTs magically arise for describing phases of that system.

Q: IS TQFT is the same sense of GZ theory for long wavelength or is it a
differnet sense?
y
It's a similar sense, but it depends what you eman by similar.
GZ is useful for dynmiacal porperties and critical exponents.
TQFT is not useful for understanding dynamics, but it's useful for undestanding
braiding paritcles, so more like kinematics than dynamics.


I'm going to briefly mention a few others.

When you hear peopel talking about topological phases of matter, they're talking
about topological band thoery or TQFT.

7.Toplogical pumps.
You have some Hamltoinian that depends on some parmater.
This hamiltonian could be periodic $H(\tau + T) = H(\tau)$.
Imagine slowly changing $H$ so that by the around it comes back you're back to
where it started, but something about the systme has changed.
As you go through this system, you can pump charge across the system for
example.
The most famous is the integer quantum Hall state on the cylinder.
As you insert flux $\Phi$, flux of $0$ is the same as flux $2\pi$.
Your wave function will slowly move to the right.
Byt he time the flux goes to $2\pi$, but htere is a net winding that pumps
charge from left to right.
This is called the \emph{Thouless pump}.
It arises in the study of the integer quantum Hall effect on a cylinder.

Recently, there has been some work where the idea of a toological pump is vastly
generalised for QFT.
There's a paper called Anomalies in the couplling constants that abstracts this
notion.

8. Effective action.
Your effective action depends on some set of fields, which has topological terms
in the effective action.
.
9. 't Hooft anomaly.
Every QFT has one.
This is invertible TQFT, cobordisms and cohomology.

10. 
Metals have a Fermi surface that separates occupied and unoccupied states.
That surface can have a topology.
In a 3D metal, the surface is a closed 2D surface,
so it could be a sphere, a torus and so on.
In apritcular, there' s an invariant called the Euler characteristic that
distinguishes these possibilities.
THere's a cool paper that says the Euler charctrsic shows up in ballistic
conductance.
No disorder, just pure sytem, add contacts, measure conductance, and it turns
our the euler characteristic turns up in that measurement.
It's well-knonw in 1D but this paper generalizes that.

11. Early attempt to 
This is a failed attempt.
Lord Kelvin and Pepter Tate attempted to think of atoms as knots in the aether.
Cool idea, because as knot is stable and has a well-defined character.
But it went down the drain when the discovered there is no aether.

Another places is when we think of world-lines of particles in 3D spacetimes.
Depending on the topology of knots and links.
It could depend only on the topological invariance of knots and links.
It comes in up in 2+1D topological quantum field thoereis.

In higher dimensions, we need membranes, world-volumes and more non-trivial
kinds of knotting.

That's it for my overview.

Any other ideas?

Studnet: Feynman diagrams planar and non-planer.

I hope the lesson is that topology shows up everywhere.
The mathematics can be veyr differnet, the actual thing you're tlaking about
cane very differnt, so you have to clear.
It's maiing about characterirising toplogical phases of matter using TQFT.
Many topological phases of matter can be understood simply in terms of
topological band theory.


\subsection{Local quantum systems}
When startin fof a sstudent learning QM and may=body QM, at some point it may
occur to you a Hamlitoinin is just a big matrix and why is it differnt ot Linaer
algebra.
The fundamental didferencis the concept of locality.
We're intersted in systems with some notion of locality, often geometry.
W'ere interested in sum of linear terms.
Wihtout locality we hayve nothing to say.
It's locality in a many-body system.
What are some basic properties local quantum systems have we can demonstrate on
general grounds.

The first thing is the Hilbert space and hte locla tensor product decomposition.
We have some quantum ssytme with some hilbert space.
We're interested in Hilbert spaces that factorize
\begin{align}
    \mathcal{H} = \bigotimes_{i=1}^{N}\mathcal{H}_i
\end{align}
where $i$ labels sites on a graph $\Lambda$, for example a lattice.
A general graph is useful.
$\mathcal{H}_i$ is the local Hilbert space, which might be finite-dimensional,
in which case it's isomorphic to $\mathbb{C}^k$.
This could be useful for a spin system for example,
so spins that can be up or down, 
or hardcore bosons, I can have bosons with hardcore repulsion.
Or it could be an infinite-dimensional space like a harmonic oscillator at that
particular site.

If we have fermions, then this $\mathcal{H}_i$ will be a \emph{graded vector
space}
which means we have to specifiy which staes have odd or even fermion numbers.
And $\mathcal{H}$ is a fermionic Fock space.
Fermions don't commute, so you have to be careful to take care of the
non-locality and anti-commutation.

I immediately have a system essentially discrete.
This is not a continuum system.
To the extent we're interested in continuum systems,
we're only interested in long-wavelength approximation.
It's hard to directly define this tensor product in the continuum,
not sure what the best way is.
If you want to do it you wind up in the world of operator algebras.

We'll be interested in continuum systems that have UV completion on $\Lambda$
iwth $\mathcal{H}$.
This menas if I go to really small scales, this is what my system actually looks
like, but if I look at low-wavelneght regimes, this continuum systme is
sufficient to describe.
Often we write down fermions in the conituum like quadaratic dispersion or
llinear band structure.
These all arise from long-wavelenth approximations of fermions hopping on a
lattice.y
That's kind of what we have in mind.

Imporatntly, not all quanutm systmes have UV completion, not all QM systems
have a tensor product factorsiation.

Q: Any examples?

Yes, not all quanutum systems factorize.c
An example is pure gauge thoery, like Maxwell theory with no electric charges,
just electric and magnetic fields.
There's Gauss law constraint where div is 0.
Such a pure gauge theory does not have all local tensor product decomposition,
because there is a constraint that has to be satisfied at every point.
You could put gauge theory on a lattice to get lattice gauge theory.
In general you can have constrained quantum systems where the Hilbert space does
not factorize into such a nice form.

This leads to the question of \emph{emergeability}:
Which QFTs can arise as long-wavelength approximations to a system that has this
local tensor-product factorisation.
For exmaple, there's a lw that all guage thoeries can emerge from such systems
iwth local tensor factorisations. 
However, the guage thoery needs to have charges in every representation of the
guage group.
There's a notion of completeness for that guage theory to be emergable.
You can think of this as emergeable from qubits if you like.
What's a qubit?
Just a local 2-state system.
This is just saying my Hilbert space is just a bunch of qubits.
The questions is whatk ind of QFTs can emerge from qubit models.
We know some QG models can arise frmo qubit models.
QG models can arise from super-Yang Mills theory and that can arise from some
interacting qubit thoeyr.

We don't know how to write the standard mdoel starting from this construction,
because it has chiral fermions, neutrinos and we don't know how to put that on a
discrieitsed structure.
Some people believei t can be done, but no one has figured it out.

So the questions is: How to get standard model of particle physics.

This questions has received attention later with TQFTs and anomalies and it has
been useful thinking about hits question.

Q: Is there a verison of the tandard model thatl ooksl ike the standard model
without neutrinos?

Yes you can put QCD on a lattice.
You can just put EM on a lattice.
But if you want everything including neutrinos,
then you have to deal with chirality.
It's sometimes inpossible to get on a lattice, but sometimes possible but very
hard.

Q: Cna you do QCD wiht weka force?

The weak force is the issue because it has neutrinos.

THer'es 2 statements: Can you usefully approximate the standard mdoel by
numerical methods? Yes but it's not honestly done.
You could do domain-wall construction, with a guage field that leaks ot hihger
dimensions.
There are tricks you cna play to get numerics.
But if want non-perturbative approximations, then no.

The systems that have some chiral fermions are on the bounadries of some
systems.
You don't want the guage field to leak out ot higher dimensiosn.

Q: Is there a relation between renormalzabilit and emergeability.

No. Don't know.

Emerge ability is a new word.
It's natural from a condensed matter perspective.
If you're big into everything comes from qubits,
then this is exactly what you're interested in.

Let me define now local Hamiltonian.

Before I do that, we have a graph $\Lambda$.
We need geometry so we need to define a distnace function dist$(i,j)$
that is symmetric and satisfies the triangle inequlaity.
The most useful thing to do do is say the distance between two points is hte
lenght of hte minimum path.

If you have a set $Z\subset \Lambda$,
then the diameter of $Z$ is $\mathrm{diam}(Z)=\max_{i,j}\mathrm{dist}(i, j)$.
Then
\begin{align}
    H = \sum_{Z\subset\Lambda} h_Z
\end{align}
where each $h_Z$ only has support on $Z$.

We're interested in geometric locality.

Q: Do we need $Z$ to be finite diameter?

Well we're interested in $N\to\infty$.
I'm not done yet.
y
Geometric locality means $|h_Z|$ decays fast enough.

\begin{align}
    |h_Z| =
    \begin{cases}
        0 & \text{if }\mathrm{diam}(Z) > k\\
        C e^{-\mu \mathrm{diam}(Z)}\\
        {(C/\mathrm{diam}(Z))}^\alpha
    \end{cases}
\end{align}

Power laws do occur, but often you have screening which essentially makes it
exponential.

Q: What is hte norm of an operator?
\begin{align}
    |\mathcal{O}| = \max_{\ket{\psi}} |\mathcal{O}\ket{\psi}|
\end{align}
If $\mathcal{O}$ is just a matrix acting on some finite-dimensional space, them
this is the maximuum eigenvalue of $\mathcal{O}$.

It turns out this isn't everything we want in the most general case.

There's something useful to add.
$\Lambda$ is some general graph,
but we do not want something with a single site talking to an extensive number
of sites,
because that could be pathological.
What we want is that every site is only involved in a few number of terms in the
Hamiltonian.
that's an additional thing.

For each $i$,
\begin{align}
    \sum_{X|i\in X} ||h_X|| \le C
\end{align}
so a single site cannot be involved in too many terms.

If you're interstedi nin  QEC, you're interested in this too.
If it's just a regular lattice like a square lattice, this is going to be
naturally bounded.

It's useful to have in mind that we can relax geometric locality and instead
just htink about \emph{sparsity}.
Here we could say that we want each term in the Hamiltonian to have low weight.
So we say that $||h_Z||\le Ce^{-|Z|\mu}$ where $|Z|$ is hte cardinality of hte
set.
We just want it to be a sum of low-weight terms,
that $H$ only has support over a low number of sites,
but we don't care where the sites are.
And we also have
\begin{align}
    \sum_{X|i\in X} \le C.
\end{align}

Sparsity shows up in the Sachdev-Ye-Kitaev model.
This mdoel is
\begin{align}
    H = \sum_{ikl} J_{j,k,l} \gamma_i\gamma_j\gamma_k\gamma_l
\end{align}
where the sum is random coupling.
They're low weight, only 4, but there's no geometric locality.
This actually faiils that constartin, but there's naother version that does.


Anohter example is finite-rate quantum error correcting codes.
Where you encounter Hamiltonains with no geometric locality,
but they do have this sparsity term.

Q: SYK have support on 4 sites, how is that not local?

Each term could be a random set of 4 sites.
I don't care about what's far and close.

Some basic properties about local Hamiltoanins.
Lee-Robinson bound.
Even though this is not relativistic,
there is an effective light cone, where $x=_{LR} t$,
the correlations will only be appreciatble inside the light code,
and will decay exponentially outside the light cone.
So there's a light cone even if there's no relativity.
This Lee-Robinson bound bounds the spread of information propagation in a local
system.
The minute you define local, you can prove htis bound.
It's extrememly powerful and geenreal.

Gapped Hamiltonian implies finite correlation length.

Q: Where is the syllabus.

Do you know elms.umd.edu.

\section{2021-09-01 Lecture 2}
\section{Lieb-Robinson Bounds}
Lightcone picture.
Last time
\begin{align}
    \| h_X \| &\le C e^{-\mu \mathrm{diam}(X)}\\
    \sum_{X| i\in X}\| h_X \| &\le \alpha
\end{align}
\begin{theorem}[Hastings]
    Suppose for all sites $i$
    \begin{align}
        \sum_{X | i\in X}
        || h_X ||
        |X|
        e^{\mu\mathrm{diam}(X)}
        \le s \le \infty
    \end{align}
    where $\mu$, $s$ are positive constants.
    Let $\mathcal{O}_X$ and $\mathcal{O}_Y$ be operators
    supported on sets $X,Y\subset\Lambda$.
    If $\mathrm{dist}(X, Y) \ge 0$
    then
    \begin{align}
        \|[\mathcal{O}_X(t), \mathcal{O}_Y]\|
        \le 2 \| \mathcal{O}_X \| \| \mathcal{O}_Y \| |X|
        e^{-\mu \mathrm{dist}(X, Y)}\left(
        e^{2s|t|} - 1
        \right).
    \end{align}
\end{theorem}
Note that the exponential term becomes
\begin{align}
    e^{-\mu(\mathrm{dist}(X, Y) - V_{LR}|t|)}
\end{align}
becomes exponentially small if $\mathrm{dist}(X, Y) > V_{LR}|t|$
where $V_{LR}=2s/\mu$.

Consider how an operator $\mathcal{O}_A^l(t)$ over some
region $A$ evolves over time.
We're going to trace out everything in $S$, which is the region outside $A$
offset by some $l$.
\begin{align}
    \mathcal{O}_A^l(t) := \frac{1}{\Tr \mathbf{1}}
    \left[
        \Tr_S \mathcal{O}_A(t)
    \right]\times \mathbf{1}_S.
\end{align}
Then
\begin{align}
    \left\|
        \mathcal{O}_A(t) - \mathcal{O}^l_A(t)
    \right\|
    \le
    C \|\mathcal{O}_A\| |A|
    e^{-\mu(l - V_{LR}|t|)}
    \left(
        1 - e^{-2s|t|}
    \right).
\end{align}

The distance between two sets is defined as
\begin{align}
    \mathrm{dist}(X, Y) = \min_{i\in X, j\in Y}
    \mathrm{dist}(i, j)
\end{align}
the minimum possible distance.
It doesn't matter which distance you use so long as it satisfies the assumption.

Note that the existence of some characteristic velocity is not that surprising
because if you had some simple Hamiltonian
\begin{align}
    H = \sum_{\langle i, j \rangle}
    J_{ij} \vec{S}_i\cdot \vec{S}_j
\end{align}
there is a characteristic time scale $\tau = \hbar/J$,
and hence some velocity scale $v_0 = a/\mathcal{t} = aJ/\hbar$.
The interesting thing about the LB bound is that it is emergent.

I'm interested in how correlations can propagate in a certain temperature.
In some states, correlations may propagate at different velocities.
So this velocity can be temperature and state-dependent.

Hence people define the \emph{butterfly velocity}
which people define as a temperature-dependent and state-dependent velocity.
This is a topic of active research.
Brian Swingle studies this a lot.

This is completely general to local Hamiltonians.

\section{Gapped Hamiltonians}
Let's specialize and talk about gapped Hamiltonians.
A gapped Hamiltonian has a spectrum that looks like the following.

I could have multiple degenerate ground states.
And then I would have a finite gap to all the other excited states.
We want $\Delta E$ to be finite in the thermodynamic limit.
This is the limit where $N=|\Lambda|\to\infty$.

It's also useful to think about a family of Hamiltonians that act on
increasingly larger systems.
Looking at the sequence of spectra, in the limit, I want there to be a finite
energy gap and $q$ degenerate ground states.
The $N\to\infty$ limit is important because every finite size system has finite
energy gaps and so the notion of gapped versus gapless is only meaningful in the
infinite limit.

\begin{question}
    Why is $N\to\infty$ the thermodynamic limit?
\end{question}
We're interested in infinitely large lattices.
We call it thermodynamic limit because in that limit thermodynamics applies.

Once we have a gapped Hamiltonian, we can talk about the important notion of
\emph{correlation length}.
If we have a finite energy gap $\Delta E$, by the uncertainty principle,
this defines a timescale $\Delta t=\hbar/\Delta E$.
So we have a characteristic time and using the Lieb-Robinson velocity we can
get a characteristic length scale
\begin{align}
    \xi = V_{LR}\Delta t = \hbar V_{LR}/\Delta E
\end{align}
that governs the exponential decay of correlations in a gapped state.
Of course the interactions must decay exponentially too.
If interactions decay as a power law correlations will also decay as a power
law.
I usually talk about correlations which decay exponentially in space.

A prerequisite for this to be useful is that the interactions must decay
exponentially.
It's not only interactions that determine this length scale,
it could by properties of the graph.

\begin{question}
    Is it useful to think of it as a thermal wavelength?
\end{question}
If I have a finite temperature system, correlations will decay in space and
time, which may be set by the thermal wave length,
but here we're at zero temperature.

I'm talking about spatial correlations in the ground state at zero temperature.
However at finite temperature, there is a length scale dependent on temperature
that could also be important for decay of correlations.
They're definitely not the same thing.

If you look at spatial correlations at finite temperature $T$,
then we have two time scales.
We have $\hbar\Delta E$ and $\hbar/k_B T$,
which gives us two length scales
$V_{LR}\hbar\Delta E$ and $V\hbar/k_B T$,
Which ever one is smaller is the important one.


\begin{question}
    How to get from the theorem to the correlation length?
\end{question}
It's a really non-trivial proof.

Let me state this precisely and that will limit confusion.

The theorem is the following.
There's been so much activity in this subject in the 2000s,
mainly from Hastings.
You'd think it's all established in the 60s with no for improvement,
but no.
\begin{theorem}[Hastings-Koma 2006]
    Let $\mathcal{O}_X,\mathcal{O}_Y$ be bosonic operators
    on sets $X$, $Y$.
    That means $[\mathcal{O}_X,\mathcal{O}_Y]=0$ if $\mathrm{dist}(X, Y) > 0$.
    Assume the system has a unique ground state with a gap $\Delta E$.
    For all sites $i$, we have
    \begin{align}
        \sum_{X| i\in X} \| h_X \| |X| e^{\mu\mathrm{diam}(X)} \le S
    \end{align}
    where $\mu,S$ are positive constants.
    Then the result is that
    \begin{align}
        | \langle \mathcal{O}_X \mathcal{O}_Y\rangle
        - \langle \mathcal{O}_X \rangle\langle \mathcal{O}_Y\rangle
        | \le
        C\| \mathcal{O}_X \| \|\mathcal{O}_Y\| e^{-l/\xi}
    \end{align}
    where $l=\mathrm{dist}(X, Y)$ and
    $\xi=\frac{2V_{LR}}{\Delta E} + \frac{1}{\mu}$.
\end{theorem}
The $1/\mu$ governs the decay of the interactions themselves.
Obviously if my interaction decays as $\frac{1}{\mu}$
then obviously the decays should decay like this too if it's dominant.

\begin{question}
    Is this a tightening of the Lieb-Robinson bound?
    There was already a length scale $1/\mu$, but now we have added another term
    to that length scale in some sense.
\end{question}
I don't think it's a tightening because this LR is just really giving a velocity
rather than length.
Hastings theorem doesn't say anything about spatial correlations.
How do these operators decay far from each other.
It's quite different, because if I set $t=0$ in the Hasting theorem then I get
zero for the commutator.

Hastings is about time, but Hastings-Koma is about equal time correlations.

\begin{question}
    The VLR is state-independent,
    but Hastings-Koma is for the ground state.
    Is that somehow strange?
\end{question}
It's not strange because $V_{LR}$ is the absolute fastest the correlations could
spread.
You could ask if there are velocities smaller than this which would reduce $C$
and cause this to fall of faster.
You could ask if there is a better bound that asks state-dependent velocities.
That's a good research question.

We've assumed a few things: bosonic operaors and ground state.
Both of these can be generalized.
You could genralise to $q$-degenerate ground states.
This term gets replaced by a projector onto the ground state subspace.
And if it's feromionic, you use anti-commutators.

I won't write the general case, but if you're intersted you could look it up.

There's also a generalization for power law decaying interactions as well,
but there's no well-defined correlation length.

\begin{question}
    When $L\to\infty$ what happens when $\Delta E\to 0$?
    How can I understand the correlation increases once I close the gap?
\end{question}
usually when the gap goes to zero the correlation gets bigger and bigger and it
diverges.
In that limit, this bound becomes useless.
Usually in a gapless system, the correlations decay albegraically as power law.
It doesn't always diverge, but it usually does.

In a gaplesss system, often the correaltion length $\xi\to\infty$.
Take a band insulator which has a finite gap.
At some point the gap will close and you have a Diract point where the
bands will cross.
You tune some parameters and you get some crossing.
The system is not gapped anymore and it's gappless.
As you decrease the energy gap, the correlation lenght gets bigger
and diverges at the point.
This theorem becomes uselss and there's no bound about how the correlations
decay.
But often what happens is that the correlations decay as a power law,
so like
\begin{align}
    \langle \mathcal{O}_X\mathcal{O}_Y\rangle
    \sim
    {\left[
        \frac{1}{\mathrm{dist}(X, Y)}
    \right]}^\alpha
\end{align}
This doesn't always happen.
The system could still be gapless, but there is still a localization lneght that
essentially bounds the correlation lenght,
but this thoerem is essentially useless.


\begin{question}
    It seems that you could get the distance of $X$ and $Y$ being zero without
    $X$ and $Y$ being the same set.
    I'm confused.
    Single-particle bosonic operaotrs, $i$ has to equal $j$.
\end{question}
Are you conofused about the definiton of Bosinci operaotr or what happens if
this is zero.
By definition, if supports don't overlap each other, that's whne the distance is
positive.
As soon as they overlap, it's non-trivial.

Often you have some microscopic energy scale for a problem,
and a charactersitic lenght scape such sa the lenth saale.
It's natural to think the correaltion enght is on the orer of the lattice
spacing.
Maybe it's 2 or 4 or whatever times.

Somtiems, this is not an emergent length scale, it could be 100 times the
spacing.
If it's 100 times the size, usually something intereseting is going on,
like being close to a critical point or transition.
The point is to remember that there's a notion of naturality in physics that
everything should be order 1.
That's what's going on here.
Often it's on the order of a lattice spacing, but it's really in general an
emergent length scale associated with an emergeent energy scale the energy gap.
Usually when the two scales are completely different, there's something special
going on.

\begin{question}
    Does the ratio of correaltion lenght with system size have anything
    significant.
\end{question}
Consider $\xi/L$.
This is an important quantitiy with finite-size scaling size analysis for
numerical simulations.
You need to pay attention if it's smaller or larger than the length scale.
That's the main thing really.

There's a natural question to ask about this theorem.
But no body has aske it so far.

\begin{question}
    Not the natural qeustion, but these are equal time correlations,
    but there's dependence on propagation of information.
    How does that connect.
\end{question}
I'm not sure.
Let me think wisely.
It's better to go through the proof to see how the velocity comes out.
I don't have intuition more than dimnesional analysis.
In the actual proof, there's some Fourier transform that takes $\Delta E$ to
time, so the Leib-Robinson velocity comes in there.
So maybe some intuition I kind of tend to think about but I'm not sure how
accurate it is.
That creates excitations, but they can only propgate in some charactersitic time
$1/\Delta E$ and how far they go depends on hte velocity and that kind of
depends away.
That's some very hand waving picture so take it with a grain of salt.

OK I htink this is the natural qeusiton.
\begin{question}
    Gapped Hamiltnoains have exponentailly decaying correlations.
    But does an exponentially decaying spatial correlation imply a gapped
    Hamiltonian?
\end{question}
No, the converse is not true.
There are states iwth short-range correaltions (exponentially decayin)
that are ground states of gapless Hamiltonians.
One example is take a Hamiltonian that is the usm of random terms.
You can prove rigorously such a Hamiltonian will typically gapless
and that hte ground state of the Hamiltonian will have exponentially decaying
correaltions.
At each point, you'll have a particle localized to these lcoal potentions,
but if you have a big enoguh system completely gapless.
There'll be some potetential when the ground state has that energy,
but wave functions decay exponnetially so so will be the correlation.


There are also translationally invariant examples,
but these examples tend to be contrived.
I tend to think they are not generic, but there are some examples that violate
the statement.
If you want to learn more I can point you people to learn from.
Any questions?


\begin{question}
    You mentioned sparse Hamiltonians as alternative to local Hamiltonians.
    There's no velocity because there's no distance,
    but is this statement true.
\end{question}
You an definitely ask it something like this.

There's been very little work outside of the SYK and QEC with finite encoding
rates, I don't know people wokring on sparse Hamiltonians.
There's the Swingle version of SYK in htis course.

\begin{question}
    For a gapped hamiltonian, we could genrate a gorund state.
    This theorem assumes a unique ground state, 
    so it assumes it's non-degenerate.
\end{question}
I did mention that this theorem can be generalized to the case of degenerate
ground states,
the case of fermionic operators
and the case of power law decay interactions.

If you want to generalize to degeerate ground states, with the projection into
the ground state.
There is a generalization.


\subsection{Localized Defects}
One thing useful about the existence of a finite $\xi$
is that you can define the notion of a localized defect.
The first case is a 0-dimensional defect.
Add a local potential that decays exponentially from some rate so
\begin{align}
    H = V(|r - r_0|).
\end{align}
That cases some kind of defect.
Outside the region, the state looks pretty much the same as the ground state.

We can define one-dimensional defects.
We add a term that is the sum over a.
A zero-dimensional defect is a point.
We could have 1-dimensional line defects.
\begin{align}
    H + \sum_{r_0\in\gamma} V(|r - r_0|).
\end{align}
I could define $p$-dimensional defects, where $p< d$.
We can talk about defects 

Topology of defects is about understanding what happens how these defects
interact with another when I glue and touch then,
and the phase.

The mathematical structure is not completely known,
but to understand these defects, we use $d$-category theory,
a hyper abstract definition people don't even know how to define yet.
It's a higher category, but mathematicians don't even have a definition.
But if you talk about 1-category or 2-category, then do you have definitions.
The properties of these defects are characterised by this $d$-category.
In this class we will use some simple examples.
This is a story very much in progress, not very fully understood.

\subsection{Quantum Phases}
Quantum really means zero-temperature phases.
I'm going to state what I actually mean by phases of gapped Hamiltonians.
This is the most popular definition.

\begin{itemize}
    \item Two gapped Hamiltonians $H_1$, $H_2$ realize the same $T=0$ (quantum)
        phase of matter if there exists an adiabatic path between them
        such that $\Delta E(\lambda)$ stays non-zero.
        That is, there is a path in Hamiltonian space $H(\lambda)$
        that starts at $H_1$ and ends at $H_2$.
\end{itemize}
The point is that the energy gap stays open as I go from one Hamiltonian to
another.
If I can do such a thing, we say the ground state of the two ground states are
the same phase of matter.
A consequence of this definition is that the ground state of the first
Hamiltonian can be adiabatically connected to the ground state of the other
Hamiltonian.
I can go down this path infinitely slowly that follows the adiabatic path.

There's a different definition that uses the ground state energy density
\begin{itemize}
    \item The ground state energy density $E_0(\lambda)/N$
        should be a smooth function of $\lambda$.
\end{itemize}
Along this path we never encounter a singularity in the ground state energy
density.
This is closer to the usual definition in phases of matter
where I look at the free energy density and a phase transition occurs when the
free energy has some singularity.
Here, we're looking at the ground state energy density.
Aside from the fact i's close to the standard definition,
you could also apply this for a gapless Hamiltonian.
So there's still a notion of whether two gapless Hamiltonians realize the same
phase of matter.

Definition 1 is only useful for gapped states of matter.

\begin{question}
    The second definition is only a distinction between phases of matter right?
\end{question}
This is the definition of when two phases are the same phase.
If they're not the same phase they're different phases.
If it's gapped and there's an energy phase.

I haven't defined what a topological phase is yet.
It's not clear exactly what definite we should use.
People use different definitions,
because there are different concepts.
For now take topological phase as gapped phase of matter.

\begin{question}
    For gapped Hamiltonians, are the two definitions equivalent.
\end{question}
I don't have a proof.

\begin{question}
    For def 1, if w ehave gapped system, if we close the gap, the two ground
    states don't connect adiabatically.
\end{question}
If the gap closes, there's a crossing, I cannot adiabatically make sure it's
the same ground state.
Adiabatic requires separation of scales.
If you get to a level crossing, you can't use the adiabatic theorem anymore,
you need to follow the right path and not get mixed up betwen the states.


\begin{question}
    You cannot ave a smooth function of $\lambda$.
    By smooth do you mean infinitely differentiable.
\end{question}
I would say yes, but that's turning to rigorous mahtematics.


\begin{question}
    Do you require 1D defects to extend infinitely in one dimension?
\end{question}
No. Everything is macrsoopic.
The 1D defect could be the shape of a ring.
But I want ot make sure the length is much larger than the correlatoin lnegth.
It could be a segment.

\begin{question}
    Why don't you consider $p=d$ where there is a cubic defect?
\end{question}
I want every sclae to be either macroscopic or I want it to be\ldots
Let me give an example.
Suppose I have a plane.
I have a defect on half the plane.
If I zoom out and keep this half finite, it becoems a 1D defect.
If I require this is a finite fraction of the total system when I zoom out,
I shouuld really thing about this as its own $d$-dimensional phase of matter,
a bulk phase.

We want to probe a phase by inserting defects.
But if it's a macroscopic part of the system,
it's just another phase, so it's not useful.


\begin{question}
    These defects are supposed to have zero measure?
\end{question}
Yes, in that sense, as you zoom out, the defect should have zero measure.


You could ask about the operator that creates hte defect,
and that could be one higher dimension.
The loop defect is created by a membrane operator.

Looking at a one-imdneiosnal defect in a two-dimensionl defect,
I need to look at operators iwht suppors on the whole two-dimensional space,
but the defect itself is a one-dimensional defect.

\section{Line Integrals}
Under what circumstances is the line integral
$\int_{A}^{B} f(z) \, dz$
independent of the path from $A$ to $B$?

\begin{theorem}[Cauchy]
    If $f(z)$ is analytic in a simply connected,
    bounded domain $D$,
    for every simple closed path $C$ in $D$,
    \begin{align}
        \oint_C f(z) \, dz = 0
        \label{eqn:1}
    \end{align}
\end{theorem}
To prove this, consider
\begin{align}
    \oint f(z)\, dz
    &= \oint_C (u + iv) (dx + i\, dy)\\
    &= \oint_C (u\,dx - v\,dy) +
    i \oint_C (v\,dx + u\,dy)
\end{align}
Then use Stokes theorm
\begin{align}
    \oint\vec{A}\cdot d\vec{r}
    = \int(\vec{\nabla}\times\vec{A})\cdot d\vec{s}.
\end{align}
Let the line integral Lie in hte $xy$ plane.
\begin{align}
    d\vec{r}
    &= dx \hat{e}_x + dy\hat{e}_y\\
    \vec{A} &= A_x \hat{e}_x + A_y \hat{e}_y.
\end{align}
Then
\begin{align}
    \oint_C \left(
        A_x\,dx + A_y\,dy
    \right)
    = \int\left(
    \frac{\partial A_y}{\partial x}
    - \frac{\partial A_x}{\partial y}
    \right)\,dx\,dy
\end{align}
if we go around the path counterclockwise.
Notice this is the same form as before.

Let us apply this to the two line integrals in Equation~\ref{eqn:1}.
For the first integral,
set $A_x = v$ and $A_y = -v$.
Notice that by the Cauchy-Riemann condition,
\begin{align}
    \oint\left(
        u\,dx - v\,dy
    \right)
    =
    \int\left(
    -\frac{\partial v}{\partial x}
    - \frac{\partial u}{\partial y}
    \right)\,dx\,dy
    =0.
\end{align}
For the second integral, set
$A_x = v$ and $A_y = u$.
Then by the other Cauchy-Riemann condition,
\begin{align}
    \oint\left(
        v\,dx + u\,dy
    \right)
    =
    \int\left(
    \frac{\partial u}{\partial x}
    - \frac{\partial v}{\partial y}
    \right)\,dx\,dy
    =0.
\end{align}
Hence
\begin{align}
    \oint_C f(z)\,dz = 0
\end{align}
Suppose $C$ is made up of two paths $C_1$ and $C_2$.
\begin{align}
    \int_{C_1} f(z)\,dz
    - \int_{C_2} f(z)\,dz
    = \oint_C f(z)\,dz
    = 0.
\end{align}
Hence
\begin{align}
    \int_{C_1}f(z)\,dz
    = \int_{C_2}f(z)\,dz
\end{align}
What Cauchy is saying is that this integral is actually a function of the final
point for fixed $A$
\begin{align}
    \int_{A}^{P}f(z)\,dz = F(P).
\end{align}


\section{Cauchy's Integral Formula}
Cauchy is saying if I know the values of an analytic function on the boundary of
a simply connected region in $\mathbf{C}$,
then I can tell you the value of the function anywhere inside the region.

I was shocked when I first saw this theorem.
There's no analogue of this for real functions.

If you consider a complex function
\begin{align}
    f(z) = u(x, y) + i v(x, y).
\end{align}
Then you can solve Laplace's equation in 2D if you know the boundary.
\begin{align}
    \frac{\partial^2 u}{\partial x^2}
     + \frac{\partial^2 v}{\partial y^2}
     = 0.
\end{align}
That's the application we're interested in.

Cauchy's integral formula states that if $f(z)$ is analytic in a simply
connected domain $D$,
then for any point $z=a$ in $D$ and any closed path $C$ in $D$ which encloses
the point $a$,
we have
\begin{align}
    \oint_C \frac{f(z)}{z - a} \, dz = 2\pi i f(a)
\end{align}
where the integration is being taken in the counterclockwise sense.
If you know the values on the boundary then you know the value of any point
inside the region.
The reason it's possible is because $f$ is composed of harmonic functions.
In principle,
if you have the values on the boundary for Laplace's equation,
then in principle you have enough information to solve Laplace's equation in the
bulk.

Define
\begin{align}
    \phi(z) := \frac{f(z)}{z - a}.
\end{align}
This is analytic everywhere in $D$ except at $z=a$.
We need to evaluate
\begin{align}
    \oint_C \phi(z)\, dz.
\end{align}
Let us go counterclockwise.
Then around $a$ draw a small circle,
and I call that circle $C'$.
Then I make a cut between $C$ and $C'$.
Let there be a point $A$ on $C$ and $B$ on $C'$.
Make a cut along $AB$.
Integrate $\phi(z)$ along the closed path from $A$ to $C$ to $B$
around $C'$ and then back to $A$.
The contribution from the cut vanishes since we integrate first in one
direction and then again in the opposite direction.
\begin{align}
    \oint_C \phi(z)\, dz + \oint_{C'}\phi(z)\, dz = 0
\end{align}
where we are going counterclockwise in $C$ and clockwise in $C'$.
Then
\begin{align}
    \oint_{C}\phi(z)\,dz = \oint_{C'}\phi(z)\,dz
\end{align}
when both integrals are performed counterclockwise.
Along the circle $C'$,
we have
$z=a + \rho e^{i\theta}$.
Remember this is a circle and the differential is
\begin{align}
    dz = i\rho e^{i\theta} \,d\theta
\end{align}
So the integral is
\begin{align}
    \oint_{C}\phi(z)\,dz &= \oint_{C'}\phi(z)\,dz\\
    &=
    \oint_{C'}\frac{f(z)}{z - a} \,dz\\
    &= \int_{0}^{2\pi}\frac{f(z)}{\rho e^{i\theta}}\left( 
        i\rho e^{i\theta}
    \right) \,d\theta\\
    &= i\int_{0}^{2\pi} f(z)\, d\theta
\end{align}
but because the circle is so small $f(z)\to f(a)$ and so
\begin{align}
    \oint_C \phi(z)\, dz
    = \oint \frac{f(z)}{z - a}\,dz
    = 2\pi i f(a).
\end{align}

The following is useful for integer $n$.
\begin{align}
    \oint \frac{dz}{{(z - z_0)}^n}
    =
    \begin{cases}
        2\pi i & \text{if } n = 1\\
        0 & \text{otherwise}
    \end{cases}
\end{align}
You can do this by using $z = z_0 + \rho e^{i\theta}$
and doing the integral explicitly on a circle.
If it's not a circle,
use the same trick with the cut like how we proved the Cauchy integral formula.

\section{Liouville's theorem}
Every analytic function is either constant or blows up somewhere in the complex
plane.
\begin{theorem}
    If $f(z)$ is analytic and bounded in absolute value in the entire complex
    plane,
    then it must be a constant.
\end{theorem}
\begin{proof}
    Assume $|f(z)|$ is bounded,
    so that $|f(z)| < M$
    for all $z\in\mathbb{C}$.
    Since $f(z)$ is analytic,
    it can be expanded like
    \begin{align}
        f(z) &= \sum_{n=0}^{\infty} a_n z^{n}.
    \end{align}
    By the Cauchy integral formula,
    \begin{align}
        |a_n|
        &= \left|\frac{1}{2\pi i} \oint \frac{f(z)}{z^{n + 1}}\, dz\right|\\
        &=
        \left|
            \sum_\gamma \frac{1}{2\pi i}
            \frac{f(z)}{z_{\gamma}^{n + 1}}
            \Dlta z_\gamma
        \right|\\
        &\le
        \sum_{\gamma}
        \left|\frac{1}{2\pi i}\right|
        \left|\frac{f(z_\gamma)}{z_\gamma^{n + 1}}\right|
        \left|\Delta z_\gamma \right|\\
        &\le
        \sum_{\gamma}
        \frac{1}{2\pi}
        \frac{f(z)|_{\max}}{R^{n + 1}} |\Delta z_\gamma|\\
        &\le \frac{1}{2\pi} \frac{f(z)|_{\max}}{R^{n + 1}} 2\pi R
    \end{align}
    where the contour is a circle centred at the origin and
    we have used the identities
    \begin{align}
        |ab| &= |a||b|\\
        |a + b| &\le |a| + |b|
    \end{align}
    Hence
    \begin{align}
        |a_n| \le \frac{M}{R^n}
    \end{align}
    for some constant $M$.
    We can take $R$ to be arbitrarily large.
    As $R\to\infty$, $a_n\to 0$ for $n\ge 1$.
    Hence $f(z)=a_0$,
    which is a constant.
\end{proof}

\section{Change of Basis}
If you get lost just \emph{insert the identity}.
Matrix multiplication works like this.
\begin{align}
    \bra{n}AB\ket{m} &=
    \bra{n} A \mathbf{1} B \ket{m}\\
    &= \sum_i \bra{n}A\ket{i} \bra{i} B\ket{m}\\
    &= \sum_i A_{ni} B_{im}
\end{align}
To change basis from $\{\ket{n}\}$ to $\{\ket{\tilde{n}}\}$
just insert the identity.
\begin{align}
    \ket{\psi} &=
    \sum_n \underbrace{\braket{n}{\psi}}_{\psi_n}\ket{n}\\
    &= \sum_n \underbrace{\braket{\tilde{n}}{\psi}}_{\tilde{\psi}_n}
    \ket{\tilde{n}}\\
    &= \sum_n
    \underbrace{\braket{\tilde{n}}{\psi}}_{\tilde{\psi}_n}
    \underbrace{\braket{m}{\tilde{n}}}_{U_{mn}}\\
    &= \sum_n \sum_m U_{mn} \tilde{\psi}_n \ket{m}
\end{align}
The result I get is the following.
\begin{align}
    \psi_n &= 
    \underbrace{U_{mn}}_{\braket{m}{\tilde{n}}}
    \tilde{\psi}_m
\end{align}
Also
\begin{align}
    \underbrace{\braket{\phi}{n}}_{\varphi_n^*}
    = \sum_n
    \underbrace{\braket{\phi}{\tilde{n}'}}_{
    U_{nn'}^* = (U^\dagger)_{n'n}
    }
    = \sum_{n'} \tilde{\phi}_{n'}^* U^\dagger_{n'n}
\end{align}
Let's do an example.
\begin{example}
    Consider functions f(x) on the real line.
    The basis of delta functions is
    $\{\delta(x - x_0)\text{ for all }x_0 \}$.
    The basis of plane waves is
    $\left\{\frac{e^{ikx}}{\sqrt{2\pi}}\text{ for all } k \right\}$.
    Do a change of basis.
\end{example}
\begin{proof}[Solution]
    Insert the identity
    \begin{align}
        \underbrace{\braket{k}{\psi}}_{\mathcal{F}[\psi](k)} =
        \int_{-\infty}^{\infty}dx\,
        \braket{k}{x}
        \braket{x}{\psi}
    \end{align}
    where
    \begin{align}
        \braket{k}{x} &=
        \int_{-\infty}^{\infty} dy \frac{e^{-iky}}{\sqrt{2\pi}}\delta(y - x)\\
        &=
    \end{align}
\end{proof}

\subsection{Eigenthings and Diagonalisation}
Sometimes, if I have the right $\ket{\psi}$ for an operator $A$,
I can get
\begin{align}
    A\ket{\psi} = \lambda\ket{\psi}.
\end{align}
This has a special name.
$\ket{\psi}$ is called the eivenvector with eigenvalue $\lambda$.

Let's do an example.
Suppose I rotate my phone.
Then I rotate it.
What's the eigenvector?
It's the axis of rotation.
What about the eigenvalue?
It's 1 because it doesn't change.

Suppose I stretch my phone along an axis.
What are the eigenvectors?
There are three of them.
Along the axis is an eigenvector, with eigenvalue the scale factor.
The other two are perpendicular, with eigenvalues 1, they are degenerate.

In 2D rotation there are no eigenvectors.

One way of thinking about the plane is a two-dimensional real space.
Then no eigenvalues.

But I can think of it as a one-dimensional complex space,
then multiplying by $i$ is a rotation.
And then I have eigenvalues.

I want to build intuition.

\begin{example}
    Consider the operator $\ket{\phi}\bra{\phi}$.
\end{example}
\begin{proof}[Solution]
    $\ket{\phi}$ is obviously an eigenvector with eigenvalue 1.
    Any other orthogonal vector is also an eigenvector,
    but with eigenvalue 0.
\end{proof}

\begin{example}
    Off diagonal $A=\sum_{n=1}^{N}\ket{n + 1}\bra{n} + \ket{1}\bra{N}$.
\end{example}
\begin{proof}[Solution]
    This is an eigenvector
    \begin{align}
        \ket{1} + \ket{2} + \cdots + \ket{N}
    \end{align}
\end{proof}
It has no importance, just practise.

\subsection{Spectral theorem}
You see spectral lines from the energy levels of atoms.
That's why it's called the spectral theorem.
\begin{theorem}
    Suppose $A$ is hermitian.
    If $A\ket{a} = a\ket{a}$,
    then
    \begin{enumerate}
        \item The $a$'s are all real,
        \item (orthonormality) The eigenvectors $\ket{a}$ can be chosen to be
            orthonormal
            $\braket{a}{b} = \delta_{ab}$,
        \item (completeness) The $a$'s form a complete basis and make a
            resolution of the identity
            $\sum_a \ket{a}\bra{a} = 1$.
    \end{enumerate}
\end{theorem}
\begin{proof}
    Let's see how I can prove it's real.
    It's very simple.
    Start from the definition
    $A\ket{a} = a\ket{a}$.
    Then take
    $\bra{a}A\ket{a} = \bra{a}a\ket{a} = a\braket{a}{a}$
    and since it's a linear operation and $\braket{a}{a}$ is real,
    But since it's hermitian,
    $\bra{a}A\ket{a} = \bra{a}A\ket{a}^*$,
    which means it's real,
    so $a$ is real.

    To prove orthonormality is slightly more complicated.
    Suppose we have
    $A\ket{a} = a\ket{a}$
    and
    $A\ket{b} = b\ket{b}$
    then take
    $\bra{b}A\ket{a} = \bra{b}a\ket{a} = a\braket{b}{a}$
    and
    $\bra{a}A\ket{b} = \bra{a}b\ket{b} = b \braket{a}{b}$.
    Note that $\braket{a}{b} = \braket{b}{a}^*$.
    From this,
    if $a\ne b$,
    then of course
    $\braket{a}{b}=0$.
    But if $a=b$,
    they are not necessarily orthogonal.
    The theorem says that we can then choose them to be orthogonal,
    which is obvious.

    The final one is a bit more difficult to prove.
    For finite $N$,
    we want to show
    \begin{align}
        \sum_{a=1}^{N}\ket{a}\bra{a} = \mathbf{1}.
    \end{align}
    Then
    \begin{align}
        (A - a\mathbf{1})\ket{a}
    \end{align}
    which implies that
    \begin{align}
        \det(A - a\mathbf{1}) = 0.
    \end{align}
    The determinant is a polynomial in $a$ of degree $N$.
    The values of $a$ for which the polynomial is zero is $a$.
    A degree-$N$ polynomial is going to have $N$ zeros,
    and so we should have $N$ eigenvalues.
    I know they are linearly independent from one another.

    But what if $N$ is not finite?
    Or worst continuous?
    That's not something I cannot prove to you.
    We haven't defined things well enough,
    whether the functions are square integrable,
    or whatever.
    I'll have to make every statement precise,
    and we're not doing physics anymore.
\end{proof}

If I see a hunk of metal,
it's a hunk of metal,
so we don't care if it's an infinitely differentiable function or a discrete
system.
It doesn't help me in the lab.
The notes have more examples you may find useful.

\begin{example}
    Consider the Hilbert space that is the set of real functions of one variable
    $f(x)$.
    We have this basis
    \begin{align}
        \left\{\frac{e^{ikx}}{\sqrt{2\pi}}\text{ for all }k\right\}
    \end{align}.
    The operator $-i\frac{d}{dx}$ is a Hermitian operator and these basis
    functions are its eigenvectors.
    You may recognise this as the momentum operator.
    The eigenvalue equation is
    \begin{align}
        -i \frac{d}{dx} f(x) = \lambda f(x)
    \end{align}
    where if you use
    $f_k(x) = e^{ikx}$
    it is satisfied.
    You can instantly recognise that $\lambda = k$.
    But this is not normalized,
    so you have to use
    $f_k(x) = e^{ikx}/\sqrt{2\pi}$.
\end{example}
By the way, the real momentum operator has a $\hbar$ in it like
$\hat{p} = -i\hbar d/dx$
and the functions should be like
\begin{align}
    f_k(x) =
    \frac{e^{ikx/\hbar}}{\sqrt{2\pi}}
\end{align}
and $\lambda = \hbar k$.

Let's do another example.
\begin{example}
    Consider the space of functions $f(x)$ again.
    Then consider the operator
    \begin{align}
        H &=
        - \frac{\hbar^2}{2m} \frac{\partial}{\partial x}
        + \frac{m\omega^2}{2}x^2
    \end{align}
\end{example}
\begin{proof}[Solution]
    The eigenvalues are
    \begin{align}
        \lambda_n &=
        \hbar\omega\left( n + \frac{1}{2} \right)
    \end{align}
    for $n=0,1,\ldots$ and the eigen functions are
    \begin{align}
        \psi_n(x) = \cdots H_n(\cdots x) e^{\cdots -x^2}
    \end{align}
    which you can look up on Wikipedia.
\end{proof}

\section{Residue Theorem}
Today I want to talk about how to evaluate integrals using the residue theorem.

Let $z_0$ be an isolated singular point of $f(z)$.
Consider the value of the closed line integral $\oint_C f(z) dz$
around a simple closed curve $C$ surrounding $z_0$
but enclosing no other singularities.
The little circle means it's a closed line as opposed to an open line.
It means it comes back to where you started.
Let $f(z)$ be expanded in a Laurent series about $z=z_0$ that converges near
$z=z_0$.

Then,
\begin{align}
    f(z) = a_0 + a_1(z - z_0) + \cdots
    + \frac{b_1}{z - z_0} + \frac{b_2}{(z - z_0)^2} + \cdots.
\end{align}

\begin{question}
    What is converging near $z_0$?
\end{question}
It means converging in the neighbourhood around that point.
This is important because you could have several Laurent series about a point
that converge in different regions.
We want the series that converges near the point we're expanding about.

The point of the residue theorem is that there are an infinite number of $b$'s,
but $b_1$ is special.
Out of all the infinite terms, this $b_1$ is special.
This $b_1$ is called the \emph{residue}.
That one is special.

The terms in the $a$ series do not contribute to the integral,
because they are analytic.
The thing is, of the terms in the $b$ series,
only $b_1$ contributes.
This is because of this identity we see again and again.
\begin{align}
    \oint \frac{dz}{\left( z - z_0 \right)^n} =
    \begin{cases}
        2\pi i & \text{if } n =1\\
        0 & \text{otherwise}
    \end{cases}
\end{align}
It's easy to prove with a circle.
Just substitute $z=\rho e^{i\theta}$.
If the curve is not a circle,
first prove arbitrary curves give the same result
using Cauchy's theorem using a cut as we showed before.
Then if you know the results of the circle,
you know the result for arbitrary curves.
Anyway,
it's a homework problem and the solution will be provided at some point.

The point is,
that if you use this integral,
the $b_1$ is the only term that satisfies $n=1$.
All the other $b$s have an $n$ that is bigger than one and doesn't contribute.
The $a$'s don't contribute because they are analytic and Cauchy's theorem tells
you they don't contribute,
and of all the $b$'s only the $b_1$ contributes because of that formula.

So then, therefore
\begin{align}
    \oint_C f(z)\, dz = 2\pi b_1
\end{align}
and this $b_1$ is called the \emph{residue} of $f(z)$ at $z=z_0$.
It's the only term that survives this integration.
Any questions about all this?

So this particular result corresponds to when you have just one singular point
inside the contour $C$.
What if there are half a dozen,
what if there is more than one singularity inside that point?
Then what?

So we have to generalize this result.
We generalize to when there are multiple isolated singularities.
And we will see there is a very simple generalization of this thing.

Now let's say this is the region [picture]
and there are two isolated singular points enclosed by the contour $C$.
Let's call this point $z_1$ and this point $z_2$.
Now what you do is draw a circle around $z_1$
and another circle around $z_2$.
This contour $C$ goes counter clockwise.
Then we do what we did before.
We create a cut like this from $C$ to $C_1$ and a cut from $C$ to $C_2$.
Now the contour is deformed.
Follow the arrows and see you have a simply connected closed loop.

A couple of things I want you to realise.
This region in between $C$ and the circles,
that is simply connected.
If I put a rubber band anywhere,
I can shrink it to a point,
and I can apply Cauchy's theorem.

So around the deformed contour,
we have
\begin{align}
    \oint f(z)\, dz = 0
\end{align}
by Cauchy's theorem,
because we have cut out the singularities and it's analytic.
The contribution to the integral from the cut vanishes.
So then you can convince yourself using the logic very similar to our proof of
the Cauchy integral formula that
\begin{align}
    \oint_C f(z) \, &=
    \oint_{C_1} f(z)\,dz
    + \oint_{C_2} f(z)\, dz
\end{align}
In other words,
the contribution from the cuts vanishes,
but you can see that $C_1$ and $C_2$ are going clockwise,
but if you make everything counterclockwise,
the signs work out.
We are going around $C$, $C_1$ and $C_2$ counterclockwise
in the equation above.

And this generalizes no matter how many singularities you have.
It is the sum of the contour integral around each point.
But we just proved the integral around one singularity is just the residue times
$2\pi i$.
Each circle contains only one singular point
and we can immediately apply the result we just derived.
So it should be $2\pi i$ times the residue of each singular point.
And this is the residue theorem we're going to apply again and again.

So this is my claim.
This leads to the residue theorem
\begin{align}
    \oint_C f(z)\, dz &= 2\pi i\times
    \text{sum of residues of $f(z)$ from singular points inside $C$}
\end{align}
This is the thing that you guys have to remember.

\begin{question}
    That's the closed contour $C$?
\end{question}
Yes it's the closed contour $C$ that contains the isolated points.
Any questions?

The rest of this class and most of the rest of the next class will be giving
examples where we use the residue theorem to evaluate integrals.
This is a place where you use complex analysis a lot.
There are many integrals where the simplest way to do them is to use a contour
integral,
sometimes you know for example,
the most convenient way to express something is as a contour integral.
This result gets used a lot,
especially if you do theoretical physics,
you will see this over and over again.
Let's do some examples.

You should look at all the literature on ELMS
and try to work through the examples there.
Do several of them.
Unfortunately I can't cover the full range of possibilities.
As you'll see,
you have to be clever in most cases,
choose your contour cunningly,
hence it's helpful to do some examples so you can learn how to choose.
But time constraints.

Let's start with some very simple ones.
\begin{example}
    Integrate $f(z) = \sin(z)/z^4$ around the unit circle counterclockwise.
\end{example}
\begin{proof}[solution]
    This potentially has a pole at $z=0$.
    There's nowhere else where it might have a pole.
    The first thing to do is expand this in Taylor series.
    \begin{align}
        \frac{\sin(z)}{z^4} &=
        \frac{1}{z^4}\left\{
            z
            - \frac{z^3}{3!}
            + \frac{z^5}{5!}
            - \cdots
        \right\}
    \end{align}
    What is the order of the pole?
    It's order 3.
    What is the residue?
    It's $-1/6$.
    You can see this
    \begin{align}
        f(z) &=
        \frac{1}{z^3}
        - \frac{1}{6} \frac{1}{z}
        + \frac{z}{5!} - \cdots
    \end{align}
    Don't forget that negative sign,
    the residue is $-\frac{1}{6}$,
    not $\frac{1}{6}$.
    So by the residue theorem,
    \begin{align}
        \oint \frac{\sin z}{z^4}\,dz &=
        2\pi i \left( -\frac{1}{6} \right)\\
        &= -\frac{\pi i}{3}
    \end{align}
\end{proof}
Any questions?
OK let's do another example.

\begin{example}
    Integrate $\frac{4 - 3z}{z^2 - z}$
    around the circle $|z|=2$ counterclockwise.
\end{example}
\begin{proof}[solution]
    This is the circle of radius 2 centred at the origin.
    Look at this and tell me where it has poles.
    $z=0$ and $z=1$ where the denominator vanishes.
    The poles are where the denominator vanishes.
    And $0$ and $1$ are both inside the circle.
    And so we really have to use the residue theorem
    but first we need to find the residue at both $0$ and $1$.
    That's the procedure.

    The function has poles at $z=0$ and $z=1$.
    Near $z=0$,
    \begin{align}
        f(z) &=
        \frac{4 - 3z}{z(z - 1)}\\
        &= \frac{1}{z}\left\{
            \frac{4 - 3z}{z - 1}
        \right\}\\
        &= -\frac{4}{z} + \text{analytic terms near $z=0$}
    \end{align}
    The fraction in the bracket is analytic at $z=0$,
    and it's equal to $-4$.
    What's where it comes from.

    Near $z=1$,
    \begin{align}
        f(z) &=
        \frac{4 - 3z}{z(z - 1)}
        = \frac{1}{z - 1}\left\{
            \frac{4 - 3z}{z}
        \right\}
    \end{align}
    Near $z=1$, the things in the braces are analytic,
    so then the whole thing behaves like
    \begin{align}
        f(z) = \frac{1}{z - 1}
    \end{align}
    and hence $1$ is the residue.
    
    By the residue theorem,
    \begin{align}
        \oint f(z)\, dz &= 2\pi i\sum \text{residues}\\
        &= 2\pi i \left\{ -4 + 1 \right\}\\
        &= -6\pi i.
    \end{align}
    These are just some really simple examples to illustrate.
\end{proof}
Let's suppose this is $|z|=1/2$.
How would this change the answer?

If you draw it,
the circle would then have radius $1/2$,
and $z=1$ is outside the circle,
so then this $z=1$ residue wouldn't have contributed
and the result would be $-8\pi i$.
The boundary is not well defined for $|z|=1$.
You may have to take some limiting procedure,
and the answer may depend on how you take that limiting procedure.

Let's go on to the next problem.
\begin{example}
    Evaluate the definite integral
    \begin{align}
        I = \int_{0}^{2\pi} \frac{d\theta}{5 + 4\cos\theta}.
    \end{align}
\end{example}
\begin{proof}[Solution]
    Strictly speaking,
    you don't need complex analysis to do this integral,
    but it's a tough one.
    You'd have to use
    \begin{align}
        t = \tan\frac{\theta}{2}
    \end{align}
    but you would have had to know it.
    I'll show you it's easy with complex analysis.
    Here's what you do.
    There's all classes of problems where you integrate trig functions.
    The steps are going to be the same.

    Change variables to $z=e^{i\theta}$.
    Then $\theta$ goes from $0$ to $2\pi$
    around the unit circle in the complex plane.
    As $\theta$ goes from $0$ to $2\pi$,
    we go around the unit circle in the complex plane,
    which of course is $|z|=1$.

    So then, you have to change variables from $\theta$ to $z$.
    \begin{align}
        dz = i e^{i\theta} d\theta\qquad
        \implies\qquad
        \frac{1}{i} \frac{dz}{z} = d\theta.
    \end{align}
    And then
    \begin{align}
        \cos\theta = \frac{1}{2}\left(
            z + \frac{1}{z}
        \right).
    \end{align}
    To see this
    just use the Euler formula
    \begin{align}
        e^{i\theta} = \cos\theta + i\sin\theta.
    \end{align}
    So then the problem has reduced to evaluating
    \begin{align}
        I &=
        \oint \frac{1}{iz}
        \frac{1}{5 + 2\left( z + \frac{1}{z} \right)}\,dz\\
        &= \frac{1}{i}\oint \frac{dz}{2z^2 + 5z + 2}.
    \end{align}
    That's what you get when the dust settles.
    Now we have to figure out what the poles of this thing are.
    You have to find the roots first.
    I'm going to skip the step and claim it can be written like this.
    \begin{align}
        I &=
        \frac{1}{i}
        \oint
        \frac{dz}{(2z + 1)(z + 2)}.
    \end{align}
    If I draw this in the complex plane,
    it has a pole here at $z=-1/2$ and then it has a pole here at $z=-2$.
    And we want the contour integral of the unit circle
    centred at $z=0$.
    You can see that only one of the two poles is inside the circle.
    We're going counterclockwise because $\theta$ is increasing this way.
    $\theta$ is increasing from $0$ to $2\pi$ so we're going counterclockwise,
    which is good,
    so we don't have to put an extra sign in
    and just blindly apply the the theorems.

    Only the pole at $z=-1/2$ contributes.
    Now let's calculate the residue.
    Let's look at this
    \begin{align}
        \frac{1}{(2z + 1)(z + 2)} &=
        \frac{1}{z + \frac{1}{2}}\left[ 
            \frac{1}{2(z + 2)}
        \right]\\
        &= \frac{1}{z + \frac{1}{2}}\left[
            \frac{1}{3}
        \right]
    \end{align}
    near $z=-1/2$,
    where we have evaluated the analytic part at $z=-1/2$.
    So $1/3$ is the residue,
    so
    \begin{align}
        I &= \frac{1}{i}2\pi i \frac{1}{3}\\
        &= \frac{2\pi}{3}
    \end{align}
\end{proof}

Any questions?
\begin{question}
    When evaluating the analytic part,
    can we ever just say this region is analytic and just plug in the value of
    $z$.
    Do we have to do the Taylor series?
\end{question}
You don't have to do the Taylor series,
only the first term matters.
This thing is non-zero,
but if it is zero,
the residue is zero.

We have to be a little bit careful.
Suppose if our function had a square in it
\begin{align}
    \frac{1}{(2z + 1)^2(z + 2)}
    = \frac{1}{\left( z + \frac{1}{2} \right)^2}\left[
    \frac{(z + \frac{1}{2})}{2(z + 2)}
    \right]
\end{align}
there is a higher order pole and you have to expand.
In physics problems you typically only have simple poles
and you don't have to expand.

You always want to make it a closed curve.

\begin{question}
    What happens if the integral is not from $0$ to $2\pi$?
\end{question}
Can we try $z=e^{i2\theta}$ for example if it's from $0$ to $\pi$.

\begin{question}
    Can you explain how you'd do it if it were a complex pole?
\end{question}
Let's say instead we had
\begin{align}
    \frac{1}{(2z + 1)^2 (z + 2)}
    = \frac{1}{\left(1 + \frac{1}{2} \right)^2}
    \frac{1}{4\left(z + 2 \right)}
\end{align}
Then you have to expand in a Taylor series,
so the coefficient of the next term in the Taylor series is what gives you the
residue.
In this situation with a higher order pole,
you need more terms in the Taylor series to work it out.

\begin{example}
    Evaluate
    \begin{align}
        I = \int_{-\infty}^{\infty}
        \frac{dx}{1 + x^2}
    \end{align}
\end{example}
\begin{proof}[Solution]
    If you were to use elementary methods,
    you would get
    \begin{align}
        \tan^{-1}(x)|_{-\infty}^{\inftyy}
        = \frac{\pi}{2}
        - \left( -\frac{\pi}{2} \right)
        = \pi.
    \end{align}
    Let's do it using complex analysis,
    just to show you the kind of contours you can integrate with.
    Consider the complex line integral
    \begin{align}
        \lim_{\rho\to\infty}
        \int_{-\rho}^{\rho}
        \frac{dz}{1 + z^2}
    \end{align}
    where the path is along the real line.
    This line integral is equal to $I$.
    Now let's draw this thing in the complex plane.
    Let's look at this integrand.
    It has poles where this $1+z^2$ vanishes.
    The denominator vanishes at $z=+i$ and $z=-i$.
    Right now the fact it has two poles is irrelevant for now.
    Consider this thing integrated not just along the real line,
    but back along a big semicircle in the upper half plane.

    Consider
    \begin{align}
        \oint \frac{dz}{1 + z^2}
    \end{align}
    integrated over the closed contour shown in the figure,
    which consists of a semicircle over the upper half plane,
    in addition to the integral over real line.

    This closed line integral we can evaluate using the residue theorem.
    Even though it has two poles,
    only $z=+i$ is inside the contour.
    Let's evalaute it.
    Close to $z=i$,
    \begin{align}
        \frac{1}{1 + z^2} = \frac{1}{(z + i)(z - i)}
        = \frac{1}{z - 1} \underbrace{\frac{1}{2i}}_{\text{residue}}
        + \text{analytic function}
    \end{align}
    Hence by the residue theorem, the contour integral is
    \begin{align}
        \oint \frac{dz}{1 + z^2} &=
        2\pi i \frac{1}{2i} = \pi.
    \end{align}
    We have evaluated this integral not over the real line,
    but over this closed contour which includes the real line,
    but also includes the contribution from this big semicircle.
    I'm going to show you the contribution from the big semicircle vanishes as
    you take $\rho\to\infty$.
    And so this closed integral is actually equal to the integral we want.
    That's going to be the logic here.

    We need to prove the line integral over the semicircle vanishes and then
    we're done.
    Let's prove that.
    \begin{align}
        \underbrace{\oint \frac{dz}{1 + z^2}}_{\pi} &=
        \underbrace{\int_{\text{real line}} \frac{dz}{1 + z^2}}_{I}
        + \underbrace{\int_{\text{semicircle}} \frac{dz}{1 + z^2}}_{?}
    \end{align}
    Set $z=\rho e^{i\theta}$.
    Then consider
    \begin{align}
        \int_{\text{semicircle}} \frac{dz}{1 + z^2} &=
        \int_{0}^{\pi} \frac{i\rho e^{i\theta}}{1 + \rho^2 e^{2i\theta}}\,
        d\theta
    \end{align}
    This has $\rho$ in the numerator and $\rho^2$ in the denominator,
    this should go to zero as $\rho\to\infty$.
    But you still have to be careful because it's complex,
    but you'll see this intuition is correct.
    It's just going to vanish.
    \begin{align}
        \left|
            \int_{\text{semicircle}} \frac{dz}{1 + z^2}
        \right|
        &=
        \left|
            \int_{0}^{\pi} \frac{i\rho e^{i\theta}}{1 + \rho^2 e^{2i\theta}}\,
            d\theta\\
        \right|
        &\le
        \int_{0}^{\pi}
        \left|
            \frac{i\rho e^{i\theta}}{1 + \rho^2 e^{2i\theta}}
        \right|
        \left|
            d\theta
        \right|\\
        &\le
        \int_{0}^{\pi}
        \frac{\rho}{\rho^2} |d\theta|\\
        &=
        \frac{\pi}{\rho}\\
        &\to 0
    \end{align}
    as $\rho\to\infty$.
    So in the end the contribution of this semicircle vanishes.
\end{proof}

This shows you can do an integral over the real line by doing an integral like
this with a contour like this.
This is one of the very standard contours where you have an integral over the
real line and make it a contour by closing the upper half plane,
or lower half plane.

\begin{question}
    What if you took the semicircle in the lower half plane?
\end{question}
Everything would work, but you're not going counterclockwise and you have to
deal with signs.
It still works though.
There's more than one way to do these problem.
Choose the most convenient one.
As you do more provlems, you'll see.
I want you guys to convince yourselves that it can be closed in the negative
half plane and get the same result.

This contour works for this function works for a lot of functions,
but it also doesn't work for a lot of functions.
You could have used either up or down in this problem,
but we're going to look at at problem where you have to use one of them.
It's actually more common.

\begin{example}
    Evalaute
    \begin{align}
        I = \int_{0}^{\infty} \frac{\cos(z)}{1 + x^2}\,dx
    \end{align}
\end{example}
\begin{proof}[Solution]
    Notice that it's similar to before.
    Notice that the integrand is even, so this is equal to
    \begin{align}
        I &=
        \frac{1}{2}\int_{-\infty}^{\infty}
        \frac{\cos(x)}{1 + x^2}\,dx.
    \end{align}
    It's an even function so we can be a bit clever.
    We want to make this an integral over some line in the complex plane.
    There's more than one way to do this,
    but here's the trick.
    Write $z = \Re e^{iz}$.
    Then write
    \begin{align}
        I &= \re\left\{
        \frac{1}{2}
        \int_{-\infty}^{\infty}
        \frac{e^{iz}}{1 + z^2}\,
        dz
        \right\}
    \end{align}
    where the integral is over the real line in the complex plane.
    Now it's beginning to look a lot like the problem we just saw.
    You have a function integrating over the whole real line.
    It has singularities at exactly the same points $z=\pm i$.
    But can we use the same contour?

    The integrand is a bit different.
    Can we convince ourseves that the contribution from this semicircle
    vanishes?
    If it does, then we can use the same contour.
    It turns out it does vanish.

    Along that big semicircle, $z$ has a real and imaginary part
    \begin{align}
        z = \rho_R + i\rho_I
    \end{align}
    Unless you're at $+\rho$ or $-\rho$,
    the imaginary part is infinite.
    If the imaginary part is positive and infinite,
    $e^{iz}$ because $e$ raised to minus infinity,
    so it goes to zero exponentially quickly
    if you're in the upper half plane.
    That means this integrand vanishes incredibly quickly in the upper half
    plane.
    So we can use exactly the same contour as we used before
    and use the residue theorem.
    At the same time,
    we can't use a contour with a semicricle in the lower half plane,
    because it blows up incredibly quickly.
    Last time we had a choice of contours up or down,
    but here we don't have a choice,
    at least we don't have that choice,
    you have to choose to go in the upper half.
\end{proof}

\section{Anomalies}
Last time we mentioned the types of TQFTs.
There's one more TQFT that you might encounter.
There's something called an extended TQFT.\@
When I define TQFT,
I assign a path integral to a $(d+1)$-dimensional manifold
and we have a path integral
$Z\left( M^{d+1} \right)\in \mathbb{C}$.

For a closed $d$-manifold $V(\Sigma^d)$.

For a closed $(d-1)$-manifold,
we can associate a $1$-category.
This is called once-extended.
The objects get more and more abstract as you go down.

And then for $(d - 2)$-manifold,
you can attach a 2-category.
For a $(d-3)$-manifold,
you attach a 3-category.

I'm not going to define these.
And then for a 0-manifold,
you get a $d$-category.

Useful to note that if you have a fully extended TQFT,
all the way down to $d$-category,
this is going to come up for us,
we won't necessarily use it,
but we will encounter them often.
Fully extended TQFTs are situations where you can write down exactly solvable
Hamiltonians for your topological phases as sums of commuting projectors.
And then the path integral can be written as an exact combinatorial state sum.

We'll encounter examples eventually.
To define the path integral,
triangular the manifold
define simplexes 1 2 3,
assign labellings,
and them sum over all labellings.
That's an exact combinatorial space.
They're useful to work with and have a lot to say about them.
Physicists don't have a good understanding of what these higher structures mean
physically,
so it's an interesting question.


\section{Invertible vs non-invertible TQFT}
The next distinction is between invertible and non-invertible TQFTs.

\begin{definition}
    A TQFT is \emph{invertible} if the magnitude of the path integral has unit
    magnitude
    \begin{align}
        |Z(M^{d+1})| = 1
    \end{align}
    on any closed manifold.
\end{definition}
In particular, if you look at $Z(\Sigma^d\times S^1) = 1
= |\dim V(\Sigma^d)|$.
This is interpreted as the dimension of the Hilbert space on this space
Physically,
this means that there is always a unique state on any closed $\Sigma^d$.
A topological state of matter described by an invertible TQFT always has a
unique ground state on any closed manifold.
Not always the case,
depending on the topology of the space,
and those will be described by non-invertible TQFTs.

The reason it's called invertible,
is that if the path integral is a phase,
there's always a well-defined inverse of the phase,
which is the complex-conjugate of the phase.

Let's say if you have a TQFT,
then the ``inverse'' TQFT has a path integral like
\begin{align}
    Z_{T^{-1}}\left( M^{d+1} \right)
    =
    Z_T^*\left( M^{d+1} \right)
\end{align}
Invertible TQFTs come up a lot.
They describe integer quantum hall states.
Topological insulators and superconductors.
All $(1+1)$-dimensional topological phases of matter are describable by
invertible TQFTs.
Any TQFT that is not invertible is a \emph{non-invertible} TQFT.\@

Actually, one more thing before I get to that.
Invertible TQFTs form an Abelian group.
Suppose that I have the path integral for two theories $T_1$ and $T_2$,
and I want the path integral for $T_1\times T_2$,
then I just multiply their path integrals.
\begin{align}
    Z_{T_1T_2}\left( M^{d+1} \right)
    =
    Z_{T_1}\left( M^{d+1} \right)
    Z_{T_2}\left( M^{d+1} \right)
\end{align}

\begin{question}
    Does the partition function define the TQFT uniquely?
\end{question}
Well $V(\Sigma^d)$ is always going to be a one-dimensional vector space.
Even if it has some boundaries,
it's just going to be a number,
a multiple of that state,
so just an amplitude.
We can extend that definition to the full TQFT.
I'm specifically talking about the invertible ones.

Even in $M^{d+1}$ is open,
the state on the boundary is one-dimensional,
I can just take the inverse theory to give me the complex conjugate.

If a theory is not invertible,
it is a non-invertible TQFT,
and it will have a non-trivial degeneracy on different manifolds.
One thing that is true,
but I'm not sure if there is a proof,
I doubt there's a counterexample,
but I think that this always goes the other way.
There's also an arrow going the other way.
If the path integral magnitude is 1,
then every path integral on the manifold is 1.
I've seen the proof for once-extended,
but I'm not sure if it's general.

So a non-invertible TQFT should have
$Z|\left( \Sigma^d\times S^1 \right)\ne 1|$
for some $\Sigma^d$.
Using ``should'' loosely here.

Let me give you a quick example of an invertible TQFT.\@
This is actually going to be an interesting example,
because it's going to show you a TQFT that is non-trivial,
but it's going to describe a trivial phase of matter.
Imagine we take our path integral on a two-manifold
and have some $U(1)$ gauge field $A$ with field strength $F$.
Then
\begin{align}
    Z\left( M^d, A \right)
    =
    e^{-\theta \int_{M^2} F}
    e^{i\phi \chi\left( M^d \right)}
\end{align}
where
\begin{align}
    \chi\left( M^d \right) = \frac{1}{2\pi}\int_{M^2}R
\end{align}
is the Euler characteristic.

Phases of matter are 1 to 1 correspondence with deformation classes of TQFTs.
In the absence of pinned phases of $\phi$,
this is an example of an invertible TQFT.\@

The first most important question studying a new phase of matter is to ask if
it's describable by an invertible TQFT or a non-invertible TQFT.\@

\section{'t Hooft anomaly}
One big concept about TQFTs I want to mention are called
\emph{'t Hooft anomaly}.
This is something that applies a general QFT,
not just TQFT.
We can consider a QFT which has a symmetry group $G$.
This is the global symmetry group of the QFT.
This may be completely well-defined in the absence of any background gauge field
associated with $G$i.
Then you try to turn on a background gauge field and see something is wrong
with the theory.
Consider your path integral $Z(M^{d+1})$ and turn on $A$,
and you find that the path integral is not gauge-invariant.
\begin{align}
    Z\left( M^{d+1}, A + d\lambda \right)
    \ne
    Z\left( M^{d+1}, A \right)
\end{align}
when $A\ne 0$.
It's not the end of the world.
You might be able to redefine the path integral and make it gauge invariant.
You may be able to define a new gauge-invariant path integral
\begin{align}
    \tilde{Z}\left( M^{d+1}, A \right)
    = Z\left( M^{d+1}, A \right)
    e^{i\int_{M^{d+1}} f(A)}
\end{align}
where the $e^{i\int_{M^{d+1}} f(A)}$
is called a local counter-term that cures the gauge invariance.

\begin{question}
    You mean the action of $G$?
\end{question}
You look at all the correlation functions and you're applying $G$ to all the
operators.
The ground state of the QFT is invariant under $G$.

\begin{question}
    When you couple the background gauge field,
    should we be doing minimal coupling?
\end{question}
It depends how you define this quantity.
Suppose you have some action that is a single Dirac fermion with a mass.
You integrate out the mass
and find a Chern-Simons term with some action.
You wouldn't have written this term down anyway if you were experineced.
But you could actually just add another local functional to this,
and now the coefficients add up to 1 and they are gauge invairant.
If you're not following this discussion it's fine.
I haven't introduced Chern-Simons theory yet.
But you're right.
Usually, you would have defined your path integral in the first place.

\begin{question}
    Is it an anomaly in global symmetry?
\end{question}
Yes. This is the only anomaly worth having a name.
Gauge anomaly means you have a 't Hooft anomaly you try to gauge.
You may not always be able to add counter terms.
I haven't defined 't Hooft anomaly yet.

My point is that a theory with a 't Hooft anomaly is a real theory.
A theory with a gauge anomaly is a theory that doesn't make sense in the first
place.

There's something called a framing anomaly.
There's gravitational anomalies.
Anomaly is an overloaded term in physics,
used everywhere for different things.

There's no such thing as a gauge theory in QM.
It's a classical redundancy in your description.
A theory can have a global symmetry that is anomalies.
That is a 't Hooft anomaly.

You write down a Lagrangian with classical gauge symmetry.
You quantize it and find the symmetry does not hold when quantized,
and that's an anomaly.
There's a whole story.

There could be situations where you can't cancel this gauge apology by adding
counter terms.
You can cancel it not with a local counter term,
but thinking your theory as the surface of a higher dimensional invertible
TQFT.

So,
we can have a situation where the gauge non-invariance is cancelled by a
$(d+2)$-dimensional invertible TQFT.
An invertible TQFT that exists in a 1-higher dimension.
You can think of your system as living on the $(d+1)$-dimension surface of a
$(d+2)$-dimensional bulk and this bulk is an invertible TQFT.
And then you can have some bulk theory
$Z_{\text{bulk}}\left( W^{d+2}, A \right)|_{\text{boundary cond.}}
= Z\left( \partial W^{d+2}, A|_{\partial W} \right)$
perfectly well-defined,
but if you evaluate with some boundary condition.
And the gauge transformation would be governed by the higher-dimensional theory.
So
\begin{align}
    Z\left( M^{d+1}, A + d\lambda \right)
    Z\left( M^{d+1}, A \right)
    = Z_{\text{bulk}}\left( M^{d+1}\times I, \tilde{A} \right)
\end{align}
So the amplitude of going from $A$ to $A + d\lambda$,
there's an extra phase that comes from the invertible TQFT in the bulk,
which cancels out the gauge non-invariance here,
but cancelling in a way that cannot be done by just a local counterterm.

So then 't Hooft anomalies can be classified by TQFTs.

\begin{question}
    Why only just one higher dimension?
\end{question}
If you have to go to 2 higher dimensions,
you can always compactify one of te dimensions
and then you're back to $+1$ extra dimensions.

Let me just explicitly write it out

A QFT has a 't Hooft anomaly if
\begin{enumerate}
    \item It has a global symmetry $G$
    \item The path integral $Z$ is not gauge invariant under gauge
        transformaiton $A\to A + d\lambda$
    \item Gauge non-invariance cannot be cancelled by a local counter-term,
        but it can be cancelled by a $(d+1)$-dimensional TQFT.
\end{enumerate}
In topologcial pahses of matter,
some are described at low energy witha 't Hooft anolmaly
and what that tells us is that it cannot be realised ni $(d+1)$ spacetime
dimensions,
but has to be the surface of a higher dimensional theory.
In condensed matter,
a theory witha 't Hooft anomaly can only be raelised microscopically
and have the symmetry act on-site,
it arises as the $(d+1)$-dimensional surface of a $(d+2)$-dimensional
topological phase.

On-site means the symmetry means a tensor product of on-site symmetry opreators
separately.
That is, 
$g=\bigotimes_i g^{(i)}$.

The symmetry group dictates that the gauge field is a connetion on a principle
$G$-bundle,
so it's a gague field of global symmetry group $G$,
so it controls what gauge transfomatinos you have.
What I'm saying is
your therory may be defined for correlation functions without $A$,
but when you turn on the $A$ and things go wrong.
Let me just mention a few examples of 't Hooft anomalies in case you may have
seen examples before.

If you haven't seem examples before,
these examples will come up in the course.

Famous example.
You can have a $(1+1)$-dimensional chiral fermion
with $U(1)$ symmetry.
It can only exist in the boundary of an integer quantum Hall state,
which is a TQFT.

Another example.
You can have a single $(2+1)$-dimensional massless Dirac fermion
with $U(1)$ and time-reversal symmetry.
But this suffers from the famous parity anomaly.
So htis can only exist at the surface of a $(3+1)$-dimensional topological
insulator.

In particular physics,
there's a famous axial anomaly,
which is an example of a 't Hooft anomaly.

These are the most famous ones.

\begin{question}
    Are massless Dirac fermions topological?
\end{question}
No,
I'm just saying the ntion of a 't Hooft anomaly applies to QFTs in general.
These examples are not topological.

\begin{question}
    Gauge-gravity duality?
\end{question}
It's related in an either loose way or such a profound way that I don't have an
aswer.
Definitely related,
but it's hard to yeah.
It's related,
but it's one of things things,
as time goes on
we're seeing more and more hints.
At the surface it seems pretty good.

\begin{question} 
    Is there some version for CFTs of diffe.rent dimensions?
\end{question}
A CFT can certainly have a 't Hooft anomaly.
They must live at the surface of a higher-dimensional TQFT.
But a higher-dimensiona CFT doesn't really make sense,
because the boundary theory only makes snese
if there are no additional degrees of freedom in the bulk.
If the bulk has degrees of freedom,
there can be leaking of these.
Here the boundary has degrees of freemdom that cannot leak into the bulk.

\begin{question}
    Why should the bulk by invertible?
\end{question}
There's two answers.
People do talk abbout non-invertible anomalies.
There are examples,
but it suffers from the thing I was describing.
If your bulk is non-inverible,
it means there are non-trivial degrees of freedom that are also non-trivial i
hte bulk,
they can leak out and there is no clean separation between the bluk and the
boundary.

An integer quantum Hall state is well-defined,
but the boundary mode cannot exist on its own.

We can open up the bulk and have it open up the space and there is a boundary
theory there.
.
\begin{question}
    So you have abulk theorey
    You restrict it to the boundary,
    you evaluate it on hte boundary?
\end{question}
It's better to htink I have a bublk theory,
and if the manifold has a boundary,
then I will specify some boundary conditions.

\begin{question}
    When you specify the boundary condition,
    what do you mean exactly?
\end{question}
It's only when I specify boundary conditions that I can get a number $Z$.

What is the bulk?
$Z_{\text{bulk}}\left( W^{d+2} \right)\in V(\partial W^{d+2})$
is just a state,
which I take an inner product with something to specify the boundary condition.

It might be more useful that we go to examples later on.
This discussion is a bit abstract.
Different theories may correspond to different boundary conditions.
In all cases, the boundary theory will have a 't Hooft anomaly.

\begin{question}
    TFQT exist on the boundary of a therory in the sense you want to put it on a
    lattice?
    If we don't put it on a lattice and don't couple it to a gauge field?
\end{question}
Call that a macroscopic realisation.

A single fermion is not emergable from a one-dimensional system.

\begin{question}
    The Standard model is not emergable?
\end{question}
If a theory has a 't Hooft anomaly,
if you're willing to give up the symmetry,
you can realise it microscopically.
If you dedmanded you wanted to preserve axial symmetry,
you have to go to the surface of a one-higher dimensino.

In the standard mdoel,
there is actually 2 anomoalies.
There's a gauge $U(1)$ anomaly for chiral fermions,
and there's a gravitaitonal anomaly,
that you cannot give up.
It's tied to energy enregy conservation.
I don't even know what that means,
because then you have a time-dependent Hamiltonian.

I think we're getting ahead of ourselves with this example.

\begin{question}
    In a 1D condensed matter system,
    in a Luttinger liquid for example?
\end{question}
That anomaly you can think of as an axial anomaly in 1+1 dimensinos.
The theory emerges out ofa sitautiaon where it doesn't exist microscopically.

If there are no more questions about TQFTs,
we're going to talk about actual topological quantum phases of matter.

\section{1D topological superconductor}
This is describable by an invertible spin TQFT.

We're going to start off with a $(1+1)$-dimensional
Majorana-Kitaev chain.
Start with a simple toy model that is the 1d spineless $p$-wave superconductor.
That's going to be described by the following.
Imagine you have 1D array of $N$ sites.
WRite the free fermion Hamiltnoia,
which you can think of as a free field for the semiconductor.

\begin{align}
    H &=
    \sum_{i=1}^{N}
    \left[ 
    -t\left( c_i^\dagger c_{i+1} + c_{i+1}^\dagger c_i\right)
    - \mu\left( c_i^\dagger c_i - \frac{1}{2} \right)
    + \Delta c_i c_{i+1}
    + \Delta^* \chi_{i+1}^\dagger c_i
    \right]
\end{align}
and consider periodic boundary conditions $c_{i+N}=c_i$.

Consider momentum space.
\begin{align}
    \tilde{c}_k &=
    \frac{1}{\sqrt{N}}
    \sum_{j=1}^{N}
    c_j e^{ik_j}
\end{align}
where $k = 2\pi n/N$,
$n=0,\ldots,N$.
The Broullin zone is $[0,2\pi] \simeq [-\pi, \pi].
$\Delta is just any compelx number here.

Define this spinner with some redundnacy.
\begin{align}
    \Phi_k =
    \begin{pmatrix}
        c_k\\
        c_{-k}^\dagger
    \end{pmatrix}
\end{align}
then the Hamiltonian is
\begin{align}
    H &=
    \frac{1}{2}\sum_{k\in BZ}
    \Psi_k^\dagger \mathcal{H}_k \Psi_k
\end{align}
So then the Bloch Hamiltonian is
\begin{align}
    \mathcal{H}_k =
    \begin{pmatrix}
        \mathcal{E}_k & \Delta_k^*\\
        \Delta_k & -\mathcal{E}_k
    \end{pmatrix}
\end{align}
where
\begin{align}
    \mathcal{E}_k &= -2t\cos k - \nu\\
    \Delta_k &= -2i\Delta\sin k
\end{align}
The dispersion relation is
\begin{align}
    E_k &= \pm
    \sqrt\left\{ \mathcal{E}_k^2 + |\Delta_k|^2 \right\}\\
    &= \pm
    \sqrt{
        (\mu + 2t\cos k)^2
        + 4|\Delta|^2\sin^2 k
    }
\end{align}2
$E_k$ is always gapped away from $k=0,\pi$.

If $\mu=-2t$,
then it is gapless at $k=0$.
If $\mu=2t$,
then it is gapless $k=\pi$

Let's draw some pictures.

[pictures]
It's called p-wave because $\Delta_k$ is linear in $k$ for small $k$.

If the Hampton is
\begin{align}
    \mathcal{H}_k &=
    h_k^0 \mathbf{1}
    + \vec{h}_k \cdot \vec{\sigma}
\end{align}
Then the combo.
\begin{align}
    h_k^x = -h_{-k}^x
    \qquad
    h_y^y = -h_{-k}^y
    \qquad
    h_k^z = h_-zk
\end{align}

If we know $\vec{h}_k$ from $[0,\pi]$,
then we know is everywhere.
\begin{align}
    \hat{h}_{k=0} &= S_0 \hat^{z}\\
    \hat{h}_{k=\pi} &= \underbrace{S_{\pi}}_{\pm 1}\hat{Z}
\end{align}
depedning on the sign of $\mathcal{E}_\pi$.
Define $\nu = S_0 S_\pi$.

The two spaces $S_0 = S_{\pi}=2$
and $S_0 = -S_{T}$.
They are distinct so long as the energy gap stays open so $h_k$ is well-defined.

[picture]

Everything is region is topological.
Everything outside is topological.
As we tune the chemical potentitoni through kinetic energy,
if we have an odd number of Fermi points,
they we're on a topological phase.

\section{Lecture 7}
37 min late, raining day.

\begin{align}
    \bra{k} U(t) \ket{k'} =  \ket{k} e^{-E_{k'}t/hbar}\ket{k'}\\
    &= e^{-E_{k'}t/hbar}\delta\left(k - k' \right)
\end{align}
and the time-evolution operator is
\begin{align}
    U(t) = \int_{-\infty}^{\infty} e^{-iE_k t/\hbar}\ket{k}\bra{k}
\end{align}
where $\psi(x) = \braket{x}{\psi}$.
Now to work it out in the position basis,
\begin{align}
    \bra{x'}U(t)\ket{\psi(0)}
    &= \bra{x'}
    \int dx\, U(t)
    \ket{x}\braket{x}{\psi(0)}\\
    &= \int_{-\infty}^{\infty}dx
    \int_{-\infty}^{\infty}dk\,
    e^{-iE_k t/\hbar}
    \underbrace{\braket{x'}{k}}_{\frac{e^{ikx'}}{\sqrt{2\pi}}}
    \underbrace{\braket{k}{x}}_{\frac{e^{ikx}}{\sqrt{2\pi}}}
    \psi(x)
\end{align}
and $E_k = \hbar^2 k^2/2m$.
So what we have is a Gaussian integral.
\begin{align}
    \bra{x'}U(t)\ket{\psi(0)}
    = \int_{-\infty}^{\infty} dx\,
    \frac{e^{\frac{i m(x - x')^2}{2\hbar t}}}{\sqrt{2\pi i t\hbar/m}}
\end{align}
You guys are physicists,
you should know what this physically means.

Suppose I start off with the wave function concentrated.
But over time it should diffuse.

Also, there are factors of $i$ here,
but it's the magnitude that matters.
There are real and imaginary parts that oscillate.

Imagine,
there is a certain probability moving to the right.
What happens to the probably of the particle moving to the right later on?
It's the same,
because the energy is the same.

The component overlap of the wave function never changes,
only the phase changes.
You don't know what the momentum is but it never changes.

I promise,
this class I do something trivial and boring,
and I blow your mind the next minute.
I promise I'll blow your mind after the break.
I'm tired now.
We'll start again at 11:00.


\section{Quantum Zeno Effect}
The hare and the tortoise.
The hare is faster.
Rabbit goes ahead, rabbit will win.
But a Greek philosopher said no.
By the time the rabbit reaches the position of the turtle,
the turtle has moved a bit.
By the time the rabbit reaches where the turtle is,
the turtle has moved ahead.
etc.
So the rabbit will never take over the tutle.

Greeks are not stupid.
They understood what's wrong.

The true argument is that with words you can prove anything you want.
Philosophers.
Silly stuff, they stopped doing this.
You can manipualate works to do whatever.
I teach classes on how to convince peopl to do stuff.
These people are sohpists.
They have quite afraction.
Some say words, who cares, you prove anything you want.

That was the old Zeno paradox.

In QM hter's soemthign slighlty simlalr to that.
This is correct as opposed to the Zeno argument,
and we call it the Zeno efefect.

Suppose some QM system stars in some initail state $\ket{i}$.
It evolves in time like
\begin{align}
    e^{i Ht/\hbar}\ket{i}
\end{align}
and the vector may chnage.
What's the probability of finding some finite state?
\begin{align}
    \bra{f}e^{i Ht/\hbar}\ket{i}
\end{align}
This is the amplitude of finding it in state $f$ later after some time.
Probability si squred.
If the evolution is very small, I can approximate
\begin{align}
    \bra{f}e^{i Ht/\hbar}\ket{i}
    = \bra{f}
    1 - \frac{i}{\hbar} Ht
    \ket{i}
\end{align}
for small $t$.
Calculate what the probability is to transition $i\to f$.
For simplicity,
assume the final state is very different,
orthogonal to each other
\begin{align}
    \braket{f}{i} = 0
\end{align}
So then
\begin{align}
    \bra{f}e^{i Ht/\hbar}\ket{i}
    = -\frac{i}{\hbar}t \bra{f}H\ket{i}
\end{align}
and it's linear in $t$, which is interesting.
Then to get the probability I need to square this.
\begin{align}
    P(i\to t) = 
    \frac{|\bra{f} H\ket{i}|^2}{\hbar^2}t^2
\end{align}
which is quadratic in $t$.
The probability of something happening is quadratic ni time,
not linaer.
Let me do an expeirment.
Start with state $\ket{i}$,
evolve with time $\delta t$.
I look at the system,
that counts as a measuremnet.
There's a probabiiyt of $i$ anda probability of $f$.
Suppose I get $i$ again.
The initail state gets reset,
collapse.
Then wait another $\delta t$.
I measure again,
say I get $i$ again.
Collapse.
I do this many times.
I do this many times,
every time there's a bit probabiliyt of $i$
and a small probabliyt of getting $f$.

Suppose instead the probability of being in time $i$ after time $\Delta t$ is
\begin{align}
    P(i \to_{\Delta t} i) = 1 - # \Delta t
\end{align}
That's thep robabiliyt of getting $i$ after a step.
But then I do it again, and I want both things to happen,
and the probability is the product
\begin{align}
    P(i \to_{2\Delta t} i) = (1 - # \Delta t)^2
\end{align}
And 3 times
\begin{align}
    P(i \to_{4\Delta t} i) = (1 - # \Delta t)^2
\end{align}
And after many many times, say $t/\Delta t$.
$t$ is the whole expeirment,
divided up into a lot of expeirments separated by $\Delta t$.
This number is going to be
\begin{align}
    P(i \to_{t} i) = \left(1 = # t \right)^{t/\Delta t}
\end{align}
and then I take the limit $\Delta t\to 0$.
Do you know what this limit is?
It is an exponential.
This is how you learn about the exponential function in high school.
It's something to expect.
If there is something unlikely to happen,
it's exponential like this.
The probabliity of staying alive is a decaying exponential.
If you wait a billion years,
I'm guaranteed to get a piano hit my head.
That's how radioactive decay happens.
There's a proabiliyt of having a decay and not decaying happens exponential over
time.
That's what happens when the probability of decaying is linear in $\Delta t$.

But the probability is not linear,
actually quadratic,
and this limit is actually
\begin{align}
    P(i \to_{t} i) = \left(1 = # t^2 \right)^{t/\Delta t} \to 1
\end{align}

So ``a watched pot will never boil''.
Quotation marks of course,
because this is a joke.
People got so excitied,
they tried to have an atom decaying,
expected exponentail decay,
but if you look at it,
it doesn't decay.
But then are you really watching it every second?
Does it count as a measurment?
Do I have to be tere,
does a grad student have to be there.
Every time peole naylse this system,
if you do a continuous measurement,
yes it does't decay.
Then you have to face the question of what is a measurement or not.

In fact,
I already argued that if you wait long enough,
you're going to go back to where you started.
In every experiment you do in normal life,
the decay is proportional in time.
You're going to go back to where you start in the beginning,
but in 99.99\% of situations,
you see the normal thing where the probability of something to happen is
proportional to the time it's going to happen.
How do I understand?
It's sublte.
The poor soul teaching you in spring is going to explain wiht perturbation
theory.

When they talk about Fermi's golden rule,
say no,
it should be quadratic in time,
how come it's not quadratic?
I want you to really think about this.
It's really subtle.

Suppose I do an experiment.
I have to measure constantly.
Someone says,
oh wlel to observe constantly,
you have to do this this and taht,
and with that of course the state is not going to change.
But if you're a little empty,
think you acomplished something by me yee.
Take a radioatcitve atome,
I can stop it just by looking at it.
By setting up sometihng that I can observe,
it is a little surprised that it doesn't decay.
You loook sceptical,
I'll dig some papers.

You'll see why it looks linear,
that you'll learn next semester.
Nag hte professor, don't mention my name by the way.

\section{Uncertainty principle}
Jumping from topic to topic.
So far,
it's generic things about quantum mechancis,
not about relativistic QM or particles on a line,
just completely generic things about QM.

Let's talk about the uncertainty principle.
You should know it.
You should know how to prove it.
You should know how to generalise it.
And you should know how not to misuse it.

Heisenberg stated the uncertainty principle,
but 80\% of what he said about it is wrong.

Let's just prove the rslut wiht standard linear algebra and talk about
consequencs.
By the way,
did I define or prove the expectation value
\begin{align}
    \bar{A} = \bra{\psi} A\ket{\psi}
\end{align}
It's just normal probability,
weighted.

For example, if there's a 90\% probability of getting 1,
and 10\% probabilty of getting 0,
the expectation value is 0.9.

By the way, with games,
this is loterry.
Tiny probability.
This is working yup'rure sure to get a little negative.
For example,
study physics and you have a right probability of losing money.
Gains and losses.

I debated myself about what notation to use.
$\langle A\rangle$ is used for expectation,
and in fact it motivated the braket notation,
but then when pepole see this,
they ask where is hte bra?
Where is the bra?
So I just write $\bar{A}$.

How do I prove this?
Well insert the identity.
\begin{align}
    \bar{A} = \bra{\psi} A\ket{\psi}
    = \sum_{m,n} \braket{\psi}{m}\underbrace{\bra{m} 
    \underbrace{A \ket{n}}_{a_n\ket{n}}}_{a_n\delta_{mn}} \braket{n}{\psi}
    = \sum_{n} a_n \underbrace{|\bra{n}{\psi}|^2}_{\text{prob of }a_n}
\end{align}
where $n,m$ are eigenstates f $A$.
So that's why it's the expectation value.

But what about the variance?
Define the operator
\begin{align}
    \Delta A = A - \bar{A}\mathbf{1}
\end{align}
This is a psychological test of authors of books.
Some people are, particular,
and they like to put in that identity.
You can also define another operator
\begin{align}
    \Delta B = B - \bar{B}\mathbf{1}
\end{align}
Then I claim that this expectation value
\begin{align}
    \sigma_A^2 &= \bra{\psi} (\Delta A)^2 \ket{\psi}\\
    &= \bra{\psi}
    A^2 - A\bar{A} \mathbf{1} - \bar{A} A + \bar{A}^2 \mathbf{1}
    \ket{\psi}
\end{align}
where $\sigma_A$ quantifies the spread of $A$.
This operator $\Delta A$ is obviously the deviation,
could be negative or positive,
and if I square it I get the dispersion around the average,
clearly measuring the standard deeviation.
You could do the sae for $B$ and get the same thing.

Define the sattes
\begin{align}
    \ket{\alpha} = \Delta A\ket{\psi}\\
    \ket{\beta} = \Delta B\ket{\psi}
\end{align}
and consider
\begin{align}
    \bra{\psi} (\Delta A)^2 \ket{\psi}
    \bra{\psi} (\Delta B)^2 \ket{\psi}
    &=
    \braket{\alpha}{\alpha}
    \braket{\beta}{\beta}
\end{align}
Have you seen an expression like this beforei nthe homework?
This is the Schwarz inequality that you proved last week.
Perectly timed,
that's a quality class.
This is going to be
\begin{align}
    \braket{\alpha}{\alpha}
    \braket{\beta}{\beta}
    \ge
    |\bra{\psi} \underbrace{\Delta A \,\Delta B}_{\braket{\beta}{\beta}}\ket{\psi}|^2
\end{align}
straight from the homework.
You know what the commutator and anticommutator means?
\begin{align}
    \bra{\alpha}(\Delta A)^2\ket{\alpha}
    \bra{\beta}(\Delta B)^2\ket{\beta}
    \ge
    \left|
        \bra{\psi}
        \frac{1}{2} [\Delta A, \Delta B]
        + \frac{1}{2}\left\{ \Delta A, \Delta B \right\}
    \right|
\end{align}
Now I need some scratch paper to show that
\begin{align}
    \left( AB - BA \right)^\dagger
    &= B^\dagger A^\dagger
    - A^\dagger B^\dagger\\
    &= BA - AB\\
    &= - [A, B]
\end{align}
so the commutator is anti-hermitian.
You can also show that the anti-commutator is hermitian.

So we have an anti-hermitian commutator,
and a hermitian anti-commutator.
Think of the complex plane,
and a complex number with real and imaginary parts.
I claim that no matter what,
the modulus is always going to be larger than the imaginary part.
\begin{align}
    \bra{\alpha}(\Delta A)^2\ket{\alpha}
    \bra{\beta}(\Delta B)^2\ket{\beta}
    &\ge
    \left|
        \bra{\psi}
        \frac{1}{2} [\Delta A, \Delta B]
        + \frac{1}{2}\left\{ \Delta A, \Delta B \right\}
    \right|^2\\
    &\ge 
    \frac{1}{4}
    \left|
        \bra{\psi}
        [\Delta A, \Delta B]
        \ket{\psi}
    \right|^2
\end{align}
Is this the usual uncertainty principle you learn in high school?
Sure is.

Let's take a spinless particle in one dimension.
The Hilbert space is the space of functions in real variables
$\psi(x)$.

So the position operator $\hat{x}$ acts like
$\hat{x}\psi(x) = x \psi(x)$.

There's another operator you learn in kindergarten
\begin{align}
    \hat{p} \psi(x)
    = -i\hbar \frac{d}{dx}\psi(x)
\end{align}
and you should also know $[\hat{x},\hat{p}] = i\hbar\mathbf{1}$.
Then you get
\begin{align}
    \sigma_x \sigma_p \ge \frac{\hbar}{2}
\end{align}
which is the traditional formula you learn.

\section{2021-09-22 Lecture}

This is a class that cannot fail,
because if it crashes and burns,
the qualifier fails.

This is my favourite lecture.
Sorry it's online.
It's conceptual.
This used to not be covered in standard classes,
but it because so popular there's no way out.
I'm going to start from something everyone knows.
The double slit experiment.

If you're not comfortable,
it means two things:
you don't know QM and you should laern straight away.

The best way to learn is to open the old Feynman lectures on physics.
Of course you can go to Youtube anddo the same thing,
with beautiful computer graphics,
but unfortanaltely many of them are wrong.
Feynman knows physics.


If there's only one hole nothing exciting happens.
The particle hits the screen,
mostly behind the hole.
Here I plot the number of particles that hit the screen.
If the hole is left,
you get particles on the lfet.
If the hole is right,
you get particles on the right.

Buty what happens when you shoot two holes?
They interferer with each other.
The result is surpricingly,
in fact the most likely place to get a particle is in the middle,
which is neither the single hole scneairos.
It's more than just the sum of the two.
You get an interference pattern.

You seen t his before.
It's pretty.
There are regions where the waves interfere desctructively and you don't get a
particle behin the screen.
that's how all waves behave.
If you make waves no thel ake,you get he same thing,
so it's not surprising.
What is surprising is tat the aprticles sometimes behave like particles,
and sometimes waves.
In fact,
what you get on the screen is this.

If you send particles one by one and you start counting,
every time you send a particle,
you get one spot on the screen,
however the probability fo getting apricles obeys the interference pattern.
That's this other way we see this particle-wave duality.

I'm going to switch to my laptop.
That's a drawing of the situation here.
I want to remind you how to descrbie this quanutm mechancially.


Let's say the left sitautions is $\ket{L}$
and the situation for the right hole open is $\ket{R}$.
When you have both holes open,
the state is going to be a linear combination
\begin{align}
    \ket{\psi} = \alpha\ket{L} + \beta\ket{R}
\end{align}
How do you get intference?
Let's measure $A_y$ that counts the number of particles,
or the probablity of the particle arriving at coordinate $y$
on hte screen.
Of course,
I want to take the expectation value.
\begin{align}
    \bra{\psi} A_y \ket{\psi} &=
    \left( \alpha^* \bra{L} + \beta^*\bra{R} \right)
    A_y
    \left( \alpha\ket{L} + \beta\ket{R} \right)\\
    &=
    |\alpha|^2 \bra{L}A_y\ket{L}
    + |\beta|^2 \bra{R}A_y\ket{R}
    + \alpha\beta^* \bra{L}A_y\ket{R}
    + \alpha^*\beta \bra{R}A_y\ket{L}
\end{align}
The first two terms are familiar.
They are just the probabilities for if the left hole is open only
and the rigt hole is open only respectively.
THe novelty of course in QM is that you have inerference terms,
the last two terms.
While the $|\alpha|^2$ and $|\beta^2|$ terms are positive,
those $\alpha\beta^*$ and $\beta\alpha^*$ terms are not necessarily positive,
so it's possible you cna have cancelation.
And that's why you get zeros in the distribution,
it's because of intereference and cancellation.

So this is well known.
The fact you have uqanutm intereference out of linear combinations,
you get this pretty well.

Let's compare this with a different situation.
Let's say I have a different siuaiton like this.
Except the two slits are never open at the same time.

I'm going to trow a dice.
If the number is odd,
I open the left hole.
If the number is even,
I open the right hole.
What am I going to get?

hal the time I'm going to get the L distributino,
and half the time I'm going to ge the R distribution.
And I get a double-bump distirbution.

So then this quantity,
The previous situation would be left AND right.
But now I will call this dice roll situation left OR right.
\begin{align}
    \bar{A}_y &=
    p_L \bra{L}A_y\ket{L}
    + p_R \bra{R}A_y\ket{R}
\end{align}
where $p_L$ and $p_R$ are the probabilities of having the left or right holes
open respectively.
You notice that this matces the first two terms of the previous formula.
What you do not find is the second two terms,
so there is no interference.
This is a different physical sitaution.

Those two things are not the same.
If you plot $A_y$ vs $y$,
in the QM superposition,
you get this
[picture]

But if you have the classical probability,
you get two bumps.
[picture]

These two sitautions are very common,
and you need to give these names.
having L and R in superpsotiion quantum mechanically,
we have a \emph{coherent} sum.
But the classical probability situation,
is the \emph{incoherent sum}.
The incorhernet sitaution happens all the time
in classical physics.

But the coherent sum is something very special to QM,
there is no classical analogue for that.

This situation where you have a combination of this,
where some of it is quanutm mehaonical,
but I also don't know what the wave function is,
appears commonly in physics too.

So ther eare two kinds of uncertainty.
There is the quantum uncertainty of the wave function,
butther'es the classical uncertainty of not knowing what the wave function is in
the first place.
There is a formalism to deal with this.

Suppose you have some observablve $A$.
Then
\begin{align}
    \bar{A} &=
    \underbrace{
    \sum_n \underbrace{p_n}_{\text{prob. in $\ket{n}$}} \underbrace{\bra{n} A \ket{n}}_{\text{quantum average}}
    }_{\text{classical average}}
\end{align}
So we have a quantum and a classical average

There's a cute way to write this as a trace.
\begin{align}
    \bar{A} = \Tr\left[
    \left( \sum_n p_n \ket{n}\bra{n} \right) A
    \right]
\end{align}
Is this the same?
It's easy to see,
because to compute the trace,
one way of computing the trace is to take a basis,
for example the same basis $\ket{n}$,
then compute the sandwich of whatever is in the trace and then sum over elements
of the basis,
but you see this is easy to compute.
\begin{align}
    \Tr\left[
    \left( \sum_n p_n \ket{n}\bra{n} \right) A
    \right]
    &=
    \sum_{m} \bra{m}
    \sum_n p_n \ket{n}\bra{n} A\ket{m}\\
    &= \sum_{m,n} \cdots
\end{align}
Then observe the following things.
Then we call that part the density matrix.
\begin{align}
    \rho = \sum_n p_n \ket{n}\bra{n}   
\end{align}
This contains all the information about the system.
It doesn't tell me what the wave function is,
because there's uncertainty about what the wave function is.
But the probabliies are al defined by this $\rho$.
States whree therare different clasical probailitesi of different wave fucntions
is called a \emph{mixed state}.
In situations wher we know what hte wave function is,
they are called \emph{pure dstates}.

To be clear,
\begin{align}
    \bar{A} = \Tr(\rho A)
\end{align}
Remember $\rho$ contains informaiton about all possible wave functions the
system may have.

Are we good up to now?
By the way,
you can raise a question by raising a hand,
but it's better to shout,
just unmute yourself and shout.
You can even go and ask questions in the chat.

\begin{question}
    Can you elaborate on mixed state?
\end{question}
If you know thw ave function of the sytem,
you call it a pure state,
that's what you learnt in QM so far.
A pure state is just a real state.
But then suppose you have a harmonic oscillator
with 30\% chance in the ground state and 70\% chance in the first excited state,
that is a mixed state.
You should not be confused though.
Take hte harmonic oscilator for exmaple.
\begin{align}
    \ket{\varphi} &=
    \frac{3}{5}\ket{0}
    + \frac{4}{5}\ket{1}
    \ne
    \rho
    =
    \left( \frac{3}{5} \right)^2 \ket{0}\bra{0}
    \left( \frac{4}{5} \right)^2 \ket{1}\bra{1}
\end{align}
These states here have the same probabiliteis,
but they are completely different states.
I can even add a phase.
\begin{align}
    \ket{\varphi} &=
    \frac{3}{5}\ket{0}
    + i\frac{4}{5}\ket{1}
    \ne
    \rho
    =
    \left( \frac{3}{5} \right)^2 \ket{0}\bra{0}
    \left( \frac{4}{5} \right)^2 \ket{1}\bra{1}
\end{align}
It's either one or the other.
No matter how smart I am,
I will know a way to measure.

here's an example.

How do we describe th sitaution where both holes are open.
We already have that,
a quantum superpsoition of left and right.
When I compuet the probablity of getting a particle on the sreen,
I get interference tersm.
If I have a phsae,
the probabilities $|\alpha|^2$ and $|\beta|^2$ don't change,
but I get a different interefernece pattern.

That's usual quantum mechanics,
but I can describe the sitaution with desity matrices too
which account for probabilities that arise because I am too dumb to know.

So I can describe the double slit experiment with density matrices too.
For example,
the incoehreent sum is
\begin{align}
    \rho &= \frac{1}{2} \ket{L}\bra{L}
    + \frac{1}{2}\ket{R}\bra{R}
\end{align}
and you can computet the probabilities
\begin{align}
    \bar{A} &=
    \Tr(\rho A_y)\\
    &=
    \Tr\left( \frac{1}{2}\ket{L}\bra{L} + \frac{1}{2}\ket{R}\bra{R} \right) A\\
    &=
    \bra{L}\left( 
    \frac{1}{2} \ket{L}\bra{L}
    + \frac{1}{2} \ket{R}\bra{R}
    \right)A_y
    \ket{R}
    + \bra{L}\left( 
    \frac{1}{2} \ket{L}\bra{L}
    + \frac{1}{2} \ket{R}\bra{R}
    \right)A_y
    \ket{R}\\
    &= \frac{1}{2}\bra{L} A_y \ket{L}
    + \frac{1}{2}\bra{R} A_{y} \ket{R}
\end{align}
Now look at what we thoguht it should be
\begin{align}
    \bar{A}_
    &= p_L\bra{L} A_y \ket{L}
    + p_R\bra{R} A_{y} \ket{R}
\end{align}
so the formalism works!
Let me tell you a couple of probabilities it's to satisfy.

Firstly, $\rho$ is a Hermitian operator.
It's weird,
because it's not an observable,
but it's the state of the system.
\begin{align}
    \rho^{\dagger} = \rho
\end{align}
Can you see this is true?
Sure
\begin{align}
    \left( \sum_n p_n \ket{n}\bra{n} \right)^{\dagger}
    &= \sum_n p_n^* \ket{n}\bra{n}\\
    &= \sum_n p_n \ket{n}\bra{n}
\end{align}
because $p_n$ are real probabilities,
so very straighforward.

Anotherp robability isthat it has unit trace.
\begin{align}
    \Tr \rho = 1
\end{align}
To see this,
\begin{align}
    \Tr \rho &=
    \Tr \sum_n p_n\ket{n}\bra{n}\\
    &= \sum_m \bra{m} \sum_n p_n\ket{n}\bra{n} \ket{m}\\
    &= \sum_{n,m} p_n \braket{m}{n} \braket{n}{m}\\
    &= \sum_n p_n\\
    &= 1
\end{align}
because the total probability is equal to 1.


Finally, $\rho$ is positive definite.
It follows from the fact that $p_n$ has to be positive or zero.
Positive definite means that
\begin{align}
    \bra{\phi}\rho\ket{\phi} &=
    \bra{\phi} \sum_n p_n\ket{n}\braket{n}{\phi}\\
    &=
    \sum_n p_n \braket{\phi}{n} \braket{n}{\phi}\\
    &= \sum_n p_n |\braket{n}{\phi}|^2\\
    &\ge 0
\end{align}

There's one final property.
For some special states, we have
\begin{align}
    \rho^2 = \rho
\end{align}
if and only if $\rho=\ket{\psi}\bra{\psi}$ is a pure state.
You can see this clearly because if
$\rho = \sum_n p_n \ket{n}\bra{n}$
and if $p_n=0$ except $p_{\bar{n}}=1$,
then obviously $\ket{\bar{n}}=\ket{n}$.
There are two things to prove,
because it's if and only if.
It's a simple test.
Suppose $\rho$ is pure,
then
\begin{align}
    \rho^2 &=
    \ket{\psi}\braket{\psi}{\psi}\bra{\psi} = \rho
\end{align}
It's a bit of extra work to go the other way around.

Suppose $\rho^2 = \rho$.
Then 
\begin{align}
    \sum_n p_n \ket{n}\bra{n}
    \sum_m p_m \ket{m}\bra{m}
    &= \sum_{n,m} p_n p_m \ket{n} 
    \underbrace{\braket{n}{m}}_{\delta_{nm}} \bra{m}\\
    &= \sum_n p_n^2 \ket{n}\bra{n}\\
    &= \sum_n p_n \ket{n}\bra{n}
\end{align}
so we know that $p_n^2=p_n$ for all $n$.
There are only two ways this can be true.
Either $p_n=0$ or $p_n=1$.
But probabilities sum to 1,
so only one of these can be $p_{\bar{n}}=1$,
and all the other $p_n=0$.
What does that prove?
In other words,
\begin{align}
    \rho = \sum_n p_n\ket{n}\bra{n} = \ket{\bar{n}}\bra{\bar{n}}
\end{align}

Oh there's a question.
It's just one word: magical.

You're just in awe of the power of density matrices and bras and kets?

By the way,
Landau invented density matrices.
He was two years too young to invent QM,
but he invented density matrices.
Bras and kets,
I'm tired of syaing it's a beautiful thing,
but I hope you can appreciate how the algebra just works without thinking.
It's importnat to understand why it works,
but once you do,
it's easy.

The name of this class is quantum and statistical physics.
Forgoet about the deep stuff.y
You're goign to use distributions of positions and momentum classical,
but the density analogue is density matrices,
where you have distributions over wave functions.


\begin{question}
    So $\rho$ is diagonal?
\end{question}
Well $\bra{n}\rho\ket{n'}$ is a diagonal matrix in this basis,
but in a different basis,
it will not necessarily be a diagonal matrix,
despite describing the same physics.
If the operator was diagonal,
I wouldn't talk about htis operator,
it would be overkill.
In this basiss to define $\rho$,
but may be diagonal,
but in another basis,
it's not diagonal
and it's not obvoius at all it's not a pure state.

\section{TQFT for Majorana Chain}
I told you that
\begin{align}
    V(S^1, \pm) \simeq \mathbb{C}
\end{align}
I also told you the path integral
\begin{align}
    Z(\Sigma_g,eta) = (-1)^{\textrm{Art}(\eta)}
\end{align}
where $\Sigma_g$ is a closed genus-$g$ surface.
I told you that genus $g$ surfaces are in 1-1 correspondence to a space of
quadratic forms.
\begin{align}
    \text{spin structures on }\Sigma_g \leftrightarrow
    \text{quadratic forms }\in Q\left( H_1\left( \Sigma_g, \mathbb{Z}_2 \right),
    \phi\right)
\end{align}
where $\phI$ is a symmetric bilinear form which is the intersection form that
tells you whether two ops are intersecting or not.
And remember
\begin{align}
    q(x + y) = q(x) + q(y) + \phi(x, y)
\end{align}
I asserted the space of quadratic forms are in 1-1 correspondence with the space
of spins structures.

As for that Arf,
I defined something called the Arf invariant of a quadratic form,
so you take the Arf invariant associated with the spin structure.

I just asserted a bunch of stuff because it was just math.

One example was the Torus, and the partition function was 
\begin{align}
    Z(T^2, \eta) &= -1, 1, 1, 1
\end{align}
depending on whether the boundary conditions are
PP, AP, PA or AA.

If you look at the partition function of a disk,
the disk has a unique spin structure I won't write.
\begin{align}
    Z\left( D^2 \right)
    \in
    V\left( S^1, - \right)
\end{align}
where the $-$ means anti-periodic boundary conditions.
I will denote $+$ for periodic.

Last time,
I asserted the following
\begin{proposition}
    Anti-periodic boundary conditions on a circle $S^1$ give a spin structure
    than can be extended to the disk $D^2$.
    These are also called \emph{bounding} spin structures,
    whereas periodic boundary conditions are \emph{non-bounding}.
\end{proposition}
If you draw a circle,
you can think of the fermion as defining a spinor in your space.
Consider 2 situations.
1 is where you have part of the spinors tangent to the circle,
and the other parts are looking like this.
This is the one that gives you anti-periodic boundary conditions,
because as you go around the circle,
the arrows wind around the circle,
and you get a $2\pi$ rotation,
which is a minus sign for fermions.
This can be extended to the disk.

On the other hand,
this spin structure (all pointing out) cannot be extended to the disk.
I don't have the time to give you background on spin structures,
so take this as a bunch of assertions.
If you want to know what's going on behind this,
we have to talk offline.
But if you do have questions, ask.

\begin{question}
    Assume if you go across the line and rotate multiple lines instead of one
    time.
    These are periodic and anti-periodic?
    We're not sure if those are also have the same.
\end{question}
You want me to convince you that some more complicated rotation on that circle
and show the anti-periodic one will be bounding?
Not sure I can do that right now?

\begin{question}
    What's a spin structure.
\end{question}
You start with a tangent bundle,
say in $d$ dimensions.
The tangent bundle is a $SO(d)$ bundle.
At every point in your manifold there is a space of tangent vectors.
If the space if curved then this will be curved.
For fermions, you don't want $SO(d)$,
you want a double cover of this.
You want a $Spin(d)$ bundle.
So $spin(d)$ is the double cover of $SO(d)$.
Basically,
on every patch originally we had a map to $SO(d)$,
so we had some map $\phi:U_\alpha\to SO(d)$.

The manifold is covered by open charts $\left\{ U_\alpha \right\}$

But instead,
you want to map to $Spin(d)$,
and you have to specify which covering of $SO(d)$ you want to cover.
So really you want to pick it with a sign
\begin{align}
    \tilde{\phi} = (\phi, \eta)
\end{align}
where $\eta$  tells which of $SO(d)\to Spin(d)$
we choose.

A spin structure is something that as you go along the manifold,
you're possibly getting rotated by $SO(d)$,
and something.
You can think of that as one basic definition.

\begin{question}
    How can a manifold not have a spin structure?
\end{question}
It might not be possible to choose $\pm$ signs that's consistent everywhere.
I don't have a 2-line explanation,
but what you find is that you want to make sure this lift is consistent on
triple overlaps.
So make sure lift is 

$SO(d)$ bundle,
there are transition functions
$f_{\alpha\beta}\in SO(d)$
which tell you how to go from one coordinate patch to another.
But when you lift it,
you want
$\tilde{f}_{\alpha\beta} = (f_{\alpha\beta},\eta)$,
and you want this consistency identity
\begin{align}
    \tilde{f}_{\alpha\beta} \tilde{f}_{\beta\gamma} \tilde{f}_{\gamma\alpha}
    = 1
\end{align}
and this is 
\begin{align}
    \delta\eta = W_2(TM)
\end{align}
You need manifolds with trivial second Stiefel-limits.

The spin chain is just 1D,
and all 1D manifolds are just circles if they're closed,
and they always admit a spin structure.
I only know an example in 4D of a manifold that doesn't admit a spin structure.
The only example of a manifold
All 1-, 2- and 3- manifolds admit spin structures.

To finish off the TQFT,
we would also want to understand the path integral on this kind of structure.
If I don't put an markings,
it's anti-periodic.
But imagine changing the boundary conditions,
like a branch cut here that makes this APP.
In principle,
these are all maps from $S^1\times S_1$ to $S^1$.
I haven't specified what the maps are,
but show that everything that is consistent you get $(-1)^{\textrm{Arf}}$.
Because these are 1D Hilbert spaces,
it's just going to be a choice of signs.
With all these signs,
you can build path integrals on closed surfaces using the multiplicative rule
and you can always get the Arf invariant.

That was an example of an invertible TQFT,
with the vector space being 1D.

The Majorana chain is an example of an invertible topological phase of matter.

Furthermore,
we have this $\mathbb{Z}_2$ invariant,
which is a $I=P_P P_A$.
If I took a double layer system,
if I had a $-1$ here, I would get $-1$.
That is,
under ``stacking'',
which means take the total Hamiltonian to be $H=H_1+ H_2$
and the Hilbert spaces to be $H=H\otimes H_2$.
If I two of them together,
I get a trivial phase.
This forms a $\mathbb{Z}_2$ group structure under stacking,
and in this case,
the state is its own inverse.

\begin{question}
    The state is its own inverse or the field?
\end{question}
All of the other.
There is some adiabatic path from
$\ket{\psi_A}\otimes \ket{\psi_B}$
to a product state.

\begin{question}
    What is a product state?
\end{question}
This should just be a state where I start with a Fock vacuum
$\ket{0}$,
I go over all sites and a put a fermion on all sites
$\prod_r c_r^\dagger \ket{0}$.
It should be a state with a fermion or not on every site.

\begin{question}
    In the trivial phase of the Majorana phase,
    it's a superconducting state?
\end{question}
The trivial phase,
there are various versions of it,
(1) was the completely empty phase $\ket{0}$.
(2) another phase was the ``atomic'' superconductor,
where on every fermion pair we had an odd number of fermions.

If we break translational symmetry,
we should be able to get an adiabatic path
$\ket{\psi_A}\otimes\ket{\psi_B} \to \ket{0}$,
when we're allowed to break symmetries.

\begin{question}
    Stacking is a way of combining 2 TQFTs?
\end{question}
Here I'm just talking about microscopic systems.
I'm literally saying it's now 2 chains instead of 1 chain.
That's all I mean by stacking here.

As far as the TQFT is concerned,
stacking means I multiply the path integrals.
$Z_{AB} = Z_A Z_B$.

\begin{question}
    What happens to the vector spaces?
\end{question}
There's a subtlety,
I have a fermion parity in the top system,
and a fermion parity in the second system,
but the parity you want to look at is both at the same time $P=P_1P_2$.
The vector spaces are all 1D and labelled by 1D or 2D and all are related by
the same.

\begin{question}
    How can you get a trivial phase just by grouping 2 together.
\end{question}
The starting Hamiltonian is decoupled but the path is definitely going to couple
them.
Maybe it's more precise to say that there's a path of Hamiltonian $H(s)$
and $H(0) = H_1 + H_2$ are decoupled.


\begin{question}
    Does the TQFT make it easier to see this factor,
    or can it be deduced from the microscopic?
\end{question}
No, this doesn't require TQFT, just that there is a $\mathbb{Z}_2$ invariant
that is either $\pm 1$,
and that together they give $+1$.
That fact is mirrored in the TQFT by $Z=(-1)^{\textrm{Arf}(\eta)}$.

Any other question?

\section{Spin bordism group}
One last point about this TQFT story is that I want to introduce the notion of a
spin bordism group.
A bordism is a high dimensional manifold between two manifolds as boundaries.

I can define a bordism group.
2 $d$-manifolds are equivalent to each other if there exists a bordism between
them.
$M_1^d \sim M_2^d$ if there exists a bordism from $M_1^d$ to $M_2^d$.
So that allows me to define equivalence classes $[M^d]$
of manifolds.
Then 
I can define multiplication by disjoint union
\begin{align}
    [M_1^d]\times [M_2^d]=
    [M_1^d \sqcup M_2^d]
\end{align}
And the inverse is just
\begin{align}
    [M_1^d]^{-1} = [\bar{M}_1^d]
\end{align}
So this group is denoted $\Omega^{\text{or}}_d$
to denote that they are oriented manifolds.
Mathematicians have computed what these groups are for each dimension.

For example, in $d=2$,
the bordism group is trivial.
In 3D it's also trivial.
Then in 4D it's $\mathbb{Z}$.
1D is also trivial.

Now consider not just bordism,
but also spin bordism,
which means each manifolds also comes equipped with a spin structure,
so we're going to have not just the manifolds.

Spin bordism:
We have $(M^d,\eta)$.
The spin structure also must extend to the higher dimension manifold,
and we define multiplication and inverse in exactly the same way.
I won't write it out,
because it's just adding spin structure,
and we call this
$\Omega_d^{\textrm{spin}}$,
which mathematicians have also calculate.

In 1D, it's $\mathbb{Z}_2$.
In 2D, it's $\mathbb{Z}_2$.
In 3D, it's trivial.
In 4D it's $\mathbb{Z}$.

What is the meaning of this?
Let's focus on $d=2$.
In $d=2$, the oriented bordism group is trivial,
which means every closed oriented 2D manifold is the boundary of some 3D
manifold.
That means every surface is boundary of a 3D manifold.

For example,
the boundary of a solid torus
is torus disjoint union nothing,
which means the torus is bordant to empty,
which means the torus is trivial as far as the bordism group is concerned.

Every 2D surface can have its interior filled,
and now I have a 3D manifold inside,
and it's a trivial bordism group.

So spin bordism is not trivial,
and that's related to the point that not all spin structures are boundaries.

On the torus,
suppose I have anti-periodic boundary conditions on both loops.
If one of them is anti-periodic that's enough.
Because if that's anti-periodic,
that's enough.
And if the other one is anti-periodic that's enough.
And so AA, AP and PA are all boundaries,
all bound.
But PP does not bound.
If both of these are periodic,
both of these spin structures cannot be extended to the interior bulk.
That's why its bordism group is $\mathbb{Z}_2$.

Already, you see a tight relation between the path integral,
and the relation of these bordism groups.

What happened is that my path integral is actually
for every element in $\Omega_d^{\textrm{spin}}$,
it gives me $\pm 1$ and that was my path integral for my TQFT.

So my TQFT path integral gave me a $\pm 1$.
Actually, I can take an element
$x \in \Omega_2^{\textrm{spin}}$
and $Z(x)=\pm 1$
depending on which bordism group it's in.
So really, the path integral is a map
\begin{align}
    Z: \Omega_2^{\textrm{spin}} \to \left\{ \pm 1 \right\}
\end{align}
That is our TQFT.
A slightly different way so saying it's a homomorphism
from $\Omega_2^{\textrm{spin}}$ to $U(1)$,
denoted
$\textrm{Hom}\left( \Omega_2^{\textrm{spin}}, U(1) \right)$.
And this is deep,
it's called the
\emph{Pontryagin dual} of $\Omega_2^{\textrm{spin}}$.

I'm telling you this because this is the property that generalises.
Manifolds can have background $g$ gauge fields,
where we define gauge field on our manifold.

\begin{question}
    Why $U(1)$?
\end{question}
$U(1)$ is the thing that generalizes.
Sometimes it's not $\mathbb{Z}_2$ but $\mathbb{Z}_m$,
in which case you get a complex phase.
For an invertible TQFT,
the path integral is just a phase,
which is why it's $U(1)$.
For non-invertible TQFT,
there's no such a nice classification.

\begin{question}
    Why do we choose $\pm 1$ here?
\end{question}
In this example it's $\pm 1$ and this example is the
Pontryagin dual.
It's a much more complex structure called an Anderson dual,
and I don't even know what the complex structure is.

\begin{question}
    Does it just have to be a complex phase or a root of unity?
\end{question}
If your bordism group is finite,
then it must be a root of unity.
But if it's a $\mathbb{Z}$ classification,
it's not clear why.
If it's finite order it's obvious.

The fact you have 2 distinct phases trivial and non-trivial,
has a mirror in the language of TQFTs,
the spin bordism group having two elements as $\mathbb{Z}_2$.

\section{Ising model and Jordan-Wigner duality}
Let's write down the transverse field Ising model in 1D.
\begin{align}
    H &=
    -J\sum_{i} \sigma_i^x \sigma_{i+1}^x
    - h\sum_{i=1}^{N}\sigma_i^{z}
\end{align}
where the $\sigma$ are just Pauli matrices.
Let's do an open chain of $N$ sites.
Now there is a global $\mathbb{Z}_2$ symmetry,
which I am going to write as
\begin{align}
    Z &=
    \prod_{i=1}^{N}\sigma_i^z
\end{align}
and this is a global symmetry that commutes with the Hamiltonian and squares to
1.
$[H,Z]=0$.

There are 2-spin systems per site.
Instead of thinking up and down ad being fermion parity even or odd.
So fermion and no fermion.
So up and down can be reinterpreted as fermion and no fermion.
For each site $i$,
I'm going to define 2 Majorana fermions.
\begin{align}
    \gamma_{2i - 1} &=
    \sigma_1^x \prod_{j=1}^{i-1}\sigma_j^{z}\\
    \gamma_{2i} &=
    \sigma_i^y \prod_{j=1}^{i-1}\sigma_i^{z}
\end{align}
which means
\begin{align}
    \sigma_i^z = -i \gamma_{2i - 1}\gamma_{2i} = P_i
\end{align}
which is the fermion parity operator.
And it's a exercise to show that these satisfy the algebraic conditions for the
Majorana operators.
And the product term becomes
\begin{align}
    \sigma_i^x \sigma_{i+1}^x &=
    -i \gamma_{2i}\gamma_{2i + 1}
\end{align}
and the $Z$ symmetry becomes
\begin{align}
    Z = P = \prod_{i=1}^{N}P_i
\end{align}
which is just the total fermion parity.

And if you substitute this into the Hamiltonian,
you find the system is just a Majorana chain.
\begin{align}
    H &=
    - J \sum_{i=1}^{N-1} i \gamma_{2i} \gamma_{2i + 1}
    + h\sum_{i=1}^{N} i \gamma_{2i - 1}\gamma_{2i}
\end{align}
So there is a mapping from Ising model and the Majorana chain.
And you see how the 2 phases of the Ising model corresponds to the trivial and
non-trivial phase of the Majorana chain.


The ferromagnetic phase $J\gg h$ is when everything is aligned  in $x$ and
that's the
topological phase in Majorana language.
Now you can see the origin of the degeneracy,
with chain,
that had Majorana zero modes,
which led us to a double degeneracy in the ground state,
with even vs odd parity.
In the Ising model,
we also have a degeneracy with all point in $+x$ or all pointing $-x$.


And then the state would be
\begin{align}
    \ket{\rightarrow \cdots \rightarrow}
    \pm
    \ket{\leftarrow \cdots \leftarrow}
\end{align}
and $Z\ket{\pm} = \pm \ket{\pm}$,
which in Majorana language means
$P= \pm 1$,
and $P \propto -i\gamma_L \gamma_R = \pm 1$.

In the $J\ll h$ limit,
we have the paramagnetic phase.

Why bother to do all this just to describe the Ising model?
Well no.
There is a non-local relationship.
That non-local relationship is why there is no local operator to distinguish the
ground states.
No local bosonic operator can distinguish the ground states and connect them.
The degeneracy comes from symmetry breaking in the Ising model,
but it's from topology in Majorana.
Spontaneous symmetry breaking
gets mapped by this non-local mapping to topological degeneracy.

\begin{question}
    Isn't there a duality between strong $J$ and weak $J$,
    how can there be different Majorana parities,
    one is topological phase and another one isn't.
    How can they have different parities if they're dual.
\end{question}
Parity in Majorana language is just total $Z$ in Ising model.
This is going to open a whole can of worms.

The Ising model has 2 dualities.
One of them is this Jordan-Wigner duality.
There's another duality,
which is not really correctly stated in the literature which is a
Kramers-Wannier ``duality'' duality where you swap
$J\leftrightarrow J$.
But it's not a true duality
because the ferromagnetic phase has 2 ground states,
but the paramagnetic phase only has 1 ground state.
So the degeneracy isn't mapped right.

However,
there is a duality,
where the Ising model is coupled to an Ising $\mathbb{Z}_2$ gauge field.

The usual ``Kramers-Wannier duality''
already has a problem with the counting of the ground sate.
That counting is corrected by coupling the Ising model with a
$\mathbb{Z}_2$ gauge field.
And that might solve all the paradoxes you might have.

\begin{question}
    If we such a system with Pauli matrices,
    can we characterise every Hamiltonian?
\end{question}
If you stick to 1D,
you can always do this transformation nd you find that the $\mathbb{Z}_2$
symmetry-broken state will always map to the topological phase.
The Jordan-Wigner in higher dimension,
you'd have to convert to higher $d$.

Yu-An has spent the last few years doing this.
Xiao-gang Wen showed this can be transformed from higher $d$ to some duality.

You have to be careful with non-local transformations to make dualities,
you need to know what's local and what's not local.
If you started with Majorana and went backwards,
you would say everyone missed something,
you could just do topological quantum computing with Ising models!
But you have to be careful because the aping is non-local.

\begin{question}
    What are the possible applications?
\end{question}
There's an application in quantum computing to simulate fermions,
and you want to simulate them without having them,
then this duality could help.
This is a baby version of duality.
If you understand this,
you can try to understand all sorts of ore complicated dualities.

\begin{question}
    I've seen this being related to being on the boundary of a toric code,
    is this related to each other,
    or is this a different kind of structure where the toric code is in 2D?
\end{question}
Um. Well we'll get to the toric code,
but if you had a 2D phase of matter and looked at the boundary,
indeed,
you could map the boundary into a 1D transverse field Ising model,
or a Majorana china.
In fact, the toric code has 2 types of boundaries,
rough and smooth,
and those map to the 2 Majorana phases of matter.
All the analysis will be useful.

Any other questions?
That's all I'm going to say for Majoranas in this lecture.
There's a few more important topics.
I gave you some model Hamiltonians for Majoranas.
How do you realise real Majoranas in a real solid state system.
In the homework I'm going to have such as system.

One big direction is how to do you actually realise Majoranas and what are the
physical systems.

\section{Chern Number}
Now we shift gears to go into higher dimensions.
That was the simplest topological phase of matter in 1D.
It didn't require anything fancy except for fermions.
Now it's associated with an invariant called the Chern number.
And now we're in $(2+1)$ dimensions.

We're going to stat off by defining invariants of band structures,
the Chern number,
but it's also sometimes called the TKNN number,
after Thouless, Kohomoto, Nightingale and den Nijys.
This is some invariant of a band.

Let's say we have a free particle
and it exists as some 2D lattice model,
so we have a tight binding model that looks like
\begin{align}
    H &=
    - \sum_{ij} c_i^\dagger c_j t_{ij} + \text{h.c.}
\end{align}
So we have a Brillouin zone.
The momenta are $(k_x, k_y)$ and we identify
\begin{align}
    \vec{k} \sim \ket{k} + \vec{G}
\end{align}
where $\vec{G}$ is the reciprocal lattice vector.
And so you do the quotient
\begin{align}
    \frac{\mathbb{R}^2}{\mathbb{Z}^2} = T^2 = S^1\times S^1
\end{align}
So the Brillouin zone is actually a torus.

The idea of a Chern number is that you have bands
$E$ vs $k$.
Every bad is going to be associated with some kind of invariant 
$C_1,C_2,C_3,\ldots$ which is an integer.
The way you define this integer for every single band is as follows.

Remember that we have Bloch states
\begin{align}
    \psi_{\vec{k}}(\vec{x})
    = e^{i\vec{k}\cdot\vec{x}}
    u_{\vec{k}}(\vec{x})
\end{align}
where these $u$ plane wave Bloch waves are periodic
\begin{align}
    u_k (x) = u_k(x + e)
\end{align}
where $\hat{e}$ is the lattice vector.
The phase of these wave functions can wind around in non-trivial ways and those
are characterised by these Chern numbers.

Mathematically we define a ``Berry connection'',
which is a gauge field defined in momentum space.
\begin{align}
    A_j(\vec{k}) =
    -i \bra{\vec{u}_k}
    \frac{\partial}{\partial k_j} \ket{\vec{u}_k}
\end{align}
The reason it's a gauge field,
is if I redefine the phase of these Bloch states $u_k$,
then it doesn't change it.
That is,
under gauge transformation
\begin{align}
    \ket{u_{\vec{k}}}
    \to
    e^{i f_{\vec{k}}}\ket{u_{\vec{k}}}
\end{align}
the gauge field transforms like
\begin{align}
    A_j(\vec{k})
    \to
    A_j(\vec{k})
    + \frac{\partial}{\partial k_j} f(\vec{k})
\end{align}
and that gauge field defines a gauge-invariant field strength,
but when you integrate that gauge field through the Brillouin zone,
you get an integer invariant.

So let us define the field strength by
\begin{align}
    F_{xy}(\vec{k})i &=
    \frac{\partial A_y}{\partial k_x}
    - \frac{\partial}{\partial k_y}A_x
\end{align}
and then we can define the Chern number as
\begin{align}
    C &=
    \frac{1}{2\pi} \int_{BZ} d^2 k F_{xy}(\vec{k})
    \in \mathbb{Z}
\end{align}
There's an elementary theorem which says if you integrate a field over a closed
surface, 
you always get an integer.

The flux through any closed surface must be an integer multiple of $2\pi$.
Imagine your surface is a sphere,
and your sphere has some flux through it.
Imagine a particle that goes along some path on the sphere.
And this particle has charge under this gauge field.
The phase it's going to pick up is the look integral
\begin{align}
    e^{i\oint A\cdot dl} =
    e^{i\int_R F}
\end{align}
However you could consider the integral of this loop to be the integral of the
complement of that region.
\begin{align}
    e^{i\oint A\cdot dl} =
    e^{i\int_R F}
\end{align}
It's minus because it has the opposite orientation.
So for these to make sense,
it must be that the flux through the entire closed surface must be 
and integer multiple of $2\pi$.

I'll end there.

[Sorry missed this one]

Something about determining the Hilbert space.

\section{Many-body definition of Chern number}
I want to talk about the many-body definition of Chern number.

For every band you can assign a Chern number.
At the face of it,
it's only a property of band structure,
of single-particle TI system.s
There's 2 pieces of evidence for something more general.

1. The Chern number dictate th Hall conductivity,
which is quantized.
So if the Hall conductivity is given by an integer invariant in the free band
theory,
as you slowly turn on interaction,
that shouldn't change,
so there should also be a quantized invariant for that
but how do you extract that from the GS wave function.
2. I had a continuum Dirac theory,
that you have a mass term,
and integrate out the other degrees of freedom.
In that field theory,
you can have higher order field theory interaction
and you get a 4-fermion interaction that is irrelevant by RG,
which you ignore,
so you still have this integer quantity that is still defined if you change
the sign of the mass.

So the Chern number should be well-defined for interacting systems that doesn't
require TI.

I will give you multiple ways of calculating the Chern number fro GS
wavefunctions.


Hall conductivity and twisted boundary conditions.
Suppose we put our system on a torus,
and we insert flux on both holes of the torus,
so consider changing the vector potential in the $x$ and $y$ directions.
\begin{align}
    \delta A_x &=
    \frac{\Phi_0}{L_x} \frac{\theta_x}{2\pi}\\
    \delta A_y &= 
\end{align}
adding these fluxes changes the Hamiltonian,
and we get a term that looks like the gauge fields coupling to the current
\begin{align}
    \delta H &=
    \int d^2x \, \delta A_i J^i\\
    &=
    - \sum_{i=x,y} \frac{\Phi_0}{L_i} \frac{\theta_1}{2\pi}
    \int d^2 x\, J^i
\end{align}
and $H[\theta_x, \theta]$
Twisted boundary conditions vs flux through the hole of the torus are related by
singular gauge transformations.
Given this Hamiltonian,
we have a ground state wave function of this Hamiltonian
\begin{align}
    \ket{\Psi(\theta_x, \theta_y)}
\end{align}
from which we can construct the Berry connection,
just like before in momentum space.

So I can define
\begin{align}
    A_j(\theta_x, \theta_y) &=
    -i \bra{\psi(\theta_x, \theta_y)}
    \frac{\partial}{\partial \theta_j}
    \ket{\psi(\theta_x, \theta_y)}
\end{align}
and from this we can define a field strength
\begin{align}
    F_{ij} &= \partial_i A_j - \partial_j A_i
\end{align}
And here the closed surface is the torus,
because $\theta_x$ can go from $0$ to $2\pi$,
and once it's $2\pi$,
we can do a large gauge transformation
to relate it to the system without the flux.
And we get
\begin{align}
    C &=
    \frac{1}{2\pi}\int d^2\theta\,
    F
    \in \mathbb{Z}
\end{align}
And this is really an integral over the torus.
\begin{align}
    \theta_x $\sim \theta_x + 2\pi\\
    \theta_y $\sim \theta_y + 2\pi
\end{align}
The BZ was a torus,
but here the space is also a torus.
We integrate this field strengh over a closed surface,
it has to be an integer from thes ame argument I gave before.

If we look at hte $\theta$-dependence,
\begin{align}
    \frac{\delta H}{\delta \theta_i} &=
    \frac{\Phi_0}{L_i} \frac{1}{2\pi}
    \underbrace{%
    \int d^2x\,
    J_i
    }_{\text{0-momentum part of current}}
\end{align}
and so we will be able to relate the Chern number
to some current-current corelation function,
which shoud be related to the Hall conductivity.
In fact,
you can prove that the Hall conducivity is
\begin{align}
    \sigma_H &=
    \frac{e^2}{h} \frac{1}{2\pi} F_{xy}(\theta_x, \theta_y)
\end{align}
The proof is in the homework.
The relation between the Hall conductivity and the Chern number is that if you
average the Hall conductivity over the space of wave funtions,
the you find
\begin{align}
    \langle \sigma_H \rangle
    = C \frac{e^2}{h}
\end{align}
that is,
the average over twisted boundary conditions.
In fact,
you find
\begin{align}
    \sigma_H &= \langle \sigma_H \rangle_{BC's}
\end{align}
People understood you get thse two formulae
and you should expect the Hall conducitvity is an average over boudnar
yconditions,
and witha gapped system,
no quantity should care what the flux is thorugh these non-contractible cycles,
in the infinite system size,
because there is a finite-correlation lngh.
If hte cycle is large,
locally you need some coherenetnce thorugh the yccle,
so you need some correlation length that scales with teh size of the system.
Intuitively,
people expected
but in a tour-de-force of mathematical physics a decade,
this was proven about 2010 by Hastings and Michalakis.

If you think of inserting flux,
or you can do a singular gauge transformation,
remove the flux,
and do some twisted boundary conditions,
and the twist is the $\theta$.
You can relate
\begin{align}
    \psi\left( r_1, r_2, \ldots, r_N \right) &=
    e^{i\theta_x}\psi\left( r_1 + L_x \hat{x}, r_2,\ldots \right)
\end{align}
and the point is,
that this is related by a singular gauge transformation.
I'm not spelling out the whole story, because I assume you know already.

The main lesson,
is that this way of defining Chern number
gives a way to define an integer invariant in a many-body system.
We can define a Chern number for any $U(1)$ symmetry,
because the moment I have a U(1) symmetry,
I can turn on a background gauge field,
and insert flux through that torus,
and define a symmetry associated with it.
So every U(1) symmetry can have a  Chern number associated with it,
and that gives a way to define a U(1) symmetry in a gapped many-body system.
And there's few things I want to emphasize here.

This thing is intrinsically many-body.
Not a single free fermions,
not assuming translationally invariant.
So not band theory.

We also assumed implicitly that the system has a unique ground state.

This definition of the Chern number so far assumes you have a unique ground
state.
We're adiabatically inserting flux,
but in some system like fractional quantum hall effect,
if you insert integer flux,
you actually wind to a different ground state,
so you need many flux quanta to wind up in the same ground state.
So here with $2\pi$ flux,
you end up with the same exact ground state.
We get this Berry phase by moving around in one direction,
and if you consider may directions,
you get a single ground ate.

We can still define a Chern number with degenerate ground sates,
but there is an index here,
and this is a non-Abelian gauge field,
and this field strength because a non-Abelian gauge field,
and we need some extra fields,
and the Chern number is the trace of that field strength.
You can still define Chern number,
but it's not a U(1).
I won't say anything more.
You can do it,
but you need to consider non-Abelian Berry gauge fields.

\begin{question}
    To go to a non-Abelian symmetry,
    how would we do that?
\end{question}
So now you're saying,
suppose we have $SU(2)$ symmetry,
can you still define a Chern number?
The quick way to answer this question is,
Chern number we associated with an effective action
and the coefficient was the Chern number.o
With SU(2),
you turn on non-Abelian SU(2) gauge fields,
then write down a non-Abelian CS theory,
and that will also have a coefficient that is also a Chern number.
There will be a Chern number,
but to define it,
you're going to consider.
Then what you do is insert flux associated with some U(1) subgroup of
SU(2)
Just compute it for some U(1) subgroup.
I said that quickly,
because I didn't explain.

\begin{question}
    what about discrete groups instead of U(1)?
\end{question}
We can write down
\begin{align}
    S_{CS}[A] = \int \frac{C}{4\pi} A\, dA
\end{align}
but this doesn't make sense if you have some net flux through your manifold,
like a net magnetic monopole,
you can't globally define $A$,
and make sense.
But here is a remedy,
and it is to go to group cohomology.
I'll just stop there.
Once you view the Chern-Simons theory through that more appropriate lens,
then that framework you can easily extend to discrete groups.

\begin{question}
    Why not glue the state to $SU(2)$ state at the boundary?
\end{question}
Let me say just a bit more about this $SU(2)$ case
For $SU(2)$ symmetry
you would consider SU(2) gauge fields.
And the effective action would be some integer
\begin{align}
    S_{\text{eff}} &=
    \frac{C}{8\pi}
    \int \Tr\left( 
    A\wedge dA
    + \frac{2}{3} A \wedge A \wedge A
    \right)
\end{align}
If the goal was just to know $C$,
you can pick $A$ to live in some $U(1)$ subgroup,
evaluate the action,
and figure out what this coefficient is.
You know the whole thing si $SU(2)$ invariatn,
so you can just pick $A$ to be in a particular $U(1)$ sugroup,
then evaluate to figure out what the coefficient is.

In general,
you could imagine you have some $SU(2)$ flux here,
but you can do an $SU(2)$ rotation,
so this $\theta_x$ lies in some $U(1)$ subgroup,
but then you have to wrory about thsi $\theta_y$.
It's not the most general thing to do,
to figure out Chern number,
just have $A$ lie in some subgroup,
and evaluate the subgroup frmo there.
Once you figured out hte Chern number,
there is only one possible Chern number.

I'm just giving you a trick to calculate the Chern number in the $SU(2)$ case.
It doesn't map analogously.

In general,
you could imagien in the $x$ direciton,
yocuould have a genera l$SU(2)$ matrix,
and you tyr to cnostruct something like the $A_j(\theat_x,\theta_y)$ formula.
But I haven't thought about tha procedure.

\begin{question}
$C$ here can be any integer,
but is it still integer for $SU(2)$?
\end{question}
The CS theory for $U(1)$,
we usually write as
\begin{align}
    S_{eff}^{U(1)} &=
    \frac{C_{U(1)}}{4\pi}\int A\, dA
\end{align}
but for $SU(2)$,
\begin{align}
    S_{\text{eff}} &=
    \frac{C}{8\pi}
    \int \Tr\left( 
    A\wedge dA
    + \frac{2}{3} A \wedge A \wedge A
    \right)
\end{align}
Note the $4\pi$ vs $8\pi$.
For bosonic systems,
\begin{align}
    C_{U(1)} &\in 2\mathbb{Z},\\
    C_{SU(1)} &\in 2\mathbb{Z},
\end{align}
But for fermionic system,s
\begin{align}
    C_{U(1)} &\in \mathbb{Z}\\
    C_{SU(1)} &\in \mathbb{Z}
\end{align}

\section{Thouless Pond}
Let's view the Chern number from polarization in 1D.
Recall if we put our system on a cylinder,
and insert flux through the cylinder.
Let's say we have some $\theta_x = \Phi$ flux.
And the cyclinder is alnog the $x$ direciton.
This causaes an electric field in the $y$ direciton,
which auses a current $j_x$.
Charge is flooiwng alon the $x$ direciton,
and if you view htis as 1D system,
the polarization is changing with time that gives rise to the current.
Think about the Chern number in terms of the polarizaiton of this effecively 1D
system.

That's the perspective,
and see how it relates to the Chern number in 2D.

Let's start off discussing dimensional reduction of a 2D tight binding model.
Suppose we have a tight-binding model on a cylinder.
$x$ is alnog, $y$ is around the cylinder,
because $y$ is periodic, $k_y$ is a good quantum number.

It's a cylinder in the sense there are boundaries at the two ends
in the $x$ direction.
We can do a fermion operator,
with band $\alpha$.
\begin{align}
    C_{k_{y,\alpha}} &=
    \frac{1}{\sqrt{L_y}}
    \sum_y C_{\alpha}(x, y)
    e^{i k_y y}
\end{align}
the Hamiltonian is
\begin{align}
    H &= -\sum_{ij} t_{ij} c_i^\dagger c_j + \textrm{h.c.}\\
    &= \sum_{k_y} H_{1D}[k_y]
\end{align}
It's a sum of independent 1D systems where $k_y$ is just some parameter now.
You can think of your system as a bunch of $L_y$ different 1D chains.
For this 1D system,
$k_y$ is effective just some number.

Let's insert flux.
\begin{align}
    A_y &= - E_y t = \frac{\Phi(t)}{L_y}\\
    A_x &= 0
\end{align}
where $E_y$ is the electric field.
You can think of it as.
\begin{align}
    H &= \sum_{k_y} H_{1D}\left( k_y + A_y \right)
\end{align}
and this causes a current,
and remember the Hamiltonian is decoupled in $k_y$,
\begin{align}
    J_x &=
    \sum_{k_y} J_{1D}(k_y).
\end{align}
Let's calculate the charge flowing across the cylinder
\begin{align}
    \Delta Q &=
    \int_{0}^{\Delta t} dt\,
    \sum_{k_y} J_{1D}(k_y)
\end{align}
Now,
what's happening is,
you can think of this $J_{1D}(k_y)$,
as having some polarization,
and the current is the time derivative of this 1D system.
\begin{align}
    J_{1D}(k_y) &=
    \frac{d P_{1D, x}(k_y)}{dt}
\end{align}
so a changing polarization leads to a current,
charged being pumped in that direction.
And this is just
\begin{align}
    \Delta Q &=
    \int_{0}^{\Delta t} dt\,
    \sum_{k_y} J_{1D}(k_y)
    =
    \left.\sum_{k_y} \Delta P_x (k_y)\right|_{0}^{\Delta t}\\
    &=
    \left.\frac{L_y}{2\pi} \int_{0}^{2\pi} dk_y\,
    \Delta P_x(k_Y)\right|_{0}^{\Delta t}
\end{align}
And we're inserting flux in the adiabatic limit,
and say we insert one flux quanta.
\begin{align}
    E\, \Delta t &=
    \frac{2\pi}{L_y}
\end{align}
and here
I'm assuming all units are 1.
And the change in polarization is just going to be
\begin{align}
    \Delta P_x (k_y) &=
    P_x\left( k_y + \frac{2\pi}{L_y} \right)
    - P_x(k_y)
\end{align}
and in the limit,
this becomes
\begin{align}
    \Delta P_x (k_y) &=
    \frac{dP_x}{dk_y} \frac{2\pi}{k_y}
\end{align}
and what we leran is that the $2\pi/L_y$ factors cancel, so
\begin{align}
    \Delta Q &=
    \int_{0}^{2\pi} dk_y\,
    \frac{dP_x}{d k_y}
\end{align}
and we can just htink of this as a look integral in the Brilluoin zone.
\begin{align}
    \Delta Q &=
    \oint dk_y\,
    \frac{dP_x}{d k_y}
\end{align}
and you can just think of $\Delta Q$
as the winding number of the polarization,
which is equal to the charge pumped across the system.

\begin{question}
    Between the time and frequency arguemnts,
    $\Delta P$ should be the difference in polarizaiton at different times?
\end{question}
The current is the change in polarization,
but the reason it's change in time is becake $k_y(t)$ is in time.
\begin{align}
    J_{1D}(k_y) &=
    \frac{dP_x\left( k_y(t) \right)}{dt}
\end{align}

\begin{question}
    What's the physical meaning of polarization winding?
\end{question}
Physically,
this is what's going on.
Physically,
take the states in a given Chern band,
and because our band has a Chern number,
we cannot write localized Wannier functions,
but they can be paritally localized Wannier functions.
Physically, localized wannier functions for the Chern band
$\ket{W(k_y, x)}$.

As $k_y$ increases,
these Wannier functions are shifting in position,
the states get shifted in $x$,
specifically let's look at the expectation values
\begin{align}
    \bra{W(k_y,x)} \hat{x} \ket{W(k_y, x)}
    = x + P(k_y)
    = \bar{x}_{k_y, x}
\end{align}
Some extra steps of algebra area needed to show this.
we have partially.
The fact the polarization winds
means that if you tried to plot this average
$\bar{x}_{k_y, x}$ vs $k_y$,
it means that if you start at lattice site 0,
you end up at lattice site $C$ after $k_y$ goes up by $2\pi$.
There are two ways of saying it.

What it's saying is that the average positiion of this partially localized
Wannier function changes by $C$ units
\begin{align}
    \bar{x}_{k_y + 2\pi, x} &=
    \bar{x}_{k_y, x} + C
\end{align}
Alternatively,
after inserting $2\pi$ flux,
this average changes by $C$ windings.
At each particular poin,
the polizariotn has changed,
and the net change in polairziaton is just hte net charge that goes from one
side toe another.
This is why you relate $J$ and $P$.

\begin{question}
    What's the picture of the winding number of $P$ around $y$?
\end{question}
The winding
\begin{align}
    \oint dk_y\, \frac{dP}{dk_y}
    &=
    p(k_y + 2\pi) - P(k_y)
\end{align} shift by $2\pi$ and see how much polarization changes.


\begin{question}
    If $C$ is zero,
    we expect this to be locallized?
\end{question}
No,
if $C=0$,
there's no obstruction to write down localied Wannier funcfions,
but we can still write paritally lcoalized Wannier functions.
As we change $k_y$,
these positions would be shifted over by $C$,
which means they don't shift over at all.


\begin{question}
    These Wannier functions are not deinite combinatinos of Bloch states?
\end{question}
You cna think of these nsulators as filling some single-articulare stares,
but which basis you think in is up to you.
You alwasy have the right ot think in the basis they are paritally localized.
winding literally means graduatlly


A nice exercise is to write down the Wannier functino in terms of the Bloch
states of the bands,
and tune the superpsoitiosn so you maximize the lcoalizeation o the wannier
function.

\begin{question}
    Do we have a known quantity for the lcoalization?
\end{question}
For the moment forget about Chern numbers entirely,
and just think about 1D systesm.
More generally,
consider some 1D system that depends on some apramter $\theta$.
\begin{align}
    H_{1D}(\theta)
\end{align}
If $\theta$ varies with time,
we could get some current in this system.
Because $\theta$ is the only thing changing in this setup,
the current is going to be some response linear function $G(\theta)$ times the
derivative.
\begin{align}
    J(t) &= G(\theta) \frac{d\theta}{dt}
\end{align}
and this is varying adiabatically in time,
so we can apply linear response,
as the sytem is always in equilibrim.
Now the current,
is almost by definition
\begin{align}
    J(t) &= \frac{dP}{dt},
\end{align}
but we can also think of it in terms of
\begin{align}
    J(t) &= \frac{dP}{dt}
    = \frac{\partial P}{\partial \theta} \frac{d\theta}{dt}
\end{align}
which means
\begin{align}
    G(\theta) = \frac{\partial P}{\partial \theta}.
\end{align}
what you find for the tight-binding model is that you get a clean expression
\begin{align}
    G(\theta) &=
    \oint \frac{dk_x}{2\pi}\left( 
    \frac{\partial A_x}{\partial \theta} - \frac{\partial A_\theta}{\partial k_x}
    \right)
\end{align}
where this $A_\theta$ is the Berry connection,
with
\begin{align}
    A_x &=
    -i \bra{u(k,\theta)} \frac{\partial}{\partial k_x}
    \ket{u(k,\theta)}\\
    A_\theta &=
    -i \bra{u(k,\theta)} \frac{\partial}{\partial \theta}
    \ket{u(k,\theta)}\\
\end{align}
and these $u$ are Bloch wave functions.
It should look familiar to you.
This is just the field strength of this Berry connection in this
$(k_x,\theta)$-space.
Now interestingly,
if $\theta$ is a periodic parameter so that
\begin{align}
    H(\theta + 2\pi) = H(\theta)
\end{align}
then ifwe take the inetegral,
\begin{align}
    \frac{1}{2\pi}\int d\theta\, G(\theta) &=
    C
\end{align}
where $C$ is some integer Chern number.
This is an amazing cool thing.
You have a family of 1D systems,
you try to look at the current floiwng through the system,
and that current is dteermined by a response function,
and that respnose fucntion
is just het integral of the field strength of the Berry connection.

So to do one mofre thing.
If we pick a gauge where $A_\theta$ is single-valued,
\begin{align}
    G(\theta) &=
    \frac{\partial}{\partial\theta}
    \oint \frac{dk_x}{2\pi} A_x
    = \frac{\partial P}{\partial \theta},
\end{align}
so we learnt something cool:
$P$ itself is equal to the Berry phase.
\begin{align}
    P &=
    \oint \frac{dk_x}{2\pi}A_x
\end{align}
The poliaziaton is taking the Berry conntion,
and looking at hte holonomy of the Berry gauge field in momentum space.
This is the so-called
\emph{Berry phase theory of 1D polarization}.
Once caveat is that this is only well-defined mod integer.
Because,
I can always do a gauge transformation of my insgle-aprticle gauge funtions,
which chagnes the flux thorugh the BZ by $2\pi$.
That is,
this is only well-defined mod $\mathbb{Z}$,
adn we can do large gauge transfomrations to chagne
$\oint_{k_x} A$ by $2\pi$.

This miplies that when $\theta$ is periodic,
\begin{align}
    \Delta Q &=
    \int J\, dt\\
    &= \int G(\theta) \frac{d\hteta}{dt}dt\\
    &=
    \int G(\theta)\, d\theta\\
    &= C \in \mathbb{Z}
\end{align}
So if we do a loop in paratere psace,
we pump charge,
and that can be related to the Chern number,
arising frmo the 1 extra dimension $\theta$.

So topological pumps in 1D are tightly connected to Chern numbers in 2D.

The cna also relate the charge density to the polarization.
The current is the time derivative of the charge,
but if you havet he gradient of the polarization,
that means you have some charge density somehwere.

By the continuity equation,
\begin{align}
    \frac{d\rho}{dt} &=
    -\frac{dJ}{dx}
    =
    -\frac{\partial^2 P(\theta)}{\partial x\, \partial t}
\end{align}
and that tells us the charge density is
\begin{align}
    \rho &=
    - \frac{\partial P(\theta)}{\partial x}
\end{align}
and combining iwth
\begin{align}
    J &= \frac{\partial P}{\partial t},
\end{align}
we get
\begin{align}
    i_{\mu} &=
    \epsilon_{\mu\nu} \frac{\partial P}{\partial x_\nu}
\end{align}
from which we get an effective action
\begin{align}
    S_{\text{eff}} &=
    \int dx\, dt\,
    P e^{\nu \mu} \partial_\nu A_\mu
\end{align}
which means
\begin{align}
    S_{\text{eff}} &=
    \int P\, F_{xt}.
\end{align}
Integrating by parts,
we can confirm that
\begin{align}
    j_{\mu} &=
    \frac{\partial S_{\text{eff}}}{\delta A_\mu}
    = -\epsilon^{\mu\nu} \partial_\nu P
\end{align}
So the Berry phase of the Bloch wave functions as you go around in momentum
space in 1D.
You can think of polairziaton in 1D,
where the effective action is literally just the field strength of $A$.
Finally,
there is a tight relation between pumps in 1D and chern numbers in 2D.

\begin{question}
    Does the correspondence carry in higher dimesnions?
\end{question}
There's a Simons cllaboartion caleled ultra-quantum matter.
There's an ultra quantum theory of polarizaiton.
Andy ou can generalize this to higher dimensions,
ubt it's a research topic.
What enters is you need translation gauge fields in higher dimensinos,
guage fields assocatied with tarnslation symmetry,
and then there's a natural way of witing down this higher dimnsional verios of
this effective action.

\begin{question}
    Why don't we need them now for 1D?
\end{question}
I think it's because in 2D you can go around in a loop,
but in 1D thereis no loop except for the toal loop,
so there are no small loops.
In some sense,
you can get a way with less structure in 1D,
because it just has a lot less structure.

\begin{question}
    How to generalize Chern numbers to 3D?
\end{question}
Several modern things have happend literally in the last few years.
ify ou want to geliraze the pump 1D to Chern number 2D,
therae ra few aperps about amoalies in thespace of coupling constants.
Ify ou chagne $\theta$,
it's not eactly invairatn,
and that's averison fo an anomaly.
You can recase everything that happened here more abstratly.

There's an orthoagonal set of things happend,
that is genralis the theroy of polarization to higher dimesnison.

Ther'es not really a connection betwen those two things,
but maybe that oculd e a resaerch project.

\section{Many-body definition of Chern number}
Let's go back ot the many-body defimtiion of chern number
using our knowledge of polarization in 1D.

Suppose we havea torus.
Consider $\oint A_y = \theta_y$.
Then we have the gorund state $\ket{\psi(\theta_y)}$.

Let me define the exponentiated polarization operator
\begin{align}
    R_x :=
    \prod_{x, y}
    e^{i \frac{2\pi x}{L_x} \hat{n}(x, y)}
\end{align}
and then I want to define
\begin{align}
    \mathcal{T} &=
    \frac{\bra{\psi(\theta_y)} R_x \ket{\psi(\theta_y)}}{%
    \braket{\psi(\theta_y)}{\psi(\theta_y)}
    }
\end{align}
The polarization is $\arg \mathcal{T}(\theta_y)$.
Then the Chern number is
\begin{align}
    C &= \frac{1}{2\pi} \oint d\hteta_y\,
    \frac{d}{d\hteta_y} P(\theta_y)
\end{align}
This is differnt to twisted boundary conditions defiiton.
that was twitisting in x and y,
constructing a berry connection, field strength.
But here,
I'm only inserting $\theta_y$
and seeing how the polarization winds.

The important thing to note about this formula is that before knowing the wave
function as a function of both $\theta_x$ and $\theta_y$,
we only consider the wave function has a function a function of 1 parameter,
$\theta_y$.

\begin{question}
    What's $R_x$?
    What's the meaning of it?
\end{question}
If you write down the formula for polarization in 1D,
it's just
\begin{align}
    P &\propto \sum x n(x)
\end{align}
If you have a 2D system,
\begin{align}
    P &\propto  \sum_{x, y} x n(x, y)
\end{align}
So this $R_x$ you can think of as an exponentiated polarization
\begin{align}
    P_x :\propto e^{i P}
\end{align}

\begin{question}
    What is the minimum number of ground states to find the Chern number?
\end{question}
I'm glad you asked.
The original definition uses the wave function in terms of 2 parameters.
We dropped than down to 1 parameter here.
It turns out you can drop it down to not depend on any parameters.

We can also extrat many-body Chern numbers fom a single GS wave funtion.

Suppose we have a cylinder iwth axis in $x$,
winding in $y$.
Then we have cylinder regions $R_1$, $R_2$, $R_3$, \ldots
and $l_y$ is the lenght in the $y$-direction.
This is pretty amazing
\begin{align}
    \mathcal{T}(\phi) &=
    \bra{0}
    W_{R_1}^\dagger(\phi) 
    \mathrm{SWAP}_{1, 3}
    W_{R_1}(\phi)
    V_{R_1 \cup R_2}
    \ket{0}
\end{align}
where
\begin{align}
    W_R &= \prod_{(x, y)\in R} e^{i\hat{n} (x, y) \phi}\\
    V_R &=
    \prod_{(x, y)\in R} e^{i \frac{2\pi y}{l_y} \hat{n}(x, y)}
\end{align}
then the Chern number is
\begin{align}
    C &= \frac{1}{2\pi} \oint d\phi\,
    \frac{d}{d\phi} \arg\mathcal{T}(\phi)
\end{align}
We have a paper on this.
You think in terms of TQFT and cutting and gluing.
There are open questions about this.
We dropped down from 2 params, 1 param, to 0 params.
But this only works on a cylinder.o
Can we get it on justa patch of space?
Empriically,
this formula works even with patches.

The question is how much topological can you extract from a single ground state
wave function on a disc.
We don't know how to do it,
complexity,
algorithsm,
etc.

\begin{question}
    You've transfoered the parameter ot the operator?
\end{question}
Whether this is really a win depends.
I think it's a win,
it's a matter of if a single ground state wave functino contains all the
topology.

Also,
the homework is due on Monday.

\section{Lecture 14}
Sorry, I missed this one!

\section{Hamiltonian Mechanics}
Recall hte Lagrangian is
\begin{align}
    L = T - U
\end{align}
which is a function $L(q_1,\ldots,q_n, \dot{q}_1,\ldots,\dot{q}_n)$.
Let use define
\begin{align}
    p_i &= \frac{\partial L}{\partial q_i}
\end{align}
where $p_i$ is called the \emph{canonical momentum} or momentum conjugate to
$q_i$.
The \emph{Legendre transform} is
\begin{align}
    H &= \sum_i p_i \dot{q}_i - L
\end{align}
where $H=H(q_i, p_i)$.

The $n$ q_i$ and $n$ p_i$ define a point in a $2n$-dimensional space,
called \emph{phase space}.
Hamilton's equations determine a unique path in phase space,
starting from any initial point.

Let's derive Hamilton's euqations for a 1D system.
\begin{align}
    H &=
    p\dot{q}(q, p) - L\left( q, \dot{q}(q, p) \right)
\end{align}
The point is that $\dot{q}$ is determined implicitly in terms of $q$ and $p$
by the definition of $p$.
\begin{align}
    \left(\frac{\partial L}{\partial \dot{q}}\right)_{q} = p.
\end{align}
where we are holding $q$ fixed in the derivative.
Let's calculate the derivative
\begin{align}
    \left( \frac{\partial H}{\partial q} \right)_p
    &=
    p \left( \frac{\partial\dot{q}}{\partial q} \right)_p
    - \left[ 
    \left( \frac{\partial L}{\partial q} \right)_p
    \underbrace{\left(\frac{\partial q}{\partial q}_\right)_p}_{1}
    +
    \underbrace{\left( \frac{\partial L}{\partial q} \right)_q}_{p}
    \left( \frac{\partial \dot{q}}{\partial q} \right)_p
    \right]
\end{align}
It's just an exercise of partial derivatives.
You see that one of the terms cancel.
So using Lagrange's equation,
\begin{align}
    \left( \frac{\partial H}{\partial q} \right)_p
    &=
    - \left( \frac{\partial L}{\partial q} \right)_{\dot{q}}\\
    &=
    -\frac{d}{dt}\left( \frac{\partial L}{\partial \dot{q}} \right)_{q}\\
    &= -\frac{dp}{dt}
\end{align}
where it's true on a classical path

Similarly,
\begin{align}
    \left( \frac{\partial H}{\partial p} \right)_{q}
    &=
    \dot{q}
    +
    p \left( \frac{\partial \dot{q}}{\partial p} \right)_{q}
    -
    \left[ 
    \left( \frac{\partial L}{\partial q} \right)_{\dot{q}}
    \underbrace{\left( \frac{\partial q}{\partial p} \right)_{q}}_{0}
    +
    \underbrace{\left( \frac{\partial L}{\partial \dot{q}}
    \right)_{q}}_{p}
    \left( \frac{\partial \dot{q}}{\partial p} \right)_{q}
    \right]\\
    &= \dot{q}
\end{align}
And so we have derived Hamilton's equations
\begin{align}
    \left( \frac{\partial H}{\partial q} \right)_{p} &= -\dot{p}\\
    \left( \frac{\partial H}{\partial p} \right)_{q} &= -\dot{q}
\end{align}
These are first order equations.
This tells you how a point $(q,p)$ changes with time in phase space.

In the Lagrangian approach,
fora system with one degrees of freedom,
we obtain a single second order equation.
In the Hamiltonian approach,
we obtain two first order equations.

\subsection{Atwood's Machine}
Once again,
we consider the Atwood's machine as an example.

We have a string on a pulley with two masses on its end
$m_1$ and $m_2$.
The length of the string on the $m_1$ is  $x$
and the length of the string on the $m_2$ side is $y$.
The constraint is that
\begin{align}
    x + y = \textrm{constant}
\end{align}
which means the string is inextensible,
so if one goes up the other goes down.

The Lagrangian is
\begin{align}
    L &=
    \frac{1}{2}m_1\dot{x}^2
    +
    \frac{1}{2}m_2 \dot{y}^2
    - \left( 
    -m_1 gx
    - m_2 gy
    \right)\\
    &=
    \frac{1}{2}m_1\dot{x}^2 + \frac{1}{2}m_2\dot{y}^2
    + m_1 gx + m_2gy\\
    &=
    \frac{1}{2}\left( m_1 + m_2 \right) \dot{x}^2
    +
    \left( m_1 - m_2 \right) gx
\end{align}
Then we can calculte the momenta
\begin{align}
    p &= \frac{\partial L}{\partial x}\\
    &=
    \left( m_1 + m_2 \right)\dot{x}
\end{align}
and write the velocity in terms of momentum
\begin{align}
    \dot{x} &= \frac{p}{m_1 + m_2}
\end{align}
which we can substitute into the Lagrangian
\begin{align}
    L &=
    \frac{1}{2} \frac{p^2}{m_1 + m_2}
    +
    \left( m_1 - m_2 \right)gx
\end{align}
so the Hamiltonian is
\begin{align}
    H &= p\dot{x} - L\\
    &=
    p \frac{p}{m_1 + m_2} - L\\
    &=
    \frac{1}{2}\frac{p^2}{m_1 + m_2}
    -
    \left( m_1 - m_2 \right)gx
\end{align}
Using Hamilton's equations
\begin{align}
    -\dot{p} &= \frac{\partial H}{\partial q}\\
    \dot{x} &= \frac{\partial H}{\partial p}
\end{align}
you can convince yourself it gives
\begin{align}
    \dot{p} &= \left( m_1 - m_2 \right)g\\
    \dot{x} &= \frac{p}{m_1 + m_2}
\end{align}
Differentiate the second equation with respect to time and we gte
\begin{align}
    \ddot{x} &=
    \frac{\left( m_1 - m_2 \right)g}{m_1 + m_2}
\end{align}

You could of course get the same differential equation from the Euler-Lagrange
equations in the Lagrangian approach.
\begin{align}
    \frac{d}{dt}\left( \frac{\partial L}{\partial \dot{x}} \right)
    - \frac{\partial L}{\partial x} &= 0\\
    \frac{d}{dt}\left[ 
    \left( m_1 + m_2 \right)x
    \right]
    -
    \left( m_1 - m_2 \right)g
    &= 0\\
    \ddot{x} &=
    \frac{\left( m_1 - m_2 \right)g}{m_1 + m_2}
\end{align}
Why bother with Hamilton's equations with a fraction of the work.

After Hamilton came up with these equations they made him a professor at Dublin
when he was just an undergrad!
So what is the great insight of Hamilton?

The minus sign changes the symmetry from the obvious one into a more subtle
symmetry.
If you are a theoretical physicist,
you see they have a symplectic geometry.
The Hamiltonian formalism leads to a symmetry between the $p$ and $q$'s.
You know how much easier it is to solve a system if it has the right coordinate
system?

Now the fact is,
there is a symmetry between the coordinates and momentum,
which means you have the option to choose coordaintes which are combinations of
coordaintes and momenta.
You can make transformations between coordaintes and moemnta together,
which increases the choics of possible coordinates,
with so much more freedom to choose coordinates.

There are problems you can solve using thismethod no one knows how to solve
otherwise.
It's because of this greater flexibility.
Unfortunatley,
there's no problem you will lsolve in this course where this is actually true.
That's why it's important even in classical mechanics.
These transformations are called \emph{canonical transformations}.

\section{Phase space orbits}
The generalization of Hamilton's equations to many degrees of freedom is
\begin{align}
    \dot{q}_i &= \frac{\partial H}{\partial p_i}\\
    -\dot{p}_i &= \frac{\partial H}{\partial q_i}
\end{align}
You will need to remember these equatinos,
but it's pretty easy to remember,
just need to know where the sign is.
If you can't remember,
just consider the free particle with Hamiltonian
$H=p^2/2m$ and so $\partial H/\partial q = p/m$ and $\dot{q}=p/m$.

\begin{align}
    \dot{q}_i &= \frac{\partial H}{\partial p_i} = f_i(q_j, p_j)\\
    -\dot{p}_i &= \frac{\partial H}{\partial q_i} = g_i(q_j, p_j)
\end{align}
Introduce the $2n$ dimensional vectors
\begin{align}
    \vec{z} &=
    \begin{pmatrix}
        q_1\\
        \vdots\\
        q_n\\
        p_1\\
        \vdots\\
        p_n
    \end{pmatrix}
\end{align}
and
\begin{align}
    \vec{h} &=
    \begin{pmatrix}
        f_1(q_j, p_j)\\
        \vdots\\
        f_n(q_j, p_j)\\
        g_1(q_j, p_j)\\
        \vdots\\
        g_n(q_j, p_j)
    \end{pmatrix}
\end{align}
so we can write all of Hamilton's equations as
\begin{align}
    \dot{\vec{z}} &= \vec{h}(\vec{z}).
\end{align}
The vector $\vec{z}$ defines the ``position'' of the system in phase space.

There's a $2n$-dimensional phase space.
There's one dimension for every $p$ and one dimension for every $q$.
If you know what $z$ is,
you know the particle position in phase space at any given time,
and this equation tells you how the particle moves in phase space as a function
of time.

Let's say you have some trajectory on phase space.
Because the vector $\dot{\vec{z}}$ is unique for a given $\vec{z}$,
there should be no crossing of paths in phase space.

\subsection{Harmonic Oscillator in 1 dimension}
The Lagrangian is
\begin{align}
    L &= \frac{1}{2}m\dot{x}^2 - \frac{1}{2}kx^2
\end{align}
so the Hamiltonian is
\begin{align}
    H &= p\dot{x} - L\\
    &= \frac{p^2}{m} - \left[ 
    \frac{1}{2}m\left( \frac{p}{m} \right)^2
    - \frac{1}{2}kx^2
    \right]^2\\
    &=
    \frac{1}{2}\frac{p^2}{m}
    + \frac{1}{2}kx^2
\end{align}
so Hamilton's equations gives
\begin{align}
    \frac{\partial H}{\partial p} &= \dot{x}
    &\implies
    \frac{p}{m} &= \dot{x}\\
    \frac{\partial H}{\partial q} &= -\dot{p}
    &\implies
    kx &= -\dot{p}
\end{align}
Eliminating $p$,
we get
\begin{align}
    m\ddot{x} &= -kx
\end{align}
and if we define $\omega^2 = k/m$,
we get the solution
\begin{align}
    x &= A \cos(\omega t - \eta)\\
    p &= m\dot{x} = - mA\omega \sin(\omega t - \eta)
\end{align}
The phase space for the harmonic oscillator is the 2D space with coordinates
$(x,p)$.
As time changes,
the particle traces out a path in phase space,
called the \emph{orbit} in phase space.
Despite the name, it's generally not a closed path.

For a Harmonic oscillator in 1 dimensions,
the orbit is an ellipse
\begin{align}
    \frac{x^2}{A^2} + \frac{p^2}{mA^2\omega^2} &= 1
\end{align}
and it's going clockwise.

\section{About the Midterm}
I've written the exam already.
If you did the homework,
understand how to solve those problems,
it should be quite straightforward to get 50\% very easily.
I think it's an exam where to get 100\% is not easy,
to get 90\% is tough.
Some questions are straightforward,
but some are had.
All carry equal points.
It's not that easy questions are first.
You can read the exam,
recognise the questions.
Make sure you do the ones you know how to do.
In the end,
the most important thing is to pass.
Nail all those you know.
Once you're confident they're right,
spend time on the tough ones.

You need to memorize the formula for the Method of Steepest descent.
There is no cheat sheet.

It's not like the qualifier,
where you need to know
all of classical, all of quantum, and you have to pass them all at once,
or some BS like that.
Trust me, the midterm is much easier than being tested on all of physics.

The questions are like this.
2 from the first homework, 2 from the second homework and 2 from the last.
All questions are are of equal weight.
On Thursday, try to get here early!
Class is 9:30 to 11:20 but try to get here early.
I will hand out the question papers early enough so you begin exactly at 09:30.
You can look at the question paper before that,
but you can only start writing at 09:30.

I will try to provide paper,
but just in case bring your own.

I suggest spending time doing the homework problems,
make sure you can absolutely do all the homework problems.
Once you've done that,
look at all the worked examples I did in class.
That's basic preparation.
If you have time beyond that,
there's all those books,
some of them have problems,
look through the problems and see if you know how to do those problems.

Do we need a calculator?
I don't think you should need a calculator.

Are we scanning or handing in physical paper at the end?
I'm thinking of handing in physical paper,
because it will take you 10 minutes of scanning.

A calculator might be useful for binomial coefficients,
but you can calculate the first 2 or 3.
You'll get some points for sure.

Will we be given the contours to integrate over?
Choosing contour shapes, should we learn?
Look very carefully through all the homework problems.
I'm not going to tell you which contour to choose,
you're going to have to figure it out yourself.
That's important.

Will the mapping for mapping problems be provided?
The choice of mapping will be provided.

Also, one last thing, there's a new set of notes online.

\section{Lecture 16}
In (1+1)D there are two types of anomaly.
Chiral anomaly $U(1)$, characterised by some integer, the Chern number.
Gravitational anomaly, characterised by Chiral central charge.
They are different,
but the minimal example of a theory with a chiral anomaly is a 1D complex
fermion.
A chiral fermion will have chiral anomaly.
When they say chiral anomaly,
they mean two different things.
If you have a left and right mover,
there is a symmetry corresponding to the left and right,
sometimes called the axial current.
The fact there's a chiral anomaly here
means that if you take two copies,
that anomaly will persist.
If you take a single chiral fermion,
and it has this gravitational anomaly.

If you have a general interacting CFT,
these don't have to be the same thing.

Hopefully I'll put up another homework assignment to cover what we've done since
last homework assignment.
We've been talking about topological insulators,
discovered by Charlie Kane who's here today.

Let me clarify some things I didn't explain,
then we'll go to 3D.

\section{2+1D case}
Here we have a topological insulator.
The fermionic symmetric group was
\begin{align}
    G_F &= U(1)^f \rtimes \mathbb{Z}_4^{T,f} / \mathbb{Z}_2
\end{align}
$U(1)$ is the same as fermion parity.
Time reversal also square the fermion parity.
This is class AII.

Then then are matrix elements of the time-reversal operator.
\begin{align}
    w_{m,n}(k) &=
    \bra{U_m(k)} \tau \ket{U_n(-k)}
\end{align}
and there are 4 time-reversal invariant momenta (TRIM) labelled $\Lambda_a$.

So we have $(0,0)$, $(\pi ,0)$, $(\pi, \pi)$, $(0, \pi)$.
Then we have $\mathbb{Z}_2$ topological band invariant.
\begin{align}
    \left( -1 \right)^{\nu} &= \prod_{a=1}^{4} \delta_a
    \prod
\end{align}
where
\begin{align}
    \delta_a &=
    \frac{\Pf(w(\Lambda_a))}{\sqrt{\det(w(\Lambda_a))}}
\end{align}
and we need to pick Bloch wave functions $\ket{u_n(k)}$ such that
$\sqrt{\det(w(k))}$ is continuous.

For 1D systems,
time-reversal polarization
\begin{align}
    (-1)^{P_T} &=
    \prod_{a=1}^{2}\delta_a = \delta_1 \delta_2
    = \delta_` \delta_2
\end{align}
When the time-reversal polarization is odd,
it has a Kramers degeneracy on the boundary,
and none if it's even.
A 1D system is not a well-defined thing,
because putting an electron at the end will ad a Kramers degeneracy.
When you do a gauge transformation the Bloch wave functions,
you can actually change $P_T$ from $0$ to $1$.

$P_T=0,1$ tells us if the boundary of a 1D system has Kramers degeneracy,
but there is a gauge transformation taking
$\ket{u_n(k)}\to e^{i\theta_k}\ket{u_n(k)}$.
So there is some phase change of the Bloch wave function that will change $P_T$
from $0$ to $1$.

There is an integer ambiguity of the charge polarization in 1D.
When we talk about Chern number,
the Chern number tells us how polarization changes affects the 1D changes.
If we had a 1D system,
if we insert flux,
the polarization as we insert flux changes from 0 to 1,
means we pump change from one end to another.
Adiabatic changes to polarization around a cycle are well defined and define
a topological pump.
Whether it's 0 to 1 in a given 1D system is not useful,
however if you think of a 2D system this way,
when you insert flux, or tune $k_y$,
the fact that the time-reversal polarization is changing between 0 and $\pi$
is the basic effect that describes what's going on here.
In equations,
this invariant is the same as
\begin{align}
    (-1)^{\nu} = (-1)^{P_T(k_y=0) - P_T(k_y = \pi)}
\end{align}
And it could be a plus.

What does it mean for the surface stats.
If $\nu$ is odd,
we have helical edge modes with counter propagating complex fermions as I said
last time.

Now, I want to explain the connection to the surface states a little more
clearly.

\section{Connection to Surface States}
The equality I didn't derive last time
\begin{align}
    (-1)^{\nu} = (-1)^{P_T(k_y=0) - P_T(k_y = \pi)}
\end{align}
I didn't describe,
so I might put that in your homework.
What does it actually mean?

We conclude that we have to have gapless surface states if the symmetry remains
intact.
Taking $k_y$ from 0 to $\pi$.
Remember $k_y + A_y$ enters the equation when we insert flux.
So $k_y + A_y$ is what enters the Hamiltonian.
So
\begin{align}
    A_y &= \Phi/L_y
\end{align}
Taking $A_y$ from $0$ to $\pi$ is equivalent to taking
$\Phi$ from $0$ to $\pi L_y$.
The cylinder is along $x$.

$2\pi$ flux is equivalent to no flux,
so this is just $\pi$ flux if $L_y$ is odd.
So take $L_y$ odd and consider inserting $\pi$ flux.
$\phi=0\to \pi$.

If we start with a Kramers singlet on a boundary,
we need to get to a Kramers doublet.
What this tells us is that if we start off with a non-degenerate ground sate,
we adiabatically insert flux.
If $\nu$ is odd,
we better get to a Kramers doublet.
That immediately tells us it's gapless?

Why?
Suppose it's gapped.
Then inserting flux can only change the ground state energy by an amount
$\Delta E \propto e^{-L_y/\xi}$.
So that means if it starts off in a Kramers singlet,
there's no way it could become a Kramers doublet.
It has to remain a Kramers singlet.

So if the boundary is gapped,
there are 2 possibilities.
Either this $\nu=0$ is trivial,
so a singlet stays singlet.
Or, we break time-reversal symmetry,
so it doesn't make sense to talk about Kramers doublet or singlet anymore.

If we break $U(1)$ charge conservation,
we have asuperconductor on the boundary,
inserting flux is well defined anymroe if you want to stay on the ground state
manifold.
If I insert flux, it gives lare enrgy cost.

We can conlude that if $\nu=1$ mod 2,
then either
\begin{enumerate}
    \item The boundary is gapped and symmetry-breaking.
    \item The boundary is gapless, so it can preserve the symmetry.
\end{enumerate}
If it's gapless it actually may or may not preserve the symmetry,
but the point is that if we want to preserve the symmetry,
the boundary must be gapless.
Remember this picture where I drew $E$ as a function of $k_y$,
where we had one bad down here and one bad up there.

At $k_y=\pi$,
they had to be Kramers degenerate,
and these Kramers pairs split at $k_y=0$.
There are two ways they can connect going from $k_y=0\to \pi$.
Either they can switch partners like this [picture].

Or they don't switch partners.
In the case of no switching pairs,
there is symmetry,
and if it's gapped, it's in conflict,
so it must be the pair switching case if $\nu$ is odd,
and not switching if $\nu$ is even.

This establishes a connection betwen the bulk nivariatn $\nu$ and the way the
edge modes connect on to the boundary,
which guarantees these robust edge states.

You could go further.
Suppose you had one of these situations,
can you convinceyouself that the insertion of $\pi$ flux would change the bands.

Now I'm going to draw the bnads from $-\pi$ to $\pi$.
Suppose your chemical potential is exactly at this point,
so htis state which isa Kramers doublet,
and bothsates in the doublet are occuppied.
And then all the states below that chemical potential are also occupied.
So then you can insert the $\pi$ flux,
which shifts all tese states over to hte right by $\pi/L_y$.
And we get a picture where one of these states goes up a little,
and hte other states shifts down.
But here there's an empty state.
So this guy becomes empty and this guy becomes filled.
Under $\pI$ flux,
these states hsfit over,
and you have occupied and unoccpied states.
This start is Kramers doublet because there's an exact anaolgous state with a
Kramers degeneracy with a degeneracy.
You can draw these diagrams in all scenarios and convince yourself that iserting
$\pi$ flux changes a state from an even number of doublets to an odd number of
doublets, and vice versa.
That makes the surface and bulk correspondnce even more explicit.

I wanted to fill in this hole last time.
I didn't make a connecino with suraace staes.

THis (1+1)D helical boudnary,
in the Chern number case,
we had a U(1) anomaly.
Something has ot cancel out the anomaly,
which si the bulk.
Here the anomaly is a mix between $U(1)$ and time-reversal.
If we want to preserve both of them so these gapless modes are robust,
that's a signature that we hav ea mixed anomaly between U(1) and TR symmetry.
The fact you insert $\pI$ flux and the Kramers degeneracy of the ground state
chagnes,
you can think of it as a symptom of the anomaly.
So the ground sate went from beinga Karmers singlet to Kramers doublet.

If you take this helical theroy,
noe way to see hwat's wrong is the study it on a spacteim that iss Klein bottle.
It's useful to put sapceteime on a non-orientable manifold,
it turns out you can also in that case see more things wrong and inconsistent
with the theory.
Just like the chiral fermion, thei nconsistency wwas curretn conservation breka
down.

Seeing anomalies wiht discrete ymmetries is a muc hmore complicated game poeple
only started to understand 10 years ago.
Studying on a Klein bottle is one thing you could do to se the anomaly.
But I'm not going to go down that path here.

So now let's go to (3+1)D.
We're still in Class AII and TI still squares to -1.
And here we have 8 TRIM,
which I write as
\begin{align}
    \Lambda_a=(n_1,n_2,n_3)
    =
    \frac{1}{2}n_1 \vec{b}_1
    + \frac{1}{2}n_2 \vec{b}_2
    + \frac{1}{2}n_3 \vec{b}_3
\end{align}
where the $b$s are primitive reciprocal lattice vectors and $n$s are iether 0 or
1.
This defines a 3D topological insulator.
The topological invariant is
\begin{align}
    (-1)^\nu &= \prod_{a=1j^{8}}\delta_a.
\end{align}
Consider a cubic lattice.
Consider a surface in the $z$ direction.
We can look at hte surface states.
At the time-reversal invaritn mometna,
at a stte in this TRIM,
we have Kramers degenercy,
but as soon as we go away,
the bands split.
So that means we have a Dirac node,
when we locally have some state at some energy.

So surface states at $(k_x,k_y)=(0,0)$, $(0,\pi)$, $(\pi, 0)$, $(\pi, \pi)$
come in Kramers pairs,
and that means we have Dirac cones every time we have states with these momenta.

We can draw a picture.
We have these different points here,
these TRIM,
and wheneter you have some state that comes iwth degeneracy,
there's a bunch of dirac cones when we hvae some staes at these moemnta.
For exmplae
at this momenta,
we have Dirac cones at vairous energis.
And just like this dimension,
just like here,
it was a question of how the states connect up.
Similarly here,
it's a uestion of how these diract cones all connect up with each other.
This index $\nu$ ultimately gives informatino about the way these cones connect
up to each other.

The fermi level could be anywhere.

The question is how do these cones connect up.
And basically hte answer is that if $\nu=1$.
Let me mention nadi mportant property of Dirac cones,
I hsould put on the homework.
If you calculate the Bery phase of going around a dirac cone,
you get a $\pI$ Berry phase,
it's the most important property of a Dirac cone.

When $\nu=1$,
the cones connect up in such a way that wheterver your chemical potential is,
if you lok tahte Fermi suaces,
and claculate the Berry phase to go aorund all your fermi surfaces,
you get a Berry phase of $\pi$.
If $\nu=1$,
the cones connect up sothat the Fermi surface encloses an odd numbero f Dirac
cones,
or more precisely there is a $\pi$ Berry phase.

For example, suppose you had $k_x$ and $k_y$.
At each $\Lambda_a$, define
\begin{align}
    \pi_a = \delta_{\Lambda_a} \delta_{\Lambda_a + \pi \hat{z}} = \pm 1
\end{align}
Depending on the sign of this guy,
what you can argue is that every time you want to go from a filled to an empty
circle,
youhave to cross the Fermi circle.
Every time you went from 0 to $\pi$ we had to cross the band so that the Fermi
surface lies here.
It's anlogous in higher dimension,
that if you want to gor from one poitn ot another,
whether you hae to cross a Fermi surface is determined by this number here.

So far I've just told you what happens,
but I haven't explaine dhtis very much,
but I won't explain this to you,
just too much band theory.
I hope you got the idea.
It's imlar to 1D.
You want to go from one side to another,
whether you cross the band depends on some invariants,
and in this case happens to be this one.
It's more confusing to think of this 3D case.
If you are interested in understanding this point in detail,
I can recommend references.

If $\nu=1$ mod 2,
then the surface can be modelled by an odd number of Dirac cones at some
chemical potnetial nd the minimal surface model would be having just 1 Dirac
cone.

So we've writtne Dirac cones before in class, for example
\begin{align}
    H &=
    \int d^2x\,
    \left[
    \Psi^\dagger\left( 
        i\sigma^x D_x - i\sigma^y D_y
    \right)\Psi
    - \mu \Psi^\dagger \Psi
    \right].
\end{align}
Can the Fermi surface be gapped?

In this model,
time-reversla symmetry can e implmented by
\begin{align}
    \mathcal{T} &= i\sigma^y K
\end{align}
where $K$ is complex conjugatino.
In 1D it's either gapped and breaks the smmetry
or it's not gapped and preserves the symmetry.
Same thing here.

A surface is either
\begin{enumerate}
    \item Trivially gapped and symmetry-breaking.
        The things you can dd to the Hamiltonian to gap out the Fermi surface
        would be a mass term
        \begin{align}
            \delta H_1 &= m \Psi^\dagger \sigma^z \Psi
        \end{align}
        which breaks TR symmetry, but conserves $U(1)$.
        If you ahvea single massive dirac fermion,
        integrate out the fermion,
        look tah te electrmagnetic response,
        you geta Chern-Simons term in the background gauge field with a value of
        half.
        so this leads to a $\frac{1}{2}$ integer Hall conductivity,
        meaning hte Hall conducatnce is
        \begin{align}
            \sigma_H &= \frac{1}{2} \frac{e^2}{h}
        \end{align}
        So breaknig TR symmetyr leads to Hall condcityity that is half.
        But if there is an integer ambiguoulity more generally,
        it's
        \begin{align}
            \sigma_n &= \left( n + \frac{1}{2} \right) \frac{e^2}{h}
        \end{align}
        the half shift of the Landau levels is related to the Dirac cone.
        You will never see the surface in a pure 2D system,
        because in a 2D system,
        you only have an even numbero f Dirac cones.
        Having an odd number of Dirac cones gives that 1/2 conduance,
        and is non-trivial.

        Another term you could add to theHamiltonain to gap the Fermi surface is
        \begin{align}
            \delta H_2 &=
            \Delta \Psi^\dagger i\sigma^y \Psi + \mathrm{h.c.}
            \propto
            \Delta \Psi_1\Psi_2 + \mathrm{h.c.}
        \end{align}
        This term by itsefl opens a gap at the Fermi surface.
        Superconducting terms usually produced gapped at the surface.
        there's no way of introducing a gap trivially,
        by trivally I mean adding a Fermion bilinear and gapping hte surface.
        These are really the only two terms you could add.
    \item The other possiblity is that you don't hvae these terms and the
        surface is symmetry-preseriving.
        But now you have in general an odd number of Dirac cones,
        with this symmetry and time-reversal,
        and this has a ``parity anomaly''.
        And in our situtiona,
        this is really a mixed anomaly between U(1) and time-reversal.
        What this mixed anomaly means is that if you take a theory of a single
        Dirac cone,
        make it well defined,
        then you run into problems and you need to regularize it,
        and any way of regualirazting necessarily breaks U(1) or T symmetry.
        For examle, compue the harge of the U(1) symmetry,
        but you need to do a sum you need to regularize,
        but ther's no way of regularlizing wihtout brekaign T.
        You need to introduce an extra dimensino.
        Alternatively,
        you could try to get teh sign of the path integral real to be
        consistentwtih TR symetry without messing the bulk.
        This discussion is probably a lecture of its own.
        There's a really nice review by Ed Witten from 5 or 6 years ago where he
        does it and he releates it to TSE index theorems,
        which I encourage you to look up.
    \item This is new to this dimensino and wasn't discovered until after 5 or 6
        years after topological insulators were discovered.
        This is the possbility that the surface is gapped and
        symmetry-preserving,
        but the surface has a non-invertible topological order.
        Somtimes people call this ``intrinsic toplogical order''.
        This means anyonic excitaitons,
        particles iwth topological nontrivial faactional statsitics,
        and the particles act on the anyons in anomalous ways that show upin a
        2+1D system.
        We haven't discussed non-invertible toolgoical ordre yet,
        so we're not in a position to discuss this third possibility,
        there there is such a possibility.
        The ground state on a nontrivial surface is degenerate,
        in fact a topological degeneracy.
        For example, if the surface of your system isa torus,
        you would havea ground sate degenarcy coming from hte boundaries,
        likethe Majoranas.
        In 1 you also have a ground state degenearcy,
        but it is trivial.
\end{enumerate}

\begin{question}
    When it says we have to enclose an odd numbero f Dirac coenes.
    Say I sweep the fermi level,
    that means it should come in paris?
    If I sweep up the Fermi level,
    and I need to retain an odd number of Dirac cones,
    then i have to pass 2 Dirac cones at a time?
\end{question}
Just ignore ``Fermi surface enclosed odd number of Dirac cones''.
What matters is that when you go around the Dirac cone,
you have a $\pi$ Berry phase.

You have to think pretty carefully about what happens in 2D, 3D, but 3D is much
more complicated.
It's analysis you have to go through.


Those are just a few words bout hte surface.o
Bulk response theory.
I assert what the bulk response theory is .

\begin{question}
    Trivial gapping is being introduced by fermion bilinears,
    by non-trivial you mean higher order terms?
\end{question}
Let me be more specific.
Trivially gapped means that the surface does not have intrinsic topological
order.
Now,
if the surface does not have intrinsic topological order,
usually those kinds of gapped surfaces which don'th ave intrinsic gapped order
can always be cancleeleld by adding fermion bilinears.
But if you add a four-fermion term,
that could lead ot expectation values for two-fermion blinears.
That's what happesn in superconductivity.
what I really mean is more naunced.

\begin{question}
    Given a material,
    would you be able to tell from this band diagram?
\end{question}
You have to find whether the symmetry is preserved or not,
which you can convince experiennatally.

\begin{question}
    Possiblity 4 with gappless and symmetry breaking?
\end{question}
That's kind of trivial.
It could be gapless and symmetry breaking.
The reason I emphasies symmetry preserving is because that's usually the
possiblity that occurs if you preserve the symmetry.

Once you break the symmetries,
you could think of what's hapening as another manifestatino of hte anomaly.
Symmetry brekaing ofa system with an anolay is an intersting subject itself.
Usulaly it means various defects have nontrivial properties because of the
anomaly.

One thing I didn't mention.
This superconductivity is superconducting in a single fermion band.
Ify ou inteoduced supercondcutivty in a single band systme,
you expect ot et Majorana zero modes.
The vortices of this superconector $\delta H_2$ carry Majorana zero modes.
This gave the first example beyond
which doesn't have a non-trivial pairing symmetry,
you can have an odd numberof so vortices carry MZMs.

It's really a single band,
there's no spin degenearcy.

\section{Electromagnetic response}
I want to discuss the electromagnetic response.
The electromagneti response has an effective action that is topological
\begin{align}
    S_{3+1}^{top}[\theta, A] &= \theta\cdot P
\end{align}
this is the so-called $\theta$ term in gauge theory.
This is called the instanton number.
\begin{align}
    P &= \frac{1}{32\pi^2} \int d^3x\, dt\,
    \epsilon^{\mu\nu\rho\sigma} F_{\mu\nu} F_{\rho\sigma}
    =
    \frac{1}{4\pi^2} \int d^3x\, dt\, \vec{E}\cdot\vec{B}
\end{align}
Let me define the non-abelian Berry conncetion,
where we take Bloch states from different bands.
\begin{align}
    a_i^{\alpha\beta} &=
    -i \bra{\alpha,k}
    \frac{\partial}{\partial k_i}
    \ket{\beta,k}
\end{align}
Before we were taking Blcoh staes from the same band,
but now we're taking them from different bands.
the non-abelian field strength
\begin{align}
    f_{ij}^{\alpha\beta} &=
    \partial_i a_j^{\alpha\beta}
    - \partial_j a_i^{\alpha\beta}
    + i[a_i, a_j]^{\alpha\beta}
\end{align}
Then the $\theta$ term is
\begin{align}
    \theta &=
    \frac{1}{2\pi}\int d^3k \, \epsilon^{ijk}\Tr\left( 
    \left( f_ij - \frac{1}{3}[a_i, a_j] \right)a_k
    \right)
\end{align}
The trace is over occupied bands.
This is called the ``theta term'',
the ``topological magnetoelectric effect''
or the magnetoelectric polarizability.

Note that $F\wedge F$ is a total derivative because
\begin{align}
    \epsilon^{\mu\nu\rho\sigma} F_{\mu\nu} F_{\rho\sigma}
    \propto
    \epsilon^{\mu\nu\rho\sigma} \partial_{\mu}\left( 
    A_\nu \partial_\rho A_\sigma
    \right)
\end{align}
One way of thinking about this is either a surface effect,
and you can think of it as a property that does something to monopoles in your
system.
We see both of those.

Just like the Chern number arises from integrating fields strength over a
surface.
But here, w'ere integrating $F\wedge F$ over a closed $4$-manifold.

$P$ is an integer if space-time is closed.
And because $P$ is an integer and spacetime is closed,
our path integral,
or effective parition function looks like
\begin{align}
    e^{i\theta P}
\end{align}
and because $P$ is an integer,
changing $\theta$ to $\theta+2\pi$ doesn't change anything,
so in the bulk it doens't change anything
$\theta \sim \theta + 2\pi$.
We're just trying to undertand the bulk of hte sytem by putting it on a closed
sacetime manifodl,
and ther's no effect changing by $2\pi$.
Then you can ask what time-versal does.c
It doesn't do anytign to $\vec{E}$, 
but it flips $\vec{B}$.
So effeively time-reversal is like flipping $\theta$.

So, you could say a time-versed system would have a $-\theta$ response,
which means that $T:\theta \to -\theta$.

And because $\theta=0$ or $\pi$,
but are consistent with timer-reversal.

\begin{question}
    The anlog of $\theat$ in 2+1 dimensisno is anot integer,
    but now we can define something up to an integer?
\end{question}
There's a fundamental distinction betwen even and odd idmensions.
In odd dmidension,
youhave $\theat$ terms.
If you have a insulator wiht no particular symemtries,
$\theat$ can be anything.
But now we add a symmetry and we quatnzie it.
The coefficient in the CS theory was integer.
The short answer is there is a fundamental distcintion betwene odd and even
dimensions.
There's a long answer how to relate 2+1 and 3+1 dimensinos,
but I can't go into that at the moment.

As an aside,
let me say that if you're intersted in the TQFT aspect of it,
you might also be interested in not only turning on a non-trival spacteime,
but also be interested in putting it on a non-trivial spacetime so it has
curvature.
So if you include the gravatiational response,
here we have not just $F\wedge F$,
but also $R\wedge R$ where $R$ is the Riemann tensor.
So
\begin{align}
    S[A, g] &=
    \theta\cdot P
    -
    \frac{1}{48} \int \frac{1}{\left( 2\pi \right)^2} \Tr R\wedge R
\end{align}
Let me mention some physical consequences of this magnetoelectric term.

\begin{enumerate}
    \item Surface quantum Hall effect.
        Suppose that we have a domain wall of $\theta$ as a function of $z$.
        And $\theta$ goes from $0$ at $z=-\infty$ to $\pi$ at $z=+\infty$.
        So then
        $\theta(z)=\pi H(z)$ as a step function and its derivatve is
        $\theta'(z) = \pi \delta(z)$.
        Then
        \begin{align}
            \frac{1}{8\pi^2}\int\theta(z)
            \epsilon^{\mu\nu\lambda}\partial_z\left( 
            A_\mu \partial_\nu A_\lambda
            \right)
            &= -\frac{1}{8\pi}\int \partial_z\theta\cdot
            \epsilon^{\mu\nu\lambda} A_\mu \partial_\nu A_\lambda\\
            &= -\frac{1}{2} \frac{1}{4\pi}\int_{z>0}
            A_\mu \partial_\nu A_\lambda \epsilon^{\mu\nu\lambda}
        \end{align}
        and so we get the Hall conductance
        \begin{align}
            \sigma_H &= -\frac{1}{2}
        \end{align}
    \item Witten effect.
        Suppose we have a bulk cube with boundaries.
        Suppose $\theta$ goes from $\pi$ to $0$ as discussed.
        Imagine a monopole form the outside and bring it inside.
        The basic effect is that monopole carries half charge.
        One quick way of seeing that is tht the surface has half-integer QHE,
        I take the monompole iwth strnght $2\pi$,
        and this monopole feels flux $2\pI$,
        so it changes by $2\pi$.
        Becasue of the half integer efefct,
        it has to bind the half charge,
        so I get a half-charge at the surface and that needs to be compensated
        somewhere else,
        and that's in the monopole.
        That's a quick way to see the half-integer hall effect.
        You could do a few lines of algebra and actually derive it a little more
        concretely.
        Let me derive it quickly.
        \begin{align}
            j^{\mu} &=
            \frac{\delta S}{\delta A^{\mu}} =
            \frac{1}{2\pi} \epsilon^{\mu\nu\lambda\rho}
            \partial_\nu \theta
            \partial_\lambda A_\rho
        \end{align}
        Suppose $\theta$ is uniform in spce,
        but time-varying
        \begin{align}
            \partial_i j^i &=
            \frac{1}{2\pi}
            \epsilon^{itjk} \partial_t \theta
            \partial_i \partial_j A_k\\
            &= -\frac{1}{2\pi} \partial_t \theta \partial_i B^i
        \end{align}
        and this is the same as the time-derivativeof the charnge
        \begin{align}
            \partial_t \rho &=
            \frac{1}{2\pi} \partial_t \theta \partial_i B^i
        \end{align}
        So then
        \begin{align}
            \rho &= \frac{1}{2\pi}\theta\nabla\cdot\vec{B}
        \end{align}
        and so tthe electric charge is half the monopole charge
        \begin{align}
            Q_e &= \frac{1}{2} Q_m
        \end{align}
\end{enumerate}

Ther'es one other consequence I want to discuss next time,
then I'll do dimensinoal reduction from 4+1D Chern insulator.

Let me just take a few minutes and do one another thing,
which is closely replated to the Hall conductivity called the Witten effect.

\section{Response Theory for (3+1)D Topological Insulator}
The response theory has action
\begin{align}
    S[A,g] &=
    \theta\cdot \frac{1}{32\pi^2}
    \int_{M^4} F_{\mu\nu}F_{\lambda\sigma} \epsilon^{\mu\nu\lambda\sigma}
    - \frac{1}{48}
    \int \frac{1}{\left( 2\pi \right)^2} \Tr R\wedge R
\end{align}
where $\theta=\pi$ for non-trivial and $\theta=0$ for trivial.
Recall that
\begin{align}
    \int F_{\mu\nu}F_{\lambda\sigma} \epsilon^{\mu\nu\lambda\sigma}
    =
    \vec{E}\cdot\vec{B}
\end{align}

The physical consequences are TME:
\begin{enumerate}
    \item Domain wall in $\theta$, $\frac{1}{2}$ integer Hall conductance at
        surface.
    \item Witten effect
        \begin{align}
            Q_e &= \frac{1}{2}Q_m
        \end{align}
    \item Magnetic field induces charge polarization
\end{enumerate}
the current is
\begin{align}
    j^{\mu} &= \frac{\delta S}{\delta A^\mu}
    = \frac{1}{2\pi} \epsilon^{\mu\nu\lambda\sigma}
    \partial_\nu \theta \partial_\lambda A_\sigma
\end{align}
Let $\theta$ be uniform and time-dependent
\begin{align}
    j^i &=
    -\frac{1}{2\pi}
    \epsilon^{tijk} \partial_t \theta \partial_j A_k\\
    &= -\frac{1}{2\pi}\partial_t \theta\cdot B^i
\end{align}
so then the current is
\begin{align}
    \vec{j} &=
    \frac{\partial\vec{p}}{\partial t}\\
    &=
    -\frac{1}{2\pi} \partial_t \theta \vec{B}
\end{align}
and so the magnetic polarizability is
\begin{align}
    \vec{P} &=
    -\frac{\theta}{2\pi}\vec{B} + \mathrm{const}
\end{align}
For a topological insulator $\theta=\pi$.
It's a bulk polarization but poarlizaiton only has an effect on the surface.
When I change $\theta$ from 0 to $3\pi$ for example,
my quantum hall effeect on the surface is $\left(1 + \frac{1}{2}\right)e^2/h$.

In 1D I can always add an integer charge on the surface.
If I put $2\pi$ here,
the efect is tackign on a quantum hall effect on the surace.
The bulk didn't play a special role,
it's just something intrinsicaly 2D you can do.

\begin{question}
    Why is QHE layer the same as surface charge?
\end{question}
The magnetic field induces the surface carge.
Charge density is attached to magnetic field in the integer quantum hall effect.

\begin{question}
    How do you physically increase $\theta$?
    Whatif you wanted to?
\end{question}
This was just a trick to get this eqution.
Every insulator in 3D has a $\theat$.
Could be anything.
Buti f it's time-reversal invariatn,
it's pinned to $0$ or $\pi$.
What $\theta$ is for a specific material,
its' a aproperty of hte matieral.

\begin{question}
    Insead of magnetic field, the charge goes ot the edge of the system?
\end{question}
If I insert a magnetic field localy,
I locally get a charge.
If I insert a uniform magnetic field then I get a uniform charge.

\begin{question}
    So the divergence of $\vec{B}$ is zero,
    does that mean there's no net charge within anywhere in the system?
\end{question}
From the Witten effect,
charge is attached to the monopole.
If you have no monopoles,
then you won't have any.

I told you all thsi stuff,
just asserted the response theory of the TI  has a $\theta$ term,
but I didn't derive it from any mdoel.
I want to give you one type of model.
This goes back to the dimensional reduction from (4+1)D.

To describe (3+1)D TI,
it's useful to start with a (4+1)D analogue of the Chern insulator.

What is a (4+1)D Chern insualtor?
In 2+1D,
we wrote the berry connection.
But now you have to define a non-abelian Berry connection.
\begin{align}
    a_i^{\alpha\beta} &=
    -i \bra{\alpha,k} \frac{\partial}{\partial k_i} \ket{\beta,k}\\
    f_{ij}^{\alpha\beta} &=
    \partial_i a_j^{\alpha\beta}
    - \partial_j a_i^{\alpha\beta}
    + i\left[ a_i, a_j \right]^{\alpha\beta}
\end{align}
where $i,j$ are spatial indices and $\alpha\beta$ are band indices.
The second Chern number is
\begin{align}
    C_2 &=
    \frac{1}{32\pi^2} \int d^4k
    \epsilon^{ijkl} \Tr f_{ij}f_{kl}
\end{align}
Here we get a (4+1)D Chern-Simons term
\begin{align}
    S_{CS,4+1} &=
    \int
    \frac{C_2}{24\pi^2} A_{\mu}\partial_\nu A_\lambda \partial_\rho A_\sigma
    \epsilon^{\mu\nu\lambda\sigma}
\end{align}
There isa Maxwell term with two factors of $A$,
but here we have3 factors of $A$,
so this is subleading,
but this term is topological,
wherease Maxwell si not toolgocial.

In (2+1)D we had a model for the Chern Insulator,
where it's basically a massive Dirac fermion you put on a lattice.

Here we write a $(4+1)D$ massive Dirac fermion instead.
So the mdoel for the $(4+1)D$ Chern insulator is a massive Dirac fermion,
which in momentum space you would write
\begin{align}
    H &= \sum_k \psi_k^\dagger d_a(k)\Gamma^a \psi_k
\end{align}
where the $\psi_k$ are 4-component fermions,
and these $\Gamma$ are $(4+1)D$ Dirac matrices which satisfy the Clifford
algebra
\begin{align}
    \left\{ \Gamma^\mu, \Gamma^\nu \right\} = 2\delta_{\mu\nu}
\end{align}
where $\mu,\nu=0,\ldots,4$.
And our lattice vectors are
\begin{align}
    d(k) &=
    \left( 
    m + c\sum_{i}\cos k_k,
    \sin k_x,
    \sin k_y,
    \sin k_z,
    \sin k_w,
    \right)
\end{align}
and to be concrete,
we can write explicityly the Gamma matrices
\begin{align}
    \Gamma^a &=
    \sigma^a \otimes \tau^2
\end{align}
for $a=1,2,3$
and
\begin{align}
    \Gamma^4 = I\otimes \tau^1\\
    \gamma^0 &= I\otimes \tau^3
\end{align}
and the lattice vectors
\begin{align}
    \hat{d}(k) &=
    \frac{\vec{d}(k)}{|d(k)|}
\end{align}
The second Chern number is
\begin{align}
    C_2 &=
    \frac{3}{8\pi^2}
    \int d^4k
    \epsilon^{abcde}
    \hat{d}_a
    \partial_x \hat{d}_b
    \partial_y \hat{d}_c
    \partial_z \hat{d}_d
    \partial_w \hat{d}_2\\
    &=
    \begin{cases}
        0 & m < -4c, m >4c\\
        1 & -4c < m < -2c
    \end{cases}
\end{align}
and just like before,
there are $(3+1)D$ gapless chiral fermions on the surface with
$|C_2|$ flavours.

The surface Hamiltonian will be
\begin{align}
    H_{\mathrm{surface}} &=
    \sgn(C_2)
    \int \frac{d^3k}{(2\pi)^3}
    \sum_{i=1}^{|C_2|}
    V_i \psi_i^\dagger \vec{\sigma}\cdot\vec{k} \psi_i
\end{align}
where these are 2-component fermions.

There is a chiral anomaly
\begin{align}
    \partial_\mu j^\mu \propto FF
\end{align}

Now that we have this $(4+1)D$ cherin insulator,
we can dimension-reduce and view one of these $A$ componetns as a parameter,
and we effectively get an $F\wedge F$ term in the dmiensinoally reduce dmouel

These chircal fermions onthe surface descend into Dirac cones on the surface of
the $(3+1)D$ topological insulator.

\begin{question}
    Why are there $C_2$ flavours?
\end{question}
It is the analogue of $C_1$ flarous in $(2+1)D$.

It's always a 2-component fermion,
buti t's a question of how many you have.
The Chern numbero f bulk is just like the $(2+1)D$ case.

\begin{question}
    Why 2 component fermions?
\end{question}
I don't have a quick answer.

The point is you decompose your fermions
with a chirality operator,
so you split the 4 componet into 2, 2 components,
and because it's chiral,
you only have 2.
They are $C_2$ flarous of the 2-compoent fermion.

\begin{question}
    How do you visualize a $(2+1)D$ chiral fermion.
\end{question}
I don't have a godo visualization.
There's chirality operator,
and you project them onto two eigenvectors.

If you have massless particles moving in a direciton,
you can attribute a handedness of the particles,
which you cannot Loretnz boost out of.

Yeah, it's basically in this language,
$1\pm \Gamma^0$,
that's how you define the Chirality operator.
Projection onto some particular chirality.

\begin{question}
In 2 spatial dimensions,
you have only clockwise and anti-clockwise.
What's the analog of this?
\end{question}
I don't usually think of chiral fermions in 3 dimensions.

\begin{question}
    In the 1D case,
    you could think of the Chern number as the wrapping of the sphere?
\end{question}
Oh yes.
The $\hat{d}$ vector is a map from the Brilluoin zone $\hat{d}:T^4\to S^4$.
This is computing the winding numbero f the map from $S^4\to S^4$.

If you take $k\to\infty$, it is the case that it all points in the same
direction,
I don't know if there's an easy way of saying it.

His point is if I take $k_x$ to $\infty$ but not the other ones,

I was plannig on presenting the dimensinoal reduction from $(4+1)D$
to $(3+1)D$.

The point is the $(4+1)D$ model can be thought of as
decoupled $(3+1)D$ models parameterized by $k_w$.
We did a version of this in $(2+1)D$,
where we thoguht of it as a $(1+1)D$ Hamiltonian parametrized by $k_y$.

Let's write the Hamiltnoian first in real space on a lattice.
On real space on a lattice,
that Hamiltnoian will look like
\begin{align}
    H_{(4+1)D} &=
    \sum_{\sigma,a=1,\ldots,4}\left[ 
    \psi_{\vec{r}}^\dagger
    \left(
    \frac{c\Gamma^0 - i\Gamma^i}{2}
    \right)
    \psi_{\vec{r} + \hat{a}}
    e^{i A_{\vec{r},\vec{r} + \hat{a}}}
    + \mathrm{h.c.}
    + m\psi_{\vec{r}}^\dagger \Gamma^0 \psi_r
    \right]
\end{align}
and then I can write it in the ``Landau gauge''
translationlaly invariatn in $w$ direction
\begin{align}
    H_{(2+1)D}[A] &=
    \sum_{k_w} H_{(3+1)D}[k_w, A]
\end{align}
and then
\begin{align}
    H_{(3+1)D} [k_w, A] &=
    \sum_{\vec{r}, a=1,\ldots,3}\left[ 
    \psi_{\vec{r}, k_w}^\dagger
    \left( 
    \frac{c\Gamma^0 - i\Gamma^a}{2}
    \right)
    \psi_{\vec{r} + \hat{a}, k_w}
    e^{i A_{\vec{r},\vec{r} + \hat{a}}}
    + 
    \mathrm{h.c.}
    \right]\\\nonumber
    &\qquad+
    \sum_{\vec{r}}
    \psi_{\vec{r}, k_w}^\dagger\left[ 
    \sin\left( k_w + A_{\vec{r}, w} \right)\Gamma^4
    +
    \left( m + c\cos\left( k_w + A_{\vec{r},w} \right)\right)\Gamma^0 
    \right]
\end{align}
So if I set $\theta=k_w + A_{\vec{r},w}$,
I just replace $k_w + A_{\vec{r},w} = \theta_{\vec{r}}$
and this gives us
$H_{(3+1)D}[A,\theta]$.

At the level of the effective action,
we can figure out the efective response theory
by doing dimensinoal reduction in the response theory.

And so
\begin{align}
    S_{(4+1)D} &=
    \frac{C_2}{24\pi^2} \int d^4x \, dt\,
    \epsilon^{\mu\nu\rho\sigma\tau}
    A_{\mu} \partial_\nu A_\rho \partial_\sigma A_\tau
\end{align}
and under dimensional reduction $C_2=$,
and I can just rewrie this action as
\begin{align}
    S &=
    \frac{1}{24\pi^2} 3
    \int d^3x \, dt\,
    \epsilon^{w \mu\nu\rho\sigma}
    A_w \partial_\mu A_\nu \partial_\rho A_\sigma\\
    &=
    \frac{\theta}{32\pi^2}
    \int \epsilon^{\mu\nu\rho\sigma} F_{\mu\nu}F_{\rho\sigma}
\end{align}

Why did we set $C_2=1$?
Because I want to describe the non-trivial topological insulator.
If I set $C_2=2$, then I would just get two copies of the TI.

You can just do perturbation theory to see what the response theory looks like.

\begin{question}
    Is $\theta$ here defined on $2\pI$ and does it save the time-reversal
    symmetry?
\end{question}
Here it does, just follows from the same argumetn I made before.

This $k_w + A_{r,w}=\theta_r$,
which you can see is periodic in $2\pi$.

If you write the time-reversal operator,
this term does not commute with the time-reversal operator,
so $\theta$ has to be $0$ or $\pI$.

\begin{question}
    When is dimensional reduction not possible?
    This is quite general argumetn.
\end{question}
You can always do dimensional reduction,
but it's not necesarily the case that you start with some theory,
and you do dimesnional reductino and yo uget an interseting and nontrivial
theory.

For example, if I set $C_2=2$,
I get two copies and I get the trivial insulator.

In $(2+1)D$ where we did dimesniaonl reductnio,
the $(1+1)D$ theory was not trivial because it realised a topological pump.

We start with a topological pahse characterised by integer $C_2$,
and then $\mathbb{Z}_2$ invariants.

You could dmiensional reduce one more time,
and yo uget the $(2+1)D$ topological insulator,
which is time-reversal invariant.

\begin{question}
    Do we need the model gapped?
\end{question}
If you start witha gapped model,
you usually wind up with a gapped model.
But if you started with a gapless mdoel,
you might not end up with a gapless model.


Now you can also do the $(2+1)D$ surface.
If we take $C_2=1$,
we would only have one flavour,
and if I dmeinosaly reduce,
and set $k_w=0$,
then I would just get a single Dirac cone.
This ia nice way to se the (2+1)D surface
would jsut beceom a (2+1)D Dirac cone.
I won't write it out,
because it's obvious from this.


Let me write out one more thing to be a little more clear.
Wen you do dimensinoal reduction,
you can think of putting a system on a cylinder,
in which case the azimuthal direction is $w$.
Then your momentum would be quantized
\begin{align}
    k_w &= \frac{2\pi}{L-w}n
\end{align}
and the low-energy mode is the one where $n=0$.
If you start with Chiral fermions and do dimesnioanl reductio
you have one massless fermion, the zero-ommentum mode
that descends to the massless dirac cone,
but you would have higher dmieinosal massive Dirac fermions.


Intersting commetn:
If you start with the $(2+1)D$ surface Dirac cone,
I mentinoed before this theory has a Perry anomaly,
hwich is a mix of U(1) and TR symmetry,
but you see the theory has problems,
you can decude the fact that the only way of preserving the symmetyra nd makig
making the theory well-define is to inroduce a bulk (3+1) respone theory.

So start with a (2+1)D surface theory with a Dirac cone,
then try to make the path integral well-defined with
$T$, $U(1)$.
Then educe a $(3+1)D$ bulk with
\begin{align}
    S_{eff} &=
    \int F\wedge F + R\wedge R
\end{align}
There's a paper by Witten in 2015,
which is nice because he uses the APS index therem by Atiyah.
I would go thorugh it,
but it would be an entire lecture.

\begin{question}
    Does it go up to higher dimensions?
\end{question}
In Witen's paper,
he only does $(3+1)D$ topological theories,
but the index theorem hsould work.

The thing is,
this parity anomaly you don't have in every dimension,
you have it in every 8 dimensisno.
There's a periodicity of 8.
Everything we did for free fermions has periodicity 8.
Deep,
but we don't care beyond 3 dimensions.

\begin{question}
    Are zero modes of (3+1)D a subset of zero modes in (4+1)D
\end{question}
I don't know, it's not obvious.

\begin{question}
    Experimental progress in realising these?
\end{question}
There are multiple generations of people who predicted various matirerals.
The first material is BismuthAntimony,
but since then many materials have been porpsoed.
There's no problem gettinga band structure exhitibing this topolgoical
insulator,
but the problem is getting the chemical potential in the gap.
You can't tune the chemical potential like in 2D,
you need to use doping,
but it's hard to get hese materials into actual insualtors.
Early on,
it was difficult getting the chmiceal potential in the right place.
But in the band structre,
you could confirm things like an odd numberof Dirac conesno the surface.
I don't know if people actually sucessfully made an insulating matieral that was
topogoloical and gets the chemcial potential in the right palce.

Was anyone following?

Now I want to switch gears a little bit.
Let's summarize what we did.

We talked about a $(1+1)D$ Majorana system with
$G_f = \mathbb{Z}_2^f$,
and there is a $\mathbb{Z}_2$ invariant.

Then we considered $(2+1)D$ systems with $G_f=\mathbb{Z}_2^f$.
There is a $\mathbb{Z}$ invariatn which si the chiral central charge,
which is realised by the $p+ip$ superconductor.
$G_f = U(1)^f$And there is a $\mathbb{Z}$ invariant Chern insulator.
And we studied 
$G_f = U(1)^f \rtimes \mathbb{Z}_4^{t,f}/\mathbb{Z}_2$
which results in a $\mathbb{Z}_2$ invariatn in class AII.

Then in $(3+1)D$,
we geta $\mathbb{Z}_2$ invariant to class AII.

All the above are examples of invertible topological phases.
They all have free fermion realisations.
These are all examples,
once we look at what states are at the free fermion level,
all states were stable to added interaciton.
They all had phsycal propblems.
the $\mathbb{Z}$ invairnat is jus the QHE conducatnce.
And we also solved for hte phsycial properties of these invariats.
We didn't formally argue it,
but it is plausible because iti s a physical quantitiy you could measure its
properties.

There are other tings tha could hapepen.
thera resiutaltios in free fermions you get $\mathbb{Z}$ invariatns,
but when you add interactinos,
it drops to $\mathbb{Z}_8$.

And then you could have invertible models with no fre fermion ralisations.
And then you have non-invertible models.

\section{Bosonic in (1+1)D}
Theseare inrinsically strongly interacting,
because if not they will condense into superfuliud state.
I necessarily need a strongly interacting system.
So all the tools we had for free fermion models become useless.
In $(1+1)D$ for bosons,
bosons can only realize a certina subclass of invertible phases.

There are some aspects.
\begin{enumerate}
    \item There's no topological order in 1D.
        This means if you forget about any symmetries of your model,
        any gapped bosonic state in 1D can be adiabatically conencted to trivila
        state.
        That's difernt to the fermionic case,
        where if you ignore all pases,
        it's still non-trivial bcause thereare stil MZMs.
        For bosons,
        if you ignore symmetries,
        everything can be adiabatically connected to the trivial phase.
        So no topological order in 1D.
    \item If you add symmetry,
        then we couldh ave a class of topological phases called SPT phases.
        SPT stands for symmetry protected topological.
        These are a subset of invertible phases,
        which has the property that if you break the symmetry,
        it can be adiabatically connected to a trivail state,
        but if you have the symmetyr,
        it cannot be in a way that preserves the symmetry.
        These are classified by a quantitiy called the group cohomology
        $H^2(G, U(1))$.
        The second cohomology group with group elements of $G$ wiht coefficients
        in $U(1)$.
\end{enumerate}
Rigorous proof will require mathematical frameowrks not in this class.
We can establish this using matrix product states.

Ther eare no invertible phaess that are not SPTs.
There are no chiral phases.
You wouldn't be able to define chiral,
the boundary is zero dmiensional,
so there are no chiral moedes.
$(2+1)D$ is special,
even bosons have an nivertible phase that is not an SPT.

Let me start off by giving an example of an interesting 1D state.
The most important example is the AKLT state.

These guys considered a spin-1 chain,
with $G=SO(3)$ symmetry.

The Hamiltonian includes a Heisenberg term,
but in addition,
there's an extra term that is the square of he Heisenberg term
and a constant offset
\begin{align}
    H_{AKLT} &=
    \frac{1}{2} \sum_{i} \left( 
    \vec{S}_i\cdot \vec{S}_{i+1}
    + \frac{1}{3} \left( \vec{S}_i\cdot\vec{S}_{i+1} \right)^2
    + \frac{2}{3}
    \right)\\
    &= \sum_i P_{i,i+1}^{(2)}
\end{align}
where $S_i^{\alpha}$ is a $3\times 3$ matrix
and $P_{i,i+1}^{(2)}$ is a projector onto spin-2 subspace of neighbouring spin
1s.

This is simply the explicit form of this projector.
For spin $1/2$ particles,
there is no square term.
Once we have this sum of projectors,
G
the ground state is the state that is annihilated by all projects.

I have a spin chain,
and I am summing over every site on the spin chain.
The first term would be $P_{1,2}$,
where I project 1 and 2 onto spin 2, etc.

Is there a non-trivial state that si annihilated by all $P$s.
You might be worried thesubspace annihlated by this is zero-dimensiona,
but it's one-dimensional on a periodic chain.

The one way to construct the ground state is a cool trick.
Think of every spin 1 as coming from 2 spin 1/2 systems,
but project onto the spin 1 subspace.

This is site $i$ for example,
would have two spin $\frac{1}{2}$ systems,
which I project onto spin-1 subspace.
You'll notice that in order for these two sites $i$ and $i=1$,
the only way this projector is going to give something non-trival
is if every pair of spin-1/2 fuses into a state of spin-1.

If any two spin-$\frac{1}{2}$ fuse to spin 0,
that subspace is going to be annihilated by this projector
$P^{(2)}$.

Imagine if you had
\begin{align}
    \frac{1}{2}\otimes \frac{1}{2}\otimes
    \frac{1}{2}\otimes \frac{1}{2}
    =
    (0 \oplus 1) \oplus (0 + 1)
\end{align}
I can consider where these spin-1/2s from neighbouring parts fouse across.o

consider a state
where each spin-1 you break down into spin-1/2s,
and then project them onto spin1.
Andthen you draw lines connecting adjacent spin-1/2s in different proejctors.

If you put this on a zero chain,
you won't hvae ege modé.o

The 0-energy ground state í the unique gapped grouned sate in AKLT.

This is a ``frustration free'' state.
There will be dangling spin-1/2 degrees of freedon.
On an open chin,

Just likewe found we have anomalies for the boundaries,
this eges system is ano.

Spin-1 degrese offreedom form faithful linear optaions of SO(3) symmetry,
wherease spin-$\frac{1}{2}$ is only a projection.

There will be dangling spin-$\frac{1}{2}$ degrees of freedom.
Edge degeneracy forms a projective representation of $SO(3)$.
Wahereas,

There's anomalies associated iwth global symemtries,
and then there are graviational anomalies,
and they are more severe.
The edges system is 

Some boudnary theories might be in conflict iwththe.

\section{1D boson SPTs}
Last time,
we started talking about SPT states,
bosons in 1D.
We talked about AKLT states,
which is the ground state for spin-1 chains.
We saw the Hamiltonian was
\begin{align}
    H &= \sum_i \vec{S}\cdot\vec{S}_j
    + \frac{1}{3} \left( S_i \cdot S_j \right)^2 + \frac{2}{3}
\end{align}
and we saw this is the sum of projects where neighbouring spins are projected
onto the spin-2 chain.
\begin{align}
    H &= \sum_i P_{i,i+1}^{(2)}
\end{align}
and then we have spin-$\frac{1}{2}$s projected onto spin-1s on each site.
Before you do that though,
you put the left and right on different sites into a spin-singlet state,
and we so we said that this ground state,
AKLT proved that this is the unique gapped ground state
of this $H_{ALKT}$.

In particular, this ground state is actually annhilated by every single term
here,
and this Hamiltnoian is called a frustration-free Hamiltonian.

It's useful to note that the motivation for AKLT to study this was actually how
Haldane conjectured that the spin-1 Heisenberg chain is gapped.

The motivation for AKLT was the Haldane conjecture.
The spin-one-half Heisenberg chain was known to be gapless,
but Dr Haldane predited the spin-1 Hesenberg chain is gapped.
That was suprising,
bcause poeple asuemd the spin half wouldh ave the same behaviour,
but Haldane's conjectre said that spin-1 and spin-half behave very
differently,
which is intrinsically QM.

\begin{question}
    Does it have Goldstone bosons?
\end{question}
No, it doesn't order, no long range order in 1D.

It's non-trivial the spin-half chain is gapless,
but the  spin-1 half is gapped.

SO this AKLT,
you can think of this thing as smoothly connected,
or close,
to the HEisenberg point,
because you can slowly ramp up this parameter from 0 to 1/3.
AKLT noticed this Hamiltonian was similar to the Heisenberg model,
but you could prove it was gapped and has this interesting ground state.

So the spin-1 Heisenberg chain is in the same phase as this,
turning from 0 to 1/3,
and show the gap is still open,
so this isa mdoel to see whath e spin-1 Heisenberg chain looks like.

The key point about this tate,
is that you see youhaave these edge modes,
whic his a key non-trivial aspect of this tate as opposed to trivial.

\begin{question}
    How do you see there's nophase transition from 0 to 1/3.
\end{question}
I think numerically you can see it?
This is 1D,
so there are all sorts of analytically solvable tricks,
maybe yes,
but I don't know.
Therei s aphase transitoin,
but it's beyond 1/3.
I forgot exactly what.

They key point si that you hvae these dangling spin-1/2 things on the edge,
which are dge modes.
The global symmetry is SO(3),
but actaully the chain has more symmetries,
it is time-reverla invaraint,
tralsationally invaraitn,
reflection symmetry,
but we're interested in the OS(3) global symmetry.

The property of SO(3) is that it acts anomalously on the boundary.
Here, anomalous means microscopically each latice site is pin-1,
so it forms a linear repseration of SO(3),
but spin-1/2 is a projective representation of SO(3).
I will remind you mathematically what projective means.

Spin-$\frac{1}{2}$ is a projective representation of SO(3),
but microscopically we have spin-1 per site,
which is a linear representation of SO(3).

So let me tell you what projective representations are,
in case you don't know exactly what this means.
The short summary is taht if you do a combination of rotaitons in SO(3),
that for examle add up to dientity,
whereaase Spin-1/2 would give you a minus sign,
but linaer represetnations would not give you a minus sign.

Suppose you have a group $G$ with a finite-dimensional representation.
\begin{align}
    R: G \to GL(d; \mathbb{C})
\end{align}
What that means is that for eveyr gorup element,
you have the group of $d\times d$ complex matrix.

A \emph{faithful} or \emph{linear} representation has the property that when you
compose group elements,
youget the representation for the product.
\begin{align}
    R_g R_h = R_{gh}
\end{align}
However, this is not always the case,
because in general you could have
\begin{align}
    R_g R_h = w(g, h) R_{gh}
\end{align}
where $w(g,h)$ is a phase in U(1).
So in other words,
\begin{align}
    w \in C^2(G, U(1))
\end{align}
so $\omega$ is a function of two group elements and it gives you a U(1) phase.
If you have this rule,
you could consider associativity for 3 group elements,
which tells you you can first multiply them in any way.
\begin{align}
    R_g R_h R_k &=
    \omega(h, k) \omega(g, hk) R_{ghk}\\
    &= \omega(g, h)\omega(gh, k) R_{ghk}
\end{align}
And this takes us to an important equation.
It tells us the phases must satisfy an important equality
\begin{align}
    \omega(h, k) \omega(g, hk)
    =
    \omega(g, h) \omega(gh, k)
\end{align}
and this is called the \emph{2-cocycle condition}.

We can define this differential map $d$ that maps $n$-cochains to
$n+1$-cochains.
\begin{align}
    d: C^n(G, U(1)) \to C^{n+1}(G, U(1))
\end{align}
and we can define this differential as
\begin{align}
    d\omega_2 (g, h, k)
    =
    \frac{\omega(h, k) \omega(h, hk)}{\omega(g, h) \omega(gh, k)}
\end{align}
and so the 2-cocyle equation is really just the equation that
\begin{align}
    d\omega_2 = 1
\end{align}
and if it satisfies this eqaution,
we call it a 2-cocycle
so actually
$\omega_2 \in Z^2(G, U(1))$,
which is the space of 2-cocycles.
In other words,
\begin{align}
    Z^2(G, U(1)) &=
    \left\{
    x_2 \in C^2(G, U(1)) \,|\,
    dx_2 = 1
    \right\}
\end{align}
However,
there is some reduencay here,
because you could change any of these by a phase and potentially get rid of
these $\omega$s.
That is,
if $R_g R_h = \omega(g, h) R_{g,h}$,
you could just redefine
\begin{align}
    R_g \to R_g \epsilon(g) = R_g'
\end{align}
so then
\begin{align}
    R_g' R_h' &= \omega_2'(g, h) R_{gh}'\\
    R_g R_h &=
    \underbrace{\frac{\omega_2^1(g, h) \epsilon(gh)}{\epsilon(g)
    \epsilon(h)}}_{\omega_2(g, h)} R_{gh}
\end{align}
and thus
\begin{align}
    \omega_2'(g,h) &=
    \omega_2(g, h) \frac{\epsilon(g) \epsilon(h)}{\epsilon(gh)}
\end{align}
So you need to consider 2-cocycles modulo this equivalence relation,
which youccan think of as equilvanet proejctive rerepstionas.
This $\epsilon(g)\in C^1(G, U(1))$ is a 1-cochain associated iwth a single grup
elements,
with
\begin{align}
    d\epsilon &=
    \frac{\epsilon(g) \epsilon(h)}{\epsilon(gh)}
\end{align}
And now you can define the space of 2-coboundaries,
\begin{align}
    B^2(G, U(1)) := \left\{ 
    X_2 \in C^2(G, U(1)) \, \mid\,
    X_2 = d\epsilon_1,\,
    \epsilon_1 \in C^1(G, U(1))
    \right\}
\end{align}
that are differentials of 1-cochains
\begin{align}
    \omega_2 \sim \omega_2 \cdot d\epsilon_1
\end{align}

So projective rpesenations should be classified by 2-cocycles modulu by
2-coboundaries.

You have these rlrestnations,
you have hese extra phases,
andthat phase,
you might be agble to get rid of it by redefingin $R_g$,
but if you can't,
then you have a non-trivial projetive rpesetnation,
and two projective represtnatinos are equivalnet if you can just rephase these
$R_g$'s so distinct ones are lcassifed by this group.

Thus distinct projective representations are classified by
the second cohomology group
\begin{align}
    \frac{Z^2(G, U(1))}{B^2(G, U(1))} =:
    H^2(G, U(1))
\end{align}
And indeed,
you could also generalize this wigh
\begin{align}
    H^n (G, U(1)) &=
    \frac{Z^n(G, U(1))}{B^n(G, U(1))}\\
    Z^n(G, U(1)) &=
    \left\{ X_n \in C^n(G, U(1)) \mid dX_n = 1 \right\}\\
    B^n\left( G, U(1) \right)
    &=
    \left\{ 
    X_n \in C^n(G, U(1)) \mid
    X_n = d\epsilon_{n-1},
    \epsilon_{n-1} \in C^{n-1}(G, U(1))
    \right\}
\end{align}
The only thing I haven't defined is how the differential map is defined,
but we only need $n=2$ for now for 1D spin-1 chains.
We will come back to higher-dimensional SPTs later.

\begin{question}
    Is $H^2$ always a group because the coefficient is $U(1)$?
\end{question}
Well $Z^2$ is always a group.
You can actually replace $U(1)$ with a module over $G$ and still define all
these things.
For $U(1)$,
we don't really need all that,
we just need $U(1)$ coefficients.

Let's just stick to the math we need to do the physics.


\begin{question}
    The stateis symmetric under $G$ means linear represatatino?
    Is there any reason why linear represetnations are special in that way?
\end{question}
No, not ncesarily.
Linear represetnations are special for a wide vairteyt of reasons,
but you could defintiely hav astate symmetric under $G$ but forms a projetcive
rpesetations.
There are theorems which govern what ht low-energy behaviour has to be.
It's imporant to know forh tsystem,
but they are both equally valid as a system and ans as symmetry.


\begin{question}
    Is there a simple quantum example of a projective prestation?
\end{question}
Yes, spin-1/2.

\begin{question}
    We can either say it forms a linear represetnation of spin-1
    or a spin
\end{question}
Let me reprhase your qeustion.
A spin-1/2 chain can be thought of as a projective rpestenation of SO(3),
or it can be thought of a linear representation of SO(3).
It's also thel inear reprsetnation of the cover of some group.
Suppose I have a spni-1/2 chian,
should I think of the symmetry of the sytem being SU(2) or SO(3).
This is actually a pretty delicate quesiton.

It depends on how many spin-1/2s you have.
if you ahve an even numberof spin-1/2s, it should be SO(3),
but if you had an odd number of spin-1/2s, it should be SU(2) technically.
It depends on odd or even.

\begin{question}
    Regardless of even or odd,
    excitaitons are alwasy even spin,
    is there somtehting to interpet?
\end{question}
The local operators for a spin-1/2 are the Pauli matrices $\sigma^i$,
and they themselves form a spin-1 represetnation of SO(3).
The way that $\sigma^i$ transforms under an SO(3) represatatnion,
is the same as a spin-1.
So the Pauli operatosr transform as Spin-1 under SO(3).
Local operators always carry integer spin,
because these $\sigma^i$ transforma nd tranform as a spin-1 represetnation.

\begin{question}
    This is true even if we have a spin-2 chain?
\end{question}
If you had a spin-2 chain,
then the local operatosr would not e pauli matrices,
it would be whatever for spin-2.
Spin-2 is not the best examle.
Spin-3/2 is better,
where they are 4x4 matrices,
but under SO(4) rotations would transform as a local spin.
The formal statement is that clocal exctiations carry integer spin.

Let me say that
\begin{align}
    H^2(SO(3), U(1)) = \mathbb{Z}_2
\end{align}
So it's linear in the spin-1 represetantio nand porjective in the SU(2)
represtnation.

There's a relation between $H^2$ and $\pi^1$.
I think there may be smoe theorem for Lie groups,
$H^2$ is related to $\pi^1$.
$H^1$ is related to $\pi^1$.
$H^1$ is athe abelianization of $\pi^1$.
This is group cohomology.

OK, so going back to AKLT,
the edge modes form this projective representation and we can actually see
what's going on there.
In general for SPT states in 1-dimensions,
suppsoe you have a ground state and you apply your symmetry operation,
on the represetantion on the full-many-body state,
and because in the bulk it's symmetric and you hav a finite-correlation length,
ify ou apply $R_g$ on a ring,
youget back the original state,
\begin{align}
    R_g \ket{\psi} = \ket{\psi}
\end{align}
but on an open chain,
you get the ground satte back,
but up to some boundary opeartions near the left boundary $U_g^{(L)}$
and some operations acting on the right boundary,
up to corrections exponentially small in the isze of teh system.
\begin{align}
    R_g \ket{\psi} \approx
    U_g^{(1)} U_{g}^{(R)} \ket{\psi}
\end{align}
and so
\begin{align}
    R_g R_h \ket{\psi} &\approx
    U_g^{(L)} U_h^{(L)} 
    U_g^{(R)} U_h^{(R)} 
    \ket{\psi}\\
    R_{gh}\ket{\psi} &= U_{gh}^{(L)} U_{gh}^{(R)} \ket{\psi}
\end{align}
Now this is a genearl statemetn that doesn't even need ALKLT.
In 1D, you hvae as ymemtric gapped state,
and you apply your symemtry transfmatoin on an open chain,
then yo hvae these local operators acting on the boundary,
and you compre aply $R_{gh}$ and $R_g R_h$,
and the only way they can be eqaul is that the $U$s are equal up to a phase.
Because some are localized onthe left and some are localized on the right
boundary,
and the only wa this could possibly hold is that
\begin{align}
    U_g^{(L)} U_{h}^{(R)} \ket{\psi} &=
    \omega_2(g, h) U_{gh}^{(L)} \ket{\psi}\\
    U_g^{(R)} U_{h}^{(R)} \ket{\psi} &=
    \omega_2^{-1}(g, h) U_{gh}^{(R)} \ket{\psi}
\end{align}
In general,
you have one projective erepsertaniotn on the left,
and the inverse one on the right.
It's just he thing about SO(3) that you find the projective rpesetnation is its
own inverse and so you have spin-$\frac{1}{2}$ in each direciton.

And so this tells us the equivalence class
\begin{align}
    [\omega_2] \in H^2(G, U(1))
\end{align}
classifies the symmetry action on the edge.

Let's do another example.
AKLT was not the simplest thing.
Even thoguh it's fustration free,
it's nontrivial to prove the ground state is simple and gapped.

There's an even simpler example.
This is called the 1D cluster state model,
sometimes called the $ZXZ$ model,
or sometimes $XZX$ model.
Here, $X$ and $Z$ are the Pauli operators.
\begin{align}
    H &=
    -\sum_{j=2}^{N-1} Z_{j-1} X_j Z_{j+1}
\end{align}
Note that every term commutes with each other.
If I have $ZXZ$ on 3 sites,
and I have another $XZX$ on 3 sites shifted by 1,
they commute with one another.
So every term in the Hamiltnoian commutes and furthermore squares to the
identity,
so the gournd state is jsut the simultaneous $+1$ eigenstate of the $XZX$.

And furthermore,
the global symmetry we're intersted here is the
$G=\mathbb{Z}_2\times \mathbb{Z}_2$,
which are generated by
\begin{align}
    \bar{X}_1 &=
    \prod_{k\mathrm{ odd}} X_k\\
    \bar{X}_2 &=
    \prod_{k\mathrm{ even}} X_k\\
\end{align}


\begin{question}
    Is this encoding $k$ logical qubits into $n$ physical qubits?
    Is it something like stabilizer groups?
\end{question}
These are just symmetries,
I wouldn't call it logical oeprators.
These are jsut symmetry operators.
They square to the identity and they commute with the whole Hamiltonian.

On each boundary,
say on the left boundary,
we actually have a set of operators that commute with the Hamiltonian exactly.
Let's call them
\begin{align}
    \bar{X}_L &= X_1 Z_2\\
    \bar{Z}_L &= Z_1
\end{align}
and these commute with the Hamiltonian.
You see $Z_1$ commutes iwth teh Hamitna,
because the only operator that hits site 1 is the first ermm,
which has a $Z$ there, so that commutes.
You can also see that with the $X_1 Z_2$.

But these have a non-trivial algebra among themselves.
They anti-commute with one another.
\begin{align}
    \bar{X}_L \bar{Z}_L = - \bar{Z}_L \bar{X}_L
\end{align}
Because they comute with te Hamiltonai,
the ground state must frm a rperestaion of this algebra,
and because they anticommute,
there must be a two-fold degeneracy on each edge.
There's a simlar one on the right one as well.
This is not an accidental degenearcy.
No local perturbation can lift this degeneracy as long as the
$\mathbb{Z}_2\times \mathbb{Z}_2$ symmetry is preserved.
Oneway to lift hte dgenracy is to add these 2 tersm to the Hamiltonian,
but these don't commute with one another,
so you cannot add these both and preservet the symmetry.

You can directly see this behaviour exactly in this model.
Suppose that you want to apply the first symmetry operator to the ground state.
Let me consider a half-infinite chain,
so we only have the left boundary,
but then it goes off to inifitey on the other end so we don't have to worry
about it.
\begin{align}
    \bar{X}_1\ket{\psi_{gs}} &=
    \bar{X}_1 \prod_{k=2}^{\infty} Z_{2k - 2} X_{2k -  1} Z_{2k}
    \ket{\psi_{gs}}
\end{align}
and then I pull some things out,
and the nice thing about pulling them out is that if I get a lot of cancelation
\begin{align}
    \bar{X}_1 Z_2 X_3 Z_4 Z_4 X_5 Z_6\cdots &=
    (X_1 X_3 X_5\cdots) Z_2 (X_3 X_5 \cdots)\\
    &= X_1 Z_2
\end{align}
so actually you have
\begin{align}
    \bar{X}_1\ket{\psi_{gs}} &=
    \bar{X}_1 \prod_{k=2}^{\infty} Z_{2k - 2} X_{2k -  1} Z_{2k}
    \ket{\psi_{gs}}\\
    &=
    X_1 Z_2 \ket{\psi_{gs}}\\
    &=
    \bar{X}_L \ket{\psi_{gs}}
\end{align}

\begin{question}
    Something about $C^*$ algebra.
\end{question}
I'll leave it as an exercise.
If you really do a more honest calfcaultion end at $N$ rather than infinity,
you should get $\bar{X}_L$ and $\bar{X}_R$.
If you simillarly did 
$\bar{X}_2\ket{\psi_{gs}}$,
and instead of pulling the even rather than the odd one,
you find.
\begin{align}
    \bar{X}_2\ket{\psi_{gs}} &=
    Z_1 \ket{\psi_{gs}}
\end{align}
and this is really similar to how
$\bar{X}_1$ and $\bar{X}_2$ act as
$\bar{X}_L$ and $\bar{Z}_2$
on the left edge

And $\bar{X}_L$, $\bar{Z}_L$ form a projective represetnation of
the global symmetry
$\mathbb{Z}_2\times \mathbb{Z}_2$.

\begin{question}
    Is it clear this is a gapped state?
\end{question}
yes, every term commutes,
and each terms is +1 or -1,
and it costs an energy of 2 to raise it frmo the ground state to the next
excited state.

If you had a gapless situation on the ege,
you wouldn't have a sitaution where exactly the boundayr gets affected but the
bulk doesn't get affected at all.

You could have a sysetm fully symmetric on a chain,
but if you did have edge modes,
you owud have poewr law correaltions,
and if wasn't exaclty symmetric,
you wouldh ave some power law decay.
Also, if it wasn't gapped,
it wouldn't give an SPT phase which is by definition gapped.

This $R_g\ket{\psi} \approx U_g&L U_g^R \ket{\psi}$
really relies on the fact that it is gapped.
Are there any other questions?

\begin{question}
    How do you see there's no edge modes connected with other symmetries and
    translations?
\end{question}
In this particular model,
you can probably just count the number of degrees of freedom and see that we've
accounted for everything.

We have $Z_1 X_2 Z_3, \cdots Z_{N-2}X_{N-1}Z_N$.
The entire Hilbert space is accounted for already by counting.

\begin{question}
    Are we just partitioning the system?
\end{question}
They coexist.
Ify ou had a symmetry that is $G\times H$,
on each edge you would have a projective represenation of $G\times H$.
Spatial symmetries like translation and reflection are special.
here, the point is that even if we have a boundary,
the systeme iwth the boudnary stil rferelcets th symmetery.
Hwoever, wiht transltaiton symemtry,
the sysetme in the prescenece of the boudndary doesn't relaly resepct that,
so it desno't really makes sense to talk about hte action of tranlation symetron
when there is a boundary.

Often wiht spatial symmetries,
you couldh ave nootrivial SPTs associtated iwth te systemery,
but there may not be edge modes.
It's a more delicate task to figure out what ist he desintct property of the
SPTs given  it's hard to probe them becasue the boundary breaks the symmetry.

\section{Full classification of bosonic 1+1D topological phases}
What we saw so far is that we have this classfificaiton of how the symmetry acts
on the edge modes.
In fact,
this is the full classification of 1D SPTs.

The full classification of $(1+1)$-D topological phases is by
$H^2(G, U(1))$.
There are 2 pieces of evidence more easy to come by.

\begin{enumerate}
    \item We can use matrix product states.
    \item Try to develop a classification of TQFTs and see that deformation
        classes.
\end{enumerate}

\section{Matrix product states}
Suppose you have a chain with $N$ spins with periodic boundary conditions.
Then a \emph{matrix product} state is of the following form
\begin{align}
    \ket{\psi} &=
    \sum_{i_1,\ldots,i_N} 
    \Tr\left( 
    A_{i_1}^{[1]}
    A_{i_2}^{[2]}
    \cdots
    A_{i_N}^{[N]}
    \right)
    \ket{i_1,\ldots,i_N}
\end{align}
where $i_k=1,\ldots,d$ where $d$ si the local qudit dimension,
each $A_{i_k}^{[k]}$ are $D\times D$ matrices
where $D$ is the bond dimension.

The beauty of this state is that there are very few parameters.
There are only $dND^2$ parameters,
as opposed to the $d^N$ parameters for the most general $N$-qudit state.
This is an efficient parameterization of a very large class of systems.

The point of these matrix product states is that the entanglement entropy of a
matrix product state is upper-bounded by its bond dimension.
So if you had a chain,
let's say you have an open chain,
and you split it up into a left and a right piece,
then the entanglement entropy would satisfy
\begin{align}
    S = -\Tr \rho_L \ln \rho_L \le 2 \ln D
\end{align}
It is bounded by the bond dimension $D$.
In 1D,
any generic gapped ground state forms an area law.
What that means is that any gapped 1D ground state has ``area law''
entanglement,
which means that if I look at the entanglement entropy of some segment,
then $S_{1d}=\mathrm{constant}$,
meaning that it is independent of system size.

Actually, the more precise statement is that the entanglement entropy $S$ is of
order
\begin{align}
    S = O\left( 
    \frac{\log_3 d}{E_{gap}}
    \right)
\end{align}
This was originally proven by Hastings in 2007,
but then it was exponentially improved 6 years later by
Arod, Kitaev, Landau, Vazirani in 2013,
and the statement above is from the latter paper.
This latter papaer also improved another result of Hastings,
which is that you can approximate any gapped ground state to an accuracy that is
$1/poly(N)$ where $N$ is system size with an MPS of sublinear bond dimension,
sublinear means the bond dimension grows slower than the size of the system.
We don't care what the polynomial is.
The more speicfic thing they proved is that
\begin{align}
    D &=
    O\left( 
    e^{\log_{3/4} N / E_{gap}^{1/4}}
    \right)
\end{align}
The main ponit is taht gapped ground states have area law entalmenet in 1D,
and MPS is a way of parameterizing states iwth area law entanglmeent.
So if you have a gapped ground state,
you can have a fairly good approximation with MPS.

You need fewer than $dND^2$ parameters to very accurately approximate the ground
state.

\begin{question}
    is the converse true?
    Any state that satisfies area law entanglement is gapped?
\end{question}
I don't have a counter example.
Every system I know has an entropy that grows logarithmically at worst.
The converse of these statements usually have some sick counterexample.
For example,
finitely correlated Hamiltonians don't imply gapped.
I don't recall hearing any for this one though.

What we can actually do is use matrix product states to prove the
classification.
In particular, there is no topological order for 1D bosonic states.

\section{Matrix Product States}
MPS closed chain.
\begin{align}
    \ket{\psi} &=
    \sum_{i_1,\ldots,i_N} \Tr\left( 
    A_{i_1}\cdots A_{i_N}
    \right)
    \ket{i_1,\ldots,i_N}
\end{align}
where $A_i$ are $D\times D$ matrices, and $i=1,\ldots, d$ where
$d$ is the dimension of the site Hilbert space.
There are $dND^2$ parameters here to describe a quantum state.

Gapped ground states of local Hamiltonians satisfy the area law for entanglement
entropy.

Can approximate gapped ground states to accuracy $1/\poly(N)$
with MPS with $D$ sublinear in $N$.

A MPS is injective if there exists finite $L_0$ such that
\begin{align}
    \tilde{A}_I &=
    A_{i_1} A_{i_2} \cdots A_{i_N}
\end{align}
spans the whole space of $D\times D$ matrices.

The question is if these matrices span the entire space of $D\times D$
matrices.

The reason injectivity is important is that you can show that an injective MPS
has the following properties.
\begin{enumerate}
    \item Finite correlation length.
    \item There are unique gapped ground states of a frustration-free parent
        Hamiltonian.
\end{enumerate}

There is a book by Xiao-Gang Wen, Xie Chen, Bei Zeng and one other author which
has a nice discussion of MPS and in particular the relation between MPS and SPT
phases.

There's another formulation called PEPS,
projected entangled pair states.

Consider a chain of maximally entangled pairs which are then projected.
So you have on each site,
imagine you have two kinds of spins.
And then you put the spins on nearby states into singlets,
and then you do a projection.

So here, 
this singlet bond,
what that really means,
you can think of it as a maximally entangled state between the Hilbert space on
the two sides.
\begin{align}
    \frac{1}{\sqrt{D}} \sum_{\alpha=1}^{D} \ket{\alpha}\ket{\alpha}
\end{align}
where $D$ is some internal bond dimension.
The idea is that each of these are some virtual spin,
and $D$ would be the dimension of the virtual spin.

And then the projection projects from two virtual spins to a physical degree of
freedom of dimension $D$.

So this projection,
if you write it in the following way
\begin{align}
    P &= \sum_{i,\alpha,\beta} A_{i,\alpha\beta} \ket{i}\bra{\alpha\beta}
\end{align}
Then this state is actually equivalent to the MPS $\ket{\psi}$.

So the MPS is exactly equivalent to this PEPS when the projection is written
like above.

Something that happened in 2010,
that led to people thinking all 1D phases are classified by $H^2(G, U(1))$,
is that if you take the bond dimension,
and what can be analysed by constant depth circuits,
is the following.

One can show that any MPS with fixed bond dimension can be disentangled
to exponential accuracy by local constant depth circuit.

The fact you can do this is evidence for no bosonic topological order,
meaning no topologically ordered phases of bosons without any symmetry.

Disentangle means toke to the direct product state.

It doesn't prove it of course,
because the bond dimension is assumed to be fixed,
but in general you should consider the bond dimension to grow with system size.

Last time in the paper,
they show the bond dimension is sublinear in $N$,
but not necessarily constant in $N$.
This discovery of disentanglement with fixed bond dimension doesn't prove there
is no topological order in 1D but it does strongly suggest there is no bosonic
topological order in 1D.

Then you could add symmetry in the circuit,
and let the circuit be symmetric,
and then you can show from there that you basically get this PEPS form.
That is,
you can show that any MPS with a fixed bond dimension and symmetry $G$
can be converted into an entangled pair form via a symmetric constant depth
circuit.

Furthermore, each virtual spin is in projective representation of $G$
characterized by this cocycle
$[w] \in H^2(G,U(1))$.

This again,
is evidence that the classification of gapped phases should be
$H^2(G,U(1))$.
The caveat is that we're considering MPS,
and constant depth local circuits that disentangle,
or take any generic MPS to this fixed point form.

I'm not going to go through the exact construction,
but this book has s nice account of it.

I want to describe a different model for thinking about SPT states in terms of
path integrals,
state sums and wave functions that can generalize to every dimension.
This whole discussion about MPS is limited to 1D.
While it's interesting and illuminated,
I'm not going to dwell more on it because it's so specific to 1D.
This is the overview of what happened 10 years ago in 1D that led people to go
further and classify higher dimensions.

\begin{question}
    Experimental tests of AKLT?
\end{question}
The spin-1 Heisenberg chain is gapped phase of matter with dangling gapped edge
modes,
and there are experiments.
The Haldane chain is half the reason why Haldane won the Nobel prize,
even though he didn't realize it's an SPT state at the time.

\begin{question}
    Is it because of ground state degeneracy?
\end{question}
Every 1D bosonic topological phase is a trivial phase,
if you forget about symmetry.
For fermions,
it's not the case because we have Majorana chains.
Even though it has no edge modes on a ring,
it's still topological.

It's not exactly because of ground state degeneracy,
because the Majorana chain does not have ground state degeneracy on a closed
chain either.

\begin{question}
    Why is the second statement evidence?
\end{question}
Every sate,
you can run it through some constant depth symmetric circuit,
and take it not a form like this,
where every one is a projective representation,
with a dangling edge state.
You could run some RG procedure that takes a constant number of steps,
but takes you to this idealized fixed-point form to good accuracy.

\begin{qeustion}
    Is the second statement still true for $d=1$.
\end{qeustion}
You can actually get from PEPS to the trivial product state by disentangling
each of these guys,
but he only way to do that is break the symmetry $G$.
But to fully disentangle,
you need that one extra step to break the symmetry.

I encourage to read the relevant chapter in the book by X.G. Wen.
There's also a book by Ignacio Cirac.

\section{Group Cohomology Model (Dijkgraaf-Witten theory)}
I want to turn to a model for SPT state,
first in 1D,
but this model actually generalized to higher dimensions as well.

The idea is that we're going to construct a TQFT that gives SPT states
by constructing the path integral.
Construct a path integral fora TQFT
\begin{align}
    Z(M^2, A)
\end{align}
We have a symmetry $G$, so there is a gauge field $A$ for that $G$ symmetry.

The first step is to triangulate spacetime.

Then the next step is define a branching structure.
A branching structure is a local ordering of the vertices.
Another way of saying what it is is to make all edges directed in a way such
that there are no closed loops.
That's all a branching structure is.

For example, take some triangle,
with vertices labelled 0, 1, 2.
Then arrows pointing from lower to higher number vertices on edge,
and you see there're no loops.

Then we introduce an orientation.
You can reverse the arrows,
and you notice you can't rotate one to make it look like the other,
because if you follow two of the arrow,
your thumb points down,
but it points up on the other orientation.
You call one the $+$ orientation and you call the other the $-$ orientation.

The next thing to do is to assign group elements to vertices.

It effectively introduces a gauge field into the problem,
because if I draw a triangle with
$g_0$, $g_1$ and $g_2$ on vertices,
the links between them will be $g_0^{-1}g_1$ from $g_0$ to $g_1$,
$g_{0}^{-1}g_2$ from $g_0$ to $g_2$ and
$g_1^{-1}g_2$ from $g_1$ to $g_2$.

[picture]

This defines a special kind of gauge field,
it's a \emph{flat} gauge field.
That is,
there is no net flux through a plaquette.

For example,
the flux through one triangle is
$A_{01}A_{12}A_{20}=1$
and in fact the product of any loop is 1 with
\begin{align}
    \prod_{\textrm{loop}} A = 1
\end{align}
so in fact $A$ is a \emph{trivial} flat gauge.

Then for each 2-simplex (triangle),
we're going to associate a phase factor,
which is an amplitude,
that depends on the group elements.

To each 2-simplex $\Delta^2$, define
\begin{align}
    \left[ \nu_2\left( g_0, g_1, g_2 \right) \right]^{S\left( \Delta^2 \right)}
\end{align}
where $\nu_2 \in U(1)$.
I assume the labels are labelled $g_0,g_1,g_2$.
$S$ is the orientation of $\Delta^2$.

\begin{question}
    Is this an invertible TQFT?
\end{question}
This will be an invertible TQFT,
and the fact that this is just a $U(1)$ phase is important why it's invertible.
You can generalize it,
but it's significantly more complex to make $U(1)$ not just a phase.

The path integral that we're going to define is defined as follows.
\begin{align}
    Z &=
    \frac{1}{|G|^{N_V}}
    \sum_\left\{ {g_i \right\}}
    \prod_{\Delta_2 \ni (i, j, k)}
    \left[ 
    \nu_2 \left( g_i, g_j, j_k \right)^{S(\Delta_2)}
    \right]
\end{align}
we're taking the product over all 2-simplices $\Delta_2$
each of which have vertices $i,j,k$ in order.
And I'm going to take the product of all these phases $\nu_2$
and I'm going to raise it to the orientation $S\left( \Delta_2 \right)$.
And then I'm going to sum over all possible choices of vertices $\left\{ g_i
\right\}$
and then I'm going to average by dividing by $|G|^{N_V}$
where $N_V$ is the number of vertices and $|G|$ is the number of elements in
$G$.

We'll come to this,
but if you're a mathematician and see this,
you might think we're crazy,
because you'll see the sum is completely useless for defining the path integral,
but the sum is here for physics.
What you find here is that every term is here is the same.
I haven't told you what $\nu_2$ are.
It turns out every term in the term is the same.

We're just summing over all possible labellings.

\begin{question}
    $G$ survives permutations?
\end{question}
We have some crazy triangulation,
and I'm just summing over every possible value of $G$.
Every vertex has a group element attached to it,
and each vertex has a different group element.

\begin{question}
    Are there as many group elements as vertices?
\end{question}
There are $N_V$ group elements,
which is different from $|G|$.
For example $G=\mathbb{Z}_2$,
then there are $2^{N_V}$.


\begin{question}
    How does it connect to the system we're trying to study?
\end{question}
Maybe you should wait.
If you remember in TQFT,
you can define a path integral for spacetime.
If your spacetime has a boundary,
you have a state on the boundary.
This path integral can give a wave function state on the boundary.


Any boundary is going to be a circle,
or a bunch of disconnected circles,
so we can get a wave function on circles that describes SPTs.
And from those circles,
we deduce an exactly solvable Hamiltonian as well.

This gives a topologically invariant path integral,
form which we extract topological invariants.

\begin{question}
    Is branching structure the same as consistent orientation?
\end{question}
No, it's more than that.
It's even more than a specific set of orientations.
It's a specific set of ordering of vertices.

\begin{question}
    Should neighbouring vertices have the same ordering?
\end{question}
Not necessarily.

\begin{question}
    We take all gauge-equivalent into one term?
\end{question}
Yes, this is going to wind up being an average over gauge transformations.

It's gauge equivalent to no $G$,
we haven't defined $\nu_2$ yet.
We'll be able to relax this product of $A=1$ condition,
so we can introduce twists and things wrap around,
and it won't be gauge-equivalent to nothing.

It's also useful to think of $Z$ in the following way,
as a sum over all $g$s normalized
over the exponential of some topological action.
\begin{align}
    Z &=
    \frac{1}{|G|^{N_V}}
    \sum_\left\{ {g_i \right\}}
    e^{i S_{\mathrm{top}}\left( \left\{ g_i \right\} \right)}
\end{align}
I'm being a bit sloppy here,
when I say $A$,
I really mean equivalence classes of $[A]$,
by gauge equivalence.
But if we know $ZA$ is gauge invariant,
I don't really need to write the square brackets.
\begin{align}
    Z = Z\left( M^2 , [A] \right)
\end{align}

This has a global symmetry,
and what I mean is that the amplitude should be the same as if multiplied by
some global $g$.
Every term in the sum is explicitly the same if I change all $g_j$'s at once by
some global $g$.
\begin{align}
    e^{iS\left( \left\{ g_i \right\} \right)}
    =
    e^{iS\left( \left\{ gg_i \right\} \right)}
\end{align}
which also means
\begin{align}
    \nu_2\left( gg_0, gg_1, gg_2 \right) &=
    \nu_2\left( g_0, g_1, g_2 \right)
\end{align}
We're assuming $g$ is unitary,
but if it's anti-unitary,
you just introduce a complex conjugate in the above equation.

What this allows is to do is define a 2-cochain.
The reason it defines a 2-cochain is because I can define the usual 2-cochain,
which I introduced last lecture,
which is
\begin{align}
    \omega_2(g_1, g_2) &=
    \nu_2 (1, g_1, g_1 g_2)
\end{align}
Furthermore,
I can multiply every element by $g_0^{-1}$ so
\begin{align}
    \nu_2 (g_0, g_1, g_2) &=
    \nu_2 \left( 
    1, g_0^{-1},g_1, g_{0}^{-1} g_1 g_1^{-1} g_2
    \right)\\
    &= \omega_2\left( g_0^{-1} g_1, g_1^{-1} g_2 \right)
\end{align}
We thought of 2-cochains as something that takes in 2 group elements and splits
out a phase factor.
But here,
we can think of it as taking 3 group elements and spits out a factor,
but it has some symmetry in the inputs.
$\nu_2$ is called a homogeneous 2-cochain,
whereas $\omega_2$ is an inhomogeneous 2-cochain.

\begin{question}
    What's inhomogeneous about it?
\end{question}
If you think about it,

\begin{question}
    What is symmetric?
\end{question}
I want the action to be symmetric.
Usually when you talk about a symmetric system,
I say the path integral is a sum over all field configurations.
But I say my system is symmetric if my action has the symmetry.

\begin{question}
    Can't we have each equality up to a phase
    so the phases cancel instead?
\end{question}
I haven't thought about that too more carefully.
I'm saying that this implies this,
but not the other way around.
You could relax things and consider it more general,
but I'm not sure how to do that in a way that's compatible with locality.
At the very least,
it gives us a way of getting a symmetric path integral.

\begin{question}
    Every group element can be mapped to another group element by another group
    element.
    Why are not all the vertices not connected to each other.
    You can always form a side that connects $g_1$ to $g_3$.
    What is the point of triangulation?
    All group elements are connected to each other.
\end{question}
I don't understand the question.
That's the graph of the group.
You can think if this as defining a gauge field on a manifold.

\begin{question}
    We're assuming $\nu_2$ to have a global $g$ symmetry,
    not a local $g$ symmetry.
\end{question}
We have a global $g$ symmetry,
which means we want every amplitude independent of $g$,
but there's also a gauge symmetry,
in that I can do a gauge transformation $A$ and not change the path integral.
There's something much stronger that's happening here in that there's also a
local gauge symmetry.


\section{Topological invariance of the path integral}
Now we get to the topological invariance of $Z$.
So far, 
we had to triangulate spacetime and add a branching structure.
That's just geometry.
But for topological invariance,
we need it to be independent of geometry and choice of branching structure.

The point is,
that we want $Z$ to be independent of triangulation and branching structure.
that's what we want to demand,
but actually we want to demand something even stringer,
that is the action $e^{iS}$ is topological,
meaning independent of triangulation.

And the way that you can require that something be independent of triangulation,
is given any triangulation,
you can always get to another triangulation by a series of moves,
and these moves are called \emph{Pacher moves}.

Suppose I have two triangles like this.
One Pacher move is to have a rhombus with a diagonal,
and change the diagonal to the other diagonal. This is a 2-2 move.

There is a 1-3 move which inserts a vertex on a triangle,
and draws rays from the new vertex.

It's just pasting a simplex of one higher dimension.
In 2D you have a 2-simplex that is a triangle.
But in one-higher dimension,
you have a tetrahedron.
Think of pasting a tetrahedron onto a triangle and flattening it.

So one way of thinking about Pachner moves is pasting $D+1$-simplices onto your
$D$ simplices and flattening it out.


But we actually need branched Pachner moves,
which also deals with the ordering.

I need some extra conditions.
For example,
in the 2-2 Pachner move,
there would be
\begin{align}
    \nu_2 (g_0, g_1, g_2)
    \nu_2 (g_0, g_2, g_3)
    =
    \nu_2 (g_0, g_1, g_3)
    \nu_2 (g_1, g_2, g_3)
\end{align}
[picutre]

To be careful,
you should also check the orientation and make sure you put the right complex
conjugation.
For the 1-3 Pachner move,
you would have
\begin{align}
    \nu_2(g_0, g_1, g_2) =
    \nu_2(g_0, g_1, g_3)
    \nu_2(g_1, g_2, g_3)
    \nu_2(g_0, g_2, g_3)
\end{align}

But secretly,
they are the same equation,
in fact they are just the 2-cocycle equation that
\begin{align}
    d\nu_2 = 1
\end{align}

To make the topological action re-triangulation invariant,
we require that $\nu_2$ to be a 2-cocycle.

And finally,
let's look at the invariant under 2-coboundaries

\subsection{Invariance under coboundaries}
Suppose that we take
\begin{align}
    \nu_2 (g_0, g_1, g_2)
    \to
    \nu_2 \cdot 
    b_1 (g_0, g_1)
    b_1 (g_1, g_2)
    \left[ b_1 (g_0, g_2) \right]^*
\end{align}
and I'm going to require
\begin{align}
    b_1 (g_1, g_2) &= b_1 (gg_1, gg_2)
\end{align}
And $e^{iS}$ is invariant because each edge (1-simplex) appears in exactly two
2-simplices with opposite induced orientation.

The reason it's invariant, is every single 1-simplex appears with 2 neighbouring
triangles.
For example, if I have a triangulation,
every 1-simplex has 2 neighbouring triangles.
And every 1-simplex has opposite orientation relative to each triangle.
That means that if one triangle changes by this factor $b_1(g_1,g_2)$,
the neighbour is going to change by a similar factor,
but it's going to appear complex conjugated in the neighbour so they're all
going to cancel out.

In terms of inhomogeneous 2-cocycles,
applying $g_0^{-1}$ to each term,
we get
\begin{align}
    \nu_2(1, g_0^{-1} g_1, g_{0}^{-1} g_2)
    &\to
    \nu_2 \cdot
    \frac{b_1(1, g_{01}) b_1(1, g_{12})}{b_1(1, g_{02})}
\end{align}
and
\begin{align}
    \omega_2(g_{01}, g_{12}) \to
    \omega_2
    \frac{\epsilon_1(g_{01}) \epsilon_1_1\left( g_{12} \right)}{\epsilon_1\left(
    g_{02} \right)}
\end{align}

That means, if we change our 2-coboundary by 2-cocycles,
the topological boundary doesn't change.
That is,
distinct $G$-symmetric path integrals are classified by
\begin{align}
    [\nu_2] \in
    H^2\left( G, U(1) \right)
    = \frac{\mathbb{Z}^2}{B^2}
\end{align}
Actually,
the last step is a bit of a jump.

Let me give more details.

On closed manifolds,
what we defined so far always gives us one with
$Z(M^2)=1$.
That's because what we have is a trivial $G$-bundle,
meaning that
$\prod_{\textrm{loop}}A=1$.

To get non-trivial results we non-trivial flat bundles with
$\prod_{\textrm{loop}}A \ne 1$.
We want it to be flat but we want holomony over non-contractible loops.

One way of doing this is this construction we've defined,
is we can cut our manifold along whatever loop we're interested with having
holomony,
then inserting some $G$-twist when we glue them back together.

Let me draw a picture.
Suppose w have a cylinder,
and we want to put a $h$ branch cut along the cylinder axis.
So then this loop at the end of the cylinder is $\gamma$,
with $\prod_\gamma A = h$.

Then if you roll out the cylinder flat,
you can draw a triangulation like this.

[picture of unrolled cylinder sheet]

The point is that if we have $g_1,g_2,\ldots$ on the bottom edge of the cut,
which we identify with the vertices on the top edge of the cut to get a
cylinder.
To do the twist,
just make the top edge vertices labelled $hg_1, hg_2, \ldots$.

For a closed $M^2$
non-trivial flat bundle,
\begin{align}
    |Z(M^2, A)| = 1
\end{align}
but the phase is going to be non-trivial,
with $Z(M^2, A)$ being a gauge-invariant polynomial in $\omega_2$.
For every element of $H^2$,
we're going to get a path integral that will spit out a $U(1)$ phase and if we
look at al possible closed manifolds and all possible closed bundles,
we find that all elements of $H^2(G, U(1))$ corresponds to a distinct path
integral $Z(M^2, A)$.

So there is a one-to-one correspondence between this group and these TQFTs by
picking appropriate fluxes along non-contractible cycles.
\begin{align}
    H^2(G, U(1)) \leftrightarrow Z(M^2, A)
\end{align}
I didn't prove that $Z(M^2)=1$ if it's flat and trivial,
because I didn't prove that the sum over $G$ is actually invariant yet.

\begin{question}
    What is the definition of $\prod A$?
\end{question}
If you have a torus.
If you take a product of lops on a contractible bundle,
you get 1.
But you get a holomony over a non-contractible loop?

\begin{question}
    What if we generalize to cellulations?
\end{question}
well this $\nu_2(g_0, g_1, g_2)$ only makes sense over triangles because it has
3 inputs,
but you could consider cellulations,
but the framework will be a different,
but ultimately you get the same answer.

\section{Quadratic Hamiltonians and the Equipartition Function}
\begin{align}
    H &= \sum_{i=1}^{N}\left( 
    \frac{p_i^2}{2m_i} + \frac{m_i\omega_i^2}{2}x_1^2
    \right)
\end{align}
The partition function is
\begin{align}
    Z &= \int_{-\infty}^{\infty} \prod_{i=1}^{N} \frac{dx_i\, dp_i}{h^N}
    e^{-\beta\sum_{i}\left( 
    \frac{p_i^2}{2m_i}
    + \frac{m_i\omega_i^2}{2} x_i^2
    \right)}\\
    &=
    \prod_{i=1}^{N}\frac{1}{h}\int_{-\infty}^{\infty} e^{%
    -\frac{\beta m_i\omega_i^2}{2}x_i^2
    }
    \int_{-\infty}^{\infty} dp_i\,
    e^{-\beta \frac{p_i^2}{2m_i}}\\
    &= {\left(\frac{k_B T}{\hbar}\right)}^{N} \frac{1}{\prod_{i=1}^{N}\omega_i}
\end{align}
Then the free energy is
\begin{align}
    F &= -k_B T \ln Z \\
    &=
    -N k_B T \ln\left( \frac{k_B T}{2} \right)
    +
    k_B T \sum_{i=1}^{N} \ln\omega_i
\end{align}
The energy is
\begin{align}
    E &= F + TS
\end{align}
where
\begin{align}
    S &= \frac{\partial F}{\partial T}\\
    &= Nk_B \ln\left( \frac{k_B T}{\hbar}  \right)
    + \frac{N k_B T}{T}
    - k_B \sum_{i=1}^{N}\ln \omega_i
\end{align}
and so the energy is
\begin{align}
    E &= Nk_B T =
    \left(
    \frac{k_B T}{2}
    + \frac{k_B T}{2}
    \right) N
\end{align}
Textbooks are confusing and students fall for it.
I think a 1D harmonic oscillator has 1 degree of freedom.
I think a 1D free particle also has 1 degree of freedom.
Forget degrees of freedom,
what matters is the number of quadratic terms in the Hamiltonian.

Each quadratic term in the Hamiltonian has energy $\frac{1}{2}k_B T$.

\begin{example}
    1D array of masses connected by springs with periodic boundary conditions.
\end{example}
The Hamiltonian is
\begin{align}
    H &=
    \sum_{i=1}^{N}\left[ 
    \frac{p_i^2}{2m}
    + \frac{k}{2} {\left( x_i - x_{i-1} \right)}^2
    \right]
\end{align}
So this is quadratic but a slightly more complicated quadratic.
You see there are cross-terms like $-2x_2x_3 + \cdots$ in there.
You have to account for boundary conditions too.
You have to say that $x_N = x_0$,
including the one that wraps around.
There are 3 terms and they are all there.

What you learn in Chacko's class is that you can do a change of variables that's
going to make this quadratic Hamiltonian that is decoupled.
This is called finding the normal modes.
It's something you're going to relearn in Chacko's class.

I'm going to state here that you can do a change in variables.
That's going to be equivalent to this
\begin{align}
    H &=
    \sum_{i=0}^{N-1}\left[ 
    \frac{p_i^2}{2m}
    + \frac{m\omega_1^2}{2} q_i^2
    \right]
\end{align}
Basically the $q_i$s are Fourier transforms of the $x_i$'s.
You will find that the magic works and the $q$'s are independent of one another.

If there s one canonical transformation to learn,
it's this one.
The range of applicability of this calculation is huge.
Basically physics is to do more and more complicated cases of harmonic
oscillators.

Now I can complete the partition function of these guys,
taking both terms and just writing the answer.

What is the average energy of this guy of temperature $T$?

It's $\frac{1}{2}k_B T$ for every quadratic term.
There are $2N$ terms.
This is going to be again
\begin{align}
    E &= \frac{1}{2}k_B T \times 2 \times N = Nk_BT
\end{align}
All you needed to know how many normal modes.
You didn't even need to find the frequency.
In physics 101,
they shake a slinky.
There are transverse modes,
longitudinal modes,
and more complicated modes I can't do with my hands.
Each one has different frequencies.
Some long wavelength ones move fast or slow,
but it doesn't matter for the average energy $E$,
it's just $\frac{1}{2}k_BT$ for every quadratic term in the Hamiltonian.

We can now state the equipartition theorem.

\begin{question}
    How do you know how many terms?
\end{question}
If you start with $N$ coordinates,
you get $N$ normal modes no matter how complicated.

\begin{theorem}[Equipartition]
    The thermal average energy of a Hamiltonian $H$ is
    \begin{align}
        E &= \frac{1}{2} k_B T N
    \end{align}
    where $N$ is hte numbero f quadratic terms in $H$.
\end{theorem}
This is true for classical systems.

Let's consider a few systems now for examples.

\begin{example}
    Imagine you have a solid insulator.
    what is the average enregy of an insulating solid?
\end{example}
Please give an example of a solid.
You give 5 things that re not solids with crystal structures.
Glass, styrofoam, plastic.
They're all rigid,
but they're not solids.
The reason I'm not talking about insulators,
is because they have electrons that move around adn that changes the energy.
I'm thinking of an insulator solid,
just a bunch of molecules in a lattice that shake.

The modules have an equilibrium position,
andthey shake around in a complicated way.
To first approximation,
they have some quadratic terms,
and you expand in powers like the distance between the atoms,
and the energy is quadratic.

You want to know how many modules htere are.
Supposey ou have $N$ modules on my chunck of rock.
How many quadratic terms do I have in the Hamiltnian?

There's a 3D knietic term,
so that's 3 quadratic terms for the kinetic enregy.
Whata bout the potential energy?
Again every module has 3 coordiantes $x,y,z$,
so there are 3 quadratic terms per molecule.
So the energy should be
\begin{align}
    E &= \left( 
    \underbrace{3N}_{\text{kinetic}}
    + \underbrace{3N}_{potential}
    \right)
    \frac{1}{2}k_B T\\
    &= 3Nk_B T
\end{align}

\begin{question}
    Are you assuming a cubic lattice?
\end{question}
No, it could bea really complicated crystal structure,
but you still have the same number of normal modes.

The specific heat of the substance is
\begin{align}
    c &= \frac{1}{N}\frac{\partial E}{\partial T} = 3k_B
\end{align}

We could check this.
Every insulating solid should have exactly the same specific heat.
That's amazing.
Andi t's not just any number,
nad it's a specific number.

You can see this thing here
What I have here isa bunch of pure elements,
andh ere I have a specific heat of hte number.
The line in  the middle is $3k_B$.

This starts from 30,
and they're all around 23.
So you know there are bad ones like potassium 29,
there's another one 23.

It's an amazing result.
See the work we did.

\begin{question}
    Are we assuming only neighbouring particles are talking to each other?
\end{question}
No, I could have $x_1$ connected to $x_{42}$.
As long as the energy is quadratic in distance,
it doesn't matter.
You could find the normal modes,
and you find there are $3N$ normal modes,
and that's the only thing that matters.

\begin{question}
    At higher temperatures in a crystal lattice,
    you'l have anharonic effects 
    like thermal expansion,
    does that influene the specifi heat?
\end{question}
There's no non-linear effects that contribute at high temperature.
How much does a rock expand when you heat a rock.

This is the Dulong-Petit law.
Validity because it really cnovineced people that convineced people statistical
mechanics was right.
It's failure is that it suggested quantum mechanics.

Before we dothat,
let's do a really classic case.
Classic in classical mechanics and classic in that you do it in high school.

\begin{example}[diatomic gas]
    Ideal, nonrelativistic, classical gas.
    What if it's diatomic?
\end{example}
The molecules now look like this.
[picture]
So what's the answer?
If there are $N$ molecules,
how many quadratic terms are you going to have?

Every particle can move in the $x,y,z$ directions,
but the two particles don't move independently,
so that's $6N$ terms.
But there is a potential term for the spring between the two particles,
so
\begin{align}
    E &= (6N + N)\frac{1}{2}k_B T = \frac{7}{2}Nk_B T
\end{align}

\section{(2+1) dimensional SPT path integral}
\begin{align}
    H_\rho^n(G, M)
    &=
    \frac{Z_\rho^n(G, M)}{B_\rho^n(G, M)}\\
    &=
    \frac{\ker d_n}{\im d_{n-1}}
\end{align}
where
\begin{align}
    d_n:
    C_\rho^n(G, M)
    \to
    C_\rho^{n+1}(G, M)
\end{align}

You can have $+$ orientation tetrahedrons.
You can also have $-$ orientation tetrahedrons.

Let $\nu_3$ be homogeneous and $\omega_3$
be inhomogeneous.
Then the path integral is
\begin{align}
    Z &=
    \frac{1}{|G|^{N_V}}
    \sum_\left\{ {g_i \right\}}
    \prod_{\Delta^3\ni (i,j,k,l)}
    \left[ 
    \nu_3\left( g_i, g_j, g_k, g_l \right)
    \right]^{S\left( \Delta^3 \right)}
\end{align}
Note that
\begin{align}
    g\cdot \nu_3\left( g_1,\ldots, g_4 \right)
    &=
    \left[ 
    \nu_3\left( gg_1, gg_2, gg_3, gg_4 \right)
    \right]^{\sigma(g)}
\end{align}
due to $g$ -action
and then due to $G$-symmetry,
\begin{align}
    g\cdot \nu_3\left( g_1,\ldots, g_4 \right)
    &=
    \left[ 
    \nu_3\left( gg_1, gg_2, gg_3, gg_4 \right)
    \right]^{\sigma(g)}
    =
    \nu_3\left( g_1, \ldots, g_4 \right)
\end{align}
Then
\begin{align}
    \nu_3\left( g_1,\ldots, g_4 \right)
    &=
    \left[ \nu_3\left( 1, g_{12}, g_{13}, g_{14} \right) \right]^{\sigma\left(
    g_1 \right)}
\end{align}
recalling $g_{ij} = g_i^{-1} g_j$.
And so
\begin{align}
    \omega_3\left( g,h,k \right)
    &=
    \nu_3\left( 
    1,g,gh,ghk
    \right)
\end{align}
Thus we have the path integral
\begin{align}
    Z\left( M^3, A \right)
    &=
    \prod_{\Delta^3\ni \left( i,j,k,l \right)}
    \left[
    \omega_3\left( A_{ij}, A_{jk}, A_{kl} \right)
    \right]^{S\left( \Delta^3 \right)}\\
    &=
    \frac{1}{|G|^{N_V}}
    \sum_\left\{ {g_i \right\}}
    \prod_{\Delta^3\ni (i,j,k,l)}
    \left[
    \omega_3\left( g_i^{-1}A_{ij} g_{j},
    g_j^{-1}A_{jk}g_k,
    g_k^{-1}A_{kl}g_l\right)
    \right]^{S\left( \Delta^3 \right)}
\end{align}

We can draw out the Pachner motes in 3D.

There is the 1-4 Pachner move which inserts a vertex inside a tetrahedron to
produce 4 tetrahedrons.

There is also the 2-3 Pacher move,
which takes two tetrahedrons sharing a face,
to produce 3 tetrahedra.
If the vertices are labelled 0123 and 01234
no the two original tetrahedra,
then new tetrahedra have vertices
01234, 1234, 0234.

We want our path integral to be invariant under Pacher moves.
In mathematics,
there is the famous \emph{pentagon equation}
\begin{align}
    \nu \nu = \nu\nu\nu
\end{align}
which implies
\begin{align}
    d\omega_3 = 1
\end{align}
so that
$\nu_3$ and $\omega_3$ are 3-cocycles.

If we change
$\omega_3 \to \omega_3\, d\mu_2$,
then this cancels out of the path integral.
Every $\mu_2$ enters the product twice with opposite orientation,
so this cancels out the $\mu_2$.

Ultimately,
distinct (2+1)D topological actions will be classified by
\begin{align}
    H_\rho^3\left( G, U(1) \right)
\end{align}
The $\rho$ is there so that every time the symmetry $G$ is anti-unitary,
it's the complex conjugate.
The $U(1)$ can be replaced by another group,
but it has to be a unitary group.
$\rho$ is just a map $\rho: G\times U(1)\to U(1)$ such that
\begin{align}
    \rho_g\left( e^{i\theta} \right)
    &=
    e^{i\theta'}
\end{align}

We can repeat this whole thing in any general dimensions,
so what we get is that in $(d+1)$ dimensions,
we have distinct actions which are classified by
$H^{d+1}\left( G, U(1) \right)$.
This gives a path integral
\begin{align}
    Z_{\left[ \omega_{d+1} \right]}
    \left( 
    M^{d+1},
    A
    \right)
\end{align}
for flat $G$ gauge field $A$
and
\begin{align}
    \left[ \omega_{d+1} \right]
    \in H^{d+1}\left( G, U(1) \right)
\end{align}
is just a cohomology class.
And this gives us invertible TQFT with background flat $G$ gauge field,
because $|Z(M,A)|=1$.

So just like in $1+1$ dimensions,
we can write down the wave function,
we can write down the wave function in $d$ dimensions,
and show there is no intrinsic order in SPT,
and this gives a general solvable model with SPT in general dimensions.

This gives us a whole class of distinct SPTs classified by $H^{d+1}$.
This is not the full classification of SPTs actually.
There is a more general classification in terms of cobordisms.
In the dimensions we're interested in,
this is often the full classification,
but often there are states beyond cohomology,
there a  whole class of SPT phases called ``beyond cohomology''.
The often appear in anti-unitary symmetries in higher dimensions.

So far we have invertible TQFT.

Let me just say a few more things in terms of how much this classifies.
In (1+1)D,
this $H^2\left( G, U(1) \right)$ is the full classification of invertible
bosonic phases.

In (2+1)D,
this $H^3(G,U(1))$ is a full classification of bosonic SPTs,
but there are invertible phases that are not SPTs.
These occur when you have chiral central charge,
you can have bosonic states with chiral edge mods.
Strictly speaking,
if you break the symmetry,
they are not adiabatically connected to the ground state.

And then in $(3+1)D$,
this $H^4\left( G, U(1) \right)$
is a full classification of invertible bosonic invertible theories if $G$ is
unitary.
But it's partial classification if $G$ has anti-unitary elements.
This leads us to so-called \emph{beyond cohomology} SPTs.

There is a theory which is expected to capture everything,
which is the cobordism theory,
which I mentioned earlier on in the course.
For now I'll leave it at this.


\begin{question}
    In 1+1 dimensions,
    there are bosonic phases?
\end{question}
In 1+1 D,
even with no symmetry there is the Majorana chain.

\begin{question}
    In 1D you can always bosonize?
\end{question}
There is a real difference between fermions and bosons.
When you bosonize,
something that is local on one side is no longer local on the other side.
It's precisely because the notion of locality changes.

\begin{question}
    The cobordism is larger than the group cohomology.
    In 6D,
    they become the same cobordism phase.
    Different phases have different topological invariants.
    How do they collect into the same/
\end{question}
I don't know.
The argument why these different phases are different,
in 1D we say the constant depth circuits which trivialize it,
but I don't know what goes wrong in higher dimensions.

\begin{question}
    Example of invertible phases in (2+1)D?
\end{question}
The basic one is called the $E_8$ phase,
which has a chiral central charge of 8.
It does not need any symmetry to exist,
but it has chiral central charge of 8.
One way of thinking about it,
is stack 8 copies of the Kitaev honeycomb model,
condense a bunch of vortices,
and you get this.
Everything is made out of these $C_{-}=8$.

\begin{question}
    Is this related to the $E_8$ the Lie group?
\end{question}
You can take $E_8$ level 1 Chern-Simon theory and you get this.
Or you can get the Cartan matrix of the Lie algebra $E_8$.
It's intriguing this exceptional group $E_8$ comes up.

\begin{question}
    ??
\end{question}
We wrote this class of modes,
but there's no reason to expect this is full classification.
This is not even proven,
I would say,
in every dimensions.
This is just the current dimension.
The current understanding is,
there is a conjecture,
which is that bosonic SPTs in $(d+1)$-dimensions
are classified these cobordism groups
\begin{align}
    \Hom\left(\Omega_{d+1}\left( BG \right), U(1)
    \right)
\end{align}
which means the manifolds are equipped with a $G$-gauge field as well.
This is the Pontryagin dual.
Nothing is really proven in terms of full classification.
To prove rigorously topological phases are fully classified,
means you have to understand all topological phases,
which is really beyond our understanding.
This is based on what we understand so far and what seems to make sense.


This sketch, what is not captured?
In 2+1D the ones not captured are he ones with chiral edge models.
Its' because the edge gives you a Hamiltonian as a sum of commuting projectors,
which you can argue will always gap the edge.
This class of models cannot exhibit fully robust chiral edge modes.
The reason some anti-unitary symmetries are outside the group cohomology models
is a deeper issue.
It ultimately has to do with the fact that group cohomologies characterise the
ability to couple to background gauge fields,
but there's something beyond background gauge fields,
and there's no reason to think $G$-gauge theory should be enough on its own
anyway.

\begin{question}
    Edge modes in 1+1D?
\end{question}
Those are captured,
but hose are gappable.
If I break the symmetry I can get rid of them.
These are ones where even if you break all symmetries you cannot get rid of
them.

\begin{question}
    ??
\end{question}
These chiral central charge has nothing to do with symmetry,
it's like thermal Hall conductance.
It's true that anti-unitary symmetries,
you cannot gauge.
When you have a background $G$-gauge field,
you can promote it to a dynamical gauge field by summing over configurations.
You can't do that if talking about anti-unitary symmetries.

Let's talk a bit more about what the physical meaning of this $H^3$ is.
So far this $H^3$ we used to describe a class of models,
we saw in 1+1D,
we saw a projective action on the edge.
You guys asked if there is an analogy for higher dimensions.
There is.

\section{Physical origin of $H^3(G, U(1))$}
Let's just briefly recall that in $(1+1)$-D,
for the $H^2(G, U(1))$ what happens.
We considered a system on an open chain,
then applied a symmetry operation of $G$,
with representation $R_g$,
which localizes a s an action on the left $U_g^L$ and an action on the right end
point $U_g^R$,
and we get a projective phase we cannot get rid of and that is why we have this
second cohomology class.
\begin{align}
    U_g^L U_h^2
    =
    \omega_2(g, h)
    U_{gh}^L
\end{align}

Now consider $(2+1)$D. (Else-Nayak 2014)
The idea is to apply some symmetry operation and see hat happens on the edge.
Say your system is on a disk,
you apply a symmetry operation,
and see what it looks like on the edge.
We have some restriction of the operator on the edge,
then we cut that operator open,
so it looks like an operator on a segment.
Then we restrict $U_g^{\left( edge \right)}$
on a segment,
and we reduce it down to a point,
and we have a projective phase that turns up.

That's the rough idea.
To get this in more detail,
I want to say more about the idea 
if we have a local unitary,
a unitary operator that arises as a constant depth local unitary circuit,
there's an idea of restricting that unitary to a subregion.

Consider a unitary $U$ acting on a space $C$.
$U$ is a local unitary,
meaning that it's a constant depth local quantum circuit.

Another local unitary $U_M$ is going to be the restriction of $U$ to a
submanifold $M$
if it acts exactly the same as $U$ on the subregion $M$ away from the
boundaries.
That is,
it acts the same as $U$ in the interior of $M$ far way from $\partial M$.
There are a few things you can say about such a restriction.

Firstly,
the restriction always exists.
Suppose we have a unitary that acts as some circuit on a bunch of qubits.
So something like this [picture].

Let's say this is $U$ acting on $C$.
I can always define a restriction
by cutting this unitary along some lines here,
and looking at the unitary only acting on the middle.
Obviously there's some ambiguity on the edge,
but in the bulk it's clear I can define a unitary that acts in the same way as
$U$.

It's important that $U$ is a constant depth circuit,
so there is a kind of light cone,
nothing on this side is going to cross to this side beyond a certain size.
Of course near the boundaries of $M$ there are ambiguities.

The restriction is obviously ambiguous near the boundaries,
and they are ambiguous up to some operators.

\begin{question}
    We must have some conditions on the size of the submanifold as well?
\end{question}
If it's too small compared to the depth of the circuit,
we can't do this.
We're assuming the region is large compared to the depth of the circuit.

The second thing is that this is defined modulo local unitaries near the
boundary $\partial M$.

This is the analogue of what we saw in $1+1$ dimensions.
We applied the symmetries,
acting like $U_g^L$ and $U_g^R$
where we could change the phase near the boundary,
and that's the ambiguity.
But here,
instead of a phase,
This is the higher-dimensional generalization of phase ambiguity.

Let's consider the symmetry operator
acting on the low energy subspace of the edge theory.
It's a unitary.
This state is gapped on the bulk,
but it's gapless on the boundary.
The states we are concerned with are where the edge is in some low energy
subspace,
so the excitations on the edge have a much lower energy than the bulk energy
gap.

Then we apply some symmetry operations.
Because the bulk is in the ground state and symmetric,
after applying $R_g$,
this is just applying $U_g^{edge}$ on the edge.


So let's assume $R_g R_h = R_{gh}$.
That's going to also be true for these edge operators.
If not,
there will be a projective representation no the edge,
which is in conflict that the ground state is unique and in the ground state.
For 1+1 D open chain,
the ground state is not symmetric,
because we have these edge states that can fuse to a singlet,
so we can have a symmetric ground state,
but we have a degeneracy when these things are far away from each other.
In higher dimensions,
the edge is connected,
even though there are gapless excitations on the edge,
it is still unique and symmetric.
\begin{align}
    U_g^{edge} U_h^{edge} u_{gh}^{edge}
\end{align}

If the edge preserves symmetry,
the ground state is unique,
then there is no projective representation for the edge.
Now we take this local unitary on the edge and we cut it open.

So suppose I take a cut.
\begin{align}
    U_g^{cut} U_h^{cut} = \Omega(g, h) U_{gh}^{cut}
\end{align}
where $\Omega(g, h)$
is a unitary operator with only support for the new end points $a$ and $b$.
For the bulk we do satisfy the group law,
but it's only at the edge boundary where we don't necessarily support the group
law.


Now we can consider associativity.
We considered
\begin{align}
    U_g^6cut U_h^{cut} U_k^{cut}
    =
    U_g^{cut}
    \Omega(h, k)
    U_{h,k}^{cut}\\
    &=
    ^g\Omega(h, k)\Omega(g, h, k) U_{ghk}^{cut}
\end{align} 
where
\begin{align}
    ^g\Omega(h, k)
    :=
    U_g^{cut} \Omega(h, k)
    \left( U_g^{cut} \right)^{-1}
\end{align}
so then
\begin{align}
    U_g^6cut U_h^{cut} U_k^{cut}
    &=
    \Omega(g, h)
    \Omega(gh, k)
    U_{ghk}^{cut}
\end{align}
And so we are led to this equation
\begin{align}
    \Omega(g, h) \Omega(gh, k)
    =
    ^g\Omega(h, k)
    \Omega(g, h, k)
\end{align}
But now we realise $\Omega$ is an operator with support on $a$ nd $b$.
Consider restriction of $\Omega(g, h)$
to $\Omega_a(g, h)$.
And so we can consider some phase
\begin{align}
    \Omega_a(g, h)
    \Omega_a(gh, k)
    =
    \omega_3(g, h, k)
    ^g\Omega_a(h, k)
    \Omega_a(g, hk)
\end{align}
where $\omega_3(g,h,k) \in U(1)$
and $\omega_3 \in C^3(G, U(1))$.
But you see that $\Omega_3$ does need to satisfy the $3$-cochain equation
\begin{align}
    d\omega_3 = 1
\end{align}
So then if we change
\begin{align}
    \Omega_a(g, h)
    \to
    \Omega_a(g, h)
    \mu_2(g,h)
\end{align}
for $\mu_2\in U(1)$,
then
\begin{align}
    \omega_3 \to \omega_3\, d\mu_2
\end{align}
This shows us the distinct set of phases is classified by $H^2(G,U(1)$.
So distinct $\omega_3$'s are classified by
\begin{align}
    H^3(G, U(1))
\end{align}
So that associativity condition is satisfied up to a phase and that's where we
get this $H^3$.

There's no projective representation anywhere.
We're just restricting and cutting
until we finally get this.

In order to get this to work,
we need to assume something more specific about how $U_g^{edge}$ looks like.
In higher dimensions,
we have to assume
\begin{align}
    U_g^{edge}
    &=
    N_g
    S_g
\end{align}
where $N_g=\sum_\alpha e^{i N_g'(\alpha)}$ is non-on site but diagonal,
and $S_g=\sum_\alpha \bra{g \alpha}\ket{\alpha}$
And in general,
the docs appear 

\begin{question}
    what happens if you had a system living on an annulus.
\end{question}
If you're taking about intrinsically 2D systems,
you can talk about the edge on a 

I'm just focusing on this operator on one segment.
On that segment,
I satisfy this group up to 1 

There are for mathematicians rigorous definitions and derivations.

I just wanted to mention one more ting.
Three's another way of phrasing these topological.

There is a compact way to define the topological action for group cohomology
models
\begin{align}
    Z\left( M^{d+1}, A \right) &=
    e^{iS_{top}(A)}
\end{align}
Firstly, every group $G$ has a corresponding space called ``classifying
spare'' $BG$.
The defining property is that
$\pi_1(GB) = G$
and
$\pi_k(BG)=0$
for $k>1$.
Every $G$ gauge field is in one-to-one correspondence with
homotopy classes of maps.
You can think of gauge fields as being a map from your manifold into your
classifying space.
$f: M^d \to BG$.
What we did when we triangulated the manifold,
and put gauge fields on links,
that actually does specify a map from $M$ to $BG$.
This is just a fancy way of saying what we did in practice.

The classifying space for the circle is
$B\mathbb{Z} = S^1$.
The classifying space for $\mathbb{Z}_2$ is
$B\mathbb{Z}_2 = \mathbb{RP}^{\infty}$.
The classifying space for phases is
$BU(1) = \mathbb{CP}^{\infty}$.

Secondly, the other thing is that group cohomology
$H^k(G, U(1))$
is equivalent to
``regular'' cohomology (singular de Rham cohomology)
Nakahara has a nice book called geometry, topology and physics..
\begin{align}
    H^k(G, U(1)) \simeq 
    H^k(BG, U(1))
\end{align}
is true for finite $G$,
but is more subtle for groups.


Thirdly,
if I had a cohomology on $Y$,
I can get a class on $X$ and the map that does this is called
$f^*$, the \emph{pullback}.
That is,
if $f:X \to Y$,
then
\begin{align}
    f^*: H_k(Y, U(1)) \to H^k(X, U(1))
\end{align}
is the pullback.

We can put these 3 facts together.
Given a class $[w_{d+1}] \in H^{d+1}\left( BG, U(1) \right) \simeq H^{d+1}(G,
U(1)$
which is isomorphic
and a $G$ gauge field determined by
\begin{align}
    f_A: M^{d+1} \to BG
\end{align}
then we can define an element of the $H^{d+1}$ cohomology class
\begin{align}
    f_A^\left[ \omega_{d+1} \right]
    \in
    H^{d=1}\left( M^{d+1}, U(1) \right)
\end{align}
and then we have our path integral
\begin{align}
    e^{iS_{top}}
    &=
    e^{i\int_{M^{d+1}} f_A^*\left[ \omega_{d+1} \right]}\\
    &= Z\left( M^{d+1}, A \right)
\end{align}
These 3 facts allow you to package this path integral in this nice way.
Unfortunately,
I would need at least 3 lectures to unpack all of this.
I'm just telling you so you know where to look to learn about this.
Are there any questions?
Actually,
if you have questions,
I might not be able to answer it.

In the last minute,
I want to mention where I'm going next.

Suppose we have a TQFT with $G$-symmetry $T$.
We can always ``gauge'' $G$.
That means we promote he background $G$ gauge field to a dynamical gauge field,
which gives a new configuration $T$ mod $G$,
which is $T/G$.
I can then sum over flat gauge field configurations.
I put brackets to say we are summing over inequivalent ones that are not related
by gauge transformation.
\begin{align}
    Z_{T/G}\left( M^{d+1} \right)
    =
    \sum_{\left[ A \right]}
    Z_{\left[ \omega_{d+1} \right]}
    \left( M^{d+1}, A \right)
\end{align}
And so dynamical, topological $G$-gauge theory is classified by
$H^{d+1}\left( G, U(1) \right)$.
We are interested in 2 classes of this phenomena.

1. Discrete and finite $G$.
This leads to models like the toric code,
and more generally quantum double models.
Like Levin-Wen models, etc.
If we start off with an invertible theory,
and we gauge $G$,
we end up with a non-irreversible theory.
SO one way of getting non-invertible theories is this.

2. Continuous $G$.
This takes us toe Chern-Simons theory in $(2+1)$D.
At this point,
the discussion bifurcates.

We can talk about non-invertible phases
you get from gauging invertible theories.

We can also label things on triangulations and sum over all labels.
Next time,
I'll start getting into this quantum Chern-Simon theories,
which allows us to describe fractional quantum hall states,
which are the only experimentally realised non-invertible theory.

Are all non-invertible topological phases obtainable by gauging an invertible
TQFT.
30 year old conjecture by Moore and Cyberg still standing.
There are candidate counterexamples,
but we're not sure.

\section{Lecture 22}
\begin{question}
    Last time for 2D systems,
    something about them not being gapped?
    In 1 spatial dimension you can have degenerate gapless modes,
    but not in 2D?
\end{question}
If I take a 1D system with boundaries,
in general you have edge modes,
and in general there is a ground state degeneracy
with dimension $\dim \mathcal{H}_L \times \dim \mathcal{H}_R$.

In 2D,
you only have a single edge,
so if the ground state is preserved,
the ground state is unique.

Of course,
you could have an annulus with two edges,
but it's still true with an edge far away.
There is no way of having just a single edge in 1D.
You always have to have two edges.
You can have an edge which is connected.

The presence of 2 edge still allow you to have 2 edges?
The edge may spontaneously break a symmetry,
but these tow are in spin singlet together and together they may preserve
symmetry,
but that's not useful and natural to do.
You talk about a single edge and its properties in 2D.

Maybe the better thing to say is that in 2D,
the edge can have a unique ground state that is symmetric,
but of course it doesn't have to be if you had multiple edges,
whereas in 1D it cannot under any circumstances,
unless maybe the edges are in a singlet with one another.


\section{(2+1)D Chern-Simons Theory}
So the idea here is that this is some gauge theory,
based on a gauge group $G$.
You can consider all kinds of gauge groups with strange properties,
but for our purposes,
we're interested in cases where this is a compact gauge group.
This gives rise to a large class of TQFTs

In general if you want to specify a Chern-Simons theory,
let me just mention that,
you want to specify the gauge group and you also want to specify some element
$[\alpha] \in H^4(BG,\mathbb{Z})\simeq H^3(BG, U(1))\simeq H^3(G, U(1))$
So a Chern-Simons theory is defined by a group and some choice of cohomology
class.

The CS theory is very powerful.
It's a class of gauge theories that is topological,
and when you quantize it,l
it leads to a class of TQFTs,
and in general these TQFTs can be non-invertible.
They typically depend on a framing,
and they may also depend on a spin structure
with spin TQFTs in the most general case.

In fact,
it is a conjecture that all $(2+1)D$ TQFTs can come from a Chern-Simons theory.
This conjecture goes back to about 1989 to 1990.
And there's still no definitive counterexample.
Although there are some candidate theories people don't know how to interpret
in terms of CS theory,
so there are some tentative counterexample,s
although it's not clear whether they actually are.
There's n definitive counterexample yet,
so CS theory is very powerful.

I want to talk about the simplest case where the gauge group is $U(1)$.

\section{$U(1)_k$ CS theory}
Consider an action
\begin{align}
    S &=
    \frac{k}{4\pi}
    \int_{M^3} a\, da\\
    &=
    \frac{k}{4\pi}
    \int_{M^3}
    \epsilon^{\mu\nu\lambda}
    a_\mu
    \partial_\nu
    a_\lambda
    \,
    d^3x
\end{align}
where $a$ is a $U(1)$ gauge field.
If you stare at it,
this does not actually make sense in general.

The reason it doesn't make sense in general,
is that a U(1) gauge field is not globally well defined in general.

A gauge field is locally just a 1-form $a_\mu \, dx^\mu$.
\begin{align}
    \int_{Y_2} F = 2\pi
\end{align}
If the field strength through some 2-cycle is $2\pi$,
you cannot pick a global gauge for $A$ if $Y_2$ is closed.
So this expression doesn't make sense,
because there's no globally well-defined choice of a $U(1)$ gauge field.
If $Y_2$ is a closed 2-manifold,
then there is no global gauge $a$.

So we cannot make sense of $S_{CS}$ itself,
but we can make sense of $e^{iS_{CS}}$.

We can in general evaluate this $\theta$ term in one dimension higher.

To make sense of this,
we pick a 4-manifold $W^4$ such that the boundary
$\omega W^3 = M^3$.
Furthermore,
we're going to extend the gauge field $a$
to be defined over $W^4$.
It turns out this is always possible.
For any 3-manifold,
you can always pick a 4-manifold whose boundary is that 3-manifold.
Conceptually,
the reason this can always be done is precisely because this cobordism group is
zero
\begin{align}
    \Omega_3 \left( BU(1) \right) = 0
\end{align}
That this is zero is that every 3-manifold with a U(1) gauge field is 
is the boundary of a 4-manifold with a U(1) gauge field.
The mathematicians tell us this fact.

\begin{question}
    What about other groups?
\end{question}
This doesn't necessarily hold for every group.

But suppose instead you were interested in SU(2).
Turns out this problem doesn't occur anymore,
because for SU(2) the first Chern class if trivial.
That we run into this problem is specific to the U(1) case.
It doesn't happen for SU(2),
but it doesn't happen for SO(3).
I'm not sure how general this statement is.
Everything that comes up in physics,
this cobordism group does always vanish.
I don't quite remember.
Let's just stick to U(1).
For SO(3) you might need to do something a little different.


Anyway, so then define
\begin{align}
    S_{CS} &=
    \frac{k}{4\pi}
    \int_{W^4} \partial_\mu a_\nu \partial_\lambda a_\sigma
    \epsilon^{\mu\nu\lambda\sigma}\\
    &=
    \frac{k}{16\pi}
    \int f_{\mu\nu} f_{\lambda\sigma} \epsilon^{\mu\nu\lambda\sigma}\\
    &=
    \frac{k}{4\pi} \int f\wedge f
\end{align}
This is completely well-defined because $f$ is a locally well-defined quantity.
The reason it gives the CS term on the boundary is because this term is a total
derivative,
and if you evaluate over the boundary it just gives you the CS term.

So this term is a total derivative and it gives a CS term on the boundary.
\begin{align}
    \partial_\mu a_\nu \partial_\lambda a_\sigma
    \epsilon^{\mu\nu\lambda\sigma}
    &=
    \partial_\mu\left( 
    a_\nu \partial_\lambda a_\sigma
    \right)
    \epsilon^{\mu\nu\lambda\sigma}
\end{align}

You might be worried,
because I'm it looks like I'm defining a (3+1)D theory,
but the result should better not depend which 4-manifold I pick,
and how I choose the gauge field in one-higher dimension.
To get an intrinsically $(2+1)D$ theory,
$S_{CS}$ should not depend on the extension $W^4$ and $a$.


In order it makes sure they're not the same,
compare 2 different extensions.

We want this difference to be $2\pI$ times an integer
\begin{align}
    \delta S &=
    \frac{k}{4\pi} \int_{W^4}f \wedge f
    - \frac{k}{4\pi} \int_{W^4'} f' \wedge f'
    \in 2\pi\mathbb{Z}
\end{align}
What I do is glue these two 4-manifolds together on the boundary which is $M^3$
and I end up with a closed 4-manifold.
\begin{align}
    X^4 &= W^4 \cap_{M^3} \overline{W_4'}
\end{align}
with
\begin{align}
    \tilde{A} &=
    \begin{cases}
        a & \text{on } W^4\\
        a' & \text{on } W^4'\\
    \end{cases}
\end{align}
Recall that
\begin{align}
    \frac{1}{4\pi^2}
    \int_{X^4} F\wedge F \in \mathbb{Z}
\end{align}
for a closed $X^4$.
And that's just integrating the entire field strength over this entire manifold.
So then
\begin{align}
    \delta S
    &=
    \frac{k}{4\pi}
    \int _{X^4} \tilde{f} \wedge \tilde{f}
\end{align}
so in order for this to be an extension,
I need $k\in 2\mathbb{Z}$ to be an even integer.

For this CS theory with $e^{iS}$ to make sense,
we need $k$ to be an even integer.

\begin{question}
    global gauge?
\end{question}
No the point is that on a 4-manifold I don't need to pick a global gauge,
because $f$ always makes sense.
On the boundary it's the same,
but it can be whether on the bulk.

Another way to see why $k$ has to be an even integer,
consider $M^3=T^3$ is a torus.
Assume
\begin{align}
    \int_{T^2} F_{12} = 2\pi
\end{align}
Then the action is
\begin{align}
    S_{CS} &=
    \int
    \frac{k}{4\pi}
    \left( 
    a_0 f_{12}
    +
    a_1 f_{21}
    +
    a_2 f_{01}
    \right)
\end{align}
and so let's do a large gauge transformation,
which takes
\begin{align}
    a_0 \to a_0 + 2\pi/L_0
\end{align}
where $L_0$ is the length in the time (0) direction.
So under this large gauge transformation,
there's going to be a change in this action
\begin{align}
    \delta S_{CS}
    &=
    \frac{k}{2}
    \int_{T^2}
    f_{12}
    =\pi k
\end{align}
and in order for this action to be independent of large gauge transformations,
we need $k$ to be an even integer.
Maybe that's a quicker way to see why $k$ needs to be an even integer

\begin{question}
    Why is it called a large gauge transformation?
\end{question}
Because this gauge transformation changes a flux through a hole by $2\pi$.
That kind of gauge transformation is not continuously connected to the identity.
Let's say
\begin{align}
    a \to a - \delta \lambda
\end{align}
what would $\lambda$ be?
We would need $\lambda = \frac{2\pi}{L_0}t$,
which is winding,
which is topologically non-trivial.

You might be a little bothered if you've done nay CS theory,
because now I'm telling you $k$ has to be even,
but we often work with $k$ being odd or 1 in integer quantum Hall states.

So here's the catch.
If you're working with fermions,
then $k$ is odd.

When working with fermions in your theory,
and you're doing QFT,
you want the manifold to have spin structure.

It's a fact that all 3-manifolds
admit a spin structure.
So we can define fermions on them,
but not all 4-manifolds admit a spin structure.
And the counter example would be $\mathbb{CP}^2$.
If you demand your theory has fermions,
then you only ant to consider extending to 4-manifolds
that admit spin structure.

So you require that $W^4$ is actually a spin manifold,
which is just a manifold that admits a spin structure.
So if $W^4$ is a spin manifold,
then $X^4$ is also going to be a spin manifold.

Extension is always possible,
and this time it's because
\begin{align}
    \Omega_3^{spin}\left( BU(1) \right) = 0
\end{align}
which we have to resort to mathematicians telling us this is the case.
Turns out if you have a spin 4-manifold,
then the second Chern class is actually even.
\begin{align}
    \frac{1}{4\pi^2}
    \int_{X^4}
    F\wegde F
    \in 2\mathbb{Z}
\end{align}
for closed spin 4-manifolds.
So what you find is
\begin{align}
    \delta S &= 2\pi k
\end{align}
which implies that
\begin{align}
    k \in \mathbb{Z}
\end{align}
and so $k$ can be odd.

If $k$ is odd,
we have a TQFT that has fermions,
that has to have spin structure,
and in the extension to 4D must also have spin structure.

So odd $k$ defines a \emph{spin} TQFT that describes a theory with fermions.

CM physicists don't worry about this issues,
because often when CM physicists work with CS theory,
they are only interested in the case when the manifold $M^3$ is $\mathbb{R}^3$
or just a ball $D^3$.
If we never consider topologically non-trivial manifolds,
then these issues don't really come up.
And we don't need to worry about the quantization of $k$.

On the other hand,
if the gauge group is $\mathbb{R}$ instead of $U(1)$,
then again we don't have to worry about the quantization of $k$.

So the quantization of $k$ only comes up when we use the fact that the gauge
group is $U(1)$,
which means for large gauge trnasofmation,
integrating magnetic field thorugh large lookps is quantized.

The quantization of $k$ only comes up when thinking of topologically non-trivial
situations.

\begin{question}
    What goes on with this argument?
    You can't even define on patcthes?
\end{question}
With fermisns,
you shouldn't even consider this 3D action,
you sould start with the 4D action,
and this issue won't come up.
You can consider the theory on $\mathbb{R}^3$ or a ball,
but you just can't do it for topologically non-trivial situations.
Maybe ther's a way of putting it together,
but I don't know.

So either you work in patches or $\mathbb{R}^3$,
or just consider the gauge group to be $\mathbb{R}$.
But for fermions,
it's even more necessary because of this issue.

Often,
we continue using this action when we care the gauge group is $U(1)$ and the
situation is topologically non-tvivial,
but usually that's when you have to hae enough expertise to know when and when
not it gives the right answers because of these subtlties.

\section{CS-Maxwell theory}
One more thing I want to just mention,
but I'll put the calculation in your homework.

Consider $\mathbb{R}^3$, but let's add the Maxwell action as well.
\begin{align}
    S_{CS-Maxwell}
    &=
    \int_{\mathbb{R}^3}
    \left( 
    \frac{k}{4\pi}
    a\, da
    -
    \frac{1}{4g^2}
    f_{\mu\nu} f^{\mu\nu}
    \right)
\end{align}
And we know Maxwell theory has gapless photons in Maxwell theory
if $k=0$.
But if $k\ne 0$,
then the photon acquires a mass,
and gets a gap.
It's away of giving the photon a mass gap,
very similar to the Higgs mechanism,
like in superconductors,
with the Meister effect which gaps out.


If we consider pure Chern-Simons theory,
just this $S_{CS} = \frac{k}{4\pi}\int a\, da$,
then the classical equation of motion tells us the field strength is actually
zero with $f_{\mu\nu} = 0$.
But since we're on $M^3 = \mathbb{R}^3$,
there's really nothing to talk about here.
We can write $a_\mu = \partial_\mu \phi$,
and the action is going to be zero because the field strength is zero.
If $S=0$, then the Hamiltonian $H=0$.
So pure classical CS theory on $\mathbb{R}^3$ has nothing in it.
Everything is completely trivial.

\section{Quantizing CS Theory}
The quantization of CS theory is non-trivial and subtle if you want to do it
completely.
We'll start off doing canonical quantization and try to understand the Hilbert
space of the system on some space.
Let me just give a preamble.

Formally,
one thing we can do is to define the path integral.
\begin{align}
    Z 
    \int \mathcal{D} a\,
    e^{i S_{CS}(a)}
\end{align}
but we need to do some gauge fixing,
as you know from QFT.

At the classical level,
we have this action that has no geometry dependence on the metric at all,
unlike maxwell.
So at a classical level,t
her theory is fully topological,
but it's not clear it will remain topological after quantizing.
You have to gauge fix to properly evaluate the path integral,
and you could lose it.
This is where the notion of framing comes in.
When gauge fixing,
you pick some metric to do gauge fixing,
then define a path integral independent of metric,
but it turns out you can't make it completely independent,
so her is some residual which is the framing.
I don't want to do all that right now.
I want to do something simpler,
just the canonical quantization of theory on $M^3$ being the form
torus cross time $M^3 = T^2 \times \mathbb{R}$.

Here things are easier.
I can always just pick a gauge where the time component $a_0=0$
I can't just forget about it entirely,
I need to remember the equation of motion for $a_0$ is non-trivial,
which hence places a constraint.
The equation of motion is
\begin{align}
    \frac{\delta S}{\delta a_0}
    \propto
    \epsilon^{ij}f_{ij} = 0
\end{align}
which becomes a constraint.
I find this explanation a bit opaque.
I prefer the following equivalent explanation.
Consider the Lagrangian
\begin{align}
    \mathcal{L}
    &=
    \frac{k}{4\pi}\left( 
    a_0 \epsilon^{0ij} \partial_i a_j
    +
    \epsilon^{i0j} a_i \dot{a}_j
    +
    \epsilon^{ij0} a_i \partial_j a_0
    \right)
\end{align}
and then I integrate by parts,
change around the order and relabel things,
terms come together
\begin{align}
    \mathcal{L} &=
    \frac{k}{2\pi}
    a_0 \epsilon^{ij} \partial_i a_j
    +
    \frac{k}{2\pi}
    a_2 \dot{a}_1
\end{align}
and note $a_0$ only enters on the first term,
and I can think of $a_0$ as a Lagrange multiplier
enforcing the constraint
\begin{align}
    \frac{k}{2\pi}\epsilon^{ij} \partial_i a_j = 0
\end{align}

\begin{question}
    By framing you mean framing of the gauge?
\end{question}
No, I haven't gotten into it,
but what I meant was fixing a gauge means choosing a metric.
Evaluate the path integral,
and you find the path integral depends on the metric.
But then you try to see if you could have done anything to save the topological
invariance.
You could get rid of the metric almost completely,
but not completely,
and it depends on a choice of framing,
which is a choice of trivialization of the tangent bundle.
You have to choose 3 vectors everywhere that don't degenerate,
which is picking a trivialization of the tangent bundle.
I'm not presenting that, so don't worry about it.


So the gauge field needs to have no magnetic field in space.
We can parametrize the flat gauge field configuration on $T^2$ with
\begin{align}
    a_i(x_1, x_2, t) &=
    \frac{2\pi}{L_i} X_i(t)
\end{align}
The gauge field should have no magnetic field through the torus,
and there will be holonomic which can be parametrized by writing in this form.
These $X_i$'s are called zero modes.
This $X_i$ is like the spatial average of $a_i$.
This is proportional to an integral over space,
and overall average if you like
\begin{align}
    X_i(t) \propto \int d^2x\, a_i
\end{align}
And so $X_i$ is related to $X_i + 1$ by large gauge transformations
with $X_i \sim X_i + 1$.
So $X_1$ and $X_2$ live on a torus.
If I substitute this form back into the Lagrangian,
remembering that $a_0$ is zero,
so only the second term remains,
and when I substitute this 
\begin{align}
    L &=
    \int_{T^2} d^2x \, \mathcal{L}\\
    &=
    2\pi X_2 \dot{X}_1
\end{align}
Doing some classical mechanics,
the canonical momentum is
\begin{align}
    P_{X_1} &=
    \frac{dL}{d \dot{X}_1} = 2\pi k X_2
\end{align}
so the Hamiltonian is
\begin{align}
    H &=
    P_{X_1}\dot{X}_1 - L = 0
\end{align}
which is strange.
But you find that your phase space itself is compact with some finite volume.
And the volume of this phase space is $2\pi k$.
This is a very unusual situation.
Almost always in classical mechanics,
your phase space is infinite volume.
Even if your coordinate is bounded,
the momentum is unbounded.
So this is a weird situation where the phase space itself has finite volume.
If you remmeber QM,
one way of thinknig about quantum sates,
for every phase space volume,
you assing one quantum state.
Here $\hbar=1$,
we get one quantum state for each $2\pi$ volume of phase space.
So we conclude the dimension of this Hilbert space on a torus has to be $k$.
\begin{align}
    \dmi \mathcal{H}\left( T^2 \right) &= k.
\end{align}

So what we learn is that
\begin{align}
    |Z(T^2 \times S^1)|
    =
    \dim \mathcal{H}(T^2) = |k|
\end{align}

\begin{question}
    Does that have something to do with the path integarl not being well defined
    yo umentinoed in the beginningof the course?
\end{question}
No.
Think aoubt this path integral,
it's integrating over all possible gauge fields.
this is perfectly well defined with phase space compact.
Usually in QFt,
you need to make sense of this paht integral hwihhci is infintie,
so you can only define correlation fucntions which are ratios of path integrals.
If you do this precisely,
you can make this path integral well-deifined and convert it to actual numbers.
I haven't told you how to do that yet.
The phase space compact is just a weridness of Chern-Simons theory.

You can repeat the above calculation on any genus $g$ surface.

Now you can think about all these fluxes through all the cycles.
Insteadof just $X_1$ and $X_2$ which parametrized 2 of hte cycles,
we're going to get $g$ copies of that kind of situation that we found,
and we're going to find that the dimension of the Hilbert space on a genus $g$
surafce ends up being
\begin{align}
    \dim \mathcal{H}\left( \Sigma_g \right) = 
    |k|^g
\end{align}
which I'll ask you to prove in your homeowrk.

There's one more generalization I'll put in your homeowrk.
You could instead consider multiple $U(1)$ gaue fields,
so you consdier $U(1)^n$ CS theory,
in which case the Lagrangian is
\begin{align}
    \mathcal{L}
    &=
    \frac{1}{4\pi} K_{IJ}
    a_\mu^{I} \partial_\nu a_\lambda^J \epsilon^{\mu\nu\lambda}
\end{align}
where $K$ is an integer symmetric matrix.
And then
\begin{align}
    \dim\mathcal{H}\left( \Sigma_g \right) = 
    |\det K|^g
\end{align}

Looking back.
So the Hailtonian is zero and we have a finite number of states in total.
There are very few degrees of freedom at all.
We could have instead considred the CS-Maxewll theory.
if you do CS-Maxwel theory,
the Hamiltonian is not zero,
and phase space will be non-compact.
And so what you'll find there is that you'll have $k^g$
ground states degeneracy,
and you'll have phase space that is noncompact,
and the photon becomes massive,
soo you havea finite gap,
but you hvae all these topologically degenerate groud states,
which are zero modes of the gauge field.
Physcally, this $a\, da$ term is only the leading term.
In the cnotex of the faracitoal quantum Hall effect,
there are gapped non-trivail excitations,
like the magneto-roton,
and that's what these excitations describe.
They're all gapped,
becaseu it's a gaped liquid,
they can e modelled by these higher order terms in $a$.
\begin{align}
    \mathcal{L} &=
    -\frac{k}{4\pi} a\, da
    +
    \frac{1}{4g^2} f^2
    + \cdots
\end{align}

When we use CS theory to describe fracitonal quantum Hall effect,
the meaning of these is the density of particles.
\begin{align}
    j^{\mu}
    &=
    \frac{1}{2\pi} \epsilon^{\mu\nu\lambda}
    \partial_\nu a_\lambda
\end{align}
So these are just hydrodynamic modes in a gapped liquid.
In Maxwell-Chern-Simons,
I don't remember the formula,
but it kas to depend on $k$.
If you take $g$ to infinitey,
all these states go away and you'rel eft with pure CS theory.
You'll have to do calcuations yourself to get the gap.

\begin{question}
In thep icutero f illing Landau levels,
what do these $k$ states correspond to?
\end{question}
You're getting two things confused
becauseI ddn't empphasises hte physcal conneiton so much.
We have 2 physical situations where CS thoery turns up.
\begin{enumerate}
 \item background $U(1)$ gauge field $A$.
     A fixed static background.
     And the effectie action for this gauge field for the $\nu=k$ integer
     quantum Hall effect,
     or if you like Chern number $k$,
     \begin{align}
         S_{eff}[A] &=
         \frac{k}{4\pi}
         \int A\, dA
     \end{align}
     The TQFT is invertible,
     with some symmetry $G$ for the gauge field.
 \item Dynamical U(1) guage field (emergent),
     the action is the same,
     but this is no longer fixed and we have to integrate over it
     \begin{align}
         S_{eff}[a] =
         \frac{k}{4\pi} \int a\, da
     \end{align}
     and the path integral is
     \begin{align}
         Z\left( M^3 \right)
         =
         \int \mathcal{D} a\,
         e^{iS[a]}
     \end{align}
     and now it's a non-inveritble TQFt.
     Here we gauge the theory,
     which si just a redundancy in the description,
     and we get a non-invertible theory.
     You could couple this to the backgorund $U(1)$ iwwth
     \begin{align}
         S_{eff}[a] =
         \frac{k}{4\pi} \int a\, da
         +
         \frac{1}{2\pi} A\, da
     \end{align}
     And you have
     \begin{align}
         Z\left( M^3 \right)
         =
         \int \mathcal{D} a\,
         e^{iS[a,A]}
     \end{align}
\end{enumerate}

\begin{question}
    Currents?
\end{question}
You're confusing 2 currents.
This $j^\mu$ is the current of flux.
If you have flux of the gauge field that moves aorund,
this is what it is.
This is hte instanton current.
There are 2 currents.
There si the current of particles charged under $a$ that is conserved wiht
$\partial_\mu j^\mu=0$,
and then there is the monopole instanton conserved current,
which is what I'm talking about.

\section{Wilson loops}
Okay,
so we've quantized the theory at least on this manifold
and we've understood the states.
But now it's useful to talk about Wilson loops.

For any curve $\gamma$,
we can write down a Wilson loop
\begin{align}
    W_\gamma &=
    e^{i\oint_\gamma a\cdot dl}
\end{align}
and let's consider $M63 = T^2\times \mathbb{R}$.
Consider the Wilson loops on the directions on the torus.
\begin{align}
    W_1 &=
    e^{i\oint_{\gamma_1} a\cdot dl} = e^{i 2\pi X_1}\\
    W_2 &=
    e^{i\oint_{\gamma_2} a\cdot dl} = e^{i 2\pi X_2}
\end{align}
And remember when we quantize,
we have
\begin{align}
    \left[ x_1, P_{x_1} \right] = i
\end{align}
we learn that
\begin{align}
    \left[ X_1, X_3 \right] = \frac{i}{2\pi k}
\end{align}
Here capital $X$ is the zero mode and $x$ is the coordinate.
This means the Wilson loops in the two directions acquire a non-trivial value.
\begin{align}
    W_1 W_2 &=
    W_2 W_1 e^{-4\pi^2\left[ X_1, X_2 \right]}\\
    &= W_2 W_1 e^{2\pi i/k}
\end{align}
Since $H=0$, $[H,W_1]=0$.
And so the ground state of the system needs to form a representation of this
algebra,
and every representation of this algebra needs to have an integer multiple of
$k$ and so the minimal representation is $k$-dimensional.
So we must have at least $k$ dimensional Hilbert space.

And we can pick a basis for these Wilson loops to act.
And you can check that this action with the states is compatible with the
algebra.
\begin{align}
    W_2\ket{n} &=
    \ket{n + 1 \mod k}\\
    W_1 \ket{n} &=
    e^{2\pi i/k}\ket{n}
\end{align}
What this means is that these particles have fractional statistics.
You can think of these Wilson loops describing particles of charge 1 moving
around the loop $\gamma$.
If I create a particle and its antiparticle,
moving them around a loop and annihilate.
That they don't commute with each other in different loops implies that these
particles carry fractional statistics.

\begin{question}
    Does this group have a name?
\end{question}
Yes. I think this group is called the Heisenberg group.

Please do your homework.
\section{U(1) Chern-Simons theory}
Last there were $|k|^g$ states on genus $g$ surface.
We also figured out we have these loop operators
\begin{align}
    W_\gamma
    &=
    e^{i\oint_\gamma a\cdot dl}
\end{align}
and if you're on a torus,
we have loops going in two directions,
which I just called 1 and 2.
And you find that
\begin{align}
    W_1 W_2 &=
    W_2 W_1 e^{-2\pi i/k}
\end{align}
What this means is that you can consider a basis of states where one of them is
diagonal
\begin{align}
    W_2 \ket{n}
    &=
    e^{2\pi in/k} \ket{n}
\end{align}
and the other one acts like a clock operator that shifts $n$ by 1.
\begin{align}
    W_1\ket{n}
    &=
    \ket{n + 1\mid k}
\end{align}
That there are $k$ ground states,
is related to the fact that around a given loop there are $k$ distinct
operators.

So that $k$ states on a torus is directly related to the fact that there are $k$
distinct Wilson loop operators along a given curve $\gamma$.
When I write a loop operator for a curve $\gamma$,
I can consider a charge $\ell$ particle going around a loop.
\begin{align}
    W_\gamma^{\ell}
    &=
    e^{i\ell \oint_\gamma a\cdot dl}
\end{align}
and furthermore this acts like the identity on the Hilbert space.
\begin{align}
    W_\gamma^k &= \mathbf{1}
\end{align}
For example, if I consider $W_1^k$,
it's going to make the phase go away.
Hence the distinct values of $\ell$ are $\ell=0,\ldots, k-1$.

From the condensed matter perspective,
there are $k$ topologically distinct types of particles.
And these $k$ distinct types of particles are what we call anyons in the system.

So far because we're working entirely in CS theory,
I haven't defined what topologically distinct,
but in CS theory,
it just means there are $k$ Wilson loop operators that act non-trivially on your
subspace.

\begin{question}
    This is difeernt from $k$ speciies of particles.
    This seems strange?
\end{question}
The way to thik about this is imagine some physical system described by CS
theory at long wave lengths.
There are $k$ distinct ground staes.
Thew ay to go from one ground state to anohter,
you create a quasiparticle, and a quasihole,
take one around the torus,
and annihliiate them.
If it is trivial,
such an operation has no effect no the ground state.
The corresponding Wilson loop should act in the ground state subspace.


\begin{question}
    There are $k$ ground states?
\end{question}
In the CS hteory,
there are only $k$ states.
In CS-Maxwell,
we have $k$ ground states and a gap to other states.
There's a ground state sector described by CS and then excitations above a gap.
there are infinitely many excitations that take you from the ground state to
anexcited state,
but they can be classified by topological operators
to make equivalence classes.
The topological operators are Wilson loops.

\begin{question}
Should we say there are $k$ different particeles?
\end{question}
I would say htere are $k$ ddistinct particles.
each are $k$ inequivalent types of paritcles,
But there is a single generator,
and elementary particle.

\begin{question}
    Discrete translation for mangetic field.
    Is htere an analogy for magnetic flux?
\end{question}
If you start with CS-Maxwell theory,
your Lagrangian would be
\begin{align}
    \mathcal{L}
    &=
    \frac{k}{4\pi}a\, da
    -
    \frac{1}{4g^2}f^2
\end{align}
If you look at the ground state sector,
you can define
\begin{align}
    X_i \propto \int d^2x \, a_i
\end{align}
but then you fnid the Lagrangian winds up looking like
\begin{align}
    L &=
    X_1 \dot{X}_2
    -
    \frac{m^2}{2}\left( 
    \dot{x}_1^2
    + \dot{x}_2^2
    \right)
\end{align}
which is literally a particle in a magnetic field.
You can think of the particle as a zero mode of the gauge field.
that makes the more specific connection.

When you go to fractional quantum Hall,
the translatioanl symmetry in FQHE has the same effect.
Translation symmetry acts the same way as Wilson loops act.
There's a deep connection there.


\begin{question}
    What was the motivaiton for WL?
\end{question}
When you have a gauge theory,
the natural thing to talk a bout his lop operaotrs.
One way to think is
suppose you hvae some action $S(a)$ for your gauge theoyr,
you could always add a term for some exeternal source current
\begin{align}
    S(a) + a_\mu j^\mu
\end{align}
So if I take my path integral,
I could always add a term here for some source current $j^\mu$,
\begin{align}
    \int \mathcal{D}a\,
    e^{i S(a)}
    \left( 
    e^{a_\mu j^\mu}
    \right)
\end{align}
some external source yo put in yoru system.
If this $j$ describes a paritcle travelling around,
this is literally the Wilson loop.
You could write the charge
\begin{align}
    j^0 &= \delta(\vec{x} - \vec{x}_*(t))
\end{align}
and then the current would be
\begin{align}
    j^i &= \frac{dx_*'}{dt} \delta^{(3)}(\vec{x} - \vec{x}_*(t))
\end{align}
So it describes a particle travelling in a loop.
And then if you write the integral,
that just becomes a loop integral on a curvve.

So the loop operator coresponds ot inserting a source in the current.

\begin{question}
    With this source current,
    that allows you todistinguish these staes?
\end{question}
Every time you hvae agauge thoery,
Wilson loops are a natural thing.
What we find ni CS theory,
we have a degenerate set of states.
If we're on a torus,
taking a paritcle around a loop 
either shifts the state or measures which state we're in.

Suppose you fix your state to be an eigenvalue,
you can insert your loop in hte other direction adn shuffle your states.
There's a deep connection betwen the number of distinct typeso f apritcles you
have and the number of ground states on your torus.
The number of ground states on your torus is equal to the number of
topologically distinct types of particles.

I havne't given you a general proof,
but we've seen it play out.

\begin{question}
    So each of these loops,
    were you able to?
\end{question}
If I loop athte Wilson loops,
I have $W_\gamma^{\ell} = e^{i\ell \oint_\gamma a\cdot dl}$
I have charge $\ell$.
And if I put $\ell=k$,
I get the trivial operation.

Now the fact that those Wlison loops have this non-trivial algebra,
you can infer that hte associated particls that go around these trajectories
have fracitonal statsitcs.

Imagine that my gournd stae is a on a torus.

SUppose I apply a Wilson loop operator for this guy,
then another one.

Suppose I do it in an oppsoite order,
and we know the phase idfferenece is $e^{2\pi i/k}$.
So if one particle goes aroudn another loop,
I get a phse $e^{2\pi i/k}$ relative to unlinked.

We can see this more precisely in the following way by looking at the equation
of motion.
From the equation of motion,
we have
\begin{align}
    j^0 &=
    \frac{\delta S}{\delta a_0}
    \\
    &=
    \frac{k}{2\pi}
    \underbrace{\left( \partial_1 a_2 - \partial_2 a_1 \right)}_{b}
\end{align}
which is just hte magnetic field inthe parentheses
\begin{align}
    b &= \frac{2\pi}{k}j^0
\end{align}
This tells us a charge 1 particle carries flux $2\pi/k$.
Because magnetic field is bound ot particle density iwth coefficeitn $2\pi/k$.
If you introduce a paritcle of charge $1$,
the CS term is that it dresses the particle with flux.
A particle with charge 1 is dressed wiht flux $2\pi /k$.
The CS theory binds flux to the charge.
That's the ssense of CS theory.
Now e see fracitonal statitcs.

SUpoose I have another paritcle of chagge 1 and also $2\pi/k$ flus.
I have charge going around flux,
so I have Aaronov-Bohm phase.
The phase for braiding this round another is going to be
\begin{align}
    e^{i2\pi/k}
\end{align}
Naively you should expect $e^{2\cdot 2\pi i/k}$,
but it turns out it's half of what you naively expect if you go through the
subtle calculations of CS theory.
Let me sketch how to do that calculation.
Let's go through the CS calculation of the fractional statistics.

Consider the manifold $M^3 = S^3$.
Consider the expectation of product of Wilson loop operators
\begin{align}
    \left\langle
    W_{\gamma_1}^{\ell_1}
    W_{\gamma_2}^{\ell_2}
    \cdots
    W_{\gamma_n}^{\ell_n}
    \right\rangle
    &=
    \int \mathcal{D} a\,
    e^{i S_{CS}(a)}
    e^{i
    \sum_\alpha \ell_\alpha \oint_{\gamma_\alpha} a\cdot dl
    }
\end{align}
Suppose I have some complicated knot that is $\gamma_1$,
and $\gamma_2$ is knotted and linked with it in some complicatd way.
Just as I mentioned Wilson loops corresponding to source current,
whatwe have is that the 
$ e^{i \sum_\alpha \ell_\alpha \oint_{\gamma_\alpha} a\cdot dl }$
is like adding an external source current.

So we can write our Lagrangian as
\begin{align}
    \mathcal{L} &=
    \frac{k}{4\pi} a_\mu \partial_\nu a_\lambda \epsilon^{\mu\nu\lambda}
    +
    a_\mu j^\mu
\end{align}
where the current is
\begin{align}
    j^\mu &=
    \sum_{\alpha} \ell_\alpha j_{\alpha}^{\mu}
\end{align}
and then the charge density part is just a particle
\begin{align}
    j_{\alpha}^{\mu}
    &=
    \delta^{(3)}\left( 
    \vec{x} - \vec{x}_{\alpha}(t)
    \right)
\end{align}
and the current
\begin{align}
    j_\alpha^i
    &=
    \frac{dx_\alpha^i}{dt}
    \delta^{(3)}\left( 
    \vec{x} - \vec{x}_{\alpha}(t)
    \right)
\end{align}

In order to understand how braiding gives fractional phases,
we need to calculatine this espcation vuae.
Therea re two ways to think about how you'd calcaulte the epectaiotn vlaue.
One is that this is a Gaussian integral,
the actoin is quadratic in hte fh field.
You could literally integrate out.

A slighly different way is to solve hte classical equation of motino
and plug that back in.
It works because this is a Gaussian integral.

To calculate the expectaiton vlaue,
solve hte quation of motion
\begin{align}
    \frac{\partial S}{\delta a_\mu}
\end{align}
which implies
\begin{align}
    \frac{k}{2\pi} e^{\mu\nu\lambda} \partial_\nu a_\lambda = -j^\mu
\end{align}
Imagine we solve for the classical solution $a^{cl}$ and plug that back into the
action.
Then substitutling it into the action,
\begin{align}
    S\left( a^{cl} \right)
    &=
    \int \left( 
    \frac{k}{4\pi} a^{cl} da^{cl} 
    + a^{cl}\cdot j
    \right)\\
    &=
    \int\left( 
    -\frac{1}{2} a^{cl} j
    + a^{cl}\cdot j
    \right)\\
    &=
    \int \frac{1}{2} a^{cl}\cdot j
\end{align}
where I'm using a shorthand to avoid writing the $\epsilon$.
And so then
\begin{align}
    \left\langle
    \prod_{\alpha} W_{\gamma_\alpha}^{\ell_\alpha}
    \right\rangle
    &=
    \mathcal{N}
    e^{-i \frac{k}{4\pi} \int a^{cl} \, da^{cl}}\\
    &=
    \mathcal{N}
    e^{i\frac{1}{2}\int a^{cl}\cdot j}
\end{align}

\begin{question}
    Does the Wilson loops not commuting matter in the ordering?
\end{question}
We can just define the expecation value using this.
The fact we're on $S^3$ also simplifies the situation,
as the only thing to talk about is hte linkingof the loops.


\begin{question}
    There should be some paht ordering in order to define the linking?
\end{question}
Usually a Wilson loop is defined like this,
but it only matters if it's a non-abelian gague field.
\begin{align}
    W &= \mathcal{P} e^{i\oint a\cdot dl}
\end{align}
Path ordering only shows up for non-Abelian things.
Here we're working on $S^3$,
so there are no non-contractible cycles where things don't commute.
T could have just written
\begin{align}
    \left\langle
    W_{\gamma_1}^{\ell_1}
    W_{\gamma_2}^{\ell_2}
    \cdots
    W_{\gamma_n}^{\ell_n}
    \right\rangle
    &=
    \int \mathcal{D} a\,
    e^{i S_{CS}(a)}
    \prod_{\alpha}
    e^{i
    \ell_\alpha \oint_{\gamma_\alpha} a\cdot dl
    }
\end{align}
and these are just phases.

\begin{question}
    How to do you classify a systme of knotted loops.
    Does that mean I just consider the knots and evaluate thep ahte integral?
\end{question}
That was Witten's point.
Even before Witten,
people know you get linking numbers of loops when you calculate these ntegarls.
Witten did this for non-Abelian CS theory,
and show the corresponding expectation value
evaluates the Jones polynomial of the knot,
which si asophisticated knot invariatn mathematicains had invented earlier.
there was no intrinsic 3D version.
He won a Fields medal.

\begin{question}
    If w'ere on a sphere,
    toologically there's no difference betwen a path that takes you around the
    north pole and a path that doesn't take you aroud the north pole.
\end{question}
You're getting confusing it with $S^2$.
We're in $S^3$.
You're right htere is a configuation in $S^2$,
but this configuration cannot happen
because a chage 1 particle has flux $2\pi /k$.
So htere has to be $k$ of them.

\begin{question}
    Does $k$ has to be even?
\end{question}
No I don't think $k$ being even is needed.

\begin{question}
    Is there a physical reason tht he charge around the fluxes is not twice?
\end{question}
No,
I don't have a good reason for that excpt that's what CS theory gives.
If you have some lattice model,
like $\mathbb{Z}^n$ toric code state,
you would get stuff consitstent with CS theory.
Our intuitition comes for Maxwell,
and this is CS theory,
whici s difernet.

\begin{question}
    For 2D,
    we sometimes impose $\pI$ flux on particles,
    which makes paritcles go from boson to fermions,
    but in this case hte full braidig braiding is trivial.
    If I take $k=2$,
    the full braiding is $e^{i\pi}$.
\end{question}
But that describes semions,
not fermions.

I might put this in your homework and guide you through it,
but the result is that you get hte Gauss linking number,
meaning htat this expectaiton vlaue of curves is equal
up to to smoe overall normalization
\begin{align}
    \left\langle
    \prod_{\alpha} W_{\gamma_\alpha}^{\ell_\alpha}
    \right\rangle
    &=
    \mathcal{N}
    e^{\frac{2\pi i}{2k}
    \sum_{\alpha,\beta}
    \ell_\alpha \ell_\beta
    L\left( \gamma_\alpha, \gamma_\beta \right)
    }
\end{align}
where the Gauss linking number is defined by
\begin{align}
    L\left( \gamma_\alpha, \gamma_{\beta} \right)
    &:=
    \frac{1}{4\pi}
    \int_{\gamma_\alpha} dx^{\mu}
    \int_{\gamma_\beta} dy^{\nu}
    \epsilon^{\mu\nu\lambda}
    \frac{x^\lambda - y^\lambda}{|x - y|^3}
\end{align}
and this means unlinking loops gives you a $e^{2\pi i/k}$ phase,
which si what we mean by fractional statistics.

But there's a problem,
when $\alpha=\beta$,
this is not well-defined,
because $x=y$ and it blows up,
so it doesn't make sens for the diagonal piece,
the self-linking.
CS theory gave us something we can't make sense of.
When ever you find infinities in QFT,
you try to get a well-defined ansewr by regularizing hte theory in some way.

In this case,
there is self-linking $L(\gamma_\alpha, \gamma_\beta)$ is ill-defined.
So to regularize,
introduce framing of the loop.
Framing means you consider it as a ribbon instead of just a string.
You have $\gamma$ and then you have $\gamma'$ that is infintetsminally pushed
off from the original $\gamma$.
You have a world-line as a ribbon,
then the self-linking is the linking between $\gamma$ and $\gamma'$.
In general,
it's completely arbitrary how I define this framing.
I could have chosen a different framing.

I could have made the ribbon twist around,
and I would get a differnt number.
So the ansewr is ambigiuos until I pick a framing,
and we know how the resutl changes when we change the framing.
So it is well-defined how it changes under changes of framing.
The point is that the flat ribbon is related ot the one-twist ribbon
by a fator of $e^{\frac{2\pi i}{2k}}$.

It's only well-defined if we pick a framing,
which is arbitrary,
but we can clacualte how te expecation vlaues chnage under framing changes.

\begin{question}
    Is it depednent on the overall framing on the manifold?
\end{question}
No, it's independent of hte overall framing.

Why does it have to be a ribbon?
If I had a particle,
it doesn't makesense to talk about it spinning aorund ont itself..
But if it's a ribbon,
you have an arrow spniing around.
You have charge and flux
in CS thery,
so your particle has a little arrow,
and you can then talk about how the arrow twists around by $2\pi$.
We talked about topological spin.

We can write
\begin{align}
    e^{\frac{2\pi i}{2k}}
    = e^{2\pi i h_1}
\end{align}
so we can define
$h_1 = \frac{1}{2k}$ as topological spin,
or more generally
\begin{align}
    h_\ell
    =
    \frac{\ell^2}{2k}
\end{align}

The idea is to think of a ribbon instead of a loop.
If you had a ribbon,
but the ribbon is in a loop,
then you can't actually smoothly deforma framing.
You owuld have to cut it,
twist it,
and glue it back.
It's nota global property of the loop.

\begin{question}
    How do we change the linking number as we change the framing?
\end{question}
The point is the framing was wa choice.
Imagine two lops with different fraings.
Then what is the hange in the path ingerals if I consider the two lops in
didfetn frames.
It's just $e^{\frac{2\pi i}{2k}}$ phase difference.

\begin{question}
    The difference comes even looknig at self-linking.
\end{question}
yes,
otherwise the off-diagonal terms are linking diferent loops,
and the selflonking wouldn't matter at all.

\begin{question}
    This almost seems like picking up an arbitrayr numbe fo reach diagonal term,
    so we just pick a finite linking nubmer?
\end{question}
Yes,
but the point is we have rules for how that number changes under hcange of
framing.
Wihtout hte rule,
this would be meaningless.
that this does havea rule means we can make sense.

\begin{question}
    What did you mean by topological spin?
\end{question}
I just meant whatehr comes in hte exponent.
The change ni phase under change in framing si topological spin.
It's $\ell^2$ because there is the $\ell_\alpha \ell_\beta$.
I implicitly assumed it was a charge 1 wilson loop,
but if it was carge $\ell$ it's $\ell^2$.

\begin{question}
    Whatis the physcial relvance of framing?
    Why not just set it all to zero and not care?
\end{question}
Suppose I can just ermove al hte idagonal terms.
The problem is that htins wouldn't relaly make sense.
The point of this thing is that these phases are important.
They contain important physics.
One way of undersatndin the physics is that these spins show up in a log of
idffertn eways.
One way they show up for example,
is suppose I hae torus and I take awilson lop of a particle thoruhg here.
I can do certain mapping class group operations,
such as Dane twists,
which you can think of cutting hte manifold alnog a cycle,
twisting it by $2\pi$,
and clugin it back together.
Thati s the same a changing the framing by 1.
thew ay the ground state changes under mapping clas group actions
is these charges.
The naive thing of forgettigng about them will fail to reproduce a lot of
physics in systems that do show up.
The point is there's something smarter you can do than just setting htem to
zero.
The point is ther is a way to regularie them.

\begin{question}
    What is hte physical meaning of Wilson loops in physical space?
\end{question}
You mean these $h_\ell$'s?
These $h$s were from changing the framing of Wilson loops.
they were not well defined,
but we need to choose a framing,
and there are many,
but they are related to geter by the phase.

\begin{question}
    This framing and expectation values of Wilson loops?
\end{question}
Suppose I have a state where I have some particles at various places.
Then I have that evole to another state.
then they go around in some braid.
The fact htese particle have factoinal statistics means
that if I have my initial state and pply a stransmtation $U$,
the point is I get some phases,
which gives me my final state.
\begin{align}
    U\ket{\psi_{in}}
    &=
    e^{i \cdots}
    \ket{\psi_{in}}
    =
    \ket{\psi_{final}}
\end{align}
These phaes are essentially given by these kinds of expecatation values.c
In this case,
consider $M^3=\mathbb{R}^2\times I$.
That causes all these particles to braid around each other.
If I take the inner product,
that essentially becomes an execation vlaueof Wilson loops.
\begin{align}
    \bra{\psi_{in}}U\ket{\psi_{in}}
\end{align}
If I thiknk of time as periodic,
this guy would be the expeation of these loops knotted in some way.
Physicaly it corespnods to the evoution of quazparticles moving aorund in space
and time and what phases you get.

\begin{question}
    If you're chosing some framing to calcualte expecation value,
    where is the comparison between these framings that inttoduces these phases.
\end{question}
When you write down this $W$,
I need to implicitly specifiy the $\gamma_\alpha'$ the shifted version.
For every $\gamma$ you need to tell me what he shiftd version is like,
and then we know the value of the expectaion value.
We would also know how differnt shiftings would relate to the orignial $\gamma$.

\begin{question}
    What's the situation you would comapre thse to?
\end{question}
You might be interested ina situation where you're intested in some evoultion
wherethe patricles evolv like this,
but you want o compare to a situatio where the paritcles eolve like this
(pictures with braidings).
In a sense you're asking what hte application is.
I feel like we'll see it when we get there.
One applicatoin isin that Dane twist picture.
If I have a paicle sittin on the plane,
I can imagine that the particle twists as it evovles in time,
relative to not twisting in time,
and you want to know hwath e phase idffernece is.

\begin{question}
    Isn't htat depending in hte framing you choose,
    how do you standadaris it?
\end{question}
In whatI was tlaking about,
imagine some evolution of your system that corespond to this (not doing
anything).
Then theren's another eovlution where you twist the space around.
Ify ou compare it to that,
if your system was evolging,
you would get smoe transition amplitud e from intial tofinal.
Insteado f donig that,youhave your system evolving,
rotate space by $2\pi$. and then come back to where you started.
Comapring htse two situatinos,
I get exaclty the same thing up to some phase,
the topological spinn.

In QFt we have the spin-statistics theorem.
It turns out
if I have some paticle that spins aorund on itself,
relative to not hvaing spun around itself,
we get this topological spin $e^{2\pi ih}$.
Another way to think about htis picture is what happend?
A particle was going here,
particle and antiparitcle got eated ot of vacuum,
then they sexhcnaged.
So three is  deep relation with exchange statsitics.
Thsi $e^{2\pi i h}$ is also the exchanges tatistcs
becasue of the spin statistics theorem.
This topological spin just tells us what hte exchange statistics is.
If I have two fermiosn,
spinnig it aorund should give $-1$,
which is the same as exchanging them.

Here I have a phase of $e^{i2\pi/k}$
but if I exchange them I also get $e^{i2\pi /k}$.

\section{Global U(1) symmetry}
This CS thoery has global U(1) symetry,
which si not to be confused with the gauge symetry that defines the teory.
We have a U(1) gauge fidl $a$ which is dynamical.
We can introduce particles that are charged under $a$,
which describes a conserved current $j$.
Then we can write down
\begin{align}
    \mathcal{L}
    &=
    \mathcal{L}_{CS}
    + a_\mu j^\mu
\end{align}
For example,
you could couple it to a scalar field,
\begin{align}
    \mathcal{L}
    &=
    \frac{k}{4\pi} a\, da
    +
    |\left( \partial + ia \right)\phi|^2
\end{align}
and this term would contain a $j\cdot a$ term,
where $j$ is the curretn of thse $\phi$ particles.

But there's another conserved cucrent beyond any kind of particle you cna couple
with $a$.
You have a conserved current
\begin{align}
    J^\mu &=
    \frac{1}{2\pi}
    \epsilon^{\mu\nu\lambda}
    \partial_\nu a_\lambda
\end{align}
with $\partial_\mu J^\mu = 0$.
We can write out the components
\begin{align}
    J^0 &= \frac{1}{2\pi}
\end{align}
the magnetic field and the eltric cfiel
\begin{align}
    J^1 &\propto \frac{1}{2\pi}e^2\\
    J^2 &\propto \frac{1}{2\pi} e^1
\end{align}
Flux of $a$ describes  conserved current.
$J^\mu$ is a conserved cucfrent for flux $a$.

This isa ddep thing for $(2+1)D$ in genreal.
Any time you have some theory iwhta glboal U(1) symmetry,,,
you can write down the symmetry for hatt ocnsierved *(1)
and yu can ``dualize'' it and get a conserved U(1) current,
and that is something deep I can discuss more if we have time.

So this conserved current for flux,
you can think of as dual to current of charges of $a$.

\begin{question}
    Is this statment true for arbitrary 2D manifolds or just $\mathbb{R}^2$.
\end{question}
I think it should be true in teeral.

We can couple a conserved U(1) current to a background U(1) gauge field.
You can couple the global U(1) to a background U(1) gauge field $A$.
So you can write a Lagrangian
\begin{align}
    \mathcal{L} &=
    \frac{k}{4\pi}\epsilon^{\mu\nu\lambda} a_\mu \partial_\nu a_\lambda
    +
    \underbrace{
    \frac{q}{2\pi}\epsilon^{\mu\nu\lambda} A_\mu \partial_\nu a_\lambda
    }_{
    a\cdot A_\mu J^\mu
    }
\end{align}
and here $a$ would be a dynmical U(1)
but $A$ is a fixed static $U(1)$ field.
And $q$ is a new parameter that tells us how to couple this with the global U(1)
symmetry.


Once consequence of this is that when we go back to the non-trivla particls of
the original theory,
those trivial particels which were associated with $2\pi/k$ flux of $a$,
but now we see those particles also carry a fractional charge under the
background $A$.
we can write down the current.
You can stare at it and find
\begin{align}
    J^0 &=
    \frac{\delta \mathcal{L}}{\delta A_0}
    =
    \frac{q}{2\pi}b
\end{align}
which means that $2\pi/k$ flux of $a$ has charge 
\begin{align}
    Q_{2\pi/k} = q/k
\end{align}
So putting it togethe.
If I have a particle that has cahrge $\ell$ under $a$,
then htere will be a flux $2\pi \ell / k$
of $a$,
which gives a charge of
\begin{align}
    Q &=
    \frac{q\ell }{k}
\end{align}
so these have fractional statistics but also fractional charge.

\section{Luttinger Liquid}
Last time we coupled a U(1) gauge theory with a background gauge field
\begin{align}
    \mathcal{L} &=
    -\frac{k}{2\pi}
    a_\mu \partial_\nu a_\lambda \epsilon^{\mu\nu\lambda}
    +
    \underbrace{
    \frac{8}{2\pi} A_\mu \partial_\nu a_\lambda \epsilon^{\mu\nu\lambda}
    }_{
    q A_\mu J^\mu
    }
\end{align}
There is a global U(1) current
\begin{align}
    J^\mu &=
    \frac{1}{2\pi}
    \epsilon^{\mu\nu\lambda}
    \partial_\nu a_\lambda
\end{align}
This means $2\pi/k$ flux of $a$ carries charge
\begin{align}
    Q_{2\pi/k} = \frac{q}{k}
\end{align}
Charge $\ell$ under $a$ implies that
$2\pi\ell /k$ flux of $a$

From the classical equation of motion
\begin{align}
    \frac{\delta S}{\delta a_0} &= 0
\end{align}
implies that
\begin{align}
    \frac{k}{2\pi}b &=
    -\frac{q}{2\pi}B
\end{align}
which tells us that
\begin{align}
    B &=
    -\frac{k}{q}b
\end{align}
This is another way of seeing that flux of $a$
comes with flux of $A$.

In particular,
$2\pi/k$ flux of $a$ is going to come with
$-2\pi/q$ flux of $A$.

One thing we can deduce from all of this is that this theory actually gives us a
fractional quantized Hall conductance associated with this background $U(1)$.
Let me explain.

\section{Fractional quantized Hall conductance}
So far we deuced $2\pi/k$ flux of $a$
comes with a $-2\pi/q$ flux of $A$.
But we also know that $2\pi/k$ fox of $a$ also comes with
$q/k$ charge under $A$.

So all together,
this tells us that $-2\pi$ flux of $a$
would give us charge of $q^2/k$.
This is precisely what you expect if the system had a Hall conductivity
\begin{align}
    \sigma_H &=
    - \frac{q^2}{k} \frac{1}{2\pi}
\end{align}
If your hall conductance was $\sigma_H = \nu \frac{1}{2\pi}$.

If you inserted $-2\pi$ flux of $A$,
after you inserted that,
you should get a $q^2/k$ charge.
Physically,
if you had an annulus,
inserting the electric field in the angular direction,
which causes charge to flow radially,
and if that is the amount of charge that's accumulated,
that means this is the Hall conductivity.

That means we have effective action given by the Lagrangian density
\begin{align}
    \mathcal{L} &=
    - \frac{q^2}{k}
    \frac{1}{4\pi}
    A_\mu \partial_\nu A_\lambda \epsilon^{\mu\nu\lambda}
\end{align}
Now another way of thinking about this is let's pick our spacetime manifold to
just be $M^3 = \mathbb{R}^3$.
Imagine your system is on space times time,
and space is the Euclidean plane.
Then we don't have to worry about non-trivial topological gauge configurations.
Let's define the part integral
\begin{align}
    Z\left( M^3, A \right)
    &=
    e^{iS_{eff}[A]}
    Z\left( M^3, A=0 \right)
\end{align}
where that effective action is
align
\begin{align}
    &=
    \int \mathcal{D} a\,
    e^{
    i\int_{\mathbb{R}^3}
    \left( 
    \frac{k}{4\pi} a\,d a
    +
    \frac{q}{2\pi} A\, da
    \right)
    }
\end{align}
We're going to shift $a$ to absorb the second term and only get a term that
depends on $A$.
So write
\begin{align}
    S[a, A] &=
    \frac{k}{4\pi}
    \left( a + \frac{q}{k} A \right)
    d\left( 
    a + \frac{q}{k}A
    \right)
    -
    \frac{1}{4\pi}
    \frac{q^2}{k}
    A\, dA
\end{align}
Now shift
\begin{align}
    a &\to
    a^1 =
    a + \frac{q}{k}A.
\end{align}
And now
\begin{align}
    Z\left( M^3, A \right) &=
    \left( 
    \int \mathcal{D} a'\,
    e^{i \frac{k}{4\pi} a'\ , a'}
    \right)
    e^{-\frac{1}{4\pi} \frac{q^2}{k}
    \int_{\mathbb{R}^3 A\, dA}
    }
\end{align}
which means we can write
\begin{align}
    S_{eff} &=
    -\frac{1}{4\pi}
    \frac{q^2}{k}
    \int_{M^3}A\, dA.
\end{align}

One thing that should bother you is that this $q^2/k$ coefficient
is a fracton.
And I told you the coefficient of the CS term should be an integer.
Why is this allowed?
What happened here?

We picked $\mathbb{R}^3$ so we don't have to worry about large gauge
transformations,
we led to the quantization restriction.

Suppose I let $M^3$ be general.
What part of this derivation did we make some wrong assumption that allowed us
to get to this point where the coefficient is not integer?
One issue is that because $a$ is a $U(1)$ gauge field,
there are certain quantisations on $a$.
Fr example,
\begin{align}
    \frac{1}{2\pi}\int_{S^2} f \in \mathbb{Z}
\end{align}
But if you shift $a$ by $\frac{q}{k}A$,
you ruin all these quantisation conditions.

Another useful perspective on what goes wrong is,
this theory if you quantize n a torus has a bunch of ground sates.
Just to regularize it,
you can imagine it has a Maxwell term,
so the actual spectrum of the quantum theory
is a bunch of degenerate ground states,
and a bunch of excited states.
When we integrate out $a$,
and forget about $a$,
we're doing something not quite allowed,
because we're integrating out something that gives rise to ground state
degeneracy,
which you shouldn't do.
It's not a high energy mode anymore.
It's a zero mode in this case.
We shouldn't integrate out $a$
because of the zero does,
so at best we should split and integrate.
But because we picked $M^3=\mathbb{R}^3$,
we don't have to worry.
But it wouldn't be valid if we did it on other manifolds.

What we can conclude from all this is that this $U(1)_{-k}$
CS theory with $q=1$
gives fractional quantized hall conductance of
\begin{align}
    \sigma_H &=
    \frac{q^2}{k}\frac{1}{2\pi}.
\end{align}
If we used SI units,
\begin{align}
    \sigma_H &=
    \frac{q^2}{k} \frac{e^2}{h}.
\end{align}
This theory describes the most prominent fractional quantum Hall states.
They are gapped phase of matter but they have this fractional quantum Hall
conductance.

Note for $k=1$,
\begin{align}
    \mathcal{L}
    &=
    -\frac{1}{4\pi} a\, da
    +
    \frac{1}{2\pi} A\, da
\end{align}
This actually describes a Chern insulator with Chern number 1,
which dis a Chern insulator or integer quantum Hall state.
This describes a fermionic theory because $k$ is odd
In fact the world line of this fermion is described by
\begin{align}
    e^{i\oint_{\gamma} a\, dl}
\end{align}
with $h=\frac{1}{2}$.
Furthermore,
it has a unique ground state.
Also  $\sigma_H=1$.
Also it is gapped.

\section{Gauge-invariance and boundary theory}
This is a boundary theory.

We studied the Chern insulator,
and we know it has a chiral.
Suppose you had an action
\begin{align}
    S &=
    \frac{m}{4\pi} dx^3 \, a\, da\,
    \partial_\mu \left( f \partial_\nu g_\lambda \right)
    \epsilon_{\mu\nu\lambda}\\
    &=
    \frac{m}{4\pi}
    \int_{y=0}
    \int dx\, dt\,
     \right)
\end{align}
To define theory,
restrict $\left. f\right_{\partial M = 0}$
Some gauge degrees of freedom of $a_\mu$
become physical degrees of freedom on body.

To study the edge theory,
consider
\begin{align}
    a_t &= 0
\end{align}
Then requiring
\begin{align}
    \frac{\partial S}{\partial a_t} &= 0
\end{align}
gives
\begin{align}
    \partial_x a_y - \partial_y a_x = 0.
\end{align}
Solve the constraint
\begin{align}
    a_i &= \partial_i \phi
\end{align}
So the action is
\begin{align}
    S &=
    \frac{m}{4\pi}
    \int d^3 x\,
    a_i \partial_t a_j \epsilon^{ij}\\
    &=
    \frac{m}{4\pi}
    \in d^3x
    \partial_ii \phi \partial_t \partial_j \phi\\
    &=
    \frac{m}{4\pi} \int_{V=0}
    dx\, dt\,
    \underbrace{
    \partial_x \phi \partial_t
    }_{pq  - H}
\end{align}
but since $H=0$,
the velocity of edge modes = 0.
Instead could have set
\begin{align}
    a_\tau &= a_t + v_a_x = 0
\end{align}
which implies
\begin{align}
    S &=
    \frac{m}{4\pi} \int_{y=0}
    dx\, dt\,
    \left( \partial_t + v\partial_x \right)\phi\cdot
    \partial_x \phi
    + v'\partial_x \phi \partial_x \phi
\end{align}
so the equation of motion is
\begin{align}
    \partial_t \phi + v\partial_v \phi &= 0
\end{align}
Modes are chiral with velocity $v$

\section{Quantizing chiral boson theory}
\begin{align}
    S &=
    \frac{m}{2\pi}
    \sum_{k>0}
    dt\,
    \left( 
    ik \phi_k dot{\phi}_k
    - vk^2 \phi_k\phi_k
    \right)
\end{align}
and because $\phi$ is real,
\begin{align}
    \phi_k &= \phi_{-k}^{*}
\end{align}
Because $\phi_k$ with $k>0$ are canonical coordinates,
The canonical momenta are
\begin{align}
    \pi_k &=
    \frac{\partial S}{\partial \dot{\phi}_k}\\
    &= \frac{m}{2\pi} ik \phi_{k}
\end{align}
Then
\begin{align}
    \left[ \phi_k, \pi_{k'} \right]
    &=
    i\delta_{k, k'}
\end{align}
which gives
\begin{align}
    [\phi_k, k'\phi_{k'}]
    &=
    \frac{2\pi}{m}\delta_{kk'}
\end{align}
We can define the density as
\begin{align}
    \rho &=
    \frac{1}{2\pi} \partial_x \phi
\end{align}
sot then
\begin{align}
    \left[ \rho_k, \pho_{k'} \right]
    &=
    -\frac{1}{4\pi^2} kk' \left[ \phi_k, \phi_{k'} \right]
\end{align}
and finally we have
\begin{align}
    \left[ \rho, \rho_{k'} \right]
    &=
    \frac{1}{2\pi m} k \delta_{k+k',0}
\end{align}
and if you have an operator that satisfy this,
you have a U(1) Kac-Moody algebra.

In real space,
if you just Fourier transform this equation,
you get
\begin{align}
    \left[ 
    \rho(x), \rho(y)
    \right]
    &=
    \frac{i}{2\pi m}
    \delta'\left( x - y \right)
\end{align}
which means that
\begin{align}
    \left[ \rho(x), \phi(y) \right]
    &=
    -\frac{i}{m} \delta\left( x - y \right)
\end{align}
You can actually integrate it one more time
to get the real space commutation relation for $\phi$.
This is telling us that
\begin{align}
    \left[ \partial_x \phi(x), \phi(y) \right]
    &=
    -\frac{2\pi i}{m} \delta\left( x - y \right)
\end{align}
and then
\begin{align}
    \left[ \phi(x), \phi(Y) \right]
    &=
    -\frac{i\pi}{m} \sgn\left( x - y \right)
\end{align}
And the derivative of that gives you the delta function.
This is a crucial commutation relation.
This means the field $\phi$ doesn't even commute with itself at different points
in space.
Usually it commutes with itself in different points in space.
Or anti commutes if fermionic.
No matter how far apart,
it doesn't commute with itself at different points.
This is the manifestation of the fact that particles have fractional statistics.
If you move world lines across each other gives fractional statistics,
and this is kind of a manifestation of that in the edge theory.

If you write the Hamiltonian in momentum space,
\begin{align}
    H &=
    2\pi m V
    \sum_{k> 0}
    \rho_k^\dagger \rho_k
\end{align}
There's some restriction on $m$ and $V$,
because you want it to be bound from below.
In order for $H$ to be bounded from below,
you're going to need $mV>0$ for stability.

This is interesting.
Just from the sign of $m$,
you can determine the sign of the velocity.
The direction of the propagation is determined by the sign of $m$.

So we basically completely understand this theory so far.
We have these modes with different momenta $k$,
and we can occupy different states at different modes.
The point is that this Hamiltonian looks like decoupled harmonic oscillators.

For each $k>0$,
there is a $\rho_k$ that satisfies $\rho_k^\dagger = \rho_{-k}$
and the commutation relations
\begin{align}
    \left[ \rho_k, \rho_k^\dagger \right]
    &=
    -\frac{1}{2\pi m} k\\
    \left[ \rho_k, \rho_{k'}^\dagger \right] &= 0
    \qquad\text{if }k \ne k'
\end{align}

So these $\rho_k$, $\rho_k^\dagger$ are creation and annihilation operations
for oscillator modes.


I can have the ground states of some bulk that is described by a CS theory,
ad the edge corresponds to these Luttinger Liquid modes.
They are like phonons,
with density fluctuations near the boundary of the system.
These can happen at arbitrarily low energy,
because you can have arbitrarily long wavelength excitations.

This $\rho_k$ excitations describe overall neutral density fluctuations
at the edge of the system.
If you let it go,
these excitations will travel in a particular direction along the boundary.

\begin{question}
    How is this associated with fractional statistics?
\end{question}
That's the next thing I'm going to say.

\begin{question}
    Has this been measured experimentally.
\end{question}
So far what would you measure?
You perturb it and watch it move in a chiral fashion.
Not really,
you don't have that much control,
to perturb it and image it at a  resolution high enough to see they move in a
particular direction,
so no that has not been directly seen.

Remember this overall theory has overall U(1) charge conservation associate
with the $A$ that we couple to.
Not only should we have neutral excitations,
which is what this $\rho_k$ are,
but there should also be charged excitations,
that is,
operators that carry charge.
Let's find what these operators are in this theory.

So operators that create charged excitations.
For example,
let's look for an operator $\Psi^\dagger$
that creates charge $1$.

Now in fractional quantum Hall context,
this $\Psi$ is the electron operator,
because our fundamental constituent that carries charge of unit 1 is really the
charge $e$ operator,
which is the electron operator.
What do we demand of an operator that carries charge 1?
It should have the commutation relation
\begin{align}
    \left[ \rho(x), \Psi^\dagger\left( x' \right) \right]
    &=
    \delta\left( x - x' \right)\Psi^\dagger(x')
\end{align}
You can think of this as the definition of an operator with charge 1.
Because it give you back the operator with density with coefficient 1.
And it turns out
\begin{align}
    \Psi \propto e^{im\phi}
\end{align}
and you can see this from the commutation relations for $\phi$ and $\rho$.
We previously learnt that
\begin{align}
    \left[ \rho, \phi \right]
    &=
    \frac{i}{m}
    \delta\left( x - y \right)
\end{align}
which implies
\begin{align}
    \left[ \rho, e^{im\phi} \right]
    &=
    \frac{i}{m}
    \left( im \right)
    e^{im\phi}
    \delta\left( x - x' \right)
\end{align}
I may have made a sign error.
But this is the operator that creates a charge 1 operation.
\begin{align}
    \Psi \propto e^{im\phi}
\end{align}
We can try to figure out what its statistics are.
If I look at
\begin{align}
    \Psi(x)\Psi(y)
    =
    e^{im\phi(x)}
    e^{im\phi(y)}
    =
    e^{im\phi(y)}
    e^{im\phi(x)}
    e^{-m^2\left[ \phi(x), \phi(y) \right]}
\end{align}
so then
\begin{align}
    \Psi(x) \Psi(y)
    &=
    \Psi(y) \Psi(x)
    \underbrace{e^{+im\pi \sgn(x - y)}}_{
    \left( -1 \right)^m
    }
\end{align}
so what you learn is that
\begin{align}
    \Psi(x) \Psi(y)
    &=
    \Psi(y) \Psi(x)
    (-1)^m
\end{align}
so if $m$ is odd then $\Psi$ is a fermionic operator,
and if $m$ is even then $\Psi$ is bosonic.

So the microscopic degrees of freedom that carry charge 1 are fermions.

I'm going to call this $\Psi$ the electron operator from now on.
Let's calculate the electron correlation function.
First you need to calculate the correlation function of the boson.
\begin{align}
    \left\langle
    \phi\left( x, t \right)
    \phi\left( 0, 0 \right)
    \right\rangle
    &=
    -\frac{1}{m}
    \ln
    \left|
    x - vt
    \right|
    +
    \text{constant}
\end{align}
That's the result for this correlation function.
I'll put it in the homework to derive this operator.
That means the Green function is
\begin{align}
    G(x, t) &:=
    \left\langle
    T\left( 
    \Psi^\dagger(x, t)
    \Psi(0, 0)
    \right)\\
    &=
    e^{m^{2}\left\langle\phi(x, t), \phi(0, 0)\right\rangle}\\
    &\propto
    \frac{1}{\left( x - vt \right)^m}
\end{align}
so this Greens function has a non-trivial integer exponent 4$m$
For the free fermion Chern insulator,
$m$ is just 1.
The fact that the exponent here is raised to $m$
signals some much more non-trivial correlations that go beyond free fermion.

Let me write the result in momentum space we will use later.
\begin{align}
    G(k, \omega) &=
    \frac{\left( vk + \omega \right)^{m-1}}{
    \omega - vk + i\sgn(\omega) 0^+
    }
\end{align}
where $0^+$ is a positive infinitesimal used to regularize things,
which you set to zero at the end of the calculation.
Then the density of states is the imaginary part of the Greens function at
$k=0$.
\begin{align}
    N(\omega) &\propto
    \Im G(k=0, \omega)
    \propto
    |\omega|^{m-1}
\end{align}

Tunneling experiments.
You could try to tunnel into these fractional quantum hall edge modes,
and have tunnelling from  a metal from the outside.
In these experiments,
you can measure the current vs voltage across the gap.
That $I$-$V$ characteristic measure the density of states.
In particular,
\begin{align}
    \frac{dI}{dV} \propto N(v) \propto |v|^{m-1}
\end{align}
The prediction is that because you have this chiral Luttinger liquid with
nontrivial $m$,
the tunneling conductance should be quantized and non-linear.

This is a fairly traditional condensed matter experiment.
It was tested in experiment,
and it looks  like there is some non-linear conductance,
but it doesn't seem to be quantized.
There's all sorts of messy physics that ruins the quantization.
We think we know why that extra messy physics comes in.
Because I'm focusing on field theory,
I didn't set up the stuff to tell you why.

But at the end of the day,
you have some confining potential that confines your electron to some disk.
It could be sharp or smooth.
In the experiments,
it's quite smooth,
and when it's smooth,
you have extra modes that contribute to the $dI/dV$ curve
and that ruins the quantization.
In principle,
if you have sharp wells,
at low enough energy and temperature,
you should see this effect.

\begin{question}
    Smooth compared to what length scale?
\end{question}
In fractional quantum Hall,
there is a characteristic length called the magnetic length
\begin{align}
    \ell_B \sim \frac{1}{\sqrt{B}}
\end{align}
So it's smooth compared to the magnetic length.

[1.5 hours late]
\section{Second Quantization notation}
We use the notation
\begin{align}
    \ket{\varphi} = \ket{n_1, n_2, \ldots}
\end{align}
where
\begin{align}
    \hat{N} &=
    \left( n_1 + n_2 + \cdots \right)
    \ket{n_1, n_2, \ldots}
\end{align}
is the total number operator and the Hamiltonian is
\begin{align}
    \het{H} \ket{\varphi} &=
    \left( 
    n_1 E_{n_1} + n_2 E_{n_2} + \cdots
    \right)
    \ket{n_1, n_2, \ldots}
\end{align}

\section{Bosonic Ideal Gas}
Second quantization notation.
We had a sum over all multiparitcle states over all multiparticle states.
Now we just have
\begin{align}
    Z_G(T, V, \mu) &=
    \sum_{n_1, n_2, \cdots = 0}^{\infty}
    \bra{n_1,n_2,\ldots} e^{-\beta(H - \mu N)} \ket{n_1,n_2,\ldots}\\
    &=
    \sum_{n_1=0}^{\infty} e^{-\beta(E_1 - \mu) n_1}
    \sum_{n_2=0}^{\infty} e^{-\beta(E_2 - \mu) n_2}
    \cdots\\
    &=
    \frac{1}{1 - e^{-\beta(E_1 - \mu)}}
    \frac{1}{1 - e^{-\beta(E_2 - \mu)}}
    \cdots
\end{align}
So then
\begin{align}
    \Omega(T, V, \mu) &=
    -k_B T \ln Z_G\\
    &=
    -k_BT \sum_k \ln\left( 1 + e^{-\beta (E_k - \mu)} \right)\\
    &\approx
    k_B TV \int \frac{d^3k}{(2\pi)^3}
    \ln\left( 1 - e^{-\beta(E_k - \mu)} \right)
\end{align}
where each
\begin{align}
    E_k &= \frac{\hbar^2 k^2}{2M}
\end{align}

It's just the same thing as the modes of a crystal vibrations.
They're the same as an EM field,
because I can find the normal modes of th EM feld.
This has nothing to do with Harmoinc oscillators except it does.
Each single aprticle state mode is like a harmonic oscillator.
The state is indexed by the particle number,
and the enregy levels are equally spaced just like the harmoinc oscillator.
That's the special property.

\section{Fermion ideal gas}
For fermions,
just change the limit of the sum from $\infy$ to $1$
and you get
\begin{align}
    \Omega(T, V, \mu) &=
    -k_B T V g \int \frac{d^3k}{(2\pi)^3}\,
    \ln\left( 
    1 + e^{\beta(E_k - \mu)}
    \right)
\end{align}
where $g$ is the number of internal states,
like the spin.

The little difference in sign changes things profoundly.

Just using the general formalism of the grand canonical ensemble,
I should be able to find the boson canonical ensemble,
etc.


For example,
if I have $\Omega$,
I can find the average $N$.
Let's compute the average number of particles over the volume so we get a
density.
\begin{align}
    \frac{N}{V} &=
    \frac{1}{V}
    \left. \frac{\partial \Omega}{\partial\mu}\right|_{T,V}\\
    &=
    - k_B T g
    \int \frac{d^3k}{(2\pi)^2}
    \frac{ \beta e^{-\beta \epsilon_k}}{%
        1 \mp e^{-\beta \epsilon_k}
    }
\end{align}
where $\epsilon_k=E_k - \mu$ which appears so frequently I may as well have a
name for it,
where the upper sign is for fermions and the lower sign is for bosons.
And then
\begin{align}
    \frac{N}{V} &=
    g \int \frac{d^3k}{(2\pi)^2}
    \frac{1}{e^{\beta\epsilon_k}\mp 1}
\end{align}
and the only difference is the sign.
The integrand is interpeted as the average occupation number.
THen I take an average over many states.
For a particular single energy state
what is the average expecation?
Well that is just the occupation number
\begin{align}
    n(E) &=
    \frac{1}{e^{\beta\epsilon_k}\mp 1}
\end{align}
which is the occupation number of hte density per unit of $k$.

Another thing I like to calculate is the enrgy density.
How do I find that?
I can thnk of this $\Omega$ here,
take a derivative in relation to $T$ to find the entropy,
then use the Lengedre trnasform to find the entorpy.
The little trick about this is the following.
This gives me $N$ as a functio of the tempreaturea nd the chemical potential
but it's hard to invert the relation to find what $\mu$ is needed to givea
aparicular $N$.
Although it's easy to say Legendre tranform,
it's hard to do in partice.

It doesn'tmatter,
you can manipulate the integrals back and forth,
I won't do this in front of you or do tell you to do it in the homework.
Which one do yo uwant?
Of course it's in the homework.

Thea nswer is something you'd expect.
It's as um over single particle states.
We're looking for the total energy.
That's the occupation number for one particular satte.
If I multiply it by the enrgy of the single particular state I get hte enrgy.
\begin{align}
    \frac{E}{N} &=
    g \int \frac{d^3k}{(2\pi)^3}
    \frac{E_k}{e^{\beta\left( E_k - \mu \right)} \mp 1}
\end{align}
where the  $g \int \frac{d^3k}{(2\pi)^3}$ means sum over single particle states,
the $\frac{1}{e^{\beta\left( E_k - \mu \right)} \mp 1}$ is the occupation number
for the $k$ and $E_k$ is the energy ofa single particle.

In the mideterm,
If I ask a question inthe midterm,
I will not give you this equation,
ecause it's so intuitive and obvious.

I'll give you the $N/V$ because the sign is not obvious,
but this $E/N$ formula is obvious from that.
It's difficult to compute but it's a simple answer in the end.

\section{Quantum Spin Liquids}
\subsection{LSM Constraints}
If Hamiltonian has symmetry
\begin{align}
    G = \underbrace{SO(3)}_{\text{spin rotation}}\times
    \underbrace{ \mathbb{Z}^d}_{\text{trans sym}}
\end{align}
and an odd number of spin-$\frac{1}{2}$ per unit cell,
then no trivial unique gapped ground state is possible.

In $d=1$, either
\begin{enumerate}
    \item Ground state either breaks symmetry
    \item Or gapless
\end{enumerate}
In $d>1$
\begin{enumerate}
    \item  ''''
    \item ''''
    \item Topological order (compatible with symmetry)
\end{enumerate}
With
\begin{align}
    G &= U(1) \times \mathbb{Z}^d
\end{align}
filling $\nu$ is charge per unit cell.

If $\nu$ is fractional then no unique gapped ground state is possible.

The general version.
\begin{align}
    G &=
    G_{\textrm{on-site}} \times \mathbb{Z}^d
\end{align}
If projective representation per unit cell
\begin{align}
    [w] \in H^2 \left( G, U(1) \right)
\end{align}
the no unique gapped ground state is possible.

Consider $U(1)$ symmetry.
Consider a cylinder with space like $T^d$.
Let $L_x=: L$.
$V$ is then umber of unit cells, and $C=V/L$.
Then translational symmetry acts like
\begin{align}
    T_x \ket{\Psi_0} &=
    e^{iP_x^0}
    \ket{\Psi_0}
\end{align}
Suppose ground state is unique and gapped.
Insert one flux quantum adiabatically,
then large gauge transformation
\begin{align}
    U &= e^{i \frac{2\pi}{L} \sum_r x n_r}
\end{align}
acts like
\begin{align}
    \ket{\Psi_0} &\to
    U c \ket{\Psi_0} = \ket{\tilde{\Psi}_0}
\end{align}
and
\begin{align}
    T_x \ket{\tilde{\Psi}_0}
    &=
    e^{i \tilde{P}_x^0}
    \ket{\tilde{\Psi}_0}
\end{align}
Then the new momentum is
\begin{align}
    \tilde{P}_x^0
    &=
    P_x^0
    +
    \frac{2\pi}{L} \sum n_r\\
    &=
    P_x^0 + \frac{2\pi N}{L}\\
    &=
    P_x&0 + 2\pi \frac{V}{L} \frac{p}{q}
\end{align}
where we let $\nu = N/V$.
Assume $\nu = p/q$ for some integers $p$ and $q$ which are coprime.
Pick $V/L$ to be coprime with $q$ so that
$\frac{V}{L} \frac{p}{q}$ is not an integer.
So then
\begin{align}
    e^{i\tilde{P}_x^0}
    \ne e^{i P_x^0}
\end{align}
which implies
$\ket{\Psi_0}$ and $\ket{\tilde{\Psi}_0}$
must be orthogonal,
which is a contradiction.

\begin{question}
    This is not a full proof?
\end{question}
No, to get a full and rigorous proof,
it was done originally by LSM.
They considered a variational state,
and showed you get could to $e^{1/L}$ energy of the ground state.
Hastings gave a long rigorous proof,
but add to give a number of assumptions about then umber of unit cells.
He was able to prove it for even cells,
but still had restrictions on the size.

\begin{question}
    How do you have the freedom to pick $V/L$.
\end{question}
We're assuming we have enough freedom to choose the case.
This proof would not work for specific vlaues of $V/L$.
Then ext step is that if I cannot have a unique ground state for this system
size,
then I also cannot have it for any system size.
The intuition is that if you did have a gapped ground state,
even if it wre degenerate,
then it has a finite coreelaiton elngth,
so the spectrum shouldn't care xactly what hte systme size is because it'sa
lgobal thing up to exponentially small corrections.
Go from speicfic system sizes to generic syste msizes,
by arguing correlation lengths.

\begin{question}
    This smells like the FQH argument,
    where $V/L$ is the number of orbitals.
\end{question}
In FQH yo uhave fractional filling,
and htis is exactly a way of proving FHQ has degenerate ground states,
and if you did the work,
then number of degenerate ground states if $q$.
If you went further,
you could show that if I insert one flux quantum,
that toggles from one state to another,
and you can go futher and argue that you have to have $q$ states.

\begin{question}
    What if there is no translational symmetry?
\end{question}
Whatwe learn her is that if you have translational symmetry and on site
symmetry,
no unique ground state is possbile.
For $d>1$ there are 3 psosiblities.
Topologcial order is compatible wiht symmetry.
The degenarcy itself is robust to breaking tansliaontl symmetry.
The topological degeneracy when you insert flux does change which degenerate
state you are in.
The main point is that once you have topolgoical order you can satisfy this
theorem,
but the teorem itself sdone't require TI,
but jsut that TI requires you have these almsot degenerate states.
The tarnslation symetry and this filling requirement imply that you musth ave a
degenearcy,
but that degenracy can still be robust to breaking these tnralsioantl symmetry.
Nothing says the egeneracy isn't robust if I brek the translational symmetry.

\begin{question}
    Here you asumed periodiciiyt in the $x$ direction,
    but where did we use it?
\end{question}
Technically we used it because I asumed it had trnalaiotnl symmetry in any
direciton,
but if I had a boundary,
I would technically break tranlationl symmetry.
And I needdd translaitonal symmetry to define a filling.
It's there implicitly,
but explicitly I didn't use it.
In practise it would get a bit more subtle wiht boundaries,
becase you would have chiral edg modes on the boundary,
and it wouldn't necessarily gapped.

This result applies to way more states if I put it on a torus.

\section{Spin liquids}
That was all motivation.
Once you have a system with fractional filling,
then topological order is the only way to still have a gapped state and be
compatible with the symmetry.
Interestingly if you have such a system with an odd number of spin-$\frac{1}{2}$ 
per unit cell,
plus the other conditions,
then that implies topological order.
You have systems with fractional filing and spin-half per unit cell,
ad if the system wants to preserve the symmetry and be gapped,
the only thing that can make that happen is topological order.

The physical picture for what the quantum spin liquid is,
comes from the idea of a resonating valence bond.
\begin{align}
    H &=
    \sum_{\langle ij\rangle}
    J_{ij} \vec{s}_{i} \cdot \vec{s}_{j}
\end{align}
for $J_{ij}>0$.

For a single pair $J_{ij} S_i S_j$,
the lowest energy state is a spin singlet.
Idea of resonating valence bond,
is the idea that it's a superpositions of different singlet configurations.
This goes back to chemistry,
where you have molecules like benzine,
where you have 6 carbon atoms,
and hydrogen out here,
and then the electrons can form singlet configurations.
But the actual state of the molecule is a superposition between this singlet
configurations and the other.
People knew about it in Chemistry,
and Linus Pauling started thinking about many-body valences bond resonance in
metal,
and that motivated Phil Anderson to think about resonating valence bond in
insulators.
Anderson in 1973 proposed the idea that you could have an electrically
insulating system where the spin degrees of freedom form resonating valence
bonds.
Meaning that the ground state is a superposition over valence bond
configurations.

The system is forming a quantum liquid of singlets,
a gorund stae superpositon of all posssible singlet configuraotns.
Andreson talked about it in 1973 andit ws completely ignored.
In 1986,
physicist dicovered high temperature superconductivity in the cuprates.
Right after that there was  big explosion in the field.
People tried to come pu with theories for why cuprates where high-temperature
superconductors,
and Anderson had this spin liquid lying in a draw,
and he said look,
maybe this spin liquid could have something to do with it.

Since 1987,
there's a huge explosion about understanding spin liquids and resonance states.
I'm not going to mention the relationship with high-temperature superconductivity,
because it's very tenuous and controversial.
It may or may not have anything to do with high-$T_c$ superconductivity,
and there are opposing camps of physicists who hate each other.
It led to the destruction of the careers of many young scientists in the
1990s,
and there should be dramatic books written about the sociology about this,
extremely fascinating.

Now,
we're getting to a point where those people are dying and retiring,
so maybe the new generation can solve this problem.
It was pretty bad.

The kind of ground states natural to think about,
once you realise that the lowest energy state is a singlet,
the natural states to think about are a crystal of these valence bonds,
sometimes called VBS or valence band solid.

These bonds form singlets,
for do so in some ordered pattern.
This kind of state breaks translation symmetry,
but it preserves the SO(3).
In the context of LSM,
it's the first possibliity where you just breka the symmetry.

Another state possible is a spin-density wave.
The simplest example is a ferromagnet.
So you have spins up and down.
An anti-ferromagnetic is a special spin density wave with wave vector $k=\pi$.
This thing breaks SO(3) as translatoin,
but it preserves some combination.

The third interesting possiblitiy s htis RVB state,.
There are two kinds of RVB states we talk about.
One is short-range RVB,
where spins near each other form valence bonds,
so only a superposition over states where nearby spins for valence bonds.
THis kind of thing means systems have finite correlation lnegth.
Anything happening in some reaosn,
spins are only connected to spins nearby,
so they come with finite corelation length,
which is typically associated with gapped states.

Ify ou had a short range RVB states,
a gapped state,
then it has to have topological order by the LSM theorem.
Specifically,
if you put it on a torus,
there iwlll be a gorund satte degenearcy as required by LSM.

Then you have long-range RVB,
where you have valnece bonds arbirarily far apart,
but of course the amplitude for having long range valence bonds will decrease
far away but perhaps power law decay.


It's the same reason why you have gapless systems at all.
A gapless system is one where the correlation decays as a power law.
How can that happen when I only have short-range interactions?
The system some how conspires to have long-range correlations,
power law decay,
but still long range.

One thing you can think of physically is suppose that these bonds are really
strong.
This guy has really strong bonds wiht this one,
but the bonds iwht this guy are weak,

So far I've given you a picutre,
from which you can construct variatonal states,
where you can write down a ground sate that is a superpositon of different
configuraitons.
You can try to figure out are these energetically favourable ground states.
That's what Anderson did.
It's not super concrete either.
I have a Hamiltonian whose ground state is going to look like this form.

\section{Quantum Dimer Model}
An importnat devlopment after 1987 was quantum dimer models,
which gave a nice way to think about spin liquids,
changing the rules a little to make thisp icutre more precise.
It changed the rules a bit so people don't like that.
But the good review article is arXiv 0809.3051,
and the original paper was Rokhsar-Kivelson 1987.

The quantum dimer model says let's forget about the spins entirely,
and think of a quantum state system of dimers,
which are just colorings on edges.

The idea is to directly model valence bonds.
We replace valence bonds with dimers.
The key point is that we treat different dimer configurations as orthogonal
quantum states.
This is the key thing.
This is actually incorrect for spin systems.
If I take a spin system,
consider some condition nation of singlets, valence bonds,
that's not going to be orthogonal to every other configuration.
But the idea of the dimer model is to forget that and just demand they are
orthogonal.

The next thing is to assume the dimers cannot touch each other.
So if one spin forms a singlet with one spin,
it cannot also form a singlet with another spin.
So dimers cannot touch.

Here we just set up the Hilbert space.
Now we want a Hamiltonian,
for example on a square lattice.
The Hamiltonian will have two terms,
the sum over plaquettes of the square lattice.
\begin{align}
    H_{QDM}
    &=
    \sum_{\square}
    -t \left(
        \ket{=}\bra{||}
        + \mathrm{h.c.}
    \right)
    +
    V
    \left( 
    \ket{=}\bra{=}
    +
    \ket{||}\bra{||}
    \right)
\end{align}
The properties of the ground state depend very sensitively what dimension we are
considering,
whether square, cubic,
and it also depends very sensitively on the geometry,
like triangle, honeycomb, etc.

Because we have these local rules which flip configurations,
let me define some terminology.

We could have a \emph{flippable plaquette}
which takes $=$ to $||$.

A non-flippable plaquette,
consider 3 plaquettes,
each with 1 singlet,
but the bonds are not touching.
Then that's not flippable.

We could also have a \emph{flippable loop}.

I can draw a fictitious loop that surrounds these guys.
For a dimer configuration on this loop,
if I can flip it with local moves to get the complementary configuration,
then if I can get from one to another by dong local moves in the interior of
the loop and maybe a little outside of it,
then that would be  flippable loop.

Once we introduce these local moves,
the question is which dimer configuration can be converted to another dimer
configuration by local moves.
And that introduces topology and winding into the system.

You can ask if there are dimer configurations that cannot be transformed into
each other by local moves,
and if there are are,
then you can argue they are topologically distinct from one another.

It turns out the biggest difference in dimer models is whether your lattice
bipartite or non-bipartite.
Bipartite means that you can colour the vertices with 2 colours such that no
nearest neighbour has the same colour.
Square lattice you can do that so it's bipartite,
but you can't do that on a triangular lattice, so it's non-bipartite.

The notion of topological sector is that you count the number of dimers that are
crossing some line.
What I mean is that suppose you draw a triangular lattice.
Then you draw a dotted line that crosses some plaquettes.
You draw your line and you count how many crossings with dimers there are.
One thing you'll notice is that the local moves only change the number of
crossings by an even amount.

Local moves here means a specific move on two triangles.
The key point it's always 2 to 2.
That's where this parity thing comes up.

\begin{question}
    Is there a physical reason we have this constraint.
\end{question}
No.
We're just writing down interesting Hamiltonians for these and considering
interesting classes of moves.
We'll find these map to lattice gauge theory,
and changing the kind of moves allowed is changing the gauge constraints,
but there's really no physical reason.
It's made up anyway.
Physically motivated,
but the actual Hamiltonian is argued from another starting point.

\begin{question}
    Perhaps it's conservation of energy?
\end{question}
There are terms you can write down like
$S_i^+ S_j^+ S_k^- S_k^-$.
But then you can write down 3 spin terms as well\ldots
Suppose I have a full dimer covering,
so there are no extra dimers you can place anywhere that is compatible with the
rules.
Then you cannot move a single dimer.
Requiring we have  full dimer covering will restrict the local moves you can
have.

The point is that if I count the parity of the number of crossings,
that's a topological invariant because it's invariant under local moves.

Putting this on a torus gives four topological sectors,
because we have 2 non-contractile cycles,
one called $\alpha$ and another called $\beta$.
Suppose the number of crossings is $n_c^\alpha$ and $n_c^\beta$,
then the invariants are
\begin{align}
    n_c^\alpha\mod 2\\
    n_c^\beta\mod 2
\end{align}
You can get these from local plaquette moves.
We don'th ave such a term n the Hamilonitna htat allowed me to do that.

On a bipartitlelattcek we have the number o crossngs $\mod 2$.

If you have a bipartitle lattice,
then you have A sites and B sites.
Imagine that all of our dimers actually have an arrow on them,
but points to the A subslattice to the B sublattice,
so they implcitly have arrows on a to b aso on.

I can count the winding number,
which is
\begin{align}
    n_{\textrm{winding}}
    &=
    n_{\textrm{left-crossings}}
    - n_{\textrm{left-crossings}}
\end{align}
Ona a torus,
there are $\mathbb{Z}\times mathbb{Z}$

\section{Topological Excitations}
There are two kind of excitations we can have.
First first one are called \emph{visons}.

\begin{question}
    It's not clear it's the same for each vison
\end{question}

???

I vison is a dual excitation that lives on plaquettes.
Imagine a line that connects the two.
Then count the dimer crossings on that line.
The number of crossings is $n_c$.

The wave function of a vison is
\begin{align}
    \ket{\mathrm{ground state}} = \sum_c (-1)^{n_c} A_c \ket{c}
\end{align}
but the ground state is
\begin{align}
    \ket{\mathrm{ground state}} = \sum_c A_c \ket{c}
\end{align}

Why are they of this form?
Because it's like vortex.
These excitations become defined finite energy excitations.

In general,
the choice of line matters,
and the energy wold even depend on\ldots

The next type of topological excitation is called a \emph{monomer}.

I can have terms like this to $H$
\begin{align}
    \ket{|}\bra{:} + \ket{:} \bra|{}
\end{align}
The dots just means separate spins by themselves.
The name vison is controversial.
It was introduced in 1997 by Matt Fischer.
All people 10-years liquid than $\mathbb{Z}_2$ vortices got renamed to something
new.
there's a reason to change the name.
One way to think about what's going on,
say you have a normal superconductor with vortices.
You can have a $+1$ vortex,
or a $-1$ vortex.
A vison is kind of a superposition between a $+1$ and $-1$ vortex.
You can think of it as all quantum superposition of odd-number winding
vortices.
The -on is for particle.
V for vortex, and is for Ising.
So vison.

The key point is that depending on whether the ground state is ordered,
valence bond crystal,
or resonating valence bond state,
a uniform superposition over all kinds of valence bond configurations,
pending on which type of ground state we're in,
these excitations will either be confined or deconfined excitations.

Confined means linear in separation.
Deconfined means you can separate them arbitrarily far apart without energy cost.

There are mutual statistics between visons and monomers.

In ordered states they are confined and in spin liquid RVB they are deconfined.

Ordered VBS confined vs spin liquid RVB deconfined.

\section{Spherical Harmonics}
Since $V\to 0$ at $r\to\infty$, then $a_{lm}=0$.
Impose boundary condition at $r=a$.
\begin{align}
    \sum_{l=0}^{n}\sum_{m=-l}^{+l}
    \frac{b_{lm}}{a^{l + 1}} Y_{lm}\left( \theta, \phi \right)
    =
    V_0 \left( \theta, \phi \right)
\end{align}
note
\begin{align}
    V\left( \theta, \phi \right)
    &=
    \sum_{l,m}
    \left( a_{lm}
    r^l
    +
    \frac{b_{lm}}{r^{l + 1}}\right)
    Y_{lm}\left( \theta, \phi \right)
\end{align}
Use orthogonality of spherical harmonics to find $b_{lm}$,
\begin{align}
    \int d\Omega \, Y_{l'm'}^* \left( \theta,\phi \right)
    \left\{ 
    \sum_{l,m}
    \frac{b_{lm}}{a^{l + 1}}
    Y_{lm\left( \theta, \phi \right)}
    \right\}
    &=
    \int d\Omega\,
    Y_{l'm'}^*\left( \theta, \phi \right)
    V_0\left( \theta, \phi \right)
\end{align}
so
\begin{align}
    b_{lm}
    &=
    a^{l + 1}
    \int d\Omega\,
    Y_{lm}^*\left( \theta, \phi \right)
    V_0\left( \theta, \phi \right)
\end{align}
so finally
\begin{align}
    V\left( r, \theta, \phi \right)
    &=
    \sum_{l=0}^{\infty}
    \int_{m=-l}^{+l}
    \left( \frac{a}{r} \right)^{l + 1}
    Y_{lm}\left( \theta, \phi \right)
    \int d\Omega'
    Y_{lm}^*\left( \theta', \phi' \right)
    V_0\left( \theta', \phi' \right)
\end{align}
I'll show you your formula sheet before the exam.
Usually you just express the answer in terms of Legendre polynomials.
Usually you're going to leave your answer in terms of the $Y_{lm}$ except where
there is spherical symmetry and you can write it in terms of Legendre
polynomials.

You can either not bother with $Y_{lm}$ and use Legendre polynomials,
or just use spherical harmonics and reduce it to the Legendre polynomials.
Remember the relation between the spherical harmonics and the Legendre
polynomials.

Let's do another example.
\begin{example}
    Find the Dirichlet Green function for a sphere of radius $b$ centered at
    origin.
    \begin{align}
        \nabla^2 G\left( \vec{r}, \vec{r}' \right)
        &=
        -4\pi \delta^{3}\left( \vec{r} - \vec{r}' \right)
    \end{align}
    such that $G\left( \vec{r}, \vec{r}' \right) = 0$
    at $\left|\vec{r}\right| = b$.
\end{example}
This is a very standard problem.
I have a charge here somewhere in the sphere and I want to find the charge here.
It's Laplace's equation except for that one point,
so we look for a solution like that.

Awa from $\vec{r}=\vec{r}'$,
the Green function satisfies Laplace's equation
\begin{align}
    G\left( \vec{r}, \vec{r}' \right)
    &=
    \sum_{l=0}
    \sum_{m=-l}^{l}
    g_{lm}\left( r \right)
    Y_{l}^{\,m}\left( \theta, \phi \right)
\end{align}
So we look for a separable solution.
Now take this and plug it in to the definition of the Green function.

And then yo get
\begin{align}
    \sum_{l,m}
    \left\{ 
    \frac{d^2 g_{lm}}{dr^2}
    +
    \frac{2}{r}
    \frac{dg_{lm}}{dr}
    -
    \frac{l(l + 1)}{r^2}g_{lm}
    \right\}
    Y_{l}^{\,m}\left( \theta, \phi \right)
    &=
    -4\pi \delta^3 \left( \vec{r} - \vec{r}' \right)
\end{align}
Then we multiply both sides by ${Y^*}_{l'}^{\, m'}\left( \theta, \phi \right)$
and integrate over $\Omega$.
\begin{align}
    \frac{d^2 g_{l'm'}}{dr^2}
    +
    \frac{2}{r} \frac{dg_{l'm'}}{dr}
    -
    \frac{l'\left( l' + 1 \right)}{r^2} g_{l'm'}
    &=
    -4\pi
    \int d\Omega\,
    {Y^*}_{l'}^{\,m'}\left( \theta, \phi \right)
    \underbrace{\delta^3\left( \vec{r} - \vec{r}' \right)}_{
    \frac{\delta\left( r - r' \right)}{r^2}
    \delta\left( \cos\theta - \cos\theta' \right)
    \delta\left( \phi - \phi' \right)
    }\\
    &=
    -\frac{4\pi}{r^2}
    {Y^*}_{l'}^{\,m'}\left( \theta, \phi \right)
\end{align}
This is much simpler because this is just an ODE,
so like the 1D problem we solve.

Away from $r=r'$,
\begin{align}
    g_{lm}\left( r \right)
    &=
    A_{lm} r^l + B_{lm} \frac{1}{r^{l + 1}}
\end{align}
So it's just like Laplace's equation.
The only difference is that you have a sphere and a delta function source
somewhere here.
This source is at $\vec{r}'$.
So we have Laplace's equation everywhere except $\vec{r}'$.
So we're going to get one value of the $A_{lm}$ for $r<r'$
and another value for $r>r'$.
So we are solving Laplace's equation for $r<r'$ and then for $r>r'$
and then we stick the two solutions together.
This is similar to what we did when we were solving for the Greens function for
the box,
where we had a point charge in the middle of the box and we had to divide the
box into two pieces,
which we solved and matched the distributions,
did using a more careful sophisticated way using the convergence theorem,
but here I'm doing it much quicker,
as the two are mathematically equivalent,
and you won't have time in an exam to do all those divergence theorems.
So this is equivalent to what we did.
Here we are breaking the sphere into two pieces,
one $r<r'$ and $r>r'$,
just done much more directly than the box,
and this is much faster,
because very quickly it gets reduced to a 1D ODE.

So in general,
$A_{lm}$ and $B_{lm}$ will be different for $r<r'$
and $r>r'$.
This $1/r^{l+1}$ blows up when $r=0$,
so since $g_{lm}(r)$ must be regular at $r=0$,
we have
\begin{align}
    g_{lm}^{<}\left( r \right)
    &=
    A_{lm}^{<} r^l
\end{align}
which is the solution for $r < r'$.
Also, we need that
\begin{align}
    g_{lm}\left( r \right) = 0
\end{align}
at $r=b$,
so
\begin{align}
    g_{lm}^{>}\left( r \right)
    &=
    A_{lm}^{>}
    \left( 
    r^l
    -
    \frac{b^{2l + 1}}{r^{l + 1}}
    \right)
\end{align}
for $r>r'$.
The solutions are different for $r<r'$ and $r>r'$,
which is similar to the solution for the box
where the solution was different above and below the box
and you had to stick the solutions together at the boundary.
So now let's stitch the solutions together.

Since solutions must be continuous at $r=r'$,
\begin{align}
    g_{lm}^{<}\left( r' \right)
    &=
    g_{lm}^{>}\left( r' \right)
\end{align}
so you get this condition
\begin{align}
    A_{lm}^{<}
    &=
    A_{lm}^{>}
    \left( 
    1
    -
    \frac{b^{2l + 1}}{r'^{2l + 1}}
    \right)
\end{align}
And then integrate the ODE on both sides over $r$
from $r=r' - \epsilon$ to $r=r' + \epsilon$,
which is the region that contains the delta function.
Then you get
\begin{align}
    \left\frac{dg_{lm}}{dr}\right|_{r' + \epsilon}
    -
    \left\frac{dg_{lm}}{dr}\right|_{r' - \epsilon}
    &=
    -\frac{4\pi}{r'^2}
    {Y^*}_{l}^{m}\left( \theta', \phi' \right)
\end{align}
So this is the second condition.
And this is a lot messier.
\begin{align}
    A_{lm}^{>}
    \left\{ 
    l {r'}^{l - 1}
    +
    \left( l + 1 \right)
    \frac{b^{2l + 1}}{ {r'}^{l + 2}}
    \right\}
    -
    A_{lm}^{,}
    \left\{ 
    l {r'}^{l - 1}
    \right\}
    &=
    - \frac{4\pi}{ {r'}^2 }
    {Y^*}_{m}^{\, l} \left( \theta', \phi' \right)
\end{align}
And then you do some algebra,
and the final answer is
\begin{align}
    G\left( \vec{r}, \vec{r}' \right)
    &=
    \sum_{l,m}
    \frac{4\pi}{\left( 2l + 1 \right)}
    r_{<}^{l}
    r_{>}^{l}
    \left\{ 
    \frac{1}{r_{>}^{2l + 1}}
    -
    \frac{1}{b^{2l + 1}}
    \right\}
    {Y^*}_{l}^{m}\left( \theta', \phi' \right)
    Y_{l}^{\;m}\left( \theta, \phi \right)
\end{align}

\section{Cylindrical Coordinates}
\begin{example}
    Consider an infinitely long cylinder of radius $b$
    that is aligned so its axis is along the $z$-axis.
    The potential on surface is independent of $z$
    and given by $V_0(\phi)$.

    Find the potential inside the cylinder.
\end{example}
So the cylinder has radius $b$ and you're given the potential $V_0(\phi$ on the
boundary surface.
The problem is to solve for $\nabla^2 V=0$.

Once again we're going to use separation of variables.
\begin{align}
    \frac{1}{\rho}
    \frac{\partial}{\partial \rho}
    \left( 
    \rho
    \frac{\partial V}{\partial \rho}
    \right)
    +
    \frac{1}{\rho^2}
    \frac{\partial^2 V}{\partial \phi^2}
    +
    \frac{\partial^2 V}{\partial z^2}
    &=
    0
\end{align}
Note it should be independent of $z$.
So we get
\begin{align}
    \frac{\partial^2 V}{\partial \rho^2}
    +
    \frac{1}{\rho}
    \frac{\partial V}{\partial \rho}
    +
    \frac{1}{\rho^2}
    \frac{\partial^2 V}{\partial \phi^2}
    &=
    0
\end{align}
Using separation of variables,
\begin{align}
    V\left( \rho, \phi \right)
    &=
    R\left( \rho \right)
    \Phi\left( \phi \right)
\end{align}
so the equation becomes separable
\begin{align}
    \frac{\rho^2}{R}
    \left\{ 
    \frac{d^2 R}{d\rho^2}
    +
    \frac{1}{\rho}
    \frac{dR}{d\rho}
    \right\}
    &=
    -\frac{1}{\Phi}
    \frac{d^2 \Phi}{d\phi^2}
    =\nu^2
\end{align}
where $\nu$ is a constant.
So we have the ODE
\begin{align}
    \frac{d^2\Phi}{d\phi^2}
    +
    \nu^2 \Phi
    &= 0
\end{align}
which has the solution
\begin{align}
    \Phi
    &=
    A \sin\left( \nu \phi \right)
    +
    B \cos\left( \nu \phi \right)
\end{align}
If $\phi$ runs from $0$ to $2\pi$,
$\Phi(\phi) = \Phi\left( \phi + 2\pi \right)$,
which means $\nu$ must be an integer.

Now for the other equation.
\begin{align}
    \frac{d^2 R}{d\rho^2}
    +
    \frac{1}{\rho}
    \frac{dR}{d\rho}
    -
    \frac{\nu^2}{\rho^2}R
    &=
    0
\end{align}
So the solution should be something like $R\sim \rho^\nu$
and $R\sim \rho^{-\nu}$.
There will be a homework problem with power law but imaginary power,
but for this problem it's not relevant now.
But what if $\nu=0$ so there are no two independent solutions?
Well in the case $\nu=0$,
then we have the special case
$R\left( \rho \right) =$ constant and $R\left( \rho \right) = \log \rho$.

Now we can expand out the solution
\begin{align}
    V\left( \rho, \phi \right)
    &=
    a_0 + b_0 \log\rho
    +
    \sum_{n=1}^{\infty}
    a_n \rho^n \sin\left( n\phi + \alpha_n \right)
    +
    \sum_{n=1}^{\infty} b_n \frac{1}{\rho^n}
    \cos\left( n \phi + \beta_n \right)
\end{align}
and this is the general solution.
Note that the $\frac{1}{\rho^n}$ blows up at the origin,
so we must have $b_n=0$ for our problem.
Suppose we were looking for solutions outside the cylinder,
then in that case it wouldn't be.
But for our case,
$b_n=0$.
In fact, $b_0$ is also $0$ because the logarithm also blows up at $\rho=0$.

So then for our particular problem,
\begin{align}
    V\left( \rho, \phi \right)
    &=
    a_0
    +
    \sum_{n=1}^{\infty}
    a_n \rho^n \sin\left( n\phi + \alpha_n \right)
\end{align}
Now we have to determine these coefficients.
What we use to determine these coefficients is that we know
this is equal to $V_0(\phi)$ at $\rho = b$,
so it's kind of like a Fourier series with the sum of trigonometric functions.
First let's rewrite this as the standards Fourier series
\begin{align}
    V\left( \rho, \phi \right)
    &=
    \frac{1}{2} c_0
    +
    \sum_{n=1}^{\infty}
    \rho^n\left\{ 
    c_n \cos\left( n\phi \right)
    +
    d_n \sin\left( n\phi \right)
    \right\}
\end{align}
so it's been expanded in sine and cosine term.s.
At the surface of the cylinder $\rho=b$,
\begin{align}
    V\left( b, \phi \right)
    &=
    V_0
    +
    \frac{1}{2} b_0
    \sum_{n=1}^{\infty}
    b^n
    \left\{ 
    c_n \cos\left( n\phi \right)
    + d_n \sin\left( n\phi \right)
    \right\}
\end{align}
and now this really is exactly the form of a Fourier series,
so now you can invert it to find the $c_n$ and $d_n$ coefficients.
You can use the standard formula for Fourier series.
\begin{align}
    c_n &=
    \frac{1}{\pi}
    \frac{1}{b^n}
    \int_{0}^{2\pi}
    d\phi'\,
    V_0\left( \phi' \right)
    \cos\left( n\phi' \right)\\
    d_n &=
    \frac{1}{\pi}
    \frac{1}{b^n}
    \int_{0}^{2\pi}
    d\phi'
    V_0\left( \phi' \right)
    \sin\left( n\phi' \right)
\end{align}
and it turns out with a page of algebra and mathematical tricks,
it turns out you can resum this.
Go through the notes to see the trick,
and there is a homework problem where you have to use this trick.
The steps themselves are not very difficult,
just follow the in the notes.

\section{Fermi Gas}
\begin{align}
    n(k) &=
    \frac{1}{e^{\beta\left( E_k - \mu \right)} + 1}
\end{align}
The energy is
\begin{align}
    E_k &=
    \frac{1\hbar^2 k^2}{2m}
\end{align}
Consider $T=0$.
If you plot $n(k)$ vs $k$,
then there is a Fermi momentum
\begin{align}
    k_F &=
    \sqrt{2m \mu}/\hbar
\end{align}
and there is a Fermi energy $E_F = \hbar k_F$.
There is a transition region near the Fermi energy.

For $T>0$ there is a transition region.
Near the transition region,
we consider where the exponential deviates significantly from 1.
\begin{align}
    \left|E_k - \mu\right| \approx 1
\end{align}
and then
\begin{align}
    \frac{\hbar^2 k^2}{2m}
    -
    \frac{\hbar^2 k_F^2}{2m}
    &=
    \frac{\hbar^2}{2m}
    \underbrace{
    \left( k + k_F \right)
    }_{\approx 2k_F}
    \left( k - k_F \right)\\
    &=
    \frac{\hbar^2}{m}k_F (k - k_F)
\end{align}
and then
\begin{align}
    \left( k - k_F \right)
    &\approx
    \frac{\mu}{k_F \hbar^2} k_B T\\
    &=
    \frac{1}{\hbar U_F} k_B T
\end{align}

The name of this sphere is the Fermi sphere.
The surface is the Fermi surface.

I'm going to apologize and do something trivial in front of you.
You're going to say it's boring,
but I promise you you're going to learn the rest of your life.
To leading order,
what you do when the temperature is exactly zero,
you should be able to do at any moment like that.

\subsection{$T=0$ limit}
I can find the density using an obvious formula for the number of particles.
\begin{align}
    N &=
    gV \int \frac{d^{3}k}{\left( 2\pi \right)^3}
    \frac{1}{e^{\beta\left( E_k - \mu \right)} + 1}
\end{align}
At zero temperature,
this approaches a step function.
\begin{align}
    n(k) \to \theta(k_F k _)
\end{align}
In that case,
the integral is going to be trivial.
\begin{align}
    N &=
    gV \frac{1}{8\pi^3}
    \int_{0}^{k_F} dk\, 4\pi k^2
\end{align}
in spherical integrals.
And then
\begin{align}
    N &= \frac{gV k_F^3}{6\pi^2}
\end{align}
and remember $k_F$ can be written in terms of the chemical potential $\mu$.
That's how you use the canonical ensemble,
you calculate $N$ in terms of $\mu$,
and then you invert it to find
\begin{align}
    k_F &= \left( \frac{6\pi^2 N}{g V} \right)^{1/3}
\end{align}
and now I can calculate anything I want from $k_F$.
To get the energy from the partition function is a bit of a pain,
but you don't have to,
you can just sum over the single-particle states,
but only on the occupied states
\begin{align}
    E &=
    gV \int^{k_F} \frac{d^3k}{\left( 2\pi \right)^3}
    \frac{\hbar^2 k^2}{2m}\\
    &=
    \frac{gV \hbar^2}{2m}
    \frac{1}{8\pi^3}
    \int_{0}^{k_F} dk\, k^2 k^2\\
    &=
    \frac{gV\hbar^2}{20\pi^2} \frac{k_F^5}{m}\\
    &=
    \frac{gV \hbar^2}{20\pi^2 m}
    \left( \frac{6\pi^2}{g} \frac{N}{V} \right)^{5/3}
\end{align}
You could do this in your head,
the fact it depends on $\prop N^{5/3}$,
if you think the right way it should be obvious.
If it were in 2 dimensions,
how would it change?
It would be $d^2k = 2\pi k\,dk$.
What if it had relativistic particles?
Just replace $E_k = ck$.

Now let's work out the pressure. 
You can do this yourself
\begin{align}
    P &=
    \left.\frac{\partial E}{\partial V}\right|_{T,N}\\
    &\sim
    \frac{1}{V^{5/3}}
\end{align}
so the pressure gets larger and larger as you compress.
But where does the pressure come from?
It's at zero temperature.
They are not moving because of thermal energy.
It's because of the exclusion principle.
There is  pressure exerted on thew alls.
The higher you are in energy,
the faster they are colliding,
and the higher the pressure.
Here you have a bunch of them,
and you're taking the average.

This has a name,
and it's good for you to know,
it's called the
\emph{degeneracy pressure}.
But the way,
the physics at low temperature of a Fermi gas
is called a degenerate gas.

The story is the following.
Stars are a big chunk of mass,
that gravitate and collapse,
but they don't.
Some nuclear reactions happen,
which produce heat,
which produces a pressure outwards.
So there is a competition between hot gas outwards and gravitational pressure
inwards.
All that can be calculated,
not that difficult.

The problem is that the nuclear reactions require fusion of elements.
At some point you fuse everything you can,
you make iron,
which gives no more energy.
So more more heat is produced,
but it does produce radiation,
but that doesn't work.
Then there are two possible results.
If the star is very heavy,
it collapses fast into a black hole,
nothing to stop it.
Black holes are boring.
If the star is a little little,
or very light like the sun,
what happens is this matter collapse,
part of it spreads all over space as supernova.
But what's left is cold,
and because of the electrons,
it doesn't collapse.
It's held up by the degeneracy pressure,
and they're called white dwarfs.

You know why they are called white dwarfs?
They are masses of planets.

Everything you need to know to calculate it you know already.
So you should be ready to calculate what is the maximum mass that can be
sustained.

White dwarves boring.
Black holes boring.
There is an intermediate neutron star.


This was Chandrasekhar.
Nobody understands how he learnt th is stuff.
TO go from India to Cambridge,
you sen  few months on a ship,
and during the trip,
he figured this out.
He was a smart cookie.

To be at the forefront of physics with a creative idea,
surrounded by nobody.

I met the guy but it wasn't a great story though.
Very respectable guy.
There was a serration with friends.
It was closed.
This guy stands up an starts talking about black holes,
and at the end,
Chandrasekhar stands up and starts ranting against the guy,
says total nonsense.
The young guy was paralysed,
knew who he was and just looked at him.
At some point Chandrasekhar sits down,
and a group of people take care of him,
after that the whole audience was weird.
By the way,
you talk about black holes.
One story leads to another,
I'm sorry.

I have another friend,
and distinguished scientist but not like Chandrasekhar.
I was watching a seminar next to him.
Very senior person.c
In the end,
he said if you ever give a talk like that,
please shoot me.
I was wondering if I should talk about this one.

Chandrasekhar stood up for 10 minutes ranting,
it was really awkward.
I could go on, but no.

Hans Bethe.
You know he discovered the sun is powered by nuclear power.
I met him when he was almost 90.
One person stood up in a talk,
and there was a thing like this projector,
moving one inch at a time.
I said oh god,
then he puts a transparent slide upside own,
and he asks where is it.
The guy focuses on the transparent slide,
flipped it,
and proceeds to give the most beautiful talk I've ever seen.
No formalism,
just one idea after another.
Goes from the early days to the recent stuff.
There was discussion about whether they found a black hole or not.
The guy was top of his game at 80 or 90,and I was like darn what about me?
But there's not going to be good talks again.

I'll give you some bullets.
Let's finish the class early for you to study.

\section{Ideal Fermi Gas}
The potential is
\begin{align}
    \Omega\left( T, V, \mu \right)
    &=
    -k _B T g
    \int \frac{d^3k}{\left( 2\pi \right)^3}
    \ln\left[ 
    1
    +
    e^{-\beta\left( E_k - \mu \right)}
    \right]
\end{align}
so the density is
\begin{align}
    \frac{N}{V}
    &=
    -\frac{\partial}{\partial \mu}
    \Omega\\
    &=
    g \int \frac{d^3k}{\left( 2\pi \right)^3}
    \underbrace{\frac{1}{e^{\beta\left( E_k - \mu \right)} + 1}}_{n(k)}
\end{align}
so the energy density is
\begin{align}
    \frac{E}{V} &=
    g \int \frac{d^3k}{\left( 2\pi \right)^3}
    \frac{E_k}{e^{\beta\left( E_k - \mu \right)} + 1}
\end{align}
The Sommerfeld expansion is in $T/T_F$.
\begin{align}
    \int_{0}^{\infty} \, n(k) k^{2\alpha}
    &= I(\alpha)\\
    &=
    \left( \frac{2m}{\hbar^2} \right)^{\alpha - \frac{1}{2}}
    \frac{m}{\hbar^2}
    \int_{0}^{\infty} dE\,
    n(E) E^{\alpha - \frac{1}{2}}
\end{align}
and recalling
\begin{align}
    E &=
    \frac{\hbar^2 k^2}{2m}
\end{align}
and
\begin{align}
    dE &=
    \frac{\hbar^2 k}{m}dk
\end{align}
we get
\begin{align}
    \int_{0}^{\infty} dE\, n(E)
    E^{\alpha - \frac{1}{2}}
    &=
    n(E)
    \left.
    \frac{E^{\alpha - \frac{1}{2}}}{\alpha + \frac{1}{2}}
    \right|_{0}^{\infty}
    -
    \int_{0}^{\infty} dE\,
    E^{\alpha + \frac{1}{2}}
    \frac{dn(E)}{dE}\\
    &=
    -\beta \int_{0}^{\infty}\left[ 
    \mu^{\alpha + \frac{1}{2}}
    +
    \frac{E - \mu}{\beta}
    \mu^{\alpha + \frac{1}{2}}
    +
    \frac{\left( E - \mu \right)^2}{2\beta^2}
    \mu^{\alpha - \frac{1}{2}}
    \left( \alpha + \frac{1}{2} \right)
    \left( \alpha - \frac{1}{2} \right)
    +
    \cdots
    \right]
    \frac{e^{\beta\left( E - \mu \right)}}{
    \left( e^{\beta\left( E - \mu \right)} + 1 \right)^2
    }\\
    &=
    \frac{1}{\alpha + \frac{1}{2}}
    \int_{-\beta \mu}^{\infty} dx\,
    \left[ 
    \mu^{\alpha + \frac{1}{2}}
    +
    \mu^{\alpha - \frac{1}{2}}
    \left( \alpha + \frac{1}{2} \right)
    x
    +
    \frac{\mu^{\alpha - \frac{1}{2}}}{2\beta^2}
    \left( \alpha + \frac{1}{2} \right)
    \left( \alpha - \frac{1}{2} \right)
    x^2
    +
    \cdots
    \right]
    \frac{e^{x}}{\left( e^x + 1 \right)^2}
\end{align}
where we have defined $x = \beta\left( E - \mu \right)$.
Then you can use the fact that
\begin{align}
    \int_{-\inty}^{\infty} dx\,
    \frac{x^n e^x}{\left( e^x + 1 \right)^2} = 0
\end{align}
for odd $n$
and
\begin{align}
    \int_{-\infty}^{\infty} dx\,
    \frac{x^{n} e^x}{\left( e^x + 1 \right)^2}
    =
    \begin{cases}
        1 & \text{if } n = 0\\
        \frac{\pi^2}{3} & \text{if } n = 2\\
        \vdots
    \end{cases}
\end{align}
Sommerfeld was probably the greatest physicist of his age,
with Heisenberg and Pauli,
all people of the same caliber.
They were just good at it.
The circle around him,
he was good,
prominent,
knew what the latest thing was.
People around him were at the forefront of science.
But he was an old man like myself.
Great physicists,
but other his best years when QM came around.
So his students got famous.
Different from today.
These days if you discover,
your advisor gets famous.
Generous man,
etc.
He wrote books,
one of them studied the picture of atoms,
using not quantum mechanics,
but using old Bohr quantum theory.
So he did things,
like relativistic corrections of the helium atom
wrote a book this thick.
By the time his book had finished,
his student Heisenberg invented quantum mechanics,
and found a correct way to calculate these things.
There's no translation into English because it's irrelevant,
but it's funny to see written in German.
HE wrote pedagogical books,
on electromagnetism, and stuff like that,
more or less the level of this class.
They're still readable today 100 years later.
You can get it from the library,
there's little development after that.

This is something for you to do at home.
Before we found that
\begin{align}
    N &\propto
    g V
    \frac{k_F^2}{6\pi^2}
\end{align}
and this is just at shorthand
\begin{align}
    k_F &= \sqrt{2M \mu}
\end{align}
But now we have a correction to this
\begin{align}
    N &\propto
    g V
    \frac{k_F^2}{6\pi^2}
    \left[ 
    1
    +
    \pi^2
    \left( \frac{k_B T}{\mu} \right)^2
    + \cdots
    \right]
\end{align}
This is hard to invert to get $\mu$ as a function of $N$.
People in my class are smart.
They understand that this is an approximation,
so this is valid only if this term is much smaller than that,
and $\mu$ is approximately given by this.
There's no point in getting the exact $\mu$ here,
the corrections are higher order terms.
You can exercise this in the homework.
You're going to invert it perturbatively.
Same thing for the other expressions.
\begin{align}
    E &=
    gV
    \frac{\hbar^2}{2m}
    k_F^5
    \frac{2^{3/2}}{5\pi^2}
    \left[ 
    1
    +
    5\pi^2
    \left( \frac{k_B T}{\mu} \right)^2
    +
    \cdots
    \right]
\end{align}
A little physical interpretation here would be good.

One more thing.
The specific heat at fixed volume.
\begin{align}
    c_V &=
    \frac{1}{N}
    \left.
    \frac{\partial E}{\partial T}
    \right|_{V,N}\\
    &\approx
    \frac{\pi^2}{2} \frac{k_B T}{T_F}
    + \cdots
\end{align}
Remember the specific heat of a classical ideal gas.
It was a constant?

With phonons?
It was $T^3$.

You might say of course,
because you can calculate in your head.
Do you know where the powers of $T$ come from?

What about black body?
Same thing,
because phonons and photons are like the same thing.

But Fermi gas has different behaviour.
It goes to zero at zero temperature,
unlike a classical gas.

So imagine a Fermi surface with radius $k_F$.
At zero temperature,
you fill up every level up to $k_F = \sqrt{2m\mu}$.
But what happens at finite temperature?

Now consider increasing the temperature by a little bit.
The distribution $n(k)$ goes from a sharp step function
to a smooth step.
I created a hole and a particle.
That is called a particle and hole.

The low-lying corresponds to this particle-hole excitation.
Take the energy of the Fermi surface and increase it slightly.

But particles deep in the Fermi surface,
cannot increase their energy by a lot,
they don't contribute much to the partition function.
Only states that are relatively close to the Fermi surface are going to be
promoted or excited.
So I'm going after the region around the Fermi surface,
where the occupation number if neither zero or 1.

The radius of this sphere,
let's calculate the volume of this spherical shell in momentum space.
Do you think the radius is going to be $\sim k_F^2 k_B T$?
No,
not really.
The correction to the energy is order $(k_BT/\mu^2$,
so this thickness is not proportional to $k_BT$,
but proportional to $(k_B T)^2$.

So the thickness of this region is 0
for zero temperature,
but it's proportional to $(k_B T)^2$
for small finite temperature.

At small temperature,
the excitations are particle-hole excitations,
and the thickness of the Fermi surface region is quadratic.

One way to think about why it's quadratic,
you can think of the step smoothing out,
because the particle part and the hole part cancel out to first order
and you're left with

Years later the only thing you'll remember is that the Fermi surface is fuzzy at
finite temperature and its thickness is going to be proportional to $T^2$.
\begin{align}
    E &\approx
    E_{T=0}
    +
    \#
    k_F^2 (k_BT)^2
\end{align}

We have a month of class left.
The syllabus is crazy.
It looks like most of you survived so far.
Pat yourself on the back a little bit.

\end{document}
