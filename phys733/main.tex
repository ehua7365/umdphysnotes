\documentclass{article}
\usepackage[utf8]{inputenc}
\usepackage{hyperref}
\usepackage{amsmath}
\usepackage{amsthm}
In conclusion, we have introduced a new topological quantum error correcting
code on a three-dimensional lattice that is a generalization
\usepackage{amssymb}
\usepackage{amsfonts}
\usepackage{xcolor}
\usepackage{physics}
\usepackage[
    a4paper,bindingoffset=0.2in, left=1in,right=1in,top=1in,bottom=1in,
    footskip=.25in
]{geometry}
\usepackage{svg}

\newtheorem{theorem}{Theorem}
\newtheorem{theorem}{Theorem}
\newtheorem{question}{Question}
\newtheorem{conjecture}{Conjecture}
\newtheorem{task}{Task}
\newtheorem{axiom}{Axiom}
\newtheorem{proposition}{Proposition}
\newcommand{\overbar}[1]{\mkern 1.5mu\overline{\mkern-1.5mu#1\mkern-1.5mu}\mkern 1.5mu}


\title{PHYS733 Topological phases of matter}
\author{Eric Huang}
\date{August 2021}

\begin{document}
\maketitle

\begin{abstract}
\end{abstract}

\section{Lecture 1}
The most famous place topology enters physics is in the discussion of
topological defects in ordered media.
For example you can have a vortex in a superfluid where a superconductor with an
order parameter winds in space.
The winding of some smooth order parameter.
Just to mention some examples, this would be vortices in superconductors.
Normal fluids are unordered media.
There are things called skyrmions in magnets, which if you think of things as
non-trivial windings along a sphere.

1. You have dislocations and disclinations in crystals and so on.


The mathematics that goes into describing this is homotopy theory.
You have some order, some symmetry that gets broken into some subgroup, then
it winds in a non-trivial way.
Mermin, RMP is a good reading.

These defects so far can be treated classically and is a well-trodden topic but
we're not going to focus on this.


Another example is the topological configurations of gauge fields
(magnetic monopoles) here the maths gets more sophisticated.
Gauge fields are connections on fibre bundles and fibre bundle has non-trivial
topology.
To really understand you need to study the topology of fibre bundles,
which is is mathematics.

You may have heard monopoles don't exist, like people have been looking for
magnetic monopoles for a long time, but they do exist in emergent gauge field
descriptions and you have monopoles of emergent gauge fields and those are
important.
They have even been experimentally observed in spin ice.
You can google to learn more about spin ice.

3.
Another place is that once you have non-trivial defect configurations like
solitons, if you have fermions living in your system, like electrons, then the
spectrumo f hte Hamiton might have zero-eenrgy states bound to the position of
these defects.
This subject is called zero modes of solitons and monopoles.
Here to understand this,
For example ou could have am assive field whre the mass changes sign.

So you have $x$ space, and $m$ mass of the field. It could go from negative to
positive, but as you cross zero there is some mode localized that is pinned to
exactly zero energy.
Understanding this brings us into the mathematics of index theory.


4. Topological band theory.
This is something the class is going to touch on a lot.
Here hte idea is that ify ou have a wave or a single particle in aperiodic
medium, hten thant wave or particle is going to acquire a dispersion relation.
As a function of momentum $k$ you're going to have enrgy bands $E(k)$.
Each band can have smoe topological character associated iwth them.
At each point in momentum space, you have states called Bloch states
$u_{f,k}(r)$ where $f$ is the band index and $k$ is the momentum.
As you track how it varies in momentum, that phase winds around in non-trivial
ways and that gives rise to topolgoical band theory.
This $u$ defines a vector bundle that can have non-trivial topology.
I don't expect you to understand what I'm talking about,
just want ot to give a broad interview.

THe most famous is the TKNN number, also called the \emph{Chern number}
which is an integer you can assign to every band once you know how to understand
how it behaves in momentum space.
This is a striking phenomenon and leads to the integer quanutm Hall efect.
When you fully fill a band a get a band instlualtion,
you get an electircal induslator, buti t has a Hall conductance, which si
atransverse response.
Iti s exactly quantized $\sigma_H= C e^2/h$.
$C$ is the Chenr number.

It's just about bands, so it's really something that describes single-particle physics.
These depict the one-elctonc states, where every electron can fill every
electron states.
You fill partilly or fuully any one of these bands.

The maths that describes this is the maths that came up early.
It's just the topology of figure bundles.
If you want a full understanding of this subject, you need to understand
$K$-theory.

Single-particle means the physics of interactions are not included in this
problem.
Also index theory shows up in this problem.

Q: What happens if you add interactions?

Sometimes the topological numbers are no longer meaningful if you add
interactions.
Then there are some cases where the topological invariant still has meaning with
interactions, but then we need new ways to describe it.
It could have different topological quantum numbers identified under
interactions, there's a possibility there are topological properties with no
analogues in theband theory too.
Both can happen.

5. What are all the possible phases of matter?
Two phases of matter are different if I tune the paramers o the sytem, I cannot
continuous go from one phase to another wihtout a phase transiton.
A phase transition is a singularity in the free energy density.

Let's try to classify all phases of matter.
Immediately when you pose htis question, topolgoy enters.
Imagine the space of all possible equilibrium systems let's just call it $X$
which is at all temperatures..
THey have some spatial dimension $d$ and some symmetyr $G$.
Imagine tunning the parameters of the system keeping these $d,G$ fixed.
As you parameters, you might encounter phase transition when the free energy
necessarily goes through a singularity.
In this vast space we're trying to imagine, you can remove all subspaces where
there's a phase transition and that gives us a new space $\tilde{X}$.
The problem of classifying spaces of matter is the problem of classifying all
the \emph{connected components} of $\tilde{X}$.
So the connected components of a space $\pi_0(\tilde{X})$ is called the zeroth
homotopy space.
Even the basic question of asking what the phases of matter are is immediate a
topology question asking for the connected components of a vast space we're
tyring to imagine.

6. Characterising topological phases of matter.
Most of this class is about this.
This class is mainnig about number 4 and 6.
When you htink about what are difrerntk inds of ordered phases.
For a lot of the 20th centruy, pepole thought you have symmetries spontaneously
broken, ask what local order parameters are from $G$ to some subgroup.
Then ask the effective free energy by expanding in gradients of the order
parmtere, then describet he long-wavelength properties by looking at smooth
breaking symmetries.
So we have the Ginzburg-Landau theory of symmetry breaking.
This is a powerful theory.
The heat capacity of a crystal.
Fluctations of atoms are phonons, then take expansions in the strain field and
you get it.
Fluids, magnets, just identify the local order parameter and write a GZ thoery.

In hte 1970s, Wegner studied $\mathbb{Z}_2$ lattice gauage theory.

He showed you can have phases with no local order parameter.
You need something more subtle.
This leads us to topological phases of matter.
In one sentence, a topolgoical phase of matter is the properties of the phase
canot be signalled by a local order parameter.
To understand it you need to look at non-local orperats and how they behave in
the ground state.

Q: Why ground state?
Good question.
Implicitly, when I talk about topological phases of matter, I'm talking about
zero-temperature, whichi shte ground state,
And also low-energy exciatitons fo the gorudn astate.
Some phases are distinct at zero-temperature, but at finite temperature the
distinction may melt away.

In topological phases of matter, the description you need is very intricate, has
a lot of fascinating mathematical structure, structure that is a more modern
topic in mathematics, not some 1940s maths we came up with before.

So instead, what you need are topological quantum numbers.
You insert defects which extend in various directions and dimensions and they
can have interesting topological properties.
Ultimately there is a topological quantum field theory that magically emerges
(TQFT) to describe the universal robust properties of the system.
The maths that goes into this is a very different kind of math.
It goes under the name of quantum topology, TQFT and Category Theory.
These topics are frontiers in pure mathematics as awell.

For me aI ifind it fascinating.
We have intricate math struytures called TQFTs, and somehow there are phases of
matter where TQFTs magically arise for describing phases of that system.

Q: IS TQFT is the same sense of GZ theory for long wavelength or is it a
differnet sense?
y
It's a similar sense, but it depends what you eman by similar.
GZ is useful for dynmiacal porperties and critical exponents.
TQFT is not useful for understanding dynamics, but it's useful for undestanding
braiding paritcles, so more like kinematics than dynamics.


I'm going to briefly mention a few others.

When you hear peopel talking about topological phases of matter, they're talking
about topological band thoery or TQFT.

7.Toplogical pumps.
You have some Hamltoinian that depends on some parmater.
This hamiltonian could be periodic $H(\tau + T) = H(\tau)$.
Imagine slowly changing $H$ so that by the around it comes back you're back to
where it started, but something about the systme has changed.
As you go through this system, you can pump charge across the system for
example.
The most famous is the integer quantum Hall state on the cylinder.
As you insert flux $\Phi$, flux of $0$ is the same as flux $2\pi$.
Your wave function will slowly move to the right.
Byt he time the flux goes to $2\pi$, but htere is a net winding that pumps
charge from left to right.
This is called the \emph{Thouless pump}.
It arises in the study of the integer quantum Hall effect on a cylinder.

Recently, there has been some work where the idea of a toological pump is vastly
generalised for QFT.
There's a paper called Anomalies in the couplling constants that abstracts this
notion.

8. Effective action.
Your effective action depends on some set of fields, which has topological terms
in the effective action.
.
9. 't Hooft anomaly.
Every QFT has one.
This is invertible TQFT, cobordisms and cohomology.

10. 
Metals have a Fermi surface that separates occupied and unoccupied states.
That surface can have a topology.
In a 3D metal, the surface is a closed 2D surface,
so it could be a sphere, a torus and so on.
In apritcular, there' s an invariant called the Euler characteristic that
distinguishes these possibilities.
THere's a cool paper that says the Euler charctrsic shows up in ballistic
conductance.
No disorder, just pure sytem, add contacts, measure conductance, and it turns
our the euler characteristic turns up in that measurement.
It's well-knonw in 1D but this paper generalizes that.

11. Early attempt to 
This is a failed attempt.
Lord Kelvin and Pepter Tate attempted to think of atoms as knots in the aether.
Cool idea, because as knot is stable and has a well-defined character.
But it went down the drain when the discovered there is no aether.

Another places is when we think of world-lines of particles in 3D spacetimes.
Depending on the topology of knots and links.
It could depend only on the topological invariance of knots and links.
It comes in up in 2+1D topological quantum field thoereis.

In higher dimensions, we need membranes, world-volumes and more non-trivial
kinds of knotting.

That's it for my overview.

Any other ideas?

Studnet: Feynman diagrams planar and non-planer.

I hope the lesson is that topology shows up everywhere.
The mathematics can be veyr differnet, the actual thing you're tlaking about
cane very differnt, so you have to clear.
It's maiing about characterirising toplogical phases of matter using TQFT.
Many topological phases of matter can be understood simply in terms of
topological band theory.


\subsection{Local quantum systems}
When startin fof a sstudent learning QM and may=body QM, at some point it may
occur to you a Hamlitoinin is just a big matrix and why is it differnt ot Linaer
algebra.
The fundamental didferencis the concept of locality.
We're intersted in systems with some notion of locality, often geometry.
W'ere interested in sum of linear terms.
Wihtout locality we hayve nothing to say.
It's locality in a many-body system.
What are some basic properties local quantum systems have we can demonstrate on
general grounds.

The first thing is the Hilbert space and hte locla tensor product decomposition.
We have some quantum ssytme with some hilbert space.
We're interested in Hilbert spaces that factorize
\begin{align}
    \mathcal{H} = \bigotimes_{i=1}^{N}\mathcal{H}_i
\end{align}
where $i$ labels sites on a graph $\Lambda$, for example a lattice.
A general graph is useful.
$\mathcal{H}_i$ is the local Hilbert space, which might be finite-dimensional,
in which case it's isomorphic to $\mathbb{C}^k$.
This could be useful for a spin system for example,
so spins that can be up or down, 
or hardcore bosons, I can have bosons with hardcore repulsion.
Or it could be an infinite-dimensional space like a harmonic oscillator at that
particular site.

If we have fermions, then this $\mathcal{H}_i$ will be a \emph{graded vector
space}
which means we have to specifiy which staes have odd or even fermion numbers.
And $\mathcal{H}$ is a fermionic Fock space.
Fermions don't commute, so you have to be careful to take care of the
non-locality and anti-commutation.

I immediately have a system essentially discrete.
This is not a continuum system.
To the extent we're interested in continuum systems,
we're only interested in long-wavelength approximation.
It's hard to directly define this tensor product in the continuum,
not sure what the best way is.
If you want to do it you wind up in the world of operator algebras.

We'll be interested in continuum systems that have UV completion on $\Lambda$
iwth $\mathcal{H}$.
This menas if I go to really small scales, this is what my system actually looks
like, but if I look at low-wavelneght regimes, this continuum systme is
sufficient to describe.
Often we write down fermions in the conituum like quadaratic dispersion or
llinear band structure.
These all arise from long-wavelenth approximations of fermions hopping on a
lattice.y
That's kind of what we have in mind.

Imporatntly, not all quanutm systmes have UV completion, not all QM systems
have a tensor product factorsiation.

Q: Any examples?

Yes, not all quanutum systems factorize.c
An example is pure gauge thoery, like Maxwell theory with no electric charges,
just electric and magnetic fields.
There's Gauss law constraint where div is 0.
Such a pure gauge theory does not have all local tensor product decomposition,
because there is a constraint that has to be satisfied at every point.
You could put gauge theory on a lattice to get lattice gauge theory.
In general you can have constrained quantum systems where the Hilbert space does
not factorize into such a nice form.

This leads to the question of \emph{emergeability}:
Which QFTs can arise as long-wavelength approximations to a system that has this
local tensor-product factorisation.
For exmaple, there's a lw that all guage thoeries can emerge from such systems
iwth local tensor factorisations. 
However, the guage thoery needs to have charges in every representation of the
guage group.
There's a notion of completeness for that guage theory to be emergable.
You can think of this as emergeable from qubits if you like.
What's a qubit?
Just a local 2-state system.
This is just saying my Hilbert space is just a bunch of qubits.
The questions is whatk ind of QFTs can emerge from qubit models.
We know some QG models can arise frmo qubit models.
QG models can arise from super-Yang Mills theory and that can arise from some
interacting qubit thoeyr.

We don't know how to write the standard mdoel starting from this construction,
because it has chiral fermions, neutrinos and we don't know how to put that on a
discrieitsed structure.
Some people believei t can be done, but no one has figured it out.

So the questions is: How to get standard model of particle physics.

This questions has received attention later with TQFTs and anomalies and it has
been useful thinking about hits question.

Q: Is there a verison of the tandard model thatl ooksl ike the standard model
without neutrinos?

Yes you can put QCD on a lattice.
You can just put EM on a lattice.
But if you want everything including neutrinos,
then you have to deal with chirality.
It's sometimes inpossible to get on a lattice, but sometimes possible but very
hard.

Q: Cna you do QCD wiht weka force?

The weak force is the issue because it has neutrinos.

THer'es 2 statements: Can you usefully approximate the standard mdoel by
numerical methods? Yes but it's not honestly done.
You could do domain-wall construction, with a guage field that leaks ot hihger
dimensions.
There are tricks you cna play to get numerics.
But if want non-perturbative approximations, then no.

The systems that have some chiral fermions are on the bounadries of some
systems.
You don't want the guage field to leak out ot higher dimensiosn.

Q: Is there a relation between renormalzabilit and emergeability.

No. Don't know.

Emerge ability is a new word.
It's natural from a condensed matter perspective.
If you're big into everything comes from qubits,
then this is exactly what you're interested in.

Let me define now local Hamiltonian.

Before I do that, we have a graph $\Lambda$.
We need geometry so we need to define a distnace function dist$(i,j)$
that is symmetric and satisfies the triangle inequlaity.
The most useful thing to do do is say the distance between two points is hte
lenght of hte minimum path.

If you have a set $Z\subset \Lambda$,
then the diameter of $Z$ is $\mathrm{diam}(Z)=\max_{i,j}\mathrm{dist}(i, j)$.
Then
\begin{align}
    H = \sum_{Z\subset\Lambda} h_Z
\end{align}
where each $h_Z$ only has support on $Z$.

We're interested in geometric locality.

Q: Do we need $Z$ to be finite diameter?

Well we're interested in $N\to\infty$.
I'm not done yet.
y
Geometric locality means $|h_Z|$ decays fast enough.

\begin{align}
    |h_Z| =
    \begin{cases}
        0 & \text{if }\mathrm{diam}(Z) > k\\
        C e^{-\mu \mathrm{diam}(Z)}\\
        {(C/\mathrm{diam}(Z))}^\alpha
    \end{cases}
\end{align}

Power laws do occur, but often you have screening which essentially makes it
exponential.

Q: What is hte norm of an operator?
\begin{align}
    |\mathcal{O}| = \max_{\ket{\psi}} |\mathcal{O}\ket{\psi}|
\end{align}
If $\mathcal{O}$ is just a matrix acting on some finite-dimensional space, them
this is the maximuum eigenvalue of $\mathcal{O}$.

It turns out this isn't everything we want in the most general case.

There's something useful to add.
$\Lambda$ is some general graph,
but we do not want something with a single site talking to an extensive number
of sites,
because that could be pathological.
What we want is that every site is only involved in a few number of terms in the
Hamiltonian.
that's an additional thing.

For each $i$,
\begin{align}
    \sum_{X|i\in X} ||h_X|| \le C
\end{align}
so a single site cannot be involved in too many terms.

If you're interstedi nin  QEC, you're interested in this too.
If it's just a regular lattice like a square lattice, this is going to be
naturally bounded.

It's useful to have in mind that we can relax geometric locality and instead
just htink about \emph{sparsity}.
Here we could say that we want each term in the Hamiltonian to have low weight.
So we say that $||h_Z||\le Ce^{-|Z|\mu}$ where $|Z|$ is hte cardinality of hte
set.
We just want it to be a sum of low-weight terms,
that $H$ only has support over a low number of sites,
but we don't care where the sites are.
And we also have
\begin{align}
    \sum_{X|i\in X} \le C.
\end{align}

Sparsity shows up in the Sachdev-Ye-Kitaev model.
This mdoel is
\begin{align}
    H = \sum_{ikl} J_{j,k,l} \gamma_i\gamma_j\gamma_k\gamma_l
\end{align}
where the sum is random coupling.
They're low weight, only 4, but there's no geometric locality.
This actually faiils that constartin, but there's naother version that does.


Anohter example is finite-rate quantum error correcting codes.
Where you encounter Hamiltonains with no geometric locality,
but they do have this sparsity term.

Q: SYK have support on 4 sites, how is that not local?

Each term could be a random set of 4 sites.
I don't care about what's far and close.

Some basic properties about local Hamiltoanins.
Lee-Robinson bound.
Even though this is not relativistic,
there is an effective light cone, where $x=_{LR} t$,
the correlations will only be appreciatble inside the light code,
and will decay exponentially outside the light cone.
So there's a light cone even if there's no relativity.
This Lee-Robinson bound bounds the spread of information propagation in a local
system.
The minute you define local, you can prove htis bound.
It's extrememly powerful and geenreal.

Gapped Hamiltonian implies finite correlation length.

Q: Where is the syllabus.

Do you know elms.umd.edu.

\section{2021-09-01 Lecture 2}
\section{Lieb-Robinson Bounds}
Lightcone picture.
Last time
\begin{align}
    \| h_X \| &\le C e^{-\mu \mathrm{diam}(X)}\\
    \sum_{X| i\in X}\| h_X \| &\le \alpha
\end{align}
\begin{theorem}[Hastings]
    Suppose for all sites $i$
    \begin{align}
        \sum_{X | i\in X}
        || h_X ||
        |X|
        e^{\mu\mathrm{diam}(X)}
        \le s \le \infty
    \end{align}
    where $\mu$, $s$ are positive constants.
    Let $\mathcal{O}_X$ and $\mathcal{O}_Y$ be operators
    supported on sets $X,Y\subset\Lambda$.
    If $\mathrm{dist}(X, Y) \ge 0$
    then
    \begin{align}
        \|[\mathcal{O}_X(t), \mathcal{O}_Y]\|
        \le 2 \| \mathcal{O}_X \| \| \mathcal{O}_Y \| |X|
        e^{-\mu \mathrm{dist}(X, Y)}\left(
        e^{2s|t|} - 1
        \right).
    \end{align}
\end{theorem}
Note that the exponential term becomes
\begin{align}
    e^{-\mu(\mathrm{dist}(X, Y) - V_{LR}|t|)}
\end{align}
becomes exponentially small if $\mathrm{dist}(X, Y) > V_{LR}|t|$
where $V_{LR}=2s/\mu$.

Consider how an operator $\mathcal{O}_A^l(t)$ over some
region $A$ evolves over time.
We're going to trace out everything in $S$, which is the region outside $A$
offset by some $l$.
\begin{align}
    \mathcal{O}_A^l(t) := \frac{1}{\Tr \mathbf{1}}
    \left[
        \Tr_S \mathcal{O}_A(t)
    \right]\times \mathbf{1}_S.
\end{align}
Then
\begin{align}
    \left\|
        \mathcal{O}_A(t) - \mathcal{O}^l_A(t)
    \right\|
    \le
    C \|\mathcal{O}_A\| |A|
    e^{-\mu(l - V_{LR}|t|)}
    \left(
        1 - e^{-2s|t|}
    \right).
\end{align}

The distance between two sets is defined as
\begin{align}
    \mathrm{dist}(X, Y) = \min_{i\in X, j\in Y}
    \mathrm{dist}(i, j)
\end{align}
the minimum possible distance.
It doesn't matter which distance you use so long as it satisfies the assumption.

Note that the existence of some characteristic velocity is not that surprising
because if you had some simple Hamiltonian
\begin{align}
    H = \sum_{\langle i, j \rangle}
    J_{ij} \vec{S}_i\cdot \vec{S}_j
\end{align}
there is a characteristic time scale $\tau = \hbar/J$,
and hence some velocity scale $v_0 = a/\mathcal{t} = aJ/\hbar$.
The interesting thing about the LB bound is that it is emergent.

I'm interested in how correlations can propagate in a certain temperature.
In some states, correlations may propagate at different velocities.
So this velocity can be temperature and state-dependent.

Hence people define the \emph{butterfly velocity}
which people define as a temperature-dependent and state-dependent velocity.
This is a topic of active research.
Brian Swingle studies this a lot.

This is completely general to local Hamiltonians.

\section{Gapped Hamiltonians}
Let's specialize and talk about gapped Hamiltonians.
A gapped Hamiltonian has a spectrum that looks like the following.

I could have multiple degenerate ground states.
And then I would have a finite gap to all the other excited states.
We want $\Delta E$ to be finite in the thermodynamic limit.
This is the limit where $N=|\Lambda|\to\infty$.

It's also useful to think about a family of Hamiltonians that act on
increasingly larger systems.
Looking at the sequence of spectra, in the limit, I want there to be a finite
energy gap and $q$ degenerate ground states.
The $N\to\infty$ limit is important because every finite size system has finite
energy gaps and so the notion of gapped versus gapless is only meaningful in the
infinite limit.

\begin{question}
    Why is $N\to\infty$ the thermodynamic limit?
\end{question}
We're interested in infinitely large lattices.
We call it thermodynamic limit because in that limit thermodynamics applies.

Once we have a gapped Hamiltonian, we can talk about the important notion of
\emph{correlation length}.
If we have a finite energy gap $\Delta E$, by the uncertainty principle,
this defines a timescale $\Delta t=\hbar/\Delta E$.
So we have a characteristic time and using the Lieb-Robinson velocity we can
get a characteristic length scale
\begin{align}
    \xi = V_{LR}\Delta t = \hbar V_{LR}/\Delta E
\end{align}
that governs the exponential decay of correlations in a gapped state.
Of course the interactions must decay exponentially too.
If interactions decay as a power law correlations will also decay as a power
law.
I usually talk about correlations which decay exponentially in space.

A prerequisite for this to be useful is that the interactions must decay
exponentially.
It's not only interactions that determine this length scale,
it could by properties of the graph.

\begin{question}
    Is it useful to think of it as a thermal wavelength?
\end{question}
If I have a finite temperature system, correlations will decay in space and
time, which may be set by the thermal wave length,
but here we're at zero temperature.

I'm talking about spatial correlations in the ground state at zero temperature.
However at finite temperature, there is a length scale dependent on temperature
that could also be important for decay of correlations.
They're definitely not the same thing.

If you look at spatial correlations at finite temperature $T$,
then we have two time scales.
We have $\hbar\Delta E$ and $\hbar/k_B T$,
which gives us two length scales
$V_{LR}\hbar\Delta E$ and $V\hbar/k_B T$,
Which ever one is smaller is the important one.


\begin{question}
    How to get from the theorem to the correlation length?
\end{question}
It's a really non-trivial proof.

Let me state this precisely and that will limit confusion.

The theorem is the following.
There's been so much activity in this subject in the 2000s,
mainly from Hastings.
You'd think it's all established in the 60s with no for improvement,
but no.
\begin{theorem}[Hastings-Koma 2006]
    Let $\mathcal{O}_X,\mathcal{O}_Y$ be bosonic operators
    on sets $X$, $Y$.
    That means $[\mathcal{O}_X,\mathcal{O}_Y]=0$ if $\mathrm{dist}(X, Y) > 0$.
    Assume the system has a unique ground state with a gap $\Delta E$.
    For all sites $i$, we have
    \begin{align}
        \sum_{X| i\in X} \| h_X \| |X| e^{\mu\mathrm{diam}(X)} \le S
    \end{align}
    where $\mu,S$ are positive constants.
    Then the result is that
    \begin{align}
        | \langle \mathcal{O}_X \mathcal{O}_Y\rangle
        - \langle \mathcal{O}_X \rangle\langle \mathcal{O}_Y\rangle
        | \le
        C\| \mathcal{O}_X \| \|\mathcal{O}_Y\| e^{-l/\xi}
    \end{align}
    where $l=\mathrm{dist}(X, Y)$ and
    $\xi=\frac{2V_{LR}}{\Delta E} + \frac{1}{\mu}$.
\end{theorem}
The $1/\mu$ governs the decay of the interactions themselves.
Obviously if my interaction decays as $\frac{1}{\mu}$
then obviously the decays should decay like this too if it's dominant.

\begin{question}
    Is this a tightening of the Lieb-Robinson bound?
    There was already a length scale $1/\mu$, but now we have added another term
    to that length scale in some sense.
\end{question}
I don't think it's a tightening because this LR is just really giving a velocity
rather than length.
Hastings theorem doesn't say anything about spatial correlations.
How do these operators decay far from each other.
It's quite different, because if I set $t=0$ in the Hasting theorem then I get
zero for the commutator.

Hastings is about time, but Hastings-Koma is about equal time correlations.

\begin{question}
    The VLR is state-independent,
    but Hastings-Koma is for the ground state.
    Is that somehow strange?
\end{question}
It's not strange because $V_{LR}$ is the absolute fastest the correlations could
spread.
You could ask if there are velocities smaller than this which would reduce $C$
and cause this to fall of faster.
You could ask if there is a better bound that asks state-dependent velocities.
That's a good research question.

We've assumed a few things: bosonic operaors and ground state.
Both of these can be generalized.
You could genralise to $q$-degenerate ground states.
This term gets replaced by a projector onto the ground state subspace.
And if it's feromionic, you use anti-commutators.

I won't write the general case, but if you're intersted you could look it up.

There's also a generalization for power law decaying interactions as well,
but there's no well-defined correlation length.

\begin{question}
    When $L\to\infty$ what happens when $\Delta E\to 0$?
    How can I understand the correlation increases once I close the gap?
\end{question}
usually when the gap goes to zero the correlation gets bigger and bigger and it
diverges.
In that limit, this bound becomes useless.
Usually in a gapless system, the correlations decay albegraically as power law.
It doesn't always diverge, but it usually does.

In a gaplesss system, often the correaltion length $\xi\to\infty$.
Take a band insulator which has a finite gap.
At some point the gap will close and you have a Diract point where the
bands will cross.
You tune some parameters and you get some crossing.
The system is not gapped anymore and it's gappless.
As you decrease the energy gap, the correlation lenght gets bigger
and diverges at the point.
This theorem becomes uselss and there's no bound about how the correlations
decay.
But often what happens is that the correlations decay as a power law,
so like
\begin{align}
    \langle \mathcal{O}_X\mathcal{O}_Y\rangle
    \sim
    {\left[
        \frac{1}{\mathrm{dist}(X, Y)}
    \right]}^\alpha
\end{align}
This doesn't always happen.
The system could still be gapless, but there is still a localization lneght that
essentially bounds the correlation lenght,
but this thoerem is essentially useless.


\begin{question}
    It seems that you could get the distance of $X$ and $Y$ being zero without
    $X$ and $Y$ being the same set.
    I'm confused.
    Single-particle bosonic operaotrs, $i$ has to equal $j$.
\end{question}
Are you conofused about the definiton of Bosinci operaotr or what happens if
this is zero.
By definition, if supports don't overlap each other, that's whne the distance is
positive.
As soon as they overlap, it's non-trivial.

Often you have some microscopic energy scale for a problem,
and a charactersitic lenght scape such sa the lenth saale.
It's natural to think the correaltion enght is on the orer of the lattice
spacing.
Maybe it's 2 or 4 or whatever times.

Somtiems, this is not an emergent length scale, it could be 100 times the
spacing.
If it's 100 times the size, usually something intereseting is going on,
like being close to a critical point or transition.
The point is to remember that there's a notion of naturality in physics that
everything should be order 1.
That's what's going on here.
Often it's on the order of a lattice spacing, but it's really in general an
emergent length scale associated with an emergeent energy scale the energy gap.
Usually when the two scales are completely different, there's something special
going on.

\begin{question}
    Does the ratio of correaltion lenght with system size have anything
    significant.
\end{question}
Consider $\xi/L$.
This is an important quantitiy with finite-size scaling size analysis for
numerical simulations.
You need to pay attention if it's smaller or larger than the length scale.
That's the main thing really.

There's a natural question to ask about this theorem.
But no body has aske it so far.

\begin{question}
    Not the natural qeustion, but these are equal time correlations,
    but there's dependence on propagation of information.
    How does that connect.
\end{question}
I'm not sure.
Let me think wisely.
It's better to go through the proof to see how the velocity comes out.
I don't have intuition more than dimnesional analysis.
In the actual proof, there's some Fourier transform that takes $\Delta E$ to
time, so the Leib-Robinson velocity comes in there.
So maybe some intuition I kind of tend to think about but I'm not sure how
accurate it is.
That creates excitations, but they can only propgate in some charactersitic time
$1/\Delta E$ and how far they go depends on hte velocity and that kind of
depends away.
That's some very hand waving picture so take it with a grain of salt.

OK I htink this is the natural qeusiton.
\begin{question}
    Gapped Hamiltnoains have exponentailly decaying correlations.
    But does an exponentially decaying spatial correlation imply a gapped
    Hamiltonian?
\end{question}
No, the converse is not true.
There are states iwth short-range correaltions (exponentially decayin)
that are ground states of gapless Hamiltonians.
One example is take a Hamiltonian that is the usm of random terms.
You can prove rigorously such a Hamiltonian will typically gapless
and that hte ground state of the Hamiltonian will have exponentially decaying
correaltions.
At each point, you'll have a particle localized to these lcoal potentions,
but if you have a big enoguh system completely gapless.
There'll be some potetential when the ground state has that energy,
but wave functions decay exponnetially so so will be the correlation.


There are also translationally invariant examples,
but these examples tend to be contrived.
I tend to think they are not generic, but there are some examples that violate
the statement.
If you want to learn more I can point you people to learn from.
Any questions?


\begin{question}
    You mentioned sparse Hamiltonians as alternative to local Hamiltonians.
    There's no velocity because there's no distance,
    but is this statement true.
\end{question}
You an definitely ask it something like this.

There's been very little work outside of the SYK and QEC with finite encoding
rates, I don't know people wokring on sparse Hamiltonians.
There's the Swingle version of SYK in htis course.

\begin{question}
    For a gapped hamiltonian, we could genrate a gorund state.
    This theorem assumes a unique ground state, 
    so it assumes it's non-degenerate.
\end{question}
I did mention that this theorem can be generalized to the case of degenerate
ground states,
the case of fermionic operators
and the case of power law decay interactions.

If you want to generalize to degeerate ground states, with the projection into
the ground state.
There is a generalization.


\subsection{Localized Defects}
One thing useful about the existence of a finite $\xi$
is that you can define the notion of a localized defect.
The first case is a 0-dimensional defect.
Add a local potential that decays exponentially from some rate so
\begin{align}
    H = V(|r - r_0|).
\end{align}
That cases some kind of defect.
Outside the region, the state looks pretty much the same as the ground state.

We can define one-dimensional defects.
We add a term that is the sum over a.
A zero-dimensional defect is a point.
We could have 1-dimensional line defects.
\begin{align}
    H + \sum_{r_0\in\gamma} V(|r - r_0|).
\end{align}
I could define $p$-dimensional defects, where $p< d$.
We can talk about defects 

Topology of defects is about understanding what happens how these defects
interact with another when I glue and touch then,
and the phase.

The mathematical structure is not completely known,
but to understand these defects, we use $d$-category theory,
a hyper abstract definition people don't even know how to define yet.
It's a higher category, but mathematicians don't even have a definition.
But if you talk about 1-category or 2-category, then do you have definitions.
The properties of these defects are characterised by this $d$-category.
In this class we will use some simple examples.
This is a story very much in progress, not very fully understood.

\subsection{Quantum Phases}
Quantum really means zero-temperature phases.
I'm going to state what I actually mean by phases of gapped Hamiltonians.
This is the most popular definition.

\begin{itemize}
    \item Two gapped Hamiltonians $H_1$, $H_2$ realize the same $T=0$ (quantum)
        phase of matter if there exists an adiabatic path between them
        such that $\Delta E(\lambda)$ stays non-zero.
        That is, there is a path in Hamiltonian space $H(\lambda)$
        that starts at $H_1$ and ends at $H_2$.
\end{itemize}
The point is that the energy gap stays open as I go from one Hamiltonian to
another.
If I can do such a thing, we say the ground state of the two ground states are
the same phase of matter.
A consequence of this definition is that the ground state of the first
Hamiltonian can be adiabatically connected to the ground state of the other
Hamiltonian.
I can go down this path infinitely slowly that follows the adiabatic path.

There's a different definition that uses the ground state energy density
\begin{itemize}
    \item The ground state energy density $E_0(\lambda)/N$
        should be a smooth function of $\lambda$.
\end{itemize}
Along this path we never encounter a singularity in the ground state energy
density.
This is closer to the usual definition in phases of matter
where I look at the free energy density and a phase transition occurs when the
free energy has some singularity.
Here, we're looking at the ground state energy density.
Aside from the fact i's close to the standard definition,
you could also apply this for a gapless Hamiltonian.
So there's still a notion of whether two gapless Hamiltonians realize the same
phase of matter.

Definition 1 is only useful for gapped states of matter.

\begin{question}
    The second definition is only a distinction between phases of matter right?
\end{question}
This is the definition of when two phases are the same phase.
If they're not the same phase they're different phases.
If it's gapped and there's an energy phase.

I haven't defined what a topological phase is yet.
It's not clear exactly what definite we should use.
People use different definitions,
because there are different concepts.
For now take topological phase as gapped phase of matter.

\begin{question}
    For gapped Hamiltonians, are the two definitions equivalent.
\end{question}
I don't have a proof.

\begin{question}
    For def 1, if w ehave gapped system, if we close the gap, the two ground
    states don't connect adiabatically.
\end{question}
If the gap closes, there's a crossing, I cannot adiabatically make sure it's
the same ground state.
Adiabatic requires separation of scales.
If you get to a level crossing, you can't use the adiabatic theorem anymore,
you need to follow the right path and not get mixed up betwen the states.


\begin{question}
    You cannot ave a smooth function of $\lambda$.
    By smooth do you mean infinitely differentiable.
\end{question}
I would say yes, but that's turning to rigorous mahtematics.


\begin{question}
    Do you require 1D defects to extend infinitely in one dimension?
\end{question}
No. Everything is macrsoopic.
The 1D defect could be the shape of a ring.
But I want ot make sure the length is much larger than the correlatoin lnegth.
It could be a segment.

\begin{question}
    Why don't you consider $p=d$ where there is a cubic defect?
\end{question}
I want every sclae to be either macroscopic or I want it to be\ldots
Let me give an example.
Suppose I have a plane.
I have a defect on half the plane.
If I zoom out and keep this half finite, it becoems a 1D defect.
If I require this is a finite fraction of the total system when I zoom out,
I shouuld really thing about this as its own $d$-dimensional phase of matter,
a bulk phase.

We want to probe a phase by inserting defects.
But if it's a macroscopic part of the system,
it's just another phase, so it's not useful.


\begin{question}
    These defects are supposed to have zero measure?
\end{question}
Yes, in that sense, as you zoom out, the defect should have zero measure.


You could ask about the operator that creates hte defect,
and that could be one higher dimension.
The loop defect is created by a membrane operator.

Looking at a one-imdneiosnal defect in a two-dimensionl defect,
I need to look at operators iwht suppors on the whole two-dimensional space,
but the defect itself is a one-dimensional defect.

\section{Line Integrals}
Under what circumstances is the line integral
$\int_{A}^{B} f(z) \, dz$
independent of the path from $A$ to $B$?

\begin{theorem}[Cauchy]
    If $f(z)$ is analytic in a simply connected,
    bounded domain $D$,
    for every simple closed path $C$ in $D$,
    \begin{align}
        \oint_C f(z) \, dz = 0
        \label{eqn:1}
    \end{align}
\end{theorem}
To prove this, consider
\begin{align}
    \oint f(z)\, dz
    &= \oint_C (u + iv) (dx + i\, dy)\\
    &= \oint_C (u\,dx - v\,dy) +
    i \oint_C (v\,dx + u\,dy)
\end{align}
Then use Stokes theorm
\begin{align}
    \oint\vec{A}\cdot d\vec{r}
    = \int(\vec{\nabla}\times\vec{A})\cdot d\vec{s}.
\end{align}
Let the line integral Lie in hte $xy$ plane.
\begin{align}
    d\vec{r}
    &= dx \hat{e}_x + dy\hat{e}_y\\
    \vec{A} &= A_x \hat{e}_x + A_y \hat{e}_y.
\end{align}
Then
\begin{align}
    \oint_C \left(
        A_x\,dx + A_y\,dy
    \right)
    = \int\left(
    \frac{\partial A_y}{\partial x}
    - \frac{\partial A_x}{\partial y}
    \right)\,dx\,dy
\end{align}
if we go around the path counterclockwise.
Notice this is the same form as before.

Let us apply this to the two line integrals in Equation~\ref{eqn:1}.
For the first integral,
set $A_x = v$ and $A_y = -v$.
Notice that by the Cauchy-Riemann condition,
\begin{align}
    \oint\left(
        u\,dx - v\,dy
    \right)
    =
    \int\left(
    -\frac{\partial v}{\partial x}
    - \frac{\partial u}{\partial y}
    \right)\,dx\,dy
    =0.
\end{align}
For the second integral, set
$A_x = v$ and $A_y = u$.
Then by the other Cauchy-Riemann condition,
\begin{align}
    \oint\left(
        v\,dx + u\,dy
    \right)
    =
    \int\left(
    \frac{\partial u}{\partial x}
    - \frac{\partial v}{\partial y}
    \right)\,dx\,dy
    =0.
\end{align}
Hence
\begin{align}
    \oint_C f(z)\,dz = 0
\end{align}
Suppose $C$ is made up of two paths $C_1$ and $C_2$.
\begin{align}
    \int_{C_1} f(z)\,dz
    - \int_{C_2} f(z)\,dz
    = \oint_C f(z)\,dz
    = 0.
\end{align}
Hence
\begin{align}
    \int_{C_1}f(z)\,dz
    = \int_{C_2}f(z)\,dz
\end{align}
What Cauchy is saying is that this integral is actually a function of the final
point for fixed $A$
\begin{align}
    \int_{A}^{P}f(z)\,dz = F(P).
\end{align}


\section{Cauchy's Integral Formula}
Cauchy is saying if I know the values of an analytic function on the boundary of
a simply connected region in $\mathbf{C}$,
then I can tell you the value of the function anywhere inside the region.

I was shocked when I first saw this theorem.
There's no analogue of this for real functions.

If you consider a complex function
\begin{align}
    f(z) = u(x, y) + i v(x, y).
\end{align}
Then you can solve Laplace's equation in 2D if you know the boundary.
\begin{align}
    \frac{\partial^2 u}{\partial x^2}
     + \frac{\partial^2 v}{\partial y^2}
     = 0.
\end{align}
That's the application we're interested in.

Cauchy's integral formula states that if $f(z)$ is analytic in a simply
connected domain $D$,
then for any point $z=a$ in $D$ and any closed path $C$ in $D$ which encloses
the point $a$,
we have
\begin{align}
    \oint_C \frac{f(z)}{z - a} \, dz = 2\pi i f(a)
\end{align}
where the integration is being taken in the counterclockwise sense.
If you know the values on the boundary then you know the value of any point
inside the region.
The reason it's possible is because $f$ is composed of harmonic functions.
In principle,
if you have the values on the boundary for Laplace's equation,
then in principle you have enough information to solve Laplace's equation in the
bulk.

Define
\begin{align}
    \phi(z) := \frac{f(z)}{z - a}.
\end{align}
This is analytic everywhere in $D$ except at $z=a$.
We need to evaluate
\begin{align}
    \oint_C \phi(z)\, dz.
\end{align}
Let us go counterclockwise.
Then around $a$ draw a small circle,
and I call that circle $C'$.
Then I make a cut between $C$ and $C'$.
Let there be a point $A$ on $C$ and $B$ on $C'$.
Make a cut along $AB$.
Integrate $\phi(z)$ along the closed path from $A$ to $C$ to $B$
around $C'$ and then back to $A$.
The contribution from the cut vanishes since we integrate first in one
direction and then again in the opposite direction.
\begin{align}
    \oint_C \phi(z)\, dz + \oint_{C'}\phi(z)\, dz = 0
\end{align}
where we are going counterclockwise in $C$ and clockwise in $C'$.
Then
\begin{align}
    \oint_{C}\phi(z)\,dz = \oint_{C'}\phi(z)\,dz
\end{align}
when both integrals are performed counterclockwise.
Along the circle $C'$,
we have
$z=a + \rho e^{i\theta}$.
Remember this is a circle and the differential is
\begin{align}
    dz = i\rho e^{i\theta} \,d\theta
\end{align}
So the integral is
\begin{align}
    \oint_{C}\phi(z)\,dz &= \oint_{C'}\phi(z)\,dz\\
    &=
    \oint_{C'}\frac{f(z)}{z - a} \,dz\\
    &= \int_{0}^{2\pi}\frac{f(z)}{\rho e^{i\theta}}\left( 
        i\rho e^{i\theta}
    \right) \,d\theta\\
    &= i\int_{0}^{2\pi} f(z)\, d\theta
\end{align}
but because the circle is so small $f(z)\to f(a)$ and so
\begin{align}
    \oint_C \phi(z)\, dz
    = \oint \frac{f(z)}{z - a}\,dz
    = 2\pi i f(a).
\end{align}

The following is useful for integer $n$.
\begin{align}
    \oint \frac{dz}{{(z - z_0)}^n}
    =
    \begin{cases}
        2\pi i & \text{if } n = 1\\
        0 & \text{otherwise}
    \end{cases}
\end{align}
You can do this by using $z = z_0 + \rho e^{i\theta}$
and doing the integral explicitly on a circle.
If it's not a circle,
use the same trick with the cut like how we proved the Cauchy integral formula.

\section{Liouville's theorem}
Every analytic function is either constant or blows up somewhere in the complex
plane.
\begin{theorem}
    If $f(z)$ is analytic and bounded in absolute value in the entire complex
    plane,
    then it must be a constant.
\end{theorem}
\begin{proof}
    Assume $|f(z)|$ is bounded,
    so that $|f(z)| < M$
    for all $z\in\mathbb{C}$.
    Since $f(z)$ is analytic,
    it can be expanded like
    \begin{align}
        f(z) &= \sum_{n=0}^{\infty} a_n z^{n}.
    \end{align}
    By the Cauchy integral formula,
    \begin{align}
        |a_n|
        &= \left|\frac{1}{2\pi i} \oint \frac{f(z)}{z^{n + 1}}\, dz\right|\\
        &=
        \left|
            \sum_\gamma \frac{1}{2\pi i}
            \frac{f(z)}{z_{\gamma}^{n + 1}}
            \Dlta z_\gamma
        \right|\\
        &\le
        \sum_{\gamma}
        \left|\frac{1}{2\pi i}\right|
        \left|\frac{f(z_\gamma)}{z_\gamma^{n + 1}}\right|
        \left|\Delta z_\gamma \right|\\
        &\le
        \sum_{\gamma}
        \frac{1}{2\pi}
        \frac{f(z)|_{\max}}{R^{n + 1}} |\Delta z_\gamma|\\
        &\le \frac{1}{2\pi} \frac{f(z)|_{\max}}{R^{n + 1}} 2\pi R
    \end{align}
    where the contour is a circle centred at the origin and
    we have used the identities
    \begin{align}
        |ab| &= |a||b|\\
        |a + b| &\le |a| + |b|
    \end{align}
    Hence
    \begin{align}
        |a_n| \le \frac{M}{R^n}
    \end{align}
    for some constant $M$.
    We can take $R$ to be arbitrarily large.
    As $R\to\infty$, $a_n\to 0$ for $n\ge 1$.
    Hence $f(z)=a_0$,
    which is a constant.
\end{proof}

\section{Change of Basis}
If you get lost just \emph{insert the identity}.
Matrix multiplication works like this.
\begin{align}
    \bra{n}AB\ket{m} &=
    \bra{n} A \mathbf{1} B \ket{m}\\
    &= \sum_i \bra{n}A\ket{i} \bra{i} B\ket{m}\\
    &= \sum_i A_{ni} B_{im}
\end{align}
To change basis from $\{\ket{n}\}$ to $\{\ket{\tilde{n}}\}$
just insert the identity.
\begin{align}
    \ket{\psi} &=
    \sum_n \underbrace{\braket{n}{\psi}}_{\psi_n}\ket{n}\\
    &= \sum_n \underbrace{\braket{\tilde{n}}{\psi}}_{\tilde{\psi}_n}
    \ket{\tilde{n}}\\
    &= \sum_n
    \underbrace{\braket{\tilde{n}}{\psi}}_{\tilde{\psi}_n}
    \underbrace{\braket{m}{\tilde{n}}}_{U_{mn}}\\
    &= \sum_n \sum_m U_{mn} \tilde{\psi}_n \ket{m}
\end{align}
The result I get is the following.
\begin{align}
    \psi_n &= 
    \underbrace{U_{mn}}_{\braket{m}{\tilde{n}}}
    \tilde{\psi}_m
\end{align}
Also
\begin{align}
    \underbrace{\braket{\phi}{n}}_{\varphi_n^*}
    = \sum_n
    \underbrace{\braket{\phi}{\tilde{n}'}}_{
    U_{nn'}^* = (U^\dagger)_{n'n}
    }
    = \sum_{n'} \tilde{\phi}_{n'}^* U^\dagger_{n'n}
\end{align}
Let's do an example.
\begin{example}
    Consider functions f(x) on the real line.
    The basis of delta functions is
    $\{\delta(x - x_0)\text{ for all }x_0 \}$.
    The basis of plane waves is
    $\left\{\frac{e^{ikx}}{\sqrt{2\pi}}\text{ for all } k \right\}$.
    Do a change of basis.
\end{example}
\begin{proof}[Solution]
    Insert the identity
    \begin{align}
        \underbrace{\braket{k}{\psi}}_{\mathcal{F}[\psi](k)} =
        \int_{-\infty}^{\infty}dx\,
        \braket{k}{x}
        \braket{x}{\psi}
    \end{align}
    where
    \begin{align}
        \braket{k}{x} &=
        \int_{-\infty}^{\infty} dy \frac{e^{-iky}}{\sqrt{2\pi}}\delta(y - x)\\
        &=
    \end{align}
\end{proof}

\subsection{Eigenthings and Diagonalisation}
Sometimes, if I have the right $\ket{\psi}$ for an operator $A$,
I can get
\begin{align}
    A\ket{\psi} = \lambda\ket{\psi}.
\end{align}
This has a special name.
$\ket{\psi}$ is called the eivenvector with eigenvalue $\lambda$.

Let's do an example.
Suppose I rotate my phone.
Then I rotate it.
What's the eigenvector?
It's the axis of rotation.
What about the eigenvalue?
It's 1 because it doesn't change.

Suppose I stretch my phone along an axis.
What are the eigenvectors?
There are three of them.
Along the axis is an eigenvector, with eigenvalue the scale factor.
The other two are perpendicular, with eigenvalues 1, they are degenerate.

In 2D rotation there are no eigenvectors.

One way of thinking about the plane is a two-dimensional real space.
Then no eigenvalues.

But I can think of it as a one-dimensional complex space,
then multiplying by $i$ is a rotation.
And then I have eigenvalues.

I want to build intuition.

\begin{example}
    Consider the operator $\ket{\phi}\bra{\phi}$.
\end{example}
\begin{proof}[Solution]
    $\ket{\phi}$ is obviously an eigenvector with eigenvalue 1.
    Any other orthogonal vector is also an eigenvector,
    but with eigenvalue 0.
\end{proof}

\begin{example}
    Off diagonal $A=\sum_{n=1}^{N}\ket{n + 1}\bra{n} + \ket{1}\bra{N}$.
\end{example}
\begin{proof}[Solution]
    This is an eigenvector
    \begin{align}
        \ket{1} + \ket{2} + \cdots + \ket{N}
    \end{align}
\end{proof}
It has no importance, just practise.

\subsection{Spectral theorem}
You see spectral lines from the energy levels of atoms.
That's why it's called the spectral theorem.
\begin{theorem}
    Suppose $A$ is hermitian.
    If $A\ket{a} = a\ket{a}$,
    then
    \begin{enumerate}
        \item The $a$'s are all real,
        \item (orthonormality) The eigenvectors $\ket{a}$ can be chosen to be
            orthonormal
            $\braket{a}{b} = \delta_{ab}$,
        \item (completeness) The $a$'s form a complete basis and make a
            resolution of the identity
            $\sum_a \ket{a}\bra{a} = 1$.
    \end{enumerate}
\end{theorem}
\begin{proof}
    Let's see how I can prove it's real.
    It's very simple.
    Start from the definition
    $A\ket{a} = a\ket{a}$.
    Then take
    $\bra{a}A\ket{a} = \bra{a}a\ket{a} = a\braket{a}{a}$
    and since it's a linear operation and $\braket{a}{a}$ is real,
    But since it's hermitian,
    $\bra{a}A\ket{a} = \bra{a}A\ket{a}^*$,
    which means it's real,
    so $a$ is real.

    To prove orthonormality is slightly more complicated.
    Suppose we have
    $A\ket{a} = a\ket{a}$
    and
    $A\ket{b} = b\ket{b}$
    then take
    $\bra{b}A\ket{a} = \bra{b}a\ket{a} = a\braket{b}{a}$
    and
    $\bra{a}A\ket{b} = \bra{a}b\ket{b} = b \braket{a}{b}$.
    Note that $\braket{a}{b} = \braket{b}{a}^*$.
    From this,
    if $a\ne b$,
    then of course
    $\braket{a}{b}=0$.
    But if $a=b$,
    they are not necessarily orthogonal.
    The theorem says that we can then choose them to be orthogonal,
    which is obvious.

    The final one is a bit more difficult to prove.
    For finite $N$,
    we want to show
    \begin{align}
        \sum_{a=1}^{N}\ket{a}\bra{a} = \mathbf{1}.
    \end{align}
    Then
    \begin{align}
        (A - a\mathbf{1})\ket{a}
    \end{align}
    which implies that
    \begin{align}
        \det(A - a\mathbf{1}) = 0.
    \end{align}
    The determinant is a polynomial in $a$ of degree $N$.
    The values of $a$ for which the polynomial is zero is $a$.
    A degree-$N$ polynomial is going to have $N$ zeros,
    and so we should have $N$ eigenvalues.
    I know they are linearly independent from one another.

    But what if $N$ is not finite?
    Or worst continuous?
    That's not something I cannot prove to you.
    We haven't defined things well enough,
    whether the functions are square integrable,
    or whatever.
    I'll have to make every statement precise,
    and we're not doing physics anymore.
\end{proof}

If I see a hunk of metal,
it's a hunk of metal,
so we don't care if it's an infinitely differentiable function or a discrete
system.
It doesn't help me in the lab.
The notes have more examples you may find useful.

\begin{example}
    Consider the Hilbert space that is the set of real functions of one variable
    $f(x)$.
    We have this basis
    \begin{align}
        \left\{\frac{e^{ikx}}{\sqrt{2\pi}}\text{ for all }k\right\}
    \end{align}.
    The operator $-i\frac{d}{dx}$ is a Hermitian operator and these basis
    functions are its eigenvectors.
    You may recognise this as the momentum operator.
    The eigenvalue equation is
    \begin{align}
        -i \frac{d}{dx} f(x) = \lambda f(x)
    \end{align}
    where if you use
    $f_k(x) = e^{ikx}$
    it is satisfied.
    You can instantly recognise that $\lambda = k$.
    But this is not normalized,
    so you have to use
    $f_k(x) = e^{ikx}/\sqrt{2\pi}$.
\end{example}
By the way, the real momentum operator has a $\hbar$ in it like
$\hat{p} = -i\hbar d/dx$
and the functions should be like
\begin{align}
    f_k(x) =
    \frac{e^{ikx/\hbar}}{\sqrt{2\pi}}
\end{align}
and $\lambda = \hbar k$.

Let's do another example.
\begin{example}
    Consider the space of functions $f(x)$ again.
    Then consider the operator
    \begin{align}
        H &=
        - \frac{\hbar^2}{2m} \frac{\partial}{\partial x}
        + \frac{m\omega^2}{2}x^2
    \end{align}
\end{example}
\begin{proof}[Solution]
    The eigenvalues are
    \begin{align}
        \lambda_n &=
        \hbar\omega\left( n + \frac{1}{2} \right)
    \end{align}
    for $n=0,1,\ldots$ and the eigen functions are
    \begin{align}
        \psi_n(x) = \cdots H_n(\cdots x) e^{\cdots -x^2}
    \end{align}
    which you can look up on Wikipedia.
\end{proof}

\section{Residue Theorem}
Today I want to talk about how to evaluate integrals using the residue theorem.

Let $z_0$ be an isolated singular point of $f(z)$.
Consider the value of the closed line integral $\oint_C f(z) dz$
around a simple closed curve $C$ surrounding $z_0$
but enclosing no other singularities.
The little circle means it's a closed line as opposed to an open line.
It means it comes back to where you started.
Let $f(z)$ be expanded in a Laurent series about $z=z_0$ that converges near
$z=z_0$.

Then,
\begin{align}
    f(z) = a_0 + a_1(z - z_0) + \cdots
    + \frac{b_1}{z - z_0} + \frac{b_2}{(z - z_0)^2} + \cdots.
\end{align}

\begin{question}
    What is converging near $z_0$?
\end{question}
It means converging in the neighbourhood around that point.
This is important because you could have several Laurent series about a point
that converge in different regions.
We want the series that converges near the point we're expanding about.

The point of the residue theorem is that there are an infinite number of $b$'s,
but $b_1$ is special.
Out of all the infinite terms, this $b_1$ is special.
This $b_1$ is called the \emph{residue}.
That one is special.

The terms in the $a$ series do not contribute to the integral,
because they are analytic.
The thing is, of the terms in the $b$ series,
only $b_1$ contributes.
This is because of this identity we see again and again.
\begin{align}
    \oint \frac{dz}{\left( z - z_0 \right)^n} =
    \begin{cases}
        2\pi i & \text{if } n =1\\
        0 & \text{otherwise}
    \end{cases}
\end{align}
It's easy to prove with a circle.
Just substitute $z=\rho e^{i\theta}$.
If the curve is not a circle,
first prove arbitrary curves give the same result
using Cauchy's theorem using a cut as we showed before.
Then if you know the results of the circle,
you know the result for arbitrary curves.
Anyway,
it's a homework problem and the solution will be provided at some point.

The point is,
that if you use this integral,
the $b_1$ is the only term that satisfies $n=1$.
All the other $b$s have an $n$ that is bigger than one and doesn't contribute.
The $a$'s don't contribute because they are analytic and Cauchy's theorem tells
you they don't contribute,
and of all the $b$'s only the $b_1$ contributes because of that formula.

So then, therefore
\begin{align}
    \oint_C f(z)\, dz = 2\pi b_1
\end{align}
and this $b_1$ is called the \emph{residue} of $f(z)$ at $z=z_0$.
It's the only term that survives this integration.
Any questions about all this?

So this particular result corresponds to when you have just one singular point
inside the contour $C$.
What if there are half a dozen,
what if there is more than one singularity inside that point?
Then what?

So we have to generalize this result.
We generalize to when there are multiple isolated singularities.
And we will see there is a very simple generalization of this thing.

Now let's say this is the region [picture]
and there are two isolated singular points enclosed by the contour $C$.
Let's call this point $z_1$ and this point $z_2$.
Now what you do is draw a circle around $z_1$
and another circle around $z_2$.
This contour $C$ goes counter clockwise.
Then we do what we did before.
We create a cut like this from $C$ to $C_1$ and a cut from $C$ to $C_2$.
Now the contour is deformed.
Follow the arrows and see you have a simply connected closed loop.

A couple of things I want you to realise.
This region in between $C$ and the circles,
that is simply connected.
If I put a rubber band anywhere,
I can shrink it to a point,
and I can apply Cauchy's theorem.

So around the deformed contour,
we have
\begin{align}
    \oint f(z)\, dz = 0
\end{align}
by Cauchy's theorem,
because we have cut out the singularities and it's analytic.
The contribution to the integral from the cut vanishes.
So then you can convince yourself using the logic very similar to our proof of
the Cauchy integral formula that
\begin{align}
    \oint_C f(z) \, &=
    \oint_{C_1} f(z)\,dz
    + \oint_{C_2} f(z)\, dz
\end{align}
In other words,
the contribution from the cuts vanishes,
but you can see that $C_1$ and $C_2$ are going clockwise,
but if you make everything counterclockwise,
the signs work out.
We are going around $C$, $C_1$ and $C_2$ counterclockwise
in the equation above.

And this generalizes no matter how many singularities you have.
It is the sum of the contour integral around each point.
But we just proved the integral around one singularity is just the residue times
$2\pi i$.
Each circle contains only one singular point
and we can immediately apply the result we just derived.
So it should be $2\pi i$ times the residue of each singular point.
And this is the residue theorem we're going to apply again and again.

So this is my claim.
This leads to the residue theorem
\begin{align}
    \oint_C f(z)\, dz &= 2\pi i\times
    \text{sum of residues of $f(z)$ from singular points inside $C$}
\end{align}
This is the thing that you guys have to remember.

\begin{question}
    That's the closed contour $C$?
\end{question}
Yes it's the closed contour $C$ that contains the isolated points.
Any questions?

The rest of this class and most of the rest of the next class will be giving
examples where we use the residue theorem to evaluate integrals.
This is a place where you use complex analysis a lot.
There are many integrals where the simplest way to do them is to use a contour
integral,
sometimes you know for example,
the most convenient way to express something is as a contour integral.
This result gets used a lot,
especially if you do theoretical physics,
you will see this over and over again.
Let's do some examples.

You should look at all the literature on ELMS
and try to work through the examples there.
Do several of them.
Unfortunately I can't cover the full range of possibilities.
As you'll see,
you have to be clever in most cases,
choose your contour cunningly,
hence it's helpful to do some examples so you can learn how to choose.
But time constraints.

Let's start with some very simple ones.
\begin{example}
    Integrate $f(z) = \sin(z)/z^4$ around the unit circle counterclockwise.
\end{example}
\begin{proof}[solution]
    This potentially has a pole at $z=0$.
    There's nowhere else where it might have a pole.
    The first thing to do is expand this in Taylor series.
    \begin{align}
        \frac{\sin(z)}{z^4} &=
        \frac{1}{z^4}\left\{
            z
            - \frac{z^3}{3!}
            + \frac{z^5}{5!}
            - \cdots
        \right\}
    \end{align}
    What is the order of the pole?
    It's order 3.
    What is the residue?
    It's $-1/6$.
    You can see this
    \begin{align}
        f(z) &=
        \frac{1}{z^3}
        - \frac{1}{6} \frac{1}{z}
        + \frac{z}{5!} - \cdots
    \end{align}
    Don't forget that negative sign,
    the residue is $-\frac{1}{6}$,
    not $\frac{1}{6}$.
    So by the residue theorem,
    \begin{align}
        \oint \frac{\sin z}{z^4}\,dz &=
        2\pi i \left( -\frac{1}{6} \right)\\
        &= -\frac{\pi i}{3}
    \end{align}
\end{proof}
Any questions?
OK let's do another example.

\begin{example}
    Integrate $\frac{4 - 3z}{z^2 - z}$
    around the circle $|z|=2$ counterclockwise.
\end{example}
\begin{proof}[solution]
    This is the circle of radius 2 centred at the origin.
    Look at this and tell me where it has poles.
    $z=0$ and $z=1$ where the denominator vanishes.
    The poles are where the denominator vanishes.
    And $0$ and $1$ are both inside the circle.
    And so we really have to use the residue theorem
    but first we need to find the residue at both $0$ and $1$.
    That's the procedure.

    The function has poles at $z=0$ and $z=1$.
    Near $z=0$,
    \begin{align}
        f(z) &=
        \frac{4 - 3z}{z(z - 1)}\\
        &= \frac{1}{z}\left\{
            \frac{4 - 3z}{z - 1}
        \right\}\\
        &= -\frac{4}{z} + \text{analytic terms near $z=0$}
    \end{align}
    The fraction in the bracket is analytic at $z=0$,
    and it's equal to $-4$.
    What's where it comes from.

    Near $z=1$,
    \begin{align}
        f(z) &=
        \frac{4 - 3z}{z(z - 1)}
        = \frac{1}{z - 1}\left\{
            \frac{4 - 3z}{z}
        \right\}
    \end{align}
    Near $z=1$, the things in the braces are analytic,
    so then the whole thing behaves like
    \begin{align}
        f(z) = \frac{1}{z - 1}
    \end{align}
    and hence $1$ is the residue.
    
    By the residue theorem,
    \begin{align}
        \oint f(z)\, dz &= 2\pi i\sum \text{residues}\\
        &= 2\pi i \left\{ -4 + 1 \right\}\\
        &= -6\pi i.
    \end{align}
    These are just some really simple examples to illustrate.
\end{proof}
Let's suppose this is $|z|=1/2$.
How would this change the answer?

If you draw it,
the circle would then have radius $1/2$,
and $z=1$ is outside the circle,
so then this $z=1$ residue wouldn't have contributed
and the result would be $-8\pi i$.
The boundary is not well defined for $|z|=1$.
You may have to take some limiting procedure,
and the answer may depend on how you take that limiting procedure.

Let's go on to the next problem.
\begin{example}
    Evaluate the definite integral
    \begin{align}
        I = \int_{0}^{2\pi} \frac{d\theta}{5 + 4\cos\theta}.
    \end{align}
\end{example}
\begin{proof}[Solution]
    Strictly speaking,
    you don't need complex analysis to do this integral,
    but it's a tough one.
    You'd have to use
    \begin{align}
        t = \tan\frac{\theta}{2}
    \end{align}
    but you would have had to know it.
    I'll show you it's easy with complex analysis.
    Here's what you do.
    There's all classes of problems where you integrate trig functions.
    The steps are going to be the same.

    Change variables to $z=e^{i\theta}$.
    Then $\theta$ goes from $0$ to $2\pi$
    around the unit circle in the complex plane.
    As $\theta$ goes from $0$ to $2\pi$,
    we go around the unit circle in the complex plane,
    which of course is $|z|=1$.

    So then, you have to change variables from $\theta$ to $z$.
    \begin{align}
        dz = i e^{i\theta} d\theta\qquad
        \implies\qquad
        \frac{1}{i} \frac{dz}{z} = d\theta.
    \end{align}
    And then
    \begin{align}
        \cos\theta = \frac{1}{2}\left(
            z + \frac{1}{z}
        \right).
    \end{align}
    To see this
    just use the Euler formula
    \begin{align}
        e^{i\theta} = \cos\theta + i\sin\theta.
    \end{align}
    So then the problem has reduced to evaluating
    \begin{align}
        I &=
        \oint \frac{1}{iz}
        \frac{1}{5 + 2\left( z + \frac{1}{z} \right)}\,dz\\
        &= \frac{1}{i}\oint \frac{dz}{2z^2 + 5z + 2}.
    \end{align}
    That's what you get when the dust settles.
    Now we have to figure out what the poles of this thing are.
    You have to find the roots first.
    I'm going to skip the step and claim it can be written like this.
    \begin{align}
        I &=
        \frac{1}{i}
        \oint
        \frac{dz}{(2z + 1)(z + 2)}.
    \end{align}
    If I draw this in the complex plane,
    it has a pole here at $z=-1/2$ and then it has a pole here at $z=-2$.
    And we want the contour integral of the unit circle
    centred at $z=0$.
    You can see that only one of the two poles is inside the circle.
    We're going counterclockwise because $\theta$ is increasing this way.
    $\theta$ is increasing from $0$ to $2\pi$ so we're going counterclockwise,
    which is good,
    so we don't have to put an extra sign in
    and just blindly apply the the theorems.

    Only the pole at $z=-1/2$ contributes.
    Now let's calculate the residue.
    Let's look at this
    \begin{align}
        \frac{1}{(2z + 1)(z + 2)} &=
        \frac{1}{z + \frac{1}{2}}\left[ 
            \frac{1}{2(z + 2)}
        \right]\\
        &= \frac{1}{z + \frac{1}{2}}\left[
            \frac{1}{3}
        \right]
    \end{align}
    near $z=-1/2$,
    where we have evaluated the analytic part at $z=-1/2$.
    So $1/3$ is the residue,
    so
    \begin{align}
        I &= \frac{1}{i}2\pi i \frac{1}{3}\\
        &= \frac{2\pi}{3}
    \end{align}
\end{proof}

Any questions?
\begin{question}
    When evaluating the analytic part,
    can we ever just say this region is analytic and just plug in the value of
    $z$.
    Do we have to do the Taylor series?
\end{question}
You don't have to do the Taylor series,
only the first term matters.
This thing is non-zero,
but if it is zero,
the residue is zero.

We have to be a little bit careful.
Suppose if our function had a square in it
\begin{align}
    \frac{1}{(2z + 1)^2(z + 2)}
    = \frac{1}{\left( z + \frac{1}{2} \right)^2}\left[
    \frac{(z + \frac{1}{2})}{2(z + 2)}
    \right]
\end{align}
there is a higher order pole and you have to expand.
In physics problems you typically only have simple poles
and you don't have to expand.

You always want to make it a closed curve.

\begin{question}
    What happens if the integral is not from $0$ to $2\pi$?
\end{question}
Can we try $z=e^{i2\theta}$ for example if it's from $0$ to $\pi$.

\begin{question}
    Can you explain how you'd do it if it were a complex pole?
\end{question}
Let's say instead we had
\begin{align}
    \frac{1}{(2z + 1)^2 (z + 2)}
    = \frac{1}{\left(1 + \frac{1}{2} \right)^2}
    \frac{1}{4\left(z + 2 \right)}
\end{align}
Then you have to expand in a Taylor series,
so the coefficient of the next term in the Taylor series is what gives you the
residue.
In this situation with a higher order pole,
you need more terms in the Taylor series to work it out.

\begin{example}
    Evaluate
    \begin{align}
        I = \int_{-\infty}^{\infty}
        \frac{dx}{1 + x^2}
    \end{align}
\end{example}
\begin{proof}[Solution]
    If you were to use elementary methods,
    you would get
    \begin{align}
        \tan^{-1}(x)|_{-\infty}^{\inftyy}
        = \frac{\pi}{2}
        - \left( -\frac{\pi}{2} \right)
        = \pi.
    \end{align}
    Let's do it using complex analysis,
    just to show you the kind of contours you can integrate with.
    Consider the complex line integral
    \begin{align}
        \lim_{\rho\to\infty}
        \int_{-\rho}^{\rho}
        \frac{dz}{1 + z^2}
    \end{align}
    where the path is along the real line.
    This line integral is equal to $I$.
    Now let's draw this thing in the complex plane.
    Let's look at this integrand.
    It has poles where this $1+z^2$ vanishes.
    The denominator vanishes at $z=+i$ and $z=-i$.
    Right now the fact it has two poles is irrelevant for now.
    Consider this thing integrated not just along the real line,
    but back along a big semicircle in the upper half plane.

    Consider
    \begin{align}
        \oint \frac{dz}{1 + z^2}
    \end{align}
    integrated over the closed contour shown in the figure,
    which consists of a semicircle over the upper half plane,
    in addition to the integral over real line.

    This closed line integral we can evaluate using the residue theorem.
    Even though it has two poles,
    only $z=+i$ is inside the contour.
    Let's evalaute it.
    Close to $z=i$,
    \begin{align}
        \frac{1}{1 + z^2} = \frac{1}{(z + i)(z - i)}
        = \frac{1}{z - 1} \underbrace{\frac{1}{2i}}_{\text{residue}}
        + \text{analytic function}
    \end{align}
    Hence by the residue theorem, the contour integral is
    \begin{align}
        \oint \frac{dz}{1 + z^2} &=
        2\pi i \frac{1}{2i} = \pi.
    \end{align}
    We have evaluated this integral not over the real line,
    but over this closed contour which includes the real line,
    but also includes the contribution from this big semicircle.
    I'm going to show you the contribution from the big semicircle vanishes as
    you take $\rho\to\infty$.
    And so this closed integral is actually equal to the integral we want.
    That's going to be the logic here.

    We need to prove the line integral over the semicircle vanishes and then
    we're done.
    Let's prove that.
    \begin{align}
        \underbrace{\oint \frac{dz}{1 + z^2}}_{\pi} &=
        \underbrace{\int_{\text{real line}} \frac{dz}{1 + z^2}}_{I}
        + \underbrace{\int_{\text{semicircle}} \frac{dz}{1 + z^2}}_{?}
    \end{align}
    Set $z=\rho e^{i\theta}$.
    Then consider
    \begin{align}
        \int_{\text{semicircle}} \frac{dz}{1 + z^2} &=
        \int_{0}^{\pi} \frac{i\rho e^{i\theta}}{1 + \rho^2 e^{2i\theta}}\,
        d\theta
    \end{align}
    This has $\rho$ in the numerator and $\rho^2$ in the denominator,
    this should go to zero as $\rho\to\infty$.
    But you still have to be careful because it's complex,
    but you'll see this intuition is correct.
    It's just going to vanish.
    \begin{align}
        \left|
            \int_{\text{semicircle}} \frac{dz}{1 + z^2}
        \right|
        &=
        \left|
            \int_{0}^{\pi} \frac{i\rho e^{i\theta}}{1 + \rho^2 e^{2i\theta}}\,
            d\theta\\
        \right|
        &\le
        \int_{0}^{\pi}
        \left|
            \frac{i\rho e^{i\theta}}{1 + \rho^2 e^{2i\theta}}
        \right|
        \left|
            d\theta
        \right|\\
        &\le
        \int_{0}^{\pi}
        \frac{\rho}{\rho^2} |d\theta|\\
        &=
        \frac{\pi}{\rho}\\
        &\to 0
    \end{align}
    as $\rho\to\infty$.
    So in the end the contribution of this semicircle vanishes.
\end{proof}

This shows you can do an integral over the real line by doing an integral like
this with a contour like this.
This is one of the very standard contours where you have an integral over the
real line and make it a contour by closing the upper half plane,
or lower half plane.

\begin{question}
    What if you took the semicircle in the lower half plane?
\end{question}
Everything would work, but you're not going counterclockwise and you have to
deal with signs.
It still works though.
There's more than one way to do these problem.
Choose the most convenient one.
As you do more provlems, you'll see.
I want you guys to convince yourselves that it can be closed in the negative
half plane and get the same result.

This contour works for this function works for a lot of functions,
but it also doesn't work for a lot of functions.
You could have used either up or down in this problem,
but we're going to look at at problem where you have to use one of them.
It's actually more common.

\begin{example}
    Evalaute
    \begin{align}
        I = \int_{0}^{\infty} \frac{\cos(z)}{1 + x^2}\,dx
    \end{align}
\end{example}
\begin{proof}[Solution]
    Notice that it's similar to before.
    Notice that the integrand is even, so this is equal to
    \begin{align}
        I &=
        \frac{1}{2}\int_{-\infty}^{\infty}
        \frac{\cos(x)}{1 + x^2}\,dx.
    \end{align}
    It's an even function so we can be a bit clever.
    We want to make this an integral over some line in the complex plane.
    There's more than one way to do this,
    but here's the trick.
    Write $z = \Re e^{iz}$.
    Then write
    \begin{align}
        I &= \re\left\{
        \frac{1}{2}
        \int_{-\infty}^{\infty}
        \frac{e^{iz}}{1 + z^2}\,
        dz
        \right\}
    \end{align}
    where the integral is over the real line in the complex plane.
    Now it's beginning to look a lot like the problem we just saw.
    You have a function integrating over the whole real line.
    It has singularities at exactly the same points $z=\pm i$.
    But can we use the same contour?

    The integrand is a bit different.
    Can we convince ourseves that the contribution from this semicircle
    vanishes?
    If it does, then we can use the same contour.
    It turns out it does vanish.

    Along that big semicircle, $z$ has a real and imaginary part
    \begin{align}
        z = \rho_R + i\rho_I
    \end{align}
    Unless you're at $+\rho$ or $-\rho$,
    the imaginary part is infinite.
    If the imaginary part is positive and infinite,
    $e^{iz}$ because $e$ raised to minus infinity,
    so it goes to zero exponentially quickly
    if you're in the upper half plane.
    That means this integrand vanishes incredibly quickly in the upper half
    plane.
    So we can use exactly the same contour as we used before
    and use the residue theorem.
    At the same time,
    we can't use a contour with a semicricle in the lower half plane,
    because it blows up incredibly quickly.
    Last time we had a choice of contours up or down,
    but here we don't have a choice,
    at least we don't have that choice,
    you have to choose to go in the upper half.
\end{proof}

\section{Anomalies}
Last time we mentioned the types of TQFTs.
There's one more TQFT that you might encounter.
There's something called an extended TQFT.\@
When I define TQFT,
I assign a path integral to a $(d+1)$-dimensional manifold
and we have a path integral
$Z\left( M^{d+1} \right)\in \mathbb{C}$.

For a closed $d$-manifold $V(\Sigma^d)$.

For a closed $(d-1)$-manifold,
we can associate a $1$-category.
This is called once-extended.
The objects get more and more abstract as you go down.

And then for $(d - 2)$-manifold,
you can attach a 2-category.
For a $(d-3)$-manifold,
you attach a 3-category.

I'm not going to define these.
And then for a 0-manifold,
you get a $d$-category.

Useful to note that if you have a fully extended TQFT,
all the way down to $d$-category,
this is going to come up for us,
we won't necessarily use it,
but we will encounter them often.
Fully extended TQFTs are situations where you can write down exactly solvable
Hamiltonians for your topological phases as sums of commuting projectors.
And then the path integral can be written as an exact combinatorial state sum.

We'll encounter examples eventually.
To define the path integral,
triangular the manifold
define simplexes 1 2 3,
assign labellings,
and them sum over all labellings.
That's an exact combinatorial space.
They're useful to work with and have a lot to say about them.
Physicists don't have a good understanding of what these higher structures mean
physically,
so it's an interesting question.


\section{Invertible vs non-invertible TQFT}
The next distinction is between invertible and non-invertible TQFTs.

\begin{definition}
    A TQFT is \emph{invertible} if the magnitude of the path integral has unit
    magnitude
    \begin{align}
        |Z(M^{d+1})| = 1
    \end{align}
    on any closed manifold.
\end{definition}
In particular, if you look at $Z(\Sigma^d\times S^1) = 1
= |\dim V(\Sigma^d)|$.
This is interpreted as the dimension of the Hilbert space on this space
Physically,
this means that there is always a unique state on any closed $\Sigma^d$.
A topological state of matter described by an invertible TQFT always has a
unique ground state on any closed manifold.
Not always the case,
depending on the topology of the space,
and those will be described by non-invertible TQFTs.

The reason it's called invertible,
is that if the path integral is a phase,
there's always a well-defined inverse of the phase,
which is the complex-conjugate of the phase.

Let's say if you have a TQFT,
then the ``inverse'' TQFT has a path integral like
\begin{align}
    Z_{T^{-1}}\left( M^{d+1} \right)
    =
    Z_T^*\left( M^{d+1} \right)
\end{align}
Invertible TQFTs come up a lot.
They describe integer quantum hall states.
Topological insulators and superconductors.
All $(1+1)$-dimensional topological phases of matter are describable by
invertible TQFTs.
Any TQFT that is not invertible is a \emph{non-invertible} TQFT.\@

Actually, one more thing before I get to that.
Invertible TQFTs form an Abelian group.
Suppose that I have the path integral for two theories $T_1$ and $T_2$,
and I want the path integral for $T_1\times T_2$,
then I just multiply their path integrals.
\begin{align}
    Z_{T_1T_2}\left( M^{d+1} \right)
    =
    Z_{T_1}\left( M^{d+1} \right)
    Z_{T_2}\left( M^{d+1} \right)
\end{align}

\begin{question}
    Does the partition function define the TQFT uniquely?
\end{question}
Well $V(\Sigma^d)$ is always going to be a one-dimensional vector space.
Even if it has some boundaries,
it's just going to be a number,
a multiple of that state,
so just an amplitude.
We can extend that definition to the full TQFT.
I'm specifically talking about the invertible ones.

Even in $M^{d+1}$ is open,
the state on the boundary is one-dimensional,
I can just take the inverse theory to give me the complex conjugate.

If a theory is not invertible,
it is a non-invertible TQFT,
and it will have a non-trivial degeneracy on different manifolds.
One thing that is true,
but I'm not sure if there is a proof,
I doubt there's a counterexample,
but I think that this always goes the other way.
There's also an arrow going the other way.
If the path integral magnitude is 1,
then every path integral on the manifold is 1.
I've seen the proof for once-extended,
but I'm not sure if it's general.

So a non-invertible TQFT should have
$Z|\left( \Sigma^d\times S^1 \right)\ne 1|$
for some $\Sigma^d$.
Using ``should'' loosely here.

Let me give you a quick example of an invertible TQFT.\@
This is actually going to be an interesting example,
because it's going to show you a TQFT that is non-trivial,
but it's going to describe a trivial phase of matter.
Imagine we take our path integral on a two-manifold
and have some $U(1)$ gauge field $A$ with field strength $F$.
Then
\begin{align}
    Z\left( M^d, A \right)
    =
    e^{-\theta \int_{M^2} F}
    e^{i\phi \chi\left( M^d \right)}
\end{align}
where
\begin{align}
    \chi\left( M^d \right) = \frac{1}{2\pi}\int_{M^2}R
\end{align}
is the Euler characteristic.

Phases of matter are 1 to 1 correspondence with deformation classes of TQFTs.
In the absence of pinned phases of $\phi$,
this is an example of an invertible TQFT.\@

The first most important question studying a new phase of matter is to ask if
it's describable by an invertible TQFT or a non-invertible TQFT.\@

\section{'t Hooft anomaly}
One big concept about TQFTs I want to mention are called
\emph{'t Hooft anomaly}.
This is something that applies a general QFT,
not just TQFT.
We can consider a QFT which has a symmetry group $G$.
This is the global symmetry group of the QFT.
This may be completely well-defined in the absence of any background gauge field
associated with $G$i.
Then you try to turn on a background gauge field and see something is wrong
with the theory.
Consider your path integral $Z(M^{d+1})$ and turn on $A$,
and you find that the path integral is not gauge-invariant.
\begin{align}
    Z\left( M^{d+1}, A + d\lambda \right)
    \ne
    Z\left( M^{d+1}, A \right)
\end{align}
when $A\ne 0$.
It's not the end of the world.
You might be able to redefine the path integral and make it gauge invariant.
You may be able to define a new gauge-invariant path integral
\begin{align}
    \tilde{Z}\left( M^{d+1}, A \right)
    = Z\left( M^{d+1}, A \right)
    e^{i\int_{M^{d+1}} f(A)}
\end{align}
where the $e^{i\int_{M^{d+1}} f(A)}$
is called a local counter-term that cures the gauge invariance.

\begin{question}
    You mean the action of $G$?
\end{question}
You look at all the correlation functions and you're applying $G$ to all the
operators.
The ground state of the QFT is invariant under $G$.

\begin{question}
    When you couple the background gauge field,
    should we be doing minimal coupling?
\end{question}
It depends how you define this quantity.
Suppose you have some action that is a single Dirac fermion with a mass.
You integrate out the mass
and find a Chern-Simons term with some action.
You wouldn't have written this term down anyway if you were experineced.
But you could actually just add another local functional to this,
and now the coefficients add up to 1 and they are gauge invairant.
If you're not following this discussion it's fine.
I haven't introduced Chern-Simons theory yet.
But you're right.
Usually, you would have defined your path integral in the first place.

\begin{question}
    Is it an anomaly in global symmetry?
\end{question}
Yes. This is the only anomaly worth having a name.
Gauge anomaly means you have a 't Hooft anomaly you try to gauge.
You may not always be able to add counter terms.
I haven't defined 't Hooft anomaly yet.

My point is that a theory with a 't Hooft anomaly is a real theory.
A theory with a gauge anomaly is a theory that doesn't make sense in the first
place.

There's something called a framing anomaly.
There's gravitational anomalies.
Anomaly is an overloaded term in physics,
used everywhere for different things.

There's no such thing as a gauge theory in QM.
It's a classical redundancy in your description.
A theory can have a global symmetry that is anomalies.
That is a 't Hooft anomaly.

You write down a Lagrangian with classical gauge symmetry.
You quantize it and find the symmetry does not hold when quantized,
and that's an anomaly.
There's a whole story.

There could be situations where you can't cancel this gauge apology by adding
counter terms.
You can cancel it not with a local counter term,
but thinking your theory as the surface of a higher dimensional invertible
TQFT.

So,
we can have a situation where the gauge non-invariance is cancelled by a
$(d+2)$-dimensional invertible TQFT.
An invertible TQFT that exists in a 1-higher dimension.
You can think of your system as living on the $(d+1)$-dimension surface of a
$(d+2)$-dimensional bulk and this bulk is an invertible TQFT.
And then you can have some bulk theory
$Z_{\text{bulk}}\left( W^{d+2}, A \right)|_{\text{boundary cond.}}
= Z\left( \partial W^{d+2}, A|_{\partial W} \right)$
perfectly well-defined,
but if you evaluate with some boundary condition.
And the gauge transformation would be governed by the higher-dimensional theory.
So
\begin{align}
    Z\left( M^{d+1}, A + d\lambda \right)
    Z\left( M^{d+1}, A \right)
    = Z_{\text{bulk}}\left( M^{d+1}\times I, \tilde{A} \right)
\end{align}
So the amplitude of going from $A$ to $A + d\lambda$,
there's an extra phase that comes from the invertible TQFT in the bulk,
which cancels out the gauge non-invariance here,
but cancelling in a way that cannot be done by just a local counterterm.

So then 't Hooft anomalies can be classified by TQFTs.

\begin{question}
    Why only just one higher dimension?
\end{question}
If you have to go to 2 higher dimensions,
you can always compactify one of te dimensions
and then you're back to $+1$ extra dimensions.

Let me just explicitly write it out

A QFT has a 't Hooft anomaly if
\begin{enumerate}
    \item It has a global symmetry $G$
    \item The path integral $Z$ is not gauge invariant under gauge
        transformaiton $A\to A + d\lambda$
    \item Gauge non-invariance cannot be cancelled by a local counter-term,
        but it can be cancelled by a $(d+1)$-dimensional TQFT.
\end{enumerate}
In topologcial pahses of matter,
some are described at low energy witha 't Hooft anolmaly
and what that tells us is that it cannot be realised ni $(d+1)$ spacetime
dimensions,
but has to be the surface of a higher dimensional theory.
In condensed matter,
a theory witha 't Hooft anomaly can only be raelised microscopically
and have the symmetry act on-site,
it arises as the $(d+1)$-dimensional surface of a $(d+2)$-dimensional
topological phase.

On-site means the symmetry means a tensor product of on-site symmetry opreators
separately.
That is, 
$g=\bigotimes_i g^{(i)}$.

The symmetry group dictates that the gauge field is a connetion on a principle
$G$-bundle,
so it's a gague field of global symmetry group $G$,
so it controls what gauge transfomatinos you have.
What I'm saying is
your therory may be defined for correlation functions without $A$,
but when you turn on the $A$ and things go wrong.
Let me just mention a few examples of 't Hooft anomalies in case you may have
seen examples before.

If you haven't seem examples before,
these examples will come up in the course.

Famous example.
You can have a $(1+1)$-dimensional chiral fermion
with $U(1)$ symmetry.
It can only exist in the boundary of an integer quantum Hall state,
which is a TQFT.

Another example.
You can have a single $(2+1)$-dimensional massless Dirac fermion
with $U(1)$ and time-reversal symmetry.
But this suffers from the famous parity anomaly.
So htis can only exist at the surface of a $(3+1)$-dimensional topological
insulator.

In particular physics,
there's a famous axial anomaly,
which is an example of a 't Hooft anomaly.

These are the most famous ones.

\begin{question}
    Are massless Dirac fermions topological?
\end{question}
No,
I'm just saying the ntion of a 't Hooft anomaly applies to QFTs in general.
These examples are not topological.

\begin{question}
    Gauge-gravity duality?
\end{question}
It's related in an either loose way or such a profound way that I don't have an
aswer.
Definitely related,
but it's hard to yeah.
It's related,
but it's one of things things,
as time goes on
we're seeing more and more hints.
At the surface it seems pretty good.

\begin{question} 
    Is there some version for CFTs of diffe.rent dimensions?
\end{question}
A CFT can certainly have a 't Hooft anomaly.
They must live at the surface of a higher-dimensional TQFT.
But a higher-dimensiona CFT doesn't really make sense,
because the boundary theory only makes snese
if there are no additional degrees of freedom in the bulk.
If the bulk has degrees of freedom,
there can be leaking of these.
Here the boundary has degrees of freemdom that cannot leak into the bulk.

\begin{question}
    Why should the bulk by invertible?
\end{question}
There's two answers.
People do talk abbout non-invertible anomalies.
There are examples,
but it suffers from the thing I was describing.
If your bulk is non-inverible,
it means there are non-trivial degrees of freedom that are also non-trivial i
hte bulk,
they can leak out and there is no clean separation between the bluk and the
boundary.

An integer quantum Hall state is well-defined,
but the boundary mode cannot exist on its own.

We can open up the bulk and have it open up the space and there is a boundary
theory there.
.
\begin{question}
    So you have abulk theorey
    You restrict it to the boundary,
    you evaluate it on hte boundary?
\end{question}
It's better to htink I have a bublk theory,
and if the manifold has a boundary,
then I will specify some boundary conditions.

\begin{question}
    When you specify the boundary condition,
    what do you mean exactly?
\end{question}
It's only when I specify boundary conditions that I can get a number $Z$.

What is the bulk?
$Z_{\text{bulk}}\left( W^{d+2} \right)\in V(\partial W^{d+2})$
is just a state,
which I take an inner product with something to specify the boundary condition.

It might be more useful that we go to examples later on.
This discussion is a bit abstract.
Different theories may correspond to different boundary conditions.
In all cases, the boundary theory will have a 't Hooft anomaly.

\begin{question}
    TFQT exist on the boundary of a therory in the sense you want to put it on a
    lattice?
    If we don't put it on a lattice and don't couple it to a gauge field?
\end{question}
Call that a macroscopic realisation.

A single fermion is not emergable from a one-dimensional system.

\begin{question}
    The Standard model is not emergable?
\end{question}
If a theory has a 't Hooft anomaly,
if you're willing to give up the symmetry,
you can realise it microscopically.
If you dedmanded you wanted to preserve axial symmetry,
you have to go to the surface of a one-higher dimensino.

In the standard mdoel,
there is actually 2 anomoalies.
There's a gauge $U(1)$ anomaly for chiral fermions,
and there's a gravitaitonal anomaly,
that you cannot give up.
It's tied to energy enregy conservation.
I don't even know what that means,
because then you have a time-dependent Hamiltonian.

I think we're getting ahead of ourselves with this example.

\begin{question}
    In a 1D condensed matter system,
    in a Luttinger liquid for example?
\end{question}
That anomaly you can think of as an axial anomaly in 1+1 dimensinos.
The theory emerges out ofa sitautiaon where it doesn't exist microscopically.

If there are no more questions about TQFTs,
we're going to talk about actual topological quantum phases of matter.

\section{1D topological superconductor}
This is describable by an invertible spin TQFT.

We're going to start off with a $(1+1)$-dimensional
Majorana-Kitaev chain.
Start with a simple toy model that is the 1d spineless $p$-wave superconductor.
That's going to be described by the following.
Imagine you have 1D array of $N$ sites.
WRite the free fermion Hamiltnoia,
which you can think of as a free field for the semiconductor.

\begin{align}
    H &=
    \sum_{i=1}^{N}
    \left[ 
    -t\left( c_i^\dagger c_{i+1} + c_{i+1}^\dagger c_i\right)
    - \mu\left( c_i^\dagger c_i - \frac{1}{2} \right)
    + \Delta c_i c_{i+1}
    + \Delta^* \chi_{i+1}^\dagger c_i
    \right]
\end{align}
and consider periodic boundary conditions $c_{i+N}=c_i$.

Consider momentum space.
\begin{align}
    \tilde{c}_k &=
    \frac{1}{\sqrt{N}}
    \sum_{j=1}^{N}
    c_j e^{ik_j}
\end{align}
where $k = 2\pi n/N$,
$n=0,\ldots,N$.
The Broullin zone is $[0,2\pi] \simeq [-\pi, \pi].
$\Delta is just any compelx number here.

Define this spinner with some redundnacy.
\begin{align}
    \Phi_k =
    \begin{pmatrix}
        c_k\\
        c_{-k}^\dagger
    \end{pmatrix}
\end{align}
then the Hamiltonian is
\begin{align}
    H &=
    \frac{1}{2}\sum_{k\in BZ}
    \Psi_k^\dagger \mathcal{H}_k \Psi_k
\end{align}
So then the Bloch Hamiltonian is
\begin{align}
    \mathcal{H}_k =
    \begin{pmatrix}
        \mathcal{E}_k & \Delta_k^*\\
        \Delta_k & -\mathcal{E}_k
    \end{pmatrix}
\end{align}
where
\begin{align}
    \mathcal{E}_k &= -2t\cos k - \nu\\
    \Delta_k &= -2i\Delta\sin k
\end{align}
The dispersion relation is
\begin{align}
    E_k &= \pm
    \sqrt\left\{ \mathcal{E}_k^2 + |\Delta_k|^2 \right\}\\
    &= \pm
    \sqrt{
        (\mu + 2t\cos k)^2
        + 4|\Delta|^2\sin^2 k
    }
\end{align}2
$E_k$ is always gapped away from $k=0,\pi$.

If $\mu=-2t$,
then it is gapless at $k=0$.
If $\mu=2t$,
then it is gapless $k=\pi$

Let's draw some pictures.

[pictures]
It's called p-wave because $\Delta_k$ is linear in $k$ for small $k$.

If the Hampton is
\begin{align}
    \mathcal{H}_k &=
    h_k^0 \mathbf{1}
    + \vec{h}_k \cdot \vec{\sigma}
\end{align}
Then the combo.
\begin{align}
    h_k^x = -h_{-k}^x
    \qquad
    h_y^y = -h_{-k}^y
    \qquad
    h_k^z = h_-zk
\end{align}

If we know $\vec{h}_k$ from $[0,\pi]$,
then we know is everywhere.
\begin{align}
    \hat{h}_{k=0} &= S_0 \hat^{z}\\
    \hat{h}_{k=\pi} &= \underbrace{S_{\pi}}_{\pm 1}\hat{Z}
\end{align}
depedning on the sign of $\mathcal{E}_\pi$.
Define $\nu = S_0 S_\pi$.

The two spaces $S_0 = S_{\pi}=2$
and $S_0 = -S_{T}$.
They are distinct so long as the energy gap stays open so $h_k$ is well-defined.

[picture]

Everything is region is topological.
Everything outside is topological.
As we tune the chemical potentitoni through kinetic energy,
if we have an odd number of Fermi points,
they we're on a topological phase.


\section{2021-09-22 Lecture}

This is a class that cannot fail,
because if it crashes and burns,
the qualifier fails.

This is my favourite lecture.
Sorry it's online.
It's conceptual.
This used to not be covered in standard classes,
but it because so popular there's no way out.
I'm going to start from something everyone knows.
The double slit experiment.

If you're not comfortable,
it means two things:
you don't know QM and you should laern straight away.

The best way to learn is to open the old Feynman lectures on physics.
Of course you can go to Youtube anddo the same thing,
with beautiful computer graphics,
but unfortanaltely many of them are wrong.
Feynman knows physics.


If there's only one hole nothing exciting happens.
The particle hits the screen,
mostly behind the hole.
Here I plot the number of particles that hit the screen.
If the hole is left,
you get particles on the lfet.
If the hole is right,
you get particles on the right.

Buty what happens when you shoot two holes?
They interferer with each other.
The result is surpricingly,
in fact the most likely place to get a particle is in the middle,
which is neither the single hole scneairos.
It's more than just the sum of the two.
You get an interference pattern.

You seen t his before.
It's pretty.
There are regions where the waves interfere desctructively and you don't get a
particle behin the screen.
that's how all waves behave.
If you make waves no thel ake,you get he same thing,
so it's not surprising.
What is surprising is tat the aprticles sometimes behave like particles,
and sometimes waves.
In fact,
what you get on the screen is this.

If you send particles one by one and you start counting,
every time you send a particle,
you get one spot on the screen,
however the probability fo getting apricles obeys the interference pattern.
That's this other way we see this particle-wave duality.

I'm going to switch to my laptop.
That's a drawing of the situation here.
I want to remind you how to descrbie this quanutm mechancially.


Let's say the left sitautions is $\ket{L}$
and the situation for the right hole open is $\ket{R}$.
When you have both holes open,
the state is going to be a linear combination
\begin{align}
    \ket{\psi} = \alpha\ket{L} + \beta\ket{R}
\end{align}
How do you get intference?
Let's measure $A_y$ that counts the number of particles,
or the probablity of the particle arriving at coordinate $y$
on hte screen.
Of course,
I want to take the expectation value.
\begin{align}
    \bra{\psi} A_y \ket{\psi} &=
    \left( \alpha^* \bra{L} + \beta^*\bra{R} \right)
    A_y
    \left( \alpha\ket{L} + \beta\ket{R} \right)\\
    &=
    |\alpha|^2 \bra{L}A_y\ket{L}
    + |\beta|^2 \bra{R}A_y\ket{R}
    + \alpha\beta^* \bra{L}A_y\ket{R}
    + \alpha^*\beta \bra{R}A_y\ket{L}
\end{align}
The first two terms are familiar.
They are just the probabilities for if the left hole is open only
and the rigt hole is open only respectively.
THe novelty of course in QM is that you have inerference terms,
the last two terms.
While the $|\alpha|^2$ and $|\beta^2|$ terms are positive,
those $\alpha\beta^*$ and $\beta\alpha^*$ terms are not necessarily positive,
so it's possible you cna have cancelation.
And that's why you get zeros in the distribution,
it's because of intereference and cancellation.

So this is well known.
The fact you have uqanutm intereference out of linear combinations,
you get this pretty well.

Let's compare this with a different situation.
Let's say I have a different siuaiton like this.
Except the two slits are never open at the same time.

I'm going to trow a dice.
If the number is odd,
I open the left hole.
If the number is even,
I open the right hole.
What am I going to get?

hal the time I'm going to get the L distributino,
and half the time I'm going to ge the R distribution.
And I get a double-bump distirbution.

So then this quantity,
The previous situation would be left AND right.
But now I will call this dice roll situation left OR right.
\begin{align}
    \bar{A}_y &=
    p_L \bra{L}A_y\ket{L}
    + p_R \bra{R}A_y\ket{R}
\end{align}
where $p_L$ and $p_R$ are the probabilities of having the left or right holes
open respectively.
You notice that this matces the first two terms of the previous formula.
What you do not find is the second two terms,
so there is no interference.
This is a different physical sitaution.

Those two things are not the same.
If you plot $A_y$ vs $y$,
in the QM superposition,
you get this
[picture]

But if you have the classical probability,
you get two bumps.
[picture]

These two sitautions are very common,
and you need to give these names.
having L and R in superpsotiion quantum mechanically,
we have a \emph{coherent} sum.
But the classical probability situation,
is the \emph{incoherent sum}.
The incorhernet sitaution happens all the time
in classical physics.

But the coherent sum is something very special to QM,
there is no classical analogue for that.

This situation where you have a combination of this,
where some of it is quanutm mehaonical,
but I also don't know what the wave function is,
appears commonly in physics too.

So ther eare two kinds of uncertainty.
There is the quantum uncertainty of the wave function,
butther'es the classical uncertainty of not knowing what the wave function is in
the first place.
There is a formalism to deal with this.

Suppose you have some observablve $A$.
Then
\begin{align}
    \bar{A} &=
    \underbrace{
    \sum_n \underbrace{p_n}_{\text{prob. in $\ket{n}$}} \underbrace{\bra{n} A \ket{n}}_{\text{quantum average}}
    }_{\text{classical average}}
\end{align}
So we have a quantum and a classical average

There's a cute way to write this as a trace.
\begin{align}
    \bar{A} = \Tr\left[
    \left( \sum_n p_n \ket{n}\bra{n} \right) A
    \right]
\end{align}
Is this the same?
It's easy to see,
because to compute the trace,
one way of computing the trace is to take a basis,
for example the same basis $\ket{n}$,
then compute the sandwich of whatever is in the trace and then sum over elements
of the basis,
but you see this is easy to compute.
\begin{align}
    \Tr\left[
    \left( \sum_n p_n \ket{n}\bra{n} \right) A
    \right]
    &=
    \sum_{m} \bra{m}
    \sum_n p_n \ket{n}\bra{n} A\ket{m}\\
    &= \sum_{m,n} \cdots
\end{align}
Then observe the following things.
Then we call that part the density matrix.
\begin{align}
    \rho = \sum_n p_n \ket{n}\bra{n}   
\end{align}
This contains all the information about the system.
It doesn't tell me what the wave function is,
because there's uncertainty about what the wave function is.
But the probabliies are al defined by this $\rho$.
States whree therare different clasical probailitesi of different wave fucntions
is called a \emph{mixed state}.
In situations wher we know what hte wave function is,
they are called \emph{pure dstates}.

To be clear,
\begin{align}
    \bar{A} = \Tr(\rho A)
\end{align}
Remember $\rho$ contains informaiton about all possible wave functions the
system may have.

Are we good up to now?
By the way,
you can raise a question by raising a hand,
but it's better to shout,
just unmute yourself and shout.
You can even go and ask questions in the chat.

\begin{question}
    Can you elaborate on mixed state?
\end{question}
If you know thw ave function of the sytem,
you call it a pure state,
that's what you learnt in QM so far.
A pure state is just a real state.
But then suppose you have a harmonic oscillator
with 30\% chance in the ground state and 70\% chance in the first excited state,
that is a mixed state.
You should not be confused though.
Take hte harmonic oscilator for exmaple.
\begin{align}
    \ket{\varphi} &=
    \frac{3}{5}\ket{0}
    + \frac{4}{5}\ket{1}
    \ne
    \rho
    =
    \left( \frac{3}{5} \right)^2 \ket{0}\bra{0}
    \left( \frac{4}{5} \right)^2 \ket{1}\bra{1}
\end{align}
These states here have the same probabiliteis,
but they are completely different states.
I can even add a phase.
\begin{align}
    \ket{\varphi} &=
    \frac{3}{5}\ket{0}
    + i\frac{4}{5}\ket{1}
    \ne
    \rho
    =
    \left( \frac{3}{5} \right)^2 \ket{0}\bra{0}
    \left( \frac{4}{5} \right)^2 \ket{1}\bra{1}
\end{align}
It's either one or the other.
No matter how smart I am,
I will know a way to measure.

here's an example.

How do we describe th sitaution where both holes are open.
We already have that,
a quantum superpsoition of left and right.
When I compuet the probablity of getting a particle on the sreen,
I get interference tersm.
If I have a phsae,
the probabilities $|\alpha|^2$ and $|\beta|^2$ don't change,
but I get a different interefernece pattern.

That's usual quantum mechanics,
but I can describe the sitaution with desity matrices too
which account for probabilities that arise because I am too dumb to know.

So I can describe the double slit experiment with density matrices too.
For example,
the incoehreent sum is
\begin{align}
    \rho &= \frac{1}{2} \ket{L}\bra{L}
    + \frac{1}{2}\ket{R}\bra{R}
\end{align}
and you can computet the probabilities
\begin{align}
    \bar{A} &=
    \Tr(\rho A_y)\\
    &=
    \Tr\left( \frac{1}{2}\ket{L}\bra{L} + \frac{1}{2}\ket{R}\bra{R} \right) A\\
    &=
    \bra{L}\left( 
    \frac{1}{2} \ket{L}\bra{L}
    + \frac{1}{2} \ket{R}\bra{R}
    \right)A_y
    \ket{R}
    + \bra{L}\left( 
    \frac{1}{2} \ket{L}\bra{L}
    + \frac{1}{2} \ket{R}\bra{R}
    \right)A_y
    \ket{R}\\
    &= \frac{1}{2}\bra{L} A_y \ket{L}
    + \frac{1}{2}\bra{R} A_{y} \ket{R}
\end{align}
Now look at what we thoguht it should be
\begin{align}
    \bar{A}_
    &= p_L\bra{L} A_y \ket{L}
    + p_R\bra{R} A_{y} \ket{R}
\end{align}
so the formalism works!
Let me tell you a couple of probabilities it's to satisfy.

Firstly, $\rho$ is a Hermitian operator.
It's weird,
because it's not an observable,
but it's the state of the system.
\begin{align}
    \rho^{\dagger} = \rho
\end{align}
Can you see this is true?
Sure
\begin{align}
    \left( \sum_n p_n \ket{n}\bra{n} \right)^{\dagger}
    &= \sum_n p_n^* \ket{n}\bra{n}\\
    &= \sum_n p_n \ket{n}\bra{n}
\end{align}
because $p_n$ are real probabilities,
so very straighforward.

Anotherp robability isthat it has unit trace.
\begin{align}
    \Tr \rho = 1
\end{align}
To see this,
\begin{align}
    \Tr \rho &=
    \Tr \sum_n p_n\ket{n}\bra{n}\\
    &= \sum_m \bra{m} \sum_n p_n\ket{n}\bra{n} \ket{m}\\
    &= \sum_{n,m} p_n \braket{m}{n} \braket{n}{m}\\
    &= \sum_n p_n\\
    &= 1
\end{align}
because the total probability is equal to 1.


Finally, $\rho$ is positive definite.
It follows from the fact that $p_n$ has to be positive or zero.
Positive definite means that
\begin{align}
    \bra{\phi}\rho\ket{\phi} &=
    \bra{\phi} \sum_n p_n\ket{n}\braket{n}{\phi}\\
    &=
    \sum_n p_n \braket{\phi}{n} \braket{n}{\phi}\\
    &= \sum_n p_n |\braket{n}{\phi}|^2\\
    &\ge 0
\end{align}

There's one final property.
For some special states, we have
\begin{align}
    \rho^2 = \rho
\end{align}
if and only if $\rho=\ket{\psi}\bra{\psi}$ is a pure state.
You can see this clearly because if
$\rho = \sum_n p_n \ket{n}\bra{n}$
and if $p_n=0$ except $p_{\bar{n}}=1$,
then obviously $\ket{\bar{n}}=\ket{n}$.
There are two things to prove,
because it's if and only if.
It's a simple test.
Suppose $\rho$ is pure,
then
\begin{align}
    \rho^2 &=
    \ket{\psi}\braket{\psi}{\psi}\bra{\psi} = \rho
\end{align}
It's a bit of extra work to go the other way around.

Suppose $\rho^2 = \rho$.
Then 
\begin{align}
    \sum_n p_n \ket{n}\bra{n}
    \sum_m p_m \ket{m}\bra{m}
    &= \sum_{n,m} p_n p_m \ket{n} 
    \underbrace{\braket{n}{m}}_{\delta_{nm}} \bra{m}\\
    &= \sum_n p_n^2 \ket{n}\bra{n}\\
    &= \sum_n p_n \ket{n}\bra{n}
\end{align}
so we know that $p_n^2=p_n$ for all $n$.
There are only two ways this can be true.
Either $p_n=0$ or $p_n=1$.
But probabilities sum to 1,
so only one of these can be $p_{\bar{n}}=1$,
and all the other $p_n=0$.
What does that prove?
In other words,
\begin{align}
    \rho = \sum_n p_n\ket{n}\bra{n} = \ket{\bar{n}}\bra{\bar{n}}
\end{align}

Oh there's a question.
It's just one word: magical.

You're just in awe of the power of density matrices and bras and kets?

By the way,
Landau invented density matrices.
He was two years too young to invent QM,
but he invented density matrices.
Bras and kets,
I'm tired of syaing it's a beautiful thing,
but I hope you can appreciate how the algebra just works without thinking.
It's importnat to understand why it works,
but once you do,
it's easy.

The name of this class is quantum and statistical physics.
Forgoet about the deep stuff.y
You're goign to use distributions of positions and momentum classical,
but the density analogue is density matrices,
where you have distributions over wave functions.


\begin{question}
    So $\rho$ is diagonal?
\end{question}
Well $\bra{n}\rho\ket{n'}$ is a diagonal matrix in this basis,
but in a different basis,
it will not necessarily be a diagonal matrix,
despite describing the same physics.
If the operator was diagonal,
I wouldn't talk about htis operator,
it would be overkill.
In this basiss to define $\rho$,
but may be diagonal,
but in another basis,
it's not diagonal
and it's not obvoius at all it's not a pure state.

\section{TQFT for Majorana Chain}
I told you that
\begin{align}
    V(S^1, \pm) \simeq \mathbb{C}
\end{align}
I also told you the path integral
\begin{align}
    Z(\Sigma_g,eta) = (-1)^{\textrm{Art}(\eta)}
\end{align}
where $\Sigma_g$ is a closed genus-$g$ surface.
I told you that genus $g$ surfaces are in 1-1 correspondence to a space of
quadratic forms.
\begin{align}
    \text{spin structures on }\Sigma_g \leftrightarrow
    \text{quadratic forms }\in Q\left( H_1\left( \Sigma_g, \mathbb{Z}_2 \right),
    \phi\right)
\end{align}
where $\phI$ is a symmetric bilinear form which is the intersection form that
tells you whether two ops are intersecting or not.
And remember
\begin{align}
    q(x + y) = q(x) + q(y) + \phi(x, y)
\end{align}
I asserted the space of quadratic forms are in 1-1 correspondence with the space
of spins structures.

As for that Arf,
I defined something called the Arf invariant of a quadratic form,
so you take the Arf invariant associated with the spin structure.

I just asserted a bunch of stuff because it was just math.

One example was the Torus, and the partition function was 
\begin{align}
    Z(T^2, \eta) &= -1, 1, 1, 1
\end{align}
depending on whether the boundary conditions are
PP, AP, PA or AA.

If you look at the partition function of a disk,
the disk has a unique spin structure I won't write.
\begin{align}
    Z\left( D^2 \right)
    \in
    V\left( S^1, - \right)
\end{align}
where the $-$ means anti-periodic boundary conditions.
I will denote $+$ for periodic.

Last time,
I asserted the following
\begin{proposition}
    Anti-periodic boundary conditions on a circle $S^1$ give a spin structure
    than can be extended to the disk $D^2$.
    These are also called \emph{bounding} spin structures,
    whereas periodic boundary conditions are \emph{non-bounding}.
\end{proposition}
If you draw a circle,
you can think of the fermion as defining a spinor in your space.
Consider 2 situations.
1 is where you have part of the spinors tangent to the circle,
and the other parts are looking like this.
This is the one that gives you anti-periodic boundary conditions,
because as you go around the circle,
the arrows wind around the circle,
and you get a $2\pi$ rotation,
which is a minus sign for fermions.
This can be extended to the disk.

On the other hand,
this spin structure (all pointing out) cannot be extended to the disk.
I don't have the time to give you background on spin structures,
so take this as a bunch of assertions.
If you want to know what's going on behind this,
we have to talk offline.
But if you do have questions, ask.

\begin{question}
    Assume if you go across the line and rotate multiple lines instead of one
    time.
    These are periodic and anti-periodic?
    We're not sure if those are also have the same.
\end{question}
You want me to convince you that some more complicated rotation on that circle
and show the anti-periodic one will be bounding?
Not sure I can do that right now?

\begin{question}
    What's a spin structure.
\end{question}
You start with a tangent bundle,
say in $d$ dimensions.
The tangent bundle is a $SO(d)$ bundle.
At every point in your manifold there is a space of tangent vectors.
If the space if curved then this will be curved.
For fermions, you don't want $SO(d)$,
you want a double cover of this.
You want a $Spin(d)$ bundle.
So $spin(d)$ is the double cover of $SO(d)$.
Basically,
on every patch originally we had a map to $SO(d)$,
so we had some map $\phi:U_\alpha\to SO(d)$.

The manifold is covered by open charts $\left\{ U_\alpha \right\}$

But instead,
you want to map to $Spin(d)$,
and you have to specify which covering of $SO(d)$ you want to cover.
So really you want to pick it with a sign
\begin{align}
    \tilde{\phi} = (\phi, \eta)
\end{align}
where $\eta$  tells which of $SO(d)\to Spin(d)$
we choose.

A spin structure is something that as you go along the manifold,
you're possibly getting rotated by $SO(d)$,
and something.
You can think of that as one basic definition.

\begin{question}
    How can a manifold not have a spin structure?
\end{question}
It might not be possible to choose $\pm$ signs that's consistent everywhere.
I don't have a 2-line explanation,
but what you find is that you want to make sure this lift is consistent on
triple overlaps.
So make sure lift is 

$SO(d)$ bundle,
there are transition functions
$f_{\alpha\beta}\in SO(d)$
which tell you how to go from one coordinate patch to another.
But when you lift it,
you want
$\tilde{f}_{\alpha\beta} = (f_{\alpha\beta},\eta)$,
and you want this consistency identity
\begin{align}
    \tilde{f}_{\alpha\beta} \tilde{f}_{\beta\gamma} \tilde{f}_{\gamma\alpha}
    = 1
\end{align}
and this is 
\begin{align}
    \delta\eta = W_2(TM)
\end{align}
You need manifolds with trivial second Stiefel-limits.

The spin chain is just 1D,
and all 1D manifolds are just circles if they're closed,
and they always admit a spin structure.
I only know an example in 4D of a manifold that doesn't admit a spin structure.
The only example of a manifold
All 1-, 2- and 3- manifolds admit spin structures.

To finish off the TQFT,
we would also want to understand the path integral on this kind of structure.
If I don't put an markings,
it's anti-periodic.
But imagine changing the boundary conditions,
like a branch cut here that makes this APP.
In principle,
these are all maps from $S^1\times S_1$ to $S^1$.
I haven't specified what the maps are,
but show that everything that is consistent you get $(-1)^{\textrm{Arf}}$.
Because these are 1D Hilbert spaces,
it's just going to be a choice of signs.
With all these signs,
you can build path integrals on closed surfaces using the multiplicative rule
and you can always get the Arf invariant.

That was an example of an invertible TQFT,
with the vector space being 1D.

The Majorana chain is an example of an invertible topological phase of matter.

Furthermore,
we have this $\mathbb{Z}_2$ invariant,
which is a $I=P_P P_A$.
If I took a double layer system,
if I had a $-1$ here, I would get $-1$.
That is,
under ``stacking'',
which means take the total Hamiltonian to be $H=H_1+ H_2$
and the Hilbert spaces to be $H=H\otimes H_2$.
If I two of them together,
I get a trivial phase.
This forms a $\mathbb{Z}_2$ group structure under stacking,
and in this case,
the state is its own inverse.

\begin{question}
    The state is its own inverse or the field?
\end{question}
All of the other.
There is some adiabatic path from
$\ket{\psi_A}\otimes \ket{\psi_B}$
to a product state.

\begin{question}
    What is a product state?
\end{question}
This should just be a state where I start with a Fock vacuum
$\ket{0}$,
I go over all sites and a put a fermion on all sites
$\prod_r c_r^\dagger \ket{0}$.
It should be a state with a fermion or not on every site.

\begin{question}
    In the trivial phase of the Majorana phase,
    it's a superconducting state?
\end{question}
The trivial phase,
there are various versions of it,
(1) was the completely empty phase $\ket{0}$.
(2) another phase was the ``atomic'' superconductor,
where on every fermion pair we had an odd number of fermions.

If we break translational symmetry,
we should be able to get an adiabatic path
$\ket{\psi_A}\otimes\ket{\psi_B} \to \ket{0}$,
when we're allowed to break symmetries.

\begin{question}
    Stacking is a way of combining 2 TQFTs?
\end{question}
Here I'm just talking about microscopic systems.
I'm literally saying it's now 2 chains instead of 1 chain.
That's all I mean by stacking here.

As far as the TQFT is concerned,
stacking means I multiply the path integrals.
$Z_{AB} = Z_A Z_B$.

\begin{question}
    What happens to the vector spaces?
\end{question}
There's a subtlety,
I have a fermion parity in the top system,
and a fermion parity in the second system,
but the parity you want to look at is both at the same time $P=P_1P_2$.
The vector spaces are all 1D and labelled by 1D or 2D and all are related by
the same.

\begin{question}
    How can you get a trivial phase just by grouping 2 together.
\end{question}
The starting Hamiltonian is decoupled but the path is definitely going to couple
them.
Maybe it's more precise to say that there's a path of Hamiltonian $H(s)$
and $H(0) = H_1 + H_2$ are decoupled.


\begin{question}
    Does the TQFT make it easier to see this factor,
    or can it be deduced from the microscopic?
\end{question}
No, this doesn't require TQFT, just that there is a $\mathbb{Z}_2$ invariant
that is either $\pm 1$,
and that together they give $+1$.
That fact is mirrored in the TQFT by $Z=(-1)^{\textrm{Arf}(\eta)}$.

Any other question?

\section{Spin bordism group}
One last point about this TQFT story is that I want to introduce the notion of a
spin bordism group.
A bordism is a high dimensional manifold between two manifolds as boundaries.

I can define a bordism group.
2 $d$-manifolds are equivalent to each other if there exists a bordism between
them.
$M_1^d \sim M_2^d$ if there exists a bordism from $M_1^d$ to $M_2^d$.
So that allows me to define equivalence classes $[M^d]$
of manifolds.
Then 
I can define multiplication by disjoint union
\begin{align}
    [M_1^d]\times [M_2^d]=
    [M_1^d \sqcup M_2^d]
\end{align}
And the inverse is just
\begin{align}
    [M_1^d]^{-1} = [\bar{M}_1^d]
\end{align}
So this group is denoted $\Omega^{\text{or}}_d$
to denote that they are oriented manifolds.
Mathematicians have computed what these groups are for each dimension.

For example, in $d=2$,
the bordism group is trivial.
In 3D it's also trivial.
Then in 4D it's $\mathbb{Z}$.
1D is also trivial.

Now consider not just bordism,
but also spin bordism,
which means each manifolds also comes equipped with a spin structure,
so we're going to have not just the manifolds.

Spin bordism:
We have $(M^d,\eta)$.
The spin structure also must extend to the higher dimension manifold,
and we define multiplication and inverse in exactly the same way.
I won't write it out,
because it's just adding spin structure,
and we call this
$\Omega_d^{\textrm{spin}}$,
which mathematicians have also calculate.

In 1D, it's $\mathbb{Z}_2$.
In 2D, it's $\mathbb{Z}_2$.
In 3D, it's trivial.
In 4D it's $\mathbb{Z}$.

What is the meaning of this?
Let's focus on $d=2$.
In $d=2$, the oriented bordism group is trivial,
which means every closed oriented 2D manifold is the boundary of some 3D
manifold.
That means every surface is boundary of a 3D manifold.

For example,
the boundary of a solid torus
is torus disjoint union nothing,
which means the torus is bordant to empty,
which means the torus is trivial as far as the bordism group is concerned.

Every 2D surface can have its interior filled,
and now I have a 3D manifold inside,
and it's a trivial bordism group.

So spin bordism is not trivial,
and that's related to the point that not all spin structures are boundaries.

On the torus,
suppose I have anti-periodic boundary conditions on both loops.
If one of them is anti-periodic that's enough.
Because if that's anti-periodic,
that's enough.
And if the other one is anti-periodic that's enough.
And so AA, AP and PA are all boundaries,
all bound.
But PP does not bound.
If both of these are periodic,
both of these spin structures cannot be extended to the interior bulk.
That's why its bordism group is $\mathbb{Z}_2$.

Already, you see a tight relation between the path integral,
and the relation of these bordism groups.

What happened is that my path integral is actually
for every element in $\Omega_d^{\textrm{spin}}$,
it gives me $\pm 1$ and that was my path integral for my TQFT.

So my TQFT path integral gave me a $\pm 1$.
Actually, I can take an element
$x \in \Omega_2^{\textrm{spin}}$
and $Z(x)=\pm 1$
depending on which bordism group it's in.
So really, the path integral is a map
\begin{align}
    Z: \Omega_2^{\textrm{spin}} \to \left\{ \pm 1 \right\}
\end{align}
That is our TQFT.
A slightly different way so saying it's a homomorphism
from $\Omega_2^{\textrm{spin}}$ to $U(1)$,
denoted
$\textrm{Hom}\left( \Omega_2^{\textrm{spin}}, U(1) \right)$.
And this is deep,
it's called the
\emph{Pontryagin dual} of $\Omega_2^{\textrm{spin}}$.

I'm telling you this because this is the property that generalises.
Manifolds can have background $g$ gauge fields,
where we define gauge field on our manifold.

\begin{question}
    Why $U(1)$?
\end{question}
$U(1)$ is the thing that generalizes.
Sometimes it's not $\mathbb{Z}_2$ but $\mathbb{Z}_m$,
in which case you get a complex phase.
For an invertible TQFT,
the path integral is just a phase,
which is why it's $U(1)$.
For non-invertible TQFT,
there's no such a nice classification.

\begin{question}
    Why do we choose $\pm 1$ here?
\end{question}
In this example it's $\pm 1$ and this example is the
Pontryagin dual.
It's a much more complex structure called an Anderson dual,
and I don't even know what the complex structure is.

\begin{question}
    Does it just have to be a complex phase or a root of unity?
\end{question}
If your bordism group is finite,
then it must be a root of unity.
But if it's a $\mathbb{Z}$ classification,
it's not clear why.
If it's finite order it's obvious.

The fact you have 2 distinct phases trivial and non-trivial,
has a mirror in the language of TQFTs,
the spin bordism group having two elements as $\mathbb{Z}_2$.

\section{Ising model and Jordan-Wigner duality}
Let's write down the transverse field Ising model in 1D.
\begin{align}
    H &=
    -J\sum_{i} \sigma_i^x \sigma_{i+1}^x
    - h\sum_{i=1}^{N}\sigma_i^{z}
\end{align}
where the $\sigma$ are just Pauli matrices.
Let's do an open chain of $N$ sites.
Now there is a global $\mathbb{Z}_2$ symmetry,
which I am going to write as
\begin{align}
    Z &=
    \prod_{i=1}^{N}\sigma_i^z
\end{align}
and this is a global symmetry that commutes with the Hamiltonian and squares to
1.
$[H,Z]=0$.

There are 2-spin systems per site.
Instead of thinking up and down ad being fermion parity even or odd.
So fermion and no fermion.
So up and down can be reinterpreted as fermion and no fermion.
For each site $i$,
I'm going to define 2 Majorana fermions.
\begin{align}
    \gamma_{2i - 1} &=
    \sigma_1^x \prod_{j=1}^{i-1}\sigma_j^{z}\\
    \gamma_{2i} &=
    \sigma_i^y \prod_{j=1}^{i-1}\sigma_i^{z}
\end{align}
which means
\begin{align}
    \sigma_i^z = -i \gamma_{2i - 1}\gamma_{2i} = P_i
\end{align}
which is the fermion parity operator.
And it's a exercise to show that these satisfy the algebraic conditions for the
Majorana operators.
And the product term becomes
\begin{align}
    \sigma_i^x \sigma_{i+1}^x &=
    -i \gamma_{2i}\gamma_{2i + 1}
\end{align}
and the $Z$ symmetry becomes
\begin{align}
    Z = P = \prod_{i=1}^{N}P_i
\end{align}
which is just the total fermion parity.

And if you substitute this into the Hamiltonian,
you find the system is just a Majorana chain.
\begin{align}
    H &=
    - J \sum_{i=1}^{N-1} i \gamma_{2i} \gamma_{2i + 1}
    + h\sum_{i=1}^{N} i \gamma_{2i - 1}\gamma_{2i}
\end{align}
So there is a mapping from Ising model and the Majorana chain.
And you see how the 2 phases of the Ising model corresponds to the trivial and
non-trivial phase of the Majorana chain.


The ferromagnetic phase $J\gg h$ is when everything is aligned  in $x$ and
that's the
topological phase in Majorana language.
Now you can see the origin of the degeneracy,
with chain,
that had Majorana zero modes,
which led us to a double degeneracy in the ground state,
with even vs odd parity.
In the Ising model,
we also have a degeneracy with all point in $+x$ or all pointing $-x$.


And then the state would be
\begin{align}
    \ket{\rightarrow \cdots \rightarrow}
    \pm
    \ket{\leftarrow \cdots \leftarrow}
\end{align}
and $Z\ket{\pm} = \pm \ket{\pm}$,
which in Majorana language means
$P= \pm 1$,
and $P \propto -i\gamma_L \gamma_R = \pm 1$.

In the $J\ll h$ limit,
we have the paramagnetic phase.

Why bother to do all this just to describe the Ising model?
Well no.
There is a non-local relationship.
That non-local relationship is why there is no local operator to distinguish the
ground states.
No local bosonic operator can distinguish the ground states and connect them.
The degeneracy comes from symmetry breaking in the Ising model,
but it's from topology in Majorana.
Spontaneous symmetry breaking
gets mapped by this non-local mapping to topological degeneracy.

\begin{question}
    Isn't there a duality between strong $J$ and weak $J$,
    how can there be different Majorana parities,
    one is topological phase and another one isn't.
    How can they have different parities if they're dual.
\end{question}
Parity in Majorana language is just total $Z$ in Ising model.
This is going to open a whole can of worms.

The Ising model has 2 dualities.
One of them is this Jordan-Wigner duality.
There's another duality,
which is not really correctly stated in the literature which is a
Kramers-Wannier ``duality'' duality where you swap
$J\leftrightarrow J$.
But it's not a true duality
because the ferromagnetic phase has 2 ground states,
but the paramagnetic phase only has 1 ground state.
So the degeneracy isn't mapped right.

However,
there is a duality,
where the Ising model is coupled to an Ising $\mathbb{Z}_2$ gauge field.

The usual ``Kramers-Wannier duality''
already has a problem with the counting of the ground sate.
That counting is corrected by coupling the Ising model with a
$\mathbb{Z}_2$ gauge field.
And that might solve all the paradoxes you might have.

\begin{question}
    If we such a system with Pauli matrices,
    can we characterise every Hamiltonian?
\end{question}
If you stick to 1D,
you can always do this transformation nd you find that the $\mathbb{Z}_2$
symmetry-broken state will always map to the topological phase.
The Jordan-Wigner in higher dimension,
you'd have to convert to higher $d$.

Yu-An has spent the last few years doing this.
Xiao-gang Wen showed this can be transformed from higher $d$ to some duality.

You have to be careful with non-local transformations to make dualities,
you need to know what's local and what's not local.
If you started with Majorana and went backwards,
you would say everyone missed something,
you could just do topological quantum computing with Ising models!
But you have to be careful because the aping is non-local.

\begin{question}
    What are the possible applications?
\end{question}
There's an application in quantum computing to simulate fermions,
and you want to simulate them without having them,
then this duality could help.
This is a baby version of duality.
If you understand this,
you can try to understand all sorts of ore complicated dualities.

\begin{question}
    I've seen this being related to being on the boundary of a toric code,
    is this related to each other,
    or is this a different kind of structure where the toric code is in 2D?
\end{question}
Um. Well we'll get to the toric code,
but if you had a 2D phase of matter and looked at the boundary,
indeed,
you could map the boundary into a 1D transverse field Ising model,
or a Majorana china.
In fact, the toric code has 2 types of boundaries,
rough and smooth,
and those map to the 2 Majorana phases of matter.
All the analysis will be useful.

Any other questions?
That's all I'm going to say for Majoranas in this lecture.
There's a few more important topics.
I gave you some model Hamiltonians for Majoranas.
How do you realise real Majoranas in a real solid state system.
In the homework I'm going to have such as system.

One big direction is how to do you actually realise Majoranas and what are the
physical systems.

\section{Chern Number}
Now we shift gears to go into higher dimensions.
That was the simplest topological phase of matter in 1D.
It didn't require anything fancy except for fermions.
Now it's associated with an invariant called the Chern number.
And now we're in $(2+1)$ dimensions.

We're going to stat off by defining invariants of band structures,
the Chern number,
but it's also sometimes called the TKNN number,
after Thouless, Kohomoto, Nightingale and den Nijys.
This is some invariant of a band.

Let's say we have a free particle
and it exists as some 2D lattice model,
so we have a tight binding model that looks like
\begin{align}
    H &=
    - \sum_{ij} c_i^\dagger c_j t_{ij} + \text{h.c.}
\end{align}
So we have a Brillouin zone.
The momenta are $(k_x, k_y)$ and we identify
\begin{align}
    \vec{k} \sim \ket{k} + \vec{G}
\end{align}
where $\vec{G}$ is the reciprocal lattice vector.
And so you do the quotient
\begin{align}
    \frac{\mathbb{R}^2}{\mathbb{Z}^2} = T^2 = S^1\times S^1
\end{align}
So the Brillouin zone is actually a torus.

The idea of a Chern number is that you have bands
$E$ vs $k$.
Every bad is going to be associated with some kind of invariant 
$C_1,C_2,C_3,\ldots$ which is an integer.
The way you define this integer for every single band is as follows.

Remember that we have Bloch states
\begin{align}
    \psi_{\vec{k}}(\vec{x})
    = e^{i\vec{k}\cdot\vec{x}}
    u_{\vec{k}}(\vec{x})
\end{align}
where these $u$ plane wave Bloch waves are periodic
\begin{align}
    u_k (x) = u_k(x + e)
\end{align}
where $\hat{e}$ is the lattice vector.
The phase of these wave functions can wind around in non-trivial ways and those
are characterised by these Chern numbers.

Mathematically we define a ``Berry connection'',
which is a gauge field defined in momentum space.
\begin{align}
    A_j(\vec{k}) =
    -i \bra{\vec{u}_k}
    \frac{\partial}{\partial k_j} \ket{\vec{u}_k}
\end{align}
The reason it's a gauge field,
is if I redefine the phase of these Bloch states $u_k$,
then it doesn't change it.
That is,
under gauge transformation
\begin{align}
    \ket{u_{\vec{k}}}
    \to
    e^{i f_{\vec{k}}}\ket{u_{\vec{k}}}
\end{align}
the gauge field transforms like
\begin{align}
    A_j(\vec{k})
    \to
    A_j(\vec{k})
    + \frac{\partial}{\partial k_j} f(\vec{k})
\end{align}
and that gauge field defines a gauge-invariant field strength,
but when you integrate that gauge field through the Brillouin zone,
you get an integer invariant.

So let us define the field strength by
\begin{align}
    F_{xy}(\vec{k})i &=
    \frac{\partial A_y}{\partial k_x}
    - \frac{\partial}{\partial k_y}A_x
\end{align}
and then we can define the Chern number as
\begin{align}
    C &=
    \frac{1}{2\pi} \int_{BZ} d^2 k F_{xy}(\vec{k})
    \in \mathbb{Z}
\end{align}
There's an elementary theorem which says if you integrate a field over a closed
surface, 
you always get an integer.

The flux through any closed surface must be an integer multiple of $2\pi$.
Imagine your surface is a sphere,
and your sphere has some flux through it.
Imagine a particle that goes along some path on the sphere.
And this particle has charge under this gauge field.
The phase it's going to pick up is the look integral
\begin{align}
    e^{i\oint A\cdot dl} =
    e^{i\int_R F}
\end{align}
However you could consider the integral of this loop to be the integral of the
complement of that region.
\begin{align}
    e^{i\oint A\cdot dl} =
    e^{i\int_R F}
\end{align}
It's minus because it has the opposite orientation.
So for these to make sense,
it must be that the flux through the entire closed surface must be 
and integer multiple of $2\pi$.

I'll end there.

\end{document}

