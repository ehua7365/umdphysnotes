\section{Luttinger Liquid}
Last time we coupled a U(1) gauge theory with a background gauge field
\begin{align}
    \mathcal{L} &=
    -\frac{k}{2\pi}
    a_\mu \partial_\nu a_\lambda \epsilon^{\mu\nu\lambda}
    +
    \underbrace{
    \frac{8}{2\pi} A_\mu \partial_\nu a_\lambda \epsilon^{\mu\nu\lambda}
    }_{
    q A_\mu J^\mu
    }
\end{align}
There is a global U(1) current
\begin{align}
    J^\mu &=
    \frac{1}{2\pi}
    \epsilon^{\mu\nu\lambda}
    \partial_\nu a_\lambda
\end{align}
This means $2\pi/k$ flux of $a$ carries charge
\begin{align}
    Q_{2\pi/k} = \frac{q}{k}
\end{align}
Charge $\ell$ under $a$ implies that
$2\pi\ell /k$ flux of $a$

From the classical equation of motion
\begin{align}
    \frac{\delta S}{\delta a_0} &= 0
\end{align}
implies that
\begin{align}
    \frac{k}{2\pi}b &=
    -\frac{q}{2\pi}B
\end{align}
which tells us that
\begin{align}
    B &=
    -\frac{k}{q}b
\end{align}
This is another way of seeing that flux of $a$
comes with flux of $A$.

In particular,
$2\pi/k$ flux of $a$ is going to come with
$-2\pi/q$ flux of $A$.

One thing we can deduce from all of this is that this theory actually gives us a
fractional quantized Hall conductance associated with this background $U(1)$.
Let me explain.

\section{Fractional quantized Hall conductance}
So far we deuced $2\pi/k$ flux of $a$
comes with a $-2\pi/q$ flux of $A$.
But we also know that $2\pi/k$ fox of $a$ also comes with
$q/k$ charge under $A$.

So all together,
this tells us that $-2\pi$ flux of $a$
would give us charge of $q^2/k$.
This is precisely what you expect if the system had a Hall conductivity
\begin{align}
    \sigma_H &=
    - \frac{q^2}{k} \frac{1}{2\pi}
\end{align}
If your hall conductance was $\sigma_H = \nu \frac{1}{2\pi}$.

If you inserted $-2\pi$ flux of $A$,
after you inserted that,
you should get a $q^2/k$ charge.
Physically,
if you had an annulus,
inserting the electric field in the angular direction,
which causes charge to flow radially,
and if that is the amount of charge that's accumulated,
that means this is the Hall conductivity.

That means we have effective action given by the Lagrangian density
\begin{align}
    \mathcal{L} &=
    - \frac{q^2}{k}
    \frac{1}{4\pi}
    A_\mu \partial_\nu A_\lambda \epsilon^{\mu\nu\lambda}
\end{align}
Now another way of thinking about this is let's pick our spacetime manifold to
just be $M^3 = \mathbb{R}^3$.
Imagine your system is on space times time,
and space is the Euclidean plane.
Then we don't have to worry about non-trivial topological gauge configurations.
Let's define the part integral
\begin{align}
    Z\left( M^3, A \right)
    &=
    e^{iS_{eff}[A]}
    Z\left( M^3, A=0 \right)
\end{align}
where that effective action is
align
\begin{align}
    &=
    \int \mathcal{D} a\,
    e^{
    i\int_{\mathbb{R}^3}
    \left( 
    \frac{k}{4\pi} a\,d a
    +
    \frac{q}{2\pi} A\, da
    \right)
    }
\end{align}
We're going to shift $a$ to absorb the second term and only get a term that
depends on $A$.
So write
\begin{align}
    S[a, A] &=
    \frac{k}{4\pi}
    \left( a + \frac{q}{k} A \right)
    d\left( 
    a + \frac{q}{k}A
    \right)
    -
    \frac{1}{4\pi}
    \frac{q^2}{k}
    A\, dA
\end{align}
Now shift
\begin{align}
    a &\to
    a^1 =
    a + \frac{q}{k}A.
\end{align}
And now
\begin{align}
    Z\left( M^3, A \right) &=
    \left( 
    \int \mathcal{D} a'\,
    e^{i \frac{k}{4\pi} a'\ , a'}
    \right)
    e^{-\frac{1}{4\pi} \frac{q^2}{k}
    \int_{\mathbb{R}^3 A\, dA}
    }
\end{align}
which means we can write
\begin{align}
    S_{eff} &=
    -\frac{1}{4\pi}
    \frac{q^2}{k}
    \int_{M^3}A\, dA.
\end{align}

One thing that should bother you is that this $q^2/k$ coefficient
is a fracton.
And I told you the coefficient of the CS term should be an integer.
Why is this allowed?
What happened here?

We picked $\mathbb{R}^3$ so we don't have to worry about large gauge
transformations,
we led to the quantization restriction.

Suppose I let $M^3$ be general.
What part of this derivation did we make some wrong assumption that allowed us
to get to this point where the coefficient is not integer?
One issue is that because $a$ is a $U(1)$ gauge field,
there are certain quantisations on $a$.
Fr example,
\begin{align}
    \frac{1}{2\pi}\int_{S^2} f \in \mathbb{Z}
\end{align}
But if you shift $a$ by $\frac{q}{k}A$,
you ruin all these quantisation conditions.

Another useful perspective on what goes wrong is,
this theory if you quantize n a torus has a bunch of ground sates.
Just to regularize it,
you can imagine it has a Maxwell term,
so the actual spectrum of the quantum theory
is a bunch of degenerate ground states,
and a bunch of excited states.
When we integrate out $a$,
and forget about $a$,
we're doing something not quite allowed,
because we're integrating out something that gives rise to ground state
degeneracy,
which you shouldn't do.
It's not a high energy mode anymore.
It's a zero mode in this case.
We shouldn't integrate out $a$
because of the zero does,
so at best we should split and integrate.
But because we picked $M^3=\mathbb{R}^3$,
we don't have to worry.
But it wouldn't be valid if we did it on other manifolds.

What we can conclude from all this is that this $U(1)_{-k}$
CS theory with $q=1$
gives fractional quantized hall conductance of
\begin{align}
    \sigma_H &=
    \frac{q^2}{k}\frac{1}{2\pi}.
\end{align}
If we used SI units,
\begin{align}
    \sigma_H &=
    \frac{q^2}{k} \frac{e^2}{h}.
\end{align}
This theory describes the most prominent fractional quantum Hall states.
They are gapped phase of matter but they have this fractional quantum Hall
conductance.

Note for $k=1$,
\begin{align}
    \mathcal{L}
    &=
    -\frac{1}{4\pi} a\, da
    +
    \frac{1}{2\pi} A\, da
\end{align}
This actually describes a Chern insulator with Chern number 1,
which dis a Chern insulator or integer quantum Hall state.
This describes a fermionic theory because $k$ is odd
In fact the world line of this fermion is described by
\begin{align}
    e^{i\oint_{\gamma} a\, dl}
\end{align}
with $h=\frac{1}{2}$.
Furthermore,
it has a unique ground state.
Also  $\sigma_H=1$.
Also it is gapped.

\section{Gauge-invariance and boundary theory}
This is a boundary theory.

We studied the Chern insulator,
and we know it has a chiral.
Suppose you had an action
\begin{align}
    S &=
    \frac{m}{4\pi} dx^3 \, a\, da\,
    \partial_\mu \left( f \partial_\nu g_\lambda \right)
    \epsilon_{\mu\nu\lambda}\\
    &=
    \frac{m}{4\pi}
    \int_{y=0}
    \int dx\, dt\,
     \right)
\end{align}
To define theory,
restrict $\left. f\right_{\partial M = 0}$
Some gauge degrees of freedom of $a_\mu$
become physical degrees of freedom on body.

To study the edge theory,
consider
\begin{align}
    a_t &= 0
\end{align}
Then requiring
\begin{align}
    \frac{\partial S}{\partial a_t} &= 0
\end{align}
gives
\begin{align}
    \partial_x a_y - \partial_y a_x = 0.
\end{align}
Solve the constraint
\begin{align}
    a_i &= \partial_i \phi
\end{align}
So the action is
\begin{align}
    S &=
    \frac{m}{4\pi}
    \int d^3 x\,
    a_i \partial_t a_j \epsilon^{ij}\\
    &=
    \frac{m}{4\pi}
    \in d^3x
    \partial_ii \phi \partial_t \partial_j \phi\\
    &=
    \frac{m}{4\pi} \int_{V=0}
    dx\, dt\,
    \underbrace{
    \partial_x \phi \partial_t
    }_{pq  - H}
\end{align}
but since $H=0$,
the velocity of edge modes = 0.
Instead could have set
\begin{align}
    a_\tau &= a_t + v_a_x = 0
\end{align}
which implies
\begin{align}
    S &=
    \frac{m}{4\pi} \int_{y=0}
    dx\, dt\,
    \left( \partial_t + v\partial_x \right)\phi\cdot
    \partial_x \phi
    + v'\partial_x \phi \partial_x \phi
\end{align}
so the equation of motion is
\begin{align}
    \partial_t \phi + v\partial_v \phi &= 0
\end{align}
Modes are chiral with velocity $v$

\section{Quantizing chiral boson theory}
\begin{align}
    S &=
    \frac{m}{2\pi}
    \sum_{k>0}
    dt\,
    \left( 
    ik \phi_k dot{\phi}_k
    - vk^2 \phi_k\phi_k
    \right)
\end{align}
and because $\phi$ is real,
\begin{align}
    \phi_k &= \phi_{-k}^{*}
\end{align}
Because $\phi_k$ with $k>0$ are canonical coordinates,
The canonical momenta are
\begin{align}
    \pi_k &=
    \frac{\partial S}{\partial \dot{\phi}_k}\\
    &= \frac{m}{2\pi} ik \phi_{k}
\end{align}
Then
\begin{align}
    \left[ \phi_k, \pi_{k'} \right]
    &=
    i\delta_{k, k'}
\end{align}
which gives
\begin{align}
    [\phi_k, k'\phi_{k'}]
    &=
    \frac{2\pi}{m}\delta_{kk'}
\end{align}
We can define the density as
\begin{align}
    \rho &=
    \frac{1}{2\pi} \partial_x \phi
\end{align}
sot then
\begin{align}
    \left[ \rho_k, \pho_{k'} \right]
    &=
    -\frac{1}{4\pi^2} kk' \left[ \phi_k, \phi_{k'} \right]
\end{align}
and finally we have
\begin{align}
    \left[ \rho, \rho_{k'} \right]
    &=
    \frac{1}{2\pi m} k \delta_{k+k',0}
\end{align}
and if you have an operator that satisfy this,
you have a U(1) Kac-Moody algebra.

In real space,
if you just Fourier transform this equation,
you get
\begin{align}
    \left[ 
    \rho(x), \rho(y)
    \right]
    &=
    \frac{i}{2\pi m}
    \delta'\left( x - y \right)
\end{align}
which means that
\begin{align}
    \left[ \rho(x), \phi(y) \right]
    &=
    -\frac{i}{m} \delta\left( x - y \right)
\end{align}
You can actually integrate it one more time
to get the real space commutation relation for $\phi$.
This is telling us that
\begin{align}
    \left[ \partial_x \phi(x), \phi(y) \right]
    &=
    -\frac{2\pi i}{m} \delta\left( x - y \right)
\end{align}
and then
\begin{align}
    \left[ \phi(x), \phi(Y) \right]
    &=
    -\frac{i\pi}{m} \sgn\left( x - y \right)
\end{align}
And the derivative of that gives you the delta function.
This is a crucial commutation relation.
This means the field $\phi$ doesn't even commute with itself at different points
in space.
Usually it commutes with itself in different points in space.
Or anti commutes if fermionic.
No matter how far apart,
it doesn't commute with itself at different points.
This is the manifestation of the fact that particles have fractional statistics.
If you move world lines across each other gives fractional statistics,
and this is kind of a manifestation of that in the edge theory.

If you write the Hamiltonian in momentum space,
\begin{align}
    H &=
    2\pi m V
    \sum_{k> 0}
    \rho_k^\dagger \rho_k
\end{align}
There's some restriction on $m$ and $V$,
because you want it to be bound from below.
In order for $H$ to be bounded from below,
you're going to need $mV>0$ for stability.

This is interesting.
Just from the sign of $m$,
you can determine the sign of the velocity.
The direction of the propagation is determined by the sign of $m$.

So we basically completely understand this theory so far.
We have these modes with different momenta $k$,
and we can occupy different states at different modes.
The point is that this Hamiltonian looks like decoupled harmonic oscillators.

For each $k>0$,
there is a $\rho_k$ that satisfies $\rho_k^\dagger = \rho_{-k}$
and the commutation relations
\begin{align}
    \left[ \rho_k, \rho_k^\dagger \right]
    &=
    -\frac{1}{2\pi m} k\\
    \left[ \rho_k, \rho_{k'}^\dagger \right] &= 0
    \qquad\text{if }k \ne k'
\end{align}

So these $\rho_k$, $\rho_k^\dagger$ are creation and annihilation operations
for oscillator modes.


I can have the ground states of some bulk that is described by a CS theory,
ad the edge corresponds to these Luttinger Liquid modes.
They are like phonons,
with density fluctuations near the boundary of the system.
These can happen at arbitrarily low energy,
because you can have arbitrarily long wavelength excitations.

This $\rho_k$ excitations describe overall neutral density fluctuations
at the edge of the system.
If you let it go,
these excitations will travel in a particular direction along the boundary.

\begin{question}
    How is this associated with fractional statistics?
\end{question}
That's the next thing I'm going to say.

\begin{question}
    Has this been measured experimentally.
\end{question}
So far what would you measure?
You perturb it and watch it move in a chiral fashion.
Not really,
you don't have that much control,
to perturb it and image it at a  resolution high enough to see they move in a
particular direction,
so no that has not been directly seen.

Remember this overall theory has overall U(1) charge conservation associate
with the $A$ that we couple to.
Not only should we have neutral excitations,
which is what this $\rho_k$ are,
but there should also be charged excitations,
that is,
operators that carry charge.
Let's find what these operators are in this theory.

So operators that create charged excitations.
For example,
let's look for an operator $\Psi^\dagger$
that creates charge $1$.

Now in fractional quantum Hall context,
this $\Psi$ is the electron operator,
because our fundamental constituent that carries charge of unit 1 is really the
charge $e$ operator,
which is the electron operator.
What do we demand of an operator that carries charge 1?
It should have the commutation relation
\begin{align}
    \left[ \rho(x), \Psi^\dagger\left( x' \right) \right]
    &=
    \delta\left( x - x' \right)\Psi^\dagger(x')
\end{align}
You can think of this as the definition of an operator with charge 1.
Because it give you back the operator with density with coefficient 1.
And it turns out
\begin{align}
    \Psi \propto e^{im\phi}
\end{align}
and you can see this from the commutation relations for $\phi$ and $\rho$.
We previously learnt that
\begin{align}
    \left[ \rho, \phi \right]
    &=
    \frac{i}{m}
    \delta\left( x - y \right)
\end{align}
which implies
\begin{align}
    \left[ \rho, e^{im\phi} \right]
    &=
    \frac{i}{m}
    \left( im \right)
    e^{im\phi}
    \delta\left( x - x' \right)
\end{align}
I may have made a sign error.
But this is the operator that creates a charge 1 operation.
\begin{align}
    \Psi \propto e^{im\phi}
\end{align}
We can try to figure out what its statistics are.
If I look at
\begin{align}
    \Psi(x)\Psi(y)
    =
    e^{im\phi(x)}
    e^{im\phi(y)}
    =
    e^{im\phi(y)}
    e^{im\phi(x)}
    e^{-m^2\left[ \phi(x), \phi(y) \right]}
\end{align}
so then
\begin{align}
    \Psi(x) \Psi(y)
    &=
    \Psi(y) \Psi(x)
    \underbrace{e^{+im\pi \sgn(x - y)}}_{
    \left( -1 \right)^m
    }
\end{align}
so what you learn is that
\begin{align}
    \Psi(x) \Psi(y)
    &=
    \Psi(y) \Psi(x)
    (-1)^m
\end{align}
so if $m$ is odd then $\Psi$ is a fermionic operator,
and if $m$ is even then $\Psi$ is bosonic.

So the microscopic degrees of freedom that carry charge 1 are fermions.

I'm going to call this $\Psi$ the electron operator from now on.
Let's calculate the electron correlation function.
First you need to calculate the correlation function of the boson.
\begin{align}
    \left\langle
    \phi\left( x, t \right)
    \phi\left( 0, 0 \right)
    \right\rangle
    &=
    -\frac{1}{m}
    \ln
    \left|
    x - vt
    \right|
    +
    \text{constant}
\end{align}
That's the result for this correlation function.
I'll put it in the homework to derive this operator.
That means the Green function is
\begin{align}
    G(x, t) &:=
    \left\langle
    T\left( 
    \Psi^\dagger(x, t)
    \Psi(0, 0)
    \right)\\
    &=
    e^{m^{2}\left\langle\phi(x, t), \phi(0, 0)\right\rangle}\\
    &\propto
    \frac{1}{\left( x - vt \right)^m}
\end{align}
so this Greens function has a non-trivial integer exponent 4$m$
For the free fermion Chern insulator,
$m$ is just 1.
The fact that the exponent here is raised to $m$
signals some much more non-trivial correlations that go beyond free fermion.

Let me write the result in momentum space we will use later.
\begin{align}
    G(k, \omega) &=
    \frac{\left( vk + \omega \right)^{m-1}}{
    \omega - vk + i\sgn(\omega) 0^+
    }
\end{align}
where $0^+$ is a positive infinitesimal used to regularize things,
which you set to zero at the end of the calculation.
Then the density of states is the imaginary part of the Greens function at
$k=0$.
\begin{align}
    N(\omega) &\propto
    \Im G(k=0, \omega)
    \propto
    |\omega|^{m-1}
\end{align}

Tunneling experiments.
You could try to tunnel into these fractional quantum hall edge modes,
and have tunnelling from  a metal from the outside.
In these experiments,
you can measure the current vs voltage across the gap.
That $I$-$V$ characteristic measure the density of states.
In particular,
\begin{align}
    \frac{dI}{dV} \propto N(v) \propto |v|^{m-1}
\end{align}
The prediction is that because you have this chiral Luttinger liquid with
nontrivial $m$,
the tunneling conductance should be quantized and non-linear.

This is a fairly traditional condensed matter experiment.
It was tested in experiment,
and it looks  like there is some non-linear conductance,
but it doesn't seem to be quantized.
There's all sorts of messy physics that ruins the quantization.
We think we know why that extra messy physics comes in.
Because I'm focusing on field theory,
I didn't set up the stuff to tell you why.

But at the end of the day,
you have some confining potential that confines your electron to some disk.
It could be sharp or smooth.
In the experiments,
it's quite smooth,
and when it's smooth,
you have extra modes that contribute to the $dI/dV$ curve
and that ruins the quantization.
In principle,
if you have sharp wells,
at low enough energy and temperature,
you should see this effect.

\begin{question}
    Smooth compared to what length scale?
\end{question}
In fractional quantum Hall,
there is a characteristic length called the magnetic length
\begin{align}
    \ell_B \sim \frac{1}{\sqrt{B}}
\end{align}
So it's smooth compared to the magnetic length.
