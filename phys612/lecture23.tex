\section{Recap}
Suppose you have a box at temperature $T$.
Maxwell equations gives you $\vec{A}(\vec{r}, t)$,
which has a lot of degrees of freedom.
The Lagrangian is
\begin{align}
    L_0 &=
    \frac{1}{8\pi} \int d^3r \left[ 
    \left( \frac{\partial A}{c\partial t} \right)^2
    - \left( \vec{\nabla}\times \vec{A} \right)^2
    \right]
\end{align}
There is some exercise to show this is equivalent to Maxwell's equations,
but that is a classical physics problem.
You could keep going and find the classical Hamiltonian,
which you know how to find and is a function of the degrees of freedom $\vec{A}$
and their canonical momentum $\Pi$.
\begin{align}
    H &= 
    \frac{1}{8\pi}
    \int d^3 r\, \left[ 
    \left( 4\pi c \right)^2 
    \Pi^2 (r, t)
    + \left( \vec{\nabla}\times\vec{A}  \right)^2
    \right]
\end{align}
where the canonical momenta are
\begin{align}
    \Pi(\vec{R}, t) &=
    \frac{\partial L}{\partial A(\vec{r}, t)}
\end{align}
It's an infinite-dimensional phase space but that's okay.
It's a quadratic Hamiltonian,
so you can carry out all the calculations.
The way to do this is that you have to go to normal coordinates.
If I had only terms like $A^2(\vec{R})$,
it only couples to itself.
But since there are derivatives of $A$,
I have terms like
$\left( A(\vec{r} + d\vec{r}, t) - A(\vec{r}, t) \right)^2$,
which is very similar to the chain of masses connected by springs
with $-k(x_1 - x_2)^2$.
If you remember the normal coordinates of the mass and springs system,
you just take the Fourier transform to decouple the Fourier modes to get the
normal coordinates.

Again,
for the electromagnetic field,
I can write the transformation to normal coordinates like
\begin{align}
    A(\vec{r},t) &=
    \sqrt{4\pi c}
    \sum_{\vec{k}}
    e^{i\vec{k}\cdot\vec{r}} Q(\vec{k}, t)
\end{align}
where $Q$ are the cavity modes inside the box,
and the wave vectors are of the form
\begin{align}
    \vec{k} &= \frac{2\pi}{L} \vec{n}
\end{align}
where $\vec{n}$ are vectors with integer components.
Then the canonical momenta are
\begin{align}
    \Pi(\vec{r}, t) &=
    \frac{1}{\sqrt{V}\sqrt{4\pi c}} \sum_{\vec{k}}
    e^{i\vec{k}\cdot\vec{r}} P(\vec{k}, t)
\end{align}
and now the degrees of freedom are decoupled and I can calculate the partition
function in my head.

When we found the Lagrangian,
we picked a certain gauge,
that is divergence of the potential is zero and the curl of $A$ is zero,
so what extra constraints do we have?
\begin{align}
    \nabla\cdot A &= 0
    \implies
    \vec{k}\cdot\vec{Q}(\vec{k}, t) = 0
\end{align}
which means one less degree of freedom,
so electromagnetic waves are not longitudinal.
It's hard to do the counting,
but I start with a real vector $A$,
and I'm left with a complex $Q$,
so the degrees of freedom doubles,
but that can't be.
The trick here is that because $A$ is real,
that implies something about these $Q$'s.
\begin{align}
    \vec{Q}(k) = \vec{Q}^*(-k)
\end{align}
which is because if you take the complex conjugate of the definition,
you should get the same thing for $-k$.
And the same thing applies for the $\vec{P}$.
One way of thinking is that if I know the real value of $Q$,
I also know its imaginary value,
so I don't double the number of degrees of freedom,
it's the same.

This transformation is canonical because it satisfies certain conditions you
should know from Chacko's class.
I am allowed to do this and what's going to happen is that I'm going to find out
then what my Hamiltonian is going to look like in terms of $P$ and $Q$.
\begin{align}
    H &= \frac{1}{2} \sum_{\vec{k}}\left[ 
    \vec{P}(\vec{k}, t)\cdot\vec{P}^*(\vec{k}, t)
    + c^2 k^2 \vec{Q}(\vec{k}, t)\cdot \vec{Q}^*(\vec{k}, t)
    \right]
\end{align}
Before the $A$'s of neighbouring parts are coupled,
but now they are not.
A valid $k$ is not next to any other valid $k$,
just a sum of completely independent parts,
one for each $k$.
One harmonic oscillator for every value of $k$.

I don't know if you know that,
but EM an all that complication with curls and divergences,
is just a bunch of this.

Do you recognize this?
\begin{align}
    H &= \frac{1}{2}p^2 + \frac{1}{2}q^2
\end{align}
That's a harmonic oscillator.
This is for many oscillators.
\begin{align}
    H &= \sum_{i}\frac{1}{2}p_i^2 + \frac{1}{2}q_i^2
\end{align}
And the EM Hamiltonian is just an infinite number of oscillators.

Interesting thing to observe is that it tells us the frequency of each mode.
$c^2 k_^2 = \omega^2$.
If you calculate the velocity of the waves,
you get the velocity of the light $c$,
which is the same for any wavelength of light.
It's much like low-momentum phonons.

Note that I could have written
\begin{align}
    \vec{P}\cdot\vec{P} =
    \left( \Re P \right)^2 + \left( \Im P \right)^2
\end{align}

So I'm done right?

I can find classically what the average energy is going to be.
By equipartition,
I should have $\frac{1}{2}k_BT$ for every quadratic term in the Hamiltonian.
\begin{align}
    E &= \sum_{\vec{k}}\frac{1}{2}k_B T\left( 
    \underbrace{2}_{\textrm{``kinetic'' + ``potential''}}
    + \underbrace{2}_{\textrm{polarization}}
    \right) = \infty
\end{align}
Only the transverse polarizations survives because
$vec{k}\cdot\vec{Q}=0$,
so you only have 2 polarizations.
Classically,
the energy does not depend on the frequency of the oscillators.
Differently from the phonons though,
this never ends,
because the maximum $k$ you could have is $\pi/a$,
but there is no lattice spacing $a$,
so this sum diverges!

It diverges because it can have modes of arbitrarily small wavelengths and each
one contributes exactly the same.

And so of course,
this is the famous ultraviolet catastrophe.
That's how quantum mechancis was invinented.
They did this calculation,
the result was not just slightly wrong,
but profoundly wrong.
And there you go.

There are some steps,
you may ask why you should believe,
but there is a reason to believe that this result is off because it's infinitely
off.

The reason is this.
Suppose now that you calculate the energy for a particular magnitude of
$k=|\vec{k}|$,
which corresponds to many different $k$ on a sphere of that radius.
Then you could try to calculate the energy density.

It's hard to do the sum because it lives on lattice,
but for large $k$,
you can do an approximation.

\begin{align}
    \frac{E}{V} &=
    =
    \frac{2k_B T}{2\pi^2} \int_{0}^{\infty} dk\, k^2\\
    &=
    \frac{8 k_B T \pi}{c^3} \int_{0}^{\infty} d\nu\, \nu^2
\end{align}
and if we define the Rayleigh-Jeans density
\begin{align}
    u_T(\nu) &= \frac{8 k_B T \pi}{c^3} \nu^2,
\end{align}
we find $u_T(\nu)$ is a quadratic.
At the time,
people did precise measurements of this thing.
I don't know how did this,
but I don't know how they measured the energy so precisely.
But they did it,
and then know that the correct curve was not quadratic.
They knew that it increased, peaked,
then decreased.

However,
they kind of agreed in the low-$\nu$ limit.
So they got something right.

And we know why.
It's because small frequencies corresponds to bulk quantum energies in the
regime
\begin{align}
    k_B T \gg \hbar \omega.
\end{align}
However, for large $\nu$,
this is violated,
and those high-frequency modes are actually frozen,
and we no longer satisfy the equipartition theorem.

That just tells you our story nowadays.

Let's take this Hamiltnoian here and don't treat it classically.
The same way that you have a particle moving on line,
you make the coordaintea nd mometa ot be operaotr,s
adn you know they have acertain commutation relation,
all that thing holds true,
except that you have an $a$ and $a^\dagger$
for every polarziation Foufier mode.
I'm not going to do this because we don't have the time,
but this is what we call \emph{quanutm field theory},
where you take a classical field,
apply quantum mechanicss to it.
the states of hte field are interpeted as particles,
like photons for example.
The particles ocrrespdongin to quantize leectromeatice field sare like phonons,
ecept hte lattice is the EM field.

To keep long story short,
quantum mechanically,
I'm going to have a Hamiltonian exactly like this,
except it doesn't change the fact I have a bunch of quantum harmonic
oscillators.

First of all,
let's find what the eigenstates of this Hamiltnoian are.
It could be that every harmonic oscilllator is in the ground sate,
the minimum amount of energy.
That's goingto have some energy.
Imagine if I take all of hte modes.

And I say only for that particular oscilators and iwht this poarlizatoin here,
I'm going to t tae the stateo f that oscillatora nd move it to the first excited
state.
Howmuch am I going to raise the energy?
It's going to be $\hbar\omega_k_1 = \hbar c k_1$.
I state like this,
we have created a photon of wavelength $2\pi/k_1$.
This is identical to what we have if we had a particle moving wit momentum
$\hbar k$.
This is the energy of the particle $cp_1$,
and rememver relativity,
this i the energy of a massless particle,
and htat's why we say it has zero mass.

What if I take that very same oscillator and I raise that energy level to the
second excited state?
Because the energy levels are equally spaced,
I would just get $2\habar\omega_{_k_1}$.

I could get two different oscillators,
and that would ocrrespond to the sum of theenrgies.
The fact hte nerg is just the sum of the enrgies of each one,
it means they are not interacting,
no potential enegy between them and that's it.

That's hte beginning of quantum electrodynamics,
let someone else teach you that.
But now I just want to compute the paritiotn function.

You've done this a million times.
Look at your notes.
After that we considered the case of phonos,
just a bunch of harmoinc oscilaotrs.
Here again,
we hve a bunchof harmoinc oscillators.
We're doing statistical mechanics of quantum field theory,
it's veyr fancy stuff.
Anyway,
what is the parittion function?

I have to take htis Hamilontina here,
put some hats on them and do the sum.
\begin{align}
    Z &=
    \Tr e^{-\beta H}\\
    &=
    \prod_{\vec{k},{\alpha}}
    \sum_{n_k=0}^{\infty}
    e^{-\beta \hbar \omega \left(n_{k} + \frac{1}{2}  \right)}\\
    &= \prod_{\vec{k},\alpha}
    \frac{e^{-\beta \omega_{\vec{k}}} \hbar/2}{%
    1 - e^{-\beta \omega_{\vec{k}}\hbar}}
\end{align}
where $\alpha=1,2$ are polarization indices.
To convince yourself it's right to take products,
\begin{align}
    \Tr e^{-\beta \left( H_1 + H_2 \right)} &=
    \Tr\left[
    e^{-\beta H_1} e^{-\beta H_2}
    \right]\\
    &=
    \sum_{n_1,n_2} \bra{n_1} e^{-\beta H_1} \ket{n_1}
    \bra{n_2} e^{-\beta H_2}\ket{n_2}
\end{align}
where $H_1$ and $H_2$ are independent.
Let me give you some advice.
Ther'es just so much more I could explain,
but I'm just feeling the time here.
Ifyo usee a large expression that doesn't have meaning to you hard to digest,
you can break it down,
by working out the trivial cases.
After you do 2,
everything becomes clear.
If you do this tomrrow,
and the day after,
and then it becomes eh,
not a big deal this infintie oscillaotrs,
it takes 30 years.
The alternative is ot manipulate formulas using rules no one told you,
and you'll feel all around.

And now we can calculate the Helmholtz free energy.
\begin{align}
    F(T, V) &=
    -k_B T \ln Z\\
    &=
    -k_B T \sum_{\vec{k},\alpha}\left[ 
    \ln\left( 1 - e^{-\beta\omega_k\hbar} \right)
    + \frac{\beta\hbar\omega_k}{2}
    \right]
\end{align}
By the way,
we invented QM becauseo f hte embarassing divergeence,
buti fy ou calcualte the ground stae of the infinite harmonic oscillaotrs,
you get $\hbar\omega/2$,
but how many ar ethere?
Infintiely many.
This theory predicts the enrgy density of hte vacuum is infinite.
That's bad right?
But that's hte predition.
We do't wrry about it,
is because if you add a photon,
you add energy,
so hte infinite amount of energy goes on both sides of the quation,
but we can ignore.
So delete that second term.
\begin{align}
    F(T, V) &=
    -k_B T \sum_{\vec{k},\alpha}\left[ 
    \ln\left( 1 - e^{-\beta\omega_k\hbar} \right)
    \right]
\end{align}

That is the energy by the way people on the internet want ot harness to move
spaceships or someting.
The rest is just like phonons.
The dispersion is not linear forever.
Here htere is no maximum value of $\vec{k}$.
Hopefuly thisp art that is temperature dependent.

There's one step I'm going to do that I'm going to defy detail.
I want ot transform this sum into an integral,
by an approxiamtion,
and when I do this,
I claim the answer is.
\begin{align}
    F(T, V) &=
    2k_B T V \int d^3k\, \ln\left( 1 - e^{-\beta \hbar ck} \right)
\end{align}
And this $2$ comes from the polarizations.
The free energy is an extensive quantity,
so you should expect it to be proportional to volume,
which makes sense.

Then we're going to move to the next topic of quantum gases next class.
Monday we have office hours.
