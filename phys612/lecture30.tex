\section{Ideal Bose gas}
\begin{align}
    N &= gV \int \frac{d^3k}{\left( 2\pi \right)^3}
    \frac{1}{\frac{1}{Z} e^{\beta E_{k}} - 1}
    + N_0\\
    &=
    \frac{gV}{\lambda_T^3} \underbrace{g_{3/2}(z)}_{\zeta(3/2)\approx 2.6\ldots}
    + N_0
\end{align}
That's it for now.
Let's do phase transitions.

The interesting stuff happens when the particles interact with each other.
Today we're going to look at one particular thing,
study something phenomenologically.
I'm going to give you some equations of state and ask you what doest hat men.
Later on,
I'll give you Hamiltonians and ask you about phase transitions.


\section{Phase transitions}
The thermodynamics of phase transitions is unjustly complicated,
it's not that complicated,
but it does require you to have good drawings.
You can download the drawings in my notes now.
I'll do my best to do good drawings,
but they're better in my lecture notes.

Example: van der Waals.

This is phenomenological,
there is no Hamiltonian for it.
\begin{align}
    P
    &=
    \frac{N k_B T}{V - bN} - \frac{a}{V^2}
\end{align}
We have two extra parameters $a,b$.
More parameters means you can fit your data better.
There's an interpretation that $b$ is the volume of a molecule.
And $a$ is motivated by why not.
There are dozens of equations of state,
some better some worse.

Let's start making pretty drawings.

I'm going to draw isothermal.
At high temperature,
this is very close to being an ideal gas.
But then I lower the temperature,
And it starts looking like this.

Each of those curves is an isothermal.
From the start,
I want to point down one fact.
Look at that region.
As I increase the volume the pressure goes up.
Does that make sense to you?
This is just nuts.

Imagine a system with that property.
I put it next to the system with a movable wall.
The pressure exerted on my system s going to go down.
And it goes further and further until it shrinks to nothing.
So this region here is just unstable.

Even if I don't couple my system.
If just one region happened to have a pressure a little larger here.
Since the more you push the lower the pressure,
so the region that had a slightly smaller pressure gets shrunk to nothing.
So nothing can exist in this region here
\begin{align}
    \left.
    \frac{\partial P}{\partial V}
    \right|_{T,N}
    > 0
    \text{ (unstable)}
\end{align}
So you should not believe the equation of state here.
Suppose I decrease the pressure from high pressure,
and it expands,
keeping the temperature fixed.
What happens when it comes here?
Is it possible for my system to have a particle temperature and have it like
this?
What happens?
It's unclear.
One attitude is to say the equation is wrong.
But no it's not.
I'll explain what happens.

The best argument is the following.

I'm going to give different names along this curve.

Everything is in this picture.
There are no equations.

[picture]

If I put some pressure $P=P_C=P_I=P_K$ by booking a book weight on the cylinder,
what's $V$?
The equation of state says there are 3.
We know it's not going to be $P_I$,
so is it $P_C$ or $_K$?

Let us use the pressure ensemble.
Consider the probability density
\begin{align}
    Z
    &\sim
    e^{-\beta\left( E - TS + PV \right)}
    =
    e^{-\beta G(T, P, N)}\\
    \rho\left(q, p, V \right)
    &\sim
    e^{-\beta\left( H(q,p) + PV \right)}
\end{align}
where $G(T, P, N)$ is the Gibbs free energy.
Since $G$ is the Legendre transform of $E$,
we can write a few things.
\begin{align}
    \left.
    \frac{\partial G}{\partial T}
    \right|_{P, N}
    &=
    -S\\
    \left.
    \frac{\partial G}{\partial P}
    \right|_{T,N}
    &=
    V\\
    \left.
    \frac{\partial G}{\partial N}
    \right|_{T,P}
    &=
    \mu
\end{align}
We want to minimize $G(T, P, N)$ obeying other constraints.
So I'm going to compute which has the largest Gibbs free energy.
They both have the same $P$ and $N$.
So which one is larger?

Consider the integral
\begin{align}
    G_K - G_C
    &=
    \int_{C}^{K} V\, dP > 0
\end{align}
so $C$ is stable.
Consider the integral

\begin{align}
    G_M - G_E
    &=
    \int_{E}^{M} V\, dP > 0
\end{align}
so $M$ is stable.

Hence we have the equal area construction.

Chocolate can be solid.
Warm it  up and everything in between.
There's chocolate of all consistencies.
There is no water of all consistencies.
It's either water or ice.

This phenomenon here is called a phase transition.
This particular one is called a first-order phase transition.

There is a thermodynamic quantity that is actually discontinuous.
Sometimes it's the derivative that is distcontinuous,
and that's a second order.

There's a region in the $P$-$V$ plot where the system is metastable.
Many things metastable are actually metastable.
The famous example is glass.
At small enough temperatures,
the ground state is messy.
The problem is ordering.
glass is someting rigorously speaking is metastable.

The way you make iron is that you melt it,
pour it the shape you want and cool down.
The atoms of iron don't have time to oraganise themselves into structure at
large scalces.
So actual pieces of metal have defects,
and so regular metals have defets.
Take a piece of metal,
hammer it,
where does it break?
It's defects!

How do yo uget rid of defects?
You heat it up,
hammer it,
force the atoms.
Then you make swords.

Funny story.
They all sent their kids to school.
They aid they were taking 2 blacksmithing classes 2 semesers,
so don't joke.

Another story.
Let's do the following experiment,
I take my substance,
I instead of fixing the pressure,
I fix the volume.
It's up to the fluid to do whatever.
What happens to the volume here?
There is no truly stable state in that region.
So What is going to happen?
What happens is phase separation.
If your volume is this,
your system can either have a much bigger volume or a much smaller volume.
Half of your molecules are going to be a state with lower volume,
half are going to be in a state of higher volume.
Inside the box you're actually going to have regions where the density is much
larger.
You see this happen many times.
Go to your fridge,
you can control the temperature to be zero celsius,
whatever farenheit that is.
You're going to have little bits floating.
This coexistence point,
you can have two phases at the same time.
You can have ice and water and both.
Is it true that if I start at zero detrees,
slightly below zero,
increase the temperature a little bit and it melts?
Yes,
but it doesn't happen simultaneously.
You can not do this easily and I will show you why not.

Consider
\begin{align}
    \left| \frac{\partial G}{\partial T} \rigth|_{P,N} &= -S
\end{align}
Let me fix a particular $P$ and $N$.
I'll have to consider the Gibbs free energy here and the Gibbs free nergy there,
(between two close isotherms) which would give me the entropy.
Now consider that on the line.c
When I calcualte $\frac{\partial G}{\partial T}$ on one sidse and another side,
I get a completely different answer.
Here I get a different entropy
even though I am coexisting.

THe entropy of ice at zero and the entropy of ice at zero celsius is different.
Which one is larger?
You have more ways of aranging the liquid.
So if it's $10^{-3}$ below zero it's ice,
but if it's $10^{-3}$ above zero it's water.
But it doesn't happen straight away.
You have to put heat into it.

Same thing for boiling water.
It transforms from liquid into gas.
It happens at 100 degrees celsius.
If it's 101 it's steam.
If it's 99 it's water.
It takes a long time to get from water to steam,
you need to put in heat.
The temperature doesn't charge,
it takes a long time for that little bit of tempeature.
That's called the latent heat.

If you have more phases,
you need more dimensions.

We'll talk about the phase diagram of water next time.
