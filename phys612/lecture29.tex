\section{Ideal Fermi Gas}
The potential is
\begin{align}
    \Omega\left( T, V, \mu \right)
    &=
    -k _B T g
    \int \frac{d^3k}{\left( 2\pi \right)^3}
    \ln\left[ 
    1
    +
    e^{-\beta\left( E_k - \mu \right)}
    \right]
\end{align}
so the density is
\begin{align}
    \frac{N}{V}
    &=
    -\frac{\partial}{\partial \mu}
    \Omega\\
    &=
    g \int \frac{d^3k}{\left( 2\pi \right)^3}
    \underbrace{\frac{1}{e^{\beta\left( E_k - \mu \right)} + 1}}_{n(k)}
\end{align}
so the energy density is
\begin{align}
    \frac{E}{V} &=
    g \int \frac{d^3k}{\left( 2\pi \right)^3}
    \frac{E_k}{e^{\beta\left( E_k - \mu \right)} + 1}
\end{align}
The Sommerfeld expansion is in $T/T_F$.
\begin{align}
    \int_{0}^{\infty} \, n(k) k^{2\alpha}
    &= I(\alpha)\\
    &=
    \left( \frac{2m}{\hbar^2} \right)^{\alpha - \frac{1}{2}}
    \frac{m}{\hbar^2}
    \int_{0}^{\infty} dE\,
    n(E) E^{\alpha - \frac{1}{2}}
\end{align}
and recalling
\begin{align}
    E &=
    \frac{\hbar^2 k^2}{2m}
\end{align}
and
\begin{align}
    dE &=
    \frac{\hbar^2 k}{m}dk
\end{align}
we get
\begin{align}
    \int_{0}^{\infty} dE\, n(E)
    E^{\alpha - \frac{1}{2}}
    &=
    n(E)
    \left.
    \frac{E^{\alpha - \frac{1}{2}}}{\alpha + \frac{1}{2}}
    \right|_{0}^{\infty}
    -
    \int_{0}^{\infty} dE\,
    E^{\alpha + \frac{1}{2}}
    \frac{dn(E)}{dE}\\
    &=
    -\beta \int_{0}^{\infty}\left[ 
    \mu^{\alpha + \frac{1}{2}}
    +
    \frac{E - \mu}{\beta}
    \mu^{\alpha + \frac{1}{2}}
    +
    \frac{\left( E - \mu \right)^2}{2\beta^2}
    \mu^{\alpha - \frac{1}{2}}
    \left( \alpha + \frac{1}{2} \right)
    \left( \alpha - \frac{1}{2} \right)
    +
    \cdots
    \right]
    \frac{e^{\beta\left( E - \mu \right)}}{
    \left( e^{\beta\left( E - \mu \right)} + 1 \right)^2
    }\\
    &=
    \frac{1}{\alpha + \frac{1}{2}}
    \int_{-\beta \mu}^{\infty} dx\,
    \left[ 
    \mu^{\alpha + \frac{1}{2}}
    +
    \mu^{\alpha - \frac{1}{2}}
    \left( \alpha + \frac{1}{2} \right)
    x
    +
    \frac{\mu^{\alpha - \frac{1}{2}}}{2\beta^2}
    \left( \alpha + \frac{1}{2} \right)
    \left( \alpha - \frac{1}{2} \right)
    x^2
    +
    \cdots
    \right]
    \frac{e^{x}}{\left( e^x + 1 \right)^2}
\end{align}
where we have defined $x = \beta\left( E - \mu \right)$.
Then you can use the fact that
\begin{align}
    \int_{-\inty}^{\infty} dx\,
    \frac{x^n e^x}{\left( e^x + 1 \right)^2} = 0
\end{align}
for odd $n$
and
\begin{align}
    \int_{-\infty}^{\infty} dx\,
    \frac{x^{n} e^x}{\left( e^x + 1 \right)^2}
    =
    \begin{cases}
        1 & \text{if } n = 0\\
        \frac{\pi^2}{3} & \text{if } n = 2\\
        \vdots
    \end{cases}
\end{align}
Sommerfeld was probably the greatest physicist of his age,
with Heisenberg and Pauli,
all people of the same caliber.
They were just good at it.
The circle around him,
he was good,
prominent,
knew what the latest thing was.
People around him were at the forefront of science.
But he was an old man like myself.
Great physicists,
but other his best years when QM came around.
So his students got famous.
Different from today.
These days if you discover,
your advisor gets famous.
Generous man,
etc.
He wrote books,
one of them studied the picture of atoms,
using not quantum mechanics,
but using old Bohr quantum theory.
So he did things,
like relativistic corrections of the helium atom
wrote a book this thick.
By the time his book had finished,
his student Heisenberg invented quantum mechanics,
and found a correct way to calculate these things.
There's no translation into English because it's irrelevant,
but it's funny to see written in German.
HE wrote pedagogical books,
on electromagnetism, and stuff like that,
more or less the level of this class.
They're still readable today 100 years later.
You can get it from the library,
there's little development after that.

This is something for you to do at home.
Before we found that
\begin{align}
    N &\propto
    g V
    \frac{k_F^2}{6\pi^2}
\end{align}
and this is just at shorthand
\begin{align}
    k_F &= \sqrt{2M \mu}
\end{align}
But now we have a correction to this
\begin{align}
    N &\propto
    g V
    \frac{k_F^2}{6\pi^2}
    \left[ 
    1
    +
    \pi^2
    \left( \frac{k_B T}{\mu} \right)^2
    + \cdots
    \right]
\end{align}
This is hard to invert to get $\mu$ as a function of $N$.
People in my class are smart.
They understand that this is an approximation,
so this is valid only if this term is much smaller than that,
and $\mu$ is approximately given by this.
There's no point in getting the exact $\mu$ here,
the corrections are higher order terms.
You can exercise this in the homework.
You're going to invert it perturbatively.
Same thing for the other expressions.
\begin{align}
    E &=
    gV
    \frac{\hbar^2}{2m}
    k_F^5
    \frac{2^{3/2}}{5\pi^2}
    \left[ 
    1
    +
    5\pi^2
    \left( \frac{k_B T}{\mu} \right)^2
    +
    \cdots
    \right]
\end{align}
A little physical interpretation here would be good.

One more thing.
The specific heat at fixed volume.
\begin{align}
    c_V &=
    \frac{1}{N}
    \left.
    \frac{\partial E}{\partial T}
    \right|_{V,N}\\
    &\approx
    \frac{\pi^2}{2} \frac{k_B T}{T_F}
    + \cdots
\end{align}
Remember the specific heat of a classical ideal gas.
It was a constant?

With phonons?
It was $T^3$.

You might say of course,
because you can calculate in your head.
Do you know where the powers of $T$ come from?

What about black body?
Same thing,
because phonons and photons are like the same thing.

But Fermi gas has different behaviour.
It goes to zero at zero temperature,
unlike a classical gas.

So imagine a Fermi surface with radius $k_F$.
At zero temperature,
you fill up every level up to $k_F = \sqrt{2m\mu}$.
But what happens at finite temperature?

Now consider increasing the temperature by a little bit.
The distribution $n(k)$ goes from a sharp step function
to a smooth step.
I created a hole and a particle.
That is called a particle and hole.

The low-lying corresponds to this particle-hole excitation.
Take the energy of the Fermi surface and increase it slightly.

But particles deep in the Fermi surface,
cannot increase their energy by a lot,
they don't contribute much to the partition function.
Only states that are relatively close to the Fermi surface are going to be
promoted or excited.
So I'm going after the region around the Fermi surface,
where the occupation number if neither zero or 1.

The radius of this sphere,
let's calculate the volume of this spherical shell in momentum space.
Do you think the radius is going to be $\sim k_F^2 k_B T$?
No,
not really.
The correction to the energy is order $(k_BT/\mu^2$,
so this thickness is not proportional to $k_BT$,
but proportional to $(k_B T)^2$.

So the thickness of this region is 0
for zero temperature,
but it's proportional to $(k_B T)^2$
for small finite temperature.

At small temperature,
the excitations are particle-hole excitations,
and the thickness of the Fermi surface region is quadratic.

One way to think about why it's quadratic,
you can think of the step smoothing out,
because the particle part and the hole part cancel out to first order
and you're left with

Years later the only thing you'll remember is that the Fermi surface is fuzzy at
finite temperature and its thickness is going to be proportional to $T^2$.
\begin{align}
    E &\approx
    E_{T=0}
    +
    \#
    k_F^2 (k_BT)^2
\end{align}

We have a month of class left.
The syllabus is crazy.
It looks like most of you survived so far.
Pat yourself on the back a little bit.
