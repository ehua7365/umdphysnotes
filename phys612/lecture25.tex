[1.5 hours late]
\section{Second Quantization notation}
We use the notation
\begin{align}
    \ket{\varphi} = \ket{n_1, n_2, \ldots}
\end{align}
where
\begin{align}
    \hat{N} &=
    \left( n_1 + n_2 + \cdots \right)
    \ket{n_1, n_2, \ldots}
\end{align}
is the total number operator and the Hamiltonian is
\begin{align}
    \het{H} \ket{\varphi} &=
    \left( 
    n_1 E_{n_1} + n_2 E_{n_2} + \cdots
    \right)
    \ket{n_1, n_2, \ldots}
\end{align}

\section{Bosonic Ideal Gas}
Second quantization notation.
We had a sum over all multiparitcle states over all multiparticle states.
Now we just have
\begin{align}
    Z_G(T, V, \mu) &=
    \sum_{n_1, n_2, \cdots = 0}^{\infty}
    \bra{n_1,n_2,\ldots} e^{-\beta(H - \mu N)} \ket{n_1,n_2,\ldots}\\
    &=
    \sum_{n_1=0}^{\infty} e^{-\beta(E_1 - \mu) n_1}
    \sum_{n_2=0}^{\infty} e^{-\beta(E_2 - \mu) n_2}
    \cdots\\
    &=
    \frac{1}{1 - e^{-\beta(E_1 - \mu)}}
    \frac{1}{1 - e^{-\beta(E_2 - \mu)}}
    \cdots
\end{align}
So then
\begin{align}
    \Omega(T, V, \mu) &=
    -k_B T \ln Z_G\\
    &=
    -k_BT \sum_k \ln\left( 1 + e^{-\beta (E_k - \mu)} \right)\\
    &\approx
    k_B TV \int \frac{d^3k}{(2\pi)^3}
    \ln\left( 1 - e^{-\beta(E_k - \mu)} \right)
\end{align}
where each
\begin{align}
    E_k &= \frac{\hbar^2 k^2}{2M}
\end{align}

It's just the same thing as the modes of a crystal vibrations.
They're the same as an EM field,
because I can find the normal modes of th EM feld.
This has nothing to do with Harmoinc oscillators except it does.
Each single aprticle state mode is like a harmonic oscillator.
The state is indexed by the particle number,
and the enregy levels are equally spaced just like the harmoinc oscillator.
That's the special property.

\section{Fermion ideal gas}
For fermions,
just change the limit of the sum from $\infy$ to $1$
and you get
\begin{align}
    \Omega(T, V, \mu) &=
    -k_B T V g \int \frac{d^3k}{(2\pi)^3}\,
    \ln\left( 
    1 + e^{\beta(E_k - \mu)}
    \right)
\end{align}
where $g$ is the number of internal states,
like the spin.

The little difference in sign changes things profoundly.

Just using the general formalism of the grand canonical ensemble,
I should be able to find the boson canonical ensemble,
etc.


For example,
if I have $\Omega$,
I can find the average $N$.
Let's compute the average number of particles over the volume so we get a
density.
\begin{align}
    \frac{N}{V} &=
    \frac{1}{V}
    \left. \frac{\partial \Omega}{\partial\mu}\right|_{T,V}\\
    &=
    - k_B T g
    \int \frac{d^3k}{(2\pi)^2}
    \frac{ \beta e^{-\beta \epsilon_k}}{%
        1 \mp e^{-\beta \epsilon_k}
    }
\end{align}
where $\epsilon_k=E_k - \mu$ which appears so frequently I may as well have a
name for it,
where the upper sign is for fermions and the lower sign is for bosons.
And then
\begin{align}
    \frac{N}{V} &=
    g \int \frac{d^3k}{(2\pi)^2}
    \frac{1}{e^{\beta\epsilon_k}\mp 1}
\end{align}
and the only difference is the sign.
The integrand is interpeted as the average occupation number.
THen I take an average over many states.
For a particular single energy state
what is the average expecation?
Well that is just the occupation number
\begin{align}
    n(E) &=
    \frac{1}{e^{\beta\epsilon_k}\mp 1}
\end{align}
which is the occupation number of hte density per unit of $k$.

Another thing I like to calculate is the enrgy density.
How do I find that?
I can thnk of this $\Omega$ here,
take a derivative in relation to $T$ to find the entropy,
then use the Lengedre trnasform to find the entorpy.
The little trick about this is the following.
This gives me $N$ as a functio of the tempreaturea nd the chemical potential
but it's hard to invert the relation to find what $\mu$ is needed to givea
aparicular $N$.
Although it's easy to say Legendre tranform,
it's hard to do in partice.

It doesn'tmatter,
you can manipulate the integrals back and forth,
I won't do this in front of you or do tell you to do it in the homework.
Which one do yo uwant?
Of course it's in the homework.

Thea nswer is something you'd expect.
It's as um over single particle states.
We're looking for the total energy.
That's the occupation number for one particular satte.
If I multiply it by the enrgy of the single particular state I get hte enrgy.
\begin{align}
    \frac{E}{N} &=
    g \int \frac{d^3k}{(2\pi)^3}
    \frac{E_k}{e^{\beta\left( E_k - \mu \right)} \mp 1}
\end{align}
where the  $g \int \frac{d^3k}{(2\pi)^3}$ means sum over single particle states,
the $\frac{1}{e^{\beta\left( E_k - \mu \right)} \mp 1}$ is the occupation number
for the $k$ and $E_k$ is the energy ofa single particle.

In the mideterm,
If I ask a question inthe midterm,
I will not give you this equation,
ecause it's so intuitive and obvious.

I'll give you the $N/V$ because the sign is not obvious,
but this $E/N$ formula is obvious from that.
It's difficult to compute but it's a simple answer in the end.
