\section{Canonical Quantization}

Classical $Q$ configuration space.
$q_i$, for $i=1,\ldots,N$
$p_i$, for $i=1,\ldots,N$
Poisson brackets
$\left\{ q_i, p_i \right\}=1$

Quantum $\mathcal{H}$ on $Q$.
$\hat{q}_i$ and $\hat{p}_i$.
Commutator
$\left[ \hat{q}_i, \hat{p}_i \right] = i\hbar \delta_{ij}$.

Hamiltonian is
\begin{align}
    \hat{H} = \mathcal{H}\left( \hat{q}_i, \hat{p}_i \right)
\end{align}

\begin{example}[1D particle]
    $\hat{x}\ket{x} = x\ket{x}$
    and
    \begin{align}
        \ket{\psi} = \int_{-\infty}^{\infty} dx\,
        \psi(x)\ket{x}
    \end{align}
    and hte momentum is
    \begin{align}
        \hat{p}\ket{x} := i\hbar \frac{d}{dx}\ket{x}
    \end{align}
    with
    \begin{align}
        \left[ \hat{x}, \hat{p} \right] = i\hbar
    \end{align}
\end{example}
\begin{proof}
    For a basis state,
    \begin{align}
        \left[ \hat{x}, \hat{p} \right]
        &=
        \hat{x} i\hbar \frac{d}{dx}\ket{x}
        - \hat{p}\hat{x}\ket{x}\\
        &=
        i\hbar
        \frac{(x + dx)\ket{x + dx} - x\ket{x}}{dx}
        - x i\hbar \frac{d}{dx}\ket{x}\\
        &=
        i\hbar\left(
            x \frac{d}{dx} + \frac{dx}{dx}\ket{x}
        \right)
        - i\hbar x \frac{d}{dx}\kket{x}\\
        &= i\hbar \ket{x}
    \end{align}
    Then more generally
    \begin{align}
        \left[ \hat{x}, \hat{p} \right] &=
        \int_{-\infty}^{\infty} dx \left[ \hat{x}, \hat{p} \right]
        \ket{x} \braket{x}{\psi}\\
        &= i\hbar\ket{\psi}
    \end{align}
\end{proof}

\begin{example}[Particle on a sphere]
    The Lagrangian is
    \begin{align}
        L &=
        \frac{m}{2} R^2\left(
            \theta^2 + \sin^2\theta \phi^2
        \right)
    \end{align}
    and the conjugate momenta are
    \begin{align}
        p_\theta &=
        \frac{dL}{d\theta}
        = mR^2\theta\\
        p_\phi &=
        \frac{dL}{d\phi}
        = mR^2 \sin^2\theta \phi
    \end{align}
    which gives the Hamiltonian
    \begin{align}
        H &=
        \frac{m}{2} R^2\left(
            \frac{p_\theta^2}{(mR^2)^2}
            + \frac{p_\phi^2}{(mR^2)^2\sin^2\theta}
        \right)
    \end{align}
    Classically the Poisson bracket is
    \begin{align}
        \left\{ \theta, p_\theta \right\} &=
        \left\{ \phi, p_\phi \right\}
        = 1
    \end{align}
    which upon quantisation leads becomes commutation relations
    \begin{align}
        \left[ \hat{\theta}, \hat{p}_\theta \right] =
        \left[ \hat{\phi}, \hat{p}_\phi \right] =
        i\hbar
    \end{align}
    We can then define the position operators acting on basis states with
    \begin{align}
        \hat{\theta}\ket{\theta,\phi} &= 
        \theta\ket{\theta,\phi}\\
        \hat{\phi}\ket{\theta,\phi} &= 
        \phi\ket{\theta,\phi}
    \end{align}
    and the momentum operators
    \begin{align}
        \hat{p}_\theta\ket{\theta,\phi}
        &=
        i\hbar \frac{\partial}{\partial\theta}\ket{\theta,\phi}\\
        \hat{p}_\phi\ket{\theta,\phi}
        &=
        i\hbar \frac{\partial}{\partial\phi}\ket{\theta,\phi}
    \end{align}
    A arbitrary state can be written as a superposition
    \begin{align}
        \ket{\psi} &=
        \underbrace{
            \int_{0}^{2\pi} d\theta\sin\theta
            \int_0^{\pi d\phi}
        }_{\displaystyle\int_{\textrm{sphere}} d\Omega}
        \psi(\theta,\phi) \ket{\theta,\phi}
    \end{align}
    However, the momentum operator does not act like how you think it does.
    \begin{align}
        \hat{p}_\theta \psi(x) &\ne
        -i\hbar \frac{\partial}{\partial\theta} \psi(\theta,\phi)
    \end{align}
    If you plug the above into the Hamiltonian,
    you don't get the Laplacian back, so it must be wrong.
    The correct expression is
    \begin{align}
        \hat{p}_\theta \psi(x) &=
        \frac{-i\hbar}{\sqrt{\sin\theta}}
        \frac{\partial}{\partial\theta}\sqrt{\sin\theta}
    \end{align}
    Then you find that
    \begin{align}
        \bra{\psi}\hat{p}\ket{\chi}
        &=
        \int_{0}^{2\pi} d\theta \sqrt{\sin\theta}
        \int_{0}^{\pi}d\phi\,
        \psi^*(\theta,\phi)\cdots
    \end{align}
    Then you find that
    the solutions are spherical harmonics $Y_{l,m}(\theta,\phi)$
    which you can use as basis.
\end{example}

\section{Operator Ording Ambiguities}
Canonical quantization is not a good algorithm for quantisation,
bcecause there is ambigiuity in the ordering of the Hamiltonian.
Suppose you have
\begin{align}
    H &= \frac{p^2}{2m}
\end{align}
and you quantize it into
\begin{align}
    \hat{H} &= \frac{\hat{p}^2}{2m}
\end{align}
However, another Hamiltonian you could have that is classically the same is
\begin{align}
    H &=
    \frac{1}{x^{100}}
    \frac{p x^{100} p}{2m}
\end{align}
which quantizes into
\begin{align}
    \hat{H} &= \frac{1}{2m}
    \frac{1}{\hat{x}^{100}}
    \hat{p} \hat{x}^{100} \hat{p}
\end{align}
which is wildly different!
