\section{Quantum Mechanics}
The state of a system is described by a ket in a Hilbert space $H$.
Every observable corresponds to a hermitian operator $A$ with
$A\ket{a}=a\ket{a}$,
where $a$ are possible outcomes of a measurement.
The probability of measuring $A=a$ is equal to
$|\braket{\psi}{a}|^2$.

The actual \emph{amplitude} is $\braket{\psi}{a}$ which is a projection onto
$\ket{a}$,
which is a complex number.
The modulus squared is the probability.

Every state has magnitude 1,
so the projection is smaller than one.

How do I calculate the probability if there is a degenerate state?
Well you project both.
If $\ket{\psi}$ is the state of the system at measurement,
and suppose I have a two-dimensional eigenspace with eigenvalue $a$.
Any vector in this space has eigenvalue $a$.
How do I calculate the probability?
I take $\ket{\psi}$ and project down onto this plane.
This probability is
\begin{align}
    \sum_{a\in\text{subspace}} |\braket{a}{\psi}|^2
\end{align}
So in the general case there is a sum here if the eigenvalue is degenerate.
Sometimes it's not so common,
so if you read Griffiths you may not have noticed.
Usual things apply,
it's a probabilistic theory, etc.


There is a misunderstood point.
Say I have a system, like my keys.
I measure the position of the car key.
Use your imagination.
It has a bunch of eigenvalues,
the angles my key makes with the vertical.
I look and I find my key here.
Good that's great.
What happens if immediately after that I make another measurement?
If the state of the system was in a state I started out,
there would be a probability.
Immediately afterwards,
there's no time for anything to happen.
What's the point of looking?
It's not like I can look and extend my 5 fingers to pick it up.
But when I move,
it's here and here.
What's the point of making a measurement?
This world would be chaos.
There would be nothing.
That's not how the universe works.

The universe works like this.
You measure something.
You measure again,
it's in the same darn place.
The world is not completely chaotic.
Of course if I take too long,
maybe they key may be in a different place.

How do I guarantee this in my formalism?

What happens is that you need this as well:
\begin{itemize}
    \item After a measurement $A=a$,
        then the state $\ket{\psi}$ magically instantaneously becomes
        \begin{align}
            \ket{\psi}\to \ket{a}
        \end{align}
\end{itemize}
But what happens immediately after?
I'll have probability $1$ of measuring $a$ again.
This has a few names.
It's called the \emph{reduction of the wave packet}.

Those eigenvectors can be thought of waves,
maybe a bump in space.
You measure and you no longer have a bump in space,
just a Fourier component for example.
That's the old lingo for that.
There's another name,
the \emph{collapse} of the wave packet,
or wave function, blah blah blah.

By the way, what happens if the eigenvalue is degenerate?
In principle I can go into any one of these.
It goes into the projection,
the closest one in the subspace.

People are immensely bothered by this,
so much that textbooks try to hide this.
They try to make you pay attention to the Schrodinger equation.
Because of this,
if some people do this lecture,
people leave this out in a lecture
because it's so embarrassing.

Have you learnt about a theory like EM where you didn't say why.
Hold this comment.

In Newtonian mechanics, you say $F=ma$,
particles accelerate, you calculate forces,
that gives me forces, etc.
Here,
the time evolution of the state of the system is given by a differential equation
that is very simple
\begin{align}
    i\hbar \frac{d}{dt} \ket{\psi(t)} = H\ket{\psi(t)}
\end{align}
where $H$ is the Hamiltonian.
I write this without the curly $H$ so it's not a Hilbert space.
The time-evolution is given by this magic operator.
The point is,
there are two ways the state of the system can evolve.
One,
according to this \emph{Schrodinger equation},
that's what it's called by the way.
That's what you do 100\% of the time in Griffiths.

Sometimes though,
Schrodinger equation stops working,
when you measure.
And then after that Schrodinger again.
For example,
if there is a guy in a white lab coat with a beer
and a lot of electronics,
and he's good at messing around with systems,
and a number pops up saying 42.
That's a measurement.
You measure the energy and got 42 Joules.
When I measure immediately afterwards I get 40 Joules.
So he with a lab coat he can stop Schrodinger equation,
and then it comes back.
What if he doesn't have a lab coat and doesn't have a beard?
Some graduate student?
So does that apply?
Yes of course,
grad students are people too.
What if they are an undergraduate?
You know in your heart they are people too.
What is it's a monkey.
You train the monkey,
he says 42,
he has no idea of what 42 means.
Do you think will the wave function will evolve.
Some people say no.
What about Neanderthals?
When did we evolve to have the ability to collapse wave functions?
That's ridiculous.

Of course the wave function collapses.
What if there's no monkey it's automatic.

People can find out it's 42.
When does the wave function collapse?
Does it collapse when the number pops up,
or when the human being looks at it?

Some say this stuff is this hippy dippy stuff.
They say
what causes this change is an interaction with the measurement apparatus.
For a while,
it's described not by the Hamiltonian of the system,
but the object and the system.
No.
It's easy to prove this is not true.

Because let's go back here.

The wave function in the future is automatically normalized.
If I write my state in the future,
in terms of state in the initial time,
there's going to be a linear operator relating those two things.
As we see, this thing is a unitary operator
\begin{align}
    \ket{\psi(t)} = U(t) \ket{\psi(0)}
\end{align}
which preserves lengths and angles.

Measurement operation is not unitary.
For example,
but part orthogonal to $\ket{a}$ got shrunk to zero,
but unitary just rotates things,
so it's not unitary.

It's too facile to say my apparatus interacted with the system,
because it's not unitary.
It's not enough.


Say my student says it was 42 measured.
When did it collapse?
When the student saw it or before that?

I make a measurement and I see a particular result.
I know there was some kind of collapse.
We can say this measurement axiom is not true,
because it poses this difficult question it is asking.

By the way,
no body has a better answer to this.

I tried my whole life to make a particular experiment
that depends whether the wave function
when the number pops up or when the grad student sees the result.
I also never seen a proof of either.
It's a hole in the literature.

There is no experiment that can explain that subtlety of when the collapse
happens.
The modern understanding is slightly better,
but only slightly.
You should do an analysis involving the system,
the apparatus,
and maybe the grad student.
This is not only unitary evolution,
but also deterministic.
I started in a state,
and there is no way to manipulate Schrodinger equation
to come up with a solution which says I collapse into this
and maybe this as well.
Think really hard about deterministic evolution
and somehow derive the measurement axiom.
You just have to live with it.
It could make you sleep bad at night,
but that's just philosophy.
I don't.
Something big and weird is happening there.

There's one more thing useful for quantum mechanics.
If you have a composite system and each one has a Hilbert space $\mathcal{H}_1$
and $\mathcal{H}_2$,
the composite system is going to have a Hilbert space
$\mathcal{H}_1\otimes\mathcal{H}_2$.
There's a lot of juice hidden here.

Say that $\mathcal{H}_1$ is 2-dimensional,
and $\mathcal{H}_2$ is 2-dimensional.
What is the composite system?
4 dimensions.
What if I had a third Hilbert space? 8.
So exponential growth.
With just 100, you have $2^{100}$.
It can be represented as a matrix,
like in kindergarten,
but how big is the matrix?
Can you write down the matrix?
Can the computer calculate?
If you transfer planet Earth,
take the galaxy,
universe,
turn it into disk space,
you still cannot write the Hamiltonian with 100 systems.
This exponential growth is a bummer,
well not really a bummer,
because then you'd be able to diagonalize everything
and I would lose my job.

That's my summary of QM,
it's really upstart,
but we're going to take this and apply to many cases.
It's very upstart,
but we've done all the work already.

We've done all the numbers already,
you know how to transform things into concrete equations,
because we spent two weeks.
Strictly speaking, we're actually done.
Go home class is over.

\begin{question}
    You call a hermitian operator measurable.
    Is that measurable the same as probability?
\end{question}
No, measurement in the physics sense.
Anything you can measure.
Go to Radio Shack, get a ruler,
that's a measurable device.

Measure is physicist lingo.

We have to distinguish things like momentum and energy for which there are
devices,
from things like kets.
I cannot by a ket-meter that tells me what the ket is.
There will never be such a thing.
That is not an observable.
That's the end of QM.
But before examples,
I want to make trivial generic comments.


\begin{itemize}
    \item An overall phase on $\ket{\psi}$ is unphysical.
        Look at my system and say this is $\ket{\psi}$,
        but my friends says no it's $e^{i\alpha}\ket{\psi}$,
        it's the same thing.
        More than that.
        Suppose I start with a state and evolve,
        but my friend has a space,
        in the end the phase doesn't matter.
        If you want to be nitty-picky,
        you should say it's not that states are represented by kets.
        It's not even represented by kets measured 1.
        It's represented by rays.
        You can take a vector,
        you can multiply by any number,
        including complex numbers.
        The whole set represents this thing.
        When I use the wave function to calculate a probability,
        we're going to get the same result.
    \item The Schrodinger equation is linear.
        If I have two solutions of the Schrodinger equation,
        I'm going to get another solution.
        It's like the Maxwell equations.
\end{itemize}
Should I stop to tell Schrodinger stories?
He was a great scientist,
but he fun part is his personal life of course.
He was Australian,
and compared to other people who invented QM
who were like 23,
he was in his 30s,
receding hairline already.
He was a 
He didn't get along with the Nazis in the 30s.
He was not Jewish,
just cut his pie.
He became very uncomfortable,
and then German invaded Austria.
He could be killed.
He wrote a letter reneging his position with the Nazis,
He regretted,
so he decided to leave.
He needed visa, a job.
He had a wife and kids.
And the lover.
And the lover's husband.
He has a lot.
It was complicated.
A few years earlier he had written that equation.
So he got invitations for jobs.
So he went to Oxford.
The people look at the situation,
say don't think so.
He wants to leave with his kids, wife, lover, lover's husband.
Then at Princeton,
there is the Institute of Advanced Studies,
when they hire Einstein, Godel, von Neumann, etc.
So he goes the New Jersey.
People at the Institute are given houses,
If you get hired there they give you Einstein's house.
It's funny,
you have to buy it from them and then you have to sell when you leave.
Einstein's house is a dump.
Nice museum, but it's not nice to live in because it doesn't have all the modern
amenities.
Can you get a job for the husband of the lover?
He was a mathematician.
People in New Jersey were not happy with that.

There was another institute of advanced studies at the time in Dublin headed by
a Catholic priest.
You say there's no chance.
Turns out he went there,
don't know how many people.
The reality is very different from the perception by majority of Americans.
Everything was fine.
He got job, visa, for his wife, kids, husband of the lover,
husband had someone else too.
Everyone came to Ireland and everyone was happy.
The only unhappy thing was Schrodinger really disliked quantum mechanics.
He didn't like collapse axiom.
After trying to fix QM,
he gave up and tried to do other things.
Some people say,
especially physicists,
he was kind of one of the first ones to think about DNA,
and how genetic information can be stored in molecules
before DNA was discovered.
I heard,
his little book \textit{What's Life}
was influential for the people who discovered DNA.
By the time he wrote this book,
everything was either known to be right or known to be wrong.
Watson and Crick weren't even biologists,
they where X-Ray physicists.

\begin{question}
    Do you know Schrodinger?
    Or other famous people?
\end{question}
Schrodinger died before I was born.
I'm not that old!
He's not one of the elder kings of QM.
I met famous people,
but not physicists.
Does that count?
There are physicists famous among physicists.
Does that count?
At this stage,
you don't know the very old people who invented QM.
I know Heisenberg's son.
I taught QM to Neils Bohr's great grandson,
he was a student here at Maryland.
How can you go into physics?
His great grandfather won a Nobel prize,
his grandfather won a Nobel prize.
His father was a slacker.
Why would you do this to yourself?

I remember these people of that generation barely walking.
Our lives did not overlap on this planet.
I met Hans Bethe.
I met some Los Alamos people,
but they were very old.
After the war,
there were a bunch of them,
very old.
I met some Rock stars,
does that count?

After telling you those things
you knew forever anyway,
we can talk about the no cloning theorem from linearity.
But, that's for Friday.

Can I write a copy of something on a Xerox?
Of course I can.
What I'm going to argue is that I cannot do this with quantum sates.
Say that I start with a system $\ket{\psi}$,
which represents blank paper.
Now I'm going to put together with another system,
which is like blank paper.
Actually, let's make this more interesting.
Suppose $\ket{\psi}$ is money.
I want to copy it.
What's the Hilbert space of the whole system together?
So my state is $\ket{\psi}\otimes\ket{\phi}$
where $\ket{\phi}$ is like blank paper.
I'm going to evolve this,
and arrange my Hamiltonian such that my system is going to be money and money
like $\ket{\psi}\otimes\ket{\psi}$.
That's how you make money.
If you do this,
you've got it.
Make a good xerox machine,
you can do that.
Not possible.
Imagine you start with a linear superposition
\begin{align}
    \frac{\ket{\psi_1} + \ket{\psi_2}}{\sqrt{2}}
    \otimes\ket{\phi}
    \to
    \frac{\ket{\psi_1}\otimes\ket{\psi_1} + \ket{\psi_2}\otimes
    \ket{\psi_2}}{\sqrt{2}}
\end{align}
But that's not what I want,
I want
\begin{align}
    \frac{\ket{\psi_1} + \ket{\psi_2}}{\sqrt{2}}
    \otimes\ket{\phi}
    \to
    \frac{\ket{\psi_1} + \ket{\psi_2}}{\sqrt{2}}
    \otimes
    \frac{\ket{\psi_1} + \ket{\psi_2}}{\sqrt{2}}
\end{align}
and you can see this is not the same.
You can see there a bunch of cross-terms like
$\ket{\psi_1}\otimes\ket{\psi_2}$.
Even if I can make copies of one-dollar bills,
I can only print one-dollar bills,
but it's not a copier.
This little result here is called the no-cloning theorem.
It's a completely trivial result.
We don't give any importance to this until Zurek
and people found it interesting.
Even if this is interesting to prove,
people forget sometimes.

If I tried to send a secret to my friend,
someone can spy on us by intercepting it make a copy.
Suppose I encode that information in a quantum state.
If someone spies on us,
they can make a copy,
but the most they can do is kill the system.
No one can make a copy of a quantum system and run away with it.
QM is a good way to transfer secrets.
When you to build a quantum computer,
a normal computer depends on copying,
but a quantum computer can't do that
and that's a bummer.
People think it's really cool,
so I want to mention this.
