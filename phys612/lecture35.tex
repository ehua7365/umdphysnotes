\section{Renormalization Group}
Block spin trnasformation for matrice slacing $a$ to $la$.
The parttion function is
\begin{align}
    Z &=
    \sum_{ \left\{ \sigma \right\}} e^{K\left[ \sigma \right]}\\
    &=
    \sum_{ \left\{ \sigma' \right\}} e^{K' \left[ \sigma' \right]}
\end{align}
where
\begin{align}
    K'
    &=
    K_0'
    +
    K'_2
    \sum_{n,m} \sigma'_{i} \sigma'_{j}
    +
    K'_3
    \sum_{n-m,n} \sigma'_i \sigma'_j
    +
    K_4' \sum_{\square}
    \sigma'_i \sigma'_j \sigma'_k \sigma'_l
    + \cdots
\end{align}
At fixed points
\begin{align}
    K^* &= R\left( K^* \right)
\end{align}
which we can expand near fixed point
\begin{align}
    K'_{\alpha}
    -
    K_{\alpha}^*
    &=
    R\left( K \right)
    -
    K^*\\
    &\approx
    R\left( K^* \right)
    +
    \left( K - K^* \right)_{\beta}
    \left.
    \frac{\partial R}{\partial K_{\beta}}
    \right|_{K=K^*}
    +
    \cdots
\end{align}
For simplicity assume the matrix $\frac{\partial R_\alpha}{\partial K_\beta}$
is symmetric.
Then if $\lambda > 1$ we say it is a relevant direciton,
but if $\lambda < 1$ it is irrelevant.
And if it's equal to zero then it's marginal.
THey are relevant for areason.

Big diagram here.
\begin{figure}[h]
    \begin{center}
        \includesvg{rgflow}
    \end{center}
    \caption{RG flow}%
    \label{fig:rgflow}
\end{figure}
\begin{align}
    K &=
    \beta J\\
    &=
    \frac{J}{k_B \left( T - T_c + T_c \right)}\\
    &\approx
    \frac{J}{k_B T_c}
    \left( 
    1
    +
    \underbrace{\frac{T - T_c}{T_c}}_{t}
    + \cdots
    \right)
\end{align}

Critical exponent $\nu$.
\begin{align}
    t' &= \lambda t\\
    t^{(n)} &= \lambda^n t\\
\end{align}
so the correlation length scales like
\begin{align}
    \xi\left( t' \right) &= \frac{1}{l} \xi\left( t \right)\\
    \xi\left( t^{(n)} \right)
    &=
    \frac{\xi\left( t \right)}{l^n}
\end{align}
with
\begin{align}
    n
    &=
    \frac{\ln \left( b/t \right)}{\ln \lambda}
\end{align}
Note
\begin{align}
    \lambda^n t
    &=
    e^{n \ln \lambda} t
    =
    e^{\frac{\ln (b/t)}{\ln \lambda} \ln \lambda} t\\
    &=
    \frac{b}{t} t\\
    &= b
\end{align}
And
\begin{align}
    l^n &=
    e^{n\ln l}\\
    &=
    e^{\frac{\ln(b/t)}{\ln \lambda} \ln \lambda}\\
    &=
    \left( \frac{b}{t} \right)^{\frac{\ln l}{\ln \lambda}}
\end{align}
so then
\begin{align}
    \xi\left( t \right)
    &=
    l^n \xi\left( t' \right)
    =
    \underbrace{l^n}_{t/b}
    \xi\left( \underbrace{\lambda^n t}_{b} \right)\\
    &\sim
    t^{- \frac{\ln l}{\ln \lambda}} = t^{-\nu}
\end{align}
Hence
\begin{align}
    \nu &=
    \frac{\ln l}{\ln \lambda}
\end{align}

Now let us calculate the $\alpha$.
Consider $f(t)$ to be the free energy per site.
\begin{align}
    l^d f(t)
    &-
    f\left( t' \right)
\end{align}
which implies
\begin{align}
    f(t)
    &=
    \underbrace{
    \frac{1}{l^{dn}}
    }_{
    \left( \frac{t}{b} \right)^{-d \underbrace{
    \frac{\ln l}{\ln \lambda}}_{
    \nu
    }}
    }
    f\left( \lambda^n t \right)\\
    &\sim t^{d\nu}
\end{align}
Recalling that
\begin{align}
    n &=
    \frac{\ln b/t}{\ln \lambda}
\end{align}
And the specific heat is proportional to
\begin{align}
    C &\sim
    -\frac{\partial^2 f}{\partial t^2}
    \sim
    t^{\underbrace{d\nu - 2}_{\alpha}}
\end{align}
hence
\begin{align}
    \alpha &= d\nu - 2
\end{align}
which is the Josephson relation.
