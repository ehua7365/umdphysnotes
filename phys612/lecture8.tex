\section{Uncertainty principle}
The uncertainty principle for observables $A$ and $B$ are
\begin{align}
    \sigma_A \sigma_B \ge \frac{1}{2}
    |\bra{\psi} [A,B] \ket{\psi}|
\end{align}
for example,
\begin{align}
    \sigma_x \sigma_p \ge \frac{\hbar}{2}.
\end{align}

Don't pretend you have to rush through the material and not get very basic
things like the uncertainty principle.

Here's something I know you understand for fact.

\section{Time-energy uncertainty}
Energy is an operator in quantum mechanics.
Time is not an operator in QM.
It is a parameter in the Schrodinger equation.
Nothing tells me the time operator.
It's special in QM.

But we shouldn't.
Time and energy satisfy an uncertainty just like position and momentum.
The proof and interoperation is different,
but it's true.

Take any observable $A$.
Compute expectation value $\bar{A}$.
Let's see how the average changes over time.
\begin{align}
    \frac{d}{dt}\bar{A} &= \frac{d}{dt}\bra{\psi} A\ket{\psi}\\
    &= \underbrace{\frac{d}{dt} \bra{\psi}}_{+\frac{i}{\hbar}\bra{\psi} H}
    A \ket{\psi}
    + \bra{\psi} A
    \underbrace{\frac{d}{dt}\ket{\psi}}_{-\frac{i}{\hbar}H\ket{\psi}}
\end{align}
and then we get the beautiful result
\begin{align}
    \frac{d}{dt}\bar{A} &=
    -\frac{i}{\hbar} \bra{\psi} [A,H] \ket{\psi}
\end{align}
which leads to
\begin{align}
    \left|\frac{d\bar{A}}{dt}\right| &=
    \frac{1}{\hbar}
    |\bra{\psi} [A,H] \ket{\psi} |\\
    &\le \frac{2}{\hbar}\sigma_A \sigma_H
\end{align}
by the uncertainty principle,
just half the commutator.

What we do is this
There is an interpretation for the formula.
I'm going to write it like this.
\begin{align}
    \boxed{
        \sigma_H \frac{\sigma_A}{\left|\frac{d\bar{A}}{dt}\right|}
        \ge \frac{\hbar}{2}.
    }
\end{align}
This $\sigma_H$ of course is the uncertainty of the energy.
In a particular state,
there is some spread in the energy I get.

What about this $\frac{\sigma_A}{\left|\frac{d\bar{A}}{dt}\right|}$?
It's the uncertainty of $A$ divided by how fast the average of $A$ changes.
This thing has units of time.
The interpretation is that it is the time it takes for $\bar{A}$
to move enough to go outside the uncertainty.
This is how long it takes for the average value of $A$ to change appreciably.
It is the time it takes for $\bar{A}$ to change by $\sigam_A$.

Let me draw a picture.
I can measure $A$,
there is an average $\bar{A}$ and there is a spread $\sigma_A$.
Over time,
the central value would change,
and it's going to take time for the central value to change more than
$\sigma_A$.
So this is the time it takes for $A$ to change such that it matters.

If you're close to an eigenstate of energy,
this is going to be small,
and this is going to be large.
The expectation is going to change not by much.
If I'm in an eigenstate,
in fact I don't change at all.
On the other hand,
if it's a superposition of many energy eigenstates,
thne the wave function chages fast,
and this thing is small,
and hte expectation value moves wildly.
There is an uncertainty relation between energy and time,
but time is just the time it takes for stuff to change,
where stuff is whatever $A$ you put here.


\begin{question}
    Eigenstates are stable?
\end{question}
They are completely stable.
If you substitute it into the Schrodinger equation,
the ket is just going to evolve by phase,
but the phase doesn't matter anyway.
The universe is not in an eigenstate of the Hamiltonian of the universe,
because it it was,
how can we be moving around?

Example.
Have you heard of quarks and gluons?
It's a complicated theory.
Not so much to state.
But to solve.
There is a state where 3 quarks live together,
they have spin that point up called $\Delta^+$.
This doesn't last a long time.
Very quick,
it splits up into a neutron $n$ and a pion $\pi^+$.
Let's try to understand what this means in QM language.

We started off with a certain state
$\ket{\Delta^+}$.
It's not an energy eigenstate of the QCD Hamiltonian,
so it will hange,
but change in many ways.
When this evolves in time,
this is going to go into a state
\begin{align}
    U(t)\ket{\Delta^+} &=
    \alpha\ket{\Delta^+} + \beta \ket{n \pi^+} + \cdots
\end{align}
so it's not an eigenstate of QCD.
You can detect the neutron,
and you can measure the energy and momenta,
and you can make a plot.
Because of conservation of energy,
this state should have the same energy as the original state.

What I find is something like this of prob vs energy $E$.
I get a certain distribution,
a distribution with some mean at 1232 MeV
and a spread of 300 MeV.
What I say is this.

It takes time for this state $\ket{\Delta^+}$
to change appreciably.
How long does it take to change appreciably?
This number here 300 MeV.
Knowing the uncertainty of the energy,
and knowing the right hand side,
I should know this $\sigma_A/|d\bar{A}/dt|$,
the time it takes to appreciably change,
otherwise known as the lifetime,
sometimes written as $\tau$.

When you see the newspaper saying the Higgs boson has a lifetime whatever,
no one timed it,
they just observed it decayed into $\Delta^+$ and $\Delta^-$.
They say a plot like this.
It didn't look pretty,
there are error bars,
and there's this bump here with two data points.
No one believed it.

Then people looked at other decay modes.
They do some plots,
and notice this other peak.
It must be a coincidence,
Calculated the probability.
They say they discovered the Higgs and celebrated.

The regular uncertainty that is not time,
it's typically saturated,
you get the equality for the ground state.
There's a similar experience here.
For ground state things,
the lowest state that 3 quarks can have for example,
is going to saturate the inequality.

It becomes a definition almost.
What they really mean when they say the Higgs has some lifetime,
it's really saying the width is something.

In some experiments,
you can measure two things independently,
lifetime and width.
The relation is an inequality.
In some situations,
there is saturated.
But I can construct situations where the lifetime is much bigger than this
inequality suggests,
but I have to think about it.
In a harmonic oscillator,
the uncertain is minimum and bigger for higher states.
One can construct a state with much bigger uncertainty than the inequality
bound.

You learn in Griffiths to calculate the hydrogen atom levels.
I can't believe we're out of time.
You solve the hydrogen atom.
What you did is htis.
You ge the Hamiltonian with Coulomb interaciton.
Nothing else.
Then you find the eigenstates.
They are there the rest of the time.
But of course they don't.
You start iwth excited states,
get cofeee
and measuredecays.
To include not onlu the Coulomb interaction etween electorn and neuclus,
but also the quantised electromagnetic field.
If you do that,
those states you found were not eigenstates of the whole thing anymore,
but they're darn close to being eigenstates.
All those eigenstates you calculated in Griffiths are really close,
because the unceratinty is so small.
Those lines in the spectra you see are not infinitely thin.
There is a little spread.
This width is related to the lifetime.
When you excite the hydrogen atom,
let it decay,
send it through a prism,
they're not going to be infinitely thin,
some are going to be bright and some are not.

can you guess a relation between the brightess and the width?
If the line is very sharp,
this time must be large,
so you get very few photons iwth htat energy.
So thin lines are going to be dim.

On the other harnd, if $\sigma_H$ is going to be big fat line,
it's going to decay slow.

As an undergrad,
you learn about eigenstates and they never change,
but suddenly people say they jump.
There's no reason for that.
Well it's beccause they missed out the rest of the Hamiotnian,
but it's good to teach as an introduction.
