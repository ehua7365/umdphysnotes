\section{2021-09-22 Lecture}

This is a class that cannot fail,
because if it crashes and burns,
the qualifier fails.

This is my favourite lecture.
Sorry it's online.
It's conceptual.
This used to not be covered in standard classes,
but it because so popular there's no way out.
I'm going to start from something everyone knows.
The double slit experiment.

If you're not comfortable,
it means two things:
you don't know QM and you should laern straight away.

The best way to learn is to open the old Feynman lectures on physics.
Of course you can go to Youtube anddo the same thing,
with beautiful computer graphics,
but unfortanaltely many of them are wrong.
Feynman knows physics.


If there's only one hole nothing exciting happens.
The particle hits the screen,
mostly behind the hole.
Here I plot the number of particles that hit the screen.
If the hole is left,
you get particles on the lfet.
If the hole is right,
you get particles on the right.

Buty what happens when you shoot two holes?
They interferer with each other.
The result is surpricingly,
in fact the most likely place to get a particle is in the middle,
which is neither the single hole scneairos.
It's more than just the sum of the two.
You get an interference pattern.

You seen t his before.
It's pretty.
There are regions where the waves interfere desctructively and you don't get a
particle behin the screen.
that's how all waves behave.
If you make waves no thel ake,you get he same thing,
so it's not surprising.
What is surprising is tat the aprticles sometimes behave like particles,
and sometimes waves.
In fact,
what you get on the screen is this.

If you send particles one by one and you start counting,
every time you send a particle,
you get one spot on the screen,
however the probability fo getting apricles obeys the interference pattern.
That's this other way we see this particle-wave duality.

I'm going to switch to my laptop.
That's a drawing of the situation here.
I want to remind you how to descrbie this quanutm mechancially.


Let's say the left sitautions is $\ket{L}$
and the situation for the right hole open is $\ket{R}$.
When you have both holes open,
the state is going to be a linear combination
\begin{align}
    \ket{\psi} = \alpha\ket{L} + \beta\ket{R}
\end{align}
How do you get intference?
Let's measure $A_y$ that counts the number of particles,
or the probablity of the particle arriving at coordinate $y$
on hte screen.
Of course,
I want to take the expectation value.
\begin{align}
    \bra{\psi} A_y \ket{\psi} &=
    \left( \alpha^* \bra{L} + \beta^*\bra{R} \right)
    A_y
    \left( \alpha\ket{L} + \beta\ket{R} \right)\\
    &=
    |\alpha|^2 \bra{L}A_y\ket{L}
    + |\beta|^2 \bra{R}A_y\ket{R}
    + \alpha\beta^* \bra{L}A_y\ket{R}
    + \alpha^*\beta \bra{R}A_y\ket{L}
\end{align}
The first two terms are familiar.
They are just the probabilities for if the left hole is open only
and the rigt hole is open only respectively.
THe novelty of course in QM is that you have inerference terms,
the last two terms.
While the $|\alpha|^2$ and $|\beta^2|$ terms are positive,
those $\alpha\beta^*$ and $\beta\alpha^*$ terms are not necessarily positive,
so it's possible you cna have cancelation.
And that's why you get zeros in the distribution,
it's because of intereference and cancellation.

So this is well known.
The fact you have uqanutm intereference out of linear combinations,
you get this pretty well.

Let's compare this with a different situation.
Let's say I have a different siuaiton like this.
Except the two slits are never open at the same time.

I'm going to trow a dice.
If the number is odd,
I open the left hole.
If the number is even,
I open the right hole.
What am I going to get?

hal the time I'm going to get the L distributino,
and half the time I'm going to ge the R distribution.
And I get a double-bump distirbution.

So then this quantity,
The previous situation would be left AND right.
But now I will call this dice roll situation left OR right.
\begin{align}
    \bar{A}_y &=
    p_L \bra{L}A_y\ket{L}
    + p_R \bra{R}A_y\ket{R}
\end{align}
where $p_L$ and $p_R$ are the probabilities of having the left or right holes
open respectively.
You notice that this matces the first two terms of the previous formula.
What you do not find is the second two terms,
so there is no interference.
This is a different physical sitaution.

Those two things are not the same.
If you plot $A_y$ vs $y$,
in the QM superposition,
you get this
[picture]

But if you have the classical probability,
you get two bumps.
[picture]

These two sitautions are very common,
and you need to give these names.
having L and R in superpsotiion quantum mechanically,
we have a \emph{coherent} sum.
But the classical probability situation,
is the \emph{incoherent sum}.
The incorhernet sitaution happens all the time
in classical physics.

But the coherent sum is something very special to QM,
there is no classical analogue for that.

This situation where you have a combination of this,
where some of it is quanutm mehaonical,
but I also don't know what the wave function is,
appears commonly in physics too.

So ther eare two kinds of uncertainty.
There is the quantum uncertainty of the wave function,
butther'es the classical uncertainty of not knowing what the wave function is in
the first place.
There is a formalism to deal with this.

Suppose you have some observablve $A$.
Then
\begin{align}
    \bar{A} &=
    \underbrace{
    \sum_n \underbrace{p_n}_{\text{prob. in $\ket{n}$}} \underbrace{\bra{n} A \ket{n}}_{\text{quantum average}}
    }_{\text{classical average}}
\end{align}
So we have a quantum and a classical average

There's a cute way to write this as a trace.
\begin{align}
    \bar{A} = \Tr\left[
    \left( \sum_n p_n \ket{n}\bra{n} \right) A
    \right]
\end{align}
Is this the same?
It's easy to see,
because to compute the trace,
one way of computing the trace is to take a basis,
for example the same basis $\ket{n}$,
then compute the sandwich of whatever is in the trace and then sum over elements
of the basis,
but you see this is easy to compute.
\begin{align}
    \Tr\left[
    \left( \sum_n p_n \ket{n}\bra{n} \right) A
    \right]
    &=
    \sum_{m} \bra{m}
    \sum_n p_n \ket{n}\bra{n} A\ket{m}\\
    &= \sum_{m,n} \cdots
\end{align}
Then observe the following things.
Then we call that part the density matrix.
\begin{align}
    \rho = \sum_n p_n \ket{n}\bra{n}   
\end{align}
This contains all the information about the system.
It doesn't tell me what the wave function is,
because there's uncertainty about what the wave function is.
But the probabliies are al defined by this $\rho$.
States whree therare different clasical probailitesi of different wave fucntions
is called a \emph{mixed state}.
In situations wher we know what hte wave function is,
they are called \emph{pure dstates}.

To be clear,
\begin{align}
    \bar{A} = \Tr(\rho A)
\end{align}
Remember $\rho$ contains informaiton about all possible wave functions the
system may have.

Are we good up to now?
By the way,
you can raise a question by raising a hand,
but it's better to shout,
just unmute yourself and shout.
You can even go and ask questions in the chat.

\begin{question}
    Can you elaborate on mixed state?
\end{question}
If you know thw ave function of the sytem,
you call it a pure state,
that's what you learnt in QM so far.
A pure state is just a real state.
But then suppose you have a harmonic oscillator
with 30\% chance in the ground state and 70\% chance in the first excited state,
that is a mixed state.
You should not be confused though.
Take hte harmonic oscilator for exmaple.
\begin{align}
    \ket{\varphi} &=
    \frac{3}{5}\ket{0}
    + \frac{4}{5}\ket{1}
    \ne
    \rho
    =
    \left( \frac{3}{5} \right)^2 \ket{0}\bra{0}
    \left( \frac{4}{5} \right)^2 \ket{1}\bra{1}
\end{align}
These states here have the same probabiliteis,
but they are completely different states.
I can even add a phase.
\begin{align}
    \ket{\varphi} &=
    \frac{3}{5}\ket{0}
    + i\frac{4}{5}\ket{1}
    \ne
    \rho
    =
    \left( \frac{3}{5} \right)^2 \ket{0}\bra{0}
    \left( \frac{4}{5} \right)^2 \ket{1}\bra{1}
\end{align}
It's either one or the other.
No matter how smart I am,
I will know a way to measure.

here's an example.

How do we describe th sitaution where both holes are open.
We already have that,
a quantum superpsoition of left and right.
When I compuet the probablity of getting a particle on the sreen,
I get interference tersm.
If I have a phsae,
the probabilities $|\alpha|^2$ and $|\beta|^2$ don't change,
but I get a different interefernece pattern.

That's usual quantum mechanics,
but I can describe the sitaution with desity matrices too
which account for probabilities that arise because I am too dumb to know.

So I can describe the double slit experiment with density matrices too.
For example,
the incoehreent sum is
\begin{align}
    \rho &= \frac{1}{2} \ket{L}\bra{L}
    + \frac{1}{2}\ket{R}\bra{R}
\end{align}
and you can computet the probabilities
\begin{align}
    \bar{A} &=
    \Tr(\rho A_y)\\
    &=
    \Tr\left( \frac{1}{2}\ket{L}\bra{L} + \frac{1}{2}\ket{R}\bra{R} \right) A\\
    &=
    \bra{L}\left( 
    \frac{1}{2} \ket{L}\bra{L}
    + \frac{1}{2} \ket{R}\bra{R}
    \right)A_y
    \ket{R}
    + \bra{L}\left( 
    \frac{1}{2} \ket{L}\bra{L}
    + \frac{1}{2} \ket{R}\bra{R}
    \right)A_y
    \ket{R}\\
    &= \frac{1}{2}\bra{L} A_y \ket{L}
    + \frac{1}{2}\bra{R} A_{y} \ket{R}
\end{align}
Now look at what we thoguht it should be
\begin{align}
    \bar{A}_
    &= p_L\bra{L} A_y \ket{L}
    + p_R\bra{R} A_{y} \ket{R}
\end{align}
so the formalism works!
Let me tell you a couple of probabilities it's to satisfy.

Firstly, $\rho$ is a Hermitian operator.
It's weird,
because it's not an observable,
but it's the state of the system.
\begin{align}
    \rho^{\dagger} = \rho
\end{align}
Can you see this is true?
Sure
\begin{align}
    \left( \sum_n p_n \ket{n}\bra{n} \right)^{\dagger}
    &= \sum_n p_n^* \ket{n}\bra{n}\\
    &= \sum_n p_n \ket{n}\bra{n}
\end{align}
because $p_n$ are real probabilities,
so very straighforward.

Anotherp robability isthat it has unit trace.
\begin{align}
    \Tr \rho = 1
\end{align}
To see this,
\begin{align}
    \Tr \rho &=
    \Tr \sum_n p_n\ket{n}\bra{n}\\
    &= \sum_m \bra{m} \sum_n p_n\ket{n}\bra{n} \ket{m}\\
    &= \sum_{n,m} p_n \braket{m}{n} \braket{n}{m}\\
    &= \sum_n p_n\\
    &= 1
\end{align}
because the total probability is equal to 1.


Finally, $\rho$ is positive definite.
It follows from the fact that $p_n$ has to be positive or zero.
Positive definite means that
\begin{align}
    \bra{\phi}\rho\ket{\phi} &=
    \bra{\phi} \sum_n p_n\ket{n}\braket{n}{\phi}\\
    &=
    \sum_n p_n \braket{\phi}{n} \braket{n}{\phi}\\
    &= \sum_n p_n |\braket{n}{\phi}|^2\\
    &\ge 0
\end{align}

There's one final property.
For some special states, we have
\begin{align}
    \rho^2 = \rho
\end{align}
if and only if $\rho=\ket{\psi}\bra{\psi}$ is a pure state.
You can see this clearly because if
$\rho = \sum_n p_n \ket{n}\bra{n}$
and if $p_n=0$ except $p_{\bar{n}}=1$,
then obviously $\ket{\bar{n}}=\ket{n}$.
There are two things to prove,
because it's if and only if.
It's a simple test.
Suppose $\rho$ is pure,
then
\begin{align}
    \rho^2 &=
    \ket{\psi}\braket{\psi}{\psi}\bra{\psi} = \rho
\end{align}
It's a bit of extra work to go the other way around.

Suppose $\rho^2 = \rho$.
Then 
\begin{align}
    \sum_n p_n \ket{n}\bra{n}
    \sum_m p_m \ket{m}\bra{m}
    &= \sum_{n,m} p_n p_m \ket{n} 
    \underbrace{\braket{n}{m}}_{\delta_{nm}} \bra{m}\\
    &= \sum_n p_n^2 \ket{n}\bra{n}\\
    &= \sum_n p_n \ket{n}\bra{n}
\end{align}
so we know that $p_n^2=p_n$ for all $n$.
There are only two ways this can be true.
Either $p_n=0$ or $p_n=1$.
But probabilities sum to 1,
so only one of these can be $p_{\bar{n}}=1$,
and all the other $p_n=0$.
What does that prove?
In other words,
\begin{align}
    \rho = \sum_n p_n\ket{n}\bra{n} = \ket{\bar{n}}\bra{\bar{n}}
\end{align}

Oh there's a question.
It's just one word: magical.

You're just in awe of the power of density matrices and bras and kets?

By the way,
Landau invented density matrices.
He was two years too young to invent QM,
but he invented density matrices.
Bras and kets,
I'm tired of syaing it's a beautiful thing,
but I hope you can appreciate how the algebra just works without thinking.
It's importnat to understand why it works,
but once you do,
it's easy.

The name of this class is quantum and statistical physics.
Forgoet about the deep stuff.y
You're goign to use distributions of positions and momentum classical,
but the density analogue is density matrices,
where you have distributions over wave functions.


\begin{question}
    So $\rho$ is diagonal?
\end{question}
Well $\bra{n}\rho\ket{n'}$ is a diagonal matrix in this basis,
but in a different basis,
it will not necessarily be a diagonal matrix,
despite describing the same physics.
If the operator was diagonal,
I wouldn't talk about htis operator,
it would be overkill.
In this basiss to define $\rho$,
but may be diagonal,
but in another basis,
it's not diagonal
and it's not obvoius at all it's not a pure state.
