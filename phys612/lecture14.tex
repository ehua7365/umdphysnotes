\section{Recap}
\begin{align}
    \rho &= \frac{e^{-\beta H}}{Z}
\end{align}
where
\begin{align}
    Z &= \Tr e^{-\beta H}
\end{align}
and for any operator $\mathcal{O}$,
\begin{align}
    \bar{\mathcal{O}} &= \Tr(\rho\mathcal{O})
\end{align}
and if it's a continuous system,
\begin{align}
    \rho(q, p) &=
    \frac{e^{-\beta H(q, p)}}{Z}
\end{align}
and the partition function is
\begin{align}
    Z &=
    \frac{\int d^Nq d^Np e^{-\beta H(q, p)}}{h^N n!}
\end{align}

\begin{example}[Harmonic oscillator]
    The partition function is
    \begin{align}
        Z &=
        \sum_{n=0}^{\infty} \bra{n} e^{-\beta H} \ket{n}\\
        &=
        \sum_{n=0}^{\infty} e^{-\beta\hbar\omega\left( n + \frac{1}{2}
        \right)}\\
        &=
        e^{-\beta \hbar\omega/2}
        \sum_{n=0}^{\infty} \left( e^{-\beta \hbar \omega} \right)^n\\
        &=
        \frac{1}{e^{\beta \hbar\omega/2} + e^{-\beta\hbar\omega/2}}\\
        &=
        \frac{1}{2}
        \frac{1}{\sinh\left(\beta\hbar\omega/2 \right)}
    \end{align}
    So then the average energy is
    \begin{align}
        \bar{E} &= \Tr(\rho H)\\
        &= \frac{1}{Z} \sum_{n=0}^{\infty}
        \bra{n} e^{-\beta H} H \ket{n}\\
        &=
        \frac{1}{Z} \sum_{n=0}^{\infty}
        E_n e^{-\beta E_n}\\
        &= \frac{1}{Z}\left( -\frac{\partial}{\partial\beta} \right) Z\\
        &=
        -\frac{\partial}{\partial\beta}\ln Z\\
        &=
        \frac{\partial}{\partial\beta}
        \ln\left( 
            Z \sinh(\beta\hbar\omega/2)
        \right)\\
        &=
        \frac{\cosh(\beta\hbar\omega/2)}{\sinh(\beta\hbar\omega/2)}
        \frac{\hbar\omega}{2}
    \end{align}
\end{example}

In the high temperature limit $k_B T \ge \hbar\omega$,
this expression approaches
\begin{align}
    \bar{E} \to k_B T
\end{align}
At high energy,
it doesn't matter what the energy is,
the energy is just going to be given by the thermal energy.
You might know this result from somewhere right?
It's closely related to the equipartition theorem.

In the low-temperature limit,
\begin{align}
    \bar{E} &=
    \hbar\omega
    \frac{1 + e^{-\beta\hbar\omega}}{1 + e^{-\beta\hbar\omega}}\\
    &\approx
    \frac{\hbar\omega}{2}\left(
        1 + 2e^{-\beta\hbar\omega} + \cdots
    \right)
\end{align}
so no matter how low temperature you get,
you can't get lower than $\hbar\omega/2$,
which is the \emph{zero point energy}.
The excitations are exponentially small.

This is how quantum mechanics was invented by the way.
If you have a cavity,
each mode behaves like a harmonic oscillator.
The only thing with different wave lengths is that the frequency changes.
Every mode of the EM rays inside the black body is going to contribute some
energy $k_B T$,
but there's an infinite number of modes,
because I can make them smaller,
so this $\bar{E} = k_B T$ must be wrong.

The correct QM calculation is that no matter how high $T$ is,
when $\omega$ is large enough,
it stays very close to the ground state,
exponentially so.
No matter what the temperature is,
for the high energy modes,
they stay in the ground state,
and they shut off,
and that's how QM saves it.

There's so much physics to talk about.
In a certain way,
we're going to derive this result over and over again,
and get all amazing results.

For example,
get a metal piece reasonably cold.
By metal,
I mean a non-conducting crystal,
so a salt,
a rock.
The goal of this class is that you can run this calculation in your head
immediately.
We're going to derive this result,
hand it to your previous self and tell me how are you going to derive it.
Here we're still understanding the formalism.

How let's go back to the classical harmonic oscillator
and the classical calculator..

\begin{example}[Classical harmonic oscillator]
    Let's say we have a simple harmonic oscillator
    \begin{align}
        H(x, p) &=
        \frac{p^2}{2m} + \frac{m\omega^2}{2}x^2
    \end{align}
    and the partition function should be an integral
    \begin{align}
        Z &=
        \int_{-\infty}^{\infty}\frac{dx\,dp}{h}
        e^{-\beta\left(
            \frac{p^2}{2m} + \frac{m\omega^2}{2}x^2
        \right)}
    \end{align}
    which is a Gaussian integral.
\end{example}
My intent for today was to give you homework was to show you how to do a
Gaussian integral,
but you have a midterm,
so it's next week.
My opinion is that every integral should be done by computer,
but this one is worth known.

Suppose you are stuck on an island.
The only way out is to build a steam engine out of coconuts and bananas.
To build a steam engine,
you need to know thermodynamics,
and so you need to calculate partition functions,
so you need to know how to calculate Gaussian integrals.

Remember this one
\begin{align}
    Z = \frac{1}{h}
    \int_{-\infty}^{\infty} dx\,
    e^{\frac{-\beta m\omega^2}{2} x^2}
    \int_{-\infty}^{\infty} dp\,
    e^{\frac{-\beta p^2}{2}}
\end{align}
Look at the first integral. By dimension analysis,
there's only one thing it could depend on,
because $x$ is of units lengths,
and the argument of the exponential is dimensionless,
so $1/\sqrt{\beta m \omega^2}$ has units of length.
\begin{align}
    \int_{-\infty}^{\infty} dx\,
    e^{\frac{-\beta\hbar\omega^2}{2} x^2}
    \propto
    \frac{1}{\sqrt{\beta m \omega}}
\end{align}
up to some dimensionless constant, and it happens to be
\begin{align}
    \int_{-\infty}^{\infty} dx\,
    e^{\frac{-\beta\hbar\omega^2}{2} x^2}
    &=
    \sqrt{\frac{2\pi k_B T}{\omega^2}}
\end{align}
and
\begin{align}
    \int_{-\infty}^{\infty} dp\,
    e^{\frac{-\beta p^2}{2}}
    &=
    \sqrt{2\pi m k_B T}
\end{align}
so
\begin{align}
    Z &= \frac{k_B T}{\hbar \omega}.
\end{align}
Relax about the $\hbar$, it's just a constant that cancels out when
you calculate expectations!

So if you're stuck on an island and your life depends it,
you'll know.

One thing you'll learn in this course is how to take derivatives with logs.
\begin{align}
    \bar{E} &=
    \frac{1}{Z} \int dx\,dp\,
    e^{-\beta H} H\\
    &=
    \frac{1}{Z}\left( \frac{-\partial}{\partial\beta} \right)Z\\
    &=
    -\frac{\partial}{\partial \beta}Z\\
    &= -\frac{\partial}{\partial\beta}\ln Z\\
    &=
    \frac{\partial}{\partial\beta} \ln(\beta\hbar\omega)\\
    &=
    \frac{1}{\beta}\\
    &= k_B T
\end{align}
which is not surprising as you did the quantum calculation
and showed this is the high energy limit.

That's why there's a delivery of liquid nitrogen and once in a blue moon liquid
helium,
because people want to do experiments at low energy.


In the old days, the notation $\langle E\rangle$ was common,
and Dirac decided to break it up into $\bra{\psi}E\ket{\psi}$.

Recall how to compute the variance.
\begin{align}
    \overline{(E - \overline{E})^2}
    &=
    \overline{E^2} - \overline{E}^2\\
    &=
    \frac{1}{Z} \int dx\,dp\,
    e^{-\beta H} H^2
    -\left( 
        \int dx\, dp
        \frac{e^{-\beta H}}{Z} H
    \right)^2\\
    &= \frac{\partial^2}{\partial\beta^2} \ln Z\\
    &= \left( k_B T \right)^2
\end{align}
Now if you recall,
\begin{align}
    \frac{\partial}{\partial\beta} \ln Z
    &=
    \frac{1}{Z} \frac{\partial}{\partial\beta} Z\\
    \frac{\partial^2}{\partial\beta^2} \ln Z &=
    - \frac{1}{Z^2}\left( \frac{\partial^2 Z}{\partial \beta^2} \right)
\end{align}
Student: Stop!
You want me to copy how to do second derivatives of a log?

Anyway,
so I'm going to consider the spread of energy,
the noise to signal ratio,
\begin{align}
    \frac{\sqrt{\overline{(E - \overbar{E})^2}}}{\bar{E}} = 1
\end{align}
which is not good,
because there's so much fluctuation.
But it's no good doing statistical mechanics on just one single harmonic
oscillator,
we need many more.

Let me do $N$ harmonic oscillators.
Instead of having just 1 $x$ and one $p$,
I'm going to have $N$ $x$s and $N$ $p$s.
So Hamiltonian is
\begin{align}
    H(x, p) &=
    \frac{p_1^2}{2m}
    + \frac{m\omega^2}{2} x_1^2
    \frac{p_2^2}{2m}
    + \frac{m\omega^2}{2} x_2^2
    + \cdots
\end{align}
and so we get a product
\begin{align}
    Z &=
    \prod_{i=1}^{N}\int_{-\infty}^{\infty}\frac{dx_i\,dp_i}{h}
    e^{-\beta\left(
        \frac{p_i^2}{2m} + \frac{m\omega^2}{2}x_i^2
    \right)}
\end{align}
And if you follow the steps,
you get
\begin{align}
    \overline{(E - \overline{E})^2} = N(k_B T)^2
\end{align}
and the noise to signal ratio drops to
\begin{align}
    \frac{\sqrt{\overline{(E - \overbar{E})^2}}}{\bar{E}} = 
    \frac{\sqrt{N} k_B T}{N k_B T}\\
    = \frac{1}{\sqrt{N}}
\end{align}
so if you have $10^{23}$ particles,
the fluctuation in total energy is very very small,
and that's why statistical mechanics works.
