\section{Quantum gases}
Grand canonical ensemble.
Second quantization. $\ket{n_1,n_2,\ldots}$.

Upper sign is for bosons,
lower sign is for fermions.
The grand potential is
\begin{align}
    \Omega &= \pm gk_B T V
    \int \frac{d^3k}{(2\pi)^3}
    \ln \left[ 
    1 \mp e^{-\beta\left( 
    \frac{\hbar^2 k^2}{2M} - \mu
    \right)}
    \right]
\end{align}
and then the number density is
\begin{align}
    \frac{N}{V} &= - \frac{1}{V}\left.\frac{\partial \Omega}{\partial
    \mu}\right|_{T,V}\\
    &=
    g \int \frac{d^3k}{(2\pi)^3}
    \frac{1}{e^{\beta\left( E_k - \mu \right) \mp 1}}
\end{align}
and the energy density is
\begin{align}
    \frac{E}{V} &=
    \frac{\Omega - TS - \mu N}{V}\\
    &=
    g \int\frac{d^3 k}{(2\pi)^3}
    \frac{E_k}{e^{\beta\left( E_k - \mu \right)} \mp 1}
\end{align}

We don't know what the chemical potential is if I give you a box of atoms.
More likely you know the density and energy.

I adjust $\mu$ and I know the temperature.
I can then get a particular values of $N$ for a particular $T$ I know.
Then I get $\mu$ to compute the temperature nd anything you want.

There is no instrument I can stick in a bottle and measure the chemical
potential.
Maybe there is but it's not practical.

I need to go through this dance of fixing this and tuning $\mU$ to get the
property density.
If I have this integral calculated,
I will do this once and I'm done.
But I can't.
So I compute the integral numerically find the value of $\mu$.

Maybe I should put it in the homework so you do it once in your life.

I do this to approximate and make the physics explicit.
To be sincere,
it is a stupid tradition that I know no computers and I have to do everything
with pen and paper.
It's not completely meaningless to do this because it does bring up some physics.

Let's pretend we do not own computes and extract things with pen and paper.
The first thing I want to consider is the classical limit.
The classical limit may surprise you,
because you think it's getting $\hbar\to 0$,
but that's fake situation,
not even God can change $\hbar$,
because it's a creation of the human mind.
If I use a unit system where $\hbar=1$,
it's gone.
So the classical limit is actually when
\begin{align}
    e^{\beta\mu} \ll 1
\end{align}
How can this be small?
Certainly possible because it's an exponential.

It's necessary that $\mu$ is negative,
very negative.
That is,
\begin{align}
    - \mu \gg k_B T
\end{align}
So then $\mu$ has to be very negative and in magnitude much larger than
$k_BT$.
I claim this is the classical limit,
because it recovers the good old results you recover in kindergarten.

Let's look at the occupation number
\begin{align}
    n(k) &=
    \frac{1}{e^{\beta\left( E_k - \mu \right) \mp 1}}
\end{align}
This exponential is always a positive number,
so some people call this quantity
$z = e^{\beta \mu}$,
called a very poetic name,
the fugacity,
the property of being temporary,
something that goes away very quickly.
I have no idea where it's used in any other circumstances.

Richard: I used to hear geochemists use that term often about the exchange of
oxygen and CO2.

If I consider the limit where the fugacity is small,
no matter what $e$ is,
this is going to be much larger than 1,
and I can neglect this $\mp 1$,
and so approximately,
\begin{align}
    n(k) &=
    \frac{1}{e^{\beta\left( E_k - \mu \right) \mp 1}}\\
    &\approx 
    e^{-\beta E_k} e^{\beta \mu}
\end{align}
and this is classical.
You see a Boltzmann factor appear here,
so it's beginning to look lie a classical gas.
And Siri is giving me a band called fugacity.
Why is the band called fugacity?
That's an optional question in the homework.

I can then do this approximation inside of the integral.
\begin{align}
    \frac{N}{V} &=
    g \int \frac{d^3k}{(2\pi)^3}
    e^{-\beta \frac{\hbar^2 k^2}{2m}} e^{\beta\mu}
\end{align}
which is a Gaussian integral you've been doing the whole semester,
and if I'm not mistaken.
\begin{align}
    \int \frac{d^3k}{(2\pi)^3}
    e^{-\beta \frac{\hbar^2 k^2}{2m}}
    &=
    \frac{1}{8\pi^2}
    \left( \frac{2mk_B T}{\hbar^2} \right)^{3/2}
    = 
\end{align}
which you could check out by dimension analysis.
I wrote it in this particular way because it has a name
Here $\lambda_T$ is the thermal length,
\begin{align}
    \frac{1}{\lambda_T^2}
    =
    \frac{2mk_B T}{\hbar^2}
\end{align}
which different books may define differently up to factors of $2\pi$.


Serious people wrote papers about how to do dimensional analysis to work out
factors of $2\pi$ in particle physics.
Completely insane.
$2\pi$ is a big number, like 6,
so if you miss it you get the number completely wrong.
It's not a science,
it's an art,
looking at integrals,
squinting and extracting powers of $2\pi$.
It's dark magic, but we're not going to do that,
just compute the integral.
\begin{align}
    \frac{N}{V} &=
    g e^{\beta N} \frac{1}{\lambda_T^3}
\end{align}

One thing I know here,
if I work in the regime of small fugacity,
This is going to be small.
Small compared to what?

Then we compute the free energy
\begin{align}
    F &= \Omega + \mu N\\
    &=
    \pm k_B T V
    \int \frac{d^3k}{(2\pi)^3}
    \ln\left[ 
    1 \mp e^{-\beta \left( E_k - \mu \right)}
    \right]
\end{align}

Story about the astronaut pen.

\begin{question}
    What would I google to dabble the dark arts of $2\pi$?
\end{question}
Places you shouldn't.
There's an actual paper called counting powers of $2\pi$.
It's not general,
only a very particular context on particle physics.
It's based on the absurd statement that the number of $\pi$'s on the left and
right have to match.
It's hilarious,
because open the textbook and find an exception to this rule.
It's an irrational number.

Anyway, then I have to do this approximation.
Look at the log.
\begin{align}
    F &\approx
    k_B T N + \mu` N
\end{align}
but which term is larger in magnitude?
$\mu$ of course.
So my energy is just
\begin{align}
    F &= N\mu
\end{align}
But what is $\mu$?
Well I can extract it from $N/V$.
\begin{align}
    F &=
    N\ln \left( \frac{N}{V} \frac{\lambda_T^3}{g} \right) \frac{1}{\beta}
\end{align}
That's what I promised before.
Explicitly then I get that.
Writing the thermal wavelength the way it's supposed to be,
I get
\begin{align}
    F & k_B T N
    \ln\left( 
    \frac{V}{N} \left( k_B T \right)^{3/2} \cdots
    \right)
\end{align}
which is what you derived before the hard way.
What's cool about this,
is that you don't have to stick the fudge factor of $N!$.

The fact we got $N$ here in the quantum derivation means we really need the $N!$
in the classical case.
Track down the $N!$ in the classical case derivation in our homework,
and you will find it will not match if you don't have the $N!$.

We had that fudge factor for 2 months for no good reason,
but finally we have a good reason for it.
\begin{question}
What's the physical interpretation of the thermal length.
\end{question}
If I have a particle in a gas,
the de Broglie wavelength is the thermal wavelength.
QM,
every particle you can think of as a wave,
and the wavelength of that wave is the wavelength of a typical particle.

The approximation was that the fugacity $z=e^{\beta N}\ll 1$,
but that's the same thing as saying the density times the thermal volume,
which is the de Broglie wavelength cubed is very small
\begin{align}
    \frac{N}{V} \lambda_T^3 \ll 1
\end{align}
and that is what I mean by a dilute gas.
What I'm saying is that to have a classical gas,
the de Broglie wavelength of a particle has to be a lot smaller than that.
If you think about it that's how quantum mechanics works.
If you have a particle moving along obstacles this big,
it's like rays.
But if you have wavelengths similar to distances,
it's behaving like waves.

Te complicated integral isn't that visible,
and the trivial thing is not.

By the way,
this is typically how you calculate the free energy,
you typically have to solve for $\mu$ from the formula for the density $N/V$.

It's a little too fast to say high temperature is classical because $T$ is
larger than $\mu$,
because not really,
because $\mu$ can also increase with $T$.

It's more useful to say that you want $z=e^{\beta N}$ to be large.
You want high density and low temperature.

You want the temperature to be large,
but large compared to what.
There's nothing large to compare against.
So you compare the thermal wavelength with the density.
That's the only thing that make sense.
There are classical gases that are not that hot,
as long as the density is smaller.

It's only the relation.

The direction is right.
Cold and dense things are quantum mechanical.
That's why every day at the physics department,
you get a truck with liquid nitrogen,
because no one wants to do high temperature.
Unless you do plasma physics,
and you live your life in classical physics.
