[30 min late]
\section{Planck's Law}
Derived Planck's law.

\section{Identical particles}
Suppose you have two particles moving along a line.
As you know, the system only has one wave function,
which is $\psi(x_1,x_2)$,
not a wave function for each but one wave function for all.
Suppose it's one dimensional and there is the kinetic energy operator
\begin{align}
    -\frac{\hbar^2}{2m}\frac{\partial}{\partial x_1}
    -\frac{\hbar^2}{2m}\frac{\partial}{\partial x_2}
\end{align}
but the new thing in QM is that you can have identical particles.
There is nothing you can do to distinguish them.
That has consequences.
Suppose someone exchanged the two particles,
and I don't measure anything else.
Should I obtain any results?
I shouldn't.

So the magnitude squared of the wave function before and after exchange should
be the same.
\begin{align}
    |\psi(x_1, x_2)|^2 &= |\psi(x_2, x_1)|^2
\end{align}
Putting those two things together,
I can have either
\begin{align}
    \psi(x_1, x_2)^2 &= \pm\psi(x_2, x_1)^2
\end{align}
There could be a phase,
but if I exchange the particle twice, I should get the same thing back.

The not obvious thing is that if I have two electrons,
any two electrons at any time,
this sign is a property of the kind of particle.

All particles in the universe are two types.
Either they don't flip the sign,
called bosons,
or they do,
called fermions.

Another interesting fact,
is particles have spin, as you know,
which could be either integer and they're always bosons,
or they could be fermions, spin in half-integers.

Give me examples of bosons?
Photons.
How do we know?

Two photons of the same wavelength,
one particle mode is excited,
and there are states in the magnetic field with an arbitrary number of photons
with the same momentum,
and they are bosons.
Same with phonons.
It could be very complicated.

In fact,
every time you have a state with two fermions,
the composite system is going to be a boson.
Every time you have an even number of fermions you have bosons.

That's why we don't know if you are a fermion or a boson,
because you'd have to know exactly how many particles you have.

All this is obvious.

Let me do an example.

Consider particles in a 3D box.
Let's say we have a single particle
\begin{align}
    \braket{xyz}{\psi} = \psi(x, y, z) =
    \left( \frac{2}{L} \right)^{3/2}
    \sin\left( \frac{n_x \pi x}{L} \right)
    \sin\left( \frac{n_y \pi x}{L} \right)
    \sin\left( \frac{n_z \pi x}{L} \right)
\end{align}
where $n_x,n_y,n_z=1,2,\ldots$.

And the energy is
\begin{align}
    E_{n_x,n_y,n_z} &=
    \frac{\hbar^2}{2mL^2}\left(
        n_x^2 + n_y^2 + n_z^2
    \right)
\end{align}
by dimension analysis.
Dimension analysis is hard to teach.
People don't know.

And so the Hamiltonian can be indexed by
\begin{align}
    H\ket{m} =  \frac{\hbar^2}{2ML^2} m^2 \ket{m}
\end{align}

All I have to do is create single particle states,
and I remember the Hamiltonian is the sum of kinetic energies.
$H=K_1 + K_2$.
Then what happens?

I act with $K_1$ on the first one,
and I leave along the second one.
Then I act $K_2$ on the second one,
and leave alone the first one.
That's really what I mean by $K_1 + K_2$,
a shorthand for
\begin{align}
    H = K_1\otimes I + I\otimes K_2
\end{align}
What's going to happen?

This gives
\begin{align}
    H\ket{m_1}\otimes\ket{m_2} &=
    E_{m_1}\ket{m_1}\otimes\ket{m_2} +
    E_{m_2}\ket{m_1}\otimes\ket{m_2}\\
    &=
    \left( E_{m_1} + E_{m_2} \right)
    \ket{m_1}\otimes\ket{m_2}
\end{align}
We'll continue next time.
