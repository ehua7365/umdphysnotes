\section{Photons}
Suppose you have a metal box at temperature $T$.
There is an electromagnetic field inside in equilibrium with the walls.
The field vanishes at the walls though.

Why should anybody study this?
Because all bodies which absorb all radiation shining on them (black bodies)
emit radiation the same way when heated and a metal cavity is the best or
easiest way to study.

The energy emitted by A equals the energy absorbed by a black body B equals the
energy emitted by B.
This is true frequency by frequency,
so $u(\nu)$ which is the energy density per unit is universal for every black
body.

We will find $u(\nu)$ by applying equilibrium statistical mechanics to the
electromagnetic field inside the box.

In order to have a Hamiltonian formalism,
we'll use the variables
$\vec{A}(\vec{r}, t)$ instead of $\vec{E}$ and $\vec{B}$.

Different $\vec{A}$'s differing by a gauge transformation,
give the same $\vec{E}$ and $\vec{B}$ and describe the same physics.
We'll impose the condition
\begin{align}
    \vec{\nabla}\cdot{A} &= 0\\
    \phi &= 0
\end{align}
to select one $\vec{A}$ for every physically distinct field $\vec{E},\vec{B}$.
Recalling that we can recover the electric and magnetic fields with
\begin{align}
    \vec{E} &= -\frac{\partial A}{c \partial t} - \vec{\nabla}\phi
    =
    -\frac{1}{c}\frac{\partial \vec{A}}{\partial t}\\
    \vec{B} &= \vec{\nabla}\times\vec{A}
\end{align}
The Lagrangian
\begin{align}
    L &=
    \frac{1}{8\pi} \int d^3 r\,
    \left[ 
    \left( \frac{\partial \vec{A}}{c \partial t} \right)^2
    - \left( \vec{\nabla}\times\vec{A} \right)^2
    \right]
\end{align}
leads to the equation of motion
\begin{align}
    \frac{1}{c^2} \frac{\partial A}{\partial t^2}
    - \nabla^2 \vec{A} &= 0
\end{align}
which is equivalent to Maxwell's equations.
The Hamiltonian is
\begin{align}
    H &=
    \int d^3r\left[
    \vec{\Pi}\cdot \frac{\partial \vec{A}}{\partial t} - H
    \right]\\
    &=
    \int d^3r\,\left[
    \Pi^2 4\pi c^2
    - \frac{(4\pi c)^2}{8\pi} \Pi^2
    + \frac{\vec{\nabla}\times\vec{A}}{8\pi}
    \right]
\end{align}
where
\begin{align}
    \vec{\Pi} &:=
    \frac{\partial H}{\partial \dot{\vec{A}}(\vec{r})}
    =
    \frac{1}{4\pi} \frac{1}{c^2} \frac{\partial \vec{A}}{\partial t}\\
\end{align}
so then
\begin{align}
    H
    &=
    \frac{1}{8\pi} \int d^3r\,\left[ 
    \underbrace{\left( 4\pi c \right)^2 \Pi^2}_{E^2}
    +
    \underbrace{\vec{\nabla}\times\vec{A}}_{B^2}
    \right]
\end{align}
Applying a change of variables to
\begin{align}
    \vec{A}(\vec{r}, t) &=
    \frac{1}{\sqrt{V}} \sum_k
    \vec{Q}(\vec{k}, t) e^{i\vec{k}\cdot \vec{r}}\\
    \vec{\Pi}(\vec{r}, t) &=
    \frac{1}{\sqrt{V}} \sum_k
    \frac{\vec{P}(\vec{k}, t)}{\sqrt{4\pi} c^2} e^{i\vec{k}\cdot\vec{r}}
\end{align}
where $\vec{k}=\frac{2\pi}{L}\vec{n}$ with $\vec{n}=0,\pm 1,\ldots$
and
\begin{align}
    \vec{Q}_{\vec{k}}^* &= \vec{Q}_{-\vec{k}}\\
    \vec{P}_{\vec{k}}^* &= \vec{P}_{-\vec{k}}
\end{align}
so then the Hamiltonian becomes decoupled as
\begin{align}
    H &=
    \frac{1}{2}\sum_{k}\left[ 
    \vec{P}_{\vec{k}}\cdot\vec{P}_{-\vec{k}}
    +
    \underbrace{c^2 k^2}_{=:\omega_k^2}
    \vec{Q}_{\vec{k}}\cdot\vec{Q}_{-\vec{k}}
    \right]
\end{align}
which is very much like the phonon Hamiltonian.
Classically,
the energy density is given by the equipartition theorem
\begin{align}
    E &=
    \sum_{k} \frac{1}{2} k_B
    \times \underbrace{2}_{\textrm{$P$ and $Q$ are quadratic}}
    \times \underbrace{2}_{\textrm{2 independent polarizations}}\\
    &\approx V \int d^3 k 2k_B T\\
    &\to \infty
\end{align}
The divergence comes because contrary to the phonon case,
there is not cutoff at high momentum $k$.
Still,
the contribution from every frequency range
\begin{align}
    \frac{E}{V} &=
    \frac{2k_B T}{2\pi^2} \int_{0}^{\infty} dk\, k^2\\
    &=
    \int_{0}^{\infty}
    \underbrace{\frac{k_B T}{\pi^2} \frac{(2\pi)^3}{c^3} \nu^2}_{u(\nu)}
    \, d\nu
\end{align}
and with $\nu = \frac{kc}{2\pi}$,
the energy density between $\nu$ and $\nu + d\nu $ is
\begin{align}
    u(\nu) \, d\nu &=
    \frac{8k_B T \pi}{c^3} \nu^2\, d\nu
\end{align}
this is known as the \emph{Rayleigh-Jeans law},
which agrees with experiments in the low frequency regime.
But not for high frequencies as it diverges!

The quantum theory is obtained by making
$\vec{A}$, $\vec{\Pi}$, $\vec{Q}$ and $\vec{P}$
into operators.

Assuming canonical commutation relations for $\vec{A}$ and $\vec{\Pi}$.
\begin{align}
    \left[ \hat{A}_i(\vec{r}), \hat{\Pi}_j(\vec{r}') \right]
    =
    i\hbar \delta_{ij} \delta\left( \vec{r} - \vec{r}' \right)
\end{align}
we find
\begin{align}
    \left[ \hat{A}_i(\vec{r}), \hat{\Pi}_j(\vec{r}') \right] &=
    \frac{1}{V} \sum_{k,k'}
    e^{i\left( \vec{k}\cdot\vec{r} + \vec{k}'\cdot\vec{r}' \right)}
    \underbrace{ \left[ Q_i(\vec{k}), P_j(\vec{k}') \right]}_{%
    i\hbar \delta_{ij} \delta_{\vec{k},-\vec{k}'}
    }\\
    &=
    \frac{i\hbar}{V} \sum_{\vec{k}}
    e^{i\vec{k}\cdot\left( \vec{r} - \vec{r}' \right)} \delta_{ij}\\
    &=
    i\hbar \delta_{ij} \delta\left( \vec{r} - \vec{r}' \right)
\end{align}
so then
\begin{align}
    \left[ Q_i\left( \vec{r} \right), P_j\left( \vec{r}' \right) \right]
    =
    i\hbar \delta_{ij} \delta_{\vec{k}, -\vec{k}'}
\end{align}
and all the other commutators between $Q_k$ and $P_k$ vanish.
We then define
\begin{align}
    \hat{a}_{\vec{k}}^{\alpha} &=
    \left( 
    \sqrt{\frac{\omega_k}{2\hbar}} \vec{Q}(\vec{k})
    + \frac{i}{\sqrt{2\omega_k\hbar}} \vec{P}(\vec{k})
    \right)
    \cdot e_{\vec{k}}^{\alpha}\\
    {\hat{a}_{\vec{k}}^{\alpha}}^\dagger &=
    \left( 
    \sqrt{\frac{\omega_k}{2\hbar}} \vec{Q}(\vec{k})
    - \frac{i}{\sqrt{2\omega_k\hbar}} \vec{P}(\vec{k})
    \right)
    \cdot e_{\vec{k}}^{\alpha}
\end{align}
where $e_{\vec{k}}^1$ and $e_{\vec{k}}^2$
are two unit vectors orthogonal to $\vec{k}$,
which are the polarization directions.
And the commutation relations
\begin{align}
    \left[
    \hat{a}_{\vec{k}}^{\alpha},
    {\hat{a}_{\vec{k}}^{\beta}}^\dagger,
    \right]
    &=
    -i \sqrt{\frac{\omega_k}{2\hbar}}
    \frac{1}{\sqrt{2\omega_{k'}\hbar}}
    \underbrace{\left[ Q_i(\vec{k}), P_j(\vec{k}'), \right]}_{%
    i\hbar \delta_{ij} \delta_{k,k'}
    }
    e_i^{\alpha}(k)
    e_j^{\beta}(k)\nonumber\\
    &\qquad
    - \sqrt{\frac{\omega_{k'}}{2\hbar}}
    \frac{1}{\sqrt{2\omega_k \hbar}}
    \underbrace{\left[ P_j(\vec{k}), Q_i(\vec{k}') \right]}_{%
    -i\hbar \delta_{ij} \delta_{k,k'}
    }
    e_j^{\alpha}(k)
    e_i^{\alpha}(k')\\
    &=
    \delta_{\alpha\beta} \delta_{\vec{k},\vec{k}'}
\end{align}
and also
\begin{align}
    \left[ \hat{a}_{\vec{k}}^{\alpha}, \hat{a}_{\vec{k}'}^{\beta} \right] =
    \left[
    {\hat{a}_{\vec{k}}^{\alpha}}^\dagger,
    {\hat{a}_{\vec{k}'}^{\beta}}^\dagger
    \right] = 0
\end{align}
Thus,
$\hat{a}^\alpha(k)$ and ${\hat{a}^{\alpha}}^\dagger(k)$
are creation and annihilation operators for photons with momentum $k$ and
polarization $e^{\alpha}(k)$.
In terms of them,
the Hamiltonian is
\begin{align}
    \hat{H} &=
    \frac{1}{2} \sum_{\alpha=1,2} V \int d^3k\,\left[
    -\frac{\hbar\omega_k}{2}
    \left( 
    \hat{a}_{\vec{k}}^{\alpha}
    - {\hat{a}_{-\vec{k}}^{\alpha}}^\dagger
    \right)
    \left( 
    \hat{a}_{-\vec{k}}^{\alpha}
    - {\hat{a}_{\vec{k}}^{\alpha}}^\dagger
    \right)
    +
    \omega_k^2
    \left( 
    \hat{a}_{\vec{k}}^{\alpha}
    - {\hat{a}_{-\vec{k}}^{\alpha}}^\dagger
    \right)
    \left( 
    \hat{a}_{-\vec{k}}^{\alpha}
    - {\hat{a}_{\vec{k}}^{\alpha}}^\dagger
    \right)
    \right]\\
    &\approx
    \frac{1}{2} \sum_{\alpha=1,2} \int d^3k\left[ 
    \frac{\hbar\omega_k}{2} 2
    \hat{a}_{k}^{\alpha}
    {\hat{a}_{k}^{\alpha}}^\dagger
    +
    {\hat{a}_{k}^{\alpha}}^\dagger
    \hat{a}_{k}^{\alpha}
    2
    \frac{\hbar\omega_k}{2}
    \right]\\
    &=
    \sum_{\alpha=1,2} \int d^3k\left[ 
    {\hat{a}_{k}^{\alpha}}^\dagger
    \hat{a}_{k}^{\alpha}
    +
    \frac{1}{2}
    \right]
    \hbar\omega_k
\end{align}
so that's one harmonic oscillator for every $\vec{k}$ and $\alpha$.

How we can calculate the partition function
\begin{align}
    Z &= \Tr e^{-\beta \hat{H}}\\
    &=
    \prod_{\vec{k},\alpha} \sum_{n_k=0}^{\infty}
    e^{-\beta\hbar\omega_k\left( n_k + \frac{1}{2} \right)}\\
    &= \prod_{\vec{k}, \alpha}
    \frac{e^{-\beta\hbar\omega_k/2}}{1 - e^{-\beta\omega_k\hbar}}\\
    &= \prod_{\vec{k},\alpha} \frac{2}{\sinh(\beta\hbar\omega_k/2)}
\end{align}
and the free energy
\begin{align}
    F &=
    -k_B T \ln Z\\
    &=
    k_B T \sum_{\vec{k}, \alpha} \ln\left( 
    \frac{1 - e^{-\beta \omega_k \hbar}}{e^{-\beta\hbar\omega_k/2}}
    \right)\\
    &=
    k_B T \sum_{\vec{k},\alpha}\left[ 
    \ln\left( 
    1 - e^{-\beta\hbar\omega_k}
    \right)
    +
    k_B T \beta \hbar\omega_k
    \right]\\
    &=
    k_B T \sum_{\vec{k},\alpha}\left[ 
    \ln\left( 
    1 - e^{-\beta\hbar\omega_k}
    \right)
    +
    \underbrace{\hbar\omega_k}_{\textrm{$T$-independent zero-point energy}}
    \right]\\
    &\approx
    2 k_B T V \int d^3k \ln\left( 
    1 - e^{-\beta \hbar\omega_k}
    \right)
    + \textrm{const}
\end{align}
which is the free energy of an ideal gas of bosons with dispersion relation
$\omega_k=kc$ and two polarizations plus a constant.

The energy is then
\begin{align}
    \frac{E}{V} &=
    2 \int d^3 k \frac{\hbar ck}{e^{\beta \hbar c k} - 1}\\
    &=
    \frac{2}{\left( \beta\hbar c \right)^3 }
    \frac{1}{2\pi^2}
    \frac{1}{\beta}
    \underbrace{\int_{0}^{\infty}
    dx\,
    \frac{x^3}{e^{x} - 1}}_{%
    \pi^4/15
    }\\
    &=
    \frac{\pi^2 \left( k_B T \right)^4}{15 \hbar^3 c^3}
\end{align}
This is the \emph{Steffan-Boltzmann law}.

Also, to get the spectrum,
\begin{align}
    \frac{E}{V} &=
    2 \frac{1}{2\pi^2} \int_{0}^{\infty} dk\,
    \frac{k^2 \hbar ck}{e^{\beta\hbar ck} - 1}\\
    &=
    \frac{\hbar c}{\pi^2} \left( \frac{2\pi}{c} \right)^4
    \int_{0}^{\infty} d\nu\,
    \frac{\nu^3}{e^{\beta h\nu} - 1}\\
    &=
    \int_{0}^{\infty} d\nu\,
    \underbrace{%
    \frac{16\pi^2\hbar}{c^3}
    \frac{\nu^3}{e^{\beta h \nu} - 1}}_{%
    u(\nu)
    }
\end{align}
where
\begin{align}
    u(\nu) = 
    \frac{16\pi^2\hbar}{c^3}
    \frac{\nu^3}{e^{\beta h \nu} - 1}
\end{align}
is the energy per volume per frequency.
This is \emph{Planck's law}.
