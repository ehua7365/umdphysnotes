\section{Fermi Gas}
\begin{align}
    n(k) &=
    \frac{1}{e^{\beta\left( E_k - \mu \right)} + 1}
\end{align}
The energy is
\begin{align}
    E_k &=
    \frac{1\hbar^2 k^2}{2m}
\end{align}
Consider $T=0$.
If you plot $n(k)$ vs $k$,
then there is a Fermi momentum
\begin{align}
    k_F &=
    \sqrt{2m \mu}/\hbar
\end{align}
and there is a Fermi energy $E_F = \hbar k_F$.
There is a transition region near the Fermi energy.

For $T>0$ there is a transition region.
Near the transition region,
we consider where the exponential deviates significantly from 1.
\begin{align}
    \left|E_k - \mu\right| \approx 1
\end{align}
and then
\begin{align}
    \frac{\hbar^2 k^2}{2m}
    -
    \frac{\hbar^2 k_F^2}{2m}
    &=
    \frac{\hbar^2}{2m}
    \underbrace{
    \left( k + k_F \right)
    }_{\approx 2k_F}
    \left( k - k_F \right)\\
    &=
    \frac{\hbar^2}{m}k_F (k - k_F)
\end{align}
and then
\begin{align}
    \left( k - k_F \right)
    &\approx
    \frac{\mu}{k_F \hbar^2} k_B T\\
    &=
    \frac{1}{\hbar U_F} k_B T
\end{align}

The name of this sphere is the Fermi sphere.
The surface is the Fermi surface.

I'm going to apologize and do something trivial in front of you.
You're going to say it's boring,
but I promise you you'regoing to learn the rest of your life.
To leading order,
what you do when the temperature is exactly zero,
you should be able to do at any moment like that.

\subsection{$T=0$ limit}
I can find the density using an obvious formula for the number of particles.
\begin{align}
    N &=
    gV \int \frac{d^{3}k}{\left( 2\pi \right)^3}
    \frac{1}{e^{\beta\left( E_k - \mu \right)} + 1}
\end{align}
At zero temperature,
this approaches a step function.
\begin{align}
    n(k) \to \theta(k_F k _)
\end{align}
In that case,
the integral is going to be trivial.
\begin{align}
    N &=
    gV \frac{1}{8\pi^3}
    \int_{0}^{k_F} dk\, 4\pi k^2
\end{align}
in spherical integrals.
And then
\begin{align}
    N &= \frac{gV k_F^3}{6\pi^2}
\end{align}
and remember $k_F$ can be written in terms of the chemical potential $\mu$.
That's how you use the canonical ensemble,
you calculate $N$ in terms of $\mu$,
and then you invert it to find
\begin{align}
    k_F &= \left( \frac{6\pi^2 N}{g V} \right)^{1/3}
\end{align}
and now I can calculate anything I want from $k_F$.
To get the energy from the partition function is a bit of a pain,
but you don't have to,
you can just sum over the single-particle states,
but only on the occupied states
\begin{align}
    E &=
    gV \int^{k_F} \frac{d^3k}{\left( 2\pi \right)^3}
    \frac{\hbar^2 k^2}{2m}\\
    &=
    \frac{gV \hbar^2}{2m}
    \frac{1}{8\pi^3}
    \int_{0}^{k_F} dk\, k^2 k^2\\
    &=
    \frac{gV\hbar^2}{20\pi^2} \frac{k_F^5}{m}\\
    &=
    \frac{gV \hbar^2}{20\pi^2 m}
    \left( \frac{6\pi^2}{g} \frac{N}{V} \right)^{5/3}
\end{align}
You could do this in your head,
the fact it depends on $\prop N^{5/3}$,
if you think the right way it should be obvious.
If it were in 2 dimensions,
how would it change?
It would be $d^2k = 2\pi k\,dk$.
What if it had relativistic particles?
Just replace $E_k = ck$.

Now let's work out the pressure. 
You can do this yourself
\begin{align}
    P &=
    \left.\frac{\partial E}{\partial V}\right|_{T,N}\\
    &\sim
    \frac{1}{V^{5/3}}
\end{align}
so the pressure gets larger and larger as you compress.
But where does the pressure come from?
It's at zero temperature.
They are not moving because of thermal energy.
It's because of the exclusion principle.
There is  pressure exerted on thew alls.
The hihger you are in energy,
the fastree they are colliiding,
and the higher the pressure.
hhere you hve a bunch of them,
and you're taking the average.

This has a name,
and it's good for you to know,
it's called the
\emph{degeneracy pressure}.
But the way,
the physics at low temperature of a Fermi gas
is called a degenerate gas.

The story is the following.
Stars are a big chunk of mass,
that graviatete and collapse,
but they don't.
Some nuclear rections happen,
which produce heat,
which produces a pressure outwards.
oSo there is a compettition between hot bas outwadrs and gvaitational pressurei
nwards.
All that can be calucalted,
not tht iddifficult.

The problem is that the nuclear reactions require fusion of elments.
At ome point you fuse everything you can,
you make iron,
which gives no more energy.
So more more heatis produced,
but it does produce radiaiton,
but that doesn't work.
Then therea re two possible rsults.
If the star is very heavy,
it collpases fast into a black hole,
nothign to stop it.
Bloack holes are boring.
If the star is a little little,
or veyr light like the sun,
what happens is this matter collapse,
part of it spreads all over space as asupernova.
But what's left is cold,
and bcause of the electrons,
it doesn't collapse.
It's held up by the degeneracy pressure,
and they're called white dwards.

You know why they are called wirte dwarfees?
They are masses of planets.

Everything you need to know to calculte it you know alaredy.
So you should be ready to caluclate what is the maximum mass that can be
sustained.

White dwarves boring.
Black holes boring.
There is an intermediate neutron star.
c

This was Chandrasakar.
Nobody undertands how he learnt th is stuff.
TO go from India to Cambridge,
you sen  few months on a ship,
and during the trip,
he figured this out.
He was a smart cookie.

To be at the forefront of physics with a creative idea,
surrounded by nobody.

I met the guy but it wasn't a great story though.
Very respectable guy.
There was a serration with friends.
It was closed.
This guy stands up an starts talking about black holes,
and at the end,
Chandrasekhar stands up and starts ranting against the guy,
says total nonsense.
The young guy was paralysed,
knew who he was and just looked at him.
At some point Chandrasekhar sits down,
and a group of people take care of him,
after that the whole audience was weird.
By the way,
you talk about black holes.
One story leads to another,
I'm sorry.

I have another friend,
and distinguished scientist but not like Chandrasekhar.
I was watching a seminar next to him.
Very senior person.c
In the end,
he said if you ever give a talk like that,
please shoot me.
I was wondering if I should talk about this one.

Chandrasekhar stood up for 10 minutes ranting,
it was really awkward.
I could go on, but no.

Hans Bethe.
You know he discovered the sun is powered by nuclear power.
I met himw hen he was almost 90.
One person stood up ni a talk,
and there was a thing like this table,
moving one inch at a time.
I said oh god,
then he puts a tranpsarenct slide upside own,
andhe asks where is it.
The guy focuses on hte tanpsanet slide,
flipped it,
and proceeds to give the most beautiful tlak I've ever seen.
No omralism,
one idea after another.
Goes from the early tdays to the recent stuff.
Terhee was disussion aout whether they foudn a black hole or not.
The guy wwas top of his game at 80 or 90,and I was like darn what about me?
But there's not going to be good talks again.

I'll give you some bullets.
Let's finish the class ealry for you to study.
