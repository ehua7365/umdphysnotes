\section{Recap}
The reservoir is not going to get colder.
It's a technique used in arts and sciences.
You want to heat something up,
you have a pan,
fill it with water,
put a smaller pan inside the water.
What's called called?
Double boiling?
I don't know.

The temperature is very constant,
so you don't burn your food.

Same thing here,
this thing stays at a very constant temperature.

What I was trying to do was the following.

I'm going to apply the ergodic principle.
All microscopic states are equally probable,
assuming fixed temperature.
If not then it's crazy.


The probability density is $ \rho_T (q, p, Q, P)$
where $q,p$ refers to the system and $Q,P$ referred to the reservoir.
This is not one variable,
but an enormous number of variables,
all the particles in the system.
I realise this is a better notation.
\begin{align}
    \rho_T (q, p, Q, P) &\sim
    \delta_\Delta(E_t - H_T(q,p,Q,P))
\end{align}
where the $\delta_\Delta$ is a finite-width delta function.

Now what is the probability of finding my subsystem here in a particular
microscopic state regardless of the reservoir state.
Well I just have to integrate over all coordinates of the reservoir
\begin{align}
    \rho(q, p) &=
    \int dQ\, dP\,
    \rho_T(q, p, Q, P)\\
    &=
    \int dQ\, dP\,
    \delta_\Delta\left( E_T - H_T(q, p, Q, P) \right)
\end{align}
The bulk is bigger than the surface for big systems,
because the bulk is cubed,
but the surface is squared,
so it is negligible,
so write the Hamiltonian as
\begin{align}
    H(q, p) - H_R(Q,P)
\end{align}
\begin{align}
    \rho(q, p) &=
    \int dQ\, dP\,
    \rho_T(q, p, Q, P)\\
    &=
    \int dQ\, dP\,
    \delta_\Delta\left( E_T - H(q, p) - H_R(Q, P)\right)\\
    &= \Gamma_R \left( E_T - H(q, p) \right)
\end{align}
where $\Gamma_R$ is the number of microstates in the reservoir in the
microcanonical ensemble.
\begin{align}
    \int dQ\, dP\, \delta_\Delta\left( E_T - H - H_R \right)
    &= \Gamma_R \left( E_T - H(q, p) \right)
\end{align}
In general,
you have to sum the microstates of the system minus the Hamiltonian of the
system.
You want to get the energy minus the energy of the subsystem.

Because now,
this is going to be related to the entropy of the system.
The number of microscopic states is the exponential of the entropy of the
reservoir.

\begin{align}
    \rho(q, p) &=
    \exp\left[ \frac{S_R}{k_B} \left( E_T - H(q, p) \right) \right]
\end{align}
and remember that $H$ is the energy of the subsystem,
much smaller than $E_T$ the energy of the huge reservoir,
so I can expand this as a Taylor series.
\begin{align}
    \rho(q, p) &=
    \exp\left\{ 
    \frac{1}{k_B}\left[ 
    S_R(E_T) - H(q, p) \frac{\partial S_R(E_T)}{\partial E_T}
    + \cdots
    \right]
    \right\}
\end{align}
Now note
\begin{align}
    \frac{\partial S_R(E_T)}{\partial E_T} = \frac{1}{T_R}
\end{align}
by definition,
and this energy is defined in the microcanonical ensemble.
When you put two systems together they exchange energy,
and because of that you get this relation.
Then we are looking for a function of $q$ and $p$,
and that dependence is in $H(q,p)$,
and everything else is just a big fat constant.
What I find is
\begin{align}
    \rho(q, p) &\sim
    \exp\left[ -\frac{H(q, p)}{k_B T} \right]
\end{align}
where there are big overall constants I don't care about.
So when I look ta the microcanonical ensemble of the subsystem,
it is a canonical ensemble.
I just take that function and normalize it,
then I get the canonical ensemble.
The whole system is in a microcanonical ensemble,
then by the ergodic hypothesis,
the subsystem is in a canonical ensemble.

Are they completely isolated?
Not really.
In the thermodynamic limit,
you get the same answer anyway.
You can take htis as a motivation for cnaoniacal ensemble.

It's motivation,
but you get the same answer anyway.
This ensemble and that ensemble are idfferent.
They look completely different but they are not.
The reason is thata of states with large energy,
and there are few states with low energies.
Only microscopic states iwth that energy actually matter.
It's usually easier to do calculatioss over all energies over all states
than to do integrals iwth this constraint.
The $e^{-H/k_BT}$ is better.

It is true that in some large systems,
you are couled to a large reservoir.

A chemical reaction in abucket.
The room is a thermal reservoir.
Everything is going to go back to 72 degrees,
because the room is big and the air conditioner is going to turn on.

The justitication for whole of statical mechanics fromes from the microcanonical
ensemble and the ergodic hypotheiss.
Most mportantly,
tHe canonical and microcanonical give the same answer anywa.

I want to continue the argument
with an important comment.
What we showed with  this,
ifwe look at the parittion function of the canonical ensemble,
I could write it as as sum over the enrgy,
the numbero f states of the energy
\begin{align}
    Z &=
    \int dE\,
    e^{\frac{S(E)}{k_B T} - \frac{E}{k_B T}}
\end{align}
and I say that
\begin{align}
    \Gamma(E) &= e^{\frac{S(E)}{k_B T}}
\end{align}
is a constant
and hte only quantity that matters is
the area around $E^*$,
because this $\Gamma(E)$ is strongly peaked around there,
so the integral becomes approximately
\begin{align}
    Z &=
    \int dE\,
    e^{\frac{S(E)}{k_B T} - \frac{E}{k_B T}}
    \approx
    \left. e^{-\beta\left( E - TS \right)}\right|_{E=E^*}
\end{align}
It is the quantity that minimizes $E-TS$.
What is it that minimizes $E-TS$?
Just find
\begin{align}
    0 &=
    \left.\frac{d}{dE}\left( E - TS(E) \right)\right|_{E=E^*}\\
    &=
    \left.
    1 - T \frac{\partial S}{\partial E} \right|_{E=E^*}
\end{align}
and you have seen this before.
Choose the value of the derivative that gives $1/T$.
Do you recognize the Legendre transform here?
This is a common mathematical construction.
It has a geometric meaning.
It was presented to you without motivation,
but I will tell you.

Take a mental picture of this,
and I'm going to go back,
and tell you,
see it's the same thing.
So take a picture of this.
This piece of mathematics
I'm not going to explain why,
actually comes from studies from the Renaissance
from presecive drawing.

Distant things dhould be smaller.
Lines that are parallel in the real world seem to converge to a point.
Completely obvious to us,
but not obvious 500 years ago.
Google Byzantine art,
they have an opposite idea.
Close things are big,
but far away things are big.
You look and it looks naturla,
but it doens't loko realisatic,
buthten you realise it's all wrong,
it's hte other way around.
Then people started figuring out the mathematics of persepctive,
and 200 years later they come up with something that has nothing to do with
persepcive.
I'm going to tell you taht.


Suppose we have af unction $f(x)$.
I need to specify what value of $f$ for each value of $x$.
That's what you learnt in 5th grade.
So the height is the $f(x)$ for each value of $x$ on the horziontal axis,
like sliced bread.

But there's another way.
Look at the slope of this function,
look at this tangent.
What if I want to specify this function
not in terms of $x$,
but in terms of the slope $\frac{df}{dx}$.
Every point has a unique slope,
so maybe I can specify the function in terms of the slope in
terms of the $x$ coordainte.
So If I defined the slope to be
\begin{align}
    p &= \frac{df(x)}{dx}
\end{align}
then what is $f(p)$ supposed to be?

For example,
suppose $f(x)=x^2$,
then $p=2x$ which implies $x=p/2$ and so
then $f(p) = (p/2)^2$.

So the question is,
how high does it have to be to get a given tangent.

The same slope same height,
draw everything to the right,
and say that's my curve.
If I specify my function in terms of the slope,
the curve is not unique
if I shift myfunction to the right.

It does matter,
because there is a constant
that is a function of the other variables,
and it's not a constant anymore.
Suppose you have $f(x - c)$ instead,
then you have many possible heights.
The non-uniqueness in enormous.

So someone invented something smarter than that.

By the way,
you see perspective here,
drawing tangents on curves?
I dno't.
It's historical.

Instead of specifying theheight as a function so the slope.
Instead,
give me a slope,
and I will give you where the tangent hits the vertical axis.
I'm not giving you the $f(x)$ hieght,
but I'm giving you where the tangent intersects the $y$ axis.
And so for a given tangent,
I'm going to give you the $y$ intercept of the tangent.

So let's draw a picture.
Let $p=dp/dx$ be the slope.
Then let the $y$-axis intersection of the tangent be $\tilde{f}(p)$.
Just by looking at this picture,
I can tell you the slope is going to be figured out by looking at
$f - \tilde{f}$.
Take the ratio  of similar triangles
\begin{align}
    p &= \frac{f - \tilde{f}}{x}
\end{align}
If that's true,
then I can isolate $\tilde{f}$ as a function of $p$.
And I can get
\begin{align}
    \tilde{f}(p) &=
    \left .f(x) - px\right|_{x\text{ such that } \frac{df}{dx}=p}
\end{align}
Now of course,
this condition is the same as saying that the
$x$ you're using is minimal.
But this is the same condition you get if you took $f(x) - px$
and minimize it in relation to $x$.
So it's the same thing.

The lesson is this.
I can give all the information about this curvei n two wasy.
I eigher specify the height $f(x)$ for every point $x$.
Or I specify this intercept point $\tilde{f}(p)$ for every possible slope $p$.
Now there is no freedom to move this around,
because the point specified is now unique.

That's the geometry I wanted you to know.

So either give me the height for every point,
or give me the intercept of the tangent for every slope,
and they contain the same information.

So let's do an example.
Suppose
\begin{align}
    f(x) &= x^2 + c
\end{align}
Then
\begin{align}
    p = \frac{df}{dx} = 2x
\end{align}
which means $x=p/2$ and so
\begin{align}
    \tilde{f}(p) &=
    \left. f - px\right|_{\frac{df}{dx}=p}\\
    &=
    \left.
    x^2 + a - pa
    \right_{x=p/2}\\
    &=
    \frac{p^2}{4} + a - \frac{p^2}{2}\\
    &= -\frac{p^2}{4} + a
\end{align}
If you have nothing better to do,
you can try to draw these tangents and points.
But,
we're not going to have to think about this geometry anymore.

Now what does this have to do with thermodynamics.
So remember that the partition function was
\begin{align}
    Z &=
    \left.
    \exp\left[ -\beta\left( E - TS \right) \right]
    \right|_{E = E^*}\\
    &=
    \left.
    \exp\left[ -\beta\left( E - TS \right) \right]
    \right|_{\left.\frac{\partial S}{\partial E}\right|_{E=E^*} =
    \frac{1}{T}}
\end{align}
So I want to write $Z$ as a function of $T=\frac{\partial E}{\partial S}$,
but I lose information,
so instead of playing with $E$,
I play with $E-TS$.
This allows me to get rid of the $S$,
in the same way the Legendre transform $\tilde{f} = f - px$
allows me to get rid of the $x$.
And htis has a name,
i'ts the Helmholtz free energy.
\begin{align}
    Z \sim
    e^{-\beta F(T, U, N)}
\end{align}

I want to talk about 3 things:
Why is this an energy?
What's free about it?
And some examples.

Why is it free? There is an explaatintio sort of I'll explain later.

Who is Helmholtz?
A physicist of course.
But actually he wasn not,
he was a professor of medicine.
Helmholtz free energy,
Helmholtz coil,
but that's not what he's famous for.
His main thing was to understand how to understand the ear works.
Psychoacoustics.
that's hwy Helmholtz appears in PDEs,
solving the wave eqution.
How does that interact iwth the ears.
His famous book is the sound of tones.
I learnt a huge amount of this as an amateur musician,
and I could tlka and talk.

The story is this:
There's a trandition in the academic route,
more theroetical than expeirmenta,
more in math than physics.
Your advisor is an important person in your life,
more important than your mother, your spouse.
It's old-fashinoed,
so there's a tradition of following your academic advisor.

There's a websitte that keeps track of those things.
One interesting happens,
those things converge on a few peple.
The world of physics was smaller 100 years ago.
The rason is because,
there are people productive in terms of students,
and others who are not.
We all had a common mother,
but same thing hapens in phycis.

My advisor,
was a student of some guy yo don't know,
who was studyent of some guy yo udont't kno,w
a student of some guy you don't know,
who was a sutdent of some guy you don't know,
who was a student of Helmholtz!

There are few schools fof physics.
In the US,
everyt student was a student of Schwinger.
Oppenheimer was not,
but he shared with Schwinger.

Sommerfeld had good students.
Heisenberg, Pauli,
and those had many others.
So Sommerfield.

Einstine has 0 zeros.
There was one guy who claimed to be,
but he was an epximerntalist and he signed his papapers.

At some point we're in the middle ages,
the only people who are scientsits are people who read and write,
and they are from the Church.
So basically it's who was the head of hte Monastary.
The Church kept records,
and then you can find themenotor of that person.
There's no PhD,
but there were records.
And then they go down to Jesus.
So I'm a direct line from Jesus.
Does that mean I should take more stdents because of that?

But Jesus was baptized by John the Baptist,
so that's his advisor.

A new continent has been discovered.
They want ot colonie the place.
They say you hvae land,
you can start your own farm.
Howm uch land?

I'm going to give you some amount of fence.
A mile.
Whatever land you can enclose in the fence,
you have it for life.
People go there and the question is what shape is best?
You're a farmer and you want as much land as possible.
But the perimeter is fixed.
What's the shape?
I'ts a circle.
Can anyone prove that?
I'm not goin to prove it but its' corect.


Then there's ano hte continent.
Fencing is very expensive.
The government fixes a certain amount of area and land they can claim.
They can surround byfence.
So for a fixed area they want otm imize the amount of fencing.
A circle too!

So if you arrive ina contentn and see all the farms are circles,
the govenremnt must hvae given them a fixed amount of frncnig each.

We play this game where we play the system,
and you have walls.
When you can change the walls,
and it's isolated from everything else.
The total energy is fixed.
$E = E_1 + E_2 + \cdots$.

But to find the final state.
The entropy is
$S=S_1+S_2+\cdots$.

How can I divide energy among the systems to maximise the energy?
There's ano hte quantit y fixed,
that's the total volume
$V=V_1+V_2+\cdots$.

We want to do this maxmiiation over a certain substate of all macroscopic
states.
The walls can move,
but I cannot change the sum of them.
I maximize the entropy under some constraints.
The constratins are the total energy $E$,
the total vomelume $V$, the totla number $N$.

Circles are the ones with largest area for a fixedp erimeter.

But hter'es naother way to characerise the same state.
You can say circles are the minimial perimeter for a given area.
For us,
you minimize the energy $E=E_1+E_2+\cdots$
assuming the energy is fixed,
that is $S,V,N$ are fixed.

After the final state,
it doesn't matter how you got there.
The same satate is characterized by thet minmimal energy
f o a fixed tempreatuer.
Even though in pracitse,
it's not mimimze the enrgy and fix the tempratuere,
theres no way ot fix the entorpy in the real world,
but it doesn't matter because mathematically it's the same thing.c
This requires a proof,
and the proof is done by a very good drawing,
because that's the best way to prove things.

You want a good drawing?
Look at my lecture notes.
It sucks to draw good on the board.

Think.

I'm going to have an axis $S$.
The energy starts out as $E_1,E_2,\ldots$,
then it's going to become
$E_1',E_2',\ldots$.
So let's have another axis $E_1'$.
I plot $S$ as a function of $E_1'$.
It looks like a curve with some maximum.
Then I have another axis,
the total energy $E$.
Suppose I did the same problem,
but with a total energy that was larger.
My claim is that the drawing would look like the following.
I have a surface $S(E,E_1')$.
Do you sense a 3D thing here?
Does that help?
Pretty good.

[picture]

Now,
the entropy goes as a the total energy gorws.
That's a feature.
The more energy,
the more microstates.

What this principle says is to ask what are the surfaces of fixed energy.

I take a surface of fixed energy $E$
and I draw this plane.
The intersection.

If you fix the energy $E$,
and look for the maximum $S$.

Now I'm going to do an identical drawing.

Now I'm going to think about this other principle here.
Constant total energy means this plane here.

Now I look for a macroscopic state with fixed entropy.
that minimizes the energy.

I screwed up the drawing,
but they should be the same point.

That's my picture proof of this principle.

This principle is called the maximum entropy principle,
which follows from the ergodic hypothesis.
This macroscopic state corresponds to the maximum number of microscopic states.

This one here is called the minimum energy principle,
and is justified because the maximum entropy principle.
I have never used the minimum energy principle,
but I am only using it as a steping stone to prove something else.


Consider the Helmholtz free energy of two systems
\begin{align}
    F = F_1 + F_2
\end{align}
and the problem is to minmize $F=F_1+F_2$ subject
to the constraint of fixed $T,V,N$.
THe idea is this.

You have some subsystem immersed in a thermal reservoir of temperature
$T$.

Think of this as two subsystems $A$ and $B$.
Then we maximize the total energy.
Altenratively,
consdier entropy fixed and minimize the energy.

The energy is fixed,
it can vary,
but states vary and you fix the entropy.
THep roblem is that you have to look at 3 different systems,
and if one if the systmes is a thermal reservoir,
it's a special system that doesn't change temperature and that simplifies
things.
So rather than apply maximal energy,
you just minimal free energy,
and I'm going toshow tat.

We want to minimze free eneergy
\begin{align}
    F &= E - TS
\end{align}
for fixed temperature $T$.
Let's start assuming that minimal energy is correct.
We want to minimize $E_T$ at fixed $S$.
What's the chnage in the energy of the whole thing?

The change in total energy is $E_T=E+E_R$,
so
\begin{align}
    dE_T &= dE + dE_R
\end{align}
and I want to set $dE_R$.
The energy doesn't change because it does work,
it doesn't change because particles corss the wall,
it changes only because heat energy isexchanged.
How is that related to the changing entropy?
Remember that
\begin{align}
    \left.\frac{dS}{dE}\right|_{V,N} &= \frac{1}{T}
\end{align}
and rememver $dE_R = T_R \,dS_R$.

The reservoir gains some energy when this looses some entroy,
so
\begin{align}
    T_R dS &= - T_R dS
\end{align}
so we get
\begin{align}
    0 &= dE - TdS = d(E - TS) = dF
\end{align}
Bottom line,
fix the energy,
find the maximmum entropy.
Altenratively,
fix the entropy
and minmiaze the energy.

But you can forget about he reservoir completely.
Then you only need to minimize the free energy.
Forget the reservoir.
Then that gives you the state.

\begin{question}
    The assumption is,
    the bath is holding the subsystme at a fixed tempeature.
\end{question}
Yes, they're in thermal equilibrim by definition.

At this point,
you might be confused by many little observations made here and there.
But that's okay.
I'm going to bring everything together.
The goal is to organize everything you learn about physics.

I actually did write this on my lecture notes.
It's not just a colletion of formluae to mmmorize.
It's the logical relation between them that mattersl
I'm going to write how you do things in the microcanonical ensemble.
And on this other board,
how to do things in the canonicla ensembler.

\subsection{Microcanonical Ensemble}
The probability of getting a particular state is
\begin{align}
    \rho(q, p) &=
    \frac{\delta_\Delta\left( E - H(q, p) \right)}{%
        \int \frac{d^Nq\, d^N}{h^N} \delta_\Delta\left( E - H(q, p) \right)
    }
\end{align}
The $h$ constant does not matter.
In the classical limit,
this gets left out here if I compute things.
But if you don't write $h$,
like when they first invented statistical mechanics,
they still got everything right.
And this denomianotr is the number o states,
which we call
\begin{align}
    \Gamma(E,V,N) &=
    \int \frac{d^Nq\, d^N}{h^N} \delta_\Delta\left( E - H(q, p) \right)
\end{align}
Then the average of an observable is a weighted average
\begin{align}
    \overline{\mathcal{O}}
    &=
    \int \frac{d^Nq \, d^Np}{h^N}
    \rho(q, p) \mathcal{O}(q, p)
\end{align}
and that's in principle all you need.
But there are ways around,
which is thermodynamics.
The entropy is
\begin{align}
    S(E,V,N) &=
    k_B \ln \Gamma(E, V, N)
\end{align}
If yo uwant to find s attae,
you should always maximize $S$ at fixed $E,V,N$.
This confused the hell out of me as a student.
I was stupid.
How can I maximize $S$?
The total $E,V,N$ cannot change,
but the distribution of $E,V,N$ can change!

Finally, a very good thing to remember is that
\begin{align}
    \left.\frac{\partial S}{\partial E}\right|_{V,N} &= \frac{1}{T}
\end{align}
has to be the same for any two systems in equilibrium.
And another thing equal for systems in equilibirum is this quantity
\begin{align}
    \left.\frac{\partial S}{\partial V}\right|_{E,N} &= \frac{P}{T}
\end{align}
and another hting is also the chemical potneital
\begin{align}
    \left.\frac{\partial S}{\partial N}\right|_{E,N} &=
    -\frac{\mu}{T}
\end{align}
We rarely do things in the microcanonical ensemble because yo uhave to compute
these crazy integrals with constraints.


\subsection{Canonical ensemble}
This is easier to work with.
The partition function is
\begin{align}
    Z(T,V,N) &=
    \int \frac{d^Nq\, d^Np}{h^N} e^{-\beta H(q,p)}
\end{align}
and the probability distribution is
\begin{align}
    \rho(q, p) &=
    \frac{e^{-\beta H(q, p)}}{%
    \int \frac{d^Nq\, d^Np}{h^N} e^{-\beta H(q,p)}
    }
    =
    \frac{e^{-\beta H(q, p)}}{%
    Z
    }
\end{align}
and the average of an observable is likewise
\begin{align}
    \overline{\mathcal{O}}
    &=
    \int \frac{d^Nq \, d^Np}{h^N}
    \rho(q, p) \mathcal{O}(q, p)
\end{align}
And then I consider the free energy which is the Legendre transform of the
energy
\begin{align}
    F(T,V,N) &= E - TS = -k_B T \ln Z(T, V, N).
\end{align}
It contains the same information as the entropy.

You could compute the $\rho$,
but it's easier to compute the denominator $Z(T,V,N)$,
then I compute the free energy from $Z$.

One thing is that
\begin{align}
    \left. \frac{\partial F}{\partial T}\right|_{V,N} &= -S\\
    \left. \frac{\partial F}{\partial U}\right|_{T,N} &= -P\\
    \left. \frac{\partial S}{\partial N}\right|_{T,V} &= N\\
\end{align}

Note
\begin{align}
    \left.\frac{\partial E}{\partial S}\right|_{V,N}
    =
    \left.\frac{\partial E}{\partial U}\right|_{S,N}
    = -P
\end{align}
and
\begin{align}
    \left.\frac{\partial E}{\partial N}\right|_{S,V} &= \mu
\end{align}
The sign comes from the fact that if you compress a system,
you increase the energy of a system.

Also,
you can take other kinds of Legrendre transforms with respect to other
variables,
enthalpy,
only chemists use it,
it's bad.
There's Gibb's free energy.
And also with respect to $N$,
you get the grand canonical potential.

So there's a family of thermodynamic potentials.
I'm not going to work out the details,
but it's very straightforward.
I'll have a different ensemble.

The point is not to memorize the relations,
the point is to remember what's varying with what.

\begin{question}
    Is there a quantum version of $E+PV$?
\end{question}
Think of $v$ as the size of the box,
so it's just a parameter in the Hamltonian.
But it's not an operator.
Imagine you have a system,
therei s a quanutm satte where particles are spread
and another one concentrated here.
There will be an operator that measures the volume distribution though.

\subsection{Quantum Version}
The density matrix,
now an opeator is
\begin{align}
    \hat{\rho} &=
    \frac{e^{-\beta \hat{H}}}{\Tr e^{-\beta\hat{H}}}
\end{align}
where the bottom is the partition function
\begin{align}
    Z(T,V,N) &= \Tr e^{-\beta\hat{H}}
\end{align}
The moment you hi thermodynamics,
those macrosconpic quantites are the same for classical and quantum.
That's weird.
When relativity awas discovered,
clasical physics had to change.
When QM was discovered,
classical physics, EM had ot change.
Butwhen stat mech was dicovered,
nothing had to change.
It's true in this universe,
true in proably many other universes we cna imagine.

There's antoher point of view.
Thermodynamics has to be true even in different worlds.

Cna I write the equivalent quantum mechnical formula for these other equations?
Yes, but it's ugly.
\begin{align}
    \bar{\mathcal{O}} &=
    \Tr(\hat{\rho}\hat{\mathcal{O}})
\end{align}
and
\begin{align}
    \hat{\rho} &=
    \frac{\displaystyle \sum_{n:E<E_n<E+\Delta} \ket{n}\bra{n}}{%
    \displaystyle\sum_{n: E<E_n<E+\Delta} 1}.
\end{align}
Not nice to compute.
Do not waste a microsecond thinking about this.
