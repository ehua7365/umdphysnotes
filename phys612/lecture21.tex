\section{Diatomic gas}
The heat capacity is
\begin{align}
    c_v &= (6N + N)\frac{1}{2}k_B = \frac{7}{2}k_B N
\end{align}
Harmonic oscillator.
For $k_B T \gg \hbar\omega$, $\bar{E} = k_B T$.

Back to solids.
Debye model.
\begin{itemize}
    \item Normal modes have $\omega_k=v k$ for all 3 modes.
        2 transverse and 1 longitudinal.
    \item Brilluoin zone is a sphere
        \begin{align}
            3N = 3V \int^{k_D} \frac{d^3 k}{(2\pi)^3}
            =
            \frac{V}{2\pi^2} k_D^3
        \end{align}
        i.e. number of normal modes $3N$ equals number of fourier modes.
\end{itemize}
We used that
\begin{align}
    \sum_{\vec{k}} f(\vec{k}) \approx
    V \int \frac{d^3k}{(2\pi)^3} f(k)
\end{align}
If you don't know this,
we'll discuss it in detail soon.

The partition function is
\begin{align}
    Z &= \Tr e^{-\beta H}\\
    &=
    \prod_{k,\alpha}\sum_{n_k=0}^{\infty} e^{-\beta\hbar\omega\left( n_k +
    \frac{1}{2}
    \right)}
\end{align}
where each term is partition function for one oscillator for one oscillator of
frequency $\omega_k$.
Here $\alpha = 1,2,3$
are the polarizations.
So
\begin{align}
    Z &=
    \prod_{k,\alpha}
    \frac{e^{-\beta\hbar\omega/2}}{1 - e^{-\beta\hbar\omega_k}}
\end{align}
The free energy is
\begin{align}
    F &= -k_B T\ln Z\\
    &=
    k_B T
    \sum_{\vec{k},\alpha}
    \left[
    \ln\left( 1 - e^{-\beta\hbar\omega_k} \right)
    +
    \underbrace{\frac{\beta\hbar\omega_k}{2}}_{\textrm{negelect zero-point}}
    \right]\\
    &\approx
    3 k_B T V \int^{k_D} \frac{d^3k}{{(2\pi)}^3}
    \ln\left( 1 - e^{-\beta\hbar vk} \right)
\end{align}
We neglect the zero-point energy because it's independent of $T$.
The factor of 3 comes from the 3 polarizations.

The entropy is then
\begin{align}
    S &= \left.\frac{-\partial F}{\partial T}\right|_{V}\\
    &=
    -3k_B V \int^{k_D} d^3k\, \ln(\cdots)
    +
    \frac{3k_B T}{k_B T^2} V
    \int^{k_D} d^3k
    \frac{\hbar v k e^{-\beta\hbar v k}}{1 - e^{-\beta\hbar v k}}
\end{align}

And finally the energy is
\begin{align}
    E &= F + TS\\
    &=
    3V \int^{k_D} \frac{d^3k}{(2\pi)^3}
    \frac{\hbar vk}{e^{\beta\hbar vk} - 1}.
\end{align}
Then the heat capacity is
\begin{align}
    C_v &=
    \left.\frac{1}{N} \frac{\partial E}{\partial T}\right|_{V,N}\\
    &=
    \frac{1}{N} V
    \frac{\partial}{\partial T}
    \int_{0}^{k_D} \frac{d^3k}{(2\pi)^3}
    \frac{3\hbar\omega_k}{e^{\beta\hbar\omega_k} - 1}\\
    &=
    \frac{3V}{N} \frac{1}{k_B T^2}
    \int_{0}^{k_D} \frac{d^3k}{(2\pi)^3}
    \frac{(\hbar\omega_k)^2}{e^{\beta\hbar\omega_k} - 1}
    e^{\beta\hbar\omega_k}\\
    &=
    \frac{3V}{N} \frac{1}{2\pi^2}
    \frac{(\hbar v)^2}{k_B T^2}
    \left( \frac{k_B T}{\hbar v} \right)^5
    \int_{0}^{\beta\hbar vk} dx\,
    \frac{x^4}{\left( e^x - 1 \right)^2} e^x
\end{align}
which for $k_B T \ll \hbar v k_D$
gives
\begin{align}
    C_V &\approx
    3 \frac{V}{N}
    \left( \frac{k_B T^}{\hbar v} \right)^3
    \frac{1}{2\pi^2}
    \int_{0}^{\infty} dx\,
    \underbrace{\frac{x^4 e^{x}}{\left( e^x - 1 \right)^2}}_{4\pi^4/15}\\
    &=
    \frac{V}{N}
    k_B
    \frac{2\pi^2}{5}
    \left( \frac{k_B T}{\hbar v} \right)^3\\
    &\sim T^3
\end{align}
which is the $C_v \sim T^3$ law for low temperatures.
