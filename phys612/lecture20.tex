\section{Quadratic Hamiltonians and the Equipartition Function}
\begin{align}
    H &= \sum_{i=1}^{N}\left( 
    \frac{p_i^2}{2m_i} + \frac{m_i\omega_i^2}{2}x_1^2
    \right)
\end{align}
The partition function is
\begin{align}
    Z &= \int_{-\infty}^{\infty} \prod_{i=1}^{N} \frac{dx_i\, dp_i}{h^N}
    e^{-\beta\sum_{i}\left( 
    \frac{p_i^2}{2m_i}
    + \frac{m_i\omega_i^2}{2} x_i^2
    \right)}\\
    &=
    \prod_{i=1}^{N}\frac{1}{h}\int_{-\infty}^{\infty} e^{%
    -\frac{\beta m_i\omega_i^2}{2}x_i^2
    }
    \int_{-\infty}^{\infty} dp_i\,
    e^{-\beta \frac{p_i^2}{2m_i}}\\
    &= {\left(\frac{k_B T}{\hbar}\right)}^{N} \frac{1}{\prod_{i=1}^{N}\omega_i}
\end{align}
Then the free energy is
\begin{align}
    F &= -k_B T \ln Z \\
    &=
    -N k_B T \ln\left( \frac{k_B T}{2} \right)
    +
    k_B T \sum_{i=1}^{N} \ln\omega_i
\end{align}
The energy is
\begin{align}
    E &= F + TS
\end{align}
where
\begin{align}
    S &= \frac{\partial F}{\partial T}\\
    &= Nk_B \ln\left( \frac{k_B T}{\hbar}  \right)
    + \frac{N k_B T}{T}
    - k_B \sum_{i=1}^{N}\ln \omega_i
\end{align}
and so the energy is
\begin{align}
    E &= Nk_B T =
    \left(
    \frac{k_B T}{2}
    + \frac{k_B T}{2}
    \right) N
\end{align}
Textbooks are confusing and students fall for it.
I think a 1D harmonic oscillator has 1 degree of freedom.
I think a 1D free particle also has 1 degree of freedom.
Forget degrees of freedom,
what matters is the number of quadratic terms in the Hamiltonian.

Each quadratic term in the Hamiltonian has energy $\frac{1}{2}k_B T$.

\begin{example}
    1D array of masses connected by springs with periodic boundary conditions.
\end{example}
The Hamiltonian is
\begin{align}
    H &=
    \sum_{i=1}^{N}\left[ 
    \frac{p_i^2}{2m}
    + \frac{k}{2} {\left( x_i - x_{i-1} \right)}^2
    \right]
\end{align}
So this is quadratic but a slightly more complicated quadratic.
You see there are cross-terms like $-2x_2x_3 + \cdots$ in there.
You have to account for boundary conditions too.
You have to say that $x_N = x_0$,
including the one that wraps around.
There are 3 terms and they are all there.

What you learn in Chacko's class is that you can do a change of variables that's
going to make this quadratic Hamiltonian that is decoupled.
This is called finding the normal modes.
It's something you're going to relearn in Chacko's class.

I'm going to state here that you can do a change in variables.
That's going to be equivalent to this
\begin{align}
    H &=
    \sum_{i=0}^{N-1}\left[ 
    \frac{p_i^2}{2m}
    + \frac{m\omega_1^2}{2} q_i^2
    \right]
\end{align}
Basically the $q_i$s are Fourier transforms of the $x_i$'s.
You will find that the magic works and the $q$'s are independent of one another.

If there s one canonical transformation to learn,
it's this one.
The range of applicability of this calculation is huge.
Basically physics is to do more and more complicated cases of harmonic
oscillators.

Now I can complete the partition function of these guys,
taking both terms and just writing the answer.

What is the average energy of this guy of temperature $T$?

It's $\frac{1}{2}k_B T$ for every quadratic term.
There are $2N$ terms.
This is going to be again
\begin{align}
    E &= \frac{1}{2}k_B T \times 2 \times N = Nk_BT
\end{align}
All you needed to know how many normal modes.
You didn't even need to find the frequency.
In physics 101,
they shake a slinky.
There are transverse modes,
longitudinal modes,
and more complicated modes I can't do with my hands.
Each one has different frequencies.
Some long wavelength ones move fast or slow,
but it doesn't matter for the average energy $E$,
it's just $\frac{1}{2}k_BT$ for every quadratic term in the Hamiltonian.

We can now state the equipartition theorem.

\begin{question}
    How do you know how many terms?
\end{question}
If you start with $N$ coordinates,
you get $N$ normal modes no matter how complicated.

\begin{theorem}[Equipartition]
    The thermal average energy of a Hamiltonian $H$ is
    \begin{align}
        E &= \frac{1}{2} k_B T N
    \end{align}
    where $N$ is hte numbero f quadratic terms in $H$.
\end{theorem}
This is true for classical systems.

Let's consider a few systems now for examples.

\begin{example}
    Imagine you have a solid insulator.
    what is the average enregy of an insulating solid?
\end{example}
Please give an example of a solid.
You give 5 things that re not solids with crystal structures.
Glass, styrofoam, plastic.
They're all rigid,
but they're not solids.
The reason I'm not talking about insulators,
is because they have electrons that move around adn that changes the energy.
I'm thinking of an insulator solid,
just a bunch of molecules in a lattice that shake.

The modules have an equilibrium position,
andthey shake around in a complicated way.
To first approximation,
they have some quadratic terms,
and you expand in powers like the distance between the atoms,
and the energy is quadratic.

You want to know how many modules htere are.
Supposey ou have $N$ modules on my chunck of rock.
How many quadratic terms do I have in the Hamiltnian?

There's a 3D knietic term,
so that's 3 quadratic terms for the kinetic enregy.
Whata bout the potential energy?
Again every module has 3 coordiantes $x,y,z$,
so there are 3 quadratic terms per molecule.
So the energy should be
\begin{align}
    E &= \left( 
    \underbrace{3N}_{\text{kinetic}}
    + \underbrace{3N}_{potential}
    \right)
    \frac{1}{2}k_B T\\
    &= 3Nk_B T
\end{align}

\begin{question}
    Are you assuming a cubic lattice?
\end{question}
No, it could bea really complicated crystal structure,
but you still have the same number of normal modes.

The specific heat of the substance is
\begin{align}
    c &= \frac{1}{N}\frac{\partial E}{\partial T} = 3k_B
\end{align}

We could check this.
Every insulating solid should have exactly the same specific heat.
That's amazing.
Andi t's not just any number,
nad it's a specific number.

You can see this thing here
What I have here isa bunch of pure elements,
andh ere I have a specific heat of hte number.
The line in  the middle is $3k_B$.

This starts from 30,
and they're all around 23.
So you know there are bad ones like potassium 29,
there's another one 23.

It's an amazing result.
See the work we did.

\begin{question}
    Are we assuming only neighbouring particles are talking to each other?
\end{question}
No, I could have $x_1$ connected to $x_{42}$.
As long as the energy is quadratic in distance,
it doesn't matter.
You could find the normal modes,
and you find there are $3N$ normal modes,
and that's the only thing that matters.

\begin{question}
    At higher temperatures in a crystal lattice,
    you'l have anharonic effects 
    like thermal expansion,
    does that influene the specifi heat?
\end{question}
There's no non-linear effects that contribute at high temperature.
How much does a rock expand when you heat a rock.

This is the Dulong-Petit law.
Validity because it really cnovineced people that convineced people statistical
mechanics was right.
It's failure is that it suggested quantum mechanics.

Before we dothat,
let's do a really classic case.
Classic in classical mechanics and classic in that you do it in high school.

\begin{example}[diatomic gas]
    Ideal, nonrelativistic, classical gas.
    What if it's diatomic?
\end{example}
The molecules now look like this.
[picture]
So what's the answer?
If there are $N$ molecules,
how many quadratic terms are you going to have?

Every particle can move in the $x,y,z$ directions,
but the two particles don't move independently,
so that's $6N$ terms.
But there is a potential term for the spring between the two particles,
so
\begin{align}
    E &= (6N + N)\frac{1}{2}k_B T = \frac{7}{2}Nk_B T
\end{align}
